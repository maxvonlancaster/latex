% -----------------------------------------------------------------
% !Mode:: "TeX:UTF-8:Hard"
% PDFLaTeX this document and view or print it from Acrobat Reader!
% -----------------------------------------------------------------
% Preamble Starts here:
% -----------------------------------------------------------------
\documentclass{article}
\usepackage[colorlinks]{hyperref}
\usepackage{graphicx}
\usepackage[utf8]{inputenc}
%\usepackage{pdfsync} % Forward PDF Search!
% EURO ---------------------------------------
% HINT: Use MiKTeX's Package Manger to install eurosym package!
% UTF-8 \u8:??? (€) Euro symbol is currently not supported with LaTeX
% unless the following package is used:
\usepackage{textcomp} % required for \texteuro
\usepackage{eurosym}  % required for \euro
% get a "nicer" looking euro symbol:
%\let\texteuro\euro   % if you want \texteuro=\euro
% WinEdt's Bullets ----------------------------------
% Allow WinEdt's Bullets (placeholders) in TeX Files!
% The bullets (eg. in automatically generated tables)
% should be replaced by actual data. In WinEdt you can
% use Ctrl+Space (Tools Menu -> Next Bullet) to navigate
% through bullets and replace them with "real" entries...
\catcode`\=13
\def{$\bowtie$}
% -----------------------------------------------------------------
% Document Starts here:
% -----------------------------------------------------------------

\begin{document}

\section{WinEdt and Unicode (UTF-8) encoding}

WinEdt is an ASCII editor and does not have full support for
Unicode or UTF-8 encoding. WinEdt always displays documents in
your default Windows code page (ACP).

\medskip

However, WinEdt can transparently convert Unicode (UTF-8)
documents into ACP when they are loaded and back to UTF-8 when
they are saved. This makes it possible to work with UTF-8
documents as long as they can be converted into ACP (Ansi Code
Page). For example, a document that contains a mix of Western
European languages will convert from UTF-8 to ACP and back.
Similarly, (properly configured) WinEdt is capable of handling
Korean UTF-8 documents.

\bigskip

\emph{WinEdt is not capable to handle Unicode (UTF-8) documents
that require more than one code page (eg. a mix of Russian and
French). Furthermore, without a support for bi-directional text
WinEdt is not suitable for editing Hebrew or Arab languages.}

\bigskip
\bigskip

There are two ways to make WinEdt properly handle document
coding. The most reliable and robust method is to include a
mode-comment that allows WinEdt determine how a document should
be handled before it is loaded.

\bigskip

\textbf{1.} A comment in the beginning of a TeX document:
\begin{verbatim}
  % !Mode:: "TeX:UTF-8"
\end{verbatim}
will ensure that the document is properly loaded and saved. A
similar convention is used by emacs:
\begin{verbatim}
  % -*-coding: utf-8 -*-
\end{verbatim}
WinEdt ``understands'' emacs coding directive for UTF-8.

\bigskip

\textbf{2.} In the Wrapping section of the configuration Wizard
you can specify the modes for which UTF-8 encoding is
automatically enabled. For example, if by default all TeX
documents should be treated as UTF-8 documents you should
enter:
\begin{verbatim}
  TeX;UTF-8|ACP;EDT;INI
\end{verbatim}
as a mode filter for UTF-8 format. Do not remove \texttt{UTF-8}
and \texttt{ACP} submodes from this filter!

\begin{center}
  \includegraphics[width=4.5in]{Images/WinEdt-Wizard}\\
\end{center}

\bigskip

Many Unicode or UTF-8 Documents start with a Byte Order Mark
(BOM). Unicode-aware applications can determine the coding of a
document from its BOM. Windows Notepad always includes BOM in
Unicode or UTF documents. Unfortunately, the BOM signature also
causes problems with many applications and compilers (including
TeX with UTF-8 encoding) and that makes it rather useless...

Without BOM and without any convention as described above it is
hard to distinguish between UTF-8 and Ansi documents.

% -----------------------------------------------------------------

\bigskip
\bigskip

\section{WinEdt and (Sub)Modes}

A document mode is initially determined from the filetype and
is stored as a local attribute of a file in WinEdt's File List
(Project File). This works in most cases for the main mode.
However, bilingual users might want to tie certain attributes
(such as dictionaries) to sub-modes that may not be apparent
from the filetype.

\medskip

Instead of setting such modes in the Document Settings Dialog
or adopting a practice to name your files with more than one
filetype (eg. Paper.fr.tex) it is possible to enter sub-modes
(as comments) in the first (or second) line of a document.

\bigskip

WinEdt Modes can be specified as a comment:
\begin{verbatim}
  % !Mode:: "Mode:Submode:Submode"
\end{verbatim}

WinEdt also recognizes mode specification as used by emacs:
\begin{verbatim}
  % -*- mode: TeX -*-
  % -*- coding: utf-8 -*-
\end{verbatim}

It is also possible to specify mode and submode in a single
comment:
\begin{verbatim}
  % -*- TeX:DE:UTF-8 -*-
\end{verbatim}
Emacs might not recognize such specification and it is better
to use WinEdt's convention \verb|!Mode:: "TeX:DE:UTF-8"| as
described above.

Furthermore, for TeX documents WinEdt also detects the language
submode from Babel and UTF-8 coding from the inputenc package:
\begin{verbatim}
  // Determine Language Submodes from babel:
  // \usepackage[french,german,italian,spanish]{babel}

  // Determine Coding (UTF-8) from the preamble:
  // \usepackage[utf8]{inputenc}
\end{verbatim}

This functionality is implemented thought the event handler
macros that are executed before a document is loaded into
WinEdt. In the Preferences| MDI Events Dialog you should
specify:
\begin{verbatim}
  [Exe("%b\Macros\Events\MDIOpen-Before.edt")]
\end{verbatim}
as the MDI Open Macro event handler (default settings already
do this)...

This will ensure that WinEdt opens and treats the document
properly. WinEdt's Help explains how to use Modes and
submodes...

The actual macro that is by default called from this event
handler is
\begin{verbatim}
  %b\Macros\Events\GetMode.edt
\end{verbatim}

If for some reason mode detection (or some portion of it) from
comments is unwanted for your style of work you can edit this
macro and comment out unwanted portions or make any other
desirable changes. However, note that this macro is executed
frequently and it has to be fast or else you'll notice delays
when opening documents or even when collecting data in
previously un-opened documents...

\bigskip

If a WinEdt mode-comment
\begin{verbatim}
  % !Mode:: "Mode:Submode:Submode"
\end{verbatim}
is detected WinEdt does not perform any additional searches for
packages or emacs convention. The presence of such a comment at
the beginning of a document makes loading of documents slightly
faster and reduces a possibility of unwanted mode changes.

\newpage

% -----------------------------------------------------------------

\section{More on WinEdt and UTF-8}

If you are writing only in English then Unicode and UTF-8 is of
no concern to you. Also, if you don't have problems with
portability of your document to other platforms (such as Linux)
then working with your default code page is by far the simplest
and most efficient method.

UTF-8, on the other hand, might be a good choice for those that
share documents with users on different platforms or when the
publisher requires your document to be in such format.

Windows has no native support for UTF-8. Thus handling UTF-8
documents \emph{always} requires (transparent) conversions when
a document is loaded and saved. WinEdt can currently only
convert it to your default code page ACP (rather than 16-bit
UNICODE). This means that there is a limit in regard to the
international characters present in your document. For example,
a mix of Greek and western characters cannot be translated into
a single document. However, all western languages fit in a
single code page.

\bigskip
\bigskip

The format and mode of a document can be changed in the
Document Settings Dialog (Document Menu). You should, however,
understand what you are doing and proceed with caution. As
always, a backup of your document (just in case if something
goes wrong) would be in order. In practice a few possibilities
might arise:

\bigskip

1. You are working on an ANSI document but you would like to
convert it to UTF-8. In Document Settings Dialog remove the
\texttt{":ACP"} submode (if necessary) and add
\texttt{":UTF-8"} submode. Then in the same dialog change the
format to UTF-8. To ensure proper treatment of the document in
the future you can add a comment:
\begin{verbatim}
  % -*- coding: utf-8 -*-
\end{verbatim}
somewhere at the beginning of a document.

\begin{center}
  \includegraphics[width=2.5in]{Images/WinEdt-DocMode}\\
\end{center}

\begin{center}
  \includegraphics[width=2.5in]{Images/WinEdt-DocFormat}\\
\end{center}

\bigskip

2. You are working on a UTF-8 document but you would like to
convert it to ANSI. In Document Settings Dialog remove the
\texttt{":UTF-8"} submode (if necessary) and optionally add
\texttt{":ACP"} submode. Then in the same dialog change the
format to ANSI. To ensure proper treatment of the document in
the future you can add a comment:
\begin{verbatim}
  % -*- coding: ACP -*-
\end{verbatim}
somewhere at the beginning of a document.

\bigskip

3. You opened a document in ANSI mode but instead of
international characters such as (à,ä,å,æ,ç, etc...) you see a
sequence of two or more "strange" looking symbols. Chances are
that your document is in UTF-8 format. Close it and use the
\texttt{":UTF-8"} mode filter in the Open dialog to properly
load the document. If you want to make sure that the document
will be properly treated from here on you can include a
comment:
\begin{verbatim}
  % -*- coding: utf-8 -*-
\end{verbatim}
somewhere at the beginning of a document.

\medskip

\textsc{Warning:} \emph{Do not change the format to UTF-8 and
then save such a document until it is properly displayed in
WinEdt! This would result in a corrupted file...}

\bigskip

4. You try to open a document that is in ANSI format as UTF-8.
WinEdt will alert you about this and it will load the document
as ANSI and mark it as modified. Upon saving the document will
be converted in UTF-8. No damage done...

\bigskip
\bigskip

Depending on whether you set UTF-8 as a default format in the
Configuration Wizard you will have use the \texttt{:ACP}
submode to handle non UTF-8 documents. WinEdt dictionaries,
macros and configuration files should not use UTF-8 format!

\newpage

% -----------------------------------------------------------------

\section{TeX and International Characters (UTF-8)}

Putting
\begin{verbatim}
  \usepackage[utf8]{inputenc}
\end{verbatim}
enables you to use UTF-8 (Unicode) coding. As long as you open
the document in WinEdt in UTF-8 mode you see the same
characters in WinEdt as in your compiled document (as is the
case with this UTF-8 document):

\bigskip

 \noindent \texttt{À Á Â Ã Ä Å Æ Ç È É Ê Ë Ì Í Î Ï Ñ Ò Ó Ô Õ Ö Ø Œ Ù Ú Û Ü Ý Ÿ Š ß ¡}

 \noindent \texttt{à á â ã ä å æ ç è é ê ë ì í î ï ñ ò ó ô õ ö ø œ ù ú û ü ý ÿ š € ¿}

 \noindent \texttt{† ‡ § ¶ © ®}

\bigskip

 \noindent {À Á Â Ã Ä Å Æ Ç È É Ê Ë Ì Í Î Ï Ñ Ò Ó Ô Õ Ö Ø Œ Ù Ú Û Ü Ý Ÿ Š ß ¡}

 \noindent {à á â ã ä å æ ç è é ê ë ì í î ï ñ ò ó ô õ ö ø œ ù ú û ü ý ÿ š € ¿}

 \noindent {† ‡ § ¶ © ®}

\bigskip

Not all UTF-8 characters are currently supported by LaTeX
unless you load extra packages. For example the \euro\ (€)
symbol requires:

\begin{verbatim}
  \usepackage{textcomp} % required for \texteuro
  \usepackage{eurosym}  % required for \euro
  % get a "nicer" looking euro symbol:
  %\let\texteuro\euro   % if you want \texteuro=\euro
\end{verbatim}

Note the difference between the shape of the \verb"\euro"
(\euro) and \verb"\texteuro" (\texteuro) symbols. Such issues
are non-WinEdt related and you will have to consult TeX's
documentation or, if needed, seek help on the appropriate forum
(such as TeX Newsgroup where \LaTeX\ related topics are
discussed).

\bigskip

\noindent Our preamble contains:

\begin{verbatim}
  \catcode`\=13
  \def{$\bowtie$}
\end{verbatim}

\noindent This allows \TeX\ to process an ``empty'' Tabular
Environment from WinEdt's Insert Menu. Bullets are represented
by :

\begin{center}
  \begin{tabular}{|c|c|c|c|c|}
    \hline
    % after \\: \hline or \cline{col1-col2} \cline{col3-col4} ...
     &  &  &  &  \\
    \hline
     &  &  &  &  \\
     &  &  &  &  \\
     &  &  &  &  \\
    \hline
  \end{tabular}
\end{center}

\noindent \emph{In WinEdt you can use Ctrl+Space (Tools Menu:
Next Bullet) to move through placeholders and fill-in the
actual data.}

\newpage

% -----------------------------------------------------------------

\section{Translation Tables}

If you prefer your documents to contain plain \TeX\ notation
for international characters (eg. \verb"\`{A}" stands for
\texttt{À}) then you should consider applying WinEdt's Read and
Write translation tables. This will make working with WinEdt
more comfortable and it is required if you want to take
advantage of WinEdt's Spell-checking ability with international
dictionaries.

\medskip

\emph{WinEdt can convert certain strings into their ANSI
equivalents when the file is being read and then translate
these characters back to the original strings representing
international characters in \TeX\ notation.}

\medskip

Suitable translation tables for \TeX\ mode are already defined
(but not enabled) in the default settings: see Options/
Settings/ Translations Dialog. The help in this dialog provides
the details.

\bigskip

For example, the default \verb"TeX_Read" and \verb"TeX_Write"
translation table contain definitions like:

\begin{verbatim}
          "{\ss}" -> "ß"               "ß" -> "{\ss}"
          "{\AA}" -> "Å"               "Å" -> "{\AA}"
          "{\AE}" -> "Æ"               "Æ" -> "{\AE}"
          "{\aa}" -> "å"               "å" -> "{\aa}"
          "{\ae}" -> "æ"               "æ" -> "{\ae}"
          "{\OE}" -> "Œ"               "Œ" -> "{\OE}"
          "{\oe}" -> "œ"               "œ" -> "{\oe}"
          "{\O}" -> "Ø"                "Ø" -> "{\O}"
          "{\o}" -> "ø"                "ø" -> "{\o}"
          "\c{C}" -> "Ç"               "Ç" -> "\c{C}"
          "\c{c}" -> "ç"               "ç" -> "\c{c}"
          "\^{A}" -> "Â"               "Â" -> "\^{A}"
          "\~{A}" -> "Ã"               "Ã" -> "\~{A}"
          "\""{A}" -> "Ä"              "Ä" -> "\""{A}"

          etc...

\end{verbatim}

\emph{Note that the last item is not a ``typo''! To specify
double quotes inside a double-quoted string they have to be
repeated twice! Failing to observe this convention may
completely corrupt WinEdt's translation table.}

\bigskip

The read translation table supports two notations (eg.
\verb"\^{A}" and \verb"{\^A}"). The write translation table
\verb"TeX_Write" is the inverse of the Read Translation Table
(except that it uses the first, preferable, notation where
applicable). You should use translation tables with some care:
make a backup copy of your documents until you verify that the
tables are set up correctly. Careless application of
translation tables may irreversibly corrupt your documents
(just like a global replace)!!!

\end{document}
