% ----------------------------------------------------------------
% -*-TeX-*- -*-Hard-*- Smart Wrapping
% ----------------------------------------------------------------
% ----------------------------------------------------------------
% Slides *********************************************************
% ----------------------------------------------------------------

\begin{slide}{}
An Extension of Burnside's Theorem:

\begin{prop}
Let $\A\subset\BH$ be a convex subset of bounded linear
operators acting on a real or complex Hilbert space $\h$.
Suppose that for every vector $g\perp{}f_0$ and
$\norm{g}\leq1$, there exists an operator $A\in\A$, satisfying
the following strict inequality for $\xi=r/\sqrt{1-r^2}$:
\begin{eqnarray*}
   \RE\seq{A\left(f_0+\xi{g}\right), f_0-\xi^{-1}g} >
    ~~~~~~~~~~~ \\
   \essnorm{\RE{}A}(1 - \norm{g}^2).
\end{eqnarray*}
Then $\A$ contains an operator $A_0$ with an eigenvalue
$\lambda$ satisfying the condition:
\[ \abs{\RE\lambda}>\essnorm{\RE{}A_0}. \]
\end{prop}
\end{slide}


\begin{slide}{}
\begin{prop}
Suppose $\h$ is a real or complex Hilbert space, and
$\lambda\in\Complex$ is a point in the spectrum of the operator
$A\in\BH$, such that
\[ \abs{\RE\lambda} > \essnorm{\RE{}A}. \]
Then the norm closure of the real algebra generated by $A$
contains a nonzero finite-rank operator.
\end{prop}

\bigskip

Note: Applies to real or complex Hilbert space

Applications: Transitive Algebras and Invariant Subspaces

\end{slide}


\begin{slide}{}
Suppose $A\in\BH$ is a set of operators acting on a real or
complex Hilbert space $\h$.

\begin{defn}
Let $\EssD\subset\h$ be the set of all non\-zero vectors
$x\in\h$ for which there exists a nonzero vector $y\in\h$
satisfying the following inequality for every operator
$A\in\A$:
\[ \RE\seq{A{}x,y} \leq \essnorm{\RE{}A} \seq{x,y}. \]
\end{defn}

Alternative: Either the norm closure of the real algebra
generated by the operators in $\A$ contains a nonzero
finite-rank operator or the set $\EssD$ is dense in $\h$.
\end{slide}


\begin{slide}{}
Applications to the invariant subspace problem for {\em
essentially self-adjoint} operators:

\smallskip

Let $\h$ be an infinite-dimensional real or complex Hilbert
space. The underlying field of real or complex numbers
(respectively) is denoted by $\Field$. Suppose $A\in\BH$ is a
fixed essentially self-adjoint operator without non-trivial
closed invariant subspaces and let $E$ denote its essential
spectrum. Furthermore, we may assume that $\essnorm{A}<1$, and
consequently: $E\subset(-1,1)$. Let $\A\subset\BH$ be an
algebra generated by $A$, i.e. $\A$ is the algebra of all
polynomials $p(A)$, with the coefficients in the underlying
field $\Field$.

The algebra of all polynomials with the coefficients in
$\Field$, equipped with the norm
\[ \norm{p}_\infty=\max_{t\in{E}} \abs{p(t)}, \]
is denoted by $\Poly$.
\end{slide}


\begin{slide}{}
There exist a pair of nonzero vectors $x,y$ such that
\[ \RE\seq{A{}x,y} \leq \essnorm{\RE{}A} \seq{x,y}. \]
Equivalently, for every $p\in\Poly$:
\[ \RE\seq{p(A)x,y} \leq \norm{\RE{}p}_\infty\seq{x,y}. \]

Consequently,
\[ \tau(p) = \seq{p(A)x,y} \]
is a (bounded) positive functional on the space of all
polynomials $\Poly$, equipped with the max norm.

Recall that such a functional is called a {\em vector state} if
$\norm{\tau}=1$, or equivalently $\seq{x,y}=1$.

Note: If $A$ is self-adjoint than
\[ \tau(p) = \seq{p(A)x,x} \]
is a vector state for every vector $x\in\h$.
\end{slide}


\begin{slide}{}
Let $\States$ be the set of all vectors $y\in\h$ for which the
functional $\tau(p)=\seq{p(A)x,y}$ is a vector state on
$\Poly$.

Then $\cal{T}$ is a proper closed and convex subset of the
hyperplane $\set{y\in\h|\seq{x,y}=1}$.

For $y\in\States$:
\[ \widehat{\tau}(p) = \tau((1-t)p(t)) =
   \seq{p(A)x,(1-A^*)y} \]
is a positive functional, and consequently,
\[ \seq{x,(1-\lambda{A^*})y}^{-1}(1-\lambda{A^*})y \in \States, \]
for any $\lambda\in(-1,1)$.

This observation immediately implies that an extreme point in
$\States$ is an eigenvector for $A^*$.

Hence: $\States$ has no extreme points.
\end{slide}


\begin{slide}{}
Fix a vector $y\in\States$ and let $T=A^{*k}$ ($k\geq0$).
Define $\Phi(\lambda)\colon(-1,1)\To\States$ by
\[ \Phi(\lambda)=\seq{x,(1-\lambda T)y}^{-1}(1-\lambda T)y. \]

Choose any vector $z\in\h$ and consider the function
$\psi\colon(-1,1)\To\RPlus$, defined by
\[ \psi(\lambda) = \norm{\Phi(\lambda)-z}^2. \]

The function $\psi$ is differentiable:
\[ \psi'(0) =
   2\RE\seq{T y, y - z - (\norm{y}^2-\RE\seq{y,z})x}. \]
\end{slide}


\begin{slide}{}
Let $P\colon\h\To\States$ be the projection on a convex set
$\States$, i.e. for every vector $z\in\h$
\[ \norm{P z - z} = \inf_{y\in\States}\norm{y-z}. \]

Recall: for $z\not\in\States$ the point $P{}z$ is called a
support point of $\States$ and $z-P{}z$ is called a support
functional at $P{}z$.

Although the set $\States$ has no extreme points it has
``plenty'' of support points. We will show that these points
are non-cyclic vectors for the real algebra generated by $A^*$.
\end{slide}


\begin{slide}{}
Choose a vector $z\in\h$ and let $y=P{z}$. Then the function
$\psi:(-1,1)\To\RPlus$, (as before) defined by
\[ \psi(\lambda)=
   \norm{\seq{x,(1-\lambda T)y}^{-1}(1-\lambda T)y-z}^2 \]
attains minimum at $\lambda=0$:
\[ \psi(0) = \norm{ P z - z }^2 = \inf_{y\in\States}\norm{y-z}. \]

Hence: $\psi'(0)=0$.

Equivalently:
\[ \RE\seq{T y, y - z - (\norm{y}^2-\RE\seq{y,z})x} = 0. \]

Conclusion: $y$ is a non-cyclic vector for the real algebra
generated by $A^*$.
\end{slide}


\begin{slide}{}
\begin{thm}
Every essentially self-adjoint operator acting on a real
infinite-dimensional Hil\-bert space has a nontrivial invariant
subspace.
\end{thm}

\medskip

Complex Case: Only Real Subspaces.

Our technique applies if $A$ admits an essentially self-adjoint
matrix representation with real entries.
\end{slide}


\begin{slide}{}
\begin{con}
Every essentially self-adjoint operator with real spectrum
admits an essentially self-adjoint matrix representation with
real entries.
\end{con}

True in finite dimensions.

Seems hard to prove.

Invariant Subspaces may exist even if the conjecture is false.

\medskip

The structure of the space of vector states: subject to further
research...
\end{slide}

% ----------------------------------------------------------------
% The end of slides
% ----------------------------------------------------------------
