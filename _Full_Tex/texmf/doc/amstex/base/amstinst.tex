%%% ====================================================================
%%% @TeX-file{
%%%   filename  = "amstinst.tex",
%%%   version   = "2.2",
%%%   date      = "2001/08/09",
%%%   time      = "18:47:33 EDT",
%%%   checksum  = "30594 415 2485 18444",
%%%   filetype  = "AMS-TeX: user documentation",
%%%   author    = "American Mathematical Society",
%%%   copyright = "Copyright 2001 American Mathematical Society,
%%%                all rights reserved.  Copying of this file is
%%%                authorized only if either:
%%%                (1) you make absolutely no changes to your copy,
%%%                    including name; OR
%%%                (2) if you do make changes, you first rename it
%%%                    to some other name.",
%%%   address   = "American Mathematical Society,
%%%                Technical Support,
%%%                P. O. Box 6248,
%%%                Providence, RI 02940,
%%%                USA",
%%%   telephone = "401-455-4080 or (in the USA and Canada)
%%%                800-321-4AMS (321-4267)",
%%%   FAX       = "401-331-3842",
%%%   email     = "tech-support@ams.org (Internet)",
%%%   codetable  = "ISO/ASCII",
%%%   keywords  = "amstex, ams-tex",
%%%   supported = "yes",
%%%   abstract  = "This file is part of the AMS-TeX distribution,
%%%                version 2.2. It contains installation instructions.
%%%                The file amsguide.tex inputs this file, but this
%%%                file is also designed so that it can be processed
%%%                separately, using only Plain TeX.",
%%%   docstring = "The checksum field above contains a CRC-16 checksum
%%%                as the first value, followed by the equivalent of
%%%                the standard UNIX wc (word count) utility output of
%%%                lines, words, and characters.  This is produced by
%%%                Robert Solovay's checksum utility.",
%%%  }
%%% ====================================================================
%
%    This file is input by amsguide.tex, which requires AMS-TeX 2.1 or
%    later.  However it can also be typeset separately using Plain \TeX{},
%    means of the mechanism below: we check to see if amsppt.sty has
%    been loaded earlier; if so, we set the catcode of the ~ character
%    to 14 so that the following section of definitions will be ignored
%    (since the definitions would be redundant); otherwise we set the
%    catcode to 9 ("ignore") so that the definitions will be carried out.
%    Several definitions differ from the ones in amsguide.tex because
%    amsppt.sty is absent.
%
%    Enclosing the definitions section within the initial \if ...
%    \else ... \fi would be problematic because of the outerness of
%    \head and \subhead in the amsppt documentstyle.
%
%    Leave the definition of \head active; it is slightly different from
%    the one in amsppt.sty, and gives a different result in Appendix C.
%
\expandafter\ifx\csname amsppt.sty\endcsname\relax
  \catcode`\~=9 \else \catcode`\~=14 \let\BYE\relax \fi
~  \let\BYE\bye
~  \hsize=30pc \vsize=47pc
  \def\head#1\endhead{\bigskip{\sc\noindent
    \leftskip0pt plus.5\hsize \rightskip=\leftskip\parfillskip=0pt
    \def\\{\break}#1\par}\nobreak\smallskip}
  \font\sc=cmcsc10
~  \def\subhead#1\endsubhead{\removelastskip\medskip{\bf\noindent
~    #1\par}\nobreak\smallskip}
~  \def\AmSTeX{{\the\textfont2 A\kern-.1667em%
~    \lower.5ex\hbox{M}\kern-.125emS}-\TeX\spacefactor 1000 }
~  \long\def\usertype#1{\smallskip
~    \moveright2pc\vbox{\def\par{\crcr}\halign{%
~      \setbox0\hbox{\tt##}%
~      \hbox\ifdim\wd0<10pc to10pc\fi{\unhbox0\hfil}%
~      \kern1pc \it $\langle$return$\rangle$\hss
~      \cr#1\crcr}}%
~    \smallskip}
~  \def\<#1>{{\it$\langle$#1\/$\rangle$}}
~  \hfuzz1pc % to suppress reporting of overfull boxes.
~  \hyphenation{amsppt}
~  \def\Textures{{\it Textures\/}}
~  \def\AMS{American Mathematical Society}
~  \def\filnam#1{{\tt\def\\{\char`\\}\ignorespaces#1\unskip}}
~  \hyphenchar\tentt=-1 % to prohibit hyphenation in tt text
~  \newdimen\exindent   \exindent=2\parindent
~  {\obeylines
~   \gdef^^M{\par\penalty9999}%
~   \gdef\beginexample#1{\medskip\bgroup %
~     \def\~{\char`\~}\def\\{\char`\\}\def\{{\char`\{}\def\}{\char`\}}%
~     \overfullrule0pt\tt\frenchspacing %
~     \parindent=0pt#1\leftskip=\exindent\obeylines}
~  }%  end \obeylines
~  \def\endexample{\endgraf\egroup\medskip}
~  \def\cs#1{\leavevmode
~    \skip0\lastskip\unskip\penalty0
~    \ifdim\skip0>0pt \hskip\skip0\fi
~    {\tt\def\{{\char`\{}\def\}{\char`\}}%
~      \def\\{\char`\\}\char`\\\ignorespaces#1\unskip}}
~  \newcount\rostercount
~  \def\roster{\par\smallskip\begingroup \rostercount=0
~    \def\par{{\endgraf}}\hangindent3pc
~    \def\item{\futurelet\next\iitem}%
~    \def\iitem{\ifx\next"\expandafter\iiitem
~      \else\advance\rostercount 1 \iiitem"(\the\rostercount)"\fi}%
~    \def\iiitem"##1"{\par \noindent\hbox to\hangindent{\hss##1\enspace}}%
~  }
~  \def\endroster{\par\smallskip\endgroup}
~  \def\newpage{\vfil\eject}
\catcode126=\active % restore ~ to normal; using `\~ here wouldn't work!
\def\TEXMF{{\tt TEXMF}}

\head Appendix B\\ Installation Procedures\endhead

\subhead B.1. Introduction\endsubhead

The \AmSTeX{} software can be used with any implementation of \TeX{}\null.
Many \TeX{} distributions include \AmSTeX{}, ready to run; check the
documentation that came with the \TeX{} distribution you are using.
When a new version of \AmSTeX{} is released, all distributors are
notified, so if your \TeX{} system is up to date, it is likely that you
don't need to do anything at all.

Most up-to-date \TeX{} installations are organized according to the
\TeX{} Directory Structure (TDS)\null.  This is a tree structure whose
root is identified as something like {\tt /usr/local/share/texmf} or
{\tt c:\char`\\sw\char`\\texmf}.  In the following instructions,
we will use the name \TEXMF{} to identify the root directory of a
TDS or similar structure.

When installing a new release of \AmSTeX{}, you may want to first
back up your old version, if you have existing documents that use it.
Although new releases are intended to be backward compatible, a backup
provides insurance in case something goes wrong.


\subhead B.2. Location of \AmSTeX{} Files in a TDS Tree\endsubhead

Files in the \AmSTeX{} distribution fall into four categories:
files for use with \TeX{}, source files, documentation, and formats.
The following list shows all the files in the current distribution,
along with their recommended locations in a TDS tree.

\medskip
\begingroup
\obeylines\obeyspaces\tt%
TEXMF/
\  tex/amstex/base/
\    amstex.tex
\    amstex.ini
\    amsppt.sty
\    amsppt.sti
\    amsppt1.tex
\  tex/plain/amsfonts/
\    amssym.def
\    amssym.tex
\  source/amstex/
\    README
\    amsppt.doc
\    amstex.bug
\  doc/ams/amstex/
\    amsguide.tex
\    amsguide.ps
\    amsppt.faq
\    amstinst.tex
\    amstinst.ps
\    joyerr.tex
\    joyerr2.tex
\  \<\TeX-implementation>/formats/
\    amstex.fmt
\endgroup

\medskip
\noindent
The \AmSTeX{} distribution can be retrieved in bundled form from the
AMS web site, e.g., as a \filnam{.zip} or \filnam{.tar} file.  In this
form, the files are already arranged according to the TDS structure.
(The format file is not included in the distribution.  It is
platform-specific, and must be created as part of the installation;
see below.)  To install the collection from a TDS-formatted bundle,
place the bundled file into a convenient directory and unpack it using
the \TEXMF{} directory as the ``Extract to'' target.  Some examples:

\halign{\quad#\hfil\quad & #\hfil \cr
 WinZip: & Click on ``Extract'' and then in the ``Extract to'' box, enter\cr
         & {\tt\char`\\sw\char`\\texmf}\quad (or whatever TEXMF is on your
           system)\cr
 gzip:   & {\tt gzip -dc amstex.tgz \char`\|\ (cd TEXMF; tar xvf -)}\cr
 unzip:  & {\tt unzip amstex2.zip -d TEXMF}\cr
}

\noindent
The bundled file may be deleted after unpacking.

If the files are obtained from CTAN (where they are mirrored from
the AMS server), they are not in TDS order.  Use the above list to
place the files into the proper location.

Note: If you currently have any of the following files from very old
releases of \AmSTeX{}, delete them before installing the new release.
They are either irrelevant or superseded in the new version of \AmSTeX{},
and it is best to remove them to avoid confusion.  If you back up your
existing \AmSTeX{} files before installing the new version, these files
should be included.

\settabs4\columns {\tt
\+\ \ amsfil.chg& amsplain.tex& amsppt.mor& amstex.chg\cr
\+\ \ cyracc.def& cyrmemo.def& cyrmemo.tex& amsplain.fmt\cr
}

\medskip
The TFM files for some of AMSFonts 2.2 (\filnam{msam*}, \filnam{msbm*},
and \filnam{eufm*}) are needed to run \AmSTeX{} with the AMSPPT
document style, even if you don't plan to actually print anything using the
AMSFonts.  (These fonts {\it are} needed to process and print the User's
Guide of which this appendix is a part.)  TFM files are available from the
AMS web site, \filnam{www.ams.org/tex/amsfonts.html}, or by anonymous FTP
from \filnam{ftp.ams.org}.  They should be placed in the directory
\filnam{TEXMF/fonts/tfm/ams/}.


\subhead B.3. What if Your \TeX{} System Isn't in a TDS Tree?\endsubhead

If your \TeX{} system is arranged in some other way, you must consult
the user documentation for guidance.  The hints that follow are just
that---hints.  If your \TeX{} system documentation recommends something
different, you should follow that procedure.

We recommend placing \AmSTeX{} files into distinct subdirectories or
folders, to simplify installation of future versions.

First, determine where \TeX{} will look for files to be input.  If
\filnam{amstex.tex} and \filnam{amssym.def} are already present, place
the files listed in section B.2 under {\tt tex/amstex/base/} and
{\tt tex/plain/amsfonts/} into those same areas.  If neither of these
files is already present, look for the file \filnam{plain.tex}.
Create appropriate subdirectories under the area where you find this
file for the new \AmSTeX{} files.

If there isn't any obvious place to put source or documentation files
(the ones listed in section B.2 under {\tt source/amstex/} and
{\tt doc/ams/amstex/}), you can put them in with the input files.
Consult the documentation for your \TeX{} distribution.

Once all the files are installed, you are ready to create a format
file.


\subhead B.4. Creating a Format File\endsubhead

Every implementation of \TeX{} uses format files to preload the macros,
fonts, and hyphenation patterns that define basic user environments
such as \AmSTeX.  Although it is not strictly necessary to use a
format file for \AmSTeX, preloading will save startup time, especially
on slower systems.  Note: Each format file takes up 150K--300K of
disk space (depending on your \TeX{} implementation).

If your \TeX{} system already includes the format file \filnam{amstex.fmt},
you may not have to create a new one; recent changes to \filnam{amstex.tex}
are cosmetic, and will not affect any math formatting features.
If the AMSPPT preprint style has been included in the format, then a new
format file is in order.  (We recommend using the name \filnam{amsppt.fmt}
for a format file with AMSPPT preloaded so that there is no question which
is present; see below.)

Creating a format file (also known as ``initialization'') requires a
special version of \TeX, a particular switch, or item in a menu.
Read the documentation for your \TeX{} distribution to learn how to
create a format file.

Before creating your format file, you will want to consider whether you
habitually use the AMSPPT document style.  If you use other document styles
rarely or never, then you would benefit from the use of a format file with
\filnam{amsppt.sty} preloaded.  If you are likely to use other document styles
periodically, then you probably do not want to preload \filnam{amsppt.sty}.
To make a simple \AmSTeX{} format file, proceed with the next paragraph.
To make a format file with \filnam{amsppt.sty} preloaded, edit the file
\filnam{amstex.ini} and remove the percent sign (comment character) at the
beginning of the line {\tt\char`\%}\cs{documentstyle\{amsppt\}}, just before
the \cs{dump} command.  Save this file with the name \filnam{amsppt.ini}.

If you intend to use Type~1 outline versions of AMSFonts, read Appendix~C
before proceeding.  Users of \Textures{} will have to comment out one more
\cs{input} statement; read the instructions.

Update the file name database (if one is used) so that \TeX{} will be able
to find \filnam{amstex.ini} (or \filnam{amsppt.ini}) and the other files.
This may be done by a command such as {\tt texhash} or via a menu option.
Consult the documentation for your \TeX{} distribution for instructions
on how to perform this update.

Now run the ``initialization'' version of \TeX{} on \filnam{amstex.ini}
(or \filnam{amsppt.ini}).  This will create an \AmSTeX{} format file named
\filnam{amstex.fmt} (or \filnam{amsppt.fmt}).  Some implementations of
\TeX{} will automatically place the format file in the proper directory;
otherwise you should move this file manually into the \TeX{} formats
directory.  Once again update the file name database to make the format
file visible to \TeX{}.


\subhead B.5. Using \AmSTeX{} 2.0+\endsubhead

With some \TeX{} implementations, a format file can be specified by preceding
its name with an ampersand:

\usertype{tex \&amstex filename}

Other implementations treat the name of the format file as a command:

\usertype{amstex filename}

Still other implementations allow the installation of this name as a
menu option.  For details of how to use format files with your
implementation of \TeX{}, see your documentation.


\subhead B.5. Getting a printed copy of the User's Guide \endsubhead

The \AmSTeX{} User's Guide can be obtained as a PDF file from the AMS
web site.  It is also included in the distribution as a PostScript file,
ready to print (\filnam{amsguide.ps}).  Or, you can use your newly
created \AmSTeX{} or AMSPPT format file to typeset the file
\filnam{amsguide.tex}; even if you print out this guide from another
source, we recommend using the file \filnam{amsguide.tex} as a test
to make sure that your newly installed \AmSTeX{} is working properly.


%%%%%%%%%%%%%%%%%%%%%%%%%%%%%%%%%%%%%%%%%%%%%%%%%%%%%%%%%%%%%%%%%%%%%%%%

\newpage
\csname firstpage\string @true\endcsname

\head Appendix C\\ Before Installing \AmSTeX: Facts About Fonts\endhead

\subhead C.1. Using AMSFonts in PostScript Type~1 Form\endsubhead

The AMSFonts Version~2.2 have been converted to PostScript Type~1 outlines
in two forms:

The AMS web site ({\tt www.ams.org/tex/type1-fonts.html}) holds the
canonical distribution.  This is a collection containing all the
typefaces, but only in 5, 7, and 10 point sizes; other sizes must be created
by scaling.  The \AmSTeX{} preprint style provides a way for authors to
indicate that fonts should be loaded under this alternative scaling
convention.  If you are using this version of AMSFonts, insert the line
\cs{PSAMSFonts} in the preamble of your file, before the \cs{topmatter}
line; see section C.3, below, for instructions on making this your local
default.

Basil Malyshev has created a collection, called BaKoMa, containing the
Computer Modern fonts as well as all sizes of the AMSFonts used in
mathematics, but excluding the \filnam{wncy*} cyrillic fonts.  The BaKoMa
fonts can be used with no special action; however, for papers or monographs
to be published by the AMS, \cs{PSAMSFonts} should be specified.


\subhead C.2. Using AMS Symbol Fonts\endsubhead

The preprint style automatically loads the Fraktur font (\filnam{eufm}) and
both fonts of extra symbols (\filnam{msam} and \filnam{msbm}), as well as
all the symbol names, as described in the sections {\bf Fonts} and
{\bf Symbol Names}.  If these will not be used, and you want a version of
\filnam{amsppt.sty} that requires less memory than the default version, you
can suppress the loading of these resources.  If \filnam{amsppt.sty} is to
be included in your format file, you must make this change before creating
the format file.  See the next section for instructions.

\subhead C.3 Setting Local Default Options\endsubhead

An ``initialization file'', \filnam{amsppt.sti}, contains settings for
some options that a user may wish to change locally.  This file is read
in automatically by \filnam{amsppt.sty}.

Two lines in \filnam{amsppt.sti} affect the inclusion of AMS symbol fonts:
\beginexample{}
\\loadeufm \\loadmsam \\loadmsbm
\\message\{symbol names\}\\UseAMSsymbols\\message\{,\}
\endexample
\noindent
To use \filnam{amsppt.sty} without AMSFonts, comment out both lines (place
a {\tt\char`\%} at the beginning of each line); to disable just the symbol
names, comment out the second line.  In either case, remove the {\tt\char`\%}
sign from the beginning of the line
\beginexample{}
\%\\define\\square\{\\vrule width.6em height.5em depth.1em\\relax\}
\endexample
\noindent
This is required for using \cs{qed} to indicate end of proof.
Extra math symbols can be loaded on demand with \cs{newsymbol} or by
including \cs{UseAMSsymbols} in the preamble of a particular document.

In order to make the Type~1 versions of the AMSFonts as distributed from
the AMS web site your local default, activate the line
\beginexample{}
\%\\PSAMSFonts
\endexample
\noindent
by removing the {\tt\char`\%} sign from the beginning of the line.


\BYE % This is = \relax if this file is input by amsguide.tex
%% \CharacterTable
%%  {Upper-case    \A\B\C\D\E\F\G\H\I\J\K\L\M\N\O\P\Q\R\S\T\U\V\W\X\Y\Z
%%   Lower-case    \a\b\c\d\e\f\g\h\i\j\k\l\m\n\o\p\q\r\s\t\u\v\w\x\y\z
%%   Digits        \0\1\2\3\4\5\6\7\8\9
%%   Exclamation   \!     Double quote  \"     Hash (number) \#
%%   Dollar        \$     Percent       \%     Ampersand     \&
%%   Acute accent  \'     Left paren    \(     Right paren   \)
%%   Asterisk      \*     Plus          \+     Comma         \,
%%   Minus         \-     Point         \.     Solidus       \/
%%   Colon         \:     Semicolon     \;     Less than     \<
%%   Equals        \=     Greater than  \>     Question mark \?
%%   Commercial at \@     Left bracket  \[     Backslash     \\
%%   Right bracket \]     Circumflex    \^     Underscore    \_
%%   Grave accent  \`     Left brace    \{     Vertical bar  \|
%%   Right brace   \}     Tilde         \~}
