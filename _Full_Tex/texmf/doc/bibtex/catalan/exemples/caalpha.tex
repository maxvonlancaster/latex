\documentclass{article}
\usepackage[latin1]{inputenc}
\usepackage[spanish,english,italian,catalan]{babel}
\title{Exemple de l'estil bibliogr�fic \textsf{caalpha}}
\author{Robert Fuster}
\date{15 de desembre de 2004}
\begin{document}
\maketitle
Aquest �s un exemple de l'estil bibliogr�fic  \textsf{caalpha}.

Entre altres documents la bibliografia inclou dues entrades del mateix autor
(\cite{lamport2,biblamport}), el document electr�nic~\cite{hyperlatex}
i algun llibre de text, com ara~\cite{rbjcjmau}. Diversos documents 
\cite{biblamport, btxdoc,lamport2,vafus,jcccrf,liang} fan refer�ncia al \TeX, al 
\LaTeX{} i al \textsc{Bib}\TeX.
\nocite{*}
\bibliographystyle{caalpha}
\bibliography{exemples}
\end{document}
