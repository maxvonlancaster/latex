\documentclass{article}
\usepackage[latin1]{inputenc}
\usepackage[spanish,english,italian,catalan]{babel}
\usepackage{natbib}
\usepackage{url}
\bibpunct{(}{)}{;}{a}{:}{,}  % Puntuacio en les citacions (veure natbib)
\newcommand{\BIBannoteformat}{%  Format de les anotacions
                \small\slshape\textbf{Comentari: }}
\title{Exemple de l'estil bibliogr�fic \textsf{caauda3}}
\author{Robert Fuster}
\date{15 de desembre de 2004}
\begin{document}
\maketitle
Aquest �s un exemple de l'estil bibliogr�fic del tipus autor-data \textsf{caauda3}.

Entre altres documents la bibliografia inclou dues entrades del mateix autor
\citep{lamport2,biblamport}, el document electr�nic~\citet{hyperlatex}
i algun llibre de text, com ara~\citet{rbjcjmau}. Diversos documents 
\citep{biblamport, btxdoc,lamport2,vafus,jcccrf,liang} fan refer�ncia al \TeX, al 
\LaTeX{} i al \textsc{Bib}\TeX.
\nocite{*}
\bibliographystyle{caauda3}
\bibliography{exemples}
\end{document}
