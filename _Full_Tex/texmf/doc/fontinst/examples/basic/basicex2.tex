%
% basicex2.tex
%
% fontinst command file for writing map file fragments for 
% the raw fonts used by the fonts made in basicex.tex.
%

\input finstmsc.sty

% At this point, you might want to change some settings
% to make the map file fragments written better adjusted
% to your particular system. Some common examples are:
% 
%   \resetstr{PSfontsuffix}{.pfb}
% 
% which is useful if the actual fonts are in PFB format.
% 
%   \AssumeLWFN
% 
% This is useful if the font files are named according to
% the old Mac OS Classic convention for PostScript fonts
% (names like AGarBolIta).

\adddriver{dvips}{dvips.map}
% This causes a .map file for dvips to be written. 
% It is probably better to choose a name that has to 
% do with the fonts than to use `dvips.map'.
% In some systems, there is a program updmap which combines
% several small .map files such as this one into a large
% psfonts.map file. 
\adddriver{dvipdfm}{dvipdfm.map}
% Similar file for dvipdfm.
\adddriver{pltotf}{pltotf-script.sh}
% A shell script that runs pltotf on those .pl files
% which you need to turn into .tfm files.
\adddriver{debug}{mapdump.txt}
% This file contains the available data in a human-readable
% format. This is useful if your dvi driver uses some other
% kind of map file.

\input basicex.recs
% This is where all the work is done.

\donedrivers
% This closes the files and does some cleanup.

\bye



