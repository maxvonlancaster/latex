\documentclass{article}
% Use
\usepackage{allrunes}
% for Metafont/pk fonts, or use
%\usepackage[type1]{allrunes}
% for type1 fonts.
\title{A Short Demo of the \textsf{allrunes} Font and Package}
\author{Carl-Gustav Werner}
\begin{document}
\maketitle
In Bj\"orketorp, Blekinge, Sweden, three
large stones are erected. On one of them a threatening
inscription is made in Primitive Norse:
\begin{quotation}
\noindent
{\arcfamily  \large
h\A idRrunoronu\\
f\A l\A h\A Kh\A ider\A g\\
in\A run\A R\A r\A geu\\
h\A er\A m\A l\A usR\\
uti\A rwel\A d\A ue\\
s\A R\th \A tb\A rutR\\
u\th\A r\A b\A sb\A}
\end{quotation}

In Torna H\"allestad, Sk\r ane, Sweden, three rune stones are set up
in the wall of the church, all with normal Scandinavian runes. (But note the
special dotted m-rune \textarn{\"m} instead of the normal \textarn{m}.)
The longest inscription of the three reads as follows:
\begin{quotation}
\noindent
{\arnfamily  \large
:askil:sati:stin:\th ansi:iftiR\\
:tuka:kur\"ms:sun:saR:hulan:\\
trutin:saR:flu:aigi:at:ub:\\
:salu\"m\\
satu:trikaR:iftiR:sin:bru\th r\\
stin:A:biarki:stu\th an:runu\"m:\th iR:\\
kur\"ms:tuka:kiku:nistiR}
\end{quotation}

In England, now at the British Museum, the stone cross from Lancaster
has the following runic inscription:
\begin{quotation}
\noindent
{\arafamily  \large
gibiD\ae\th fo\\
r\ae cynibal\\
\th cu\th bere}
\end{quotation}

In H\"og, H\"arjedalen, Sweden, a stone has an inscription made with staveless runes:
\begin{quotation}
\noindent
{\arlfamily \large \withlines
ku\th niutr\tripledot \th rusun\tripledot litritastin\th ina
\tripledot akbrukir\th i\tripledot aftiRbru\th rsina\tripledot
asbiurn:\\
akatku\th laf}
\end{quotation}

\end{document}
