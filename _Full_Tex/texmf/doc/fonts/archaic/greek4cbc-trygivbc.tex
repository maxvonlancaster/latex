% trygivbc.tex    Test 4thC BC Greek font
\documentclass{article}
\usepackage{greek4cbc}

\newcommand{\abc}{a b g d e z h T i k l m n x o p r s t y X f P O}
\newcommand{\Acomms}{\Aalpha%\
                     \Abeta%\
                     \Agamma%\
                     \Adelta%\
                     \Aepsilon%\
                     \Azeta%\
                     \Aeta%\
                     \Atheta%\
                     \Aiota%\
                     \Akappa%\
                     \Alambda%\
                     \Amu%\
                     \Anu%\
                     \Axi%\
                     \Aomicron%\
                     \Api%\
                     \Arho%\
                     \Asigma%\
                     \Atau%\
                     \Aupsilon%\
                     \Achi%\
                     \Aphi%\
                     \Apsi%\
                     \Aomega}

\newcommand{\ARcomms}{\ARalpha%\
                     \ARbeta%\
                     \ARgamma%\
                     \ARdelta%\
                     \ARepsilon%\
                     \ARzeta%\
                     \AReta%\
                     \ARtheta%\
                     \ARiota%\
                     \ARkappa%\
                     \ARlambda%\
                     \ARmu%\
                     \ARnu%\
                     \ARxi%\
                     \ARomicron%\
                     \ARpi%\
                     \ARrho%\
                     \ARsigma%\
                     \ARtau%\
                     \ARupsilon%\
                     \ARchi%\
                     \ARphi%\
                     \ARpsi%\
                     \ARomega}


\title{Try 4th Century BC Greek Font}
%%\author{Peter Wilson \\ \texttt{herries dot press at earthlink dot net}}
\author{}
\date{}
\begin{document}
\maketitle

    This provides a short test of the characters in the 4th century \textsc{bc} Greek 
font
--- the \verb|givbc| font family.

\begin{center}
The smooth font in Large size \\
{\givbcfamily\Large \abc \par
}
\end{center}

\begin{center}
The font in its normal size, both smooth and rough, and for comparison, transliterated into 
Modern Greek. \\
\textgivbc{\Acomms} \\ \textgivbc{\ARcomms} \\
\translitgivbc{\Acomms} \\
\end{center}

    GREEK in Greek is: \textgivbc{greek}.

The next lines of `Greek' are produced by the following code (note that
the Greeks had no punctuation marks): 
\begin{verbatim}
{\givbcfamily this font gomes in both a rough and a smooT 
form \quad
Tis is Te smooT form \quad at Tis time Te
greeks no longer used boustrofedron riting}
\end{verbatim}

{\givbcfamily this font gomes in both a rough and a smooT 
form \quad
Tis is Te smooT form \quad at Tis time Te
greeks no longer used boustrofedron riting}

\begin{table}
\centering
\caption{Alphabet and commands}
\begin{tabular}{|c|c||l|l|} \hline
Glyph         & ASCII & Smooth & Rough \\ \hline
\textgivbc{a} & a      & \verb|\Aalpha|   & \verb|\ARalpha|      \\
\textgivbc{b} & b      & \verb|\Abeta|    & \verb|\ARbeta|      \\
\textgivbc{g} & g      & \verb|\Agamma|   & \verb|\ARgamma|      \\
\textgivbc{d} & d      & \verb|\Adelta|   & \verb|\ARdelta|      \\
\textgivbc{e} & e      & \verb|\Aepsilon| & \verb|\ARepsilon|      \\
\textgivbc{z} & z      & \verb|\Azeta|    & \verb|\ARzeta|      \\
\textgivbc{h} & h      & \verb|\Aeta|     & \verb|\AReta|      \\
\textgivbc{T} & T      & \verb|\Atheta|   & \verb|\ARtheta| \\
\textgivbc{i} & i      & \verb|\Aiota|    & \verb|\ARiota|      \\
\textgivbc{k} & k      & \verb|\Akappa|   & \verb|\ARkappa|      \\
\textgivbc{l} & l      & \verb|\Alambda|  & \verb|\ARlambda|      \\
\textgivbc{m} & m      & \verb|\Amu|      & \verb|\ARmu|      \\
\textgivbc{n} & n      & \verb|\Anu|      & \verb|\ARnu|      \\
\textgivbc{x} & x      & \verb|\Axi|      & \verb|\ARxi| \\
\textgivbc{o} & o      & \verb|\Aomicron| & \verb|\ARomicron|      \\
\textgivbc{p} & p      & \verb|\Api|      & \verb|\ARpi|      \\
\textgivbc{r} & r      & \verb|\Arho|     & \verb|\ARrho|      \\
\textgivbc{s} & s      & \verb|\Asigma|   & \verb|\ARsigma|      \\
\textgivbc{t} & t      & \verb|\Atau|     & \verb|\ARtau|      \\
\textgivbc{y} & y      & \verb|\Aupsilon| & \verb|\ARupsilon|      \\
\textgivbc{X} & X      & \verb|\Achi|     & \verb|\ARchi|      \\
\textgivbc{f} & f      & \verb|\Aphi|     & \verb|\ARphi| \\
\textgivbc{P} & P      & \verb|\Apsi|     & \verb|\ARpsi| \\
\textgivbc{O} & O      & \verb|\Aomega|   & \verb|\ARomega| \\
\hline
\end{tabular}
\end{table}

\end{document}
