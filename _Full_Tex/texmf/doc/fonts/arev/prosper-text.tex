%prosper-text.tex

\frame[t]
{
  \frametitle{Characterization of the Imaginary Forms}
  
  \blue{\textbf{Theorem.}}

  Let $S=\{v_1,v_2,\ldots,v_k\}$ be a \green{set of vectors} in $\mathbb{R}^n$.

  \bigskip

  Consider $\mathcal{F}(S)=\sum_{i=1}^k \delta(v_i v_j w) \sigma_{i,j}$.

  \bigskip

  If \red{$\mathcal{F}(S)\le \varepsilon$}, then

  \[
     \phi(S,\alpha)=\frac{1}{2\pi i} \int_{-\infty}^{753}
     \frac{\tilde{W}_{n}(\gamma)\cos\left(\sqrt{x^{2}}\right)}{f'(x) R/a}dx
     =\det\left(\begin{array}{cc}
       \alpha^{2} & \Pi\\
       \omega & x\otimes y\end{array}\right)
  \]

  \bigskip \

  \emph{Note}: If $\beta\in\Gamma$, then the form is \red{undefined} at the points in $S\cap\Gamma$, and the integral $I_l(i_1)$ diverges as $\varepsilon \rightarrow 0$.  This pathological behavior \purple{can be handled} by taking
  $\Gamma \subseteq S$.
  
}

\end{document}
