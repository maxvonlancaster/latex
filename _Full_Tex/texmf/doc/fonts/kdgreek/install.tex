% This file requires plain TeX for typesetting and NO special fonts
% Use tex install.tex to produce the file install.dvi
% install.tex 7-Dec-92
%
%INSTALLATION PROCEDURE FOR KD FONTS
\magnification=\magstep1
\parindent=0pt
\centerline{\bf INSTALLATION PROCEDURE}
\smallskip
\centerline{\bf FOR Greek\TeX\ PACKAGE}
\bigskip
\centerline{\bf NOTE: This document does not describe installation of %
the grlatex part of the package}
\bigskip
This documents gives a brief description of the installation
procedure of the Greek\TeX\ Package. It is assumed that \TeX\ version
3.0 or higher is installed in your system and you also have access to
Metafont (v 2.00 or higher). Most of the systems are also equiped
with a scirpt, or batch file or even a small program that will create
a font at a given magnification with a single command. Let us assume
that this command is {\tt newfont} $<FontName>$ {\tt scaled}
$<Magnfification>$. You must also have access to {\tt plain.tex} and
{\tt hyphen.tex} (plain format version 3.00)  or higher, as well as
to INITEX for building formats. If you are interested in building
the grlatex format as well, the the files for building lplain must
also be available.

\medskip
\leftline{\bf Installing the Fonts}
\smallskip
Here is a table of the fonts and suggested magnifications:
$$\vbox{\divide\hsize by2
\halign{\strut\tt #\hfill\tabskip=2em&\hfil#\hfil&\hfil#\hfil&\hfil#\hfil&%
\hfil#\hfil&\hfil#\hfil&\hfil#\hfil&\hfil#\hfil\cr
&750&800&900&1000&1200&1440&1795\cr
kdgr10&&&&*&*&*&*\cr
kdgr9&&&&*&*&*&*\cr
kdgr8&*&&&*&*&*&*\cr
kdbf10&&&&*&*&*&*\cr
kdbf9&&&&*&*&*&*\cr
kdbf8&*&&&*&*&*&*\cr
kdsl10&&*&*&*&*&*&*\cr
kdti10&&*&*&*&*&*&*\cr
kdtt10&&*&*&*&*&*&*\cr
}}
$$
Each font is installed by a command of the form\par
{\tt newfont kdgr10 scaled 1200}\par
After installing all the fonts you are ready to move to the
installation of the greek format.
\medskip
\leftline{\bf Format Installation}
\smallskip
Make sure that the file {\tt plain.tex} and {\tt hyphen.tex} are in
your path and enter the command:\par
{\tt INITEX greek}\par
When INITEX has finished enter {\tt $\backslash$ dump}. Hopefully the file
{\tt greek.fmt} must be ready to use. Just place it in your \TeX\ path
and issue the command\par
{\tt tex \& greek foo.tex}\par
to run GREEK\TeX.

You can also place the file {\tt greektex.tex} in you TEXINPUT path
so that it can be included in short greek documents or replace the
greek format file if the latter does not exist in your system. Note
that the hyphenation will only work under the greek format and {\bf
not} when you include the {\tt greektex} file to your document.
Hyphenation apart, no other differences exist between the greek
format and {\tt greektex.tex}.


If you are building the latex extension grlatex, issue the command:\par
{\tt INITEX lgreek}\par
This will built the lgreek format. The lgreek format can be used
instead of the standard latex format since it is simply and extension
of the latex package enriched with several language dependant macros
and the hyphenation patterns for greek tex. 

If during the procedure of creating the greek latex INITEX complains
about too many font families then the lplain format must be modified
to define only 10 font families (older version of lfonts.tex defined
the {\tt uit} family of fonts which you might be able to eliminate
without mamy losses).



\bye
