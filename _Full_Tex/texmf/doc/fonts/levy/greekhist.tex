%\magnification=\magstep1
\def\midspace#1{}
\def\caption#1{}

\input greekmacros

\greekdelims

\def\plus{\math{}+{}\math}
\font\man=logo10 at 10true pt % font used for special symbols (old version)
  \def\MF{{\man META}\-{\man FONT}}%
% macros for verbatim scanning
\newskip\ttglue \ttglue=.5em plus.25em minus.15em
\chardef\other=12
\def\ttverbatim{\begingroup
  \catcode`\\=\other
  \catcode`\{=\other
  \catcode`\}=\other
  \catcode`\$=\other
  \catcode`\&=\other
  \catcode`\#=\other
  \catcode`\%=\other
  \catcode`\~=\other
  \catcode`\_=\other
  \catcode`\^=\other
  \obeyspaces \obeylines \tt}
 
\outer\def\begintt{$$\let\par=\endgraf \ttverbatim \parskip=\z@
  \catcode`\|=0 \rightskip-5pc \ttfinish}
{\catcode`\|=0 |catcode`|\=\other % | is temporary escape character
  |obeylines % end of line is active
  |gdef|ttfinish#1^^M#2\endtt{#1|vbox{#2}|endgroup$$}}

\catcode`\|=\active
{\obeylines \gdef|{\ttverbatim \spaceskip\ttglue \let^^M=\  \let|=\endgroup}}

%\documentstyle{amsppt}
%\nologo
%
%\topmatter
%\title Typesetting Greek \endtitle
%\author Silvio Levy \endauthor
%\affil Princeton University \endaffil
%\address{Fine Hall, Washington Road, Princeton, NJ, 08544}
%\abstract{
%We discuss the design, creation and use of a family of Greek fonts for
%\TeX.  The fonts can be used for modern or classical
%Greek, by themselves or in combination with the Computer Modern family
%of fonts.  A short historical introduction is followed by a discussion
%of special topics, including the handling of accents and breathings,
%hyphenation, and the two varieties of sigma.
%}
%\endtopmatter
%\document
%

\font\eightpoint=cmr8
\font\twelverm=cmr10 at 12pt
\def\heading#1\endheading{\bigskip\centerline{\twelverm#1}\medskip}

%formatting conventions for proceedings
\catcode`@=11
\font\titlefont=cmr17
\font\namefont=cmcsc10 scaled \magstep1
\font\twelvebd=cmbx10 scaled \magstep1
\font\twelverm=cmr10 scaled \magstep1
\font\sixrm=cmr6
\vsize=43pc
\hsize=29pc
\parindent=16pt
\nopagenumbers
\normalbaselines
\footline={\hss\sixrm\folio\hss}
\def\makefootline{\baselineskip3pc\line{\the\footline}}
\def\plainoutput{\shipout\vbox{\makeheadline\pagebody\makefootline}%
  \advancepageno
  \vsize=45pc
  \ifnum\outputpenalty>-\@MM \else\dosupereject\fi}
\def\makeheadline{\vbox to\z@{\vskip-22.5\p@
  \line{\vbox to8.5\p@{}\ifnum\count0=1
         \else\ifodd\count0
                \hfill\the\doctitle\ignorespaces
             \else \the\authorname\hfill\fi\fi}\vss}\nointerlineskip}

\newtoks\doctitle  \newtoks\authorname

\doctitle={Typesetting Greek}
\authorname={Silvio Levy}
\catcode`@=12 % at signs are no longer letters
\tenrm
\begingroup
\titlefont
\the\doctitle
\vskip 3pc
\namefont\the\authorname
\tenrm
\vskip 1pc
{\obeylines
Mathematics Department
Princeton University
Princeton, NJ, 08544
}
\vskip 3pc
\line{\hfill\twelvebd ABSTRACT\hfill}
\vskip 3pc
\leftskip 4pc \rightskip 4pc
We discuss the design, creation and use of a family of Greek fonts for
\TeX.  The fonts can be used for modern or classical
Greek, by themselves or in combination with the Computer Modern family
of fonts.  A short historical introduction is followed by a discussion
of special topics, including the handling of accents and breathings,
hyphenation, and the two varieties of sigma.
\par
\endgroup
\vskip 3pc

\heading A Bit of History \endheading

During the first four centuries after the introduction of the printing
press in Europe, the printing of Greek was hampered by the relative
inadequacy of existing types, at least in comparison with the quality
and variety of the best roman fonts.  This was partly a consequence
of the antiquity of the language and its consequent evolution in both
writing and pronunciation: not only did the letterforms change over time,
but they came to be adorned with a multitude of diacritical marks, a
legacy of zealous scribes and grammarians anxious to preserve the
pristine state of the language that lent them their prestige.

In the early days we find that the few printers that attempted to cut
Greek type generally ignored all this complexity, either disregarding
the diacritics or casting them separately and setting them
above each line of text.  Some texts were printed in
lowercase only; others would make do with roman capitals like `A' and
even some lowercase roman glyphs, like `v' for `$n$'.  (See figures
1--12 in [Brit.~Mus.~1927], an excellent historical survey.)

In 1495 Aldus Manutius introduced cursive greeks.
They became a resounding success, as they reproduced the florid
and idiosyncratic handwriting of the day, full of abbreviations
and ligatures.  The inordinate number of ligatures is in fact the
most striking feature of such fonts: for example, Fell's ``Great
Primer Greek'' [Morison 1967, p.~102] has sorts for many
four-letter words and prefixes.

Although skilfully typeset books using cursive fonts are undeniably
beautiful (see [Brit.~Mus.~1927], figure 29), the use of so many
ligatures was a nightmare for compositor and reader alike, since
in some cases the component letters are virtually unidentifiable.
Gradually the fonts were scaled down in size, but the practice
of imitating handwriting remained until the end of the eighteenth
century, when Bodoni, G\"oschen and Didot paved the way back to a
more sober course.

Still problems remained.  Porson's greek [Mosley 1960], first used
in 1826 and destined to become the standard of the English-speaking
world, continues the illogical tradition of mixing upright capitals
with slanted minuscules; in particular the blending of Greek and roman
text leads to poor results (figure 1).  Didot's design was somewhat
more felicitous, and became standard in Italy, France and Greece itself
(figure 2), but the type is narrow and irregular, bearing, according
to Scholderer [Brit.~Mus.~1927, p.~14], the `malign mark' of Bodoni's
greeks (which were admittedly not a match for his inspired romans).
\midspace{1.1truein}
\caption{\eightpoint Figure 1}

\midspace{1.2truein}
\caption{\eightpoint Figure 2}

(Incidentally, the different treatment of capitals established itself
in mathematical typesetting.  In English books, Greek capitals are the
only math letters that are not slanted; in French and German books
they conform with the rest.  The Computer Modern Greek faces are
based on Porson.  They are, of course, meant for math; when used
for text, as in [Wonneberger 1987], the results fall short of
the perfection achievable with \TeX.)

It was not until the beginning of this century that really
well-designed Greek types became available, including Scholderer's
`New Hellenic' type (figure 3), based on a pre-Aldine model,
and the Monotype font shown in figure 4, a much improved version
of Didot's design and perhaps the typeface most favored for
high-quality printing in Greece during the last few decades.
\midspace{1.1truein}
\caption{\eightpoint Figure 3}

\midspace{1.3truein}
\caption{\eightpoint Figure 4}

\heading Requirements \endheading

My main motivation in designing a family of Greek fonts is the
preparation of a Modern Greek--English dictionary.  This in itself
makes the design more exacting: the fonts should not only look good
individually and in combination, but also blend well with Computer
Modern fonts; there should be at least three main styles, different
enough that no special effort to distinguish between them is required,
even in small point size; and the typing should be as painless as
possible on a standard (English) keyboard.
This last condition is, of course, open to interpretation, but I
define it to mean that each grapheme (letter, accent, etc.)~should
require as few keystrokes as possible---generally one, but
occasionally more, like accents in English.

In esthetic terms there are a lot of details to work out, but the
foundation has been laid.  I follow loosely the type shown in figure 4,
which shares several features with the Computer Modern roman:
sharp contrast between thicks and thins, similar letter widths,
and a wealth of texture and detail.  (This is not surprising since
the `modern' family of fonts goes back to Bodoni and Didot.)
But it also has distinguishing features, which recapture some of the
beauty of cursive writing: its strokes are more fluid, and there
is no left-right symmetry (compare a roman `o' with a greek `$o$').

Modern Greek has traditionally been typeset with the diacritic
apparatus of ancient Greek, which is very rich: it contains two accents,
the acute $'$ and the grave $`$, that can go over any vowel, plus one,
the circumflex $~$, that can go over the vowels $ahiuw$; two so-called
breathings $<>$, one of which goes over every vowel or diphthong
in initial position; the diaeresis $"$, that can go over $i$ and $u$;
and the iota subscript $|$, that can go under $ahw$.  An accent
plus a breathing, or an accent plus the dieresis, can coexist;
furthermore any accent, breathing or accent-breathing combination
can share a vowel with an iota subscript.  Here, for example,
are the 24 possible varieties of lowercase $a$:
\display
\hbox{$a 'a `a ~a <a <'a <`a <~a >a >'a >`a >~a
a| 'a| `a| ~a| <a| <'a| <`a| <~a| >a| >'a| >`a| >~a|$}
\display

In the last several decades the tendency has been to get rid of these
complications, and now the ``official'' system in use in Greece includes
only one accent and the dieresis, without breathings or iota subscripts.
Even though I work primarily with modern Greek, however, I thought
it short-sighted not to include the whole apparatus, having in mind
both classicists and those traditionalists who
still prefer the three-accent system, as do some publishing houses.

It was clear right away that these diacritics should be implemented as
ligatures, not as \TeX\ accents, because in \TeX\ words that include
accents cannot be hyphenated.  This implied that 128 font
positions were not enough.   Fortunately \TeX\ and \MF\ are
well equipped to handle 256-position fonts, though most device
drivers are not (see the section `Other Problems').

Following the one-keystroke-per-grapheme rule, then, one can conjure
up the last alpha in the display above by typing |>~a|{\tt\char`\|},
which accesses a four-character ligature.  The remaining conventions
for diacritics are: |'| and |`| for the acute and grave, respectively;
|"| for the dieresis; and |<| for the rough breathing.  (If you're
wondering how to produce quotation marks, or $((e>isagwgik'a))$,
it's by typing |((| and |))|.)

\heading The Font Layout \endheading

A further complication has to do with the letter sigma.  According to
Knuth [1980], the letter `s' is in a class by itself in terms of
complexity of design; its Greek counterpart is similarly difficult, though
for different reasons.  A sigma in initial or medial position in the word
is written $c$, but in final position it is written $s$.  Since the
alternation is entirely conditioned, it seemed a pity to require different
characters in the \TeX\ file to represent the two varieties of sigma.

The first solution I tried made no assumptions on the font.  It
consisted in making `s' an active character, which checked the next
token and printed $c$ if it was a letter, $s$ otherwise.  There were
two drawbacks to this procedure: the letter `s' could not be used
in control sequences, and the check was expensive.  (It also didn't
work when the next token was a control sequence that expanded to one
or more letters.)

A better idea is to save one position in the font for an
invisible dummy character.  All punctuation marks (and the space)
are made active; they expand to the dummy end-of-word character,
plus the punctuation character.  An `s' by itself prints as $c$,
while an `s' in ligature with the dummy---which is to say, at the end
of a word---prints as $s$.  All other letters yield themselves when
in ligature with the dummy.  This is still not ideal for two reasons:
\TeX\ is always ``obeying spaces,'' and words separated by something
other than punctuation or spaces (say |\par|) count as one.  (But
this might be the only practicable solution for the Hebrew alphabet,
which has five chameleon letters.)

The solution I chose avoids the drawbacks above, at the expense of
several positions in the font.  Namely, the font contains each of
the possible combinations $sa$, $sb$, \dots, $sw$, which are
automatically accessed as ligatures when the word contains `s'
followed by a letter.

\pageinsert
{
\font\twelverm=cmr10 scaled \magstep1
\font\twelvei=cmmi10 scaled \magstep1\skewchar\twelvei='177
\font\twelvesy=cmsy10 scaled \magstep1\skewchar\twelvesy='60
\font\twelvebf=cmbx10 scaled \magstep1
\font\twelvett=cmtt10 scaled \magstep1\hyphenchar\twelvett=-1
\font\twelvesl=cmsl10 scaled \magstep1
\font\twelveit=cmti10 scaled \magstep1

\def\twelvepoint{\def\rm{\fam0\twelverm}%
  \textfont0=\twelverm
  \textfont1=\twelvei
  \textfont2=\twelvesy
  \textfont3=\tenex
  \def\it{\fam\itfam\twelveit}%
  \textfont\itfam=\twelveit
  \def\sl{\fam\slfam\twelvesl}%
  \textfont\slfam=\twelvesl
  \def\bf{\fam\bffam\twelvebf}%
  \textfont\bffam=\twelvebf \scriptfont\bffam=\sevenbf
   \scriptscriptfont\bffam=\fivebf
  \def\tt{\fam\ttfam\twelvett}%
  \textfont\ttfam=\twelvett
  \tt \ttglue=.5em plus.25em minus.15em
  \normalbaselineskip=14.5pt
  \setbox\strutbox=\hbox{\vrule height10.5pt depth4pt width0pt}%
  \normalbaselines\rm}

\twelvepoint
\font\my=grreg10 scaled \magstep1
\centerline{
\def\:{\char\count255\global\advance\count255 by 1}
\count255='0
\vbox{
\hbox{\hbox to 40pt{\it\hfill0\/\hfill}%
\hbox to 40pt{\it\hfill1\/\hfill}%
\hbox to 40pt{\it\hfill2\/\hfill}%
\hbox to 40pt{\it\hfill3\/\hfill}%
\hbox to 40pt{\it\hfill4\/\hfill}%
\hbox to 40pt{\it\hfill5\/\hfill}%
\hbox to 40pt{\it\hfill6\/\hfill}%
\hbox to 40pt{\it\hfill7\/\hfill}}
\vskip 4pt
\hrule
\def\^{\vrule height 10.5pt depth 4.5pt}
\halign{\hbox to 0pt{\hskip -24pt {\rm\'{}\kern-.2em\it#}\hfill}&\^
\hbox to 40pt{\my\hfill#\hfill\^}&
&\hbox to 40pt{\my\hfill#\hfill\^}\cr
00&\:&\:&\:&\:&\:&\:&\:&\:\cr\noalign{\hrule}
01&\:&\:&\:&\:&\:&\:&\:&\:\cr\noalign{\hrule}
02&\:&\:&\:&\:&\:&\:&\:&\:\cr\noalign{\hrule}
03&\:&\:&\:&\:&\:&\:&\:&\:\cr\noalign{\hrule}
04&\:&\:&\:&\:&\:&\:&\:&\:\cr\noalign{\hrule}
05&\:&\:&\:&\:&\:&\:&\:&\:\cr\noalign{\hrule}
06&\:&\:&\:&\:&\:&\:&\:&\:\cr\noalign{\hrule}
07&\:&\:&\:&\:&\:&\:&\:&\:\cr\noalign{\hrule}
10&\:&\:&\:&\:&\:&\:&\:&\:\cr\noalign{\hrule}
11&\:&\:&\:&\:&\:&\:&\:&\:\cr\noalign{\hrule}
12&\:&\:&\:&\:&\:&\:&\:&\:\cr\noalign{\hrule}
13&\:&\:&\:&\:&\:&\:&\:&\:\cr\noalign{\hrule}
14&\:&\:&\:&\:&\:&\:&\:&\:\cr\noalign{\hrule}
15&\:&\:&\:&\:&\:&\:&\:&\:\cr\noalign{\hrule}
16&\:&\:&\:&\:&\:&\:&\:&\:\cr\noalign{\hrule}
17&\:&\:&\:&\:&\:&\:&\:&\:\cr\noalign{\hrule}
20&\:&\:&\:&\:&\:&\:&\:&\:\cr\noalign{\hrule}
21&\:&\:&\:&\:&\:&\:&\:&\:\cr\noalign{\hrule}
22&\:&\:&\:&\:&\:&\:&\:&\:\cr\noalign{\hrule}
23&\:&\:&\:&\:&\:&\:&\:&\:\cr\noalign{\hrule}
24&\:&\:&\:&\:&\:&\:&\:&\:\cr\noalign{\hrule}
25&\:&\:&\:&\:&\:&\:&\:&\:\cr\noalign{\hrule}
26&\:&\:&\:&\:&\:&\:&\:&\:\cr\noalign{\hrule}
27&\:&\:&\:&\:&\:&\:&\:&\:\cr\noalign{\hrule}
30&\:&\:&\:&\:&\:&\:&\:&\:\cr\noalign{\hrule}
31&\:&\:&\:&\:&\:&\:&\:&\:\cr\noalign{\hrule}
32&\:&\:&\:&\:&\:&\:&\:&\:\cr\noalign{\hrule}
33&\:&\:&\:&\:&\:&\:&\:&\:\cr\noalign{\hrule}
34&\:&\:&\:&\:&\:&\:&\:&\:\cr\noalign{\hrule}
35&\:&\:&\:&\:&\:&\:&\:&\:\cr\noalign{\hrule}
36&\:&\:&\:&\:&\:&\:&\:&\:\cr\noalign{\hrule}
37&\:&\:&\:&\:&\:&\:&\:&\:\cr\noalign{\hrule}}}
}
}
\endinsert

With all the s\plus letter combinations and all the vowels with
diacritics, it turns out that not even 256 characters are enough.
Since I couldn't push the font size any further, I decided to
eliminate some of the vowel\plus diacritic combinations.  The obvious
candidates were the combinations of breathings with grave accent,
which can only occur in a restricted number of monosyllables, and
thus can be typeset as accents, because no hyphenation is required
anyway.  So I made the characters |<| and |>| active.  Depending on
whether or not the next character is $`$, these active characters
expand to an |\accent| or to a |\char|, the latter meant to form
a ligature with what comes next.  (Actually, things are not quite
so simple.  A breathing or accent over a capital vowel is
traditionally written before the letter, so the |\accent| control
sequence is only emitted if the following character is lowercase.)

The complete layout of the fonts is shown on the next page.  There
are a few unfilled positions, two of which I'm saving for the digamma,
whose design I haven't yet tackled.

\heading Other Problems \endheading

In order to write continuous text, I had to prepare a modern Greek
hyphenation table, which I'm currently testing. 
The hyphenation of modern Greek follows closely that of ancient Greek,
which is straightforward (that is, described by a fairly short set of
rules) because Greek had originally a phonetic script---one letter
for each phoneme and vice versa.  This conservatism means that
hyphenation does not necessarily occur between syllables, as in
the word $stau-r'os$ `cross', now pronounced
[sta\'{}vros]; but this apparently doesn't bother anyone.

One difficulty, however, cannot be resolved by means of mechanical
rules: the digraphs $mp$, $nt$ and $gk$ are sometimes pronounced as
nasal\plus voiced stop, sometimes as voiced stop alone, depending on
the word (also sometimes on the speaker).  In the first case, the group
can be split, but in the second, one should hyphenate before the group.
This problem seems to be solvable only by trial and error.

The last problem I want to discuss is that of device drivers.
We have been using various |dvi|-to-PostScript drivers in Princeton,
and I found that none of them would work with
256-character fonts, even though both \TeX\ and PostScript are
designed to handle such fonts.  I was, however, able to adapt Nelson
Beebe's excellent driver |dvialw| [Beebe 1987] after making only
three changes, because the program is well written and well documented.
I exhort all those who write |dvi| drivers to include 256-character
fonts in their design, since they will certainly become more necessary
as \TeX\ extends its reach around the world.

\heading Bibliography \endheading

\def\\{\smallskip\hangindent=\parindent\hangafter=1\noindent\ignorespaces}
\\
Nelson H. F. Beebe, Public domain \TeX\ DVI driver family,
{\it TUGboat}, {\bf 8}:1 (1987), 41--42.
\\
British Museum, {\it Greek Printing Types}, London, British Museum, 1927.
\\
Donald E. Knuth, The Letter S, {\it The Mathematical Intelligencer},
{\bf 2} (1980), 114--122.
\\
Stanley Morison, {\it John Fell: the University Press and the `Fell'
Types}, Oxford, Clarendon Press, 1967.
\\
J. M. Mosley, Porson's Greek Types, {\it Penrose Annual}, {\bf 54} (1960),
36--40.
\\
Reinhard Wonneberger, Typesetting `Normaltext', {\it TUGboat},
{\bf 8}:1 (1987), 63--72.
\end
