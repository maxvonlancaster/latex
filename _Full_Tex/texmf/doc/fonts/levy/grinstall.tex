\magnification=\magstep1
\def\.#1{{\tt #1}}
\def\disp#1{\line{\tt #1\hfil}}
\def\angles#1{{$\langle$\rm #1$\rangle$}}
\chardef\!`\\

\def\verbatim{\begingroup
  \def\do##1{\catcode`##1=12 } \dospecials
  \parskip 0pt \parindent 0pt
  \catcode`\ =13 \catcode`\^^M=13
  \tt \catcode`\?=0 \verbatimdefs \verbatimgobble}
{\catcode`\^^M=13{\catcode`\ =13\gdef\verbatimdefs{\def^^M{\ \par}\let =\ }} %
  \gdef\verbatimgobble#1^^M{}}

\centerline{INSTALLATION INSTRUCTIONS}

\medskip

(Note to non-UNIX sites: These instructions should be followed verbatim
only on systems running UNIX.  I've tried to make the instructions 
understandable to people without knowledge of UNIX, so that if you're
familiar with your own operating system it should not be hard to
figure out the right thing to do under it.  If the separator between
components of a file pathname is not `/', you should change the
definition of \.{readfrom} in \.{grbase.mf}.

See the \.{README} file for an important note concerning required
METAFONT capabilities.)

In order to make the fonts, you need an executable version of
METAFONT, version 1.0 or newer, with the `plain' base preloaded
(see p.~279 of {\it The METAFONTbook\/} if you're not familiar with this).
You will also need the \.{cmbase.mf} file, which is distributed
with METAFONT, unless you have an executable METAFONT with that
file already preloaded.  The shell script (batch file) \.{makefont}
assumes that \.{cmmf} is an executable METAFONT with \.{cmbase.mf}
preloaded; if you don't have that you can replace \.{cmmf} by
\.{mf} in \.{makefont}.  (Non-UNIX sites: you'll have to edit
\.{makefont} anyway, as well as \.{testfont} and \.{makeall} below.
In each case the comments inside the file should guide you in finding
the right incantations under your operating system.)

If METAFONT has never been used at your site before, you will
also need to create a file \.{local.mf} and preload it together
with \.{plain.mf}.  (See pp.~278--9 of {\it The METAFONTbook}.)
This file
should contain the characteristics of the different output
devices that you will generate fonts for.  The parameters
for many output devices can be found on page 269 of the November
1987 issue of TUGboat; and two more devices (the Apple laserwriter
and Sun workstation) which are not in that table can be found
in \.{modes.mf} (in this distribution).  The values in \.{modes.mf}
are experimental; if you feel like fiddling with them and think
you can get better results, let me know.  In any case, your
\.{local.mf} should contain mode definitions for all output devices
you intend to use the fonts with, and it should be preloaded
together with \.{plain.mf}, so that when you issue the command
\.{mf} or \.{cmmf} all this information is already in memory.

Next you should test one font by saying

\smallskip
\disp{makefont grreg10 [\angles{modename}]}
\smallskip

(where \angles{modename} stands for \.{alw} if you're generating
for the laserwriter, for example; the default is
\.{localmode}, which should be defined in \.{local.mf}).
It will take a while to make the font, and when
it's done you can look in \.{grreg10.log} to see if everything went
smoothly.  If everything is OK, move \.{grreg10.tfm}
and \.{grreg10.NNNgf} to the system's \TeX\ font directory (often
\.{/usr/lib/tex/fonts}) and do a test run:

\smallskip
\disp{\% tex}
\disp{\angles{blurb...}}
\disp{**\!font\!test=grreg10 \!test ABC \!end}
\disp{\angles{more blurbs; hopefully no error messages}}
\disp{\%}
\smallskip

The text typed during this test should be very simple, as above:
no accents, ligatures, etc.  If you got a \.{texput.dvi} file without
problems, you should try to print it, using your local dvi driver.
Some dvi drivers cannot read font files of the form \.{*.NNNgf},
only those of the form \.{*.NNNpk}; you can use the utility
\.{gftopk}, which comes with the METAFONT distribution, to carry
out the conversion.  If your driver can only read \.{*.NNNpxl} fonts,
it's time to get a new one.

Another problem with some dvi drivers is that they don't
take 256 character fonts, although they should, since 256 characters
is part of standard \TeX.  (The reason is, presumably, that Computer
Modern fonts only have 128 characters, and programmers were counting
on all other fonts being similarly restricted.)  To test this
aspect of your driver, create a dvi file with a character in the
range 128-255:

\smallskip
\disp{\% tex}
\disp{\angles{blurb...}}
\disp{**\!font\!test=grreg10 \!test 'a \!end}
\disp{\angles{more blurbs}}
\disp{\%}
\smallskip

When you try to print this, you may get an error message, or
a core dump, or simply a blank page.  If so, you will have to
fix your driver or get a new one.  Fixing your driver involves
understanding fairly well the dvi format (carefully explained
in {\it\TeX: The Program}), but should not be beyond anyone with
a modicum of systems programming experience.  Getting a driver
that works is probably easier if you're well connected,
electronically speaking.  I recommend Nelson Beebe's family
of drivers (page 41 of the April 1987 TUGboat).  Here are
some hints from Nelson Beebe for obtaining his drivers: 

\medskip
\begingroup
\verbatim
> From BEEBE@SCIENCE.UTAH.EDU Mon Dec 21 10:28:08 1987
> From: "Nelson H.F. Beebe" <Beebe@SCIENCE.UTAH.EDU>
> Subject: Re: dvi drivers
> X-Us-Mail: "Center for Scientific Computing, South Physics,
>   University of Utah, Salt Lake City, UT 84112"
> X-Telephone: (801) 581-5254
> 
> Internet users may retrieve the DVI driver family via ANONYMOUS
> FTP to SCIENCE.UTAH.EDU; get the via 00README.TXT in the login
> directory for details.  The files 00PCDOS.TXT, 00TOPS20.TXT and
> 00VMSSETUP.TXT in the same directory describe TeX and DVI
> directory layouts and system logical names on IBM PC DOS, TOPS-20
> and VAX VMS.  Compressed Unix tar files are available in several
> directories.
> 
> VAX VMS BACKUP savesets (compressed and uncompressed) are
> available via ANONYMOUS FTP to CTRSCI.UTAH.EDU; the file
> 00README.TXT in the login directory has further details.
> 
> Australian ACSnet users may retrieve the DVI drivers from
> LaTrobe University, Bundoora, Victoria; send mail to
> 	"munnari!latvax8.lat.oz.au!ccmk"@uunet.uu.net
> for retrieval details.
> 
> European Bitnet users may contact $92@dhdurz1.bitnet
> at the University of Heidelberg.  Here is how:
> 	With the command directed at LISTSERV at DHDURZ1.BITNET
> 		GET DRIVER FILELIST
> 	you will receive a directory from the driver
> 	subdirectory. To receive a file:
> 		GET filename filetype
> 
> European DECNET users may retrieve the DVI drivers (and a
> large collection of other TeX-related software) from the
> University of Padova in Padova, Italy; send mail to 
> 	CALVANI%VAXFPD.INFNET@IBOINFN.BITNET
> for retrieval details.
> 
> UK Janet users should contact AbbottP@uk.ac.aston.mail (Aston
> University).  Here is the latest retrieval information from
> TeXHaX #92 (03-Nov-87):
> 	The DVI family of drivers software is now stored
> 	under (and has been for some time) public.texdvi208.
> 	I have been notified that version 2.10 is being
> 	shipped and this will be made available under
> 	public.texdv210. The index for version 2.08 is
> 		aston.kirk::[public.texdvi208]000index.list
> 	If you want to know the latest state of the
> 	archive the file is
> 		aston.kirk::[public]000aston.readme
?endgroup
\endgroup
\medskip

Once you've got the driver working, you can type

\smallskip
\disp{testfont grreg10}
\smallskip

for a full-blown test of the font \.{grreg10}.  This creates
a dvi file \.{grtestfont.dvi} and moves it to \.{grreg10.dvi}, which you
should then print.

Finally, you can say

\smallskip
\disp{makeall [ \angles{modename} [ \angles{mag} ]]}
\smallskip

for all modes and magnifications desired.  Don't forget to move
the \.{*.tfm} and \.{*.NNNgf} files thus obtained to their destination
directory.

The last step is to TeX the user's manual page (\.{doc/readme.tex})
and make it available to users.
\end
