\documentclass[ngerman]{libertinedoku}
\begin{document}
\pageTitle{Biolinum - Shadow}

\section{Einbindung}

Der Biolinum-Shadow-Font steht mit den Schnitten

\texttt{fxb-s-n}, \texttt{fxb-s-it}, \texttt{fxb-s-b},
\texttt{fxb-s-sc}, \texttt{fxb-s-ic} und \texttt{fxb-s-bc}

zur Verfügung.

\section{Aufruf}

\verb|\usefont{T1}{fxb}{s}{n}\fontsize{1.5cm}{1.6cm}\selectfont|

oder

\verb|\usefont{T1}{fxb}{s}{it}\fontsize{1.5cm}{1.6cm}\selectfont|

oder

\verb|\usefont{T1}{fxb}{s}{b}\fontsize{1.5cm}{1.6cm}\selectfont|

\subsection{Beispiel: Shadow - normal}

{\usefont{T1}{fxb}{s}{n}\fontsize{1.5cm}{1.6cm}\selectfont
Dies ist ein Beispiel!\par 0123456789\par}

\subsection{Beispiel: Shadow - italic}

{\usefont{T1}{fxb}{s}{it}\fontsize{1.5cm}{1.6cm}\selectfont
Dies ist ein Beispiel!\par 0123456789\par}

\subsection{Beispiel: Shadow - bold}

{\usefont{T1}{fxb}{s}{b}\fontsize{1.5cm}{1.6cm}\selectfont
Dies ist ein Beispiel!\par 0123456789\par}


\subsection{Beispiel: Shadow - normal - sc}

{\usefont{T1}{fxb}{s}{sc}\fontsize{1.5cm}{1.6cm}\selectfont
Dies ist ein Beispiel!\par 0123456789\par}

\subsection{Beispiel: Shadow - italic - sc}

{\usefont{T1}{fxb}{s}{ic}\fontsize{1.5cm}{1.6cm}\selectfont
Dies ist ein Beispiel!\par 0123456789\par}

\subsection{Beispiel: Shadow - bold - sc}

{\usefont{T1}{fxb}{s}{bc}\fontsize{1.5cm}{1.6cm}\selectfont
Dies ist ein Beispiel!\par 0123456789\par}


\end{document}
