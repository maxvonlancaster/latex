\documentclass[ngerman]{libertinedoku}

\usepackage{eqlist}
\usepackage{libertinecomp}

\begin{document}
\pageTitle{Libertine \& Biolinum: Kompatiblitäts-Modus}

\section{für Umsteiger}

In der aktuellen Version stehen für die Glyphenauswahl folgende Makros zur Verfügung:

\begin{eqlist}[\def\makelabel#1{\textbf{\bs\texttt{#1}}}]
\item [libertineGlyph\{<glyphname>\}] Verwendung eines Zeichens mit dem Glyphnamen (Font Libertine).
\item [biolinumGlyph\{<glyphname>\}] Verwendung eines Zeichens mit dem Glyphnamen (Font Biolinum).
\end{eqlist}

Um das alte Makro \verb|\useTextGlyph| für alte Dokumente weiter zu verwenden, kann das
Makro über das Paket \texttt{libertinecomp.sty} zur Verfügung gestellt werden.

\begin{minipage}{10cm}
\verb|{\Huge\useTextGlyph{fxl}{uni211A}}| \hfill {\Huge\useTextGlyph{fxl}{uni211A}} \par
\verb|{\Huge\useTextGlyph{fxl}{uni263A}}| \hfill {\Huge\useTextGlyph{fxl}{uni263A}} \par
\verb|{\Huge\useTextGlyph{fxl}{Tux}}| \hfill {\Huge\useTextGlyph{fxl}{Tux}} \par

\bigskip
\verb|{\Huge\useTextGlyph{fxb}{uni211A}}| \hfill {\Huge\useTextGlyph{fxb}{uni211A}} \par
\end{minipage}

\end{document}
