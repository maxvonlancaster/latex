\documentclass[ngerman]{libertinedoku}



\begin{document}
\pageTitle{Libertine Glyphen}

\section{Libertine}

{%
\setlength{\columnseprule}{.5pt}
\setlength{\columnsep}{1cm}
\begin{multicols}{3}
   \renewcommand*\DeclareTextGlyphX[5]{\index{#4 (Libertine)}%
   \makebox[3cm][l]{%\hypertarget{glyph.#4}{}\hyperlink{gglyph.#4}{#4}
   \texttt{#4}}\hfill%
   {\Huge\fbox{\libertineGlyph{#4}\strut}}\hfill\mbox{}\newline}
   \catcode`\_=12%
   \InputIfFileExists{fxl.inc}{}{}
\end{multicols}
}

\newpage
\section{Biolinum}

{%
\setlength{\columnseprule}{.5pt}
\setlength{\columnsep}{1cm}
\begin{multicols}{3}
   \renewcommand*\DeclareTextGlyphX[5]{\index{#4 (Biolinum)}%
   \makebox[3cm][l]{%\hypertarget{glyph.#4}{}\hyperlink{gglyph.#4}{#4}
   \texttt{#4}}\hfill%
   {\Huge\fbox{\libertineGlyph{#4}\strut}}\hfill\mbox{}\newline}
   \catcode`\_=12%
   \InputIfFileExists{fxb.inc}{}{}
\end{multicols}
}

\section{Index}

{%
\catcode`\_=12%
\def\indexcolumn{3}
\printindex
}
\end{document}
