\documentclass[ngerman]{libertinedoku}

\usepackage{libertinelist}
\usepackage{forloop}

\begin{document}
\pageTitle{Aufzählungslisten}

\section{Nummern}

Spezielle Nummern lassen sich über \verb|\useListNumberGlyph[<Bereich>]{<Nummer>}|
ansprechen.

\bigskip
{\Huge
\useListNumberGlyph{1}
\useListNumberGlyph{2}
\useListNumberGlyph{3}
\useListNumberGlyph{4}
\dots
\useListNumberGlyph{20}
}

\bigskip
Standardmäßig wird der Bereich \texttt{circle} verwendet. Dieser wurde über das Makro\\
\verb|\DeclareListNumberGlyph{circle}{1}{uni2460}|\\ definiert. Dabei wird der Bereich,
die Nummer und der zu verwendende Glyphname angegeben. Der Bereich \texttt{circle} kann
Zahlen von 1 bis 20 darstellen.


\section{Aufzählung mit Nummern}

Mit der Umgebung \texttt{libertineenumerate} wird eine Aufzählungsumgebung wie
\texttt{enumerate} zur Verfügung gestellt. Mit dem optionalen Parameter
wird der Bereich für die Nummern festgelegt.

\subsection{circle}

Der Bereich \texttt{circle} kann Zahlen von 1 bis 20 darstellen.

\forloop{enumi}{1}{\value{enumi}<21}{\useListNumberGlyph[circle]{\arabic{enumi}}\quad}

\begin{minipage}{\linewidth}
\begin{minipage}{.48\linewidth}
\begin{lstlisting}
\begin{libertineenumerate}
\item Punkt 1
\item Punkt 2
\item Punkt 3
\end{libertineenumerate}
\end{lstlisting}
\end{minipage}\hfill
\begin{minipage}{.48\linewidth}
\begin{libertineenumerate}
\item Punkt 1
\item Punkt 2
\item Punkt 3
\end{libertineenumerate}
\end{minipage}
\end{minipage}

\subsection{bracket}

Der Bereich \texttt{bracket} kann Zahlen von 1 bis 20 darstellen.

\forloop{enumi}{1}{\value{enumi}<21}{\useListNumberGlyph[bracket]{\arabic{enumi}}\quad}

\begin{minipage}{\linewidth}
\begin{minipage}{.48\linewidth}
\begin{lstlisting}
\begin{libertineenumerate}[bracket]
\item Punkt 1
\item Punkt 2
\item Punkt 3
\end{libertineenumerate}
\end{lstlisting}
\end{minipage}\hfill
\begin{minipage}{.48\linewidth}
\begin{libertineenumerate}[bracket]
\item Punkt 1
\item Punkt 2
\item Punkt 3
\end{libertineenumerate}
\end{minipage}
\end{minipage}

\subsection{Alpha}

Der Bereich \texttt{Alpha} kann Zahlen von 1 bis 26 in Großbuchstaben und
\texttt{alpha} in Kleinbuchstaben darstellen.

\forloop{enumi}{1}{\value{enumi}<27}{\useListNumberGlyph[Alpha]{\arabic{enumi}}\quad}

\forloop{enumi}{1}{\value{enumi}<27}{\useListNumberGlyph[alpha]{\arabic{enumi}}\quad}


\begin{minipage}{\linewidth}
\begin{minipage}{.48\linewidth}
\begin{lstlisting}
\begin{libertineenumerate}[Alpha]
\item Punkt 1
\item Punkt 2
\item Punkt 3
\end{libertineenumerate}
\end{lstlisting}
\end{minipage}\hfill
\begin{minipage}{.48\linewidth}
\begin{libertineenumerate}[Alpha]
\item Punkt 1
\item Punkt 2
\item Punkt 3
\end{libertineenumerate}
\end{minipage}
\end{minipage}

\begin{minipage}{\linewidth}
\begin{minipage}{.48\linewidth}
\begin{lstlisting}
\begin{libertineenumerate}[alpha]
\item Punkt 1
\item Punkt 2
\item Punkt 3
\end{libertineenumerate}
\end{lstlisting}
\end{minipage}\hfill
\begin{minipage}{.48\linewidth}
\begin{libertineenumerate}[alpha]
\item Punkt 1
\item Punkt 2
\item Punkt 3
\end{libertineenumerate}
\end{minipage}
\end{minipage}

\subsection{doublecircle}

Der Bereich \texttt{doublecircle} kann Zahlen von 1 bis 10 darstellen.

\forloop{enumi}{1}{\value{enumi}<21}{\useListNumberGlyph[doublecircle]{\arabic{enumi}}\quad}


\begin{minipage}{\linewidth}
\begin{minipage}{.48\linewidth}
\begin{lstlisting}
\begin{libertineenumerate}[doublecircle]
\item Punkt 1
\item Punkt 2
\item Punkt 3
\end{libertineenumerate}
\end{lstlisting}
\end{minipage}\hfill
\begin{minipage}{.48\linewidth}
\begin{libertineenumerate}[doublecircle]
\item Punkt 1
\item Punkt 2
\item Punkt 3
\end{libertineenumerate}
\end{minipage}
\end{minipage}

\subsection{darkcircle}

Der Bereich \texttt{darkcircle} kann Zahlen von 1 bis 20 darstellen.

\forloop{enumi}{1}{\value{enumi}<21}{\useListNumberGlyph[darkcircle]{\arabic{enumi}}\quad}

\begin{minipage}{\linewidth}
\begin{minipage}{.48\linewidth}
\begin{lstlisting}
\begin{libertineenumerate}[darkcircle]
\item Punkt 1
\item Punkt 2
\item Punkt 3
\end{libertineenumerate}
\end{lstlisting}
\end{minipage}\hfill
\begin{minipage}{.48\linewidth}
\begin{libertineenumerate}[darkcircle]
\item Punkt 1
\item Punkt 2
\item Punkt 3
\end{libertineenumerate}
\end{minipage}
\end{minipage}





\end{document}
