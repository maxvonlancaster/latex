\documentclass[ngerman]{libertinedoku}

\usepackage{eqlist}

\begin{document}
\pageTitle{Libertine \& Biolinum}

\section{Einbindung}

Der Libertine-Font wird über
\begin{lstlisting}
\usepackage{libertine}
\end{lstlisting}
eingebunden. Er steht dann unter \texttt{pdflatex} und \texttt{xelatex} zur Verfügung.
Für die Serifenschrift wird dabei Libertine und für die serifenlose Schrift Biolinum verwendet.

\section{Optionen}

Es stehen dabei folgende Optionen zur Verfügung:

\begin{eqlist}[\def\makelabel#1{\textbf{#1}}]
\item [nf]  Es werden normale Ziffern verwendet (Grundeinstellung).
\item [osf] Es werden anstelle der normalen Ziffern Medivalziffern bzw. Minuskelziffern verwendet.
\item [ss]  Es wird \textsc{ss} anstelle von \textsc{ß} verwendet.
\item [osfss] Es wird \textsc{ss} anstelle von \textsc{ß} und Medivalziffern verwendet.
\item [scaled] Skaliert den Font.
\end{eqlist}

\section{Makros}

Es stehen folgende Makros zur Verfügung:

\begin{eqlist}[\def\makelabel#1{\textbf{\bs\texttt{#1}}}]
\item [libertine] Schaltet auf den Libertine-Font um.
\item [biolinum] Schaltet auf den Biolinum-Font um.
\item [libertineGlyph\{<glyphname>\}] Verwendung eines Zeichens mit dem Glyphnamen (Font Libertine).
\item [biolinumGlyph\{<glyphname>\}] Verwendung eines Zeichens mit dem Glyphnamen (Font Biolinum).
\end{eqlist}

\begin{minipage}{10cm}
\verb|{\Huge\libertineGlyph{uni211A}}| \hfill {\Huge\libertineGlyph{uni211A}} \par\medskip
\verb|{\Huge\libertineGlyph{uni263A}}| \hfill {\Huge\libertineGlyph{uni263A}} \par\medskip
\verb|{\Huge\libertineGlyph{Tux}}| \hfill {\Huge\libertineGlyph{Tux}} \par\medskip
\verb|{\Huge\libertineGlyph{uni20A8}}| \hfill {\Huge\libertineGlyph{uni20A8}} \par

\bigskip
\verb|{\Huge\libertineGlyph{uni211A}}| \hfill {\Huge\libertineGlyph{uni211A}} \par\medskip
\verb|{\Huge\textbf{\libertineGlyph{uni211A}}}| \hfill{\Huge\textbf{\libertineGlyph{uni211A}}} \par\medskip
\verb|{\Huge\textit{\libertineGlyph{uni211A}}}| \hfill {\Huge\textit{\libertineGlyph{uni211A}}} \par\medskip

\end{minipage}

Eine Liste aller Glyphen findet sich in der Glyphentabelle.

\newpage
\section{Beliebige Fontauswahl}

\subsection{pdflatex}

Ansonsten können Sie jeden Teilbereich über z.\,B.
\begin{lstlisting}
\usefont{T1}{fxl}{m}{n}\selectfont
\end{lstlisting}
auswählen. Siehe hierzu auch die Fonttabellen.

Für \texttt{T1} stehen folgende Schnitte zur Verfügung.

\subsubsection{Libertine - fxl}

\begin{lstlisting}
\usefont{T1}{fxl}{m}{n}\selectfont
\usefont{T1}{fxl}{m}{it}\selectfont
\usefont{T1}{fxl}{m}{sl}\selectfont
\usefont{T1}{fxl}{b}{n}\selectfont
\usefont{T1}{fxl}{b}{it}\selectfont
\usefont{T1}{fxl}{b}{sl}\selectfont
\usefont{T1}{fxl}{m}{sc}\selectfont
\usefont{T1}{fxl}{b}{sc}\selectfont
\usefont{T1}{fxl}{m}{ic}\selectfont
\usefont{T1}{fxl}{b}{ic}\selectfont
\end{lstlisting}

\subsubsection{Biolinum - fxb}

\begin{lstlisting}
\usefont{T1}{fxb}{m}{n}\selectfont
\usefont{T1}{fxb}{m}{it}\selectfont
\usefont{T1}{fxb}{m}{sl}\selectfont
\usefont{T1}{fxb}{b}{n}\selectfont
\usefont{T1}{fxb}{b}{sl}\selectfont
\usefont{T1}{fxb}{m}{sc}\selectfont
\usefont{T1}{fxb}{b}{sc}\selectfont
\usefont{T1}{fxb}{o}{n}\selectfont
\usefont{T1}{fxb}{o}{it}\selectfont
\usefont{T1}{fxb}{o}{b}\selectfont
\usefont{T1}{fxb}{s}{n}\selectfont
\usefont{T1}{fxb}{s}{it}\selectfont
\usefont{T1}{fxb}{s}{b}\selectfont
\end{lstlisting}

\subsection{xelatex}

In \texttt{xelatex} wird der Stil des Fonts über \texttt{RawFeature}, z.\,B.
\texttt{RawFeature=+liga;+onum} festgelegt. Der eigentliche Font wird über seinen
Namen ausgewählt.

\begin{lstlisting}
Linux Libertine O
Linux Biolinum O
Linux Libertine O Bold
Linux Libertine O Italic
...
\end{lstlisting}

Mehr dazu im Paket \texttt{libertine.sty}.

\newpage
\section{Beispiele}

\subsection{Zahlen}

\begin{tabularx}{\linewidth}{lXXXXX}
\textbf{Stile} & \textbf{Libertine} &
                 \textbf{Libertine bold} & \textbf{Libertine italic} &
                 \textbf{Libertine bold italic} \\
\textbf{normale} & \usefontx{fxl}{m}{n}{\FontNumber}  &
                   \usefontx{fxl}{b}{n}{\FontNumber}  &
                   \usefontx{fxl}{m}{it}{\FontNumber}  &
                   \usefontx{fxl}{b}{it}{\FontNumber}  \\
\textbf{old stlye} & \usefontx{fxlj}{m}{n}{\FontNumber}  &
                   \usefontx{fxlj}{b}{n}{\FontNumber}  &
                   \usefontx{fxlj}{m}{it}{\FontNumber}  &
                   \usefontx{fxlj}{b}{it}{\FontNumber}  \\
\textbf{fitted} & \usefontx{fxlf}{m}{n}{\FontNumber}  &
                   \usefontx{fxlf}{b}{n}{\FontNumber}  &
                   \usefontx{fxlf}{m}{it}{\FontNumber}  &
                   \usefontx{fxlf}{b}{it}{\FontNumber}  \\

\end{tabularx}

\begin{tabularx}{\linewidth}{lXXXXX}
\textbf{Stile} & \textbf{Biolinum} &
                 \textbf{Biolinum bold} & \textbf{Biolinum italic} &
                 \textbf{Biolinum bold slanted} \\
\textbf{normale} & \usefontx{fxb}{m}{n}{\FontNumber}  &
                   \usefontx{fxb}{b}{n}{\FontNumber}  &
                   \usefontx{fxb}{m}{it}{\FontNumber}  &
                   \usefontx{fxb}{b}{sl}{\FontNumber}  \\
\textbf{old stlye} & \usefontx{fxbj}{m}{n}{\FontNumber}  &
                   \usefontx{fxbj}{b}{n}{\FontNumber}  &
                   \usefontx{fxbj}{m}{it}{\FontNumber}  &
                   \usefontx{fxbj}{b}{sl}{\FontNumber}  \\
\textbf{fitted} & \usefontx{fxbf}{m}{n}{\FontNumber}  &
                   \usefontx{fxbf}{b}{n}{\FontNumber}  &
                   \usefontx{fxbf}{m}{it}{\FontNumber}  &
                   \usefontx{fxbf}{b}{sl}{\FontNumber}  \\
\end{tabularx}


\end{document}
