% install.tex -- how to install Malvern fonts 
% Copyright 1994 P. Damian Cugley

%%% @TeX-document {
%%%   filename       = "install.tex",
%%%   version        = "X",
%%%   date           = "pdc 1994.09.26",
%%%   package        = "Malvern 1.2",
%%%   author         = "P. Damian Cugley",
%%%   email          = "damian.cugley@comlab.ox.ac.uk",
%%%   address        = "Oxford University Computing Laboratory,
%%%                     Parks Road, Oxford  OX1 3QD, UK",
%%%   codetable      = "USASCII",
%%%   keywords       = "Malvern, METAFONT, font, typefont, TeX",
%%%   supported      = "Maybe",
%%%   abstract       = "Installation summary for  the Malvern
%%%                     font family.",
%%%   dependencies   = "none",
%%% }

%  See the Malvern Handbook (maman.tex) for more info about Malvern.
%  This software is available freely but without warranty.
%  See the file COPYING for details.

%  This document is written in plain TeX and has NO dependencies
%  other than the fonts (of course) which are all standard CM fonts.
%  The hope is that this way anyone will be able to TeX this file
%  without it all going horribly wrong...

% To change the paper size, change the assignments to \hsize 
%  and \vsize below (after the definition of \margin).

\newdimen\margin \margin=25mm	% size of margins

%  Change the following two lines to select a different paper size:
\hsize=210mm \advance\hsize-2\margin
\vsize=297mm \advance\vsize-2\margin

% change the following 1 line if you do have the LOGO10 font:
\newif\iflogo \logofalse

\def\package{Malvern~1.2}
\def\ttpackage{Malvern-1.2}

%  Adjust \vsize to be a multiple of baselineskip + topskip:

\baselineskip=14pt
\newcount\counta
\counta=\vsize \advance\counta-\topskip 
\divide\counta\baselineskip
\vsize=\counta\baselineskip \advance\vsize\topskip

\hoffset=\margin \advance\hoffset-1in
\voffset=\margin \advance\voffset-1in

%
%  FONTS
%

\def\fnt#1#2{\font#1=#2 \fontdimen3#1=0pt \fontdimen4#1=0pt }

\fnt\twelverm{cmr12} 		\def\rm{\twelverm\fam0 }
\fnt\twelvei{cmmi12} 
\fnt\twelvesy{cmsy10 at 12pt}
\fnt\twelvebf{cmb10 at 12pt}	\def\bf{\twelvebf\fam\bffam}
\fnt\twelveit{cmti12}		\def\it{\twelveit\fam\itfam}
\fnt\twelvett{cmtt12} 		\def\tt{\twelvett\fam\ttfam}
\iflogo
    \fnt\twelvemf{logo10 scaled 1200} 	\def\mf{\twelvemf}
\fi
\fnt\twelvermc{cmcsc10 scaled 1200}	\def\sc#1{{\twelvermc \lowercase{#1}}}

%  Add just enough maths stuff:
\textfont0=\twelverm \scriptfont0=\tenrm \scriptscriptfont0=\sevenrm
\textfont1=\twelvei \scriptfont1=\teni \scriptscriptfont1=\seveni
\textfont2=\twelvesy \scriptfont2=\tensy \scriptscriptfont2=\sevensy
\textfont\itfam=\twelveit \scriptfont\itfam=\tenit  
\textfont\ttfam=\twelvett \scriptfont\ttfam=\tentt  

\fnt\titlerm{cmss17}
\fnt\headingrm{cmss12 scaled 1200}
\fnt\headingit{cmssi12 scaled 1200}
\fnt\subheadingrm{cmss12}
\fnt\subheadingit{cmssi12}

\rm
\rightskip=0pt plus 2em
\tolerance=1000
\emergencystretch=3em

%
%  MACROS
%

\def\.{\mathhexbox201}

\def\<#1>{$\langle\hbox{\it #1\/}\rangle$}

\def\allcapabbrev#1{{\the\scriptfont\fam #1}}

\iflogo
    \def\MF{{\mf META}\-{\mf FONT}}
\else
    \def\MF{\allcapabbrev{META}\-\allcapabbrev{FONT}}
\fi
\def\fontinst{{\tt font}\-{\tt inst}}

% No blank lines after section and subsection commands!
\newcount\seccount \seccount=0
\def\section#1%
{
    \medbreak
    \advance\seccount1 \subseccount=0
    \begingroup \headingrm \let\it=\headingit
	\rightskip=0pt plus 2em
	\parskip=0pt \parindent=0pt
	\the\seccount\enspace #1
	\nobreak\smallskip
    \endgroup
    \noindent\ignorespaces
}

\newcount\subseccount
\def\subsection#1%
{
    \smallbreak
    \advance\subseccount1
    \begingroup \subheadingrm \let\it=\subheadingit 
	\rightskip=0pt plus 2em
	\parskip=0pt \parindent=0pt
	\the\seccount\sfcdot\the\subseccount\enspace #1\quad
    \endgroup
    \ignorespaces
}

\def\sfcdot{\leavevmode\raise0.5ex\hbox{.}}

\def\begintable
{%
    $$\vbox\bgroup\halign\bgroup
	##\hfil&&\quad##\hfil\cr
	\noalign{\hrule height 1pt \kern2\jot}%
}
\def\endtable
{%
	\crcr
	\noalign{\kern1.5\jot \hrule height 1pt}%
    	\egroup\egroup$$
    \ignorespaces
}


\def\flushtop#1%
{{%
    \setbox0=\hbox{#1}\setbox1=\hbox{X}%
    \dimen0=\ht1 \advance\dimen0-\ht0
    \raise\dimen0\box0
}}

\def\pt{\,{\rm  pt}} \mathcode`.="0201
\def\La{L\kern-0.15em\flushtop{a}}
\def\LaTeX{\La\kern-0.05em \TeX}

%
%  DOCUMENT PROPER STARTS HERE
%

\leftline{\titlerm Installing Malvern fonts}
\bigskip

%  Print my address:

\rightline{\vbox{\halign{\strut#\hfil\cr
	P. Damian Cugley\cr
\noalign{\smallskip}%
	Oxford University Computing Laboratory\cr
   	Parks Road\cr
	Oxford \sc{OX$\mit1\ 3$QD}\cr
	UK\cr
\noalign{\smallskip}%
	\tt Damian.Cugley@comlab.ox.ac.uk\cr
}}}
\bigskip

	(Where this document and the corresponding chapter in the {\it
	Handbook} differ, this version is probably the more up-to-date.)

\section{About Malvern}
	Malvern is a font family designed and implemented in \MF.  It is
	a sanserif font intended mainly for non-technical documents.  It
	does not blend particularly well with the Computer Modern fonts.

\subsection{Generic driver file}
	Malvern is unusual amongst \MF\ families in that it is supplied
	with one {\it generic driver file} which could in theory be used
	instead of having one driver file for each size and style the
	family comes in.\footnote*{The package includes a selection of
	driver files nevertheless.} The generic driver file ({\tt
	ma.mf}) inspects the values of variables like {\it designsize}
	and {\it weight} and gives the ad-hoc parameters appropriate
	values, before reading the program files.  On some UNIX systems
	a small program called {\tt mff} can be used to generate fonts
	using this system.  Otherwise other measures will have to be
	taken, described below.

\subsection{Malvern encodings}
	Malvern uses a nonstandard encoding -- in fact a family of
	encodings, for example, Malvern~A (latin alphabets) and
	Malvern~G (Greek).  To create fonts for use with \LaTeX~2e and
	the NFSS~2 font selection macros, use Alan Jeffrey's {\tt
	fontinst} package.  This generates the {\tt fd} files and
	virtual fonts so that Malvern fonts may be used in the same way
	as other \LaTeX~2e families.  (The glyphs needed to make fonts
	with the Cork (T1) encoding are in Malvern~A and Malvern~B.)
	This has already been done to make a selection of styles
	available with both \TeX\ Text and T1 encodings.

\subsection{Malvern font names}
	The Malvern fonts generated with \MF\ will have \TeX\ names of
	the following form:
$$
	\hbox{\tt ma \it style enc size}
$$
 	where the {\it style} is a two-digit style code (described
	below), {\it enc} is one- or two-letter code identifying a
	Malvern encoding (such as `{\tt a}' or `{\tt az}'), and {\it
	size} is the size in points (e.g., `{\tt 12}' for $12\pt$, `{\tt
	17p28}' for $17.28\pt$). Thus {\tt ma55a12} (12-pt Malvern 55,
	encoding~A), {\tt ma56g12} (12-pt Malvern~56, Greek encoding).

	The two-letter style codes are based on Adrien Frutiger's
	system, devised for the Univers family.  The first digit
	specifies the weight, from 1 (ultra-light) to 9 (ultra-bold),
	with 5 meaning medium and 7 bold.  The second digit gives the
	width: 5 or 6 is normal width, 7 or 8 is compressed, 3 or 4 is
	expanded.  An odd number means an upright font, an even number
	italic.  Thus Malvern~55 is roman, Malvern~56 is italic,
	Malvern~75 boldface, Malvern~68 is demibold compressed italic.

\edef\refstylecodes{\S\thinspace\the\seccount\.\the\subseccount}

	The \MF\ programs are also able to generate fonts named in Karl
	Berry's system,\footnote\dag{Karl Berry, {\it File Names for \TeX\
	Fonts}, version 1\.6 ({\tt
	file:\allowbreak//ftp.cs.umb.edu\slash pub\slash tex\slash fontname}).}
	starting with `{\tt fmv}'.  The correspondance between Malvern's
	style digits and weight, variant and expansion letters is given
	in the tables below.  The encoding used for these fonts is \TeX\
	text, so for example {\tt fmvr12} and {\tt ma55s12} are names
	for the same font.  This is so that these fonts may be used in
	plain-\TeX\ and \LaTeX-2\.09 documents without too much
	confusion.  NFSS~2 systems (indeed, any that use non-\TeX-text
	encodings) will use virtual fonts.

	As of October 1994, however, the font names standard does not
	allow for combinations of more than two variants (where
	`variants' includes encodings and slant) without breaking the
	eight-character limit required by stupid filesystems like MS-DOS
	and ISO~9960.  As an extreme example, Malvern~58 with the
	caps-and-small-caps, old-style-numerals variation of the \TeX\
	Text encoding is {\tt fmvr9ci7tc12} (12 chars).  The only
	temporary solution I can offer is to extend the naming scheme
	used for the `{\tt ma}-' fonts to include encoding codes for the
	virtual fonts needed for plain \LaTeX, NFSS and NFSS~2 (e.g.,
	{\tt ma58ab12}).  Therefore, the ability to create {\tt fmv}-
	fonts is not being used at present.

\section{Unpacking Malvern}
	The Malvern distribution includes \MF\ source files, some \TeX\
	files such as this documentation, and a few other miscellaneous
	files.  

	All files have names that should work with MS-DOS and ISO-9960
	filesystems.

\subsection{Source files ({\tt/source})}
	On most \TeX\ systems, \MF\ fonts end up with their source files
	stored in one directory.  I~suggest that Malvern instead be
	given its own file area.  This avoids problems with different
	font families having files with the same name, and makes it easy
	to replace all the Malvern files in one go if you upgrade to a
	newer version of Malvern.

	With a new-style directory tree,\footnote*{At the time of
	writing, the \TeX\ Directory Standard (TeDiouS) is still in
	discussion.\break   This is my best guess based on a \TeX\ system I
	recently installed.} the {\tt\ttpackage} directory may be
	placed in the `{\tt texmf/fonts/public}' directory, and the {\tt
	source} directory and {\tt driver} directories merged and
	renamed so that \MF\ can find them.

	The files {\tt mff.rc} and {\tt fmv.mff} are input files for
	{\tt mff} and may be ignored if you are not using {\tt mff} to
	generate fonts.  If the Malvern files are being mixed in with a
	bunch of others, rename {\tt mff.rc} to {\tt ma.mff}.

\subsection{Driver files ({\tt/drivers})}
	Fonts can be generated by including assignments on the \MF\
	command line, but driver files are more convenient (unless you
	have {\tt mff}).  These are also used when assignments to the
	generic driver file's parameters on the \MF\ command line is
	impossible, such as when using the standard {\tt MakeTeXPK}
	script.  A driver file is included for each {\tt tfm} supplied.

\subsection{\TeX\ input files ({\tt/tex})}
	These files belong in a system-wide \TeX\ inputs area.  The
	files ending -{\tt.fd} are font family description files, used
	by NFSS~2.

	On new-style directory trees, the best approach is probably to
	link or rename the {\tt texmf/fonts/public/\ttpackage/tex}
	directory to {\tt texmf/tex/\ttpackage}.

\subsection{Documentation files ({\tt/doc}, {\tt/dvi})}
	Files ending in `{\tt.tex}' are plain \TeX\ documents, and will
	not work with \LaTeX.  Compiled ({\tt dvi}) files are supplied
	to save the installer having to run \TeX.  Read this document,
	{\tt install.tex}, first.  The Greek text encoding is described
	in {\tt magrman.tex}.


	Please do not install any of the macro files used to typeset the
	documentation in the system-wide \TeX\ inputs area.  They are
	not supported, not necessarily generally useful, and very
	nonstandard.  Earlier releases of Malvern gave the macro files
	generic enough names that they might clash with other macro
	files.  They have been renamed to start with `{\tt pdc}' in the
	hope that this will avoid clashes if they are accidentally
	installed.

\subsection{Virtual font files ({\tt/vf})}
	These virtual fonts are generated using Alan Jeffrey's
	\fontinst\ package, and are Malvern fonts with the Cork encoding
	(and some of the trickier \TeX\ Text variations).  With NFSS~2
	they are used by specifying encoding `{\tt T1}' and family `{\tt
	fmv}'.s

	The virtual fonts may have the following encoding codes:
\begintable
	\it Encoding&\it Code&\it Description\cr
\noalign{\kern\jot\hrule\kern1.5\jot}
	OT1c&	\tt aa&
		\TeX\ text with caps\&small caps\cr
	OT1c9&	\tt ab&
		ditto, with old-style numerals\cr
	T1&	\tt ar&
		Cork\cr
	T19&	\tt as&
		Cork, with old style numerals\cr
	T1c&	\tt at&
		Cork, with caps\&small caps\cr
	T1c9&	\tt au&
		Cork, with c\&sc and o.s. numerals\cr
\endtable
	For example, {\tt ma55ar12} has encoding T1.  Note that
	replacing lower-case letters with small-capitals counts as a
	different encoding scheme rather than a `font shape'.

\subsection{Font metric files ({\tt/tfm})}
	A selection of precompiled font metric files, including those
	for virtual fonts in {\tt /vf} and the actual fonts needed to
	use them, as well as fonts needed to print the documentation in
	{\tt/dvi}.

\section{Using mff to generate Malvern fonts}
	First, install {\tt mff} and arrange that \MF\ fill be able to
	find the {\tt mf} files.  Then to create a Malvern font, for
	example 12-pt Malvern~55 ({\tt ma55a12}), give the command:
$$
	\hbox{\tt mff ma55a12}
$$
	To generate Malvern~55 and 56 at magsteps 0, half and 1, for 300-dpi
	and 1000-dpi printers, you would type:
$$
	\hbox{\tt mff --magstep=0,h,1 
		--dpi=300,1000 ma55a10 ma56a10}
$$
	If all goes well, the {\tt tfm} and {\tt pk} files generated
	will be installed in the correct directories automatically.

\section{Generating Malvern fonts without mff}
	This section presumes you know how to install a normal \MF\
	font.

	Create a driver file for each style of Malvern you want to be
	able to use.  It should have the following form:
\begintable
\tt	\% \<name of file> -- generate \<size>-pt Malvern \<style> \cr
\noalign{\smallskip}
\tt	font\char95 size \<size> pt\#; \cr
\tt	encoding = \<number>; \cr
\tt	weight = \<number>; hratio = \<number>; \cr
\tt	slant = \<number>; italicness = \<number>; \cr
\tt	input ma \cr
\tt	bye \cr
\endtable
 	where \<size> is the design size desired in points, and the
	values for the various variables are given below.  The \<style>
	is the two-digit style code described in \refstylecodes.
%  You
%	might want to append an verbose description of the style:
%$$\hbox{\tt \% ma76a12.mf -- generate 12-pt Malvern 76 bold italic}$$ 
	To
	generate '{\tt fmv}-' fonts, the second-last line is changed to
	`{\tt input fmv}'.

	These driver files should be named after the font they
	correspond to -- for example, `{\tt ma76a12.mf}' to generate
	Malvern~76.  Then they are used as usual with \MF:
$$
	\hbox{\tt mf \char92 mode=luxo; mag={\it m}; input ma55a12}
$$
 	where {\it m} is the magnification wanted, or `{\tt 1.0}' for no
	magnification.

\subsection{Values for {\it encoding}}
	The variable {\it encoding} specifies the encoding to use -- in
	other words, the subset of Malvern's glyphs to be generated.
	It should be given one of the following values:
\begintable
	\it Encoding&\it Code&\it Description\cr
\noalign{\kern\jot\hrule\kern1.5\jot}
	${\it encoding} = 1$&	\tt a&
		Malvern A (latin alphabets {\it a--z}, numerals, etc.)\cr
	${\it encoding} = 2$&	\tt b&
		Malvern B (superscripts, symbols etc.)\cr
	${\it encoding} = 3$&	\tt c&
		Malvern C (Cyrillic)\cr
	${\it encoding} = 5$&	\tt e&
		Malvern E (composite letters {\it \`a--\v z})\cr
	${\it encoding} = 7$&	 \tt g&
		Malvern G (Greek letters $\alpha$--$\omega$, 
			$\it\acute\alpha$--$\it\tilde\omega$)\cr
	${\it encoding} = 19$&	\tt s& \TeX\ text (the same as CM Roman)\cr
	${\it encoding} = 26$&	\tt az& \TeX\ text, old-style numerals\cr
\endtable
 	The code is the used in the font name -- for example, {\tt
	ma55s10} for Malvern with the \TeX\ text encoding (${\it
	encoding} = 19$), and so on.

	The Malvern programs used to attempt to generate other encoding
	schemes; with {\tt fontinst} this all becomes redundant and
	those codes are obsolete.

\subsection{Values for {\it weight}}
	The variable {\it weight} specifies the weight (boldness) of the
	font.  It has the following values:
\begintable
	\it Weight&\it Style&\it NFSS~2& \it Berry&\it Description\cr
		  &\it digit&\it code&\it code\cr
\noalign{\kern\jot\hrule\kern1.5\jot}
	${\it weight} = 1/4$&	\tt 1& \tt ul& \tt t&	ultra-light\cr
	${\it weight} = 1/2$&	\tt 2& \tt el& \tt i& 	extra-light\cr
	${\it weight} = 3/4$&	\tt 3& \tt\ l& \tt l& 	light\cr
	${\it weight} = 7/8$&	\tt 4& \tt sl& \tt b& 	semi-light\cr
	${\it weight} = 1$&	\tt 5& \tt\ m& \tt m& 	medium\cr
	${\it weight} = 1\.3$&	\tt 6& \tt sb& \tt d& 	semi-bold\cr
	${\it weight} = 1\.6$&	\tt 7& \tt\ b& \tt b& 	bold \cr
	${\it weight} = 2$&	\tt 8& \tt eb& \tt x& 	extra-bold\cr
	${\it weight} = 3$&	\tt 9& \tt ub& \tt u& 	ultra-bold\cr
\endtable
 	The `style digit' is the first digit in the two-digit style
	codes described in \refstylecodes.  The `NFSS~2\ code' is the
	first half of a corresponding NFSS~2\ `font series' code.  The
	`Berry code' is the code for this weight in Karl Berry's font
	naming scheme.

 	Note that some of the character programs produce ugly results
	for large values of {\it weight}.

\subsection{Values for {\it hratio}}
	The variable {\it hratio} specifies the ratio between horizontal
	and vertical measurements: in other words, whether the font is
	compressed or expanded.  It may have the following values:
\begintable
	\it Hratio&\it Style&\it NFSS~2& \it Berry&\it Description\cr
		  &\it digit&\it code&\it code\cr
\noalign{\kern\jot\hrule\kern1.5\jot}
	${\it hratio} = 0\.50$&	{\tt 9} or {\tt 0}&
		\tt ec&\tt o&	extra compressed\cr
	${\it hratio} = 0\.80$&	{\tt 7} or {\tt 8}&
		\tt\ c&\tt c& 	compressed\cr
	${\it hratio} = 1\.00$&	{\tt 5} or {\tt 6}&
		\tt\ m&\tt r& 	normal width\cr
	${\it hratio} = 1\.15$&	{\tt 3} or {\tt 4}&
		\tt\ x&\tt x&	expanded\cr
	${\it hratio} = 1\.30$&	{\tt 1} or {\tt 2}&
		\tt ex&\tt w&	extra expanded\cr
\endtable
 	The 'style digit' is the second half of the two-digit style
	codes described in \refstylecodes.  The `NFSS~2\ code' is the
	second half of an NFSS~2\ `font series' code (for example,
	bold expanded is ${\tt b} + {\tt x} = {\tt bx}$, compressed is
	${\tt m} + {\tt c} = {\tt c}$).  The `Berry code' is for Karl
	Berry's scheme.

 	Beware that the character programs do not all produce good
	results when {\it hratio} is not 1.

\subsection{Values for {\it slant} and {\it italicness}}
	These two variables between them specify whether a font is to be
	produced with italic letterforms or not:
\begintable
	\it Slant&\it Italic? &\it Style&\it NFSS~2&\it Berry&\it Description\cr
	&&\it digit&\it code&\it code\cr
\noalign{\kern\jot\hrule\kern1.5\jot}
	${\it slant} = 0$&	${\it italicness} = 0$&	odd&
		\tt n& \tt r& 	upright\cr
	${\it slant} = 1/8$&	${\it italicness} = 1$&	even&	
		{\tt ti} or {\tt i}& \tt i&	italic\cr
	${\it slant} = 0$&	${\it italicness} = 1$&	--&
		\tt ui& \tt u& 	`upright italic'\cr
	${\it slant} = 1/8$&	${\it italicness} = 0$&	--&	
		{\tt sl} 	& \tt o&	oblique\cr
\endtable
 	The `NFSS~2\ code' is the `font shape' code.  For italic this
	might be `{\tt ti}' or `{\tt i}' depending on whom you ask.  The
	`Berry codes' are variant letters for Karl Berry's font naming
	scheme.

 	The variable {\it slant} is common to many \MF\ programs, and
	causes the glyphs to be obliqued.  The {\it italicness} variable
	signals that italic letterforms should be used for some letters.
	It is possible to generate an obliqued font or an `upright
	italic' with appropriate settings, but these cannot have `{\tt
	ma}-' names because I~have not defined style codes for them.

\section{Note on the {\it Handbook}}
	The {\it Malvern Handbook}, {\tt doc/maman.tex} produces
	cross-references automatically via an {\tt aux} file, and will
	need to be run through \TeX\ twice to get the cross-references
	right.  The front matter (preface and table of contents) are
	printed {\it last}, and should be transferred to the front of
	the handbook before binding.

\rightline{pdc 1994.10.07}
\bye

%Local Variables:
%fill-prefix: "	"
%End:
