% mabib.tex -- Bibliography for Malvern Handbook
% Copyright 1994 P. Damian Cugley

%%% @TeX-document {
%%%   filename       = "mabib.tex",
%%%   version        = "X",
%%%   date           = "pdc 1994.10.11",
%%%   package        = "Malvern 1.2",
%%%   author         = "P. Damian Cugley",
%%%   email          = "damian.cugley@comlab.ox.ac.uk",
%%%   address        = "Oxford University Computing Laboratory,
%%%                     Parks Road, Oxford  OX1 3QD, UK",
%%%   codetable      = "USASCII",
%%%   keywords       = "Malvern, METAFONT, font, typefont, TeX",
%%%   supported      = "Maybe",
%%%   abstract       = "Character programs for the Malvern
%%%                     font family.",
%%%   dependencies   = "ma55doc.tex, pdc*.tex, maamac.tex, texnical.tex,
%%%                     mabib.tex, and many Malvern fonts",
%%% }

%  This software is available freely but without warranty.
%  See the file 0copying.txt for details.

%{{{  bibliography

\section{References}

\iflong\else
	This section includes references not cited in the "Abridged
	Malvern Handbook".
\fi

	The publication \TUGboat, referred to below, is the journal of
	the \TeX\ Users Group (TUG), PO~Box~\n{9506}, Providence,
	RI~$02940$--$9506$, United States of America.  Their electronic
	mail address is |TUG@Math.AMS.com|.

	URLs (universal resource locators) describe documents available
	over Internet.  A URL of the form
\begin display
	|file://|\<host>|/|\<directory>|/|\<name>\cr
\end display
	represents a file called \<name> in directory \<directory> on an
	FTP server called \<host>.  CTAN is the Comprehensive \TeX\
	Archive, for example
\begin display
        |file://ftp.tex.ac.uk/tex-archive|\cr
	|gopher://gopher.tex.ac.uk/11/archive/|\cr\quad|Archive%20directory|\cr
       	|http://www.tex.ac.uk/tex-archive/|\cr
\end display
	and similarly
\begin display
	|file://ftp.dante.de/tex-archive|\cr
        |file://ftp.shsu.edu/tex-archive|\cr
\end display
	and many mirrors.

\def\Ibid{Ibid.\spacefactor1000 }

\xreflabel{bib}{\S{\thesecno}}
\begingroup
\def\bibitem#1#2%
{%
    \smallbreak
    \def\tmp{#1}
    \edef\tmp{\string\expcs\string\def{xref-#2}%
	{\expandafter\TOCtrim\meaning\tmp}}
    \write\auxfile\expandafter{\tmp}
    \setbox0=\hbox{\subheadingfonts {#1\/}}
    \ifdim\wd0>\leftmargin
	\noindent\hskip-\leftmargin \box0 \quad
    \else
	\noindent\llap{\hbox to \leftmargin{\hfil \unhbox0 \quad}}%
    \fi
    \ignorespaces
}

\counta=0
\def\nbibitem%
{%
    \advance\counta 1
    \expandafter\nnbibitem\expandafter{\the\counta}
}
\def\nnbibitem#1{\bibitem{\n{#1}}}

\def\TB#1#2#3#4#5%
{%
    \TUGboat~\n{#1.#3} (\n{#2}), pp.\thinspace\n{#4}--\n{#5}%
}

\nbibitem{PS}
	Adobe Systems Incorporated, {\it PostScript\registered\ Language
	Reference Manual}, $2$nd Edition (Addison--Wesley, $1990$).  The
	definitive description of the Level-$2$ PostScript
	page-decsription language.

\nbibitem{Berry}
	Karl Berry, `Filenames for Fonts', \TB{11}{1990}{4}{517}{520}.
	This describes a system of conventions for assigning \TeX\
	external names to fonts.

\nbibitem{Berry2}
	Karl Berry, {\it Filenames for Fonts}, version $1.6$
	\url{file}{ftp.cs.umb.edu}{pub/tex}{fontnames-1.6.tar.gz}.  As
	above, with revisions, a longer list of family names, and some
	ideas for long fontnames.

\nbibitem{Bien}
	Janusz S. Bie\'n, `On Standards for Computer-Modern Font
	Extensions', \TB{11}{1990}{2}{175}{183}.  Ennumerates the
	special letters used in several languages.

\nbibitem{KD}
	K.~J. Dryllerakis (|kd@doc.ic.ac.uk|), {\it Typesetting Greek
	Texts with Greek\TeX}, Greek\TeX~\n{3.1} (also known as KD Greek)
	(CTAN |fonts/greek/kd|).


\nbibitem{Cork}
	Michael~J. Ferguson, `Report on Multilingual Activities',
	\TB{11}{1990}{4}{514}{516}.  This describes the font encoding
	that resulted from discussions at the \TeX\ User Group
	conference in Cork in September $1990$.

\nbibitem{GPL}
	Free Software Foundation, {\it GNU General Public License},
	Version~$1$ (Free Software Foundation, Inc., $675$ Mass Ave,
	Cambridge, MA~$02139$, USA) (Feb.\thinspace $1989$).  The license
	agreement descibes the rights and responsibilities of users of
	GNU software.  (Supplied as a file |COPYING| in all GNU
	distributions.)

\nbibitem{Hara}
	Yannis Haralambous, `\TeX\ and Latin-Alphabet Languages',
	\TB{10}{1989}{3}{342}{345}.  An attempt to devise a system of
	ligatures to cover all European Latin-alphabet languages.
	Incorrectly states that no English words use composite letters.

\nbibitem{Hart's}
	Horace Hart, {\it Hart's Rules for Compositors and Readers at
	the University Press, Oxford}, $39$th Edition (corrected)
	(Oxford University Press, $1990$).  A small book describing the
	house style of the Oxford University Press.

\nbibitem{Hart's13} \Ibid, p.\,$13$.
\nbibitem{Hart's30} \Ibid, p.\,$30$
\nbibitem{Hart's62} \Ibid, p.\,$62$
\nbibitem{Hart's102} \Ibid, p.\,$102$.
\nbibitem{Hart's120} \Ibid, pp.\,$120$--$121$.
\nbibitem{Hart's135} \Ibid, p.\,$135$.

\nbibitem{fontinst}
	Alan Jeffrey ({\tt alanje@cogs.susc.ac.uk}), {\it The |fontinst|
	package} (CTAN |fonts/utilities/fontinst|, June \n{1994}).  See
	\xref{s-fontinst}.

\nbibitem{MFbook}
	Donald E. Knuth, {\it The \MF book} (Addison--Wesley,
	$1986$).  Also published as Volume~C of the Computers and
	Typesetting Series.  The definitive user manual for \MF.

\nbibitem{MFApp.F}
	\Ibid, Appendix F.

\nbibitem{TeXbook}
	Donald E. Knuth, {\it The \TeX book} ($n$th printing,
	Addison--Wesley, $1990$).  Also published as Volume~A of the
	Computers and Typesetting Series.  The definitive user guide to
	\TeX.

\nbibitem{TeXchar}
	{\TeX book}, Chapters $25$ and~$26$.  \<character> is defined at the
	start of Ch.\thinspace $26$ (Summary of Math Mode), but the concept is
	used in Ch.\thinspace $25$ (Summary of Horizontal Mode).

\nbibitem{TeX9}
	{\it \TeX book}, Chapter \n9.
\nbibitem{TeXB4}
	{\it \TeX book}, Appendix B, \S$4$

\nbibitem{TeXD1}
	{\it \TeX book}, Appendix D, \S$1$.

\nbibitem{TeXE}
	{\it \TeX book}, Appendix E.

\nbibitem{TeXF1}
	{\it \TeX book}, Appendix F, Figure~\n1.

\nbibitem{LaTeX}
	Leslie Lamport, {\it \LaTeX: A Document Preparation System}
	(Addison--Wesley, $1986$).  An introduction and user guide for
	\LaTeX.

\nbibitem{lfonts}
	Leslie Lamport et al., |lfonts.tex| (part of the standard
	\LaTeX\ distribution).  The only documentation for the standard
	\LaTeX\ font-loading macros.

\nbibitem{NFSS}
	Frank Mittelbach and Rainer Sch\"opf, `The New Font Family
	Selection: User Interface to Standard \LaTeX',
	\TB{11}{1990}{2}{298}{}.  This describes a version of the
	so-called New Font Selection Scheme (\mc{NFSS}), used in some
	\LaTeX\ installations.

\nbibitem{ODWE}
	{\it The Oxford Dictionary for Writers and Editors} (Oxford
	University Press, $1990$).  The companion dictionary to
	\rcite{Hart's} above.

\nbibitem{ODWEacc}
	\Ibid, under "accents and special sorts".

\nbibitem{ODWEcap}
	\Ibid, under "capitalization/Postcodes".  (Also the first line
	of the title verso of the same book.)

\nbibitem{Pei}
	Mario A. Pei, {\it The World's Chief Languages}, $3$rd Edition
	(Allen \& Unwin, $1949$).  The only reference in my local
	library to many of the languages that latin fonts are expected
	to support.

\nbibitem{nfss-malvern}
	Sebastian Rahtz, {\tt malvern.sty} (in directory {\tt
	/contrib/spqr} of the \package\ distribution).  Specifies
	Malvern fonts for \LaTeX \n{2.09} with NFSS.
\par
\endgroup 
%}}}  bibliography

%Local variables:
%fold-folded-p: t
%tex-mother-file: "maman"
%fold-folded-p: t
%fill-prefix: "\t"
%End:
