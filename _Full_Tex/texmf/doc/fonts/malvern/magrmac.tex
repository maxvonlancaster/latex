% %M% 1.%I% %E% -- hacks for typesetting Greek

%  These macros are intended to be useful with the Malvern G fonts,
%  and should also work pretty much with the Levy or KD Greek fonts.  In
%  the long term, I expect to merge what little extra support is needed
%  for Malvern into the GreekTeX distribution in some sense.

%  These macros are not in final form and are not intended to be
%  supported.  Use them at your own risk.  Enjoy!

%  Damian Cugley 
%  Oxford University Computing Laboratory, UK
%  <Damian.Cugley@comlab.ox.ac.uk>

%{{{ macros for greek

\newtoks\everygreek

%{{{  lunate sigma, curly phi

{\catcode`\s=13 \catcode`\S=13
\gdef\\%
{%
    \catcode`\s=13 \chardef s=`c%
    \catcode`\S=13 \chardef S=`C%
}}
\let\grlunatesigma=\\

{\uccode`\~=`\f \uppercase{
\gdef\\%
}}  lunate sigma
%{{{  accent

{\catcode`\<=13 \catcode`\>=13 \catcode`\'=13 \catcode`\`=13 
\gdef\monotoniko
{%
    \catcode\lq\'=13 \catcode\lq\`=13 \catcode\lq\~=13
    \catcode\lq\<=13 \catcode\lq\>=13  \catcode\lq\|=13 
    \chardef'=3  \chardef`=3 \chardef~=3
    \chardef<=64 \chardef>=64 \chardef|=64
}
\gdef\plutoniko
}}  accent
%{{{  define delims

%  Define #1 to be a "greek shift" in the way $ is a maths shift.
%  Like $, #1 stands for either \begingreek or \endgreek as required.
%  #1 must be a single token that is definable with \def -- either
%  a control sequence or an (already) active character.

\def\grdelimiter#1%
{
    \def#1{\begingreek \let#1=\endgreek}
}

%}}}  define delims

\def\begingreek
{%
    \begingroup
    \catcode`\~=12 \catcode`\|=12 \catcode`\"=12
    \the\everygreek
}
\def\endgreek
}} macros for greek

% Local variables:
% fold-folded-p: t
% tex-macros-p: t
% End:
