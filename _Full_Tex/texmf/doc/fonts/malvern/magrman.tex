% magman.tex -- info about Malvern Greek.
%{{{ Malvern Greek manual
%{{{  preamble

\input cmdoc
\input magrmac
\input pdcidx

\font\notegr=ma55g9
\font\bodygr=ma55g11 
\macappend\BODYtemplate{\f{gr}{ma55g}}
\bodyfonts

\everygreek{\gr \baselineskip=14pt}

\grdelimiter*

\headline={\hfil\global\headline={\bodyfonts \hfil Typesetting Greek using
	Malvern fonts\quad{\bf\folio}}}

\def\thesecno
{%
    \number\subsecno
}

%{{{   font tables

%%  Code to typeset a font table -- lifted from my testfont.tex

\newcount\tableN
\newcount\hexcount
\def\hexdigit#1{\ifcase#1\relax 0\or 1\or 2\or 3\or 4\or 5\or 6\or 7\or
	8\or 9\or A\or B\or C\or D\or E\or F\fi}

\def\ntablecr
{%
    \cr
    \noalign{\nointerlineskip}
    \multispan2\hfill &\multispan{33}\hrulefill
}
\def\ntable
{
    \medskip
    \begingroup \openup1\jot
	\def\\{\char\tableN \global\advance\tableN 1}
	\def\0##1{&\omit&\sevenrm##1}
	\halign to \hsize
	  {%
	    \chartstrut\hss##\tabskip=0pt plus 10pt &
	    &\hss##\hss&##\vrule\cr
	    \lower 6.5pt\null
	    &\00\01\02\03\04\05\06\07\08\09\0A\0B\0C\0D\0E\0F
	    \ntablecr 
	    \global\tableN=0
	    \ntablelines
	    \crcr
	  }
	\medbreak
    \endgroup
}

\def\ntablelines
{%
    \ifnum\tableN<256 
	\let\next\ntablecontinuation
    \else
	\let\next\relax
    \fi
    \next
}

\newcount\ntabtmp

\def\ntablecontinuation
{%
    % Find out if none of this row are defined by making a horizontal
    % list of all of them preceeded by a penalty of 1; if any of them
    % are defined then \lastpenalty will be something other than 1:
	\setbox0=\hbox{\penalty1
	    \def~{\char\tableN \advance\tableN 1}%
	    ~~~~~~~~~~~~~~~~\global\ntabtmp=\lastpenalty}%
    % Now set the row in the table iff ntabtmp # 1:
	\ifnum\ntabtmp=1 
	    \global\advance\tableN 16 \let\next=\ntablelines
	\else
	    \let\next=\ntablecontinuationcontinuation
	\fi
    \next
}

\def\ntablecontinuationcontinuation
}}   font tables
%{{{  two-column quotations

\def\twocolgr
{
    \smallskip 
    \setbox0=\vbox\bgroup
	\hsize=\bodywd \advance\hsize-\colsep \divide\hsize by 2
	\begingreek \strut \ignorespaces
}
\def\endtwocolgr
}}  two-column quotations
%}}}  preamble
%{{{  bibliography

\def\TB{{\it TUGboat}}

\counta=0
\newtoks\bibliography
\newcount\bibcount

\def\FAQnote
{%
	\par {\bf Note on FAQs}\quad An FAQ is an electronic document, posted
	regularly to a USENET newsgroup, usually in the form of a list of
	answers to frequently asked questions.  Many FAQs are available on
	the archive site |pit-manager.mit.edu| (alias |rtfm.mit.edu|)
	[18.172.1.27] in the directory |pub/usenet/news.answers|.  The
	reference gives the name of the compiler, the title, the
	last-modified date (in lieu of an edition or version number) and
	name under which the FAQ is archived appears in parentheses.
    \par
}

\def\bibdef#1#2%
{
    \expcs\edef{cite#1}%
    {%
        \noexpand\docite\noexpcs{cite#1}\noexpcs{refer#1}%
    }
    \toks0={#2}
    \expcs\edef{refer#1}{\noexpand\bibitem{\expcs\noexpand{cite#1}}
	\the\toks0}
}
\def\docite#1#2%
{%
	\global\advance\bibcount1
	\global\bibliography=\expandafter
		{\the\bibliography #2}%
        \global\edef#1%
	{[\the\bibcount]}#1%
}

\bibdef{Allen}
{
    W. Sidney Allen, {\it Vox Graeca: A Guide to Pronunciation of Classical
	Greek}, 3rd Ed.\ (Cambridge University Press, 1987),
	pp.\thinspace177--179.
}

\bibdef{KD}
{
    K.~J. Dryllerakis (|kd@doc.ic.ac.uk|), {\it Typesetting Greek Texts with
	Greek\TeX}, Greek\TeX~3.1 (also known as KDGreek) (CTAN
	|fonts/greek/kd|).
}

\bibdef{FAQ}
{
    Nikolaos Fotis (ed.), {\it soc.culture.greek} FAQ --
	Linguistics ({\tt greek-\penalty\exhyphenpenalty faq\slash
	linguistics}, 1993/03/06).
}

\bibdef{Yannis}
{
    Yannis Haralambous and Klaus Thull, Typesetting Modern Greek with 128
	Character Codes, \TB\/ 10 (1989), pp.\thinspace354--359 (CTAN
	|fonts/greek/yannis|).
}

\bibdef{Harts}
{
    Horace Hart, {\it Hart's Rules for Compositors and Readers at the
	University Press, Oxford}, 39th Edition, revised (Oxford
	University Press, 1989), pp.\thinspace111--116.
}

\bibdef{Levy}
{
    Silvio Levy, Using Greek Fonts with \TeX, \TB\/ 9 (1988),
	pp.\thinspace20--24 (CTAN |fonts/greek/levy|).
}

\bibdef{COD}
{
	{\it The Concise Oxford Dictionary of Current English}, 8th Ed.
	(Oxford University Press, 1990), p.\thinspace1453.
}

\def\bibitem#1%
}}  bibliography

\leftline{\headingfonts Typesetting Greek using Malvern fonts}
\bigskip
\leftline{P. Damian Cugley}
\leftline{Oxford University Computing Laboratory}
\leftline{(|Damian.Cugley@comlab.ox.ac.uk|)}
\leftline{April 1993}
\bigskip

\noindent
	Malvern is a sanserif font family, implemented in \MF.  This
	document describes an early version of Malvern encoding~G (the Greek
	alphabet).  The Greek letters used to appear in Malvern~B, but I
	decided to give them an encoding (code page?)\ of their own so that
	there would be space for composite letters.

\subsec{Using Malvern G}

	Because most Latin-alphabet languages require additional composite
	letter glyphs to be properly typeset by \TeX, most Malvern fonts
	will be used as the raw material for composite fonts.  In contrast,
	the Malvern Greek font can be used directly to typeset Greek, both
	with the old accent-and-breathing systems and the newer (post-1974)
	one-accent system.  It follows the conventions of Silvio Levy's
	original |gr| family \citeLevy\ and K.~J.  Dryllerakis's Greek\TeX\
	(|kd| fonts) \citeKD.  It is almost compatible with the reduced
	Greek fonts (|rgr| and |mrgr| families) described by Yannis
	Haralambous and Karl Thull \citeYannis.

%{{{  typesetting

\subsec{Macro file}

	The file |grhacks.tex| has some macros for setting texts in plain
	\TeX.  A future production release may instead work as an option or
	extension of Greek\TeX, with the objective of making it possible to
	switch an existing document to Malvern without much editing.

	Greek text is set within a \dfn{Greek mode}, delimited by macros
	|\begingreek| and |\endgreek|.  Each time Greek mode is entered, the
	contents of the token register |\everygreek| is scanned first.
	Since |grhacks| does {\it not} load a particular font by default, a
	manuscript using |grhacks| will want to include an assignment like
\begin display
	|\everygreek={\gr}|\cr
	|\catcode`\*=\active \grdelimiter*|
\end display
	where |\gr| has been bound to some Greek font or another such as
	|ma55g12|.  The macro |\grdelimiter| takes one parameter, an active
	character (or control sequence) and makes that character into a
	self-matching delimiter for Greek mode (like |$|--|$| works for
	mathematics mode).  Thus, `|*This is Greek*|' produces `*This is
	Greek*'.

  
\subsec{The alphabet}

	The letters are transliterated as
	follows:
\begin table \hfil#\hfil&&\enspace\hfil#\hfil\cr
	\gr a&\gr b&\gr g&\gr d&\gr e&\gr z&\gr h&\gr j&\gr i&\gr k&\gr
	l&\gr m&\gr n&\gr x&\gr o&\gr p&\gr r&\gr ss&\gr c&\gr t&\gr 
	u&\gr f&\gr q&\gr y&\gr w\cr
	\tt a&\tt b&\tt g&\tt d&\tt e&\tt z&\tt h&\tt j&\tt i&\tt k&\tt
	l&\tt m&\tt n&\tt x&\tt o&\tt p&\tt r&\tt s&\tt c&\tt t&\tt u&\tt 
	f&\tt q&\tt y&\tt w\cr
\noalign{\smallskip}%
	\gr A&\gr B&\gr G&\gr D&\gr E&\gr Z&\gr H&\gr J&\gr I&\gr K&\gr
	L&\gr M&\gr N&\gr X&\gr O&\gr P&\gr R&\gr S&\gr C&\gr T&\gr 
	U&\gr F&\gr Q&\gr Y&\gr W\cr
	\tt A&\tt B&\tt G&\tt D&\tt E&\tt Z&\tt H&\tt J&\tt I&\tt K&\tt
	L&\tt M&\tt N&\tt X&\tt O&\tt P&\tt R&\tt S&\tt C&\tt T&\tt U&\tt 
	F&\tt Q&\tt Y&\tt W\cr
\end table
 	The letter *C c* is \dfn{lunate sigma} (see below).

	A lower case sigma `*s@*' at the end of a word (or followed by
	punctuation) is automatically changed to the final form `*s*'
	through \TeX's ligature mechanism.\note{Occasionally \TeX's notion
	of word boundaries causes the wrong glyph to appear.  There are two
	special invisible glyphs that can be used to control this in obscure
	circumstances.  The so-called \dfn{compound word mark} (or cwm) acts
	as a word-boundary in the middle of a word.  It has code~32, so
	`|as^^`a|' produces `*as^^`a*', for example.  The \dfn{null glyph}
	is invisible, but, because it is not a boundary character, it can be
	used to prevent a ligature with a following word boundary.  It has
	code 64, so `|as@|' produces `*as@*'.}

\subsec{Composite letters}

	The marks that go above the letters are obtained with the
	following characters, which go before the letter:
\begin table \gr#\hfil&\quad#\hfil&\quad #\hfil& \quad *#*\hfil\cr
\noalign{\hrule height 1pt \vskip1\jot}%
	\omit Mark\hfil&\omit\quad Char\hfil&
		Name&\omit\quad Name in \citeYannis\hfil\cr
\noalign{\vskip\jot \hrule \vskip\jot}
	<@&	|<|&	asper (rough, = h)& 	dase'ia\cr
	>@&	|>|&	lenis (smooth)&	yil'h\cr
	"@&	|"| (double quote)&
			diaeresis&		dialutik'a\cr
	'@&	|'| (quote)& 
			acute accent&		>oxe'ia\cr
	`@&	|`|&	grave accent&		bare'ia\cr
	\char126 @& |~|& circumflex&		perispwm'enh\cr
\noalign{\vskip1\jot \hrule height 1pt\smallskip}%
\end table
	A breathing or diaeresis can be combined with an accent, thus
	`|<~a|' for *<~a*, `|>'e|' for *>'e*.

	The vowels *a*, *h* and *w* may have an iota beneath them, called
	`subscript': *a|*, *h|*, *w|*.  These are produced with a vertical
	bar `||||' after the letter in Greek mode: `|a|||' makes *a|*,
	`|<a|||' makes `*<a|*', and so on.

	As an example: `*>en >arq~h| >~hn <o l'ogos*' is generated with `|>en
	>arq~h|| >~hn <o l'ogos|'.

%	In old-style Greek, all vowels beginning a word require a breathing
%	-- but *u* may only have an asper (rough).  In diphthongs (*ai*,
%	*ei*, *oi*, *ui*, *au*, *eu*, *hu*, *ou*, *wu*), breathings and
%	accents stand over the second letter: *a<i*, *o>u*.  Double *rr*
%	used to be written *>r<r*.
	
%\subsec{Apostrophe (sign of elision)}

%	The lenis is also used like an apostrophe, to mark the elision of
%	vowels at the end of a word when the next word starts with a vowel.
%	The Malvern-G fonts, like the Levy-derived fonts, also has a
%	separate apostrophe glyph, which can be set with `|''|' (two quote
%	characters) or `|'|' (one quote) at the end of a word.  Thus you
%	have a choice of `|d>|' or `|d'|' (*d>* or *d'*).


\subsec{Punctuation}

	Here's a table of correspondences for punctuation:
\begin table \hfil#\hfil&&\enspace\hfil#\hfil\cr
	*.*&*,*&*;*&*:*&*!*&*?*&*''*&*((*&*))*\cr
	|.|&|,|&|;|&|:|&|!|&|?|&|''|&|((|&|))|\cr
\end table
	The apostrophe is also generated by a single single-quote at the end
	of a word: `*d'*' can be generated by `|d'|'.  There are also
	digits, parentheses, brackets, hyphen, dashes, slash, per cent sign,
	asterisk, plus and equals signs.

\subsec{One-accent Greek}

	The easiest way to write modern Greek is simply to use |'| for the
	accent and otherwise to use |\begingreek|--|\endgreek| as before:
	`*en arq'h hn o l'ogos*' is generated with `|en arq'h hn o l'ogos|'.

	Malvern also has glyphs for a symmetrical accent (*\char3*) and
	composite letters (such as *\char3 a*, *"\char3u*).  The macro
	|\monotoniko|\note{This macro is so named for compatibility with the
	|rgr|/|mrgr| macros \citeYannis.  It might usefully go in
	|\everygreek|.} in Greek mode makes |'| an active character,
	expanding to the symmetrical accent.  Then `|En'w|' produces
	`*En\char3w*'.  In fact, the other accent characters |`| and |~| are
	also made to substitute `*\char3*', and |<|, |>| and |||| expand to
	the null glyph, so that a text with all the breathings and accents
	and be hacked into almost-correct one-accent Greek.\note{Again, for
	compatibility with the |rgr|/|mrgr| macros.  Because monosyllables
	should not have accents at all in one-accent Greek, a complete
	conversion from old-style to new-style requires changes to the
	manuscript anyway, so the usefulness of this feature is debatable!}
	For example, `|>En~w|||' becomes `*\monotoniko >En~w|*'.

\subsec{Variant glyphs}

	Sigma has a variant called lunate sigma, written *Cc*.  These
	letters can be obtained directly (produced by the character `|c|' in
	the manuscript).  So that a given text can be switched between the
	two styles, there is also a macro |\grlunatesigma|, which redefines
	`|s|' is as active character equivalent to `|c|'.  Obviously this is
	only useful when Greek mode is only used for plain texts, since it
	prevents `|s|' from being used in control sequence names.

	Similarly, there are two ways to write lower case phi: *f* and
	*\char92 * (`|f|' and glyph~92).  There is no difference in meaning
	between *f* and *\char92 *; the choice of one or the other is only
	made on aesthetic grounds.  The macro |\grvarphi| redefines `|f|' as
	an active character equivalent to `*\char92 *', in effect replacing
	*f* with *\char92 *.

	Normally these macros, if used, will go in |\everygreek|.
	
%}}}  typesetting
\iffalse
%{{{  background
\subsec{Background}

	This section is optional reading.

\subsubsec{Development of the Levy-derived font families}

	Silvio Levy's |gr| fonts \citeLevy\ used the \TeX~2 ligature system
	to make *s* become *s@* when followed by a letter, and had glyphs
	for almost all the composite letters.

	Yannis Haralambous and Klaus Thull created a set of reduced
	(128-glyph) fonts, for \TeX\ systems still unable to manage
	256-glyph fonts \citeYannis.  These fonts (the |rgr| and |mrgr|
	families) still used Levy's character programs.  Because they could
	not include glyphs for all the composite letters, many composite
	letters had to be obtained with macros rather than ligatures.  The
	|mrgr| family is for new-style (one-accent) Greek.

	Finally K.~J. Dryllerakis's Greek\TeX\ package (also called KDGreek)
	\citeKD\ uses the Levy character programs again, but takes advantage
	of the new ligature features of \TeX~3 to make *s@* become *s*
	(without needing compound glyphs for every combination of *s@* with
	a letter).  The fonts are given names starting with |kd|.
	Greek\TeX\ also includes format files for plain \TeX\ and \LaTeX,
	transliteration programs and other useful things.

\subsubsec{Designing Greek letters for Malvern}

	My Malvern-G encoding is intended to be `ligature-compatible' with
	Levy's and Dryllerakis's, meaning that `|<'a|||' will produce
	`*<'a|*' in each of them, even though this is glyph~141 in
	|grreg10|, 201 in |kdgr10| and some random number in |ma55g10|.  The
	punctuation characters have the same encoding.

	Designing a sanserif Greek lower case alphabet (*alfabet*?) is tricky
	because the lower case letters have a very cursive style, as if
	hand-drawn with a brush by a scribe in a hurry -- which is pretty
	much the opposite of most sanserif styles.  I~made sketches (on
	paper) of Greek letters forced into the same sorts of shapes as
	other Malvern letters, and the results were uniformly horrid.  The
	original idea of Malvern was to make a humanist/geometric hybrid, so
	I tried approaching it from the from the humanist end this time.

%}}}  background
\fi
%{{{  references

\subsec{References}

	Since I don't speak a word of Greek -- the only word I know is
	*barbaroi* and I don't know how to spell it -- I have relied on
	reading between the lines of the documentation for other systems for
	typesetting Greek, including Hart's Rules \citeHarts.

	CTAN is the Comprehensive \TeX\ Archive Network, a collection of FTP
	sites (such as |ftp.tex.ac.uk|, under directory |tex-archive|).

\begingroup \parskip=0pt \parindent=0pt
	\the\bibliography
\endgroup

%}}}  postamble
\vfill\eject
%{{{  samples

\subsec{Examples of text in Malvern 55 and 56}

\iffalse
	This quotation is used by Haralambous and Thull \citeYannis\ as an
	example of the old-style accent-and-breathing system:
%{{{   Popess Johanna

\begin twocolgr
	>En~w| <esp'eran tin`a >exantl'hsas t`a murol'ogi'a tou
	>ekoim~ato <o Froum'entios >ep`i t~hs >'ammou t~hs paral'ias,
	katab`as >ex o>uran~wn <o >ap'ostolos >eke~inos t~wn Sax'onwn
	>'hnoixe di`a maqa'iras t`a st'hjh to~u koimwm'enou,
	e>is'hgage to`us <iero`us dakt'ulous tou e>is t`hn >op`hn kai
	>exag`wn t`hn kard`ian >eb'ujisen a>ut'hn e>is l'akkon pl'hrh
	<'udatos, <'oper <hg'iasen prohgoum'enws.  <H fl'egousa
	>eke'inh kard'ia >'efrizen e>is t`o <'udwr <ws smap`is >ent`os
	to~u thgan'iou, >afo~u d`e >ekr'uwsen, >'ejese p'alin a>ut`hn
	<o <'agios e<is t`on t'opon ths ka`i kle'isas t`hn plhg`hn
	>ep'estreyen e>is t`on >idik'on tou.

	>'Etuq'e pote, >anagn~wst'a mou, n`a >apokoimhj~h|s m`e
	>anup'oforon b~hqa, koim'wmenos n`a <idr'wsh|s ka`i >exupn'hsas
	n`a e<urej~h|s >iatreum'enos?  >Agn~wn <'oti e>~isai
	kal`a >ano'igeis mhqanik~ws t`o st'oma, <'ina plhr'wsh|s e>is
	t`on >epikat'araton b~hqa t`on sun'hjh f'oron.  >All`a
	p'oshn a>isj'anesai qar'an, m`h e<ur'iskwn e>is t`on l'arugga
	t`o >oqlhp`on jhr'ion! O<'utw <'ama >'hnoixe ka`i <o
	Froum'entios to`us >ofjalmo'us, <htoim'asjn n`a prosf'erh|
	e>is t`hn >aq'ariston >Iw'annan t`hn sun'hjh dakr'uwn
	spond'hn, >all`a par`a p~asan prosdok'ian o<i >ofjalmo'i
	tou e<ur'ejhsan xhro`i ka'i n`a progeumat'ish| m~allon >`h
	n`a kla'ush| >h|sj'aneto >'orexin met`a polu'hmeron nhste'ian
	<o kal`os Benedekt~inos.
\end twocolgr
\rightline{\csc{Emmanouil Ro\"{\i}dis}, `The Popess Johanna (1896)'}
\vskip 0pt plus \baselineskip

%}}}   Popess Johanna
\fi

	The following paragraph is set in 11-point Malvern~55 (11-point
	Malvern has comprable x-height to 12-point Computer Modern).  This
	quotation is used as one of the examples in Levy's article
	\citeLevy:
\begin twocolgr
	>All' >ako'usontai, >enper e>~u dok~h|s l'egein.  t'ode d'e sou
	>ene'ohsa <'ama l'egontos, ka`i pr`os >emaut`on skop~w; e>i <'oti
	m'alist'a me E>uj'ufrwn did'axeien, <ws o<i jeo`i <'apantes t`on
	toio~uton j'anaton <hgo~untai >'adikon e>~inai, t'i m~allon >eg`w
	mem'ajhka par' E>uj'ufronos, t'i pot' >est`in t`o <'osi'on te ka`i
	t`o >an'osion?  jeomis`es m`en g`ar to~uto t`o >'epgon, <ws
	>'eoiken, e>'in >'an; >all`a g`ar o>u to'utw| >ef'anh >'arti
	<wrism'ena t`o <'osion ka`i m'h; t`o g`ar jeomis`es >`on ka`i
	jeofil`es >ef'anh.  <wste to'utou m`en >af'ihm'i se, >~w E>uj'ufron;
	e>i bo'ulei, p'antes a>ut`o <hge'isjwn jeo`i >'adikon ka`i p'antes
	miso'untwn.  >all' >~ara to~uto n~un >epanorj'wmeja >en t~w| l'ogw|,
	<ws <`o m`en p'antes o<i jeo`i mis~wsin, >an'osi'on >estin, <`o d'
	>`an fil~wsin, <`osion; <`o d' >`an o<i m`en fil~wsin, o<i d`e
	mis~wsin, o>ud'etera >`h amf'otera?  >~ar' o<'utw bo'ulei <hm~in
	<wr'isjai n~un per`i to~u <os'iou ka`i to~u >anos'iou?
\end twocolgr
	\rightline{Plato, `Euthyphro'}
\vskip 1\medskipamount plus \baselineskip


	This quotation (similarly stolen) is in Malvern~56 and has
	|\grvarphi| and |\grlunatesigma| in effect:

\begin twocolgr \font\f=ma56g11 \f \grvarphi \grlunatesigma
	T`h stigm`h to'uth ni'wjw p'oso bar'u ''nai t`o must'hrio t~hs
	xomol'oghshs.  <Ws t'wra, kane`is d`en x'erei p~ws p'erasa t`a du`o
	qr'onia mou st`o <'Agion >'Oros.  O<i f'iloi mou jarro~un p`ws p~hga
	n`a d~w buzantin`a kon'ismata >`h >ap`o mustikop'ajeia n`a z'hsw
	mi`a perasm'enh >epoq'h.  Ka`i t'wra, n'a, ntr'epomai n`a mil'hsw.

	P~ws n`a t`o p~w?  Jumo~umai <'ena >anoixi'atiko deilin'o, po`u
	kat'ebaina t`on Ta"'ugeto, mi`a xafnik`h j'uella m`e k'uklwse kont`a
	sto'us Pentaulo'us.  T'oso fober`os >anemos'ifounas, po`u >'epesa
	katag~hs gi`a n`a m`hn gkremist~w.  O<i >astrap`es m' >'exwsan
	<olo~uje ki >'ekleisa t`a m'atia m`hn tuflwj~w, ka`i kat'aqama,
	p'istoma, per'imena.  <'Olo t`o pan'uyhlo boun`o >'etreme, ka`i du`o
	>'elata d'ipla mou tsak'isthkan >ap' t`h m'esh ka`i br'onthxan
	q'amou.  >'Eniwja t`o jei'afi to~u kerauno~u st`on >a'era, ka`i
	xafnik`a x'espase <h mp'ora, >'epesen <o >'anemos, ka`i qontr'es,
	jerm'es st'ales broq`h qt'uphsan t`a dentr`a ka`i t`o q~wma.  T`o
	jum'ari, <h jro'umpa, t`o fask'omhlo, t`o flisko'uni, qtuphm'ena
	>ap' t`o ner'o, t'inaxan t`is murwdi'es tous ki <'olh <h g~hs
	m'urise.
\end twocolgr
\rightline{\csc{Kazantzakis}, `Symposium'}
\vskip 0pt plus \baselineskip

%}}}  samples
\iffalse
%{{{  table

*\ntable*
\smallskip \noindent
	This is a working encoding, and will likely be different in future
	versions.
%}}}  table
\fi
%{{{  24-pt sample

\font\biggr=ma55g24
\bigbreak
\moveleft\leftmargin \vbox{ \hsize=\bodywd
\hrule height 1pt \bigskip
\begingreek \biggr\baselineskip=30pt
\centerline{((ABGDEZHJIKLMNXOPRSCTUFQYW))}
\centerline{](abgdezhjiklmnxoprss^^`ctuf\char92 qyw)[}
\centerline{=0123456789---\%\char42 +/--''!,:;?.-}
\centerline{\char3 '@ `@ ~@ <@ <'@ <`@ <~@ >@ >'@ >`@ >~@ "@ "'@ "`@ "~@
	"\char3 @}
\endgreek
\medskip\hrule height 1pt
}
%}}}  24-pt sample

\bye
%}}} Malvern Greek manual

% Local variables:
% fill-prefix: "\t"
% fill-column: 76
% fold-folded-p: t
% End:
