% maman.tex -- Top level of Malvern documentation
% Copyright 1994 P. Damian Cugley

%%% @TeX-document {
%%%   filename       = "maman.tex",
%%%   version        = "X",
%%%   date           = "pdc 1994.07.20",
%%%   package        = "Malvern 1.2",
%%%   author         = "P. Damian Cugley",
%%%   email          = "damian.cugley@comlab.ox.ac.uk",
%%%   address        = "Oxford University Computing Laboratory,
%%%                     Parks Road, Oxford  OX1 3QD, UK",
%%%   codetable      = "USASCII",
%%%   keywords       = "Malvern, METAFONT, font, typefont, TeX",
%%%   supported      = "Maybe",
%%%   abstract       = "Character programs for the Malvern
%%%                     font family.",
%%%   dependencies   = "ma55doc.tex, pdc*.tex, maamac.tex, texnical.tex,
%%%                     mabib.tex, and many Malvern fonts",
%%% }

%  This software is available freely but without warranty.
%  See the file 0copying.txt for details.

%{{{ maman.tex
%{{{ Preamble

%**start of header

\input ma55doc
\input pdcidx
\input pdcdcap
	
%{{{   fonts

\twosidedfalse	% change to TRUE for book-style printing, FALSE for 1-sided
\newif\ifPS \PSfalse	% change to TRUE iff have PostScript fonts
\newif\iflong \longfalse % chnage to TRUE if you want as longer manual

\newfam\cpfam \newfam\xpfam \newfam\rmbfam \newfam\rmcfam \newfam\grfam
\newfam\rmxfam \newfam\rmsfam \newfam\cyfam
\def\extrafontsmap{\f{rmb}{ma55b}\f{rmc}{ma55c}%
    \@\f{rmx}{fmvmq10}\@\f{rms}{fmvm10}%
    \f{cp}{ma57a}\f{xp}{ma53a}\f{gr}{ma55g}\f{cy}{ma55c}}
\xfontset{tiny}\extrafontsmap{7}
\xfontset{small}\extrafontsmap{9}
\xfontset{note}\extrafontsmap{10}
\xfontset{body}\extrafontsmap{12}
\def\\{\macappend\BODYtemplate} \expandafter\\\expandafter{\extrafontsmap}
\bodyfonts
%}}}   fonts
%{{{   customize footnotes & environments

\def\footnotetextmark#1{{\rmb#1}}
\def\footnotenotemark#1{{\rmb#1}}
\def\TOCentrysubsubsec#1#2#3{}

\def\n#1{$ \textfont0=\font #1 $}	
\mathcode`\.="70AE		% make "." in maths mode centred dot
	
%  In the index, references are enclosed in $...$, so
%  we have to change the "encapsulation" macros:
\def\idxunderline#1{\underline{#1}}  
\def\see#1#2{$ {\rmb \char1 } #1 $}

%%%% \everylines={\smallfonts}

\everybnf={\def\>{{\rmb\char1 }}\def\\{{\rmb\char124 }}}

\def\notepar
{
    \smallskip
    \index{note}
    \begingroup \it
    \setbox\parbox=\hbox{\it note}
}
\def\endnotepar
{
    \smallbreak
    \endgroup
}

\def\example
{
    \penalty\predisplaypenalty
    \smallskip \parskip=0pt 
    \obeylines	% every line is indented; blank lines ignored
    \the\everylines
    \let\end=\linesend
}
\def\endexample
{
    \smallbreak \global\noindenttrue
}
%}}}   environment tweaking
%{{{   full-page Part headings

% if in long mode, part headings omitted.
\def\BIGmap{\f{rm}{ma55a}\@\f{mf}{logo10}}
\fontset{big}\BIGmap{48}{56pt}{big}{big}
\ldfont\hugerm{ma75a72}

\def\part#1%
{%
\iflong
    \advance\partno1
    \mark{{\thesecno}{\thesecno}}
    \vfill\eject
    \iflong \line{} \vfill \fi
    \begingroup
	\parskip=0pt \parindent=0pt \hyphenpenalty=10000
        \bigfonts
	\rightskip=0pt plus 2em 
	\iflong
		{\hugerm \thepartno}\bigskip
	\else
		\noindent\llap{\thepartno\enspace}%
	\fi
	#1 \par
	\TOCwrite\TOCentrypart{\thepartno}{#1}
	\counta=\secno \advance\counta1 
		\mark{{\n{\the\counta}}{\n{\the\counta}}}
	\iflong \headline={\hfil} \footline={\hfil} \eject \fi
    \endgroup
\fi
}
%}}}   full-page part headings
%{{{   symbol tables

\def\symtabrule#1%
{%
    \nointerlineskip
    \moveleft\leftmargin \hbox
    {%
	\vrule height #1 width \dimen0 depth 0pt
    }%
    \nointerlineskip
}

\def\symtab
{%
    \noindent$$  % matching $$ is in endsymtab
    \def\dag{{\rm\char170 }}
    \let\\=\symtabentry  \counta=0
    \openup1\jot
    \dimen0\hsize \advance\dimen0\leftmargin
    \halign to \hsize\bgroup\hskip-\leftmargin
	\hfil##\unskip\hfil \tabskip=0pt plus 1em minus 0.5em
		&\quad\tt##\unskip\hfil
	&&\qquad\hfil##\unskip\hfil&\quad\tt##\unskip\hfil\cr
	\noalign{\symtabrule{1pt}\medskip}%
}
\def\endsymtab
{%
        \crcr
	\noalign{\nobreak\medskip \symtabrule{0.5pt}}%
    \egroup$$  %% matches $$ in above
}

\def\symtabentry#1|#2|%
{%
    #1&
    \def\tmp{#2}\expandafter\stripMacro\meaning\tmp
    \global\advance\counta1 
    \ifnum\counta=3 
	\global\counta=0
	\def\next{\cr}%
    \else
	\def\next{&}%
    \fi
    \next
}

\def\stripMacro#1:->{}

%}}}   symbol tables
%{{{   font tables

%%  Code to typeset a font table -- lifted from my testfont.tex

\newcount\tableN
\newcount\hexcount
\def\hexdigit#1{\ifcase#1\relax 0\or 1\or 2\or 3\or 4\or 5\or 6\or 7\or
	8\or 9\or A\or B\or C\or D\or E\or F\fi}

\def\ntablecr
{%
    \cr
    \noalign{\nointerlineskip}
    \multispan2\hfill &\multispan{33}\hrulefill
}
\def\ntable
{
    \medskip
    \begingroup \openup1\jot
	\def\\{\char\tableN \global\advance\tableN 1}
	\def\0##1{&\omit&\sevenrm##1}
	\halign to \hsize
	  {%
	    \chartstrut\hss##\tabskip=0pt plus 10pt &
	    &\hss##\hss&##\vrule\cr
	    \lower 6.5pt\null
	    &\00\01\02\03\04\05\06\07\08\09\0A\0B\0C\0D\0E\0F
	    \ntablecr 
	    \global\tableN=0
	    \ntablelines
	    \crcr
	  }
	\medbreak
    \endgroup
}

\def\ntablelines
{%
    \ifnum\tableN<256 
	\let\next\ntablecontinuation
    \else
	\let\next\relax
    \fi
    \next
}

\newcount\ntabtmp

\def\ntablecontinuation
{%
    % Find out if none of this row are defined by making a horizontal
    % list of all of them preceeded by a penalty of 1; if any of them
    % are defined then \lastpenalty will be something other than 1:
	\setbox0=\hbox{\penalty1
	    \def~{\char\tableN \advance\tableN 1}%
	    ~~~~~~~~~~~~~~~~\global\ntabtmp=\lastpenalty}%
    % Now set the row in the table iff ntabtmp # 1:
	\ifnum\ntabtmp=1 
	    \global\advance\tableN 16 \let\next=\ntablelines
	\else
	    \let\next=\ntablecontinuationcontinuation
	\fi
    \next
}

\def\ntablecontinuationcontinuation
}}   font tables
%{{{   aux file
%%
%%  Hacks using the ".aux" file to store information between runs
%%
%%  This allows cross-references and per-page footnotes.
%%  Normally I don't bother if a document is short enough not to need
%%  cross-references or an index.
%%
%%  Often TeX must be run twice to get the Xrefs sorted out :-(
%%

%{{{   xrefs

%  Cross-refernces require two passes; one to write the xref
%  to the .aux file with \xreflabel, and one to use the resulting \def.
%  A footnote to the effect that a ref has been omitted is generated.
%  Perhaps this is not a great idea.

\def\xref#1%
{%
    \expcs\ifx{xref-#1}\relax
	\message{Warning: xref (#1) not known; you will need to TeX
	this file again}%
	<<{#1}>>%
	\footnote{{\rm\dag}}{Cross-references in this document are
	produced automatically.  The values of Xrefs are written
	into a file called \jobname.aux, which is read at the start of
	the next run of \TeX\ on this file.  Unknown Xrefs are replaced with the tag
	name used within the document, enclosed in guillemets.}%
	\gdef\xref##1{\expcs\ifx{xref-##1}\relax <<{##1}>>%
	    \else\csname xref-##1\endcsname\fi}%
    \else
	\csname xref-#1\endcsname
    \fi
}

%  #1  -- string of characters -- internal label for this xref
%  #2  -- TeX text -- what is produced by \xref{#1} on subsequent runs.
\def\xreflabel#1#2%
{%
    \edef\tmp{\string\expcs\def{xref-#1}{#2\ on page~\n{\noexpand\folio}}}%
    \write\auxfile\expandafter{\tmp}%
}

\def\rcite#1{{\xref{#1}}}		% direct reference
\def\cite#1{[{\xref{#1}\/}]}		% parenthetically
\def\ncite#1#2{[{\xref{#1}\/}, #2]}	% with a note

\def\lcite#1%  #1 is \\-list macro of cites!
{%
    \def\tmplist{#1}%
    \lop\tmp\tmplist
    [{\xref\tmp\def\\##1{, \xref{##1}}\tmplist}]%
}

%}}}
%{{{   footnotes -- numbered within pages!

%  Data accumulated during the previous TeX run is used to
%  number footnotes consecutively within pages.  (Obviously
%  this fails when footnotes changed -- but the same applies to
%  xrefs anyway.)

%  This is done by keeping two numbers per footnote -- the reference
%  number (increases throughout the document, and is used internally)
%  and the mark number (used in the completed document).

%  Before the first footnote on a given page, the mark number is reset to 0.
%  This can't be done by simply assigning it during the output routine
%  because TeX might have set the footnote on the next page just before
%  shipping out the current one.  Therefore I keep a list -- call it xs
%  -- of footnotes that were the 1st note on a page during the previous
%  run.  Thus xs will be an increasing list starting with 1 (because the
%  1st footnote in a file is automatically the first on a page!)

%  Then the algorithm for each note is to increment the reference
%  number, and check it against head(xs); if it is equal then set the
%  mark number to 1 and xs := tail(xs).  Otherwise, it is less than head(xs)
%  hence on the same page as previously, so increment the mark number.

%  For efficiency (ha!) xs is represented by \notelisthead (a countdef
%  name, set to head(xs)), and \notelist (a list macro, set to tail(xs)).

\def\notelist{}  % set by reading .aux file
\newcount\notelisthead

\newcount\notemarkcount
\newcount\noterefcount

\def\note			% start a footnote
{%
    \global\advance\noterefcount1
    \ifnum\noterefcount=\notelisthead
	\ifx\notelist\empty
	    \global\notelisthead=10000
	\else
	    \glop\tmp\notelist
	    \global\notelisthead=\tmp
	\fi
	\notemarkcount=0
    \fi
    \global\advance\notemarkcount1
    \edef\tmp{\string\notedatum{\the\noterefcount}{\noexpand\folio}}%
    \write\auxfile\expandafter{\tmp}%
    \footnote{\number\notemarkcount}%
}


%  The data in the .aux file is (reference number, folio) pairs
%  This is converted into a list of reference numbers for which
%  the folio is different from the preceeding one, stored in macro
%  \notedatumlastpage.

\def\notedatumlastpage{--!--}

\def\notedatum#1#2%
{
    \def\tmp{#2}
    \ifx\notedatumlastpage\tmp \else
	\gappend\notelist{#1}
	\gdef\notedatumlastpage{#2}
    \fi
}

%}}}   footnotes

\catcode`@=11
\inputifexists{\jobname.aux}
\catcode`@=12
\newwrite\auxfile \openout\auxfile=\jobname.aux

\ifx\notelist\empty 
    \notelisthead=10000 	%  No footnote data
\else 
    \lop\tmp\notelist \notelisthead=\tmp  %  set up notelisthead
\fi
%}}}   aux file
%{{{   abbrevs

\def\url#1#2#3#4{(\urlnoparens{#1}{#2}{#3}{#4})}
\def\urlnoparens#1#2#3#4%
{%
	URL {\tt#1:\allowbreak
	//#2/\allowbreak
	#3/\allowbreak
	#4}%
}

\newif\ifkbnames

%  Captions for \topins items
\newcount\tablecount
\def\caption#1{\global\advance\tablecount1
	\medskip\noindent {\bf Table~\n{\the\tablecount}}\quad #1\par}

\def\cmrname#1{{\tt\lowercase{#1}}}

\def\ISO#1{ISO~\n{#1}}

% Use \vn{N} for version number after `\LaTeX' or `NFSS': \LaTeX\vn2e
%  This is because they prefer no space between name and version ID
%  whereas I think there ought to be one.
\def\vn{~\n}
\def\LaTeXe{\LaTeX\vn2e\note{Note that `\LaTeX\thinspace\n2\lower0.5ex\hbox{\gr
	e}' and `\LaTeX\vn2e' are the same thing.  The lowered `{\gr e}'
	is merely an excessively fancy way of saying
	`e'.}\global\def\LaTeXe{\LaTeX\vn2e}}

\def\texchar#1{\hbox{`{\tt#1}'}}
\def\oct#1{\hbox{\showmark\'$\it#1$}}
\def\hex#1{\hbox{\showmark\H\tt#1}}
	
\def\TUGboat{{\it TUGboat}}

\def\frac#1/#2%
{%
	{\rmb\lowercase{#1}/\uppercase{#2}}%
}

\def\showmark#1%	#1 is an \accent command
{\hbox to 0.5em{%
    \def\typeImark##1{\char##1\relax}%
    \def\typeIImark##1{\char##1\relax}%
    \hss #1{}\hss
}}


\def\marginchar#1%
{%
    \vadjust
      {
	\moveleft\leftmargin \vbox to 0pt
	  {
	    \vss
	    \hbox to \leftmargin{\bigrm \hss#1\hss\hskip1pc}%
	  }
      }%
}

%%  Put this LAST!!
\active\" \append\verbatimplains\"
\def"#1"{\hbox{\it#1\/}}

	
%}}}   abbrevs

\def\package{Malvern~\n{1.2}}
\def\ttpackage{Malvern-1.2}

%**end of header

%}}}  preamble
%{{{ Introduction
%{{{  demo

\vfill\eject
\begingroup \def\line#1{\hbox to \hsize{\kern-\leftmargin #1}}
\bigrm \baselineskip=52pt plus 1pt
\line{A\A\AE BCD\DH E\E FG\hfil.,:;`',,``''}
\line{HIJKL\L MN\NG O\O\OE\hfil?`!`!?}
\line{PQRST\TH UV\hfil a\a\ae bcd\dh\vd}
\line{WXYZ\hfil  e\e f\/ff\/fi\/fl\/ffi\/fflgh\h i\i}
\line{-\/-->>\hfil   j\j kl\l\vl mn\ng o\o\oe pqrs\ss}
\line{\copyright\registered\hfil <<--- t\vt\th uvwxy\&z}
\line{\rbrace\rbrack\char41 \rangle\hfil
	>\degrees\char42 \trademark\Box \careof \char152\char153
	\char155 <
	\hfil \langle\char40 \lbrack\lbrace}
\line{\spaceskip=0pt plus 1fil
	\dag\ \ddag\ \P{}\ \S{}\ @\ \#\ \$\ \%\ \&\ \cents\ \pounds\
	\currency\ \permille\ \yen\ \florin}
\line{123456\hfil\cdot/+=\times\char175\bullet\hfil\uppercase{1234}}
\line{7890 \hfil\uppercase{567890}}

\vskip-\prevdepth
\nointerlineskip
\vbox to 0pt
{
    \vss
    \centerline{\headingrm 48-pt Malvern 55}
}

\headline={\hfil}

\eject
\endgroup

%}}} demo.tex
	
\part{Introduction}

\iflong
%{{{  about TeX and LaTeX

\section{About \TeX\ and \LaTeX}

\subsec{How to pronounce `\TeX'}

	The "X" in "\TeX" is intended to stand for a Greek letter chi
	({\gr Qq}), which is like the "ch" in Scottish "loch" or German
	"ach", and is usually pronounced `k' by people whose languages
	do not include that sound.  Therefore `\TeX' should be
	pronounced either `te"ch"' or simply `teck'.  It is *not*
	pronounced `tecks'.

	Moreover, the lowering of the "E" in "\TeX" is intended to
	remind us that it stands for epsilon ({\gr Ee}).  When it is not
	possible to print the "E" lowered, \TeX\ should be referred to
	as `TeX' (but not `Tex' or `TEX').

	^^{Lamport, Leslie} Lamport suggests that "\LaTeX" might be
	pronounced `lay-\TeX', `lah-\TeX', or even `lay-teks'!  On
	systems without the capability to exactly reproduce his logo you
	might try `L\kern-0.2em\raise 0.5ex\hbox{a}\TeX', `La\TeX', or
	`LaTeX'.

\subsec{\TeX}

	\TeX\ is a powerful and flexible typestting program.  It was
	devised by ^^{Knuth, Donald E.} Donald~E.\ Knuth with the
	intention of producing with a computer typeset text (including
	mathematics and complex tables) as good as that of a
	professionally-produced book.  \TeX\ goes to extraordinary
	lengths to produce optimal line-breaks and correctly aligned and
	spaced mathematics, in a way in which no DTP program currently
	does.  Because \TeX\ is a free program---that is, anyone can get
	a copy of the |WEB| code and create a version that runs on their
	computer---and uses plain text files, it is very widely used for
	producing documents and exchanging them between different
	computer systems, especially those containing a lot of
	mathematical notation.

	Knuth describes \TeX\ in {\it The \TeX book}~\cite{TeXbook}.
	While this manual lacks the three-level section numbering that
	computer users are accustomed to, it does describe everything
	there is to know about \TeX, from writing simple manuscripts to
	designing macros and producing new formats.

	\TeX\ was designed to be used on a variety of different systems,
	at a time when the pictorial user-interfaces popular today were
	still rare.  Therefore \TeX\ is entirely text-based: the user
	feeds \TeX\ a manuscript with typesetting commands embedded in
	it, produced with a normal text editor.  \TeX\ operates in a
	manner similar to a compiler, converting the textual
	representation of the document into a compact format called DVI
	(from `device-independent'), analogous to machine code.  The
	file of DVI information is printed using a separate
	printer-driver program.

\subsec{\LaTeX}

	\LaTeX\ is a front-end to \TeX, written in \TeX's own macro
	language.  It is often used in an attempt to evade the full
	complexity of operating with \TeX.  Strictly speaking, \LaTeX\
	is a \TeX\ \dfn{format}, that is, a package of \TeX\ macros
	compiled into a compact form.

	As the first macro package to offer facilities for automated
	sectioning, bibliography, and table-of-contents generation,
	\LaTeX\ has become a very popular way of producing technical
	documents.

	The standard introduction to \LaTeX\ is {\it\LaTeX: A Document
	Preparation System}~\cite{LaTeX} by the designer of \LaTeX,
	^^{Lamport, Leslie} Leslie Lamport.  This book includes a
	summary of the \LaTeX\ macros, but stops short of describing how
	to go about augmenting \LaTeX's facilities with new styles and
	style options.

	At the time of writing, two new versions of \LaTeX\ are under
	development.  \LaTeX\vn3 will be a complete rewrite; in the
	meantime, \LaTeXe\ will be released as a new standard \LaTeX.

\subsec{Fonts for \TeX}

	The fonts used with \TeX\ documents are described in two parts.
	The first part, the font metrics (called the TFM file), is used
	by \TeX\ itself when generating the DVI file.  The second
	describes how the characters are drawn by the printer; it is in
	a format that depends on the particular font, driver and
	printer.  Common formats for the second part of a font
	description are GF (`generic fonts' produced by \MF), PK
	(produced from GF files and used by many printer drivers), VF
	(composite fonts, that is, fonts formed by combining several
	other fonts), and \PS\ types~$1$ or $3$ (used on \PS\ printers).


	Normally, computer fonts are created from designs drawn on paper
	at a large size (a cap-height of $100\,\rm mm$, say) and then
	the outlines of the letters are scanned in by locating points
	along the curves with a probe (or by scanning the shapes and
	having a program calculate the outlines from the resulting
	bitmaps).  Normally by the time the font is being scanned, all
	the design decisions have been taken and the role of the
	computer operators is merely to produce a faithful rendering in
	the correct format.

	\MF\ is a program which converts outlines to bitmaps and creates
	GF files (which contain a bitmap for each glyph).  The outlines
	are described using \MF's own programming language, with a
	\dfn{character program} for each letter.  Traditionally \MF\
	font design delays some aesthetic decisions until well into the
	programming stage---in Knuth's example ("\MF\/book", Ch.~$5$),
	sixteen trial letters "I" are generated with different stem
	widths so that they can be compared against a sample letter "O".

	\MF\ programs can be arbitrarily complex (although the peverse
	nature of the \MF\ language ensures that any attempt to be too
	clever is severely punished).  A given program can be used to
	generate different fonts---light and boldface, for example.
	Because \MF\ programs know what device the font is being
	generated for, they can include code intended to make the font
	digitize well (that is, ensure that the bitmaps resemble the
	original intention of the designer even on low-resolution
	devices).  The practical upshot of this is that the facilities
	of the latest fancy formats for other systems (`multiple
	masters', `intelligent scaling', `hints') were available in \MF\
	systems years ago, but the programming skill required to use
	them prevented them from being widely used.

%}}}  about TeX and LaTeX
\fi
%{{{  Nomenclature

\section{Conventions used in this handbook}
	
	A distinctive font will be used for examples of literal text
	such as commands to the computer and the contents and names of
	text files:
\begin example
	|ABCDEFGHIJKLMNOPQRSTUVWXYZ|
	|abcdefghijklmnopqrstuvwxyz|
	|0123456789"#$%&@*+-=,.;:?!|
	|()<>[]{}`'\||/_^~|
\end example
	Italic letters are used for the names of some \MF\ variables,
	for metasyntactic variables, for words mentioned in sentences
	(such as the word "word" here), and for the titles of books and
	so on.

	Boldface is sometimes used for \MF\ `sparks'.  A bold italic
	face is used for the first occurence of technical terms.

\subsec{Citations and references}

	Numbers appearing in square brackets (for example, [$1$]) refer
	the reader to other sources for more information; they are
	listed in \xref{bib}.

\subsec{Syntax descriptions}

	Occasional syntax descriptions will be given in the usual
	^^{Backus--Naur formalism}^{extended Backus--Naur Formalism}
	(extended BNF, or EBNF): in other words, an extension of the
	notation described at the start of Chapter~$24$ of the {\it\TeX
	book} \cite{TeXbook}, `Summary of Vertical Mode'.  Syntactic
	entities are written with angle brackets, thusly: \langle{\it
	dimen\/}\rangle.  The arrow `{\rmb\char1}' is read as `is
	defined as', and the vertical bar `{\rmb\char124}' as `or'.
	Literal text is enclosed in quotation marks.

	Square brackets on the RHS of the arrow enclose optional text.
\iflong
	As an example, take this definition from Chapter~$24$:
\begin bnf
	\<unit of measure> \> \<optional spaces> \<internal unit> 
		\\~\<optional {\tt true}> \<physical unit>.

	\<optional {\tt true}> \> |true| \\ \<empty>.

	\<optional spaces> \> \<empty> \hfil\break 
		\\ \<space token> \<optional spaces>.
\end bnf
	Using brackets the first two rules can be replaced by just one:
\begin bnf
	\<unit of measure> \> \<optional spaces> \<internal unit> 
		\hfil\break
		\\~[ |true| ] \<physical unit>.
\end bnf
\fi
	Braces enclose text that may repeated, or rather text which may
	appear any number of times including zero.  
\iflong
	For example,
	\<optional spaces> can be defined as:
\begin bnf
	\<optional spaces> \> \{ \<space token> \}.
\end bnf
	Or we could rewrite the whole of the above definition of \<unit
	of measure> as follows:
\begin bnf
	\<unit of measure> \> \{ \<space token> \} \<internal unit> 
		\hfil\break
		\\~[ |true| ] \<physical unit>.
\end bnf
\fi
	There is no conflict with the use of brace tokens in the
	{\it\TeX book}\/: all such tokens will be written in quotation
	marks in syntax rules (`|{|' and `|}|' rather then \lbrace\ and
	\rbrace), and besides, I~will not be referring to brace tokens.

\iflong
\subsec{Typefaces, founts and fonts}

	In traditional English usage, a \dfn{typeface} (or \dfn{face})
	is the design of a set of letters, figures and symbols at any
	size: an abstract concept.  A \dfn{fount} is an implementation
	of a typeface at a particular size; normally this refers to a
	phyiscal object, such as a set of punches or a phototypesetter
	film strip.  By analogy with "program", I will use "font" for an
	analagous computer resource, whether it describes a typeface at
	any size (as a \PS\ scalable font does) or at a particular
	design size (as \TeX\ fonts do).

\subsec{Italic, slanted, oblique, latin, roman and upright}

	The term \dfn{italic} used to refer to a cursive variation of a
	typeface.  In phototypesetting days, sanserif faces with no
	italic could have an obliqued form produced using prisms.
	Generally \dfn{slanted} was used to mean either italic (if
	available) or obliqued.  Nowadays almost all faces are supplied
	with a slanted form; for sanserif fonts these are inconsistently
	referred to as "italic" or "oblique".  So far as I~know,
	Computer Modern is the only type family to include three
	different slanted alphabets---italic, obliqued and so-called
	`math italic'.

	The word "roman" is used to mean ordinary upright letters, as
	distinct from italic or bold and the like.  I~shall generally
	distinguish between this use of "roman" and the use of "latin"
	to refer to the alphabets derived from the alphabet devised for
	writing the Latin language in classical times, as compared with
	the Greek and Cyrillic alphabets, for example.

\iffalse
\subsec{Families and super-families}

	A family is a collection of typefaces, intended to be compatible
	with each other.  For example, the ITC Avant-Garde Gothic family
	has members such as ITC Avant-Garde Gothic Demi and ITC
	Avant-Garde Gothic Light Oblique, and Univers has family members
	(also called variants) like Univers~$55$ and Univers~$48$.

	Most families consist of a few different weights of italic and
	upright faces.  There are a few `super-families', such as ITC
	Stone, Lucida and Computer Modern, which contain many different
	variants such as sanserif, typewriter, informal.  ITC Stone
	Sans, ITC Stone Serif and ITC Stone Informal may be viewed as
	three separate families that were designed together
	(collectively a `super-family'); or, ITC Stone can be viewed as
	is a family with Sans Semibold Italic, Informal Italic and so on
	as faces within it.

	I~personally prefer the `splitter' approach; it keeps the number
	of names of font variants a document manager needs to know about
	relatively small, and does not greatly increase the number of
	family names, because there are very few super-families compared
	with the enormous number of normal-sized families.  It also
	makes the design of menu-based font-selectors more
	straightforward.  (The names Adobe uses for its fonts indicate
	that they are `splitters' too.)  Unfortunately, \LaTeX\ was
	designed on the implicit assumption that `sans serif', `slanted'
	and `caps and small caps' are variants present in *any* family.

	Super-families are an intuitively pleasant  concept for
	classifying fonts, but they are not relevant to any
	document-processing program I~can think of off the top of my
	head.  That is, given document~$A$ with ITC Stone Serif and Sans
	and document~$B$ with Palatino and Univers, there's nothing
	useful that a computer can do with the information that $A$
	takes all its faces from one super-family and $B$ does not.

\fi	% end of iffalse
\fi	% end if iflong

\subsec{Composite letters}

	I will use the phrase \dfn{composite letter} to refer to letters
	which, regardless of their meaning in their respective
	languages, are *written* (not necessarily typeset) as one of the
	letters {\it A}--{\it Z} (called a \dfn{base glyph}) with that
	addition of some sort \dfn{mark}.  Thus composite letters
	include German umlauts ({\it\"a}, {\it\"o}, {\it\"u}), accented
	letters in languages like Spanish ({\it\'a}, {\it\'e}, etc.),
	and special letters like {\it\aa}, {\it\"a} and {\it\"o} in
	Scandinavian languages.

	As well as marks that go above letters, there are some that go
	below ({\it\c c}, {\it\a}) and through letters ({\it\l},
	{\it\o}).  Normally it will be useful to distinguish between
	composite letters with the mark above them and the latter two
	categories.  For example, "\A" is the same height as the letter
	"A", but if the letter "\AA" is to have the same height as "A"
	then the base glyph must be shrunk slightly.

\subsec{Hyphens and dashes}

	To prevent confusion, I~shall use the word "dash" to refer only
	to the en- and em-dash symbols (`--' and `---'), and I~will use
	the word "hyphen" for hyphens (`-').  	
\iflong
	Hyphens (produced with
	one ASCII minus sign) are used in English for joining words
	("ear-ring", "get-at-able"), to disambiguate a few odd compounds
	("re-cover", "co-operate"), and when dividing words at the ends
	of lines.  Short (en) dashes (|--|) are used in joining pairs
	where there is movement or tension between them ("London--Oxford
	route", "\n{1980}--\n{85}").  Long (em) dashes (|---|) are used
	in a similar way to parentheses or a colon.
\fi

\subsec{Sundry quotation marks}

	In English printing, the ^{quotation marks} are an ^{apostrophe}
	(') and an ^{inverted comma}---with hot metal typesetting this
	was a literally a ^{comma} that had been put in place
	upside-down.  In German, a rather more sensible arrangement is
	used where two commas start quotations, and two inverted commas
	close them: ,,so``.  To prevent confusion, I~shall consistently
	refer to the character `\,`\,' as an inverted comma rather than
	`^{left quote}', because it is not always a *left* quotation mark:
\begin display
	, & comma & ,, & double comma\cr
	` & inverted comma & ``	& double inverted comma\cr
	' & apostrophe & '' & double apostrophe\cr
\end display
	There is some potential for confusion with regards to the ASCII
	characters `|`|'and `|'|', used by \TeX\ to stand for an
	inverted comma and apostrophe respectively.  ASCII was designed
	with typewriter-like devices in mind, and |'| was intended as a
	combined neutral single quotation mark, acute accent and
	apostrophe, which is not quite the same as \TeX's use for it.
	Traditional UNIX names for these two characters are \dfn{quote}
	and \dfn{backquote}.
	
	When I~am referring to French ^{guillemets}, I~shall use `left
	guillemet' to refer to the one that points to the left (<<).  In
	French and many other languages, this is used as an opening
	quotation mark <<\,thus\,>>; in German it is sometimes used as a
	closing quotation mark >>so<<.
	
\subsec{How to refer to Malvern fonts}

	The name `Malvern' is spelled as a normal proper noun (no fancy
	typography is necessary).  In English it is pronounced with the
	accent on the first syllable, with the "a" pronounced as in
	either "pal" or "pall" (natives of the town it is named after
	say `*mawl*-v{\rmb\char116 }n', I~tend to pronounce it
	`*mal*-v{\rmb\char116 }n').  In other languages it may as well
	be pronounced and declined according to the orthography of the
	language.

	When describing a particular version of Malvern, convention
	favours an unbreakable space between name and number:
	Malvern~$1.1$ is `|Malvern~1.1|' (like Henry~VIII, Occam~$2$,
	Fortran~$9$X).  The same applies to naming members of the
	Malvern font family (Malvern~$55$, Malvern~$76$) and Malvern
	font encoding family (Malvern~A, Malvern~B).  In the latter case
	the letters need not be italicized.

%}}}   Nomenclature
%{{{  Numeric style suffixes

\section{Malvern's font names}

%{{{  Univers

\subsec{Numeric style codes}

	The first font family designed as a unit was ^{Univers}.  The
	designer, Adrien Frutiger\index{Frutiger, Adrien}, introduced
	two-figure codes to describe the different variations, instead
	of a names like `bold italic' and `light condensed'.  For
	example, the `roman' face is called Univers~$55$, italic
	Univers~$56$, boldface Univers~$75$ and lightweight condensed
	italic Univers~$38$.  Frutiger has used his system in all his
	subsequent font families.  ^{Malvern} and other new fonts I~make
	use a similar system.

%}}}  univers
%{{{  Malvern suffixes

\subsec{The style codes used with Malvern}

\xreflabel{s-stylecodes}{\S\thesecno}
	Just as with Frutiger's system, the first digit gives the weight
	of the font and the second digit gives the width (in the sense
	of `{\cp compressed}' vs.\ `{\xp expanded}'), as in the
	following table:
\begin table \strut$#$\hfil\quad&#\hfil\qquad&$#$\hfil\quad&#\hfil\cr
	  & \it first digit		&   & \it second digit\cr
    \noalign{\smallskip}%
	1 & ultra-light (hairline) 	& 1, 2 & extra expanded \cr
	2 & extra-light (thin) \cr
	3 & light 			& 3, 4 & expanded/extended \cr
	4 & book (semi-light) \cr
	5 & medium 			& 5, 6 & normal width \cr
	6 & demi (semi-bold) \cr
	7 & bold 			& 7, 8 & condensed/compressed \cr
	8 & extra-bold \cr
	9 & ultra-bold  		& 9, 0 & extra compressed \cr
%%%%	0 & black \cr
\end table
	The second digit is even for a slanted face, odd for an
	upright face---so that the italic version of Malvern~$55$ is
	$56$, and of $19$ is~$10$.

	This in theory gives $90$ styles.  The middle odd-numbered
	weights ($3$, $5$, and $7$) are expected to be the more commonly
	used ones.  Similarly, the three middle widths are expected to
	be used more often than $1$ and~$9$.  The practical upshot of
	this is (a)~they are more likely to be available as \mc{PK}
	files and (b)~they are more likely to have had their \MF\
	programs tested and debugged.

	This system of names only allows one slanted form and cannot
	distinguish `compressed' from `condensed'.  Normally this
	fine---before Knuth's Computer Modern came along, it was only
	heretics with \PS\ or phototypesetters without italics available
	who would resort to obliqued fonts.  The \MF\ programs that
	describe Malvern have the capacity to produce `vertical italic'
	and obliqued forms, but these are not part of the standard
	Malvern pantheon.  They are *not* intended to be used in normal
	bookwork.
%}}}  Malvern suffixes
%{{{  TeX fonts names

\subsec{\TeX\ font names}

\xreflabel{s-font-names}{\S\thesecno}
	The question now is how to represent these fonts using
	reasonably portable \TeX\ font names.  Recall that only
	lowercase letters and digits can be used, and that names must be
	limited to a length that even stupid filesystems like ^{MS-DOS}
	can handle.

	I~have my own system of external font names of the following
	form:
\begin bnf
	^\<font name> \> \<family> \<style> \<encoding> \<size>.

	\<family> \> \{ \<letter> \\ \<digit> \} \<letter>.

	\<style> \> \<digit> \<digit>.

	\<encoding> \> \<letter> [ \<letter> ].
\end bnf
	Where a \<size> is the design size in points, possibly including
	`|p|' as a decimal point, or `|m|' to indicate millimetres:
\begin bnf
	^\<size> \> \<points> \\ \<millimetres>.

	\<points> \> \<digits> [ |p| \<digits> ].

	\<millimetres> \> \<digits> |m| [ \<digits> ].

	\<digits> \> \<digit> \{ \<digit> \}.
\end bnf
	In the case of Malvern, the \<family> is |ma|.  The \<encoding>
	describes the subset of all the Malvern characters that this
	font contains---for example |s| is the \TeX\ Text character set.

	Thus $10$-pt Malvern~$55$ with \TeX\ Text encoding is
	given the name |ma55s10|, which fits nicely into the eight
	characters allowed by ^{MS-DOS} filenames.  More eccentric
	sizes, like $7\cdot2\pt$ have names like |ma55s7p2| (eight
	characters) or |ma55s12p5| (nine characters).

	Here's a partial list of encoding codes:
\begin display
	\tt a&	Malvern A (latin), see \xref{tab-charcodes-a}\cr
	\tt b&	Malvern B (supplement), see \xref{tab-charcodes-b}\cr
	\tt c&	Malvern C (Cyrillic), see \xref{tab-charcodes-c}\cr
	\tt g&	Malvern G (Greek), see \xref{tab-charcodes-g}\cr
	\tt s&	\TeX\ Text \cite{TeXF1}\cr
	\tt az&	\TeX\ Text with old-style numerals\cr
	\tt aa&	\TeX\ Text with small capitals replacing lower case letters\cr
	\tt ab&	\TeX\ Text with old-style numerals and small capitals\cr
	\tt ar&	\TeX\ Extended Text---Latin (Cork) \cite{Cork}\cr
	\tt as&	Cork with old-style numerals\cr
	\tt at&	Cork with small capitals\cr
	\tt au&	Cork with old-style numerals and small capitals\cr
\end display
	\xreflabel{tab-encoding-codes}{\S\thesecno}

%}}}  TeX fonts names
%{{{  Berry's scheme

\subsec{Naming Malvern fonts in  Karl Berry's system}

	Karl Berry has described a system for naming fonts newly
	introduced to the `\TeX\ world' \lcite{\\{Berry}\\{Berry2}}.  A
	subset of the Karl Berry's syntax for font names is as follows:
\begin bnf
	\<font name> \> |fmv| \<style> \<size>.

	\<style> \> \<weight>  \<variant> \{ \<variant> \} [ |r| \\ \<width> ]
		\hfil\break \\ \<weight> |r| \<width>
		\hfil\break \\ \<weight>.

	\<weight> \> |t| \\ |i| \\ |l| \\ |k| \\ |m| \\ |d| \\ |b| \\
		|x| \\ |u| \\ |c|.

	\<variant> \>  |i| \\ |9| \\ |o| \\ |u| \\ |q| \\ |c|.

	\<width> \> |x| \\ |c|.

	\<size> \> \<digit> [ \<digit> ].
\end bnf
	This definition is incomplete, because the rules governing the
	order of variant letters and when variant and width letters may
	be omitted are complicated\iffalse:
\begin bullets
\\
	If \<variant> letters are present, arrange them in alphabetical
	order.
\\
	If there is a \<variant>, and there is no \<width>, and the
	last \<variant> is a digit or a valid \<width>, then the
	trailing |r| must be present.  Otherwise it must be omitted.
\end bullets
\else.\fi

	The \<weight> letters listed above correspond to the first
	digit's values of $1$, \dots, $9$, $0$.  A \<variant> of |i|,
	|o| or |u| indicates a slanted, obliqued or `upright italic'
	style instead of upright; |9| is old-style digits; |c| is caps
	\& small caps; |q| is the Cork encoding (instead of the \TeX\
	Text encoding).  Fonts with |c| or |q| variants have to be
	created as virtual fonts, using Alan Jeffrey's ^|fontinst|
	package (\xref{s-fontinst}).  Fonts using \TeX\ Text encoding
	may be generated directly with \MF.
	
	This gives names like `|fmvm12|' for $12$-point Malvern~$55$,
	`|fmvdic18|' for Malvern~$68$.  Malvern~\n{55} with old-style
	numerals would be |fmvm9r10|.  On the other hand, Malvern~\n{68}
	with old-style numerals would be `|fmvd9ic10|'---nine
	characters---and we haven't even specified a base encoding yet.

\ifkbnames\else
\begin notepar
	As of October~\n{1994}, the font names standard does not allow
	for combinations for more than two variants (where `variants'
	includes encodings and font shape) without breaking the
	\n8-character limit imposed by stupid filesystems like MS-DOS
	and ISO~\n{9960}.  The only temporary solution I~can offer is to
	extend the nonstandard `\/{\tt ma}\/-' naming scheme to include
	encoding codes for the fonts used by plain \LaTeX, NFSS and
	NFSS\vn2.  Therefore `\/{\tt fmv}\/-' names are not used at
	present.
\end notepar
\fi	

\xreflabel{s-kb-names}{\S\thesecno}

%}}}  Berry's scheme
%}}}
%}}} Introduction
%{{{ Using Malvern

\part{Using Malvern in \TeX\ documents}
%{{{  LaTeX 2e
\section{Using Malvern with \LaTeXe\ or NFSS\vn2}

\index{LaTeX2e@\LaTeXe|+}
\index{NFSS\vn2|+}

\subsec{Background to \LaTeXe}

	The release of the so-called New Font Selection Scheme (^{NFSS})
	has left the \LaTeX\ world in a slightly confused state, with
	neither plain \LaTeX\ nor \LaTeX\ augmented with NFSS considered
	`standard' anymore.  This is partly because \LaTeX+NFSS is not
	quite compatible with plain \LaTeX, so older documents might
	have to be edited to suit the new system.

	\LaTeX\ version \n3 is intended to put an end to this confusion,
	but it will be some time before ^^{LaTeX3@\LaTeX\vn3} \LaTeX\vn3
	is generally available.  In the meantime a new, `standard'
	version has been released, called \LaTeXe.  \LaTeXe\ is
	backwards-compatible with \LaTeX-\n{2.09} documents, using a new
	syntax to access the newer features.

\subsec{NFSS\vn2}

	\LaTeXe\ includes a new FSS called NFSS\vn2 which uses font
	family definition ({\tt fd}) files to tell it how to obtain
	fonts in a given font family and encoding family.  For example,
	a file describing Cork-encoded Malvern is called `^|T1fmv.fd|':
	"T\n1" is the NFSS\vn2 code for the Cork encoding, and "fmv" is
	Karl Berry's code for Malvern.  Once such an {\tt fd} file is
	installed, to use Malvern fonts in a document you would use
	something like the following:
\begin display
	|\fontencoding{T1}|\cr
	|\fontfamily{fmv}|\cr
	|\selectfont|\cr
\end display
	This might go in a document preamble, but normally such changes
	to the overall look of a document belong in a package.

\subsec{Malvern in NFSS\vn2}

	The Malvern includes a sample set of {\tt tfm}, driver, {\tt vf}
	and {\tt fd} files so that Malvern should be usable with
	\LaTeXe\ or NFSS\vn2 without very much effort, both with the
	\TeX\ Text and Cork encodings (OT$1$ and T$1$).  This assumes
	that
\begin bullets
     \\	{\tt dvi} previewers and printer drivers that understand virtual
	fonts are available;
     \\ printer fonts (such as {\tt pk} files) will be generated
	automatically when needed (for example, by invoking a
	^|MakeTeXPK| script on UNIX-\TeX\ systems).
\end bullets
	These files were generated with ^^{Jeffrey, Alan} Alan Jeffrey's
	^|fontinst| package (see \xref{s-fontinst}).  They do not
	include support for italic caps-and-small-caps because there is
	at present no font-shape name for this combination.

\index{LaTeX2e@\LaTeXe|-}
\index{NFSS\vn2|-}

%}}}  LaTeX 2e
%{{{  LaTeX/NFSS

\section{Using Malvern with \LaTeX~\n{2.09}}

\index{LaTeX2.09@\LaTeX~\n{2.09}|+}

	In \LaTeX, It is possible to use a font or two in an
	ad-hoc fashion using the ^|\newfont| command:
\begin display
	|\newfont{twlgr}{ma55g12}|\cr
	|\newcommand{\textgreek}[1]{{\twlgr#1}}|\cr
\end display
	Then `|\textgreek{alfa}|' generates `{\gr alfa}'.  The problem
	here is that fonts loaded this way do not change size
	automatically when commands like ^|\footnotesize| or ^|\large|
	are used; as a result they will not work properly in footnotes
	or section headings.  They also will not work in maths mode.
		
\subsec{Using NFSS}

	The so-called New Font Selection Scheme (^{NFSS}) is an
	extension of \LaTeX~\n{2.09} to make it easier to switch between
	fonts \cite{NFSS}.  The \LaTeX\ system I have access to does not
	use NFSS, so I~can not test NFSS code and I~do not know how much
	documentation on adding new fonts is supplied with the package.
	I have included \index{Rahtz, Sebastian P.~Q.} Sebastian Rahtz's
	^|malvern.sty| \cite{nfss-malvern}, and also style files named
	^|fmvnfss.sty| and ^|fmv9nfss.sty|\note{"Fmv" is Karl Berry's
	code for Malvern, and "nfss" distinguishes it from the plain
	\LaTeX\ version and any \LaTeXe\ version.  The digit `$9$' is
	the Berry code for old style figures.} which are modelled on
	|malvern.sty|, but which use the fonts included in the Malvern
	distribution.  They do not have support for the ^|\sc|
	declaration.
	
\subsec{Making fonts change size without NFSS}

	To make fonts that change size properly requires a style file
	that reprograms plain \LaTeX's size-changing commands (I~shall
	refer to \LaTeX\ version~\n{2.09} without NFSS as `plain
	\LaTeX~\n{2.09}', and to \LaTeX~\n{2.09} with NFSS as
	\LaTeX~\n{2.09} + NFSS').  To do this requires some knowledge of
	the way plain \LaTeX's size-changing and font-changing commands
	work \cite{lfonts}.

	The ^{size-changing commands} like |\normalsize| and
	|\footnotesize| work by calling internal commands |\|"size"|pt|
	and |\@|"size"|pt|, where "size" is the type size in points,
	expressed as roman numerals.  For example, in a $10$-pt
	document, |\normalsize| invokes |\xpt| and |\@xpt| in order.
	The |\@|"size"|pt| macros are initially defined to expand to
	nothing.  The |\|"size"|pt| macros link font nicknames |\rm|,
	|\it| etc.\ with real fonts like |\tenrm| or invocations of
	|\@getfont|\index{getfont@{\tt\char92\char64 getfont}}.  The
	macro |\@getfont| allows some fonts to be loaded on demand
	rather than preloaded when \LaTeX\ starts up (saving some memory
	when they are not used).  When expanded, |\@getfont| appends
	assignments to |\@|"size"|pt| that override the definition in
	|\|"size"|pt|.  The practical upshot of this is that |\@getfont|
	is only invoked the first time the font is referred to;
	subsequent uses of that font use the assignment added to
	|\@|"size"|pt|.

	There are two ways to add new fonts to add new fonts to \LaTeX's
	lists.  The quick kludge is to let |\@getfont| do the work.  For
	example, the following code might go in a {\tt sty} file:
\ifkbnames
\begin lines
	|\@getfont{\prm}{0}{\@xpt}{fmvr10}|
	|\@getfont{\prm}{0}{\@xipt}{fmvr11}|
	|\@getfont{\prm}{0}{\@xiipt}{fmvr12}|
\smallskip
	|\@normalsize|
\end lines
\else
\begin lines
	|\@getfont{\prm}{0}{\@xpt}{ma55s10}|
	|\@getfont{\prm}{0}{\@xipt}{ma55s11}|
	|\@getfont{\prm}{0}{\@xiipt}{ma55s12}|
\smallskip
	|\@normalsize|
\end lines
\fi
	This loads the fonts and adds assignments to the |\@|-commands
	so that they are used in $10$, $11$ and $12$-pt text.  It also
	has unwanted side-effects like changing the size and current
	font, which are cancelled out by the
	^^{normalsize@{\tt\char92\char64 normalsize}} |\@normalsize| at
	the end.

	A solution that is more efficient if more than a handful of
	fonts are being changed is to make a copy of |lfonts.tex| and
	edit it to load Malvern fonts instead of Computer Modern fonts.
	This approach allows the new fonts to be demand-loaded using
	^^{getfont@{\tt\char92\char64 getfont}} |\@getfont| and so on.
	This is used two sample style files included in the
	distribution, ^|fmvpltx.sty| and ^|fmv9pltx.sty|.\note{"Fmv" is
	the code for Malvern in Karl Berry's system, and "pltx" short
	for `plain \LaTeX'.}  They do not include support for the
	caps-and-small-caps declaration ^|\sc|.

\subsec{Which fonts to use with \LaTeX\ \n{2.09}}

	In the ^{Malvern~\n{1.0}} documentation I~suggested that Malvern
	fonts with the standardized font names
	\lcite{\\{Berry}\\{Berry2}} and Cork encoding \cite{Cork} be
	used with \LaTeX, on the grounds that the Cork encoding is due
	to become the standard for \LaTeX.  Since then the situation has
	changed a little.  The introduction of NFSS\vn2 means that
	future versions of \LaTeX\ will not be tied down to just one
	encoding scheme after all.  Also, new versions of the
	font-naming scheme include encoding codes amongst the variant
	letters, implying that the default is the standard \TeX\ Text
	encoding.

	It follows that the standard names (starting `{\tt fmv}-')
	should be used with the standard `\TeX\ Text' encoding when
	using Malvern fonts with \LaTeX~\n{2.09}, whether with NFSS or
	not.  This way the macros that work with the old encoding (such
	as |\ae|, |\'|, and the like) will not need to be changed.
	NFSS~\n2 will take care of redefining these macros when
	appropriate.

\ifkbnames\else
\begin notepar
	As of October~\n{1994}, the font names standard does not allow
	for combinations for more than two variants (where `variants'
	includes encodings and font shape) without breaking the
	\n8-character limit imposed by stupid filesystems like MS-DOS
	and ISO~\n{9960}.  The only temporary solution I~can offer is to
	extend the nonstandard `\/{\tt ma}\/-' naming scheme to include
	encoding codes for the fonts used by plain \LaTeX, NFSS and
	NFSS\vn2.  Therefore `\/{\tt fmv}\/-' names are not used at
	present.
\end notepar
\fi

\xreflabel{s-LaTeX-fnames}{\S\thesecno}

\ifkbnames
	Malvern \n{75} is intended as the boldface counterpart to
	Malvern~\n{55}, so the font used for \n{10}-pt |\bf| should be
	|fmvb10| rather than |fmvbrx10|.
\fi

\index{LaTeX2.09@\LaTeX~\n{2.09}|-}
%}}}  LaTeX/NFSS
%{{{  plain TeX -- maamac.tex
\section{Using Malvern with plain \TeX: {\tt maamac.tex}}

\xreflabel{s-maamac}{\S\thesecno}
	This ^^{maamac.tex@{\tt maamac.tex}|+} section describes a file
	^|maamac.tex| of definitions to customize \TeX\ to work with the
	Malvern~A conventions.  It covers roughly the same territory as
	Chapter~\n9 of the "\TeX book" \cite{TeXbook}, and assumes you
	have already have macros in place to load Malvern fonts with the
	Malvern~A encoding \lcite{\\{TeXB4}\\{TeXE}}.  At a minimum:
\begin example
	|\input maamac|
	|\font rm=ma55a10 \rm|
\end example

\begin notepar
	The file |maamac.tex| (`*Ma*lvern *A* *mac*ros') was formerly
	called simply ^|malvern.tex|.  The new name is intended to be
	less ambiguous, while still acceptable to file systems with
	short file names.
\end notepar

\subsec{Ligatures}

	The letter ligatures "ff", "fi", "fl", "ffi", "ffl" and
	punctuation ligatures `--', `---', `!`', `?`',
	`\thinspace``\thinspace' and `\thinspace''\thinspace' work as in
	Computer Modern.  In addition, |fj| produces "fj" and |<<| and
	|>>| produce the guillemets `<<' and `>>'.  Note that you must
	use `|''|' to stand for a double-apostrophe---with Computer
	Modern, some people use `|"|', which will not work.

\subsec{Special letters}
	In addition to the special letters "\oe", "\ae", "\aa", "\o",
	"\l", "\ss", "\i" and "\j" there are:
\begin display
	|\A|&\A&|\a|&\a&	Polish "a" with ogonek\cr
	|\E|&\E&|\e|&\e&	Polish "e" with ogonek\cr
	|\TH|&\TH&|\th|&\th&	Icelandic thorn\cr
	|\DH|&\DH&|\dh|&\dh&	Icelandic eth\cr
	|\NG|&\NG&|\ng|&\ng&	Lappish eng\cr
	|\vd|&\vd&|\vt|&\vt&	alternatives to "\v d", "\v t"\cr
	|\vl|&\vl&|\h|&\h&	alternatives to "\v l", "\^h"\cr
\end display
	The commands producing composite letters (|\'e| for "\'e",
	etc.)\ work as before, except that they use different marks on
	capital letters: "\'E" instead of "\accent8 E"\iflong\space (see
	\xref{s-composites} for more information)\fi.

\subsec{Sundry symbols}

	The commands |\$|, |\#|, |\%| and |\&| produce `\$', `\#', `\%'
	and `\&' as in plain \TeX\ (|\$| does not produce a pounds sign
	`\pounds' in italics).  
	In addition, the commands |\P|, |\S|, |\dag| and |\ddag| produce
	signs "\P", "\S", "\dag", "\ddag" that come from the current
	font rather than the mathematical symbol font.

	There are the following additional symbols:
\begin display
	^|\pounds|&	\pounds&	
		Pounds-sterling sign\cr
	^|\cents|&	\cents&		
		Cents sign (alternative to "c")\cr
	^|\currency|&	\currency&	
		Currency sign\iflong , see \cite{s-currency}\fi\cr
	^|\permille|&	\permille&
		Per-mille (per-thousand) sign\cr
	^|\yen|&		\yen&
		Japanese Yen sign\cr
	^|\florin|&	\florin&
		Florin sign (alternative to italic "f\/")\cr
\noalign{\smallbreak}%
	^|\times|&	\times&		
		multiplication sign\cr
	^|\minus|&	\minus&		
		minus sign\cr
	^|\langle|, ^|\rangle|& \langle, \rangle& 
		angle brackets\cr
	|\{|, |\}|& \{, \}&		
		braces\cr
	^|\cdot|&	\cdot&		
		raised dot (British decimal point)\cr
	^|\bullet|&	\bullet&	
		a bullet\cr
	^|\Box|&		\Box&		
		a ballot box\cr
	^|\degrees|&	\degrees&	
		degrees sign\cr
\noalign{\smallbreak}%
	^|\Mc|&		\Mc&
		`Mac', as in \Mc{Donald}\cr
	^|\No|&		\No&
		`Number' or `numero'\cr
	^|\orda|&	\orda&
		feminine ordinal numbers\cr
	^|\ordo|&	\ordo&
		masculine ordinal numbers\cr
	^|\careof|&	\careof&
		indicates indirect address\cr
	^|\copyright|&	\copyright&	
		international sign of copyright\cr
	^|\registered|&	\registered&	
		registered trade mark sign\cr
	^|\trademark|&	\trademark&	
		trade mark sign\cr
\end display
	The inclusion of `\times', `\minus', `\degrees', `\cdot' makes
	it possible to say `\minus17\cdot6\thinspace\degrees C' or
	`50\thinspace mm \times\ 100\thinspace mm' without using maths
	mode (and therefore in the current font).  The symbols that are
	named the same as mathematics symbols in plain \TeX\ use
	|\ifmmode| so that they can use the plain \TeX\ definition in
	formulas and the Malvern character in horizontal mode.

\subsec{Old-style and ranging numerals}

	By default, ^{old style numerals} are used: 0123456789.  These
	are for use in non-technical text, where old style numerals are
	usual when using serif fonts.  To get ^{ranging numerals}
	$0123456789$ use one of:
\begin display
	|$0123456789$|\cr
	|$\textfont0=\font 0123456789$|\cr
	|\uppercase{0123456789}|\cr
	|\caps{0123456789}|\cr
\end display
	The switch to ranging numerals in mathematics mode reflects the
	assumption that mathematics mode is mainly used in documents
	with technical content (besides, old style numerals look wrong
	in anything but the simplest formulas).  Normally a given
	document should not use both styles---which implies that if you
	are using formulas then all numerals in the manuscript must be
	encosed in mathematics delimiters.

\xreflabel{s-ranging}{\S\thinspace\thesecno}

\subsec{Small captials}

	The following macros do not work in \TeX's mouth, meaning that
	they will not have the intended result when expanded in an
	|\edef|, for example\iflong
	\space (see \xref{s-alphabets} for more
	information)\fi.
\begin display
	|\sc{ABC\DH}|&\sc{ABC\DH}&
		Even small capitals\cr
	|\mc{ABC\DH}|&\mc{ABC\DH}&
		Medium capitals\cr
	|\csc{ABC\DH abc\dh}|&\csc{ABC\DH abc\dh}&
		Caps and small caps\cr
	|\caps{ABC\DH abc\dh}|&\caps{ABC\DH abc\dh}&
		All-capitals\cr
	|\lc{ABC\DH abc\dh}|&\lc{ABC\DH abc\dh}&
		Lower case\cr
\end display
	^^|\csc| ^^|\sc| ^^|\mc| ^^|\caps| ^^|\lc| For example, I set my
	postcode with `|\sc{OX1~3QD}|' to get `\sc{OX1~3QD}', and I can
	write `\csc{PostScript}' with `|\csc{PostScript}|'.  Because the
	small capital alphabets are included in all styles, we can write
	`|\csc{\it Fred}|' to get `\csc{\it Fred}', and even produce the
	\LaTeX\ logo in italics: {\it\LaTeX}.  Note that these macros
	work differently from the ^|\sc| or ^|\smc| declarations in
	\LaTeX\ and other formats, which switch to ^|cmcsc10|, a
	separate caps-and-small-caps font.

	The last two transliterate into capitals and lower case---the
	differences between these macros and the |\uppercase| and
	|\lowercase| primitives are:
\begin bullets
    \\	|\caps| and |\lc| (and all the others) make assignments, and so
	can not be expanded in \TeX's mouth; and
    \\	|\uppercase| and |\lowercase| do not affect letters like {\it\O}
	and {\it\ae} when they are produced with control sequences (like
	^|\O| and ^|\ae|).
\end bullets
	The practical upshot is that the primitives are useful when
	implementing strange macros involving creating control sequence
	names on the fly etc.\ and for introducing strange characters
	into definitions, whereas to produce an all-capitals headline
	(or whatever), the macros are better.

	^^{maamac.tex@{\tt maamac.tex}|-}

%}}}   plain TeX -- maamac.tex
\iflong
\input texnical
\fi
% \input ldfontsdoc
	
%}}} Unsing Malvern
\iflong
%{{{ Implementing Malvern

\part{Implementing Malvern}

\iffalse % This section largely superceeded by the installation summary
%{{{  organization of the files
\section{Organization of the files}

	The implementation of Malvern is split into several files.  The
	file |ma.mf| is the `root' file, also called the generic driver
	file.  This reads in |makit.mf| (a collection of macros which
	describe various pieces of letters), |maparams.mf| (which sets
	most of the {\it ad hoc} parameters, and |maencode.mf| (which
	describes the character encoding to be used), before reading in
	the program files (which contain `character programs' which
	actually produce the glyphs) and producing the ligature table.

\subsec{The generic driver file}

	The difference between my generic driver file and a driver file
	like |cmbx12.mf| is that it must have certain variables given
	values before it is read.  At the very least the internal
	\MF\ variable $designsize$ (set with the {\bf font\_size}
	macro) should have a value.  Other variables can be given values
	to signal to the driver to produce a font other than roman (with
	the \TeX\ text encoding):
\begin table $#$\hfil
	& \quad\vtop{\hsize=0.75\hsize \noindent#\strut\smallskip}\hfil\cr
weight & Controls the average thickness of the strokes that make up the
	letterforms.  A value of $1$ means the normal weight, $1\cdot5$
	is bolder, $0\cdot75$ is lighter.\cr
hratio & Horizontal ratio: controls the average width of the characters.
	The shapes of the characters will be squeezed or expanded (as if
	by a |xscaled| transform) but the thickness of strokes will
	remain the same.  This is used to obtain compressed or expanded
	fonts.\cr
slant & The transform `slanted $slant$' is catenated to
	$currenttransform$, so that all the glyphs are obliqued.
	Normally only non-zero if $italicness$ is defined and
	nonzero.\cr
italicness & A number from $0$ to $1$, where $1$ represents an italic
	font (some letters are drawn in a more flamboyant manner) and
	$0$ a roman font.\cr
encoding & An integer, used to select one of a set of character
	encodings for the font (see \xref{s-encoding}).\cr
\end table
	If these are left unknown, then $weight$ and $hratio$ default to
	$1\cdot0$ (medium, normal width), $slant$ and $italicness$
	default to $0$ (upright) and $encoding$ defaults to $0$ (the
	standard \TeX\ encoding).

\subsec{Creating a font}

	The procedure for creating Malvern fonts is not much different
	from normal \MF\ fonts.  Normally the \MF\ program
	is run with a `first line' (after the `|**|' prompt) along the
	lines of
\begin example
	|\mode=...; font_size 10pt#; weight=1.4; input ma; bye|
\end example
	This produces files called something like `|ma.tfm|' and
	`|ma.200gf|'.  These generic file names must be changed to
	`|ma65s10.tfm|' and `|ma65s10.200gf|', say, and the |gf| file
	might need to be converted into a different format used by the
	printer driver (for example, |GFtoPK| might be used to create a
	file `|ma65s10.200pk|').

\subsubsec{Automating this process with {\tt mff}}

	This is altogether too much bother, especially if, like me, you
	must produce fonts on demand rather than installing a large
	number of them once and for all.  I~use a small UNIX program
	|mff| which does all the above automatically: it deduces the
	values for the parameters $weight$ etc.~from the font name.  The
	command
\begin example
	|mff ma65s10|
\end example
	will do all the above automagically.  (The tables used to parse
	the font name are described in a file |mff.rc| shipped with the
	\MF\ files, and also allow the use of names in Karl
	Berry's font-naming scheme, such as `|fmvd10|'.)
	
	The |mff| is intended to be useful for maintaining any family of
	\MF\ fonts, not just Malvern\allowbreak---I~use it for all my
	fonts.  It should be available from the same source that 
	you obtained Malvern itself (the current version is $2.9$).

\subsubsec{An alternative approach using less-generic driver files}

	It is possible to avoid having to type such long commands to
	\MF, at the expense of having many more small files lying
	about.  The setting of parameters is performed by small driver
	files along these lines:
\begin lines
	|% fmvd.mf -- generic driver for demibold Malvern|
\smallskip	
	|weight := 1.4; % demibold|
	|encoding := -200; % Cork TeX Extended Text -- Latin |
	|input ma|
	|bye|
\end lines
	Then the command line to \MF\ is reduced to
\begin example
	|\mode=localfont; font_size 10pt#; input fmvd|
\end example
	(The files produced by \MF\ will now have names starting
	`|fmvd.|', and will still need to be renamed.)  

	This can be taken one step further, by having driver files for
	every font, thus |fmvd10.mf| would contain just the line
	`|font_size 10pt#; input fmvd|'.  Then the basename of the |.mf|
	file is the same as that which the font is to be installed as.
	(This is less of an advantage than it might at first seem,
	because installing a font normally requires moving it into a
	different file area, and therefore a renaming operation.)

\subsubsec{Automating the process with {\tt MFjob}}

	The program |MFjob| is part of em\TeX, a public-domain \TeX\
	distribution for IBM PS/$1$s and PS/$2$s.  It has the capacity
	to generate use driver files in the style of |fmvd.mf| above to
	create fonts at arbitrary sizes.

%}}} organization
\fi
%{{{  file names
\section{File names}

	\MF\ source files for Malvern have names starting with with
	`|ma|' (for "Malvern").  I have tried to choose names that fit
	in with MS-DOS's lamentable naming conventions.
\begin table \tt#\hfil&&\quad#\hfil\cr
	ma.mf&		Top-level generic driver file\cr
	maaenc.mf&	Malvern A encoding description\cr
	mabenc.mf&	Malvern B encoding description\cr
	macenc.mf&	Malvern C encoding description\cr
	macy.mf&	Malvern Cyrillic\cr
	madenc.mf&	Malvern D encoding description\cr
	maencode.mf&	Malvern encoding vectors\cr
	mafigs.mf&	glyph programs for numerals\cr
	magenc.mf&	Malvern G encoding description\cr
	maglcaps.mf&	glyph programs for Greco-latin capital letters\cr
	magrcaps.mf&	glyph programs for Greek capital letters\cr
	magrlc.mf&	glyph programs for Greek lower case\cr
	makit.mf&	macro definitions\cr
	malc.mf&	glyph programs for lower case\cr
	malcco.mf&	glyph programs for lower case Composite-Only\cr
	malcnc.mf&	glyph programs for lower case Non-Composite\cr
	mamarks.mf&	glyph programs for marks for making composites\cr
	maparams.mf&	computing the ad-hoc paramters\cr
	mapunct.mf&	glyph programs for punctuation\cr
	masenc.mf&	\TeX\ Text encoding description\cr
	masyms.mf&	glyph programs for miscellaneous symbols\cr
\end table
%}}}  file names
\iffalse
%{{{  macro conventions
\section{Conventions for macros that draw letters}

	Definitions for macros that draw letters are written so that
	they may be used to draw that letter in an position---for
	example, so that the program for `c' may be used in drawing
	`\copyright'.

	By convention the header for letter "a" looks like:
\begin display
	|vardef draw_|"a"|@#(expr l, b, r, t, bl)|"other parameters"| =|
\end display
	Where $(l, b)$ is the bottom left and $(r,t)$ the top right
	corner of the cell in which the glyph will be drawn, and $bl$ is
	the $y$-value of the baseline.  All should be whole numbers.
	For a normal letter, they are |(l, -d, r, h, 0)|.  The "other
	parameters" are parameters specific to the letter itself: for
	example, giving the relative distance between certain features
	within the letter.

%}}}  macros conventions
\fi

\iflong 
\input encoding
\fi
% \input multiple
	
%}}} Implementing Malvern
%{{{ About Malvern

\part{About Malvern}

%{{{  design
\section{The design of the Malvern font}

	This section describes some of the design decisions made for
	Malvern---this may not be strictly relevant to most people's use
	of it.  I~will start with general, overall style and then
	continue to more specific details.

	My main motivation for developing Malvern was that I~was that
	I~do not use \TeX\ exclusively for technical documents.
	I~produce  leaflets, booklets and magazines in
	non-scientific fields, where the determinedly old-fashioned
	appearance of Computer Modern would jar terribly.  These were
	the sorts of situations where the simplicity and informality of
	a sanserif font would be best suited.\note{Americans seem to be
	more frail than Europeans when it comes to reading sanserif
	text.  It is true that serif faces---particularly oldstyle and
	transitional faces---are more readable in continuous text, but
	the difference is not so great as to override all other design
	considerations.  (The United States is noted for its addiction
	to Bodoni, which is neither particularly readable nor
	particularly legible, and is ugly besides.)  Perhaps this
	handbook is rather long to be set all in sanserif; I~hope that
	not too many readers will be struck blind as a result, and that
	they will allow that it is reasonable for the handbook to act as
	a showcase for the typeface it describes.}

%{{{   types of sanserif
\subsec{Types of sanserif font}

	One way of sub-dividing sanserif fonts is into the four flavours
	\dfn{grotesque}, \dfn{humanist}, \dfn{geometric} and
	\dfn{neo-grotesque}.  While these are not cut-and-dried
	divisions (there are many faces with a flavour all their own),
	they can be a useful approach to thinking about the design of
	typefaces.

\subsubsec{Grotesque}

	The grotesque styles were created during the Victorian period,
	when poster printers out-did each other in combining as many
	different typestyles as possible (one of many examples of how
	the Victorians, who considered themselves the acme of civility
	and good taste, exhibited terrible taste).  They were so dubbed
	because contemporary designers thought they were horribly ugly.
	Today `grotesque' is simply a label, and has no more inherent
	meaning than `modern' or `dutch' do when applied to serif
	styles.  Computer Modern Sans Serif is derived from Computer
	Modern in the same way that grotesque fonts were derived from
	the modern fonts (variations on Bodoni) of the time, so it might
	be described as a grotesque.

\subsubsec{Humanist}

	The humanist styles emerged from the calligraphy revival of the
	early twentieth century.  They reflected the new emphasis on
	carefully-crafted letter-shapes and proportions embodied in
	Edward Johnston's Foundational Hand.  Gill Sans and Johnston's
	Underground Font---used in all London Regional Transport---are
	examples.

\subsubsec{Geometric}

	The Bauhaus movement in $1930$s Germany gave rise to geometric
	styles, in which the letter-shapes were progressively simplified
	and unified, producing alphabets composed of arcs of circles and
	straight lines, with many letters sharing components.
	Latter-day geometric styles retain the appearance of being
	constructed with ruler and compass, but incorporate subtle
	shadings and distortions that allow for optical illusions and
	distortions due to printing technology.  ITC Avant Garde Gothic
	is an example of a geometric font.

\subsubsec{Neo-Grotesque}

	Finally, neo-grotesques are more refined descendants of the
	original grotesque typefaces.  Neo-grotesque designs emphasize
	legibility and uniformity of design.  Their function is to
	deliver the text of a message to the reader in as unobtrusively
	as possible---they are ideal for signs, where legibility is of
	paramount importance (more important than readability, for
	example).  The British Rail Alphabet---used in airports and the
	like as well as BR itself---and the ubiquitous Helvetica are
	neo-grotesque designs.

%}}}   types of sanserif
%{{{   what flavour is Malvern?
\subsec{What flavour is Malvern?}

	When designing Malvern, I~was aiming for a mixture of geometric
	and humanist features.  I~have always liked the look of humanist
	faces like Johnston's Underground Font.  They are described as
	crude, but I~think that this gives them more character than the
	refined and refined neo-grotesques: Helvetica is a very
	well-designed typeface, but text set in Helvetica looks boring.
	I~wanted a face that was appealing to read in itself.  

	At the same time I~wanted to make a typeface that was simple and
	elegant, without (apparant) variation in stroke width and with
	an emphasis on rounded shapes (I~particularly wanted to have a
	circular `O').  Since this was my first attempt to design a
	typesetter font it seemed sensible to aim for character and
	simplicity rather than sophistication.

%}}}   what flavour is Malvern?
%{{{   global design features
\subsec{Global design features}

	There is one feature of Malvern that I~based on the Johnston
	Underground Font, which is that the bowls and arches of letters
	join the stems at a large angle, rather than curving toward a
	tangent with the stem:
\begin display 
	\font\sf=cmss17 scaled \magstep1\sf bpnu&
	\ifPS \font\sf=phvr at 18pt \sf bpnu& \fi
	\headingrm bpnu \cr
	\noterm CMSS\n{17}&
	\ifPS \noterm Helvetica& \fi
	\noterm Malvern \n{65}\cr
\end display
	As well as forming a nice, uncluttered shape, this avoids having
	to narrow the width of the stroke to avoid dark spots.  (A
	fortuitous side-effect is that these joins have a good chance of
	digitizing well at low resolutions.)

	Malvern has rounded terminals---that is, the ends of the strokes\marginchar{nb}
	are rounded rather than squared off.  Together with the lack of
	tapering at curve joins, this means that Malvern looks as if it
	were drawn with a circular pen.  All this makes Malvern rather
	easier to implement than most typefaces would be, because \MF\
	has built-in operators to produce just such an effect.  This was
	not the  reason for choosing to make Malvern look this way,
	although it did affect my decision to make a stab at producing
	Malvern rather than some other typeface.

%}}}   global design features
%{{{   currency sign

\subsec{The ISO currency symbol}

	The symbol `\currency' is the \ISO{646} currency symbol.  It
	takes the place of the dollar sign in position \n{36} of the
	\ISO{646} character encoding (which is the international
	standard equivalent of ASCII).  Thus, in theory, non-Americans
	should all use \currency\ instead of \${}, and it should stand
	for the local currency.  (It has been used in some Scandinavian
	countries at least: I~remember a book on \mc{BASIC} which had
	\mc{A\currency\ = CHR\currency}($13$) instead of \mc{A\${} =
	CHR\${}}($13$)).

	Most fonts I~have encountered which have a `\currency'
	character\marginchar{\currency} draw it as a small circle with
	the four `ears' at right angles to each other: a small, square,
	symbol that floats above the baseline like a binary operator.
	My design is unusual, but I~prefer it, because is is the same
	approximate size as other currency symbols (and looks like it
	belongs in the font).

\xreflabel{s-currency}{\S\thesecno}

%}}}  currency sign
%{{{   odd symbols

\subsec{Sundry oddly shaped symbols}

	With some symbols, like the the per-cent sign, I have
	deliberately ignored their origins, so `\%' does not much
	resemble `$0/0$' or `\frac0/0', `\S' does not look much like two
	`s's stacked on top of each other, and `\pounds' does not look
	much like a script letter "L".  This is partly in recognition of
	the fact that they have long since been used as symbols in and
	of themselves, and the link to their origins is weak.

	So `\%' has similar proportions to other currency-like
	signs\marginchar{\%\S{}} (\${}, \yen, \P, and the like), while
	still having two generously-sized, circular rings.  (My first
	reaction to CM Typewriter was `How can I change that per cent
	sign?')  The crooked design for the section sign `\S{}' was
	partly an attempt to ensure that the lines met at large angles,
	to avoid blots without making it a taller character than the
	dollar sign.

%}}}  odd symbols
%}}}  design
%}}} About Malvern
\fi
%{{{ Appendix
\part{Appendix}

%{{{  encoding

\section{The encoding of Malvern fonts}


	An \dfn{encoding} for a computer font is a mapping from a
	\dfn{character code} (a nonnegative integer less than
	$256$) to characters in the font.  (Encodings are like the `code
	pages' defined by ISO standards such as $646$ and $8859$.)
	Unlike \PS, any given \TeX/\MF\ font has its encoding hard-wired
	into it, because characters are always referred to by code.

	This section discusses some general aspects of \TeX\ font
	encodings, and some `standard' font encodings, before describing
	the Malvern font encodings.  The encoding used is reflected in
	the external name of the font (see \xref{s-font-names}).

\iflong
%{{{   in put != output

\subsec{Input and output encodings need not be the same}

\xreflabel{s-input-output}{\S\thesecno}
	For the purposes of this handbook, we shall only need to discuss
	the encodings of fonts produced with \MF\ to be used to typeset
	pages produced by \TeX.  The encoding used to interpret the
	characters used as input to \TeX\ is usually different, because
	it is usually dictated by what symbols the computer keyboard has
	printed on its keys and how they are displayed on the screen,
	whereas the encodings of fonts referred to by \mc{DVI} files
	suffer no such restriction.  

	In fact, the encoding used for \TeX's output fonts need have *no
	connection whatever* to the ASCII character set.  This is
	important, because good typesetting requires glyphs like `fi',
	`\thinspace``\thinspace', `\S{}' and the like that do not exist
	in the ASCII\ or \ISO{8859} characters sets (instead they have
	characters like `{\rmb\char34 }' and `{\rmb\char94 }' which are
	not of use in fine printing, except when simulating a computer
	or typewriter).

	In practice, the (assumed) input encoding and the encoding of
	latin text fonts will have a common sub-encoding, because it is
	so much easier if we arrange that the code of common characters
	is the same as the code of the corresponding character in \TeX's
	internal encoding.  Thus `A', say, has encoding $65$ in text
	fonts, matching the character `|A|' in position $65$ of ASCII.

%}}}  input != output
%{{{   tex text

\subsec{The \TeX\ Text encoding}

	The encoding used for the Computer Modern text fonts is simply
	called `\TeX\ Text' \cite{MFApp.F}, which has two
	variations, `\TeX\ Typewriter Text' and `\TeX\ Text without
	f-ligatures' (if changing the dollar sign into a pounds-sterling
	sign in \mc{CM} Text-Italic is not counted as a different
	encoding).  The first two of these are displayed in Appendix~F of
	the {\it\TeX book}.  

	Knuth stresses that other \TeX\ fonts might have different
	conventions (addressed with different macros), so perhaps the
	use of `\TeX' rather than `Computer Modern' in those names
	should be interpreted as a historical artefact from the days
	when \TeX\ and Computer Modern were almost synonymous.  (Calling
	them `CM Text', `CM Typewriter Text' etc.~would not have
	prevented non-CM fonts from using the CM conventions, of
	course.)

	The \TeX\ Text encodings were devised when many programs in the
	\TeX/\MF\ system limited fonts to $128$ characters, and so it
	only uses the first $128$ codes.  As well as the capital and
	lowercase alphabets, ranging figures and usual punctuation and
	symbols, it includes the ligatures used in Englsh-language
	typesetting ("fi", "ffl", etc.), marks to make composite letters
	with, and the uppercase ^{Greek} letters that do not have
	similar Latin letters: {\gr G}\negthinspace, {\gr D},
	\dots\thinspace, {\gr W}.

	It lacks the Polish ogonek mark (\showmark\ogonek\enspace) used
	to make "\A", "\E", "\a" and "\e";\note{And Lithuanian "\ogonek
	i" and "\ogonek u" \cite{Pei}?} guillemets (the quotation marks
	<< and >>); pounds-sterling and a few other currency-sign-like
	symbols (like "\yen" and "\P{}").

	
	Malvern fonts named using \index{Berry, Karl} Karl Berry's
	standard short font names (|fmv|\dots) use this encoding if no
	other encoding is specified.

\medskip

	{\rms \ntable}
\medskip
\noindent
	Notes:
\smallskip
\halign
{#\hfil\quad&\hfil#&\quad\hfil#&\quad#\hfil\cr
	\oct{000}--\oct{012}&\hex{00}--\hex{0A}&$0$--$10$&
		non-latin Greek letters\cr
	\oct{022}--\oct{030}&\hex{12}--\hex{18}&$18$--$24$&
		marks for composite letters\cr
	\oct{040}&\hex{20}&$32$&
		slash that makes "\l" and "\L"\cr
}

%}}}  tex text
%{{{   Cork

\subsec{The Cork encoding}

	A new proposed standard encoding is described by Michael
	Ferguson \cite{Cork}.  I shall refer to this as the Cork
	encoding, because it was devised at the \TeX\ users conference
	held in Cork in September~$1990$. I~think that it is intended to
	be used as \TeX's internal encoding as well as for `standard'
	text fonts.\note{\TeX\ converts documents from the local
	character set to its own as the first stage of the parsing
	process---normally the local character set is ASCII, the same as
	\TeX, so this stage is transparent.  The advantage of extending
	\TeX's internal encoding to include composite letters, assuming
	\TeX\ is configured properly, would be that the code for
	`{\tt\accent'23 e}' on a user's keyboard would be transformed
	into the code for `\'e' in the Cork encoding.  The disadvantage
	of using the same encoding for \TeX's internal representation of
	the input file and in the DVI file is that it requires the \MF\
	fonts to match ASCII for codes $33$--$126$ (|'|, |-| and |`| are
	each used in ASCII to stand for several different symbols, so
	the match can't be exact).  It also requires the input encoding
	to include characters like `ffl', which are of little or no
	utility in a computer character set.  I feel it would make more
	sense to have two encodings.} The Cork encoding includes a large
	number of composite letters (all those of \ISO{8859/1} and
	\ISO{8859/2}), so that most European languages can be typeset
	without using the |\accent| primitive (Esperanto and some
	European languages are not covered; and it lacks the dotted
	consonants used when setting Irish in the traditional Celtic
	alphabet, and dotted/underlined letters used in transliterating
	Arabic).

	It also includes a duplicate hyphen, and the \dfn{compound word
	mark} (cwm), an invisible character.  The point of cwm is to
	allow the breaking of ligatures; in a language like German where
	the `fl' ligature is sometimes used but is incorrect if the `f'
	ends a syllable, this would be written as |f^^Wl| (or a control
	sequence expanding to this).

	The duplicate hyphen is intended as a kludge to cope with the
	fact that \TeX\ will not break words containing hyphens except
	at the hyphens, even if those hyphens are really part of a dash
	ligature.  The relevant paragraph of the article\note{`It
	includes both a ``dash'' and an explicit ``hyphen char''.  This
	capability allows font designers the option of replacing the
	``-'' with an ``='' without losing the dash.  Since the ``-'' is
	no longer the hyphen char, it allows words with explicit dashes,
	such as INRS-T\'el\'ecommunications[,] to be hyphenated.'} is
	unclear because Ferguson uses `dash' without specifying whether
	he includes hyphens as dashes.  My guess is that the hyphen and
	its duplicate are expected to both look the same, and that they
	intend that \TeX\ normally be set up to hyphenate with character
	$45$ as usual.  People who want to treat hyphens as a normal
	letter character will set |\hyphenchar| to the duplicate hyphen.

	A discouraging point in the article is that the inclusion of the
	typewriter-style ugly quotation mark character is described as a
	desirable feature of the proposed standard:
\begin quotation
	This \dots\ removes an irritant in the use of [double quote
	ligatures] in these fonts.  The font designer can decide whether
	[``], [''], and [{\rmb\char34 }] are distinguishable.
\end quotation
	Presumably the aim here is to make it as easy for \TeX ers to
	produce lousy typesetting as it is for people using Macintoshes
	and \MSDOS\ machines.  I can accept that if it is assumed that
	the same encoding must be used for \TeX's internal encoding as
	for DVI output, then the ugly-quote character is unavoidable;
	but it pains me to see it presented primarily as a boon for lazy
	typists.  One of the worst effects of the `DTP revolution' has
	been the way in which typewriter-style quotation marks and
	apostrophes have become common in typeset text, because they are
	so much easier to generate on a Macintosh than the correct
	quotation marks.

%	The article also states
%\begin quotation
%	The standard also includes |<| |>| in the normal ASCII
%	location[s].
%\end quotation
%	but this does not address the fact that the ASCII symbols `|<|'
%	and `|>|' are used both as the less-than and greater-than
%	relations {\rmb<} and {\rmb>} and as angle brackets \langle\ and
%	\rangle.  These keys might also reasonably be used for single
%	guillemets < and > or normal (double) guillemets << and >>.  


	Future versions of \LaTeX\ will expect fonts using this encoding
	to exist.  In the Malvern~\n{1.0} documentation I~suggested that
	\index{Berry, Karl} Karl Berry's names be used for these; there
	is now a specal code letter `|q|' for the Cork encoding.
\medskip
	{\rmx \ntable}
\medskip
\noindent
	Notes:
\smallskip
\halign
{#\hfil\quad&\hfil#&\quad\hfil#&\quad#\hfil\cr
	\oct{000}--\oct{014}&\hex{00}--\hex{0C}&$0$--$12$&
		marks for composite letters\cr
	\oct{015}&\hex{0D}&$13$&
		German single quotation mark\cr
	\oct{016}--\oct{017}&\hex{0E}--\hex{0F}&$14$--$15$&
		single guillemets\cr
	\oct{027}&\hex{17}&$23$&
		compound word mark (cwm)\cr
	\oct{030}&\hex{18}&$24$&
		ring that combines with \% to make \permille\cr
	\oct{040}&\hex{20}&$32$&
		visible space\cr
	\oct{042}&\hex{22}&$34$&
		typewriter-style neutral quotation mark\cr
	\oct{136}&\hex{5E}&$94$&
		ASCII circumflex\cr
	\oct{137}&\hex{5F}&$95$&
		ASCII under{\rmb\char95 }score\cr
	\oct{176}&\hex{7E}&$126$&
		ASCII tilde/swung dash\cr
	\oct{177}&\hex{7F}&$127$&
		duplicate hyphen\cr
}



%}}}  Cork
%{{{   bag-of-characters

\subsec{The `bag-of-characters' alternative}

	The Cork encoding has as a central assumption that it is
	possible and overridingly desirable to have *one* encoding that
	is used in typesetting anything, and that this encoding shall be
	used for input to \TeX, fonts used in DVI files, and as the
	output of \MF.  I~am of the opinion that it is not useful or
	convenient to try to impose One True Encoding on all text fonts.

	In fact, the Cork article does have as an unstated assumption
	that different fonts are used for different languages (or rather
	groups of languages): for example, the discussion of the use of
	the cwm to separate ligatures in German but to join ligatures in
	other contexts suggests a plethora of fonts with the identical
	encoding and glyphs but different ligtables.

	I~would argue that if separate fonts are being used then they
	might as well have different encodings as well.  This would
	allow a German font to omit, say, "ffl" and "ffi" (not used in
	German) and perhaps add "ck", "tz", "ch" ligatures.  True {\it
	Fraktur} faces require more ligatures, but do not include
	composite letters other than umlauts.  Guillemets for French use
	should be created with extra space inside them <<\thinspace like
	so\thinspace>>, whereas Germanic fonts should have guillemets
	without the extra space >>like so<< \cite{Hart's102}.  Allowing
	different encodings would put Esperanto, Irish, Lappish, Welsh
	etc.\ on equal footing with the larger languages.  Assuming that
	switching between languages already implies a switch in
	hyphenation tables and other parameters of \TeX, switching fonts
	as well seems like a minor problem, especially as it allows
	great benefits.

	Furthermore, the best way within the \TeX\ system to produce
	several different fonts with almost the same collection of
	glyphs is to use composite fonts.  A composite font can draw its
	characters from one or more base fonts.  These base fonts are in
	effect a bag of characters from which different selections can
	be drawn to form language-group-specific fonts.  There is no
	reason for the base font(s) to have the same encoding as the
	tailored fonts used by \TeX\ itself.

%}}}  bag-of-characters
\fi
%{{{   Malvern

\subsec{The Malvern font-encoding conventions}

	The Malvern typeface includes more than $256$ characters, so it
	follows that no one font will contain every Malvern character.
	Instead I have coding schemes called Malvern~A, Malvern~B,
	Malvern~C and Malvern~G.\note{The use of `Malvern' in these
	names is in order give them a unique name, not to restrict the
	use of these encodings to Malvern fonts.  If the encoding
	conventions of later versions of Malvern are different for some
	reason, then the version number can be appended to distinguish
	the different encodings, giving `Malvern A $1.0$', say.}
	Malvern~A is the simple latin text font; it has the base
	alphabets (that is, it has "A--Z" and the rest but not "\`A--\v
	Z").  Malvern~C and G are Cyrillic and Greek respectively
	(Malvern~C is stil incomplete).  Malvern~B is the latin text
	supplement.  (There might also be a Malvern~E of composite
	letter glyphs in the future.)

\xreflabel{tab-charcodes}{\S\thesecno}

%}}}  Malvern
%{{{   macros for table notes

%  Much to my annoyance, the smart play is to give the codes in octal
%  since TeX can convert to hex and decimal easily.

\def\hexnumber#1%
{%
    \hex{\chardef\tmp=#1 \expandafter\striptodoublequote\meaning\tmp}%
}
\begingroup\catcode`\"=12 \toks0={\endgroup
	\def\striptodoublequote#1"{}
}\the\toks0
\def\decnumber#1%
{%
    \n{\number#1}%
}


\def\noteone#1#2%
{%
	\oct{#1}&\hexnumber{'#1}&\decnumber{'#1}&#2\cr
}
\def\notetwo#1#2#3%
{%
	\oct{#1}, \oct{#2}&
	\hexnumber{'#1}, \hexnumber{'#2}&
	\decnumber{'#1}, \decnumber{'#2}&#3\cr
}
\def\noterange#1#2#3%
}}   macros for table notes
%{{{   Malvern A
%{{{   insert table

\pageinsert	% this one is too big for a normal insert
\ntable
\medskip
\noindent
	Notes:
\smallskip
\halign
{#\hfil\quad&#\hfil&\quad#\hfil&\quad#\hfil\cr
	\oct{006}--\oct{017}&\hex{086}--\hex{0F}&$6$--$15$&
		marks for l.c. composites\cr
	\oct{026}--\oct{027}&\hex{16}--\hex{17}&$22$--$23$&
		marks for U\&lc composites\cr
	\oct{060}--\oct{071}&\hex{30}--\hex{39}&$48$--$57$&
		oldstyle digits\cr
	\oct{074}, \oct{076}&\hex{3C}, \hex{3E}&$60$, $62$&
		single guillemets\cr
	\oct{206}--\oct{217}&\hex{86}--\hex{8F}&$134$--$143$&
		marks for cap.\ composites\cr
	\oct{234}&\hex{9C}&$156$& lower-case "d" with hook (\,=\,"\v d")\cr
	\oct{235}&\hex{9D}&$157$& lower-case "t" with hook (\,=\,"\v t")\cr
	\oct{236}&\hex{9E}&$158$& lower-case "h" with circumflex\cr
	\oct{237}&\hex{9F}&$159$& lower-case "L" with hook (\,=\,"\v l"?)\cr
	\oct{240}&\hex{A0}&$160$& ballot box\cr
	\oct{244}&\hex{A4}&$164$& ISO-$646$ currency sign\cr
	\oct{251}&\hex{A9}&$169$& `florin sign' (variant italic "f")\cr
	\oct{256}&\hex{AE}&$174$& raised dot (British decimal point)\cr
	\oct{260}--\oct{271}&\hex{B0}--\hex{B9}&$176$--$185$&
		ranging figures\cr
	\oct{300}&\hex{C0}&$192$& degrees sign\cr
	\oct{333}, \oct{335}&\hex{DB}, \hex{DD}&$219$, $221$&
		angle brackets\cr
}
\caption{Malvern A encoding.}
\xreflabel{tab-charcodes}{Table~\n{\the\tablecount}}
\endinsert
%}}}   insert table
\subsec{Malvern A Encoding}

\xreflabel{tab-charcodes-a}{\S\thesecno}
	This is the latin text font, containing alphabets, two sets of
	figures, punctuation marks, various symbols, and some special
	letters.  Although the eventual plan is that language-group
	specific fonts be created, this encoding has been designed so
	that it can be used on its own at a pinch.  As with the \TeX\
	Text encoding, words with composite letters in them will not
	hyphenate properly, and any composite-letter characters
	available on the user's keyboard will have to be made to expand
	to control sequences to generate that character.

	As well as the usual (large) capitals and lower case alphabets,
	Malvern has two more capital alphabets: small capitals and
	medium capitals.  The font is divided into two halves, with the
	first $128$ slots being an extension of a subset of normal
	ASCII, and the upper $128$ characters largely `shadowing' the
	lower half.  For example, small capitals and medium capitals
	`shadow' the lower case and large capital alphabets (`a' is
	character \hex{61}, and `\sc{A}' (small cap.)\ is $\hex{61} +
	\hex{80} = \hex{E1}$).

	There are two sets of digits: old style (\lowercase{1234567890})
	are the default; to get ranging digits (\uppercase{1234567890})
	see \xref{s-ranging}.  The usual block of marks (which may be
	used to make composite letters) is designed to suit the lower
	case letters, and the shadow set is designed to suit the medium
	capitals\iflong (see \xref{sec-type-A})\fi.



%}}}   malvern A
%{{{   Malvern B
%{{{    insert table

\topinsert
	{\rmb \ntable}
\medskip
\noindent
	Notes:
\smallskip
\halign
{#\hfil\quad&\hfil#&\quad\hfil#&\quad#\hfil\cr
    \noteone{40}{blank space}
    \noteone{42}{seconds (and ASCII doublequote)}
    \noteone{45}{Cork's ring for per mille}
    \noteone{47}{minutes (and ASCII quote)}
    \noteone{54}{German single quote}
    \noteone{56}{three-dot ellipsis\dag}
    \noteone{57}{slash for fractions}
    \noterange{60}{72}{infoerior figures}
    \noterange{60}{72}{inferior figures}
    \notetwo{74}{76}{greater- \& less-than signs}
    \noterange{101}{105}{Old English caps\dag}
    \noterange{106}{145}{Old English lower case\dag}
    \noterange{146}{153}{Ligatures "ch", "ck", "ct", "ft", "ij", "ll"\dag}
    \noterange{154}{163}{Long "s" and ligatures}
    \noteone{164}{Schwa}
    \noterange{165}{173}{Font-specific variant letters}
    \noterange{134}{137}{Cork glyphs}
    \noteone{140}{reverse tick (ASCII backquote)}
    \noteone{240}{Cork visible space}
    \noterange{241}{246}{ISO \n{8859}/\n{1} glyphs}
}
\caption{Malvern B encoding.}
\endinsert
%}}}   inster table
\subsec{Malvern B Encoding}

\xreflabel{tab-charcodes-b}{\S\thesecno}
	This font is essentially a collection of random symbols and
	alphabetical characters, some of them mathematical.  I~have not
	yet settled on a firm design for it---most of it is left as gray
	areas to be filled in in later versions of Malvern.  Also, even
	in the final form of this font, some areas will be designated as
	`font-specific'; other fonts using the Malvern conventions will
	use them for whatever variant letters and unique characters the
	designer wishes to include.
	Superior and inferior---that is, raised and lowered---figures
	(\lowercase{{\rmb 1234567890}}/\uppercase{{\rmb 123456789}})
	take the place of the old-style and ranging figures.  The glyphs
	required by the Cork encoding, such as this visible space
	`{\rmb\char160 }' and ASCII doublequote `{\rmb\char34 }', are
	also included here.


%}}}   malvern B
%{{{   Malvern C
	
\subsec{Malvern C Encoding}

\xreflabel{tab-charcodes-c}{\S\thesecno}
	This will be the ^{Cyrillic} subset of Malvern, using the same
	or simlar conventions to the existing ^|wncy| fonts.  As yet,
	several of these glyphs are missing.  (Perversely enough, mainly
	the ones appearing in the Latin and Greek alphabets, so I can
	only say `AH{\gr G}{\cy LIQ}AH{\cy I}H' by borrowing `A' and `H'
	from Malvern~A and `{\gr G}' from Malvern~G!)  I'm not really in
	a good position to produce a good Cyrillic font---I don't have
	the slightest knowledge of Russian or any other language using
	the Cyrillic alphabet.

\topinsert
	{\rmc \ntable}
\caption{Malvern~C encoding (so far).}
\endinsert

%}}}   Malvern C
%{{{   Malvern G
	
\subsec{Malvern G encoding}

\xreflabel{tab-charcodes-g}{\S\thesecno}
	This is the ^{Greek} text font encoding, based largely on the
	Greek\TeX\ \cite{KD} fonts, so that, for example, `|<'a|||'
	produces an alpha with breathing, accent and iota subscript:
	`{\gr <'a\char124 }'.  The document |magrman.tex| in the Malvern
	distribution describes the use of this font.

\topinsert
	{\gr \ntable}
\caption{Malvern G encoding.}
\endinsert

%}}}   Malvern G
%}}}  encoding
\dosupereject
\iflong
%{{{   type sizes
\vfil
\penalty-100
\vfilneg
\section{Sizes of type}

	This is a table of sizes of type, descibed both in \TeX's
	Anglo-American points ($0\cdot351\,{\rm mm}$ or $0\cdot01383''$)
	and in millimetres, and also the cap heights for Malvern~55.
	The choice of point sizes is a mixture of traditional sizes
	($10\pt$, $12\pt$, \dots, $48\pt$, $60\pt$ etc.) and the sizes
	Computer Modern fonts are supplied in ($10\pt$, $12\pt$,
	$17\cdot28\pt$ etc.), and a few sizes that are round figures in
	metric units ($10\mm$ etc.).
\medskip
\moveleft\leftmargin\vbox{\advance\hsize\leftmargin
\halign to \hsize{\hfil $#$\tabskip=0pt& $#$\hfil \tabskip=0pt plus 1fil&
    &\quad\hfil $#$\tabskip=0pt& $#$\hfil \tabskip=0pt plus 1fil
\cr
	\noalign{\hrule height 1pt \vskip 1.5\jot}%
		\multispan4Body size\hfil &
		\multispan4\quad Cap.\ ht\hfil &
		\multispan8\quad Pixels per em\hfil\cr
	\noalign{\vskip\jot\nointerlineskip}%
		\multispan4\hrulefill &
		\multispan4\quad\hrulefill &
		\multispan{12}\quad\hrulefill \cr
		\multispan2pt\hfil &
		\multispan2\quad mm\hfil &
		\multispan2\quad pt\hfil &
		\multispan2\quad mm\hfil &
		\multispan2\quad $120$ &
		\multispan2\quad $200$ &
		\multispan2\quad $300$ &
		\multispan2\quad $400$ &
		\multispan2\quad $600$ &
		\multispan2\quad $1000$\cr
	\noalign{\vskip\jot \hrule \vskip 1.5\jot}%
	\input typesizes
	\noalign{\vskip2\jot \hrule height 1pt}
}}
\smallskip

%}}}   type sizes
\fi
%{{{   composite letters

\section{Composite letters used in some languages}

	This table lists composite letters listed as being
	required to typeset several European languages.  This list is
	not particularly canonical\allowbreak---I~have merely combined
	the OUP list \cite{ODWEacc}, the resources of my local library,
	and \TUGboat\ articles \lcite{\\{Bien}\\{Hara}}.  There are
	probably omissions, especially of composites used in
	loan-words\allowbreak---and I~would appreciate any information
	which I~can include in later revisions.

%{{{    macros to do footnotes in table

\newcount\tabcount 
%\tabcount=7 \multiply\tabcount16 \advance\tabcount\itfam 
%\multiply\tabcount256 \advance\tabcount`a \advance\tabcount-1

%%  Normally footnotes into a table are lowercase letters
%%  In this case it seemed better to use numbers given that the
%%  entries are all letters...
%%  
\def\tabnote#1%
{%
    \global\advance\tabcount1 
    \global\edef#1{\footnotetextmark{\the\tabcount}}%
    #1%
}
\def\tabnotetext#1%
{%
%    \ifvmode
%	\noindentfalse
%	\leavevmode\llap{#1}%
%    \else 
%	\unskip \penalty10000 \hbox to \parindent{\hfil#1}%
%    \fi
    \par \indent\llap{#1\enspace}%
    \ignorespaces
}

%}}}
%{{{    the table

\medskip
\moveleft\leftmargin\vbox\bgroup
\advance\hsize\leftmargin
\hrule height 1pt
\vskip2\jot
\halign to \hsize\bgroup #\unskip\hfil \tabskip=3pt plus 10pt&&#\unskip\hfil\cr
Czech		& \'a	& \v c \v D\vd\tabnote\czech
				&\'e\v e&	& \'\i	& \v n	& \'o
	& \v r\v s \v T\vt\czech
										& \'u\ringmark u 
											&	& \'y	& \v z	\cr
Danish		&\aa &	&	&	&	&	& \o	&	&	&	&	&	& \ae	\cr
Dutch\tabnote\dutch &	&	&\'e\"e	&	&	&	&\'o\"o	\cr
English		&\aa\tabnote\angstrom
			& \c c	&\'e\`e\tabnote\egravenote\"e 
					&	&\"\i
							&	&\^o\"o\tabnote\odiaeresis &&&	&	&	& \ae\oe\tabnote\aelig	\cr
Esperanto 	& 	& \^c	&	& \^g\h	&	& \^\j	&	& \^s	& \u u	\cr
Finnish		&\aa\"a
			&	&	&	&	&	& \"o	\cr
French%\tabnote\frenchnote
		&\`a\^a	& \c c	& \'e\`e\"e &	&\^\i\"\i
							&	& \^o	&	& \`u\^u\"u	\cr
German		& \"a	&	&	&	&	&	& \"o	&	& \"u 	&	&	&	& \ss%\tabnote\eszetnote 
\cr
Hungarian	& \'a	&	& \'e	&	&\'\i &	&\'o\"o\H o &	& \'u\"u\H u \cr
Icelandic	& \'a	&	& \'e	&	&\'\i &	&	&	&	&	&	&	& \th\dh \cr
Norwegian	&\aa
			&	&	&	&	&	& \o	&	&	&	&	&	& \ae \cr
Polish		& \a	& \'c	& \e	&	&	& \l\'n	& \'o	& \'s	&	&	&	& \'z\.z \cr
Portuguese	& \'a\`a\^a\~a 
			& \c c	& \'e\`e\^e &	&\'\i\`\i
							&	& \'o\`o\^o\~o && \'u\`u\"u\tabnote\portuguese \cr
Romanian	& \`a\^a\u a &	& \`e	&	&\`\i\^\i &&	&\c{s}\c{t}
										& \`u \cr
Russian\tabnote\russian &&	&\`e\"e	&	&\u{\i} & 	&
	&	&	&	& \=y	& \v z 	& \rmb '\thinspace \char34  \cr
Spanish%\tabnote\spanish
		& \'a	& 	& \'e	&	&\'\i 
							& \~n	& \'o	&	& \'u\"u \cr
Swedish		&\aa\"a &&	&	&	&	& \"o	\cr
Turkish		& \^a	& \c c	&	&\u{g}
						&\.I\i\tabnote\turkish \^\i
							&	& \"o	&\c{s}
										& \"u \cr
Welsh%\tabnote\welsh
		&\^a\'a	&	&\^e\"e	&	&\^\i\"\i
							&	& \^o\"o &	&\^u	& \^w	& \^y \cr
\crcr \egroup
\vskip1\jot
\hrule height 1pt
\egroup
\smallskip

%}}}
%{{{    text of footnotes

\tabnotetext\czech
	The letters "\vd" and "\vt" are used as alternatives to placing
	a \showmark\v on letters with ascenders.  They can also be
	written \xyaccent{0.4 }{\dimen0=0pt }{13}{d} and \v{t}.
\tabnotetext\dutch
	"IJ~ij" counts as one letter.
\tabnotetext\angstrom
	"\aa ngstr\"om".
\tabnotetext\egravenote
	In poetry, "-\`ed" is us\`ed to show a syllable normally mute is
	to be separately pronounced \cite{Hart's30}.
\tabnotetext\odiaeresis 
	American "co\"operate" (Br. spelling is "co-operate"), "\aa ngstr\"om".
\tabnotetext\aelig
	Using the vowel ligatures "\ae" and "\oe" in English words
	with Latin roots is a Victorian affectation---it is usual
	nowadays to write "aesthetic", not "\ae sthetic"
	\cite{Hart's62}.  Americans write "esthetic", of
	course.
\tabnotetext\portuguese
	The letter "\"u" is used in Brazil but not in Portugal
	\cite{Hart's135}.
\tabnotetext\russian
	These are letters used in transliterating Russian and the other
	languages that use the Cyrillic alphabet (Belorussian etc.)
	\cite{Hart's120}.
\tabnotetext\turkish
	In Turkish, "I~\i" and "\.I~i" are separate letters.  Sometimes
	"\ringmark \i" or "\=\i" is used in manuscripts to indicate~"\i"
	\cite{Hart's135}.

%\tabnotetext\spanish
%	The digraphs "ch" and "ll" count as single letters.
%\tabnotetext\welsh
%	Has $7$~digraphs: "ch", "dd", "ff", "ng" (ranked after "g"),
%	"ll", "ph", "rh" (except after vowels), "th".
%\tabnotetext\eszetnote
%	"Es-zet", or `sharp "s"', originating as a {\it Fraktur\/}
%	ligature for long "s" and "z" ("\char157 3"), but treated as
%	"ss" in alphabetizing, and the upper-case equivalent is "SS".
%\tabnotetext\ocircumflex 
%	This is inconsistently used for words naturalized from French --
%	"role" (no circumflex) but "m\^el\'ee" \cite{ODWE}.

%}}}
%}}}   composite letters
\iflong
%{{{  list maamac.tex
\section{Listing of {\tt malvern.tex}}

	This is a complete listing of |maamac.tex|, a file of \TeX\
	definitions that redefine the usual accent commands etc.\ and
	make some other definitions appropriate to the Malvern encoding.
	It might be used by plain~\TeX

\everylisting{\smallfonts}
\listing{maamac.tex}
%}}}  malvern.tex
\fi
%{{{  installation hints

\section{Installation summary}

\subsec{Getting Malvern via FTP}

	Malvern should be available from CTAN (the
	Comprehensive \TeX\ Archive Network), in directory
	|fonts/malvern|.

	Where appropriate, Malvern now uses the same naming convention
	as GNU software, so there should be a bundled distribution
	called something like
\begin display
	{\tt\ttpackage.tar.gz}
\end display
	The file, as the suffix suggests, is a |tar| archive compressed
	with GNU Zip, and on UNIX systems will likely be unpacked with
	something like
\begin display
	{\tt zcat \ttpackage.tar.gz \char`\| tar xvf -}
\end display

	A version which has been `patched' will have a version ID like
	`$1.2.05$', the `$05$' being the \dfn{patch level}.  You apply
	the patches (with ^^{Wall, Larry} Larry Wall's ^|patch|
	program) in order (patch $01$ first, then $02$, and so on) to
	the main distribution.  Patch files have names like
	`|malvern-1.2-patch05|'.

	There is a document |dvi/install.dvi| included in teh package
	that contains installation hints.

\subsec{Generic driver file}
	Malvern is unusual amongst \MF\ families in that it is supplied
	with one {\it generic driver file} instead of one driver file
	for each size and style the family comes in.\note{The package
	also includes a selection of driver files.} The generic driver
	file ({\tt ma.mf}) inspects the values of variables like
	"designsize" and "weight" and gives the ad-hoc parameters
	appropriate values, before reading the program files.  On some
	UNIX systems a small program called ^|mff| can be used to
	generate fonts using this system.  Otherwise other measures will
	have to be taken, described below.

\subsec{Malvern encodings}
	Malvern uses a nonstandard encoding---in fact a family of
	encodings, for example, Malvern~A (latin alphabets) and
	Malvern~G (Greek).  To create fonts for use with \LaTeXe\ and
	the NFSS\vn2 font selection macros, use ^^{Jeffrey, Alan} Alan
	Jeffrey's ^|fontinst| package \cite{fontinst} (see also
	\xref{s-fontinst}).  This generates the |fd| files and virtual
	fonts so that Malvern fonts may be used in the same way as other
	\LaTeXe\ families.  (The glyphs needed to make fonts with the
	Cork (T$1$) encoding are in Malvern~A and Malvern~B.)  This has
	already been done for you to make a selection of styles
	available with both \TeX\ Text and T\n1 encodings.

\subsec{Malvern font names}
	The Malvern fonts generated with \MF\ will have \TeX\ names of
	the following form:
$$
	\hbox{{\tt ma}\<style>\<encoding>\<size>}
$$
 	where the \<style> is a two digit style code (described below),
	\<encoding> is a code identifying a Malvern encoding (such as
	`{\tt s}' or `{\tt az}'), and \<size> is the size in points
	(e.g., `{\tt 12}' for $12\pt$). Thus {\tt ma55a12} ($12$-pt
	Malvern $55$, encoding~A).

	The two-digit style codes are based on Adrian Frutiger's system,
	devised for the Univers family; see \xref{s-stylecodes}.

	The \MF\ programs can also produce fonts in ^^{Berry, Karl}
	Karl Berry's font naming scheme, with the \TeX\ Text encoding.
	The names start with `{\tt fmv}'.  This is so that these fonts
	may be used in plain-\TeX\ and \LaTeX-$2.09$ documents without
	too much confusion.  The correspondance between Malvern's style
	digits and weight, variant and expansion letters is given in the
	tables below.  NFSS\vn2 systems (indeed, any that use
	non-\TeX-text encodings) will have to use virtual fonts.

\ifkbnames\else
\begin notepar
	`\/{\tt fmv}\/-' names are not used at present; use
	corresponding `\/|ma|\/-' names instead.  See the note at the
	end of \xref{s-kb-names}.
\end notepar
\fi	

\subsec{Unpacking Malvern}
	The Malvern distribution includes \MF\ source files, some \TeX\
	files such as this documentation, and a few other miscellaneous
	files.  

\subsubsec{Source files ({\tt/source})}
	On most \TeX\ systems, \MF\ fonts end up with their source files
	stored in \MF's input file area.  I~suggest that Malvern instead
	be given its own file area.  This avoids problems with different
	font families having files with the same name, and makes it easy
	to replace all the Malvern files in one go if you upgrade to a
	newer version of Malvern.

	With a new-style directory tree,\note{At the time of writing,
	the \TeX\ Directory Standard (TeDiouS) is still in discussion.
	This is my best guess based on a \TeX\ system I~recently had
	installed.} the {\tt\ttpackage} directory may be placed in the
	`|texmf/fonts/public|' directory, and the directories |source|
	and |drivers| merged and renamed so that \MF\ can find them.

	The files {\tt mff.rc} and {\tt fmv.mff} are input files for
	{\tt mff} and may be ignored if you are not using {\tt mff} to
	generate fonts.

\subsubsec{Driver files ({\tt/drivers})}
	These files are not strictly necessary but are included for
	convenience.  They are used when assignments to the generic
	driver file's parameters on the \MF\ command line is impossible,
	such as when using the standard {\tt MakeTeXPK} script.  A
	driver file is included for each {\tt tfm} supplied.

\subsubsec{\TeX\ input files ({\tt/tex})}
	These files belong in a system-wide \TeX\ inputs area.  The
	files ending with `|.fd|' are font family definition files,
	used by NFSS\vn2 and \LaTeXe.

	On new-style directory trees, the best approach is probably to
	link or rename {\tt texmf/fonts/public/\ttpackage/tex} to
	{\tt texmf/tex/\ttpackage}.

\subsubsec{Documentation files ({\tt/doc}, {\tt/dvi})}
	Files ending in `{\tt.tex}' are plain \TeX\ documents, and will
	not work with \LaTeX.  Compiled ({\tt dvi}) files are supplied
	to save the installer having to run \TeX.  Read {\tt
	install.tex}, first.  The ^{Greek} text encoding is described in
	|magrman.tex|.

	The "Malvern Handbook", |maman.tex| produces cross-references
	automatically via an |aux| file, and will need to be run through
	\TeX\ twice to get the cross-references right.  The front matter
	(preface and table of contents) are printed *last* and should be
	transferred to the front of the handbook before binding.

\begin notepar
	Please do not install any of the macro files used to typeset the
	documentation in the system-wide \TeX\ inputs area.  They are
	not supported, not necessarily generally useful, and very
	nonstandard.  Earlier releases of Malvern gave the macro files
	generic enough names that they might clash with other macro
	files.  They have been renamed to start with `{\tt pdc}' in the
	hope that this will avoid clashes if they are accidentally
	installed.
\end notepar

\subsubsec{Virtual font files ({\tt/vf})}
	These virtual fonts are generated using ^^{Jeffrey, Alan}Alan
	Jeffrey's ^|fontinst| package \cite{fontinst}, and are Malvern
	fonts with the Cork encoding.  With NFSS\vn2 (or \LaTeXe) they
	are used by specifying encoding `{\tt T1}' and family `{\tt
	fmv}'.  See \xref{tab-encoding-codes} for a list of encoding
	codes.

\subsubsec{Font metric files ({\tt/tfm})}
	A selection of precompiled font metric files, including those
	for virtual fonts in {\tt /vf} and the actual fonts needed to
	use them, as well as fonts needed to print the documentation in
	{\tt/dvi}.

\subsec{Using mff to generate Malvern fonts}
	First, install {\tt mff} and arrange that \MF\ will be able to
	find the {\tt mf} files.  Then to create a Malvern font, for
	example $12$-pt Malvern~$55$ ({\tt ma55a12}), give the command:
\begin example
	 |mff ma55a12|
\end example
	To generate Malvern~$55$ and $56$ at magsteps $0$, \frac1/2 and
	$1$, for $300$-dpi and $1000$-dpi printers, you would type:
\begin example
	 |mff --magstep=0,h,1 --dpi=300,1000 ma55a10 ma56a10|
\end example
	If all goes well, the {\tt tfm} and {\tt pk} files generated
	will be installed in the correct directories automatically.

\subsec{Generating Malvern fonts without mff}
	This section presumes you know how to install a normal \MF\
	font.

	Create a driver file for each style of Malvern you want to be
	able to use.  It should have the following form:
\begin lines
	|% |\<name of file>| -- generate |\<size>|-pt Malvern |\<style>
\smallskip
	|font_size |\<size>| pt#;|
	|encoding = |\<number>|; |
	|weight = |\<number>|; hratio = |\<number>|;|
	|slant = |\<number>|; italicness = |\<number>|;|
	|input ma|
	|bye.|
\end lines
 	where \<size> is the design size  in points, and the
	values for the various variables are given below.  The \<style>
	is the two-digit style code described in \xref{s-stylecodes}.

	These driver files should be named after the font they
	correspond to---for example, `{\tt ma76a12.mf}' to generate
	Malvern~$76$.  Then they are used as usual with \MF:
\begin example
	|mf \mode=luxo; mag=|\<number>|; input ma55a12|
\end example
 	where the \<number> is the magnification wanted, or `{\tt 1.0}'
	for no magnification.

\subsubsec{Values for {\it encoding}}
	The variable {\it encoding} specifies the encoding to use---in
	other words, the subset of Malvern's glyphs to be generated.  It
	should be given one of the following values:
\begin display
	\it Encoding&\it Letter&\it Description\cr
\noalign{\kern\jot\hrule\kern1.5\jot}
	${\it encoding} = 1$&	\tt a&
		Malvern A (latin alphabets ABCdef.)\cr
	${\it encoding} = 2$&	\tt b&
		Malvern B (superscripts, symbols etc.)\cr
	${\it encoding} = 3$&	\tt c&
		Malvern C (Cyrillic A{\cy DIlz\char126 })\cr
	${\it encoding} = 5$&	\tt e&
		Malvern E (composite letters \AA \c C\^D\'e\u g\^\i)\cr
	${\it encoding} = 7$&	 \tt g&
		Malvern G (Greek letters {\gr ABGdez})\cr
	${\it encoding} = 19$&	\tt s&
		\TeX\ text \cite{TeXF1}\cr
	${\it encoding} = 26$&	\tt az&
		\TeX\ text, old-style numerals\cr
\end display
 	The letter is the letter used in the font name -- for example,
	{\tt ma55s10} for Malvern with the \TeX\ text encoding (${\it
	encoding} = 19$), and so on.

\begin notepar
	The Malvern programs used to attempt to generate other encoding
	schemes; with ^|fontinst| this all becomes redundant and
	those codes are obsolete.
\end notepar

\subsubsec{Values for {\it weight}}
	The variable {\it weight} specifies the weight (boldness) of the
	font.  It has the following values:
\begin display
	\it Weight&\it Style&\it NFSS\vn2& \it Berry&\it Description\cr
		  &\it digit&\it code&\it code\cr
\noalign{\kern\jot\hrule\kern1.5\jot}
	${\it weight} = 1/4$&	\tt 1& \tt ul& \tt t&	ultra-light\cr
	${\it weight} = 1/2$&	\tt 2& \tt el& \tt i& 	extra-light\cr
	${\it weight} = 3/4$&	\tt 3& \tt\ l& \tt l& 	light\cr
	${\it weight} = 7/8$&	\tt 4& \tt sl& \tt b& 	semi-light\cr
	${\it weight} = 1$&	\tt 5& \tt\ m& \tt r& 	medium\cr
	${\it weight} = 1.3$&	\tt 6& \tt sb& \tt d& 	semi-bold\cr
	${\it weight} = 1.6$&	\tt 7& \tt\ b& \tt b& 	bold \cr
	${\it weight} = 2$&	\tt 8& \tt eb& \tt x& 	extra-bold\cr
	${\it weight} = 3$&	\tt 9& \tt ub& \tt u& 	ultra-bold\cr
\end display
 	The `style digit' is the first digit in the two-digit style
	codes described in \xref{s-stylecodes}.  The `NFSS\vn2\ code' is the
	first half of a corresponding NFSS\vn2\ `font series' code.  The
	`Berry code' is the code for this weight in Karl Berry's font
	naming scheme.

 	Note that some of the character programs produce ugly results
	for large values of {\it weight}.

\subsubsec{Values for {\it hratio}}
	The variable {\it hratio} specifies the ratio between horizontal
	and vertical measurements: in other words, whether the font is
	compressed or expanded.  It may have the following values:
\begin display
	\it Hratio&\it Style&\it NFSS\vn2& \it Berry&\it Description\cr
		  &\it digit&\it code&\it code\cr
\noalign{\kern\jot\hrule\kern1.5\jot}
	${\it hratio} = 0.50$&	{\tt 9} or {\tt 0}&
		\tt ec&\tt o&	extra compressed\cr
	${\it hratio} = 0.80$&	{\tt 7} or {\tt 8}&
		\tt\ c&\tt c& 	compressed\cr
	${\it hratio} = 1.00$&	{\tt 5} or {\tt 6}&
		\tt\ m&\tt r& 	normal width\cr
	${\it hratio} = 1.15$&	{\tt 3} or {\tt 4}&
		\tt\ x&\tt x&	expanded\cr
	${\it hratio} = 1.30$&	{\tt 1} or {\tt 2}&
		\tt ex&\tt w&	extra expanded\cr
\end display
 	The `style digit' is the second half of the two-digit style
	codes described in \xref{s-stylecodes}.  The `NFSS\vn2\ code' is the
	second half of an NFSS\vn2\ `font series' code (for example,
	bold expanded is ${\tt b} + {\tt x} = {\tt bx}$, compressed is
	${\tt m} + {\tt c} = {\tt c}$).  The `Berry code' is for Karl
	Berry's scheme.

 	Beware that the character programs do not all produce good
	results when {\it hratio} is not $1$.

\subsubsec{Values for {\it slant} and {\it italicness}}
	These two variables between them specify whether a font is to be
	produced with italic letterforms or not:
\begin display
	\it Slant&\it Italic? &\it Style&\it NFSS\vn2&\it Berry&\it Description\cr
	&&\it digit&\it code&\it code\cr
\noalign{\kern\jot\hrule\kern1.5\jot}
	${\it slant} = 0$&	${\it italicness} = 0$&	odd&
		\tt n& \tt r& 	upright\cr
	${\it slant} = 1/8$&	${\it italicness} = 1$&	even&	
		\tt it& \tt i&	italic\cr
	${\it slant} = 0$&	${\it italicness} = 1$&	odd&
		\tt ui& \tt u& 	upright italic (!)\cr
	${\it slant} = 1/8$&	${\it italicness} = 0$&	even&	
		\tt sl& \tt o&	oblique\cr
\end display
 	The `NFSS\vn2\ code' is the `font shape' code.  The `Berry
	codes' are variant letters for Karl Berry's font naming scheme.

 	The variable {\it slant} is common to many \MF\ programs, and
	causes the glyphs to be obliqued.  The {\it italicness} variable
	signals that italic letterforms should be used for some letters.
	It is possible to generate an obliqued font or an `upright
	italic' with appropriate settings, but these cannot have `{\tt
	ma}-' names, because I~have not defined style codes for them.

%}}}  installation hints
%{{{  availability

\section{Copying Malvern and reporting bugs}

	Like \TeX\ and \MF\ themselves, the implementation of Malvern in
	\MF\ for use with \TeX\ is free software in the sense used by
	the ^{Free Software Foundation} \cite{GPL}.  (The word `free' is
	used to mean `free of restrictions', rather than `available for
	free'; free software may be bought and sold on the understanding
	that the buyer may make copies and distribute them.)  
^^{distribution}
^^{copying}

	There is a ^{mailing list} for discussion of Malvern problems
	and solutions.  Send ^^{bugs} bug reports, suggestions, and
	ideas to |malvern@comlab.ox.ac.uk|.  To subscribe to the list,
	mail me at |malvern-request@comlab.ox.ac.uk|.

%}}}  availability
%{{{  fontinst

\section{Fontinst}

	^^{Jeffrey, Alan} Alan Jeffrey's ^|foninst| package
	\cite{fontinst} is a set of \TeX\ macros which allows users to
	install ^{virtual fonts}.  It can combine fonts represented by
	Adobe Font Metric ({\tt afm}) or \TeX\ font metric ({\tt pl})
	files into virtual fonts.  These virtual fonts can then be used
	by \TeX.

\xreflabel{s-fontinst}{\S\thesecno}

	The package
\begin bullets
     \\	is written in \TeX, for maximum portability (at the cost of
	speed);
     \\	supports the full ^{Cork encoding};
     \\	allows arbitrary `fake' glyphs -- for example, creating an "ij"
	ligature by placing "i" next to a "j";
     \\	allows capital-and-small-capital fonts with letter-spacing and
	kerning;
     \\	allows kerning information to be copied between glyphs -- for
	example "ij" can be made to kern on the left like "i" and on the
	right like "j";
     \\	allows glyphs from several fonts to be combined to make a new one;
     \\	automatically generates a {\tt fd} file for NFSS~\n2 and \LaTeXe;
     \\	can deal with arbitrary font encodings.
\end bullets
	The current version is a beta release.  It can be obtained from
	the Comprehensive \TeX\ Archive Network (CTAN), in directory
	|fonts/utilities/fontinst|.

	By including some Cork-specific glyphs in the Malvern B
	encoding, I~have managed to spin Cork-encoded Malvern fonts
	(with variant encodings like cap \& small caps) from Malvern~A
	and B fonts.  The fonts generated have names in Karl Berry's
	scheme, like `|fmvmq12|'.

	
%}}}  fontinst
%{{{  to do

\section{To-do list}

	So far both Malvern and this handbook is incomplete---Malvern is
	a spare-time project and progress has slowed to a crawl over the
	last few months.  For the present, the best strategy seems to be
	to release a version that can be picked apart by more
	knowledgable people in order that a later, complete, release be
	that much better.

	Tasks that need attention include:
\begin bullets
\\
	Make guillemets space themselves automagically.
%\\
%	Fix design of ogonek letters \A, \E, \a, \e.
\\
	Make head of \P{} solid.
%\\
%	Tidy up the ^|fontinst| encoding files so that they can be
%	included in future releases.
%\\
%	Need a two drop-in \LaTeX\ style option files---one for standard
%	\LaTeX, one for \LaTeX+NFSS.
\\
	The kerning table is incomplete---I need to develop a
	systematic method for thrashing one out in the absence of a
	decent typesetter or enough specialized knowledge.  Donation of
	a kerning table or clever software for generating same would be
	appreciated.
%\\
%	Need some way of generating VF or VPL files automatically.
\\
	Some composite letters are missing or hastily designed.  I need
	to do more research into other latin-alphabet languages (so that
	I know how to draw their special letters).
\\
	Find example texts containing some of the more esoteric special
	letters (such as "\NG").
\\
	Rethink the proportions of small-capital letters.
%\\
%	Obtain more greek newspapers and magazines and design a
%	lower-case greek alphabet.
\\
	Learn Russian and design a Cyrillic alphabet.
\\
	Script alphabets.  (Just an idea.)
%\\
%	The figures need to be overhauled---the `8' fails horribly in
%	all sorts of conditions.
%\\
%	Lower-case "e" should be different in italic.
\\
	Documentation of the \MF\ code is incomplete---and some of the
	code might usefully be rationalized.
\end bullets

%}}}   to do
\input mabib
%{{{  index

\vfill\eject
\begingroup \leftmargin=0pt \setncolumns3 \notefonts 
	\section{Index}

	\leftskip=1em \parindent=-\leftskip \parskip=0pt plus 1pt
	\let\vn=\n

\inputifexists{\jobname.ind}

\vfill\eject

\endgroup

\index{New Font Selection Scheme|see{NFSS}}


%}}}  index
\iflong
%{{{  colphon

\vfill\eject

\section{Colophon}

	This manual was of course typeset using \TeX\ with Malvern
	fonts.  The manuscript was edited using Richard Stallman's GNU
	Emacs extensible text-editor, with the CMU\TeX\ Major Mode by
	Olin Shivers et al.  The DVI files were previewed with |xdvi|,
	and converted into \PS\ using Tomas Rokicki's excellent |dvips|
	printer driver.

	The body text is set in 10/12 pt Malvern~55, with Malvern~56
	italics and Malvern~75 boldface.

%}2}}  colphon
\fi
	
%}}} Appendix
%{{{ preliminary matter

\vfill\eject
\pageno=-1
\mark{{}{}}

\begingroup
\headline={\hfil} \footline={\hfil}
\iftwosided
%{{{   half-title

\null
\vskip 0pt plus 1fil

\line{\bigrm\spaceskip=0pt plus 1fil M A L V E R N}

\vskip 0pt plus 1.5fil
\eject

\line{}\vfill\eject		% fronticepiece
	
%}}}
\fi
%{{{   title page + title verso

\null
\vfill

\leftline{\bigrm The \iflong\else abridged\fi}
\vskip 24pt
\leftline{\hugerm\spaceskip=-0.05em  M A L\kern-0.1em V E R N}
\vskip 18pt
\rightline{\hugerm\spaceskip=-0.05em  h a n d b o\kern-0.025em o k}
\vskip36pt
	\leftline{\headingfonts P. Damian Cugley}
\vfill

\iftwosided 
	Alleged Literature, Oxford.

	\eject\null\vfill 
\fi

\parindent=0pt \parskip=1ex \leftskip=-\leftmargin

	\copyright\ $1991$--$1994$ P. Damian Cugley.

	The right of Damian Cugley to be identified as the author of
	this work has been asserted in accordance with British copyright
	law.

	Permission is granted by the copyright holder to copy, use and
	and distribute verbatim copies of the Malvern package and this
	handbook without fee so long as ($1$)~the above copyright
	messages and this permissions message are preserved intact on
	all copies; and ($2$)~neither the name of Damian Cugley nor that
	of Oxford University be used in any related promotion or
	advertising without prior written consent.  (Merely
	acknowledging copyright would not count as contradicting~($2$).)

	{\bf Caveat}\quad This documentation is incomplete.  

\bigskip
	This document describes \package.

	\TeX\ was run on this file on \today.
\eject
%}}}
\endgroup

%{{{   preface
\sectionheading{Preface}

\dropcap\hugerm
	Malvern is a sanserif typeface family: a collection of typefaces
	that have been designed together and are intended to coordinate
	with each other.  (`Sanserif' comes from the French {\it sans
	serif\/} and means that the arms and stems of letters do not
	have the small finishing strokes called serifs.)  This handbook
	describes an implementation of Malvern created for use with
	\TeX, a fiendish typesetting system devised by ^^{Knuth, Donald
	E.} Donald~E.\ Knuth.  Like Knuth's own Computer Modern family,
	Malvern is implemented as a set of files which are interpreted
	by \MF, a companion program to \TeX\ which is included in most
	\TeX\ systems.

\iflong
	The reader may be surprised that a typeface needs a `handbook'
	like this one at all---after all, we do not see books called
	`The Garamond User Guide' or `How to get more from your Bodoni'.
	This is partly because of the nature and rarity of \MF\ fonts: they
	are individually-crafted programs, whereas commercial fonts are
	produced systematically using standard software.  Partly it is
	because the Computer Modern family is so thoroughly installed
	into plain \TeX\ and standard \LaTeX\ that some effort is needed
	to switch it over to another font family.  And finally, there is
	the fact that Malvern is itself eccentric and these
	eccentricities need explaining.

	My motivation to create Malvern comes from two sources.
	Firstly, the \TeX\ system has always suffered from a dearth of
	font designs: In the standard \TeX\ distribution, there are two
	sanserif families (Computer Modern Sans Serif and CM Sans Serif
	Quotation) both of which are derived from the same character
	programs as the Computer Modern fonts.  This is a nice gimmick,
	but in practice the natures of sanserif and serif faces are so
	much at odds that they cannot be usefully merged into one
	`meta-design'.
%	In particular, the capitals of \cmrname{CMSS}
%	and \cmrname{CMSSQ} do not seem to me to coordinate well with
%	their lower case.  The $x$-height for \cmrname{CMSS} is small
%	for a sanserif font, and makes it look top-heavy---but a
%	side-effect of increasing the $x$-height in \cmrname{CMSSQ} is
%	to make the capitals so extended as to be distended.  
%	My attempts to coax what I~considered to be a nice sanserif face
%	out of the CM programs---with all sorts of hacks and
%	fudges---became more and more elaborate, but to no avail.
	Of course there are many wonderful faces available
	commercially---but I have no money with which to buy them.  (One
	of the great things about \TeX\ is that it is free, so that
	quality typesetting is available to any sufficiently-motivated
	pauper with access to a computer.)

	The second main impetus was more immediate---it was the
	publication of the proposed standard \TeX\ font encoding in
	\TUGboat\ (the \TeX\ User Group journal).  I~wanted to write an
	article or letter about this---but to do so I~needed to be able
	to include many peculiar symbols in a \TeX\ document.  After my
	first attempts (creating a font to augment Computer Modern) it
	seemed obvious that designing a new font from scratch would be
	less effort in the long run than trying to come up with a bodge
	to patch  existing fonts.
\fi

	This handbook describes my implementation of Malvern and some
	ways in which Malvern can be used with \TeX\ (both plain \TeX\
	and with the \LaTeX\ macro package).  
\iflong
	It also includes
	information that relates to Malvern in general, as opposed to
	any one implementation in particular.  It is possible that at
	some future date someone will make versions of Malvern fonts for
	other systems, in which case they will surely supply their own
	handbook to supplement this one.

	Different parts of this handbook will be relevant to different
	people.  The first sections describe Malvern as it is used in
	\TeX\ (or \LaTeX) documents, including the developement of \TeX\
	macros to take advantage of Malvern's features (these are aimed
	at \TeX\ experts and should be ignored by people who want merely
	to have a system that works).

	Later sections give an outline of the \MF\ code used to
	implement Malvern.  I~will not describe it in great detail but
	I~will attempt to outline some of its weirder features.  This
	section is for the satisfaction of myself and other \MF-hackers,
	and is of even less interest to the typical Malvern user than
	the \TeX\ programming.
\fi

	The appendix includes a summary of how to install Malvern, some
	of the code described in previous sections, some tables (useful
	or otherwise) and a list of books and articles referenced
	elsewhere.

\rightline{Damian Cugley, June \n{1994}}

%}}}
%{{{   TOC
\sectionheading{Table of Contents}

\immediate\closeout\TOCfile
\begingroup

\def\TOCentrysubsec#1#2#3{#1\ #2\quad}

\input \jobname.toc
\endgroup
%}}}
%}}}  preliminary matter
\bye
%}}} maman.tex


%Local variables: 
%fold-folded-p: t
%tex-has-children: t
%fill-prefix: "\t"
%End: 

