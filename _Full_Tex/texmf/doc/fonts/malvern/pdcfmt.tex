% pdcfmt.tex 1.1.9 1994/07/20 -- macros for larger-scale formatting of text
% Copyright  1991, 1992, 1993 P. Damian Cugley.

%%% @TeX-macro-file {
%%%   filename       = "pdcfmt.tex",
%%%   version        = "1.1.9",
%%%   date           = "1994/07/20",
%%%   package        = "Malvern 1.1",
%%%   author         = "P. Damian Cugley",
%%%   email          = "damian.cugley@comlab.ox.ac.uk",
%%%   address        = "Oxford University Computing Laboratory,
%%%                     Parks Road, Oxford  OX1 3QD, UK",
%%%   codetable      = "USASCII",
%%%   keywords       = "TeX, plain TeX, macros",
%%%   supported      = "Maybe",
%%%   abstract       = "Formatting macros for plain TeX documents.",
%%%   dependencies   = "pdcutil.tex",
%%% }

%{{{ pdcfmt.tex
\begingroup\catcode`\%=12 \toks0={\endgroup
	\gdef\version{1.1.9} \gdef\lastedit{pdc 1994/07/20}
}\the\toks0
\message{\version\space \lastedit}
\ifx\utilsversion\UNDEFINED \input pdcutil \fi

%{{{  \everypar -- \noindenttrue and \parbox

\newif\ifnoindent
%  If set then next paragraph is not indented

\newbox\parbox	
\newdimen\parboxsep \parboxsep=1pc
%  box to go at left of current paragraph -- usually a \vtop or void:
%  If not void then next paragraph is not indented, and
%  the box is set \parboxsep from left margin.

\everypar=
{%
    \ifvoid\parbox
	\ifnoindent {\setbox0=\lastbox}\global\noindentfalse \fi 
	    % after \noindent this has no effect anyway
    \else
	{\setbox0=\lastbox}\global\noindentfalse % implied noindenttrue
	\dp\parbox=0pt
	\hbox to 0pt{\hss \box\parbox \hskip\parboxsep}%
    \fi
}
%}}}
\newdimen\envindent \envindent=1pc  % used when \parindent=0pt
\def\endenv
{
    \par
    \ifdim\lastskip<\smallskipamount \removelastskip\smallskip \fi
    \global\noindenttrue
}  
	% generic end of environ.
%{{{  Bullets

\newcount\bulletlevel \bulletlevel=-1
\def\bulletsign
{%
    \ifcase\bulletlevel \bullet\or --\or \circ\else \cdot\fi
}

\def\bullets
{%
    \par 
    \advance\bulletlevel 1
    \def\\{\smallskip\noindent\llap{\bulletsign\enspace}\ignorespaces}
    \ifdim\parindent>0pt
	\advance\leftskip\parindent
    \else
	\advance\leftskip\envindent
    \fi
    \the\everybullets
}

\let\endbullets=\endenv

\newcount\enumerateitem
\def\enumerate
{%
    \global\enumerateitem=0
    \def\bulletsign{\global\advance\enumerateitem1 \n{\the\enumerateitem}.}%
    \bullets
}

\let\endenumerate=\endbullets
%}}}
%{{{  Tagged

\newdimen\taglabelwidth
\def\tagged			% like bullets, but with tags
{%
    \par
    \def\\%
    {
	\smallskip\noindent
	\setbox0=\hbox\bgroup	% matched by \finishtag
	\the\everytag\ignorespaces
	\let\\\finishtag
    }
    \def\finishtag
    {
	\unskip\enspace\egroup		% matches \\
	\ifdim \wd0 < \taglabelwidth
	    \wd0=\taglabelwidth
	    \llap{\box0}%
	\else
	    \hskip-\taglabelwidth
	    \unhbox0
	    \hfil\break
	    \vadjust{\nobreak}%
	\fi
  	\ignorespaces
    }
    \ifdim\leftmargin=0pt 
	\ifdim\parindent>0pt
	    \envindent=2\parindent
	\fi
	\leftskip=\envindent 
	\taglabelwidth=\envindent
    \else
	\taglabelwidth=\leftmargin
    \fi
    \the\everytagged
}

\let\endtagged=\endenv
%}}}
%{{{  Quotations

\def\quotation
{
    \smallskip
    \ifdim\parindent>0pt
	\advance\leftskip\parindent % \advance\rightskip\parindent
    \else
	\advance\leftskip\envindent % \advance\rightskip\envindent
    \fi
    \noindenttrue
    \the\everyquotation
}
\let\endquotation=\endenv
%}}}
%{{{  lines (for verbatim listings etc)

\def\lines
{
    \smallskip\hrule\nobreak\smallskip
    \obeylines
    \parindent=0pt \parskip=0pt 
    \parfillskip=0pt plus 1fil
    \the\everylines % user chooses font to use
    \counta=0 \countb=5 \everypar{\linenum}%	number lines
    % %  Cut into left margin to allow for at least 72 columns:
    % \leftskip=-\leftmargin \advance\leftskip2em % space for ~3 digits
    % \setbox0=\hbox{\tt x}\dimen0=-72\wd0 % estimate of 72 \tt characters
    % \advance\dimen0\hsize  
    % \ifdim\leftskip<\dimen0 \leftskip=\dimen0 \fi
    \let\end\linesend
}

\let\normalend\end
{\obeylines   \gdef\linesend#1^^M{\normalend #1 }}

\def\endlines
{
    \par
    \ifdim\lastskip<\smallskipamount \removelastskip\nobreak\smallskip \fi
    \hrule\smallskip
    \global\noindenttrue
}  

\def\linesskipped#1%
}}
%{{{  Displays, Tables -- wrapper about \halign

%  \begin table #\hfil&#\cr
%    ...&...\cr
%  \end table

\def\table
{
    \noindent
    $$ % matching $$ is in \endtable
    \halign \bgroup\indent
}
\def\endtable
{
	\crcr \egroup $$ % matches $$ in \table
    \global\noindenttrue
}

%  Random displayed text
%  
%  \begin display
%    ... [&...\cr...]
%  \end display


\def\display
{%
    $$%					$$ for matching
	\halign\bgroup\indent##\hfil&&\quad##\hfil\cr
}

\def\enddisplay
{%
	    \crcr
	\egroup%						$$ for matching
    $$
}

%}}}
%{{{  (E)BNF

\def\bnf
{
    \nobreak\smallskip
    \advance\leftskip2\parindent \parindent=-\parindent 
    \parskip=0.5\smallskipamount
    \rightskip=1\rightskip plus 3em
    \def\\{$\mid$}  % use \\ for alternatives
    \def\>{\unskip\enspace$::=$\enspace\ignorespaces}
    \def|{`\begingroup\tt\setupverbatim\def|{\endgroup'}}
    \chardef\*=`\*
    \the\everybnf
}
\def\endbnf
{
    \smallskip
    \global\noindenttrue
}

\newtoks\everybnf

%}}}
%{{{  headings

\newdimen\leftmargin  % size of left margin
%  Set this to 0pt for headings in text, and > 0pt for left margin headingfs
%  

%  #1 is skip before -- usu bigskip or medskip
%  #2 is pre-text text -- usu. style: e.g. \headerfonts or just \bf
%  #3 is post-text text -- e.g. to do a rule, maybe?
%  #4 is text of heading
%
\newskip\headingtemp
\def\doheading#1#2#3#4%
{
    \ifdim\lastskip<#1\relax \removelastskip \vskip#1\relax \fi
    \ifdim \leftmargin>0pt
	\global\setbox\parbox=\vtop
	  {%
	    \hsize=\leftmargin \advance\hsize-\parboxsep
	    \parindent=0pt
	    \leftskip=0pt \rightskip=0pt plus 3em
	    \hyphenpenalty=10000 \exhyphenpenalty=5000
	    \strut#2#4#3
	  }
    \else
	\begingroup
	    \parindent=0pt \parfillskip=0pt plus 1fil
	    \leftskip=0pt \rightskip=0pt plus0.25\hsize 
	    \hyphenpenalty=10000 \exhyphenpenalty=5000
	    \strut#2#4#3
	    \global\headingtemp=\baselineskip % sets fixed part only
	    \par
	\endgroup
	\advance\headingtemp-\baselineskip 
%	\advance\headingtemp\smallskipamount
	\nobreak
	\vskip 1.0\headingtemp
  	\smallskip
	\noindenttrue
    \fi
}

%  #1 is white space at top of page
%  #2 is pre-text
%  #3 is post-text
%  #4 is text of heading
%
\def\newpageheading#1#2#3#4%
{
    \vfill\supereject % ensure no insertions still floating
    \null\vskip#1\relax
    \moveleft\leftmargin\vbox
      {
	\advance\hsize\leftmargin
	\parindent=0pt \parfillskip=0pt plus 1fil
	\leftskip=0pt \rightskip=0pt plus0.25\hsize 
	\hyphenpenalty=10000 \exhyphenpenalty=5000
	\strut#2#4#3
	\global\headingtemp=\baselineskip % sets fixed part only
	\par
      }
    \advance\headingtemp-\baselineskip 
    \advance\headingtemp\smallskipamount
    \vskip \headingtemp 
    \smallskip
    \noindenttrue
    \def\tmp{#4}
    \message{*\expandafter\TOCtrim\meaning\tmp.  }
}
%}}}
%{{{  footnotes

\newcount\notecount		% initially 0
\def\note
{%
    \global\advance\notecount+1
    \footnote{\number\notecount}%
}

\catcode`\@=11			%  See TeXbook p.363:
\def\footnote#1{\let\@sf\empty
  \ifhmode\edef\@sf{\spacefactor\the\spacefactor}\/\fi
  \flushtop{\footnotetextmark{#1}}\@sf\vfootnote{#1}}
\def\vfootnote#1%
{\insert\footins\bgroup		% matched by \@foot
    \interlinepenalty=\interfootnotelinepenalty
    \leftskip=0pt 
    \the\everyfootnote
    \splittopskip=\ht\strutbox \splitmaxdepth=\dp\strutbox
    \floatingpenalty=20000
    \indent\footstrut
    \ifdim\parindent>1em 
	\llap{\footnotenotemark{#1}\enspace}%
    \else
	\footnotenotemark{#1}\enspace
    \fi
    \futurelet\next\fo@t
}
\def\@foot{\smallskip\egroup}
\catcode`\@=12
\def\footnotetextmark#1{$^{#1}$}
\def\footnotenotemark#1{$^{#1}$}
%}}}
%{{{  Hooks

\newtoks\everybullets
\newtoks\everytagged
\newtoks\everytag % e.g., \everytag={\bf}
\newtoks\everyquotation	% e.g., \everyquotation={\smallfonts}
\newtoks\everylines
\newtoks\everyfootnote
\newtoks\everyfootnotemark%} }}
%}}} pdcfmt.tex

%Local variables:
%fold-folded-p:t
%tex-macros-p:t
%End:
