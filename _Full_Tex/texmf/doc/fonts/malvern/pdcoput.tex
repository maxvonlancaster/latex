% pdcoput.tex 4.1.6 1994/07/20 -- output routines for Stomping Scene
% Copyright 1994 P. Damian Cugley

%%% @TeX-macro-file {
%%%   filename       = "pdcoput.tex",
%%%   version        = "4.1.6",
%%%   date           = "1994/07/20",
%%%   package        = "Malvern 1.1",
%%%   author         = "P. Damian Cugley",
%%%   email          = "damian.cugley@comlab.ox.ac.uk",
%%%   address        = "Oxford University Computing Laboratory,
%%%                     Parks Road, Oxford  OX1 3QD, UK",
%%%   codetable      = "USASCII",
%%%   keywords       = "TeX, plain TeX, output routine",
%%%   supported      = "Maybe",
%%%   abstract       = "An output routine for Plain TeX docs.",
%%%   dependencies   = "pdcutil.tex",
%%% }

%  See the Malvern Handbook (maman.tex) for more info about Malvern.
%  This software is available freely but without warranty.
%  See the file COPYING for details.

%{{{ pdcoput.tex
%{{{  preamble
\ifx\utilsversion\UNDEFINED \input pdcutil \fi

\begingroup\catcode`\%=12 \toks0={\endgroup
    \def\pdcoputversion{4.1.6 <pdc 1994/07/20>}
}\the\toks0
\message{\pdcoputversion}

%}}}  preamble
%{{{  parameters

\newdimen\paperwd \newdimen\paperht 	% Size of paper
\newdimen\bodywd \newdimen\bodyht	% size of page body (text etc.)
\newdimen\gridwd \newdimen\gridht	% size of grid slots
\newdimen\colwd \newdimen\colht		% width & height of text columns

\newdimen\topmarge	\topmarge=20mm
\newdimen\outmarge	\outmarge=20mm
\newdimen\innmarge	\innmarge=20mm
\newdimen\botmarge	\botmarge=27mm
%  Size of top, bottom, inner and outer margins between paper edge and
%  edge of page body.
%  botmarge is an approximation -- the margin at bottom will be
%  increased slightly in order to make \bodyht a good value wrt
%  \baselineskip.

\newskip\headlineskip 	% distance betw baselines of headline & 1st body line
\newskip\footlineskip	% ditto for footline
\headlineskip=10mm \footlineskip=10mm

\newdimen\outputrule	\outputrule=0.1mm % default thickness for rules

\newdimen\caprule	\caprule=\outputrule
\newskip\capsep		\capsep=5mm
%  rule and separation between caption and body text

\newdimen\colrule	\colrule=\outputrule
\newdimen\colsep	\colsep=5mm
%  ditto for between columns

\newcount\ncols		\ncols=1
%  Number of columns

\newbox\partialpage  % stores partial pages when balancing columns

\newbox\topbox \newbox\botbox	% vboxes placed around body
\newbox\leftbox \newbox\rightbox % hboxes placed around body

\chardef\pagebox=255

\let\collist\empty		% list macro of \makecolumns
\let\pagelist\empty		% list macro of \makepages

\newtoks\everypage	% expanded at start of vbox being shipped out 
\newtoks\everycaption	% expanded at start of every caption

%}}}  parameters
%{{{  make a column (vbox)

\newdimen\toptotal \newdimen\bottotal

% Helper macro -- calculate total size of space + caption for
% top/bot of a column
%  #1  <dimen name>	caculate total into here
%  #2  <8-bit number>	place cap in this box
%  #3  <dimen>		space to be left
%  #4	text		text of caption, or empty
\def\calctotal#1#2#3#4%
{%
    \setbox#2=\vbox{\the\everycaption#4}%
    #1=\ht#2%
    \advance#1\baselineskip \advance#1-1sp
    \divide#1\baselineskip \multiply#1\baselineskip
    \advance#1#3%
    \ifdim#1>0pt \advance#1\capsep \fi
}

% Create one column of stuff of height \colht, as a vbox
% If \capsep=\baselineskip and captions done in fonts of same
% baselineskip as body, then baselines should all line up neatly
% (this takes some doing!).
%  #1  <dimen>		space at top of col
%  #2  text		text of top caption, or empty
%  #3  <8-bit number>	box to make middle out of
%  #4  text		text of bot cap, or empty
%  #5  <dimen>		space at bottom of col
\def\makecolumn#1#2#3#4#5%
{%
    \vbox to \colht
      {
	\calctotal\toptotal0{#1}{#2}%
	\calctotal\bottotal2{#5}{#4}%
	%
	%  Chop off enough text to fill in gap:
	%
	\dimen0=\colht 
	\ifvoid\footins\else	% allow space for footnotes
	    \advance\dimen0-\ht\footins
	    \advance\dimen0-\skip\footins
	\fi
	\ifvoid\topins\else	% allow for a \topinsert as well
	    \advance\dimen0-\ht\topins
	    \advance\dimen0-\skip\topins
	\fi
	\advance\dimen0-\toptotal
	\advance\dimen0-\bottotal
	%
	\splittopskip=\topskip
	\splitmaxdepth=\maxdepth
	\setbox1=\vsplit#3to\dimen0
	%
	%  Sandwich it all together:
	\kern\toptotal
	\ifdim \toptotal>0pt   
	    \vbox to 0pt
	      {
		\vss
		\box0
%%%%		\kern-\dp\strutbox
		\capseprule
		\kern-\dp\strutbox
	 	\kern\baselineskip
		\kern-\topskip
	      } 
	\fi
	\ifvoid\topins\else \unvbox\topins \vskip\skip\topins \fi
	\dimen0=\dp1	% this will be prevdepth for bot caption
	\ifvoid1 \vfil \else \unvbox1 \fi
	\ifvoid\footins \else
	    \vskip\skip\footins
	    \footnoterule
	    \dimen0=\dp\footins  %  revise prevdepth-to-be
	    \unvbox\footins
	\fi
	\ifdim \bottotal>0pt
	    \vbox to 0pt
	      {
		\prevdepth=\dimen0
		\capseprule
		\vtop{\unvbox2 }
		\vss
	      } 
	\kern\bottotal
	\fi
	\kern-\dimen0
      }%
}

\def\capseprule
{%
    \hbox to \colwd
      {%
	\dimen0=0.5ex \advance\dimen0-0.5\caprule
	\dimen1=\dimen0 \advance\dimen1\caprule
	\strut
	\vrule height \dimen1 depth-\dimen0 width\colwd
      }%
}

%}}}  make a column
%{{{  make body part of page (hbox)

%  Generate a column of text to be put in the current page body
\def\makecolumnfromcollist#1%
{%
    \ifx\collist\empty
	\makecolumn{0pt}{}{#1}{}{0pt}%
    \else
	\glop\temp\collist
	\temp
    \fi
}

\def\makebody#1%
{%
    \hbox to \bodywd
      {%
	\ifvoid\leftbox\else 
	    \box\leftbox 
	    \hfil \vrule width \colrule \hfil 
	\fi
	\counta=\ncols
	\makecolumnfromcollist{#1}%
	\loop \ifnum\counta>1
	    \hfil  \vrule width \colrule depth 0pt  \hfil
	    \makecolumnfromcollist{#1}%
	    \advance\counta -1
	\repeat
	\ifvoid\rightbox\else 
	    \hfil \vrule width \colrule \hfil 
	    \box\rightbox 
	\fi
      }%
}

%}}}  make insides of page (hbox)
%{{{  make a page (vbox)

%  Generate a box to be shipped out as the current page.  Used in \output.
\def\makepage
{%
    \vbox
      {
	\special{papersize=\the\paperwd,\the\paperht}%
	\colht=\bodyht \advance\colht-\ht\partialpage
	\advance\colht-\ht\topbox \advance\colht-\ht\botbox
	\the\everypage
	\makeheadline \nointerlineskip
	\box\partialpage
	\box\topbox
	\ifx\pagelist\empty
	    \makebody\pagebox
	\else
	    \glop\temp\pagelist 
	    \temp
	\fi
	\unvbox\botbox
	\makefootline
      }%
}

%  Generate vertical meterial with zero height to produce the headline.
%  Expanded at very top of vbox being shipped out,
%  immediately after the \everypage tokens have been read.
%  Used in \makepage.
\def\makeheadline
{
    \vbox to 0pt
      {
	\skip0=\topskip \advance\skip0-2\ht\strutbox 
	\advance\skip0-\headlineskip
	\vskip\skip0
	\hbox to \bodywd{\strut\the\headline}
	\vss
      }
}

%  Generate vertical material that gives the footline of the page.
%  Expanded immediately after the botbox, as the last thing in the
%  box that is shipped out.  (Does not need to be zero-height).
\def\makefootline
{{
    \baselineskip=\footlineskip
    \hbox to \bodywd{\the\footline}
}}

%}}}  make a page (vbox)
%{{{  output

\def\ssoutput
}}  output
%{{{  automated column setting

%  User macro -- set paper size and other corresponding dimens
%  Use when body font + baselineskip have already been selected.
%  #1  <dimen>  width
%  #2  <dimen>  height
\def\setpaper#1#2%
{%
    \paperwd=#1%
    \paperht=#2%
    \calcbodywdht
    \calctopskip
}

%  helper macro -- set bodywd, bodyht
\def\calcbodywdht
{
    \bodywd=\paperwd \advance\bodywd-\outmarge \advance\bodywd-\innmarge
    \bodyht=\paperht \advance\bodyht-\topmarge \advance\bodyht-\botmarge
    \calctopskip
}

%  set topskip to suit current font
%  bodyht is munged to be of the form N*baselineskip + topskip
\def\calctopskip
{
    \setbox0=\hbox{Xbl()!gyJQ,}
    \topskip=\ht0 \maxdepth=\dp0
    \advance\bodyht-\topskip \divide\bodyht \baselineskip
    \multiply\bodyht \baselineskip \advance\bodyht\topskip
}

%  helper macro
%  set hsize and vsize appropriate to current #cols, topskip, bodyht,
%  and bodywd
%  if bodyht = N * baselineskip + topskip, for some N, then set 
%  vsize to ncols * (N + 1) * baselineskip - baselineskip + topskip
\def\sethsizevsize
{
    \hsize=\colwd
    \vsize=\bodyht 
    \advance\vsize-\topskip \advance\vsize\baselineskip \multiply\vsize\ncols 
    \advance\vsize-\baselineskip \advance\vsize\topskip
}

%  User macro
% divide into grid of n, with k grid widths making 1 column of body
%  #1  <number>  number of grid slots making up body width
%  #2  <number>  number of grid slots making up one column
%   so there will be floor(#1/#2) body columns per page
\def\setnkgrid#1#2%  
{%
    \gridwd=\bodywd 
    \advance\gridwd-#1\colsep \advance\gridwd\colsep \divide\gridwd by #1
    \colwd=#2\gridwd \advance\colwd#2\colsep \advance\colwd-\colsep
    \ncols=#1 \divide\ncols#2
    \gridht=\baselineskip
    \capsep=\gridht
    \sethsizevsize
}

%  User macro
%  divide into n columns
%  #1  <number>  number of columns per page
\def\setncolumns#1%
{%
    \colwd=\bodywd
    \advance\colwd-#1\colsep \advance\colwd\colsep \divide\colwd by #1
    \gridwd=\colwd
    \ncols=#1
    \sethsizevsize
}

\def\setpaperA#1%
{
    \ifcase#1
	\setpaper{841mm}{1189mm}\or	% A0
	\setpaper{594mm}{841mm}\or	% A1
	\setpaper{420mm}{594mm}\or	% A2
	\setpaper{297mm}{420mm}\or	% A3
	\setpaper{210mm}{297mm}\or	% A4
	\setpaper{148mm}{210mm}		% A5
    \else
	\errmessage{Dunno how big DIN A#1 paper is, sorry.}
    \fi
}
%  default settings (using default font):
\setpaperA4
\setncolumns1

%}}}  automated column setting
%{{{  balancing columns

%  Knuth's rigid balancing routine  TeXbook App. D Sec. 7 (p. 396--397)

\newcount\RGDncols \newdimen\RGDtopskip \newif\ifRGDfirstcol
\newtoks\RGDsep

%  Take a rigid, regular box, and split into several approx equal
%  columns.  Produces assignments followed by a box -- this can be
%  retrieved using \lastbox.
%  #1  <8-bit number>	box to chop
%  #2  <number>   	#columns to make
%  #3  <dimen>		value of \topskip for box
%  #4  <dimen>		width of the whole shebang
%  #5  TeX code		what goes between columns (e.g., \hfil)
\def\rigidbalance#1#2#3#4#5%
{%
    \setbox0=\box#1\relax \RGDncols=#2\relax \RGDtopskip=#3\relax \RGDsep={#5}%
    \hbox to #4\relax
      {%
	\splittopskip=\RGDtopskip \vbadness=10000 
	\RGDfirstcoltrue
	\valign{##\vfil\cr \doRGDsplits}%
      }%
}

%  recursive helper function for above - produces the valign's body
\def\doRGDsplits
{%
    \ifnum\RGDncols>0 
	%  if not first folumn, put in separator
	\ifRGDfirstcol 
	    \global\RGDfirstcolfalse 
	\else
	    \noalign{\the\RGDsep}%
	\fi
	%  split off one column (see TeXbook for justification
	%  for this value of \dimen0):
	\dimen0=\ht0  \divide\dimen0 \RGDncols  \advance\dimen0 \RGDtopskip
	\setbox1=\vsplit0 to \dimen0
	\unvbox1
	%
	\global\advance\RGDncols-1
	\cr
	\doRGDsplits
    \fi
}

%}}}  balancing columns
%}}} pdcoput.tex
%Local variables:
%fold-folded-p: t
%tex-macros-p: t
%End:
