\documentclass{article}

\usepackage[monospacemath]{mmasym}

%%%%%%%%%%%%%%%%%%%%%%%%%%%%%%%%%%%%%%%%%%
\raggedbottom 

\font\TTT=ptmr7t at 7pt
\newcount\cno
\def\TT{\T\setbox0=\hbox{\char\cno}\ifdim\wd0>0pt
   \box0\lower4pt\hbox{\TTT\the\cno}\else
   \ifdim\ht0>0pt \box0\lower4pt\hbox{\TTT\the\cno}\fi\fi
   \global\advance\cno by1
}
\def\showfont#1{\font\T=#1 at 10pt\global\cno=0
 \tabskip1pt plus2pt minus1pt\halign to\textwidth{&\hss\TT ##\hss\cr
 \multispan{16}\hfil \texttt{Font #1}\hfil\cr\noalign{\smallskip}
 &&&&&&&&&&&&&&&\cr
 &&&&&&&&&&&&&&&\cr
 &&&&&&&&&&&&&&&\cr
 &&&&&&&&&&&&&&&\cr
 &&&&&&&&&&&&&&&\cr
 &&&&&&&&&&&&&&&\cr
 &&&&&&&&&&&&&&&\cr
 &&&&&&&&&&&&&&&\cr
 &&&&&&&&&&&&&&&\cr
 &&&&&&&&&&&&&&&\cr
 &&&&&&&&&&&&&&&\cr
 &&&&&&&&&&&&&&&\cr
 &&&&&&&&&&&&&&&\cr
 &&&&&&&&&&&&&&&\cr
 &&&&&&&&&&&&&&&\cr
 &&&&&&&&&&&&&&&\cr
}}


\begin{document} 

\title{
The \MathLogo{} Virtual Font Package\\
\MathLogo{} fonts for \LaTeXe{}\\
1.0 Release}
\date{26 September 1997}
\author{Jens-Peer Kuska}
\maketitle

\section{Introduction}

\MathLogo{} comes with a full set of mathematical 
fonts in PostScript format. These fonts can be used
to typeset mathematical texts together with the standard 
PostScript font, Times-Roman. To use the fonts together
with \TeX{} and the macro package \LaTeXe{} \cite{LTeXComp}, more 
than just the plain PostScript Type1 fonts are needed. \TeX{}
must be informed about the dimensions of the characters and,
for typesetting mathematics, \TeX{} needs the information on how
to change the sizes of the operators and delimiters coded in the fonts \cite{TeXBook}.

\TeX{} provides the virtual font mechanism to borrow 
characters from certain fonts and to assemble new ones. 
\MathLogo{} has introduced some new symbols like 
$\ii=\sqrt{-1}$ and $\ee$ for the base of the natural logarithm and
the mathematical alphabets $\mathbb{DoubleStruck}$, $\mathfrak{Gothic}$
and $\mathcal{Script}$ with lower case letters. Similar mathematical
alphabets can be found in the font sets of the American Mathematical
Society. The virtual fonts \textsf{zw*.vf}, \textsf{zw*.tfm} and
the style file \textsf{mmasym.sty} replace the standard computer modern
fonts with the Times-Roman family for normal text, Helvetica family for 
sansserif text, and the Courier family for monotype text. 
The four mathematical fonts for
operators, letters, symbols and extensible symbols are replaced by the
virtual fonts of the \textsf{zwa} family. The first 128 characters of the
new fonts conform to the standard \TeX{} encoding for mathematical 
fonts. Some of the slots with higher character codes are used for the
new symbols. The main part of \textsf{mmasym.sty} deals with the
setup of the new symbols.

The virtual fonts where created with Alan Jeffrey's fontinst package \cite{GrComp}.

\section{Files and Installation}

To use the fonts an DVI-driver that understands virtual fonts like 
\textsf{dvips} is needed.
The new fonts all start with the \textsf{zw} letters. The combination
with Times is indicated by the letter \textsf{a}, the mono spaced fonts
by the letter \textsf{c}. A single 
character follows for the kind of font and the next character
refers to the weight of the font.
\begin{center}
\begin{tabular}{|rll | c | l|}
\hline\hline
\textsf{zwa}&\textsf{r}&\textsf{m} & OT1 & medium weight operator\\
\textsf{zwa}&\textsf{m}&\textsf{m} & OML & medium weight math italic\\
\textsf{zwa}&\textsf{y}&\textsf{m} & OMS & medium weight symbol\\
\textsf{zwa}&\textsf{v}&\textsf{m} & OMX & medium weight math extensions\\
\hline
\textsf{zwc}&\textsf{r}&\textsf{m} & OT1 & Courier medium weight operator\\
\textsf{zwc}&\textsf{m}&\textsf{m} & OML & Courier medium weight math italic\\
\textsf{zwc}&\textsf{y}&\textsf{m} & OMS & Courier medium weight symbol\\
\textsf{zwc}&\textsf{v}&\textsf{m} & OMX & Courier medium weight math extensions\\
\hline
\textsf{zwa}&\textsf{r}&\textsf{b} & OT1 & bold operator\\
\textsf{zwa}&\textsf{m}&\textsf{b} & OML & bold math italic\\
\textsf{zwa}&\textsf{y}&\textsf{b} & OMS & bold symbol\\
\textsf{zwa}&\textsf{v}&\textsf{b} & OMX & bold math extensions\\
\hline
\hline
\textsf{zwc}&\textsf{r}&\textsf{b} & OT1 & Courier bold operator\\
\textsf{zwc}&\textsf{m}&\textsf{b} & OML & Courier bold math italic\\
\textsf{zwc}&\textsf{y}&\textsf{b} & OMS & Courier bold symbol\\
\textsf{zwc}&\textsf{v}&\textsf{b} & OMX & Courier bold math extensions\\
\hline
\textsf{zws} & & \textsf{m}         &  U  &  keyboard and text symbols \\
\hline
\end{tabular}
\end{center}

\begin{sloppypar}
\TeX{} needs the font metric files (\textsf{*.tfm}),
the virtual fonts (\textsf{*.vf}) and the native 
font metric files from the \MathLogo{} fonts (i{.}e{.} 
\textsf{Math1.tfm}, \textsf{Math2.tfm}, \ldots).
\LaTeXe{} needs the font definition files 
\textsf{OT1zw*r.fd}, \textsf{OMLzw*m.fd}, \textsf{OMSzw*y.fd} and
\textsf{OMXzw*v.fd} to access the fonts. Additionally, the 
font definitions for the Times-Roman (\textsf{OT1ptm.fd}), Helvetica
(\textsf{OT1phv.fd}) and Courier fonts (\textsf{OT1pcr.fd})  are needed
for the style file \textsf{mmasym.sty}.	The last three files comes usual
with the \LaTeXe{} packages in the \textsf{psnfss} directory.
\end{sloppypar}
A DVI driver creating PostScript output must be informed 
to download the Type 1 \MathLogo{} fonts. For \textsf{dvips} this is done by
adding the map file \textsf{mma.map}
%
\begin{verbatim}
Math1       Math1       <Math1.pfa
Math1-Bold  Math1-Bold  <Math1-Bold.pfa
Math2       Math2       <Math2.pfa
Math2-Bold  Math2-Bold  <Math2-Bold.pfa
Math3       Math3       <Math3.pfa
Math3-Bold  Math3-Bold  <Math3-Bold.pfa
Math4       Math4       <Math4.pfa
Math4-Bold  Math4-Bold  <Math4-Bold.pfa
Math5       Math5       <Math5.pfa
Math5-Bold  Math5-Bold  <Math5-Bold.pfa
Math1Mono      Math1Mono <Math1Mono.pfa
Math1Mono-Bold Math1Mono-Bold <Math1Mono-Bold.pfa
Math2Mono      Math2Mono <Math2Mono.pfa
Math2Mono-Bold Math2Mono-Bold <Math2Mono-Bold.pfa
Math3Mono      Math3Mono <Math3Mono.pfa
Math3Mono-Bold Math3Mono-Bold <Math3Mono-Bold.pfa
Math4Mono      Math4Mono <Math4Mono.pfa
Math4Mono-Bold Math4Mono-Bold <Math4Mono-Bold.pfa
Math5Mono      Math5Mono <Math5Mono.pfa
Math5Mono-Bold Math5Mono-Bold <Math5Mono-Bold.pfa
\end{verbatim}
%
to the \textsf{config.ps} file with
\begin{verbatim}
p+ mma.map
\end{verbatim}
Additionally the environment variables of \textsf{dvips} must be updated to
append the directory with the fonts.


\textbf{Important Note:} 
This map file includes the \MathLogo{} fonts into your PostScript output and
these fonts can be extracted from the resulting PostScript file. 
If the PostScript file is
sent to people without a \MathLogo{} license, Wolfram Research, Inc., will
probably interpret this as a violation of its copyright.

One way to avoid this problem is to make the fonts available to 
your resident PostScript interpretor 
(Your PostScript interpretor might be your copy of GhostScript,
your PostScript printer, or some other PostScript rasterizing system.)
and then to not include the 
fonts into your PostScript file.  
The file would then contain only
references to the appropriate \MathLogo{} fonts instead of 
the fonts themselves.

Once the fonts are available to your PostScript interpretor,
you need not include them into your PostScript file, and you
can freely distribute your PostScript file. 
Only the people who already have the 
\MathLogo{} fonts will be able to correctly 
view your PostScript file.  This behavior can
be achieved with the following revised mapping file.

\goodbreak

\begin{verbatim}
Math1       Math1
Math1-Bold  Math1-Bold
Math2       Math2
Math2-Bold  Math2-Bold
Math3       Math3 
Math3-Bold  Math3-Bold
Math4       Math4
Math4-Bold  Math4-Bold
Math5       Math5
Math5-Bold  Math5-Bold
Math1Mono      Math1Mono 
Math1Mono-Bold Math1Mono-Bold
Math2Mono      Math2Mono 
Math2Mono-Bold Math2Mono-Bold
Math3Mono      Math3Mono
Math3Mono-Bold Math3Mono-Bold 
Math4Mono      Math4Mono
Math4Mono-Bold Math4Mono-Bold
Math5Mono      Math5Mono
Math5Mono-Bold Math5Mono-Bold
\end{verbatim}
%%%%%%%%%%%%%%%%%%%%%%%%%%%%%%%%%%%%%%%%%%%%%%%%%%%%%%%%%%%%%%%%%%%%%%%
%
If the \MathLogo{} fonts aren't available to your PostScript
interpretor,
\textsf{dvips} must find the  fonts for inclusion and
the environment variable \texttt{DVIPSHEADERS} must include
the \MathLogo{} font directory that contains the Type 1 fonts.

To setup the fonts for GhostScript two things must be done. At first add 
the PostScript fonts to the \textsf{fontmap} file in the GhostScript 
directory. This is done simply by appending the lines:
\begin{verbatim}
/Math1-Bold       (Math1-Bold.pfa);
/Math1            (Math1.pfa);
/Math1Mono-Bold   (Math1Mono-Bold.pfa);
/Math1Mono        (Math1Mono.pfa);
/Math2-Bold       (Math2-Bold.pfa);
/Math2            (Math2.pfa);
/Math2Mono-Bold   (Math2Mono-Bold.pfa);
/Math2Mono        (Math2Mono.pfa);
/Math3-Bold       (Math3-Bold.pfa);
/Math3            (Math3.pfa);
/Math3Mono-Bold   (Math3Mono-Bold.pfa);
/Math3Mono        (Math3Mono.pfa);
/Math4-Bold       (Math4-Bold.pfa);
/Math4            (Math4.pfa);
/Math4Mono-Bold   (Math4Mono-Bold.pfa);
/Math4Mono        (Math4Mono.pfa);
/Math5-Bold       (Math5-Bold.pfa);
/Math5            (Math5.pfa);
/Math5Mono-Bold   (Math5Mono-Bold.pfa);
/Math5Mono        (Math5Mono.pfa);
\end{verbatim}
to the font map. The second step is to tell GhostScript where the fonts can be found.
This is typical done by setting/appending the directory with the \MathLogo{} fonts
(typical \textsf{/usr/local/mathematica/SystemFiles/Fonts/Type1})
to the \verb|GS_LIB| environment variable in Your login shell script.

All the stuff comes in an archive file \textsf{mmafnt.zip}
with a TDS conform structure. Change to the directory where the
\textsf{texmf} directory resides and unzip it with the full directory
information. Modify the \textsf{config.ps} file and 
the \texttt{DVIPSHEADERS} environment variable and this document
should compile with \LaTeXe.

To use the package with your own files add the line
\begin{verbatim}
\usepackage{mmasym}
\end{verbatim}
%
to the preamble of your \LaTeXe{} document.


\section{Typesetting \MathLogo{} Notebooks}

Apart from the production of bitmap free \TeX{} output You may use
mono spaced mathematical fonts. This is of limited interest
in usual mathematical texts but needed  for typesetting the \texttt{In[]} and
\texttt{Out[]} cells of \MathLogo{} notebooks.
For that propose \textsf{mmasym.sty} introduce two new mathematic styles.
The \textsf{mmasym.sty} package has an option \texttt{monospacemath}.
If the package is loaded with:

\begin{verbatim}
\usepackage[monospacemath]{mmasym}
\end{verbatim}

the monospaced fonts will be present and two new mathversions are defined.
For medium weight mono spaced output the math-style \texttt{mono} is 
introduced and for bold mono spaced mathematics the \texttt{monobold}.
By default the mono spaced fonts will not be loaded.
\begin{sloppypar}
Like the \texcmd{boldmath} or \verb|\mathversion{bold}| the command
\texcmd{monomath} or \verb|\mathversion{mono}| and \texcmd{monoboldmath}
or \verb|\mathversion{monobold}| will switch to the new styles. Remember
that the switch must be outside of a mathematical formula.
Typesetting notebooks will need mutch more macros than \textsf{mmasym.sty}
introduces. Here are a few examples	of the different styles.
\end{sloppypar}

\def\half{{1\over 2}}

\def\mathsample{\fbox{\begin{minipage}{0.95\textwidth}
\noindent The Dirac equation for a free particle:
\[ \ii \hbar {\partial \psi\over
              \partial t} = {\hat H}_f \psi = 
	\left( \dsc\, {\hat{\vec \alpha}}\,{\hat{\vec p}} + m_0\, \dsc^2\, {\hat \beta}\right)\psi
\]
\noindent
The integral representation of Bessel function $J_\nu(z)$ :
\begin{eqnarray}
J_\nu(z)&=&{1\over
          \pi}\int_0^\pi \cos( z \sin(\theta)-\nu\,\theta)\,\dd\theta\nonumber\\
		  &&\qquad-
		  {\sin(\nu\pi)\over
		   \pi} \int_0^{\infty} \ee^{z \sinh t - \nu\,t}\,\dd t
		   \,\, (|\arg z|<{1\over 2}\pi)\nonumber
\end{eqnarray}
The expansion of Coulomb wave functions in terms of Bessel-Clifford functions:
\[
F_L(\eta,\varrho)=C_L(\eta) {(2\,L+1)!\over
                             (2\,\eta)^{2\,L+1}} \varrho^{-L} 
							 \sum_{k=2\,L+1}^\infty b_k t^{k/2} I_k(2\sqrt{t})
\]
with $b_{2\,L+1}=1$, $b_{2\,L+2}=0$ and 
$4\eta^2\,(k-2\,L)\,b_{k+1}+k\,b_{k-1} + b_{k-2}=0$.

\noindent A radical idenity:
\[
\sqrt{\half}\cdot\sqrt{\half+\half\,\sqrt{\half}}\cdot
\sqrt{
  \half +\half\sqrt{
    \half+\half\sqrt{\half}
   }
 }\ldots = {2\over \pi}	
\]
\end{minipage}
}
}

\begin{figure}
\mathsample
\caption{The normal mathematics style}
\end{figure}

\begin{figure}
\boldmath\mathsample
\caption{The bold mathematics style}
\end{figure}

\begin{figure}
\monomath\mathsample
\caption{The mono mathematics style}
\end{figure}

\begin{figure}
\monoboldmath\mathsample
\caption{The mono bold mathematics style}
\end{figure}
\mathversion{normal}

\begin{sloppypar}
One difference between the mono spaced output and the fonts used for
``The \MathLogo{} Book'' is that the variables typesetted in italic.
This is the correct behavior for mathematics but it looks strange
for constructs like 
\mathversion{mono}$$Expand[(x^2+1)^{100}]$$\mathversion{normal}
\end{sloppypar}

I recommend definitions like
\begin{verbatim}
\newcommand{\Expand}{{\mathop{\mathrm{Expand}}}}
\end{verbatim}

\newcommand{\Expand}{{\mathop{\mathrm{Expand}}}}

to get \mathversion{mono}$$\Expand[(x^2+1)^{100}].$$\mathversion{normal}



\section{Symbol Names}

The \MathLogo{} names, in most cases, are too long. The \TeX{}
command for a symbol is the name of the 
corresponding AMS font symbol
if it exists, otherwise a \MathLogo{} alias or name is used.
Negated relations start always with the letter \texttt{n} and
the \TeX{}, AMS-\TeX{} name follows. Even if it not explicit
listed in the following tables, an alias due to the \MathLogo{}
naming convention may exist.

\begin{sloppypar}
For the additional alphabets the full \MathLogo{} name, the aliases
of the frontend, the AMS font switching mechanism using \verb|\mathcal{}|
for script, \verb|\mathfrak{}| for \MathLogo{}s gothic, and
\verb|\mathbb{}| for double struck characters, are all working. For
single letters I recommend the alias of the frontend because the
macros for character replacement in the \verb|\mathcal{}|, \verb|mathbb{}|
and \verb|\mathfrak{}| commands are a bit time consuming.
\end{sloppypar}

\begin{thebibliography}{9}
\bibitem{GrComp} Michel Gossens, Sebastian Rahtz, Frank Mittlebach,
                 \textit{The \LaTeX{} Graphics Companion}, Addison-Wesley, 1997
\bibitem{LTeXComp} Michel Gossens, Frank Mittlebach, Alexander Samarin,
                 \textit{The \LaTeX{} Companion}, Addison-Wesley, 1994
\bibitem{TeXBook} Donald E. Knuth, \textit{The \TeX{}book}, Addison-Wesley, 1984
\end{thebibliography}

\appendix

\section{Character Tables}

The following tables give the reference of the defined characters and symbols
when the \textsf{mmasym.sty} package is used. 

\def\charrow#1#2#3{\texcmd{#1} & \texcmd{#2} & $#3$ & {\boldmath $#3$}\\}

\begin{table}
\caption{Additional Characters}
\begin{center}
\begin{tabular}{|c|c|c|c|}
\hline
Name & Alias & normal & bold \\
\hline
\charrow{ee}{ExponetialE}{\ee}
\charrow{ii}{ComplexI}{\ii}
\charrow{jj}{ComplexJ}{\jj}
\charrow{dd}{DifferentialD}{\dd}
\charrow{DD}{CapitalDifferentialD}{\DD}
\charrow{DoublePi}{}{\DoublePi}
\charrow{EulerGamma}{}{\EulerGamma}
\charrow{ScriptDotlessI}{}{\ScriptDotlessI}
\charrow{ScriptDotlessJ}{}{\ScriptDotlessJ}
\hline
\charrow{HBar}{hbar}{\hbar}
\charrow{Mho}{}{\Mho}
\charrow{lambdaslash}{}{\lambdaslash}
\charrow{Angstroem}{}{\Angstroem}
\hline
\charrow{beth}{}{\beth}
\charrow{daleth}{}{\daleth}
\charrow{gimel}{}{\gimel}
\hline
\charrow{Digamma}{}{\Digamma}
\charrow{Stigma}{}{\Stigma}
\charrow{Koppa}{}{\Koppa}
\charrow{Sampi}{}{\Sampi}
\charrow{digamma}{}{\digamma}
\charrow{stigma}{}{\stigma}
\charrow{koppa}{}{\koppa}
\charrow{sampi}{}{\sampi}
\charrow{varkappa}{}{\varkappa}
\charrow{Euler}{euler}{\Euler}
\charrow{Micro}{}{\Micro}
\hline
\end{tabular}
\end{center}
\end{table}

\def\SDCharRow#1#2#3#4{\texcmd{#1} & \texcmd{#2} &
                 $\csname #2\endcsname$  
				 &\texcmd{#3} & \texcmd{#4} &
                 $\csname #4\endcsname$ \\}

\def\DCharRow#1#2#3#4{\texcmd{#1} & \texcmd{#2} &
                 $\csname #2\endcsname$ 
				 & {\boldmath $\csname #1\endcsname$} 
				 &\texcmd{#3} & \texcmd{#4} &
                 $\csname #4\endcsname$ 
				 & {\boldmath $\csname #3\endcsname$}\\}
\begin{table}
\caption{Script Characters, 
the \texcmd{mathcal} can be used to get 
several $\mathcal{Script}$ characters.}
\begin{center}
\begin{tabular}{|c|c|c|c|c|c|}
\hline
\SDCharRow{ScriptCapitalA}{scA}{ScriptA}{sca}
\SDCharRow{ScriptCapitalB}{scB}{ScriptB}{scb}
\SDCharRow{ScriptCapitalC}{scC}{ScriptC}{scc}
\SDCharRow{ScriptCapitalD}{scD}{ScriptD}{scd}
\SDCharRow{ScriptCapitalE}{scE}{ScriptE}{sce}
\SDCharRow{ScriptCapitalF}{scF}{ScriptF}{scf}
\SDCharRow{ScriptCapitalG}{scG}{ScriptG}{scg}
\SDCharRow{ScriptCapitalH}{scH}{ScriptH}{sch}
\SDCharRow{ScriptCapitalI}{scI}{ScriptI}{sci}
\SDCharRow{ScriptCapitalJ}{scJ}{ScriptJ}{scj}
\SDCharRow{ScriptCapitalK}{scK}{ScriptK}{sck}
\SDCharRow{ScriptCapitalL}{scL}{ScriptL}{scl}
\SDCharRow{ScriptCapitalM}{scM}{ScriptM}{scm}
\SDCharRow{ScriptCapitalN}{scN}{ScriptN}{scn}
\SDCharRow{ScriptCapitalO}{scO}{ScriptO}{sco}
\SDCharRow{ScriptCapitalP}{scP}{ScriptP}{scp}
\SDCharRow{ScriptCapitalQ}{scQ}{ScriptQ}{scq}
\SDCharRow{ScriptCapitalR}{scR}{ScriptR}{scr}
\SDCharRow{ScriptCapitalS}{scS}{ScriptS}{scs}
\SDCharRow{ScriptCapitalT}{scT}{ScriptT}{sct}
\SDCharRow{ScriptCapitalU}{scU}{ScriptU}{scu}
\SDCharRow{ScriptCapitalV}{scV}{ScriptV}{scv}
\SDCharRow{ScriptCapitalW}{scW}{ScriptW}{scw}
\SDCharRow{ScriptCapitalX}{scX}{ScriptX}{scx}
\SDCharRow{ScriptCapitalY}{scY}{ScriptY}{scy}
\SDCharRow{ScriptCapitalZ}{scZ}{ScriptZ}{scz}
\hline
\end{tabular}
\end{center}
\end{table}

\begin{table}
\caption{Double Struck Characters, 
the \texcmd{mathbb} can be used to get 
several $\mathbb{DoubleStruck}$ characters.}
\begin{center}
\begin{tabular}{|c|c|c|c|c|c|}
\hline
\SDCharRow{DoubleStruckCapitalA}{dsA}{DoubleStruckA}{dsa}
\SDCharRow{DoubleStruckCapitalB}{dsB}{DoubleStruckB}{dsb}
\SDCharRow{DoubleStruckCapitalC}{dsC}{DoubleStruckC}{dsc}
\SDCharRow{DoubleStruckCapitalD}{dsD}{DoubleStruckD}{dsd}
\SDCharRow{DoubleStruckCapitalE}{dsE}{DoubleStruckE}{dse}
\SDCharRow{DoubleStruckCapitalF}{dsF}{DoubleStruckF}{dsf}
\SDCharRow{DoubleStruckCapitalG}{dsG}{DoubleStruckG}{dsg}
\SDCharRow{DoubleStruckCapitalH}{dsH}{DoubleStruckH}{dsh}
\SDCharRow{DoubleStruckCapitalI}{dsI}{DoubleStruckI}{dsi}
\SDCharRow{DoubleStruckCapitalJ}{dsJ}{DoubleStruckJ}{dsj}
\SDCharRow{DoubleStruckCapitalK}{dsK}{DoubleStruckK}{dsk}
\SDCharRow{DoubleStruckCapitalL}{dsL}{DoubleStruckL}{dsl}
\SDCharRow{DoubleStruckCapitalM}{dsM}{DoubleStruckM}{dsm}
\SDCharRow{DoubleStruckCapitalN}{dsN}{DoubleStruckN}{dsn}
\SDCharRow{DoubleStruckCapitalO}{dsO}{DoubleStruckO}{dso}
\SDCharRow{DoubleStruckCapitalP}{dsP}{DoubleStruckP}{dsp}
\SDCharRow{DoubleStruckCapitalQ}{dsQ}{DoubleStruckQ}{dsq}
\SDCharRow{DoubleStruckCapitalR}{dsR}{DoubleStruckR}{dsr}
\SDCharRow{DoubleStruckCapitalS}{dsS}{DoubleStruckS}{dss}
\SDCharRow{DoubleStruckCapitalT}{dsT}{DoubleStruckT}{dst}
\SDCharRow{DoubleStruckCapitalU}{dsU}{DoubleStruckU}{dsu}
\SDCharRow{DoubleStruckCapitalV}{dsV}{DoubleStruckV}{dsv}
\SDCharRow{DoubleStruckCapitalW}{dsW}{DoubleStruckW}{dsw}
\SDCharRow{DoubleStruckCapitalX}{dsX}{DoubleStruckX}{dsx}
\SDCharRow{DoubleStruckCapitalY}{dsY}{DoubleStruckY}{dsy}
\SDCharRow{DoubleStruckCapitalZ}{dsZ}{DoubleStruckZ}{dsz}
\hline
\end{tabular}
\end{center}
\end{table}

\begin{table}
\caption{Gothic Characters, 
the \texcmd{mathfrak} can be used to get
$\mathfrak{Gothic}$ characters.}
\begin{center}
\begin{tabular}{|c|c|c|c|c|c|c|c|}
\hline
\DCharRow{GothicCapitalA}{goA}{GothicA}{goa}
\DCharRow{GothicCapitalB}{goB}{GothicB}{gob}
\DCharRow{GothicCapitalC}{goC}{GothicC}{goc}
\DCharRow{GothicCapitalD}{goD}{GothicD}{god}
\DCharRow{GothicCapitalE}{goE}{GothicE}{goe}
\DCharRow{GothicCapitalF}{goF}{GothicF}{gof}
\DCharRow{GothicCapitalG}{goG}{GothicG}{gog}
\DCharRow{GothicCapitalH}{goH}{GothicH}{goh}
\DCharRow{GothicCapitalI}{goI}{GothicI}{goi}
\DCharRow{GothicCapitalJ}{goJ}{GothicJ}{goj}
\DCharRow{GothicCapitalK}{goK}{GothicK}{gok}
\DCharRow{GothicCapitalL}{goL}{GothicL}{gol}
\DCharRow{GothicCapitalM}{goM}{GothicM}{gom}
\DCharRow{GothicCapitalN}{goN}{GothicN}{gon}
\DCharRow{GothicCapitalO}{goO}{GothicO}{goo}
\DCharRow{GothicCapitalP}{goP}{GothicP}{gop}
\DCharRow{GothicCapitalQ}{goQ}{GothicQ}{goq}
\DCharRow{GothicCapitalR}{goR}{GothicR}{gor}
\DCharRow{GothicCapitalS}{goS}{GothicS}{gos}
\DCharRow{GothicCapitalT}{goT}{GothicT}{got}
\DCharRow{GothicCapitalU}{goU}{GothicU}{gou}
\DCharRow{GothicCapitalV}{goV}{GothicV}{gov}
\DCharRow{GothicCapitalW}{goW}{GothicW}{gow}
\DCharRow{GothicCapitalX}{goX}{GothicX}{gox}
\DCharRow{GothicCapitalY}{goY}{GothicY}{goy}
\DCharRow{GothicCapitalZ}{goZ}{GothicZ}{goz}
\hline
\end{tabular}
\end{center}
\end{table}


\def\samplerow#1#2#3{\texcmd{#1} & $#2$ & $#3$ & $\displaystyle#3$\\ 
&  &\mathversion{mono} $#3$ &\mathversion{mono} $\displaystyle#3$ \\}
\begin{table}
\caption{Integral signs}
\begin{center}
\begin{tabular}{|l|l|c|c|}
\hline
\TeX-Command  &  &  & \\
\hline
\samplerow{int}{\int}{\int_{a}^{b} f(x)\,\dd x}
\samplerow{oint}{\oint}{\oint_{C} f(\zeta)\,\dd \zeta}
\samplerow{dbloint}{\dbloint}{\dbloint_{\Gamma} f(u,v)\, \dd u\, \dd v}
\samplerow{clockoint}{\clockoint}{\clockoint_{\Gamma} f(z)\, \dd z}
\samplerow{cntclockoint}{\cntclockoint}{\cntclockoint_{\Gamma} f(z)\, \dd z}
\samplerow{sqrint}{\sqrint}{\sqrint_{\Gamma} f(z)\, \dd z}
\samplerow{fint}{\fint}{\fint_{-\infty}^{\infty} {f(x)\over x} \, \dd x}
\hline
\end{tabular}
\end{center}
\end{table}

\def\samplearrows#1{%
\texcmd{#1} & $\csname#1\endcsname$ & $a\csname#1\endcsname b$ \\}
\def\samplevarrows#1{%
\texcmd{#1} & $\csname#1\endcsname$ & $\bigm{\csname#1\endcsname}$ 
                                       $\Bigm{\csname#1\endcsname}$
									   $\biggm{\csname#1\endcsname}$ 
									   $\Biggm{\csname#1\endcsname}$ \\}
\begin{table}
\caption{Additional and changed arrows 1}
\begin{center}
\begin{tabular}{|c|c|c|}
\hline
\samplearrows{HookLeftArrow}
\samplearrows{HookRightArrow}
\samplearrows{MapsTo}
\samplearrows{MapsFrom}
\samplevarrows{MapsUp}
\samplevarrows{MapsDown}
\hline
\samplearrows{ShortUpArrow}
\samplearrows{ShortDownArrow}
\samplearrows{ShortRightArrow}
\samplearrows{ShortLeftArrow}
\samplearrows{LongLeftArrow}
\samplearrows{longleftarrow}
\samplearrows{LongRightArrow}
\samplearrows{longrightarrow}
\samplearrows{LongLeftRightArrow}
\samplearrows{longleftrightarrow}
\hline
\samplearrows{DblLongLeftArrow}
\samplearrows{Longleftarrow}
\samplearrows{DblLongRightArrow}
\samplearrows{Longrightarrow}
\samplearrows{DblLongLeftRightArrow}
\samplearrows{Longleftrightarrow}
\hline
\end{tabular}
\end{center}
\end{table}

\begin{table}
\caption{Additional and changed arrows 2}
\begin{center}
\begin{tabular}{|c|c|c|}
\hline
\samplearrows{RightVectorBar}
\samplearrows{LeftVectorBar}
\samplearrows{DownRightVectorBar}
\samplearrows{DownLeftVectorBar}
\samplearrows{RightTeeVector}
\samplearrows{LeftTeeVector}
\samplearrows{DownRightTeeVector}
\samplearrows{DownLeftTeeVector}
\samplearrows{RightArrowBar}
\samplearrows{LeftArrowBar}
\samplearrows{leftrightharpoonup}
\samplearrows{leftrightharpoondown}
\samplearrows{equilibrium}
\samplearrows{revequilibrium}
\samplearrows{Equilibrium}
\samplearrows{RevEquilibrium}
\samplevarrows{upharpoonleftup}
\samplevarrows{upharpoonleftdown}
\samplevarrows{upharpoonrightup}
\samplevarrows{upharpoonrightdown}
\samplevarrows{leftupdownharpoon}
\samplevarrows{rightupdownharpoon}
\samplevarrows{UpArrowBar}
\samplevarrows{DownArrowBar}
\hline
\end{tabular}
\end{center}
\end{table}

\begin{table}
\caption{Additional and changed arrows 3}
\begin{center}
\begin{tabular}{|c|c|c|}
\hline						 
\samplevarrows{LeftUpTeeVector}
\samplevarrows{RightUpTeeVector}
\samplevarrows{LeftDownTeeVector}
\samplevarrows{RightDownTeeVector}
\samplevarrows{LeftUpVectorBar}
\samplevarrows{RightUpVectorBar}
\samplevarrows{LeftDownVectorBar}
\samplevarrows{RightDownVectorBar}
%
\samplevarrows{upequilibrium}
\samplevarrows{uprevequilibrium}
\hline
\samplearrows{rightleftarrow}
\samplearrows{leftrightarrow}
\hline
\samplevarrows{uparrowdownarrow}
\samplevarrows{downarrowuparrow}
\hline
\end{tabular}
\end{center}
\end{table}

\def\samplexarrows#1#2{%
\texcmd{#1}\textit{[length]} & $\csname#1\endcsname$ & $a\csname#1\endcsname[#2] b$ \\}
\begin{table}
\caption{Extensible Horizontal Arrows. 
All arrows have a length argument.}
\begin{center}
\begin{tabular}{|c|c|c|}
\hline
\TeX command & Symbol & Example \\
\hline		   
\samplexarrows{RightArrowFill}{24pt}
\samplexarrows{LeftArrowFill}{24pt}
\samplexarrows{LRArrowFill}{24pt}
\samplexarrows{DblRightArrowFill}{24pt}
\samplexarrows{DblLeftArrowFill}{24pt}
\samplexarrows{DblLRArrowFill}{24pt}
\samplexarrows{RightHarpoonUpFill}{24pt}
\samplexarrows{LeftHarpoonUpFill}{24pt}
\samplexarrows{RightHarpoonDownFill}{24pt}
\samplexarrows{LeftHarpoonDownFill}{24pt}
\samplexarrows{LRHarpoonUpFill}{24pt}
\samplexarrows{LRHarpoonDownFill}{24pt}
\samplexarrows{EquilibriumFill}{24pt}
\samplexarrows{RevEquilibriumFill}{24pt}
\samplexarrows{RightLeftArrowFill}{24pt}
\samplexarrows{LeftRightArrowFill}{24pt}
\hline
\end{tabular}
\end{center}
\end{table}

\begin{table}
\caption{Mathematica specials}
\begin{center}
\begin{tabular}{|c|c|c|}
\hline
\samplearrows{Rule}
\samplearrows{RuleDelayed}
\samplearrows{SetDelayed}
\samplearrows{Equal}
\samplearrows{Same}
\hline
\multicolumn{3}{|c|}{Double brackets}\\
\hline
\samplevarrows{lpart}
\samplevarrows{rpart}
\samplevarrows{llbracket}
\samplevarrows{rrbracket}
\hline
\end{tabular}
\end{center}
\end{table}

\begin{table}
\caption{Dot's as time derivative}
\begin{center}
\begin{tabular}{|l|c|c|}
\hline
\texcmd{Dot}   & ${\Dot a}(t)$   & {\boldmath ${\Dot a}(t)$}\\
\texcmd{DDot}  & ${\DDot a}(t)$  & {\boldmath ${\DDot a}(t)$}\\
\texcmd{DDDot} & ${\DDDot a}(t)$ & {\boldmath ${\DDDot a}(t)$}\\
\texcmd{vec} & ${\vec A}$    & {\boldmath ${\vec A}$}\\
\texcmd{lrvec} & ${\lrvec A}$    & {\boldmath ${\lrvec A}$}\\
\texcmd{lvec} & ${\lvec A}$    & {\boldmath ${\lvec A}$}\\
\texcmd{Vec} & ${\Vec A}$    & {\boldmath ${\Vec A}$}\\
\texcmd{LRVec} & ${\LRVec A}$    & {\boldmath ${\LRVec A}$}\\
\texcmd{LVec} & ${\LVec A}$    & {\boldmath ${\LVec A}$}\\
\hline
\end{tabular}
\end{center}
\end{table}


\def\sampleoverunder#1{
\texcmd{#1}$\{$\textit{argument}$\}$ 
            & $\csname#1\endcsname{a+b}$ 
            & $\csname#1\endcsname{a+b+c}$
			& $\csname#1\endcsname{a+b+x+y}$ \\}
\begin{table}
\caption{Over- and underbraces, brackets \ldots}
\begin{center}
\begin{tabular}{|l|c|c|c|}
\hline
\sampleoverunder{overparen}
\sampleoverunder{underparen}
\hline
\sampleoverunder{overbracket}
\sampleoverunder{underbracket}
\sampleoverunder{OverBracket}
\sampleoverunder{UnderBracket}
\hline
\sampleoverunder{overbrace}
\sampleoverunder{underbrace}
\hline
\sampleoverunder{overleftarrow}
\sampleoverunder{overrightarrow}
\hline
\sampleoverunder{overleftharpoon}
\sampleoverunder{overrightharpoon}
\sampleoverunder{overlrharpoon}
\hline
\end{tabular}
\end{center}
\end{table}

\def\relrow#1{\texcmd{#1} & $ a\csname #1\endcsname b$ \\}

\begin{table}
\caption{Relations and negated binary relations 1}
\begin{center}
\begin{tabular}{|c|c|}
\hline
\relrow{therefore}
\relrow{because}
\relrow{Proportion}
\relrow{neq}
\relrow{dotequal}
\relrow{nasymp}
\relrow{nequiv}
\relrow{nsupseteq}
\relrow{nsubseteq}
\relrow{nsqsupseteq}
\relrow{nsqsubseteq}
\relrow{nleq}
\relrow{ngeq}
\relrow{npreceq}
\relrow{nsucceq}
\relrow{nsim}
\relrow{cong}
\relrow{ncong}
\relrow{napprox}
\relrow{nsubset}
\relrow{nsupset}
\relrow{nll}
\relrow{ngg}
\relrow{nprec}
\relrow{nsucc}
\relrow{nin}
\relrow{nni}
\relrow{nless}
\relrow{ngtr}
\relrow{bumpeq}
\relrow{Bumpeq}
\relrow{nbumpeq}
\relrow{nBumpeq}
\hline
\end{tabular}
\end{center}
\end{table}

\begin{table}
\caption{Relations and negated binary relations 2}
\begin{center}
\begin{tabular}{|c|c|}
\hline
\relrow{unlhd}
\relrow{unrhd}
\relrow{nunlhd}
\relrow{nunrhd}
\relrow{backepsilon}
\relrow{TildeEqual}
\relrow{NotTildeEqual}
\relrow{NestedLessLess}
\relrow{NotNestedLessLess}
\relrow{NestedGreaterGreater}
\relrow{NotNestedGreaterGreater}
\relrow{GreaterLess}
\relrow{NotGreaterLess}
\relrow{GreaterTilde}
\relrow{LessTilde}
\relrow{NotGreaterTilde}
\relrow{NotLessTilde}
\relrow{PrecedesSlantEqual}
\relrow{SucceedsSlantEqual}
\relrow{NotPrecedesSlantEqual}
\relrow{NotSucceedsSlantEqual}
\relrow{PrecedesTilde}
\relrow{SucceedsTilde}
\relrow{NotPrecedesTilde}
\relrow{NotSucceedsTilde}
\relrow{RightTriangle}
\relrow{LeftTriangle}
\relrow{NotRightTriangle}
\relrow{NotLeftTriangle}
\relrow{RightTriangleBar}
\relrow{LeftTriangleBar}
\relrow{NotRightTriangleBar}
\relrow{NotLeftTriangleBar}
\hline
\end{tabular}
\end{center}
\end{table}


\begin{table}
\caption{Angle}
\begin{center}
\begin{tabular}{|c|c|c|c|}
\hline
Name & Alias & normal & bold \\
\hline
\charrow{Angle}{angle}{\Angle}
\charrow{rightangle}{RightAngle}{\rightangle}
\charrow{measuredangle}{MeasuredAngle}{\measuredangle}
\charrow{sphericalangle}{SphericalAngle}{\sphericalangle}
\hline
\end{tabular}
\end{center}
\end{table}

\def\texsymrow#1#2{\texcmd{#1} & \csname#1\endcsname &\texcmd{#2} & \csname#2\endcsname\\}

\begin{table}
\caption{Text symbols, the text symbols are 
all defined with a closing \texcmd{xspace}.}
\begin{center}
\begin{tabular}{|c|c||c|c|}
\hline
\texsymrow{MathLogo}{MathIcon}
\texsymrow{KernelIcon}{Wolf}
\texsymrow{WatchIcon}{LightBulb}
\texsymrow{HappySmiley}{NeutralSmiley}
\texsymrow{SadSmiley}{FreakedSmiley}
\texsymrow{WaringSign}{AliasDelimiter}
\hline
\texsymrow{CommandKey}{ControlKey}
\texsymrow{AltKey}{ModeOneKey}
\texsymrow{ModeTwoKey}{CloverLeaf}
\texsymrow{ReturnIndicator}{DottedSquare}
\texsymrow{LeftModfied}{RightModfied}
%
\texsymrow{EscapeKey}{ReturnKey}
\texsymrow{ShiftKey}{SpaceKey}
\texsymrow{BackspaceKey}{HomeKey}
\texsymrow{PageUpKey}{PageDownKey}
\texsymrow{EndKey}{TabKey}
%
\hline
\texcmd{DownQuestion} & \DownQuestion & & \\
\hline
\end{tabular}
\end{center}
\end{table}

\def\mathtextrow#1{\texcmd{#1} & \csname#1\endcsname & $\csname#1\endcsname$ \\}
\begin{table}
\caption{Symbols that exists in math and in text mode.}
\begin{center}
\begin{tabular}{|c|c|c|}
\hline
\mathtextrow{SpaceIndicator}
\mathtextrow{RoundSpaceIndicator}
\mathtextrow{Continuation}
\mathtextrow{ErrorIndicator}
\mathtextrow{UnknownGlyph}
\mathtextrow{SelectionPlaceholder}
\mathtextrow{Placeholder}
\mathtextrow{SixPointedStar}
\mathtextrow{Rectangle}
\mathtextrow{GrayRectangle}
\mathtextrow{EmptyRectangle}
\mathtextrow{Square}
\mathtextrow{GraySquare}
\mathtextrow{EmptySquare}
\mathtextrow{Circle}
\mathtextrow{GrayCircle}
\mathtextrow{EmptyCircle}
\mathtextrow{Ellipsis}
\mathtextrow{CenterEllipsis}
\mathtextrow{VerticalEllipsis}
\mathtextrow{AscendingEllipsis}
\mathtextrow{DescendingEllipsis}
\hline
\end{tabular}
\end{center}
\end{table}
\clearpage
\section{Bug Reports}

I have still some free positions in the virtual fonts. All users
are asked to contribute requests for symbols from the PostScript
fonts that are missed in the \TeX{} fonts. 

The virtual fonts (spacing, italic correction, placement of super- and
subscripts, \ldots) as well as the style file \texttt{mmasym.sty} may
still have some errors. Please report this errors via e-mail to:

\begin{verbatim}
kuska@osf1.mpae.gwdg.de
\end{verbatim}

Please start the subject with the string \verb|MMA Font Bug| and attatch
a \LaTeXe{} file that shows the error. I will try to fix the errors
as soon as posible.

\section{Copyright}

The copyright of the Type 1 PostSript fonts and the
afm-files belongs to Wolfram Research Inc. The copygright of
all files in the ``\MathLogo{} Virtual Font Package'' belongs
to Jens-Peer Kuska.

The files may be freely copied, distributed and used, 
provided that no changes whatsoever are made. 
All users are asked to help keep the virtual font files 
and the style consistent and ``uncorrupted,''
identical everywhere in the world. 
Changes are permissible only
if the modified file is given a new name, 
different from the names of existing files, 
and only if the modified file is clearly 
identified as not being part of the ``\MathLogo{} 
Virtual Font Package''.
    
The author has tried his best to produce correct 
and useful macros and fonts, in order to 
help promote computer science research and
\MathLogo, but no warranty of any kind should 
be assumed.
\end{document} 


