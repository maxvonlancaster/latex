\documentclass[12pt]{article}
\usepackage{mxedruli,xucuri}
\parindent0pt
\parskip1.5ex
\pagestyle{empty}

\begin{document}
\section*{Georgian Writing Systems}
The following shows the same Text in the two Alphabets
used in Georgian, {\it Mxedruli\/} and {\it Xucuri\/}.
Both variants of {\it Xucuri\/} (Majuscules or {\it Mrg(v)lovani\/}
and minuscules or {\it \d Kutxovani\/}) are demonstrated here.

\subsection*{\emph{Mxedruli} \mxedb mxedruli}


\begin{itemize}
\item From: Bedi Kartlisa, 43 (1963) p. 17:

\begin{mxedr}
naxes ucxo moqme vinme, +jda m.tirali .cqlisa .pirsa,\\
+savi cxeni sadavata hqva lomsa da vita gmirsa,\\
.k+sirad esxa margali.ti lagam-ab+jar-unagirsa,\\
cremlsa vardi daetrtvila, gulsa mdu.grad ana.tirsa.
\end{mxedr}
\end{itemize}

\subsection*{\emph{Xucuri -- Mrgvlovani} \xucr XUCURI -- MRGVLOVANI}


\begin{itemize}
\item Form: Bedi Kartlisa, 43 (1963) p. 17:

\begin{xucr}
NAXES UCXO MOQME VINME, +JDA M.TIRALI .CQLISA .PIRSA,\\
+SAVI CXENI SADAVATA HQVA LOMSA DA VITA GMIRSA,\\
.K+SIRAD ESXA MARGALI.TI LAGAM-AB+JAR-UNAGIRSA,\\
CREMLSA VARDI DAETRTVILA, GULSA MDU.GRAD ANA.TIRSA.
\end{xucr}

\item From: Nikolai Marr and Maurice Bri\`ere, La Langue G\'eorgienne, Paris
1931, S. 595

\begin{xucr}

\begin{center}
SAXAREBAI1 MATEES TAVISAI1.

B
\end{center}


{\rm 1.} --- XOLO IESOW KRIS.TEES +SUBASA BETLEMS
HOWRIAS\-.TANISASA. D.GETA HERODE MEPISATA. AHA MOGOWNI A.GMO\-SAVALIT
MOVIDES IEROWSALEEMD DA I.TQODES:

{\rm 2.} --- SADA ARS ROMELI IGI I+SVA. MEOWPEE HOWRIATAI1? RAI1RETOW
VIXILET VARS.KOWLAVI MISI A.GMOSAVALIT DA MOVEDIT TAVQOWANIS--CEMAD
MISA.

{\rm 3.} --- VITARCA ESMA ESE HERODES MEPESA. +SEJR.COWNDA DA +SOVELI
IEROWSALEEMI MISTANA.

\end{xucr}
\end{itemize}

\subsection*{\emph{Xucuri -- \d  Kutxovani} \xucr xucuri -- .kutxovani}


From: {\it N. Marr and M. Bri\`ere, La Langue G\'eor\-gienne,
Paris 1931\/}, p. 599

\begin{xucr}
\begin{center}\Large
saxarebai1 lu.kai1s tavisai1.

ie
\end{center}

{\rm 11.} --- merme i.tqoda da tkua: .kacsa visme escnes
or je.

{\rm 12.} --- da hrkua umr.cemesman man mamasa twssa:
mamao. momec me romeli mxudebis na.cilidam.kw drebelisa. da ganuqo
mat sacxovrebeli igi.
\end{xucr}



\end{document}
