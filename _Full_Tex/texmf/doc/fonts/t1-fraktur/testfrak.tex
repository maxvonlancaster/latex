
\documentclass[12pt]{article}
\usepackage{mf-frak}
\ifx\german\undefined \def\german{\language=7}\fi
\def\ital#1{{\slshape #1\/}}
\def\Ital#1{{\bfseries\slshape #1\/}}
\def\3{\ss}
\def\<{``}
\def\>{''}
\def\Pageref{p.~\pageref}
\def\medrule{\par\medskip\centerline{\vrule width 0.5\hsize}\par\medskip}
\def\ecole{{\sf\bfseries SPORT{}}}
\def\FFME{{\SetCmDefault\rm FFME{}}}
\def\COSIROC{{\SetCmDefault\rm COSIROC{}}}
\begin{document}
\german

%\SetCmDefault

\title{\Large\bf Das| Klettern in Frankreich}
\author{\SetCmDefault\sl D. Taupin}
\date{{\SetCmDefault\bf 16 juillet 2002}}
\date{{\SetCmDefault\rm 16 juillet 2002}}
\maketitle

{\huge (a) (e)}

\section{Was| gibt es| neu?} \rm

\begin{enumerate}
 \item Der Hauptheraus|geber dieser sechsten Aus|gabe des| \ital{Guide des|
sites| naturels| d'escalade} ist jetzt die \FFME{} (F\'ed\'eration fran\c
caise de la montagne et de l'escalade), zusammen mit dem \COSIROC, nicht
anders| herum. Hier wurde der Tatsache Rechnung getragen, das{}s| im Zuge der
Weiterentwicklung des| Klettersports| immer mehr die Departement-Komitees|
sowie die regionalen Komitees| der \FFME{} die Klettergebiete erschlossen
haben und uns| die Informationen liefern.

 \item Die Anzahl der Klettergebiete steigt weiterhin, auch wenn man momentan
eher anstrebt, die vorhandenen Gebiete zu erweitern oder dichter einzubohren,
als| neue zu schaffen, die ihrerseits| Probleme aufwerfen; sei es| mit den
Eigent\"umern, mit den zust\"andigen Verwaltungen, oder den besorgten Umwelt-
oder Sicherheitsbeauftragten.

 \item Abgesehen vom Hochgebirge werden immer noch viele, einstmals| alpine,
Touren eingebohrt, aber sicher sportlicher als| in den Klettergarten, und
stellen dadurch ein relativ unberechenbares| Zwischending zwischen dem
\"ubersicherten Klettergarten und unabgesichertem Gel\"ande dar. Aus| diesem
Grund mussten wir die Aus|wahlkriterien f\"ur die in diesem F\"uhrer genannten
Gebiete \"uberarbeiten:
 \begin{itemize}
 \item Wir f\"uhren alle Felsen --- egal ob alpines| oder Sportklettern ---,
unter 1600~m H\"ohe auf, von denen es| bekannt ist, das{}s| sie beklettert wurden,
oder das{}s| sie kletterbar sind.
 \item Dar\"uber hinaus| f\"uhren wir die Gebiete auf, die aus|reichend
erschlossen sind, um mit Hilfe vom Klemmkeilen beklettert zu werden, daf\"ur
aber keine, f\"ur die Bohrmaschine und Haken erforderlich sind.
 \item Auch wenn man teilweise Pl\"attchen oder zementierte Haken findet, haben
wir absichtlich das| Hochgebirge weggelassen, also Klettereien, f\"ur die auch im
Sommer Eispickel, Steigeisen, Bergschuhe erforderlich sind, auch f\"ur Zu- oder
Abstieg.

 \end{itemize}

 \item Ein anderes| wichtiges| und gleicherma\3en schlimmes| Ph\"anomen ist
der Befall unseres| Landes| --- oder praktisch ganz Westeuropas| --- von
dieser verh\"angnis|vollen amerikanischen Manie, bei jedem Unfall, der sich
ereignet, einen Verantwortlichen zu suchen, bzw. sogar einen Schuldigen. Das|
zieht f\"ur unsere Sportart, wie im \"ubrigen f\"ur alle Natursportarten,
mehrere nachteilige Konsequenzen nach sich:


 \begin{itemize}
 \item viele Anf\"anger im Klettern, Wandern, Skifahren oder Bergsteigen
entziehen sich
jegliche Verantwortung: welche Dummheit ich auch immer begehe, der
Schuldige ist der ``andere'', der F\"uhrer, der Kletterlehrer, der
B\"urgermeister, der Pr\"afekt, der Grundst\"uckseigent\"umer, der Kollege... oder der
F\"uhrerautor;

 \item all das| besch\"aftigt die Juristen, vor allem aber erh\"oht das| die
Versicherungs|kosten, die unerschwinglich werden f\"ur junge Leute, Leute mit
niedrigen Einkommen und f\"ur all die kleinen Gemeinden, in deren
Einzugsbereich es| ungl\"ucklicherweise ein Klettergebiet gibt, bzw. H\"ohlen
oder Canyoningm\"oglichkeiten;

 \item es| kommt noch schlimmer: die Verwundeten oder Angeh\"origen der
Unfallopfer geben sich nicht damit zufrieden, Schadenersatz zu
verlangen, der von der Haftptlichtversicherung des| mutma\3lichen
Verantwortlichen gezahlt wird, sie klagen denselben wahrhaftiger
Verbrechen an wie fahrl\"assige K\"orperverletzung, Gef\"ahrdung fremden
Lebens|, Totschlag, usw.; und hier entscheidet nicht mehr das|
Zivilgericht, sondem das| Strafgericht, mit hohen Geldstrafen --- die niemals|
durch die Versicherungen abgedeckt sind --- oder sogar Gef\"angnisstrafen.
  \end{itemize}

\end{enumerate}

 Die Reaktion des| potentiellen Schuldigen liegt in dem Fall auf der Hand:
drakonische und damit abschreckende Bedingungen auferlegen (was| der
Fall ist bei den Clubs| oder Freizeitorganisationen), oder schlichtweg Verbote
erteilen (der Fall der B\"urgermeister, Eigent\"umer oder Pr\"afekten).
  
\section{Unsere juristischen Vorkehrungen}

 Im Artikel 17 des| Gesetzes| zum Sport von 1984 steht: \<In jeder sportlichen
Disziplin und f\"ur einen abgegrenzten Zeitraum erh\"alt ein einziger Verband den
Auftrag des| Ministers| f\"ur Sport, sportliche Wettk\"ampfe zu organisieren (...)
und die entsprechenden Qualifikationen durchzuf\"uhren. Dieser Verband
definiert, unter Achtung der internationalen Reglements|, die seiner Sportart
eigenen technischen Regeln.\>
   
 Der beauftragte Verband, der unter anderem f\"ur den Klettersport und den
Alpinismus| diese technischen Regeln definiert hat, ist die \FFME. Sie hat daher
1993 eine gewisse Anzahl an Sicherheits|vorschriften  (siehe \ital{Consignes|
f\'ed\'erales| de s\'ecurit\'e} im franz\"osischen Text, \Pageref{consfede})
festgelegt; Vorschriften, deren  Mi\3achtung einen technischen Fehler
darstellt, f\"ur den der unvorsichtige Aus|\"ubende selbst voll verantwortlich
ist und in gleichem Ma\3 der \<andere Schuldige\>, den manche zu suchen versucht
sind, nicht verantwortlich.

 Zusammengefasst bedeuten diese Vorschriften, das{}s| der Kletterer die
Aus|r\"ustung, die Qualit\"at des| Gesteins|, die Wetterverh\"altnisse des|
Kletterortes| selbtst \"uberpr\"ufen mus{}s|, entweder durch visuelle Beobachtung,
mit Hilfe der Literatur oder anderer Informationsquellen, deren
Zuverl\"assigkeit er selbst absch\"atzen muss. Wenn all da ungen\"ugend ist,
soll er selbstverst\"andlich verzichten oder absteigen.

 Weitere Vorsichtregeln sollen hier noch gegeben werden: jeder, der sich ohne
Aus|bildung oder vorangegangenes| Training auf ein Gel\"ande von mittlerer
Steigung \"uber 50~\% (30 Grad) begibt, der sich zu Fu\3 einem Felsmassiv auf
einen Abstand, weniger als| die halbe H\"ohe des| Massivs|, n\"ahert oder der sich
auf dem Gipfel in wehiger als| 5~m Abstand von der Abbruchkante aufh\"alt
(aus|genommen touristische Einrichtung), handelt \<fahrl\"assig\>. Diese
Fahrl\"assigkeit beinhaltet damit die eigene Verantwortung in dem Ma\3e wie der
Zugang zu diesem Ort nicht einen Notwendigkeits|charakter hatte wie z.B. der
Weg zur Arbeits|st\"atte, zur Wohnortst\"atte oder auch zu einem Freizeitort
(Aktivit\"aten in freier Natur aus|genommen).


\section{Die beschriebenen Kletterorte}

Mit dem Aus|druck \<nat\"urliche Klettergebiete\> meinen wir nicht nur die
Kletterfelsen, die sich aus| Naturwirkung ergeben haben, sondern auch gewisse
Bauwerke, die urspr\"unglich nicht f\"ur das| Klettern
bestimmt waren (z.b.: Eisenbahn- und Festungwerke, D\"amme, Steinbr\"uche).
Wir haben erw\"agt, das{}s| die k\"unstlichen Kletterw\"ande eines| besonderen
F\"uhrers| w\"urdig seien.

 Die Klettergebiete wurden in vier Kategorien unterteilt:

 \begin{enumerate}

 \item Bezeichnung \<Site sportif d'Escalade\> (Klettergarten) f\"ur eine
be\-grenz\-te An\-zahl von Felsenw\"anden, die folgende Kriterien
erf\"ullen: Instand\-haltung, Sicherheit, schriftliche Unterlagen usw.

 \item Bezeichnung \<Initiation\> (Anf\"anger\-klettergarten) f\"ur eine noch
begrenzte Anzahl von Klettergebieten, die besonders| die
Sicherheits|\-erfordernisse der Kletter\-beginner erf\"ullen, und die eine
gute Aus|wahl von Beginner\-routen (3 bis| 5) bieten.

 \item Im Gegensatz dazu hei\3en \<Terrains| d'Aventure\> andere
Klettergebiete, die nicht diese Kriterien erf\"ullen, und wo der Kletterer auf
alles| gefasst sein muss: keine Haken oder eine gro\3e Anzahl von Haken in
schlechtem Zustand, irrige Routenf\"uhrer, verschrobene Schwierigkeitskalen
usw.

 \item Schlie\3lich wurden als| \<Blocs\> (Boulder) Felsbl\"ocke von geringer
H\"ohe be\-zeich\-net, f\"ur die keinerlei Aus|r\"ustung ben\"otigt wird und
auf die die oben beschriebene Unterscheidung nicht angewandt werden kann.

 \end{enumerate}
\section{Einige Empfehlungen} \rm

 Der Klettersport hat die Besonderheit, das{}s| er an Orten
betrieben wird, die von normalen Wohn- und Arbeits|gebieten entfernt
liegen und die sich von ihnen unterscheiden. Trotzdem sind diese
Gebiete nicht unbewohnt und es| kommt leider manchmal zu
Aus|einandersetzungen zwischen den Ortsans\"assigen und den Kletterern.

Auch wenn dem eine gewisse Ablehung der seltsamen Fremden, die die
Kletterer ja sind, durch die Einheimischen zu Grunde liegen kann, so
liegt doch auch ein gro\3er Teil der Schuld im Verhalten des|
Kletterers. Auch wenn man in etwa zugeben kann, das{}s| die Vorg\"ange in
einer Wand nur die Kletterer und den Naturschutz betreffen, so sei
doch darauf hingewiesen, das{}s| der Zugang zu vielen Steilw\"anden aus|
landwirtschaftlichen, forstwirtschaftlichen oder selbst aus|
kommerziell gen\"utzten Gebieten besteht. Die Kletterer sind bei weitem
nicht die Einzigen, die ein Anrecht auf diese Gebiete haben, die f\"ur
die Einheimischen Arbeit und Verdienst bedeuten. Die Kletterer sollen
sich daher nicht wie Eroberer sondern wie G\"aste benehmen.

 Deswegen bitten wir aus|dr\"ucklich die Kletterer in den landwirtschaftlichen
und forstwirtschaftlichen Gegenden, die W\"ande nur \"uber die bestehenden
Feldwege oder, zur Not, \"uber brachliegende Felder zu erreichen. Es| ist
nicht \"ubertrieben, einen Umweg von einer Stunde oder ein Abseilen zu
machen, wenn nur auf diese Weise ein Feld geschont werden kann.

 Davon aus|gehend, das{}s| viele Kletterer St\"adter sind, erscheinen uns|
einige Verhaltensratschl\"age f\"ur das| Landleben angebracht: Obwohl man in
der Stadt jedes| nicht eingez\"aunte Grundst\"uck betreten darf (das| Schild
\<Propri\'et\'e Priv\'ee\>, d.h. \Ital{Privatgrundst\"uck}, ist nur ein
Hinweis), unter der Bedingung, das{}s| man keinen Schaden verursacht,
betritt man niemals| ein Feld oder
eine Heuwiese, auch wenn diese nicht eingez\"aunt sind. Es| ist keine
Entschuldigung, wenn St\"adter ein brachliegendes| Feld nicht von einer
Heuwiese unterscheiden k\"onnen. Hingegen gibt es| zwei Arten von
Einfriedungen, die nur schwer voneinander zu unterscheiden sind: Tiergatter,
die ein \"Uberqueren nicht verbieten, wenn man keinen Schaden anrichtet und
die Pforte schlie\3t, und Z\"aune, die unbefugtes| Betreten verhindern
sollen. So sei den Kletterern eindringlichst empfohlen, sich an Ort und
Stelle \"uber die Verh\"altnisse zu erkundigen, bevor Sie Ihren Weg zur
gew\"unschten Wand w\"ahlen.

 Noch gr\"o\3ere Vorsicht sei Autofahrern geboten. Ein Auto darf
niemals| au\3er\-halb der Abstellpl\"atze geparkt werden: wenn Fu\3g\"anger
auf einem brachliegenden Feld nur wenig Schaden anrichten, so lassen
sich Autospuren nur schwer wieder entfernen; auch wenn auf schmalen
Feldwegen ein parkendes| Auto die Durchfahrt f\"ur ein anderes| Auto noch
freil\"a\3t, so versperrt es| doch den Weg f\"ur land- oder
forstwirtschaftliches| Ger\"at; auch kann ein starker Verkehr auf den
Feldwegen die Landarbeit behindern.

 Bringen Sie nicht Ihren vollst\"andigen \Ital{Verpflegung} mit, sondern kaufen
Sie sie bei den \Ital{lokalen Gesch\"aften}. Wenn m\"oglich, besuchen Sie die
Bars|, Gasth\"ofe und Zeltpl\"atze der Gemeinde. Es| sei bemerkt, das{}s| die
Gesch\"aftsleute im Konfliktfall unsere besten Verteidiger sind.

 Man bedauert, das{}s| noch kein Gesetzt aus|dr\"ucklich berechtigt, zu
den Klettergebieten Zugang zu haben. Aber man mus{}s| sich auch dessen
bewusst sein, das{}s| ein Gesetzestext zwar einen privaten Eigent\"umer
zwingen kann, den Zugang zu erlauben, das{}s| er aber nicht eine
feindlich eingestellte Bev\"olkerung \"uberzeugen kann, sich nicht der
sch\"adlichen Eindringlinge zu entledigen.

\medskip
 Abf\"alle: viele Kletterer glauben, das{}s| es| gen\"ugt, wenn Sie Ihre
Abf\"alle sorgf\"altig eingepackt am Zugang zum Parkplatz oder am Rande
der Stra\3e hinterlassen, da ja die M\"ullabfuhr dann alles| einsammelt.
Dabei wird aber vergessen, das{}s| die meisten l\"andlichen Gemeinden nicht
die Mittel haben, eine M\"ullabfuhr f\"ur alle Stra\3en und Wege zu
unterhalten. Im besten Falle wird der M\"ull in der Ortsmitte abgeholt,
meistens| aber bringen die Einwohner ihre Abf\"alle selbst zum
M\"ullabladeplatz oder sie verbrennen sie. Auch mus{}s| davon Abstand
genommen werden, die Abf\"alle selbst zum \"ortlichen Depot zu bringen:
dieses| reicht vielleicht f\"ur ein Dorf von hundert Einwohnern aus|, es|
ist aber unzureichend f\"ur hunderte von Kletterern, die jeden Sonntag
einfallen. Man darf sich auch nicht auf die \"uberall montierten
Abfallbeh\"alter verlassen: sie sind oft zu klein und werden nur
unregelm\"a\3ig geleert. Wir m\"ochten Sie bitten, diese Beh\"alter den
Leuten zu \"uberlassen, die weniger gewissenhaft sind als| Sie, und die
nicht diese Zeilen lesen. Stattdessen schlagen wir vor:

 \medskip\centerline{\LARGE\bf Nehmen Sie alle ihre abf\"alle wieder mit !}

\centerline{\Large\bf(... und auch ein wenig mehr wenn m\"oglich!)}

\medskip
 Wenn Sie weiter weg wohnen, so k\"onnen Sie auch Ihren M\"ull in einer
gr\"o\3eren Stadt abladen, indem Sie Ihren M\"ullsack (gut verschlossen) zu
anderm M\"ull stellen, der auf den Abtransport wartet. In den meisten
F\"allen sind keine Toiletten vorhanden: vergraben Sie also das|
Toilettenpapier um die biologische Zersetzung zu beschleunigen; gr\"o\3ere
Mengen k\"onnen verbrannt werden, wenn kein Sicherheits|risiko besteht. In
Extremfall wird es| mit Ihrem M\"ull eingesammelt.

 \medrule

 Unabh\"angig von den Ortsans\"assigen ist der Kletterer nicht der einzige,
der auf die Felswand, die er beklettert, ein Anrecht hat. Arch\"aologische
Sch\"atze und eine au\3ergew\"ohnliche Vegetation k\"onnen ein Kletterverbot
nach sich ziehen. Vegetation und historische Sch\"atze leiden nur, wenn sie
zertreten oder entfernt werden. Hingegen kann schon die blo\3e N\"ahe von
Menschen die Vermehrung der V\"ogel, die in den Felsen nisten,
beeintr\"achtigen. Es| mus{}s| betont werden, das{}s| im Gegensatz zum \Ital{homo
erectus}, den man etwas| zu Unrecht als| \Ital{sapiens} qualifiziert, eine ganze
Reihe von Vogelarten vom Aus|sterben bedroht sind.

 Bei uns| sind vor allem die Arten betroffen, die in den Kletterw\"anden nisten
und sich dort vermehren: Wanderfalke, K\"onigsadler, Raubgeier, Uhu, gro\3e
Saatkr\"ahe. Von diesen Arten gibt es| manchmal nur noch einige dutzend in ganz
Frankreich. Werden sie nur einmal beim Nestbau oder w\"ahrend der Brutzeit
gest\"ort (im allgemeinen vom 1. Februar bis| zum 15 Juni), so stirbt die ganze
Brut, die 1/10 der n\"achsten Vogelgeneration aus|macht. Man kann gewis{}s| sagen,
das{}s| die Aus|rottung vor allem Pestizide, Jagd und Dummheit verschuldet ist,
aber es| w\"are traurig, wenn der Gnadensto\3{ }von den Kletterern k\"ame.

 Um drakonische Ma\3nahmen zu vermeiden, die zu einem generellen
Verbot f\"uhren w\"urden, raten wir eindringlichst davon ab, au\3er\-halb
der mit {\ecole} gekennzeichneten Stellen von Februar bis|
einschlie\3lich Juni zu klettern. Auf jeden Fall sollte man mit einem
Fernglas| mehrere Tage lang untersuchen, ob auch keine Nester
vorhanden sind.

 Es| sei noch bemerkt, das{}s| in gewissen {\slshape d\'epartements} (Ain,
Doubs|, Jura, C\^ote d'Or) ein Erlas{}s| des| Pr\"afekten das| Klettern im
Fr\"uhling in den meisten gro\3en Kletterw\"anden streng verbietet. Diese
Felsen werden von den Vogel\-schutz\-gesell\-schaften dauenrd \"uber\-wacht,
die bei \"Uber\-tretung die Wald- oder Land\-ver\-waltungs|\-beh\"orde
unmit\-tel\-bar benach\-richtigen.
Geldstrafen von 2~000~{\SetCmDefault\rm FF} bis|  60~000~{\SetCmDefault\rm FF} (1986).

 \medskip\centerline{\vrule width 0.5\hsize height 0.4pt}

 \medskip
 Bez\"uglich der Aus|r\"ustung der Sport\-kletter\-gebiete hat die \FFME{}
gem\"a\3ihres| Auftrags| als| Sportbund bestimmt, das{}s| diese zu dem
Aufgaben\-bereich ihrer Departement\-komitees| geh\"ore. Oft werden die lokalen
Vereine f\"ur diesen Auftrag von den Departement\-komitees| der \FFME{}
beauftragt.

 Deshalb, wird eine Route im Kletterf\"uhrer als| \<aus|ger\"ustet\>
(\Ital{\'equip\'ee}) be\-zeich\-net, so sind die vorhandenen Installationen
aus|reichend f\"ur den Kletterer, f\"ur den der allgemeine Schwierigkeits|grad
der Route angepa\3t ist. Es| geht also nicht, das{}s| man die Aus|r\"ustung
auf eigene Faust zeitweilig ver\"andert. Andernfalls| riskiert man ernsthafte
Zusammenst\"o\3e (vieleicht Prozesse). F\"ur diese Routen, die
als| aus|ger\"ustet be\-zeich\-net werden, sollte man also weder Haken noch
Hammer mit sich f\"uhren, sondern nur Klemmkeile und Schlingen. Dies|
bedeutet, das{}s| man sich \"uber die am Ort \"ublichen
Schwierigkeit\-skalenabst\"ande, und \"uber die G\"ultigkeit der
Kletterf\"uhrer informiert und so selbst nachpr\"uft, ob die angegebenen
Aus|r\"ustungen noch vorhanden sind. Der Kletterer, der au\3erstande ist,
eine Strecke zu bew\"altigen, soll sich einen Notaus|stieg suchen, sich von
oben ein Seil zuwerfen lassen, absteigen oder abseilen. Er darf \Ital{keine
zus\"atzlichen Haken anbringen}. Hingegen verlangen der \COSIROC{} und die \FFME{}
von den \"ortlichen Betreuern, das{}s| die vorhandene Aus|r\"ustung dem Niveau
der Kletterer, die die jeweilige Routen hinsichtlich ihres|
Schwierigkeits|\-grades| w\"ahlen, angepa\3t ist, und nicht nur von dem
Niveau der besten Kletterer der Gegend aus|geht.

 % \input sichermo.pub


 Wird eine Route als| \<nicht aus|ger\"ustet\> be\-zeich\-net, so empfehlen wir,
entweder ein gutes| Werk zu verrichten, und die Route \Ital{in Zusammenarbeit mit
den \"ortlichen Betreuern oder mit dem Departement\-komitee der \FFME}
aus|zur\"usten, oder so wenig wie m\"oglich Haken zu verwenden, die das| Gestein
besch\"adigen, unter weitgehendster Verwendung der nat\"urlichen Sicherungen.

In den \Ital{terrains| d'aventure} (Abenteuergebiete, Alpingel\"ande) sind die
Regeln nicht so streng, aber wir empfehlen den Kletterern dringendst, ihre
Aus|r\"ustung auf das| Ersetzen von gef\"ahrlichen Haken zu beschr\"anken, und
\Ital{niemals| eine systematische Aus|r\"ustung der Routen ohne Erlaubnis| von den
Departement\-komitees| der \FFME{} zu unternehmen}. In der Tat folgt die geringe
Anzahl von Haken oder die Aus|r\"ustung\-losigkeit in manchen gro\3en Routen
oft aus| der Absicht, diese Routen so aus|gesetzt, wie sie urspr\"unglich
geklettert wurden, zu erhalten. Dagegen ist das| Abrei\3en von
ungesch\"utzten Gew\"achs| und Moos| und die Entfernung von gef\"ahrlichen
Blocken immer ein gutes| Werk.

 \medskip\centerline{\vrule width 0.5\hsize height 0.4pt}

\medskip Verwenden Sie Magnesia nur in den \"au\3ersten F\"allen: dieses|
wei\3e Pulver saugt Schwei\3 auf und gibt so den H\"anden
vor\"ubergehend besseren Halt. Aber leider kann dieses| bl\"ode Pulver nicht
den edelen Schwei\3 des| Kletterers| von der Luftfeuchtigkeit unterscheiden.
Nach Gebrauch saugt es| sich mit Wasser voll und erinnert dann sehr an
Schmierseife, wobei es| auch noch die nat\"urlichen Unebenheiten des| Gesteins|
zuschmiert. Es| wird denn notwendig immer gr\"o\3ere Mengen Magnesia zu
verwenden, da nicht nur der Schwei\3, sondern auch der Tau der letzten
Nacht getrocknet werden muss. Wir empfehlen statt dessen gemahlenes| Harz
(\<pof\> auf franz\"osichem Kletterjargon), auch Kolophonium genannt. Es| gibt
H\"anden und F\"u\3en mehr Reibung und ist biologisch und durch
Sonneneinwirkung abbaubar. Magnesia ist in verd\"unnter Salzs\"aure
l\"oslich.

\end{document}
