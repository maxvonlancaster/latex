%%%%%%%%%%%%%%%%%%%%%%%%%%%%%%%%%%%%%%%%%%%%%%%%%%%%%%%%%%%%
% File: yi4latex.tex
% Date: September 2, 1997
% Author: Oliver Corff
% (c) 1997 Ulaanbaatar, Beijing, Berlin
%
% A package for providing the Yi script to LaTeX-users.
%
% NB: This package requires LaTeX2e!
%
%%%%%%%%%%%%%%%%%%%%%%%%%%%%%%%%%%%%%%%%%%%%%%%%%%%%%%%%%%%%
\documentclass[11pt,a4paper]{article}
\usepackage{yi}
\title{\huge\textbf{\Ynuo\Ysu\Ybbur\Yma}\\
	\vspace{.25cm}
	\sf Yi for \LaTeX , v.~0.1}
\author{Oliver Corff\protect\footnote{%
	The \YiL\ package was developed while receiving 
	a DFG (German Research Council) grant to which
	the author expresses his sincere gratitude.}}
\date{Sep.~2nd, 1997}
\newcommand{\YiL}{\textsf{Yi for \LaTeX}}
\markboth{\YiL}{\YiL}
\pagestyle{myheadings}
\begin{document}
\maketitle
\thispagestyle{myheadings}

\section{Introduction}
{\YiL} is a package providing the Yi script for \LaTeX\ users.
The Yi live mainly in Sichuan, Yunnan, Guizhou and Guangxi, and
have a population of about 6.5 million (1990). There are various
viewpoints on when the Yi script originated; some scholars claim
this happened during the Tang dynasty, others say it occurred
only during the Ming dynasty, and others again place the date
well back by 6000 years. A detailed summary of arguments is
presented by Ding Chunshou in his treaty (\emph{On Yi})%
\footnote{Ding Chunshou: \emph{Yiwen Lun}. Sichuan Minzu Chubanshe,
Chengdu 1993. ISBN 7-5409-0659-6/H.41}

The historical Yi script comprised somewhere close to 10,000
characters which were used in a way similar to Chinese, i.~e.~they
had ideographic properties. Modern Yi, however, takes a radically
different approach: All syllables have a unique sound but no
intrinsic meaning whatsoever. This system was promulgated by the
State Council as Standardized Yi in 1980 and subsequently popularized
by a number of dictionaries, grammars etc.

The \YiL\ package provides all 1165 Yi syllables as defined in the State
Council decree of 1980, arranged by a consonant order which is a
hybrid between the syllabic arrangement by phonetics and a pure
alphabetical arrangement. This arrangement was not chosen by the
author; it reflects the current status of discussions on creating
ISO and Unicode standards for Yi. If this order is preserved,
the only thing needed for using these fonts in 16-bit environments 
is to indicate the offset into this code space.

Since the number of syllables is so large, five encodings had
to be defined.  These are labelled as local encodings: the
series LYA, \dots\ LYE stands for [L]ocal [Y]i, [A] to [E].

Regardless of the physical arrangement of the syllables in the font,
the user always has access to all syllables via their syllable name.


\section{How to use the Yi package}
This section describes which files are necessary and which commands
are needed for invoking the Yi syllables.


\subsection{Necessary files}
The Yi package consists of the following files:
\begin{raggedright}
\begin{description}
\item [Package] The file {\tt yi.sty} provides all necessary
	declarations and commands for selecting and entering the
	Yi script. Place this file somewhere where your
	\LaTeX\ installation will find it. In {\tt emtex} systems,
	this could be the path \verb"\emtex\texinput\yi" assuming you
	have a subdirectory named \verb"yi" which holds everything
	necessary for Yi Language Support.
\item [Fonts] The font files are:
	\begin{enumerate}
		\item {\tt yir-[1-5].mf}: These are the font drivers
					for generating regular typeface. 
		\item {\tt yib-[1-5].mf}: These are the font drivers
					for generating boldface. 
		\item {\tt yis-[1-5].mf}: These files contain the
					individual syllable definitions.
		\item {\tt yi[rb]parms.mf}: These files contain the
					settings for different typefaces.
		\item {\tt yirdefs.mf}: This file contains general
					macro definitions used for
					the syllable definitions.
	\end{enumerate}
	These 18 files should
	reside in a directory where {\sc Metafont} can find them, e.~g.
	\verb"\emtex\mfinput\yi" if using \verb"emtex", or perhaps in
	\verb"/usr/local/tex.local/texmf/fonts/source/public/yi/" when
	using \verb"teTeX".

\item [Font Metrics] The ten font metrics files {\tt yi[rb]-[1-5].tfm} 
	go into \verb"\emtex\tfm\yi" or 
	\verb"/usr/local/tex.local/texmf/fonts/tfm/yi/".

\item [Documentation] There are two files, a character table
	named \verb"yi-table.tex" and a description of the
	package named {\tt yi4latex.tex}. You are reading it at
	the moment. Both files can be placed in the same
	directory as {\tt yi.sty} or in a separate directory in the 
	\verb"\emtex\doc" branch, or in
	\verb"/usr/local/tex.local/texmf/doc/".
\end{description}
\end{raggedright}

\subsection{Declarations and Input Methods}
In order to use the Yi script, the document preamble 
must contain the declaration \verb"\usepackage{yi}". This
is the only necessary step.

Entering Yi syllables into any text is straightforward. In the
\YiL\ package, all syllable names are defined as text commands.
The syllable names are prefixed by \verb"\Y". Two typefaces,
regular and boldface, are supported, and they are addressed as
in any textmode. Thus, the title
of this paper is generated with the following commands:
\begin{center}
	regular: \verb"\Ynuo\Ysu\Ybbur\Yma" \\
	\Ynuo\Ysu\Ybbur\Yma \\
	boldface: \verb"{\textbf{\Ynuo\Ysu\Ybbur\Yma}}" \\
	{\Large\textbf{\Ynuo\Ysu\Ybbur\Yma}}\\
\end{center} 

Instead of stating \verb"\Y"\emph{syllable} it is also possible 
to use the form \verb"\Y{"\emph{syllable}\verb"}".

Besides that, the Yi numbers can be entered with the commands
\verb"\YOne", \verb"YTwo", \verb"\YThree" up to \verb"\YTen":
\YOne, \YTwo, \YThree, \YFour, \YFive, \YSix, \YSeven, \YEight,
\YNine\ and \YTen.

Another command is useful when writing about Yi (rather than in Yi):
The pronounciation of each syllable can be attached to that syllable
by saying \verb"\YP{"\emph{syllable}\verb"}":
\begin{center}
	Input: \verb"\YP{nuo}\YP{su}\YP{bbur}\YP{ma}" \\
	Result: \YP{nuo}\YP{su}\YP{bbur}\YP{ma} \\
\end{center} 
The text example on the next page shows the result of applying this
command to a whole text.

The syllable reduplication mark \YP{wu} can also be entered as \YP{w}.
This is done intentionally.

\section{Known Problems}

There seems to be only one genuine problem so far: In \TeX\
environments which do not go through PostScript to generate
printer output, e.~g.~\verb"emtex", it is possible that downloading
a large amount of large character bitmaps into the printer can cause
a printer memory overflow. This happened with a Hewlett-Packard
LaserJet IV when trying to print the whole font table (comprising
1165 characters) with Yi characters set at \verb"|\Huge". The remedy
is simple: Print until the printer complains and gives up, and start
a new printing job with the first page set to where the printer died
on the last job.  This will take you safely through the rest of your
printing job.


\section{Legal Status}

The \YiL\ package is protected by copyright: it is not in the public
domain. Non-commercial, academic and private use is permitted;
commercial parties willing to use \YiL\ are requested to obtain a licence.

You are allowed and indeed encouraged to share this software with
others; you are not allowed to modify the software and redistribute
the modified software under the same name. Modified versions of this
software may only be distributed if the name is different from the
original software and if the modified software is accompanied by
the original software.

This software package may only be distributed in complete form,
preferably in form of the archive as which is was obtained from
its source.

\section{Desiderata}
%\begin{sloppypar}
The package in its present stage cannot be regarded complete.
Some of the letter shapes still need a bit of refinement, and for
a future research project it is desirable to encode the
pre-standardization Yi characters (of which there are around 10,000
or so). This, however, is work which cannot be done from books alone;
the support of native Yi speakers is required.

The punctuation marks used in Yi literature are in Chinese style;
though the use of Western style punctuation may be permissible, a
full set of suitable punctuation marks waits to be implemented.

The input method works but can barely called a user interface; an
input preprocessor which allows the entering of Yi text without specific
syllable markup would be fine, and waits to be written. Either a
small C preprocessor or a \TeX\ program are acceptable; the latter
to be preferred since no additional user interaction would be necessary.

It would be a great thing if Omega appears soon; the characters could
then be mapped to the Unicode world simply by defining an offset into
the 16-bit code space, and the LYA to LYE business would become obsolete.

Comments and suggestions are highly appreciated and should be
directed to the author of these lines at \verb"corff@zedat.fu-berlin.de".
%\end{sloppypar}

\pagebreak

\section{A text sample%
	\protect\footnote{See Ding:1993, p.~120}:
		\Ypup\Yguop\Ymgu\Yxi\Yndit}

\YP{ngop} \YP{ox} \YP{ge} \YP{ap} \YP{mop} \YP{su} \YP{pup}
\YP{guop} \YP{li}, \YP{mux} \YP{pyr} \YP{a} \YP{jjy},
\YP{ddur} \YP{lu} \YP{ax} \YP{nyi} \YP{mu} \YP{jjo}, 
\YP{zzur} \YP{pup} \YP{ax} \YP{li} \YP{xip} \YP{ma}
\YP{nge}.  \YP{guop} \YP{jiet} \YP{cyx} \YP{ma} \YP{xy}
\YP{wa} \YP{mu}, \YP{mu} \YP{wa} \YP{vo} \YP{le} \YP{ggup}
\YP{mu} \YP{sot} \YP{go} \YP{ho} \YP{ne} \YP{kop} \YP{ggu}
\YP{hxa} \YP{fut} \YP{ci} \YP{vat} \YP{nra} \YP{ma}
\YP{jjo}; \YP{co} \YP{cux} \YP{vyt} \YP{nyi} \YP{nge}
\YP{cix} \YP{nge} \YP{bbup} \YP{nra} \YP{jjo}, \YP{co}
\YP{go} \YP{ci} \YP{vat} \YP{w} \YP{o} \YP{hmy} \YP{jjo}.

\YP{ngop} \YP{pup} \YP{guop} \YP{cyx} \YP{ma} \YP{li},
``\YP{zho} \YP{huop} \YP{vo} \YP{co} \YP{jjy} \YP{gex}
 \YP{nzix} \YP{nyi} \YP{guop} '' \YP{hmi}. \YP{guop}
\YP{jiet} \YP{ly} \YP{kop} \YP{rep} \YP{dde} \YP{ne}
\YP{bip} \YP{ji} \YP{lur} \YP{kur} \YP{nge}. \YP{guop}
\YP{jiet} \YP{gu} \YP{ho} \YP{sat} \YP{ma} \YP{nge}, 
\YP{mu} \YP{jy} \YP{nge} \YP{ma} \YP{bbop} \YP{jjie}
\YP{lap} \YP{vut} \YP{tie} \YP{nga} \YP{mop} \YP{jjip}, 
\YP{ggup} \YP{lep} \YP{w} \YP{ne} \YP{che} \YP{si} \YP{nip}
\YP{zzie} \YP{ma} \YP{sip} \YP{zhop} \YP{mga} \YP{mu} \YP{da}
\YP{ma} \YP{nge}. \YP{guop} \YP{jiet} \YP{po} \YP{bbo}
\YP{li}, \YP{po} \YP{bbo} \YP{a} \YP{hni} \YP{bbut} \YP{go}
\YP{jyx} \YP{nge} \YP{rry} \YP{nge} \YP{ma} \YP{dit} \YP{da}
\YP{bbut} \YP{nge}. \YP{guop} \YP{jiet} \YP{shyp} \YP{si}
\YP{hmox} \YP{li} \YP{zho} \YP{guop} \YP{gop} \YP{cha}
\YP{dax} \YP{nge}.

\YP{ngop} \YP{pup} \YP{guop} \YP{mgu} \YP{xi} \YP{ndit}
\YP{cyx} \YP{ma} \YP{li}, \YP{zzyt} \YP{mu} \YP{gax} \YP{tuo}
\YP{guop} \YP{jiet} \YP{co} \YP{ma} \YP{ax} \YP{nyi} \YP{lop}
\YP{ax} \YP{nyi} \YP{max} \YP{su}, \YP{mux} \YP{pyr}
\YP{nyi} \YP{a} \YP{jjy} \YP{lop} \YP{a} \YP{jjy} \YP{suo}
\YP{ma} \YP{su} \YP{nge}. \YP{guop} \YP{jiet} \YP{xy}
\YP{wa} \YP{nbi} \YP{se} \YP{nyip} \YP{zi} \YP{nyix} \YP{ma},
\YP{zyt} \YP{jie} \YP{jux} \YP{dde} \YP{qu} \YP{nge}
\YP{ma}, \YP{lur} \YP{kur} \YP{shyp} \YP{ax} \YP{yy}
\YP{suo} \YP{ma}, \YP{jjy} \YP{gex} \YP{mu} \YP{jux} \YP{dde}
\YP{suo} \YP{ci} \YP{ma} \YP{mu} \YP{da}.

\YP{ngop} \YP{ox} \YP{pup} \YP{guop} \YP{li}, \YP{zzur}
\YP{pup} \YP{ax} \YP{li}, \YP{mo} \YP{nyop}, \YP{get} \YP{nyop},
\YP{syr} \YP{nyop}, \YP{bbot} \YP{nyop}, \YP{hxe} \YP{nyop}
\YP{cyp} \YP{xix} \YP{cyp} \YP{yiet} \YP{ap} \YP{ddur} \YP{ddu}
\YP{ap} \YP{jjo}, \YP{syp} \YP{lu} \YP{mo} \YP{lu} \YP{get}
\YP{lu} \YP{bbo} \YP{pat} \YP{ry} \YP{su} \YP{nyi} \YP{ax}
\YP{nyi} \YP{mu} \YP{jjox} \YP{ma} \YP{nge}. \YP{yy} \YP{hmy}
\YP{jox} \YP{zhy} \YP{dop} \YP{yit}, \YP{tep} \YP{yy} \YP{su}
\YP{qu} \YP{ddie} \YP{su}, \YP{cy} \YP{mop} \YP{ddie} \YP{su}
\YP{si} \YP{nip} \YP{tep} \YP{yy} \YP{bbur} \YP{ma} \YP{zhet}
\YP{su} \YP{jix} \YP{po} \YP{cyx} \YP{ly} \YP{yiet} \YP{jjy}
\YP{gex} \YP{ngop} \YP{guop} \YP{jiet} \YP{miep} \YP{lie}
\YP{bbo} \YP{pat} \YP{shu} \YP{la} \YP{su} \YP{nge}.
\end{document}
