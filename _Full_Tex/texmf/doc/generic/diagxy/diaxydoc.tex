
\documentclass[12pt]{article}
 \input diagxy
 \xyoption{curve}
 \textwidth 6in
 \oddsidemargin 0pt
 \makeindex
\begin{document}
\def\xypic{\hbox{\rm\Xy-pic}}

\title{A new diagram package (Version \thedate)}
\author{Michael Barr\\Dept of Math and Stats\\McGill University\\805
Sherbrooke St. W\\Montreal, QC Canada H3A 2K6\\[5pt] barr@barrs.org}
 \date{}
 \maketitle

\makeatletter
\contentsline {section}{\numberline {1}Why another diagram package?}{2}
\contentsline {subsection}{\numberline {1.1}The latest version}{3}
\contentsline {subsection}{\numberline {1.2}Main features}{3}
\contentsline {subsection}{\numberline {1.3}Compatibility}{3}
\contentsline {subsection}{\numberline {1.4}Errant spaces}{4}
\contentsline {subsection}{\numberline {1.5}Font sizes}{4}
\contentsline {subsection}{\numberline {1.6}On \LaTeX \ 2.09 and \LaTeX \ 2e}{5}
\contentsline {subsection}{\numberline {1.7}Acknowledgments}{5}
\contentsline {subsection}{\numberline {1.8}License}{6}
\contentsline {section}{\numberline {2}The basic syntax}{6}
\contentsline {subsection}{\numberline {2.1}A word about parameters}{7}
\contentsline {section}{\numberline {3}Defined diagrams}{8}
\contentsline {section}{\numberline {4}Examples}{10}
\contentsline {subsection}{\numberline {4.1}Complex diagrams}{19}
\contentsline {subsection}{\numberline {4.2}Empty placement and moving labels}{23}
\contentsline {subsection}{\numberline {4.3}Inline macros}{25}
\contentsline {subsubsection}{\numberline {4.3.1}Added:}{26}
\contentsline {subsection}{\numberline {4.4}2-arrows}{26}
\contentsline {subsection}{\numberline {4.5}Mixing \hbox {\rm \unhbox \voidb@x \hbox {\kern -.1em X\kern -.3em\lower .4ex\hbox {Y\kern -.15em}}-pic}\ code}{27}
\contentsline {subsection}{\numberline {4.6}Diagram from TTT}{29}
\makeatother

\kern 40pt
{\narrower
For the fashion of Minas Tirith was such that it was built on seven levels,
each delved into a hill, and about each was set a wall, and in each wall
was a gate.

-- J.R.R.\ Tolkien, ``The Return of the King''\footnote{These macros are
at the seventh level of nesting.  The first level is the code actually
executed directly by the processor.  The second is the microcode, burned
into the chip, that interprets the assembly language instructions into
the ones that are directly executed.  Level three is the language,
usually C, but originally Pascal, into which the Web code, the fourth
level, is compiled.  The fifth level is \TeX\ itself, which is best seen
as a high level programming language for mathematical text processing.
Level six is the \xypic\ code in which these macros, level seven, are
written.}

}


\kern 5pt
 Note:  This is the first increase in functionality since the original
general release.  The macros \index{\backslash to}\verb.\to. and
\index{\backslash two}\verb.\two. that make inline arrows and pairs of arrows now
accept an option to vary the types of the arrow, for example to get a
pair of adjoints going in opposite directions.  A macro that makes three
arrows has been added as well as a macro for a 3 x 3 diagram that can be
expanded into a map between two exact sequences.  A new optional
parameter has been added to \index{\backslash to}\verb.\to.,
\index{\backslash two}\verb.\two. and \index{\backslash three}\verb.\three. to allow the
user to override the automatic computation of the arrow size.  See the
sections on inline arrows, page~\pageref{added} and
page~\pageref{complex} for details.

\section{Why another diagram package?}

This started when a user of my old package, diagram, wrote to ask me if
dashed lines were possible.  The old package had dashed lines for
horizontal and vertical arrows, but not any other direction.  The reason
for this was that \LaTeX\ used rules for horizontal and vertical arrows,
but had its own fonts for other directions.  While rules could be made
any size, the smallest lines in other directions were too long for
decent looking dashes.  Presumably, Lamport was worried about compile
time and file size if the lines were too short, considerations that have
diminished over the years since.  Also arrows could be drawn in only 48
different directions, which is limiting.  My macros were not very well
implemented for slopes like 4 and 1/4, since I never used such lines.

There was certainly an alternative, \xypic, for those who wanted
something better.  But it was hard to learn and I was not entirely happy
with the results.  The basic interface used an
\index{\backslash halign}\verb.\halign..  This
meant that no extra space was allotted to an arrow that had a large
label.  In addition, the different slots could have different horizontal
size, which could result in a misshapen diagram.  I used it in a paper
that had a `W' shaped diagram whose nodes had different widths and the
result was quite obviously misshapen.  On the other hand, the graphics
engine underlying \xypic\ is really quite remarkable and it occurred to
me that I could try to redraft diagram as a front end.  The result is
the current package.  It has been tested mainly with version 3.7, so
there is no guarantee it would work with any earlier version (or, for
that matter, later).  A limited amount of testing with version 3.6
suggests that the only thing that does not work is the 2-arrows; these
come out vanishingly short.

So despite the desire for logical programming, it still seems that for
some purposes, it is desirable to have a system in which you specify
what goes where and where the arrows are drawn.

\subsection{The latest version}
The latest version can be downloaded at
\verb,ftp://ftp.math.mcgill.ca/pub/barr/diagxy.zip, or by anonymous ftp
from \verb,ftp.math.mcgill.ca//pub/barr/diagxy.zip,.


\subsection{Main features}

\begin{itemize}
 \item A general arrow drawing function
\index{\backslash morphism}\verb.\morphism..
 \item Various common diagram shapes such as squares, triangles, etc.
 \item A few special shapes such as cube and $3\times3$ diagrams.
 \item Small 2-arrows that can be placed anywhere in a diagram, much
like \LaTeX's
picture
 environment.
 \item A uniform syntax, while allowing access to all of \xypic's
capabilities
 \item Never expires and does not request acknowledgment.
\end{itemize}

\subsection{Compatibility}
The syntax described below is not compatible with the original diagram
package.  Every front end of this sort represents a trade-off between
simplicity and utility.  A package that simply upgraded the syntax to
allow more dashed (and dotted) lines would have been just as easy to
implement, but would have made poor use of the wonderful possibilities
of the underlying \xypic\ package.  Also there would have been too many
different arrow specifications (they would have had to go at least in
the range $[-9,9]$) for easy memory) and still would not have included
things like inclusion arrows.  To those who would have liked a simpler
syntax, I apologize.  Those who would want more flexibility, I remind
that the entire \xypic\ package is there for use.

\subsection{Errant spaces}
 There is one point that cannot be made too strongly.  {\em Watch for
errant spaces.} Unlike the old diagram package, which was carried out in
math mode so that spaces were ignored, \xypic\ is not in math mode and
spaces will mess up your diagrams.  On the other hand, it will not be
necessary to enclose duplicate nodes inside
\index{\backslash phantom}\verb.\phantom., since the
registration between different nodes is perfect.  In the old package,
for reasons that I now understand, objects did not always register
properly.  This was a flaw built in to the very heart of the package and
is not worth correcting, although it could have been done better in the
first place.  If you see an object in double vision, then the almost
certain cause is that a space has gotten in there somewhere.  I have
attempted to prevent this by liberal use of
\index{\backslash ignorespaces}\verb.\ignorespaces.  in the
definitions, but one cannot always be sure and while testing, I found a
number seemingly innocuous spaces that messed up diagrams.  When in
doubt, always terminate a specification with a
\index{%}\verb.%..  See the
examples.

\subsection{Font sizes}
 According to the documentation of \xypic, the declarations
\begin{verbatim}
\xyoption{tips}
\SelectTips{xy}{12}
\end{verbatim}
 \index{tips} will cause 12 point tips to be used.  This does not appear
to be the case.  The only way I can seem to get larger font sizes is to
add \index{\backslash fontscale{x}}\verb.\fontscale{x}., with \index{x}\verb.x.
taking on the values \verb.0,h,1,2,3,4,5., after \xypic\ is loaded.
With \verb.11pt. you will want \verb.h. and with \verb.12pt. you will
want 1. The others are in case you use larger sizes for transparencies
or for later reduction.  While on the subject, I might mention that if
you want thicker arrow shafts without enlarging anything else you could
add the declaration
\begin{verbatim}
\font\xydashfont=xydash scaled \magstep1
\end{verbatim}
or even larger.  However, this does not thicken the arrow tips and is
not really recommended.  There is probably no way (short of creating
your own fonts) to thicken the tips without also lengthening them.


\subsection{On \LaTeX\ 2.09 and \LaTeX\ 2e}
 Since I have no other forum to do so, I use this space to indulge in a
personal rant, which you may skip since it has nothing to do with the
package (except periphally, since it explains why I have not made 14,
17, and 20 point style files for \LaTeX\ 2e).  Recent distributions of
\TeX\ have omitted latex.tex, the source file for \LaTeX\ 2.09.  If you
make a posting on comp.text.tex, admitting to using the older version,
you are certain to get one or more replies admonishing you that \LaTeX\
2.09 is obsolete and should not be used.  There are at least a couple of
good reasons for continuing to use the older version (along with some
reasons for switching, especially if you are using a language other than
English).  For one thing, the compatibility mode is useless if you have
used additional fonts, as I have (most importantly, cmssbx, where many
use bbbold, one of the ugliest fonts to see the light of day).  In fact,
the most significant change in \LaTeX\ 2e is that it uses the new font
selection system and the compatibility mode makes no attempt to be
compatible with the old one.  The second reason for not switching is
that, ``If it ain't broke, don't fix it.''  For many the old \LaTeX\
worked fine, why change?  If you want to use new fonts in \LaTeX\ 2e it
is certainly possible, but the instructions for doing so are embedded
holographically in Chapter 7 of The \LaTeX\ Companion and took me days
to decipher.  I am still not convinced that I am doing it right,
although it seems to work.  The third---and to me most important---is
that the file latex.tex is extensively documented internally and it is
relatively easy to make whatever changes one wants.  I have never
changed the source, but always put the changes in another file.  In this
way, I have changed, for example,
\index{\backslash @startsection}\verb.\@startsection.  in various ways.
It is clear that Lamport intended that his version of \LaTeX\ be
user-modifiable.  It is equally clear that the implementers of the new
version had no interest in that.  They didn't go out of their way to
make it hard, but they did nothing to encourage it and the documentation
is entirely on how to use it.

\subsection{Acknowledgments}
 First, I would like to thank Kris Rose for a superb programming job
that made this package possible.  The original versions of \xypic\ were
based on matrices, but this version includes code to place an arrow with
two nodes (more precisely, their centers) a user-definable distance
apart and that is exactly what I needed.  Second I would like to thank
Ross Moore who answered innumerable questions about the use of the
macros that the documentation did not answer or at least I could not
understand what it said.  I thank Donald Arsenau answered several of my
questions about the internals of \TeX\ that I did not understand
clearly and Steve Bloom who found a couple
of typos in the first version of these notes.  Finally, I would like to
thank all those, but especially Charles Wells, who gave me opinions on
the best syntax.  But I made the final decisions and if you have any
complaints, you can direct them to me.  Not that I am likely to change
anything at this date.

\subsection{License}
 The use of this package is unrestricted.  It may be freely distributed,
unchanged, for non-commercial or commercial use.  If changed, it must be
renamed.  Inclusion in a commercial software package is also permitted,
but I would appreciate receiving a free copy for my personal examination
and use.  There are no guarantees that this package is good for
anything.  I have tested it with LaTeX 2e, LaTeX 2.09 and Plain TeX.
Although I know of no reason it will not work with AMSTeX, I have not
tested it.


\section{The basic syntax}

The basic syntax is built around an operation
\index{\backslash morphism}\verb.\morphism.  that is
used as \index{\backslash morphism}
 \begin{verbatim}
 \morphism(x,y)|p|/{sh}/<dx,dy>[N`N;L]
 \end{verbatim}
 The last group of parameters is required.  They are the source and
target nodes of the arrow and the label.  The remaining parameters are
optional and default to commonly used values.  Currently, the
\verb.L.\index{label} is set in \index{\backslash scriptstyle}\verb.\scriptstyle.
size, but this can easily be changed by putting
\index{\backslash let\backslash labelstyle=\backslash textstyle}
\verb.\let\labelstyle=\textstyle. in
your file.

 The parameters \verb.x. and \verb.y. give the location of the source
node within a fixed local coordinate system and \verb-x+dx- and
\verb-y+dy- locate the target.  To be precise, the first coordinate is
the horizontal center of the node and the second is that of the base
line.  These distances are given in terms of \index{\backslash ul}\verb.\ul.'s,
(for unitlength), which is user assignable, but currectly is .01em.  The
placement parameter \verb.p. is one of \verb.a,b,l,r,m. that stand for
above, below, left, right, or mid and describe the positioning of the
arrow label on the arrow.  If it is given any value other than those
five, it is ignored.  We describe below why you might want it ignored.
The label \index{m}\verb.m. stands for a label positioned in the middle
of a split arrow.  When used on a non-vertical arrow, \index{a}\verb.a.
positions the label above the arrow and \index{b}\verb.b. positions it
below.  On a vertical arrow, the positioning depends on whether the
arrow points up or down.  Similarly, when used on a non-horizontal
arrow, \index{l}\verb.l. positions the label to the left and
\index{r}\verb.r. positions it to the right, while on a horizontal arrow
it depends on which way it points.

The shape parameter \verb.sh. describes the shape of the arrow.  An
arrow is thought of as consisting of three parts, the tail, the shaft
and the head.  You may specify just the head, in which case the shaft
will be an ordinary line, or all three.  However, since the tail can be
(and usually is) empty, in practice you can also describe the shaft and
tail.  In addition, it is possible to modify the arrow in various ways.
Although the parameter is shown within braces, the braces can be omitted
unless one of the modifier characters is \verb./., in which case, {\em
the entire parameter} must be put in braces.  It is important to note
that it will not work just to put the \verb./. inside the braces, since
this will interfere with the internal parsing of \xypic.  The head and
tail shapes are basically one of \index{>}\verb.>., \index{>>}\verb.>>.,
\verb.(., \verb.)., \index{>}\verb*+>+, and \index{<}\verb*+<+.  Each of
these may also be preceded by \index{^}\verb.^. or \index{_}\verb._. and
others are user definable.  For details, see the \xypic\ reference
manual.  The first of these is an ordinary head, while the second is for
an epic arrow.  The third is not much used, but the superscripted
version makes and inclusion tail, as will be illustrated below.  The
reverse ones give reversed arrowheads.  The sign \verb*+ + stands for an
obligatory space and it leaves extra space for a tailed (monic) arrow,
which otherwise runs into the source node.  Although there are many
possibilities for shafts, including alphanumeric characters, the ones
that mainly interest us are:  \index{-}\verb.-., which is an ordinary
shaft, \index{--}\verb.--., which produces a dashed arrow,
\index{=}\verb.=., which gives a double arrow (although with only one
arrowhead), and ., which makes a dotted arrow.  Thus \index{>}\verb.>.
or \index{->}\verb.->. will produce an ordinary arrow,
\index{->>}\verb.->>. an epic arrow, \index{ >->}\verb*+ >->+ makes a
monic arrow, and \index{-->>}\verb.-->>. would make a dashed epic arrow.
The descriptions \index{<-}\index{<<-}\verb.<-,<<-,.  \index{<-<
}\verb*+<-< +, and \index{<<--}\verb.<<--. give the reversed versions.
Note that \index{<}\verb.<. does not give a reversed arrow, since
\xypic\ interprets that as a reversed head, not a tail.

If the shape parameter begins with an \index{@}\verb.@., it is
interpreted differently.  In that case, it has the form
\index{@{shape}@}\verb.@{shape}@. modifier, where the modifier is as
described in the \xypic\ reference guide.  I just mention a couple of
them.  The parameter \index{@{->}@<3pt>}\verb.@{->}@<3pt>., for example,
would give an ordinary arrow moved three points in the direction
perpendicular to that of the arrow.  If you give
\index{{@{-->}@/^5pt/}}\verb.{@{-->}@/^5pt/}., you will get an epic
arrow that is curved in the direction perpendicular to the direction of
the arrow by five points.  It is imperative that a specification such as
\index{@{>}@/5pt/}\verb.@{>}@/5pt/. be enclosed in braces because of the
\index{/}\verb./.s.

\subsection{A word about parameters}
 I have already mentioned the necessity of enclosing certain arrow shape
specifications in braces.  Because of the way \TeX\ operates, I have
used a number of different delimiters:  \index{(}\verb.(.,
\index{)}\verb.)., \index{|}\verb.|., \index{/}\verb./.,
\index{<}\verb.<., \index{>}\verb.>., \index{[}\verb.[.,
\index{]}\verb.]., \index{`}\verb.`., and \index{;}\verb.;..  Any of
these that appear inside an argument could conceivably cause problems.
They were chosen as the least likely to appear inside mathematics
(except for \index{(,)}\verb.(,). which appear in positions that are
unlikely to cause problems).  However, be warned that this is a possible
cause of mysterious error messages.  If this happens, enclosing the
offending parameter in braces should cure the problem.  The exceptions
come when the braces interfere with \xypic's somewhat arcane parsing
mechanism.  One place it imperative to use braces is if you attempt to
use a disk as a tip.  Although most tips do not have to be enclosed in
braces, you get a small solid disk tip from \index{{*}}\verb.{*}. and
not from \index{*}\verb.*.  (see the note at the bottom of the first
column of page 42 in the reference manual.  The first three of
\begin{verbatim}
\morphism(0,300)|a|/-*/<500,0>[A`B;f]
\morphism(0,0)|a|/{-*}/<500,0>[A`B;f]
\morphism(0,0)|a|/{*}/<500,0>[A`B;f]
\morphism(0,0)|a|/-{*}/<500,0>[A`B;f]
\end{verbatim}
 give error messages.  Only the fourth works:
 $$\bfig
\morphism(0,0)|a|/-{*}/<500,0>[A`B;f]
 \efig$$

In addition to the diagrams, there are macros that are intended to be
used inline to make horizontal arrows, pointing left or right, plain,
monic, epic, or user-definable shapes, and calculating their own length
to fit the label.  Finally, there is a macro for making short 2-arrows
that may be placed (actually, \index{\backslash place}\verb.\place.d) anywhere in
a diagram.


\section{Defined diagrams}

Using the basic
\index{\backslash morphism}\verb.\morphism.  macro, I have a defined a number of
diagrams:  squares (really rectangles) and variously oriented triangles
and a few compound diagrams.  The basic shapes are exactly the same as
in the old diagram package, but the options are done entirely
differently.  Here is the syntax of the
\index{\backslash square}\verb.\square.  macro:
 \begin{verbatim}
 \square(x,y)|pppp|/{sh}`{sh}`{sh}`{sh}/<dx,dy>[N`N`N`N;L`L`L`L]
 \end{verbatim}
 Each of the first four sets of parameters is optional and any subset of
them can be omitted.  (Note that only the sets can be omitted, once you
give \index{dx}\verb.dx., you also have to give \index{dy}\verb.dy. and
so on.)  The first set gives the horizontal and vertical position (in a
local coordinate system) of the lower left corner of the square, the
next four give the label placements using the same five characters
previously described.  The next four give the shapes of the arrows using
the same syntax as discussed above.  The last group is horizontal and
vertical size of the square.  More precisely, the \index{x}\verb.x.
coordinate is that of the midpoint of the node, while the
\index{y}\verb.y. coordinate is that of the baseline of the node.  This
is entirely based on \xypic.

In the case of other shapes, discussed below, the positioning parameter
may be different.  The \index{x}\verb.x. coordinate is the midpoint of
the leftmost node and the \index{y}\verb.y. coordinate is the baseline
of the lowest node.  In the case of the
\index{\backslash qtriangle}\verb.\qtriangle.,
\index{\backslash Vtriangle}\verb.\Vtriangle., and
\index{\backslash Ctriangle}\verb.\Ctriangle. described below, these are different
nodes.  What this positioning means is that if you specify the
coordinates and sizes correctly the shapes will automatically fit
together.  The last example on Page~\pageref{TTTdiag} illustrates this.

Here is a listing of the shapes, together with the groups of parameters.
In all cases, the first four groups are optional and any subset of them
will work.  However, they must come in the order given.  Note that the
names of the triangles are related to the shape as the shape that best
approximates the shape of the letter.  For example, a
\index{\backslash ptriangle}\verb.\ptriangle. is a right triangle that has its
hypotenuse going from upper right to lower left.  Triangles with lower
case names have their legs horizontal and vertical and the dimension
parameters are the lengths of the legs.  Those with capitalized names
have their hypotenuse horizontal or vertical.  In those cases, one of
\index{dx}\verb.dx. or \index{dy}\verb.dy. is the length of a leg and
the other is {\em half} the length of the hypotenuse.  In all cases, the
order of the nodes and of the arrows is linguistic, first moving from
left to right and then down.  The defaults are reasonable, but with
triangles, there is not always a natural direction for arrows.  I always
made mistakes in the order with my macros and this is certainly a
defect.  But the order is the same.  In every case the braces around the
shape specification can be removed unless it includes the following
delimiter (that is, \index{`}\verb.`. or \index{/}\verb./., as the case
may be.)
 \begin{verbatim}
\square(x,y)|pppp|/{sh}`{sh}`{sh}`{sh}/<dx,dy>%
 [N`N`N`N;L`L`L`L]
\ptriangle(x,y)|ppp|/{sh}`{sh}`{sh}/<dx,dy>[N`N`N;L`L`L]
\qtriangle(x,y)|ppp|/{sh}`{sh}`{sh}/<dx,dy>[N`N`N;L`L`L]
\dtriangle(x,y)|ppp|/{sh}`{sh}`{sh}/<dx,dy>[N`N`N;L`L`L]
\btriangle(x,y)|ppp|/{sh}`{sh}`{sh}/<dx,dy>[N`N`N;L`L`L]
\Atriangle(x,y)|ppp|/{sh}`{sh}`{sh}/<dx,dy>[N`N`N;L`L`L]
\Vtriangle(x,y)|ppp|/{sh}`{sh}`{sh}/<dx,dy>[N`N`N;L`L`L]
\Ctriangle(x,y)|ppp|/{sh}`{sh}`{sh}/<dx,dy>[N`N`N;L`L`L]
\Dtriangle(x,y)|ppp|/{sh}`{sh}`{sh}/<dx,dy>[N`N`N;L`L`L]
\Atrianglepair(x,y)|ppppp|/{sh}`{sh}`{sh}`{sh}`{sh}/%
<dx,dy>[N`N`N`N;L`L`L`L`L]
\Vtrianglepair(x,y)|ppppp|/{sh}`{sh}`{sh}`{sh}`{sh}/%
<dx,dy>[N`N`N`N;L`L`L`L`L]
\Ctrianglepair(x,y)|ppppp|/{sh}`{sh}`{sh}`{sh}`{sh}/%
<dx,dy>[N`N`N`N;L`L`L`L`L]
\Dtrianglepair(x,y)|ppppp|/{sh}`{sh}`{sh}`{sh}`{sh}/%
<dx,dy>[N`N`N`N;L`L`L`L`L]
\end{verbatim}
 \index{\backslash square} \index{\backslash ptriangle} \index{\backslash qtriangle} \index{\backslash dtriangle}
\index{\backslash btriangle} \index{\backslash Atriangle} \index{\backslash Vtriangle}
\index{\backslash Ctriangle} \index{\backslash Dtriangle} \index{\backslash Atrianglepair}
\index{\backslash Vtrianglepair} \index{\backslash Ctrianglepair} \index{\backslash Dtrianglepair}
Note that the
\verb.%.  signs are required if you break the
macro at such points.  See also the discussion of errant spaces above.

To make a diagram, you have to enclose it inside \index{\backslash xy ...
\backslash endxy}.\verb,\xy ...  \endxy,.  You will usually want it
displayed,
for which the simplest way is to enclose it in \index{$$\xy ...
\endxy$$}\verb,$$\xy...\endxy$$,.  For old times' sake, I have also let
\index{\backslash bfig}\verb.\bfig. and \index{\backslash efig}\verb.\efig. be synonyms for
\index{\backslash xy}\verb.\xy. and \index{\backslash endxy}\verb.\endxy., resp.  (In case
you wonder, these go all the way back to a main frame formatter running
at McGill when Charles Wells and I were first writing TTT, where
\verb,.BFIG, and \verb,.EFIG, were used to make display---at least as
far as that primitive formatter was capable of.)

\section{Examples}

Many people---including me---learn mainly by example and
I will give a number of examples here.  The formal
syntax that is not given here can be learned in the \xypic\ reference
manual. More samples from an actual paper can be found in
\verb,ftp.math.mcgill.ca/pub/barr/derfun.tex,.  If you want to compile
that paper, you will need \verb,tac.cls,, available from \verb,tac.ca,.
We begin with \index{\backslash morphism}
\begin{verbatim}
 $$\bfig
 \morphism[A`B;f]
 \morphism(0,300)[A`B;f]
 \morphism(0,600)|m|[A`B;f]
 \morphism(0,900)/<-/[A`B;f]
 \morphism(900,500)<0,-500>[A`B;f]
 \morphism(1200,0)<0,500>[A`B;f]
 \efig$$
\end{verbatim}
which gives the diagram
 $$\bfig
 \morphism[A`B;f]
 \morphism(0,300)[A`B;f]
 \morphism(0,600)|m|[A`B;f]
 \morphism(0,900)/<-/[A`B;f]
 \morphism(900,500)<0,-500>[A`B;f]
 \morphism(1200,0)<0,500>[A`B;f]
 \efig$$
\begin{verbatim}
 $$\bfig
\square[A`B`C`D;e`f`g`m]
 \efig$$
\end{verbatim}
 produces
 $$\bfig
\square[A`B`C`D;e`f`g`m]
 \efig$$
 This can be modified, for example
\begin{verbatim}
 $$\bfig
\square/>>`>`>` >->/[A`B`C`D;e`f`g`m]
 \efig$$
\end{verbatim}
 produces
 $$\bfig
\square/>>`>`>` >->/[A`B`C`D;e`f`g`m]
 \efig$$
 This can be put together with a morphism as follows:
\begin{verbatim}
 $$\bfig
\square/>>`>`>` >->/[A`B`C`D;e`f`g`m]
 \morphism(500,500)|m|/.>/<-500,-500>[B`C;h]
 \efig$$
\end{verbatim}
 \index{\backslash square}
 which makes a familiar diagram:
 $$\bfig
\square/>>`>`>` >->/[A`B`C`D;e`f`g`m]
 \morphism(500,500)|m|/.>/<-500,-500>[B`C;h]
 \efig$$
 The same diagram could have been made by
\begin{verbatim}
 $$\bfig
 \ptriangle|alm|/>>`>`.>/[A`B`C;e`f`h]
 \dtriangle/`>` >->/[B`C`D;`g`m]
 \efig$$
\end{verbatim}
\index{\backslash ptriangle} \index{\backslash dtriangle}

There are four macros for making pairs of triangles put together:
 $$\bfig
 \Vtrianglepair[A`B`C`D;f`g`h`k`l]
 \efig$$
 comes from
\begin{verbatim}
 $$\bfig
 \Vtrianglepair[A`B`C`D;f`g`h`k`l]
 \efig$$
\end{verbatim}
 \index{\backslash Vtrianglepair}
 \index{\backslash Atrianglepair}
 \index{\backslash Ctrianglepair}
 \index{\backslash Dtrianglepair}

 The other three are called \verb.\Atrianglepair.,
\verb.\Ctrianglepair., and \verb.\Dtrianglepair..

 You can fit two squares together, horizontally:
\begin{verbatim}
 $$\bfig
\square|almb|[A`B`C`D;f`g`h`k]
\square(500,0)/>``>`>/[B`E`D`F;l``m`n]
 \efig$$
\end{verbatim}
 $$\bfig
\square|almb|[A`B`C`D;f`g`h`k]
\square(500,0)/>``>`>/[B`E`D`F;l``m`n]
 \efig$$
 or vertically
\begin{verbatim}
 $$\bfig
\square(0,500)|alrm|[A`B`C`D;f`g`h`k]
\square/`>`>`>/[C`D`E`F;`l`m`n]
 \efig$$
\end{verbatim}
 $$\bfig
\square(0,500)|alrm|[A`B`C`D;f`g`h`k]
\square/`>`>`>/[C`D`E`F;`l`m`n]
 \efig$$
or a square and a triangle
\begin{verbatim}
 $$\bfig
\Ctriangle/<-`>`>/<400,400>[\hbox{\rm rec}(A,B)`B`X;r_0(A,B)`f`t_0]
 \square(400,0)/<-``>`<-/<1000,800>[\hbox{\rm rec}(A,B)`A\times\hbox{\rm
   rec}(A,B)`X`A\times X;r(A,B)``\hbox{\rm id}_A\times f`t]
 \efig$$
\end{verbatim}
 gives the diagram
 $$\bfig
\Ctriangle/<-`>`>/<400,400>[\hbox{\rm rec}(A,B)`B`X;r_0(A,B)`f`t_0]
 \square(400,0)/<-``>`<-/<1000,800>[\hbox{\rm rec}(A,B)`A\times\hbox{\rm
   rec}(A,B)`X`A\times X;r(A,B)``\hbox{\rm id}_A\times f`t]
 \efig$$
 This diagram is on page 361 of the third edition of Category Theory for
Computing Science to describe recursion.
 Here is an example using the procedure for sliding an arrow sideways.
This one could even be used in a text,
 \index{sliding arrows}
$\xy \morphism(0,0)|a|/@{>}@<3pt>/<400,0>[A`B;d]
\morphism(0,0)|b|/@{>}@<-3pt>/<400,0>[A`B;e]\endxy$
 which was made using
\begin{verbatim}
$\xy \morphism(0,0)|a|/@{>}@<3pt>/<400,0>[A`B;d]
\morphism(0,0)|b|/@{>}@<-3pt>/<400,0>[A`B;e]\endxy$
\end{verbatim}
 Indidentally, if you don't put this in math mode, the diagram will come
out too low, for reasons I do not understand but must be buried within
the \xypic\ code.
Later we will introduce a number of inline procedures.

Something a bit different that illustrates the use of another shaft

\index{=}\verb.=.  that gives a 2-arrow, as well as curved arrows:
\begin{verbatim}
 $$\bfig
\morphism(0,0)|a|/{@{>}@/^1em/}/<500,0>[A`B;f]
\morphism(0,0)|b|/{@{>}@/_1em/}/<500,0>[A`B;g]
\morphism(250,50)|a|/=>/<0,-100>[``]
 \efig$$
\end{verbatim}
 $$\bfig
\morphism(0,0)|a|/{@{>}@/^1em/}/<500,0>[A`B;f]
\morphism(0,0)|b|/{@{>}@/_1em/}/<500,0>[A`B;g]
\morphism(250,50)|a|/=>/<0,-100>[`;]
 \efig$$
 In order to use curved arrows, you must insert
\index{\backslash xyoption{curve}}\verb.\xyoption{curve}.
into your file. Here are two ways of doing three arrows between two
objects, depending on what you like:
\begin{verbatim}
 $$\bfig
 \morphism(0,0)|a|/@{>}@<5pt>/<500,0>[A`B;f]
 \morphism(0,0)|m|/@{>}/<500,0>[A`B;g]
 \morphism(0,0)|b|/@{>}@<-5pt>/<500,0>[A`B;h]
 \efig$$
\end{verbatim}
 which gives
 $$\bfig
 \morphism(0,0)|a|/@{>}@<5pt>/<500,0>[A`B;f]
 \morphism(0,0)|m|/@{>}/<500,0>[A`B;g]
 \morphism(0,0)|b|/@{>}@<-5pt>/<500,0>[A`B;h]
 \efig$$
and
\begin{verbatim}
 $$\bfig
 \morphism(0,0)|a|/{@{>}@/^5pt/}/<500,0>[A`B;f]
 \morphism(0,0)|m|/@{>}/<500,0>[A`B;g]
 \morphism(0,0)|b|/{@{>}@/^-5pt/}/<500,0>[A`B;h]
 \efig$$
\end{verbatim}
 which gives
 $$\bfig
 \morphism(0,0)|a|/{@{>}@/^5pt/}/<500,0>[A`B;f]
 \morphism(0,0)|m|/@{>}/<500,0>[A`B;g]
 \morphism(0,0)|b|/{@{>}@/^-5pt/}/<500,0>[A`B;h]
 \efig$$
 Either of these could also be used inline.

 There is a macro
\index{\backslash place}\verb.\place.
that places that object anywhere.  I have changed the name from
\verb.\put. in order to avoid clashing with the \LaTeX\ picture mode's
\index{\backslash put}\verb.\put..
Here is an example that uses a construction that is undocumented here,
but uses a documented \Xy\ construction:
\begin{verbatim}
\newbox\anglebox
\setbox\anglebox=\hbox{\xy \POS(75,0)\ar@{-} (0,0) \ar@{-} (75,75)\endxy}
 \def\angle{\copy\anglebox}
 $$\bfig
 \square[A`B`C`D;f`g`h`k]
 \place(100,400)[\angle]
 \efig$$
\end{verbatim}
\newbox\anglebox
\setbox\anglebox=\hbox{\xy \POS(75,0)\ar@{-} (0,0) \ar@{-} (75,75)\endxy}
 \def\angle{\copy\anglebox}
 $$\bfig
 \square[A`B`C`D;f`g`h`k]
 \place(100,400)[\angle]
 \efig$$
 Notice that you get a headless arrow by using
\index{\backslash ar@{-}}\verb.\ar@{-}..

Here is a special code installed at the request of Jonathon Funk:
\begin{verbatim}
 $$\bfig
\pullback|brrb|<800,800>[P`X`Y`Z;t`u`v`w]%
|amb|/>`-->`>/<500,500>[A;f`g`h]
 \efig$$
\end{verbatim}
 \index{\backslash pullback}
 $$\bfig
\pullback|brrb|<800,800>[P`X`Y`Z;t`u`v`w]%
|amb|/>`-->`>/<500,500>[A;f`g`h]
 \efig$$
The full syntax for this is
\begin{verbatim}
\pullback(x,y)|pppp|/{sh}`{sh}`{sh}`{sh}/<dx`dy>[N`N`N`N;L`L`L]%
    |ppp|/{sh}`{sh}`{sh}/<dx,dy>[N;L`L`L]
\end{verbatim}
 Of these only the nodes placed inside brackets are obligatory.  The
first sets of parameters are exactly as for
\index{\backslash square}\verb.\square.  and the
remaining parameters are for the nodes and labels of the outer arrows.
There is no positioning parameters for them; rather you set the
horizontal and vertical separations of the outer node from the square.

Here are some more special constructions.  In general, if you are doing
a square, you should use
\index{\backslash Square}\verb.\Square.  instead of
\index{\backslash square}\verb.\square.
because if figures its own width.  The syntax is almost the same, except
that
\index{dx}\verb.dx.  is omitted.  For example,
\begin{verbatim}
 $$\bfig
 \Square/^{ (}->`>`>`^{ (}->/<350>[{\rm Hom}(A,2^B)`{\rm Sub}(A\times B)`
 {\rm Hom}(A',2^{B'})`{\rm Sub}(A'\times B');\alpha(A,B)```\alpha(A',B')]
 \efig$$
\end{verbatim}
 will produce the square
 $$\bfig
 \Square/^{ (}->`>`>`^{ (}->/<350>[{\rm Hom}(A,2^B)`{\rm Sub}(A\times B)`
 {\rm Hom}(A',2^{B'})`{\rm Sub}(A'\times B');\alpha(A,B)```\alpha(A',B')]
 \efig$$
 There are a couple of points to note here.  Note the use of the
argument
\index{^{ (}->}\verb.^{ (}->.  to get the inclusion arrow.  The
complication is
created by the necessity of adding a bit of extra space before the hook.
You get pretty much the same effect by putting a bit of extra space
after the node:
\begin{verbatim}
 $$\bfig
 \Square/^(->`>`>`^(->/<350>[{\rm Hom}(A,2^B)\,`{\rm Sub}(A\times B)`
 {\rm Hom}(A',2^{B'})\,`{\rm Sub}(A'\times B');\alpha(A,B)```\alpha(A',B')]
 \efig$$
\end{verbatim}
 The full syntax is
\begin{verbatim}
 \Square(x,y)|pppp|/{sh}`{sh}`{sh}`{sh}/<dy>[N`N`N`N;L`L`L`L]
\end{verbatim}

There are also macros for placing two
\index{\backslash Square}\verb.\Square.s together
horizontally or vertically.  The first is
\index{\backslash hSquares}\verb.\hSquares.  with the
syntax
\begin{verbatim}
\hSquares(x,y)|ppppppp|/{sh}`{sh}`{sh}`{sh}`{sh}`{sh}`{sh}/%
<dy>[N`N`N`N`N`N;L`L`L`L`L`L`L]
\end{verbatim}
The second is
\index{\backslash vSquares}\verb.\vSquares.  with a similar syntex except that there
are two
\index{dy}\verb.dy.  parameters, one for each square:
\begin{verbatim}
\hSquares(x,y)|ppppppp|/{sh}`{sh}`{sh}`{sh}`{sh}`{sh}`{sh}/%
<dy,dy>[N`N`N`N`N`N;L`L`L`L`L`L`L]
\end{verbatim}
 Similarly, there are four macros for making pairs of triangles put
together.  For example,
 $$\bfig
 \Vtrianglepair[A`B`C`D;f`g`h`k`l]
 \efig$$
 comes from
\begin{verbatim}
 $$\bfig
 \Vtrianglepair[A`B`C`D;f`g`h`k`l]
 \efig$$
\end{verbatim}

There is a macro for making cubes\index{\backslash cube}.  The syntax is
\begin{verbatim}
 \cube(x,y)|pppp|/{sh}`{sh}`{sh}`{sh}/<dx,dy>[N`N`N`N;L`L`L`L]%
      (x,y)|pppp|/{sh}`{sh}`{sh}`{sh}/<dx,dy>[N`N`N`N;L`L`L`L]%
      |pppp|/{sh}`{sh}`{sh}`{sh}/[L`L`L`L]
\end{verbatim}
 The first line of parameters is for the outer square and the second for
the inner square, while the remaining parameters are for the arrows
between the squares.  Only the parameters in square brackets are
required; there are defaults for the others.  Here is an example:
\begin{verbatim}
 $$\bfig
\cube(0,0)|arlb|/ >->` >->`>`>/<1500,1500>[A`B`C`D;f`g`h`k]%
(300,300)|arlb|/>`>`>`>/<400,400>[A'`B'`C'`D';f'`g'`h'`k']%
 |mmmm|/<-`<-`<-`<-/[\alpha`\beta`\gamma`\delta]
 \efig$$
\end{verbatim}
 gives the somewhat misshapen diagram
 $$\bfig
\cube(0,0)|arlb|/ >->` >->`>`>/<1500,1500>[A`B`C`D;f`g`h`k]%
(300,300)|arlb|/>`>`>`>/<400,400>[A'`B'`C'`D';f'`g'`h'`k']%
 |mmmm|/<-`<-`<-`<-/[\alpha`\beta`\gamma`\delta]
 \efig$$
 because the parameters were oddly chosen.  The defaults center the
squares.  I discovered accidently, while debugging the cube that what I
thought was an out-of-range choice of parameters would produce an offset
cube:
\begin{verbatim}
 $$\bfig
\cube|arlb|/ >->` >->`>`>/<1000,1000>[A`B`C`D;f`g`h`k]%
(400,400)|arlb|/>`>`>`>/<900,900>[A'`B'`C'`D';f'`g'`h'`k']%
 |rrrr|/<-`<-`<-`<-/[\alpha`\beta`\gamma`\delta]
 \efig$$
\end{verbatim}
 gives
 $$\bfig
\cube|arlb|/ >->` >->`>`>/<1000,1000>[A`B`C`D;f`g`h`k]%
(400,400)|arlb|/>`>`>`>/<900,900>[A'`B'`C'`D';f'`g'`h'`k']%
 |r`r`r`r|/<-`<-`<-`<-/[\alpha`\beta`\gamma`\delta]
 \efig$$


\subsection{Complex diagrams}\label{complex}
In homological algebra one often has a $3\times3$ diagram, with or
without 0's on the margins.  There is a macro to do that:
 \index{3 by 3}\index{\backslash iiixiii}
\begin{verbatim}
 $$\bfig
\iiixiii(0,0)|aammbblmrlmr|/>`>`>`>`>`>`>`>`>`>`>`>/<500,500>{'5436}%
 <400,400>[A'`B'`C'`A`B`C`A''`B''`C'';f'`g'`f`g`f''`g''`u`v`w`u'`v'`w']
 \efig$$
\end{verbatim}
 that gives
 $$\bfig
\iiixiii(0,0)|aammbblmrlmr|/>`>`>`>`>`>`>`>`>`>`>`>/<500,500>{'5436}%
 <400,400>[A'`B'`C'`A`B`C`A''`B''`C'';f'`g'`f`g`f''`g''`u`v`w`u'`v'`w']
 \efig$$
 Here is the explanation.  The arrow parameters are not given in the
usual order, but rather first all the horizontal arrows and then all the
vertical ones, each in their usual order.  The nodes are given in the
usual order.  The fifth parameter is the octal number
\verb.`5436.   (In
\TeX, the
\verb.'.  introduces octal numbers and
\verb.".  introduces
hexadecimal numbers) which corresponds to the binary number
101,100,011,110 that describes the distribution of the 0's around the
margins.  The same results would have been obtained if the number had
been the hexadecimal number
\verb."B1E.  or the decimal number

\verb.2846..  The sixth parameter gives the horizontal and vertical
offset of the 0's, which you often want smaller than the others.  You
must not give the sixth parameter unless you have given a value (which
could be 0) to the fifth or an error condition will result.  Note that
the positioning parameters ignore the 0's so that it is the lower left
node that appears at the position
\verb.(x,y)..  As usual, there are
defaults.
 \begin{verbatim}
 $$\bfig
\iiixiii[A'`B'`C'`A`B`C`A''`B''`C'';f'`g'`f`g`f''`g''`u`v`w`u'`v'`w']
 \efig$$
 $$\bfig
\iiixiii%
 {'5436}[A'`B'`C'`A`B`C`A''`B''`C'';f'`g'`f`g`f''`g''`u`v`w`u'`v'`w']
 \efig$$
 $$\bfig
\iiixiii{'5436}<250,450>[A'`B'`C'`A`B`C`A''`B''`C'';%
f'`g'`f`g`f''`g''`u`v`w`u'`v'`w']
 \efig$$
\end{verbatim}
 $$\bfig
\iiixiii[A'`B'`C'`A`B`C`A''`B''`C'';f'`g'`f`g`f''`g''`u`v`w`u'`v'`w']
 \efig$$
 $$\bfig
\iiixiii%
 {'5436}[A'`B'`C'`A`B`C`A''`B''`C'';f'`g'`f`g`f''`g''`u`v`w`u'`v'`w']
 \efig$$
 $$\bfig
\iiixiii{'5436}<250,450>[A'`B'`C'`A`B`C`A''`B''`C'';%
f'`g'`f`g`f''`g''`u`v`w`u'`v'`w']
 \efig$$

A similar macro\index{3 by 2}%
\index{\backslash iiixii}\verb.\iiixii.  has been added for a map between exact
sequences, with parameters similar to the above.  An actual example is
\begin{verbatim}
 $$\bfig
 \iiixii|aaaalmr|<1000,800>[H`G`F`H\oplus H_0`G\oplus H_0\oplus F_0`
F\oplus F_0; f`g`\pmatrix{f&0\cr0&1\cr0&0}`\pmatrix{g&0&0\cr0&0&1}`
 \pmatrix{1\cr0}`\pmatrix{1\cr0\cr0}`\pmatrix{1\cr0}]
\efig$$
\end{verbatim}
                which gives
 $$\bfig
 \iiixii|aaaalmr|<1000,800>[H`G`F`H\oplus H_0`G\oplus H_0\oplus F_0`
F\oplus F_0; f`g`\pmatrix{f&0\cr0&1\cr0&0}`\pmatrix{g&0&0\cr0&0&1}`
 \pmatrix{1\cr0}`\pmatrix{1\cr0\cr0}`\pmatrix{1\cr0}]
\efig$$
 The general syntax is
\begin{verbatim}
 \iiixii(x,y)|ppppppp|/1`2`3`4`5`6`7/<dx,dy>{n}<dx'>[...]
\end{verbatim}
 with the usual meaning.  The number n is a number between 0 and 15
(default 15) that specifies whether and where 0's appear (think binary,
with the high bit at the upper left) and dx' specifies the separation of
the zeroes.  You get two squares side by side if both n and dx' are 0.

\subsection{Empty placement and moving labels}
 The label placements within \verb.|p|. is valid only for
\verb.x=a,b,r,l,m..  If you use any other value (or leave it empty) the
label entry is ignored, but you can use any valid \xypic\ label, as
described in Figure 13 of the reference manual.  One place you might
want to use this is for the placement of the labels along an arrow.  In
\xypic\ the default placement of the label is midway between the
midpoints of the nodes.  If the two nodes are of widely different sizes,
this can result in strange placements; therefore I always place them
midway along the arrow.  However, as the following illustrates, this can
be changed.
 \begin{verbatim}
 $$\bfig
\morphism(0,900)||/@{->}^<>(0.7){f}/<800,0>[A^B\times B^C\times C`C;]
\morphism(0,600)||/@{->}^<(0.7){f}/<800,0>[A^B\times B^C\times C`C;]
\morphism(0,300)||/@{->}^>(0.7){f}/<800,0>[A^B\times B^C\times C`C;]
\morphism(0,0)||/@{->}^(0.7){f}/<800,0>[A^B\times B^C\times C`C;]
 \efig$$
\end{verbatim}
 which produces
 $$\bfig
\morphism(0,900)||/@{->}^<>(0.7){f}/<800,0>[A^B\times B^C\times C`C;]
\morphism(0,600)||/@{->}^<(0.7){f}/<800,0>[A^B\times B^C\times C`C;]
\morphism(0,300)||/@{->}^>(0.7){f}/<800,0>[A^B\times B^C\times C`C;]
\morphism(0,0)||/@{->}^(0.7){f}/<800,0>[A^B\times B^C\times C`C;]
 \efig$$
 Here is the explanation.  The label placement argument is empty (it
cannot be omitted) and the arrow entry is empty.  However, placing
\verb.^(0.7){f}. inside the arrow shape places the label $f$ 7/10 of the
way between the nodes.  Unmodified, this places it 7/10 of the way
between the centers of the nodes.  This may be modified by
\index{<}\verb.<., which moves the first (here the left) reference point
to the beginning of the arrow, \index{>}\verb.>. which moves the second
reference point to the end of the arrow, or by both, which moves both
reference points.  In most cases, you will want both.  Incidentally,
\index{-}\verb.-. is a synonym for the sequence
\index{<>(}5).\verb.<>(.5). and that is the default placement in my
package.

Here are some more examples that illustrates the special sequence
\index{\backslash hole}\verb.\hole. used in conjunction with \index{|}\verb.|.
that implements \index{m}\verb.m. as well as the fact that these things
can be stacked.  For more details, I must refer you to the \xypic\
reference manual.
 \begin{verbatim}
 $$\bfig
\morphism(0,600)||/@{->}|-\hole/<800,0>[A^B\times B^C\times C`C;]
\morphism(0,300)||/@{->}|-\hole^<>(.7)f/<800,0>[A^B\times B^C\times C`C;]
\morphism(0,0)||/@{->}|-\hole^<>(.7)f_<>(.3)g/<800,0>[A^B\times
B^C\times C`C;]
 \efig$$
\end{verbatim}
 produces
 $$\bfig
\morphism(0,600)||/@{->}|-\hole/<800,0>[A^B\times B^C\times C`C;]
\morphism(0,300)||/@{->}|-\hole^<>(.7)f/<800,0>[A^B\times B^C\times C`C;]
\morphism(0,0)||/@{->}|-\hole^<>(.7)f_<>(.3)g/<800,0>[A^B\times
B^C\times C`C;]
 \efig$$

Here is another version of the cube we looked at above, using these
special placements and\index{\backslash cube}
\index{\backslash hole}\verb.\hole.'s to break some lines and make it
neater.
\begin{verbatim}
 $$\bfig
\cube|arlb|/@{ >->}^<>(.6){f}` >->`@{>}_<>(.4){h}`>/%
<1000,1000>[A`B`C`D;`g``k]%
(400,400)|axxb|/>`@{>}|!{(300,1000);(500,1000)}\hole^<>(.6){g'}`>`@{>}%
|!{(1000,500);(1000,300)}\hole_<>(.4){k'}/<900,900>[A'`B'`C'`D';f'``h'`]%
 |rrrr|/<-`<-`<-`<-/[\alpha`\beta`\gamma`\delta]
 \efig$$
\end{verbatim}
 $$\bfig
\cube|arlb|/@{ >->}^<>(.6){f}` >->`@{>}_<>(.4){h}`>/%
<1000,1000>[A`B`C`D;`g``k]%
(400,400)|axxb|/>`@{>}|!{(300,1000);(500,1000)}\hole^<>(.6){g'}`>`@{>}%
|!{(1000,500);(1000,300)}\hole_<>(.4){k'}/<900,900>[A'`B'`C'`D';f'``h'`]%
 |rrrr|/<-`<-`<-`<-/[\alpha`\beta`\gamma`\delta]
 \efig$$
 This one is probably worth saving as a template.  Later I will explain
the meaning of the strings \verb.!{(300,1000);(500,1000)}\hole. and
 \verb.!{(1000,500);(1000,300)}\hole. along with a caveat on their use.
If the nodes are unusually large, the cube may be magnified using
\index{\backslash scalefactor}\verb.\scalefactor.  .

\subsection{Inline macros}
 Here we illustrate a few of the macros for inline---or
displayed---equations the package contains.  In each case, the macro may
have a superscript or subscript or both (in which case the superscript
must come first) and the arrow(s) grow long enough to hold the super- or
subscript.  If you type\par\noindent
\verb.$A\to B\to^f C\to_g D\to^h_{{\rm Hom}(X,Y)} E$., you get
 $A\to B\to^f C\to_g D\to^h_{{\rm Hom}(X,Y)} E$.  Similarly, the macro
\index{\backslash toleft}\verb.\toleft. reverses the arrows.  The remaining macros
of this sort are \index{\backslash mon}\verb.\mon. which gives a monic arrow,
\index{\backslash epi}\verb.\epi. which gives an epic arrow,
\index{\backslash two}\verb.\two. that gives a pair of arrows, as well as
leftwards pointing versions, \index{\backslash monleft}\verb.\monleft.,
\index{\backslash epileft}\verb.\epileft., and \index{\backslash twoleft}\verb.\twoleft. of
each of them.  Here is one more example:
\begin{verbatim}
 $A\twoleft B\twoleft^f C\twoleft_g D\twoleft^h_{{\rm Hom}(X,Y)} E$
\end{verbatim}
 gives  $A\twoleft B\twoleft^f C\twoleft_g D\twoleft^h_{{\rm Hom}(X,Y)} E$.
There is an almost unlimited variety of such procedures possible.  The
ones that are provided can be used as templates to define new ones with,
say, curved arrows or three arrows or whatever a user might have need
of.

\subsubsection{Added:}\label{added} The macros \index{\backslash to}\verb.\to. and
\index{\backslash two}\verb.\two. can each have optional parameters of the form
\index{/{sh}/<dx>}\verb./{sh}/<dx>. and
\index{/{sh}`{sh}/<dx>}\verb./{sh}`{sh}/<dx>., resp. that allow you to
specify the shapes of the arrows and to override the automatic
computation of the lengths of the arrows.  For example,
 \begin{verbatim}
$$A\to/<-/ B\to^f C \to/ >->/<500>_g D\to/<-< /^f_g E$$
$$A\two/<-`->/<100> B\two^f C \two/ >->` >->/_g D\two/<-< `<-< /^f_g E$$
\end{verbatim}
 gives
$$A\to/<-/ B\to^f C \to/ >->/<500>_g D\to/<-< /^f_g E$$
$$A\two/<-`->/<100> B\two^f C \two/ >->` >->/_g D\two/<-< `<-< /^f_g E$$
 This renders \index{\backslash mon}\verb.\mon., \index{\backslash epi}\verb.\epi.,
\index{\backslash toleft}\verb.\toleft., \index{\backslash monleft}\verb.\monleft.,
\index{\backslash epileft}\verb.\epileft., and \index{\backslash twoleft}\verb.\twoleft.
obsolete, but they have been retained for back compatibility and
convenience.  A three arrow macro that works similarly has been added.
For example
 \begin{verbatim}
 $$A\threepppp/>`<-`>/<400>^{d^0}|{s^0}_{d^1}B\three<100>
C\three/->>`<-< `->>/ D$$
\end{verbatim}
 gives
 $$A\threepppp/>`<-`>/<400>^{d^0}|{s^0}_{d^1}B\three<100>
C\three/->>`<-< `->>/ D$$

\subsection{2-arrows}
 There is a macro for making 2-arrows of a fixed size, but varying
orientation.  They should be put at the appropriate position in a
diagram.  The two parameters are two integers \index{dx}\verb.dx. and
\index{dy}\verb.dy. whose ratio is the slope of the arrow.  They need
not be relatively prime, but arithmetic overflow could occur if they are
too large.  Note that although \index{(dx,dy)}\verb.(dx,dy). and
\index{(-dx,-dy)}\verb.(-dx,-dy). describe the same slope, the arrows
point in opposite directions.  Here is a sampler
 \begin{verbatim}
 $$\bfig
 \place(0,0)[\twoar(1,0)]
 \place(200,0)[\twoar(0,1)]
 \place(400,0)[\twoar(1,1)]
 \place(600,0)[\twoar(0,-1)]
 \place(800,0)[\twoar(1,2)]
 \place(1000,0)[\twoar(1,3)]
 \place(1200,0)[\twoar(1,-3)]
 \place(1400,0)[\twoar(-3,1)]
 \place(1600,0)[\twoar(-1,-3)]
 \place(1800,0)[\twoar(255,77)]
 \efig$$
\end{verbatim}
 $$\bfig
 \place(0,0)[\twoar(1,0)]
 \place(200,0)[\twoar(0,1)]
 \place(400,0)[\twoar(1,1)]
 \place(600,0)[\twoar(0,-1)]
 \place(800,0)[\twoar(1,2)]
 \place(1000,0)[\twoar(1,3)]
 \place(1200,0)[\twoar(1,-3)]
 \place(1400,0)[\twoar(-3,1)]
 \place(1600,0)[\twoar(-1,-3)]
 \place(1800,0)[\twoar(255,77)]
 \efig$$

Here is little amusement.
\begin{verbatim}
 $$\bfig
 \square/@3{->}`~)`=o`--x/[A`B`C`D;```]
 \place(400,100)[\twoar(-1,-1)]
 \place(100,400)[\twoar(1,1)]
 \morphism(500,500)||/{*}.{*}/<-500,-500>[B`C;]
 \efig$$
\end{verbatim}
 $$\bfig
 \square/@3{->}`~)`=o`--x/[A`B`C`D;```]
 \place(400,100)[\twoar(-1,-1)]
 \place(100,400)[\twoar(1,1)]
 \morphism(500,500)||/{*}.{*}/<-500,-500>[B`C;]
 \efig$$

\subsection{Mixing \xypic\ code}
 Here is a sample in which I have mixed code from \xypic\ with my own.
\let\tilde\widetilde
\let\hat\widehat
\begin{verbatim}
 $$\bfig
 \square(1500,500)/>`>`>`@{>}^<>(.2){r_{n-1}}/[T_{n-1}`T_{n-2}`R_{n-1}
`R_{n-2}; t_{n-1}``\sigma_{n-2}`]
 \square(1500,0)/`>`>`>/[R_{n-1}`R_{n-2}`S_{n-1}`S_{n-2};
`f_{n-1}`f_{n-2}`s_{n-1}]
 \morphism|b|[\tilde S_{n+1}`S_n;\tilde s_{n+1}]
 \square(1000,500)/>`>`>`>/[\tilde T_n`T_{n-1}`\tilde
R_n`{R_{n-1}};\tilde t_n`\tilde \sigma_n`\sigma_{n-1}`]
 \square(1000,0)/>`>``>/[\tilde R_n`R_{n-1}`\tilde S_n`{S_{n-1}};\tilde
r_n\quad ```\tilde s_n]
 \square(500,0)/>`>``>/[R_n`{R_n}`{S_n}`{S_{n}};
\hat r_n`f_n`\tilde f_n`\hat s_n]
 \POS(1500,1000)*+!!<0ex,.75ex>{T_{n-1}}
 \ar@{-}|!{(1000,500);(1500,500)}\hole(1167,334)%
 \POS(1167,334)\ar|!{(1000,500);(1000,0)}\hole_<>(.6){h_{n-1}}
  (500,0)*+!!<0ex,.75ex>{S_n}
 \morphism(2000,1000)/@{>}|\hole^<>(.8){h_{n-2}}/%
 <-500,-1000>[T_{n-2}`S_{n-1};]
 \efig$$
\end{verbatim}
 $$\bfig
 \square(1500,500)/>`>`>`@{>}^<>(.2){r_{n-1}}/[T_{n-1}`T_{n-2}`R_{n-1}
`R_{n-2}; t_{n-1}``\sigma_{n-2}`]
 \square(1500,0)/`>`>`>/[R_{n-1}`R_{n-2}`S_{n-1}`S_{n-2};
`f_{n-1}`f_{n-2}`s_{n-1}]
 \morphism|b|[\tilde S_{n+1}`S_n;\tilde s_{n+1}]
 \square(1000,500)/>`>`>`>/[\tilde T_n`T_{n-1}`\tilde
R_n`{R_{n-1}};\tilde t_n`\tilde \sigma_n`\sigma_{n-1}`]
 \square(1000,0)/>`>``>/[\tilde R_n`R_{n-1}`\tilde S_n`{S_{n-1}};\tilde
r_n\quad ```\tilde s_n]
 \square(500,0)/>`>``>/[R_n`{R_n}`{S_n}`{S_{n}};
\hat r_n`f_n`\tilde f_n`\hat s_n]
 \POS(1500,1000)*+!!<0ex,.75ex>{T_{n-1}}
 \ar@{-}|!{(1000,500);(1500,500)}\hole(1167,334)%
 \POS(1167,334)\ar|!{(1000,500);(1000,0)}\hole_<>(.6){h_{n-1}}
  (500,0)*+!!<0ex,.75ex>{S_n}
 \morphism(2000,1000)/@{>}|\hole^<>(.8){h_{n-2}}/%
 <-500,-1000>[T_{n-2}`S_{n-1};]
 \efig$$
 There are three points to note here in connection with the two lines
that begin with \index{\backslash POS}\verb.\POS.  First the objects that are the
source of the first and the target of the second are preceded by
\verb;!!<0ex,.75ex>;\verb;!!<0ex,.75ex>;.  The effect is to reset the
baseline to the baseline of the object (rather than the vertical center)
and then to lower that by 3/4 of xheight so that the arrow goes in the
right place.  This string precedes all objects.  Without that, an object
like $\tilde R $ would be set lower that $R$.  Second, the first arrow
has no target and the second no source.  This does not give the same
result as empty source and target since in the latter cases, there would
be spaces allowed around them and then the two lines would not meet.  It
would be possible to add code that tests for empty nodes, but it comes
up so seldom that I have refrained.  In the meantime, the only recourse
is to revert to the underlying \xypic\ code.  Thirdly the string
\verb.|!{(1000,500);(1500,500)}\hole. specifies that the line should be
broken at the place where the current arrow intersects the line between
the nodes located at (1000,500) and (1500,500).  One must be careful
using this construction, however, as it does not seem to work correctly
if the line segment fails to intersect the current line, or if it does
intersect, but happens to be too long.  I have not worked out how long
is too long, but you can get odd results.  I assume that this bug will
be fixed eventually.  (Ross Moore says that it works correctly in the
version he has, which, however, has not been released.)  There is a
similar string, with similar effect, in the following line.  The last
line uses simply \index{|\hole}\verb.|\hole. which positions the gap in
the middle of the arrow.

\subsection{Diagram from TTT}
 The last example is a complicated diagram from TTT.  If you have the
documentation from the old diagram macros (or the errata from TTT), you
can see how much easier it is to describe this diagram with these
macros.  Note the use of \index{\backslash scalefactor}\verb.\scalefactor. to
change the default length from 500 to 700 that made it unnecessary to
specify the scales on the squares and triangles.
 \begin{verbatim}
 $$\bfig
 \scalefactor{1.4}%
 \qtriangle(0,1000)/>`>`/[TT`T`TTT';\mu`TT\eta'`]%
 \btriangle(500,1000)/`>`@<-14\ul>/[T`TTT'`TT';`T\eta`T\sigma]%
 \morphism(0,1500)|l|/>/<0,-1000>[TT`TT'T;T\eta'T]%
 \square(500,500)|ammx|/@<14\ul>`>`>`/[TTT'`TT'`TT'TT'`TT'T';%
    \mu T'`T\eta'TT'`T\eta'T'`]%
 \morphism(1000,1000)|r|/>/<500,-500>[TT'`TT';\hbox{\rm id}]%
 \square/>`>``>/[TT'T`TT'TT'`T'T`T'TT';TT'T\eta'`\sigma T``T'T\eta']%
 \square(500,0)|ammb|[TT'TT'`TT'T'`T'TT'`T'T';%
    TT'\sigma`\sigma TT'`\sigma T'`T'\sigma T']%
 \square(1000,0)/>``>`>/[TT'T'`TT'`T'T'`T';T\mu'``\sigma`\mu']%
\place(500,1250)[1]\place(215,1000)[2]\place(750,750)[3]%
\place(215,250)[4]\place(750,250)[5]\place(1140,750)[6]%
\place(1250,250)[7]%
 \efig$$
\end{verbatim}
 $$\bfig\label{TTTdiag}
 \scalefactor{1.4}%
 \qtriangle(0,1000)/>`>`/[TT`T`TTT';\mu`TT\eta'`]%
 \btriangle(500,1000)/`>`@<-14\ul>/[T`TTT'`TT';`T\eta`T\sigma]%
 \morphism(0,1500)|l|/>/<0,-1000>[TT`TT'T;T\eta'T]%
 \square(500,500)|ammx|/@<14\ul>`>`>`/[TTT'`TT'`TT'TT'`TT'T';\mu T'`%
    T\eta'TT'`T\eta'T'`]%
 \morphism(1000,1000)|r|/>/<500,-500>[TT'`TT';\hbox{\rm id}]%
 \square/>`>``>/[TT'T`TT'TT'`T'T`T'TT';TT'T\eta'`\sigma T``T'T\eta']%
 \square(500,0)|ammb|[TT'TT'`TT'T'`T'TT'`T'T';TT'\sigma`\sigma TT'%
    `\sigma T'`T'\sigma T']%
 \square(1000,0)/>``>`>/[TT'T'`TT'`T'T'`T';T\mu'``\sigma`\mu']%
\place(500,1250)[1]\place(215,1000)[2]\place(750,750)[3]%
\place(215,250)[4]\place(750,250)[5]\place(1140,750)[6]%
\place(1250,250)[7]%
 \efig$$


\begin{theindex}

  \item \verb@%@, 3
  \item \verb@(@, 7
  \item \verb@(@,), 7
  \item \verb@(-dx@,-dy), 25
  \item \verb@(dx@,dy), 25
  \item \verb@)@, 7
  \item \verb@*@, 7
  \item \verb@-@, 6, 23
  \item \verb@--@, 6
  \item \verb@-->>@, 7
  \item \verb@->@, 7
  \item \verb@->>@, 7
  \item \verb@/@, 7, 9
  \item \verb@/{sh}/<dx>@, 25
  \item \verb@/{sh}`{sh}/<dx>@, 25
  \item \verb@;@, 7
  \item \verb@<@, 6, 7, 22
  \item \verb@<-@, 7
  \item \verb@<-< @, 7
  \item \verb@<<-@, 7
  \item \verb@<<--@, 7
  \item \verb@<>(@, 23
  \item \verb@=@, 6, 13
  \item \verb@>@, 6, 7, 22
  \item \verb@>>@, 6
  \item \verb@[@, 7
  \item \verb@\startsection@, 5
  \item \verb@\Atriangle@, 9
  \item \verb@\Atrianglepair@, 9, 12
  \item \verb@\Ctriangle@, 8, 9
  \item \verb@\Ctrianglepair@, 9, 12
  \item \verb@\Dtriangle@, 9
  \item \verb@\Dtrianglepair@, 9, 12
  \item \verb@\POS@, 27
  \item \verb@\Square@, 15, 16
  \item \verb@\Vtriangle@, 8, 9
  \item \verb@\Vtrianglepair@, 9, 12
  \item \verb@{-}@, 14
  \item \verb@\bfig@, 9
  \item \verb@\btriangle@, 9
  \item \verb@\cube@, 16, 23
  \item \verb@\dtriangle@, 9, 11
  \item \verb@\efig@, 9
  \item \verb@\endxy@, 9
  \item \verb@\epi@, 24, 25
  \item \verb@\epileft@, 24, 25
  \item \verb@\fontscale{x}@, 4
  \item \verb@\hSquares@, 16
  \item \verb@\halign@, 2
  \item \verb@\hole@, 23
  \item \verb@\ignorespaces@, 3
  \item \verb@\iiixii@, 21
  \item \verb@\iiixiii@, 18
  \item \verb@\let\labelstyle=\textstyle@, 6
  \item \verb@\mon@, 24, 25
  \item \verb@\monleft@, 24, 25
  \item \verb@\morphism@, 2, 5, 8, 10
  \item \verb@\phantom@, 3
  \item \verb@\place@, 8, 14
  \item \verb@\ptriangle@, 8, 9, 11
  \item \verb@\pullback@, 15
  \item \verb@\put@, 14
  \item \verb@\qtriangle@, 8, 9
  \item \verb@\scalefactor@, 24, 28
  \item \verb@\scriptstyle@, 6
  \item \verb@\square@, 8, 9, 11, 15
  \item \verb@\three@, 1
  \item \verb@\to@, 1, 25
  \item \verb@\toleft@, 24, 25
  \item \verb@\two@, 1, 24, 25
  \item \verb@\twoleft@, 24, 25
  \item \verb@\ul@, 6
  \item \verb@\vSquares@, 16
  \item \verb@\xy@, 9
  \item \verb@\xy ... \endxy@, 9
  \item \verb@\xyoption{curve}@, 13
  \item \verb@]@, 7
  \item \verb@^@, 6
  \item \verb@^{ (}->@, 16
  \item \verb@_@, 6
  \item \verb@`@, 7, 9
  \item \verb@{*}@, 7
  \item \verb@3 by 2@, 21
  \item \verb@3 by 3@, 18

  \indexspace

  \item \verb@ >->@, 7

  \indexspace

  \item \verb@a@, 6

  \indexspace

  \item \verb@b@, 6

  \indexspace

  \item \verb@dx@, 8, 15, 25
  \item \verb@dy@, 8, 16, 25

  \indexspace

  \item \verb@l@, 6
  \item \verb@label@, 6

  \indexspace

  \item \verb@m@, 6, 23

  \indexspace

  \item \verb@r@, 6

  \indexspace

  \item \verb@sliding arrows@, 13

  \indexspace

  \item \verb@tips@, 4

  \indexspace

  \item \verb@x@, 4, 8

  \indexspace

  \item \verb@y@, 8

\end{theindex}


\end{document}




