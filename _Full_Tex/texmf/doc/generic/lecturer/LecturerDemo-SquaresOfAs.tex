% This is a demonstration file distributed with the
% Lecturer package (see lecturer-doc.pdf).
%
% You can recompile the file with a basic TeX implementation,
% using pdfTeX or LuaTeX with the plain format.
% 
% There's no cutoff point for the reusable part here
% because code changes between slides. It is easily
% retrievable, though.
%
% Author: Paul Isambert.
% Date: July 2010.


% For the fonts with LuaTeX.
% Comment this first line out if you want to use pdfTeX,
% and don't forget to change the fonts!
\input luaotfload.sty
\input lecturer

\hfuzz=\maxdimen
\vfuzz=\maxdimen % We might have some bad boxes...
\lineskiplimit=-1000pt % So baseline distances are respected, no matter what.

% The measure. Could be anything.
\newdimen\squarewidth
\squarewidth=4cm
\newdimen\squaresep
\squaresep=\dimexpr\squarewidth/6

% Uncomment these to show the grid.
%\showgrid{.5\squaresep}
%\showgrid{\squaresep}[red]

\setparameter job:
  background = white
  mode       = presentation
  fullscreen = true

\newcolor{grey}{grey}{.7}
\setparameter slide:
  width height = "3\squarewidth+4\squaresep"
  bottom       = 0cm
% This is just for the last slide.
% The negative left is to accomodate the a's spur.
  left         = "-.8\squaresep"
  right        = 0cm
  hpos         = "rr"
  vpos         = top
  foreground   = grey
  background   = white

\setarea{area1 area2 area3 area4 area5 area6 area7 area8 area9}
  width height = \squarewidth
  left right   = ".3\squaresep"
  hpos         = rr
  baselineskip = ".18\squarewidth"
  topskip      = ".1\squarewidth"
  vpos         = center
  visible      = step % This will be changed after slide 1
  background   = grey
  foreground   = white
  font         = \mainfont
  frame        = "width=-.5\squaresep, corner=round"

\setarea{area1 area2 area3}
  vshift = \squaresep

\setarea{area4 area5 area6}
  vshift = "\squarewidth+2\squaresep"

\setarea{area7 area8 area9}
  vshift* = \squaresep

\setarea{area1 area4 area7}
  hshift = \squaresep

\setarea{area2 area5 area8}
  hshift = "\squarewidth+2\squaresep"

\setarea{area3 area6 area9}
  hshift* = \squaresep

\setarea{area8} % The area for the big a's.
  topskip           = \squarewidth
  vpos              = bottom
  left right        = 0pt
  font              = \afont % This font will change at each slide



% Some shorthands.
\long\def\dosquare#1#2{%
  \position{area#1}{#2}%
  }
\long\def\Step#1#2{%
  \step\dosquare{#1}{#2}%
  }
% #2 is the font
\def\doasquare#1#2{\position{area#1}{#2a}}




\font\mainfont=cmssbx10 at .3\squarewidth
\font\afont=cmssbx10 at 1.93\squarewidth



\slide[Introduction]

\Step1{I}
\Step2{really}
\Step3{find}
\Step6{the}
\Step9{letter}
\step\position{area8}{a}
\Step7{fascinating}
\Step4{don't}
\Step5{you?}

\endslide






\setarea{area1 area2 area3 area4 area5 area6 area7 area8 area9}
  visible = true

\font\mainfont=cmssbx10 at .2\squarewidth
\font\afont=cmr10 at 2.04\squarewidth

\slide[Computer Modern]

\position{area8}{a}
\dosquare1{For instance, here's one.}
\Step4{That's Computer Modern.}
\Step3{(Guy with the computer?)}
\Step9{See how it curls?}
\Step7{Like it has a cowlick.}
\Step2{Not very fond of it but...}
\Step6{... it has a charm of its own.}
\Step5{A study in \hbox{roundness}.}

\endslide







\font\afont=Electra.ttf at 2.09\squarewidth

\slide[Electra]

\position{area8}{a}
\dosquare7{And here's an Electra.}
\Step9{It stands proudly upright.}
\Step6{Square shoulder(s)...}
\Step4{... and an impatient spur.}
\Step5{A letter of purpose.}
\Step3{A hint of insolence.}
\Step1{Its bowl makes it friendly, though.}
\Step2{It jumps right out of the book.}

\endslide







\font\afont=Figural.ttf at 1.98\squarewidth

\slide[Figural]

\position{area8}{a}
\dosquare5{That's a Figural.}
\Step6{Built with strength too...}
\Step1{... out of pen and hand.}
\Step7{The bowl isn't larger than the arm.}
\Step9{Hence the squareness.}
\Step3{And itself is a rectangle.}
\Step2{(Meaning the bowl.)}
\Step4{The beauty of humanism.}

\endslide





% Now all areas except 3 and 5 take a big a.

\font\afont=Walbaum.ttf at 1.82\squarewidth
\font\chap=ChaparralPro-Regular.otf at 2.13\squarewidth
\font\cent=Centaur.ttf at 2.41\squarewidth
\font\sab =Sabon.ttf at 2\squarewidth
\font\four=FournierMTStd-Regular.ttf at 2.3\squarewidth
\font\espr=Esprit.ttf at 1.98\squarewidth
\font\mend=Mendoza.ttf at 1.83\squarewidth


\setarea{area1 area2 area4 area6 area7 area8 area9}
  topskip           = \squarewidth
  vpos              = bottom
  left right        = 0pt

% Freely positioned material is not meant
% to interact nicely with the area's vpos,
% since it doesn't really exist and can occur
% anywhere among real material. Hence this hack.
% 
\long\def\Dosquare#1{%
  \position{area3}[0cm,0cm]{\vbox to \squarewidth{\vfil#1\vfil}}%
  }
\setarea{area3}
  topskip = \squarewidth
  vpos    = top

%
% The designers in the central square.
%
\font\designers=cmssbx10 at .065\squarewidth
\def\Font#1: #2, #3.{#2 (#1, #3)}
\setarea{area5}
  topskip baselineskip = ".065\squarewidth"
  vpos                 = top
  left right           = ".01\squarewidth"
  hpos                 = fr
  font                 = \designers



\slide[Showcase]

\step[visible=true,off=chap]
\Dosquare{An intimidating Walbaum}
\step[visible=true]\position{area8}{a}

\step[chap,off=cent]
\Dosquare{A deceptively na\"\i ve Chaparral}
\step[on=chap]\doasquare4\chap

\step[cent,off=sab]
\Dosquare{A s(w)inging Centaur}
\step[on=cent]\doasquare6\cent

\step[sab,off=four]
\Dosquare{A no-nonsense Sabon}
\step[on=sab]\doasquare2\sab

\step[four,off=espr]
\Dosquare{A delicate Fournier}
\step[on=four]\doasquare9\four

\step[espr,off=mend]
\Dosquare{A whimsical Esprit}
\step[on=espr]\doasquare7\espr

\step[mend,off=des]
\Dosquare{An open Mendoza}
\step[on=mend]\doasquare1\mend

\step[des]
\Dosquare{And the designers of all these fonts}
\dosquare5{%
  \Font Computer Modern: Donald Knuth, past 30 years.,
  \Font Electra: W.A. Dwiggins, 1935.,
  \Font Figural: Old\v rich Menhart, 1940.,
  \Font Walbaum: Justus Erich Walbaum, 1800.,
  \Font Chaparral: Carol Twombly, 1997.,
  \Font Centaur: Bru\-ce Rogers, 1914 after Jenson, 1469.,
  \Font Sabon: Jan Tschichold, 1964 after Sabon, 16th century.,
  \Font Fournier: Pierre-Simon Fournier, 18th century.,
  \Font Esprit: Jovica Veljovi\'c, 1985.,
  \Font Mendoza: Jos\'e Mendoza y Al\-meida, 1991..
  }

\endslide



% And one big final a. The title of the slide
% comes from my impression that this glyph
% is almost an abstract symbol without any relation to
% the letter (especially when compared to the others
% on the previous slide, actually). Ok, a glyph is
% an abstract anyway, but you see what I mean.


\setarea{area4}
  background = white
  foreground = grey

\font\chapp=ChaparralPro-Regular.otf at 2.275\pdfpageheight

\slide[Abstraction,areas=area4,font=\chapp]

a

\endslide


\bye