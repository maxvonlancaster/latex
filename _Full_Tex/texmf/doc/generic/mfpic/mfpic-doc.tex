%%% File: mfpic-doc.tex
%%% A part of mfpic 1.05 2010/06/10
%%%
%  Documentation of mfpic macros
\documentclass[letterpaper]{article}

% Fonts: TimesRoman, CM Sans serif, and LuxiMono for TeX commands.
\usepackage[T1]{fontenc}
\usepackage{mathptmx}
\usepackage[scaled=.85]{luximono}
\renewcommand\sfdefault{cmss}
\newcommand\sgn{\mathop{\mathrm{sgn}}\nolimits}
% Fake chapters (really sections):
\usepackage[chapters]{mfpdoc}
\pagestyle{mfpdoc}

\usepackage{makeidx}
\makeindex

\usepackage{graphics}

\ifpdf
\expandafter\usepackage\expandafter
  [\mfpHyOpts,pdfpagelabels=true,hyperindex]{hyperref}
\expandafter\pdfstringdefDisableCommands\expandafter
  {\mfpHyDisable}
\fi

\stepcounter{secnumdepth}

\title{\Mfp{}: Pictures in \TeX{}\\ with Metafont and
MetaPost\thanks{Copyright 2002--2010, Daniel H. Luecking}}

\author{%
Daniel H. Luecking%
    \thanks{\email {luecking@uark.edu}: Communications regarding \mfp{}
    should be sent to this author. Any first-person references in this
    manual refer to Dr.~Luecking.}
\and Thomas E. Leathrum
\and Geoffrey Tobin}

\date{\mfpdate}

\begin{document}

\pagenumbering{roman}
\maketitle
\tableofcontents

\clearpage
\pagenumbering{arabic}

\chapter{Introduction}\label{introduction}
\thispagestyle{plain}

\section{Why?}\label{why}

Tom got the idea for \mfp{}%
    \footnote{`\Mfp{}' is pronounced by spelling the first two letters:
    `em-eff-pick'.}
mostly out of a feeling of frustration. Different output mechanisms for
printing or viewing \TeX{} DVI files each have their own ways to include
pictures.  More often than not, there are provisions for including
graphic objects into a \prog{DVI} file using \TeX{} \cs{special}'s.
However, this technique seemed far from \TeX{}'s ideal of device
independence because different \TeX{} output drivers recognize different
\cs{special's}, and handle them in different ways.

\LaTeX{}'s \env{picture} environment has a hopelessly limited supply of
available objects to draw---if you want to draw a graph of a polynomial
curve, you're out of luck.

There was, of course, \PiCTeX{}, which was wonderfully flexible and
general, but its most obvious feature was its speed---or rather lack of
it. Processing a single picture in \PiCTeX{} (in those days) could often
take several seconds.

It occurred to Tom that it might be possible to take advantage of the
fact that \MF{} is \emph{designed} for drawing things. The result of
pursuing this idea was \mfp{}, a set of macros for \TeX{} and
\MF{} which incorporate \MF{}-drawn pictures into a \TeX{} file.

With the creation of \MP{} by John Hobby, and the almost universal
availability of free \PS{} interpreters like \GS, some \mfp{} users
wanted to run their \mfp{} output through \MP{}, to produce \PS{}
pictures. Moreover, users wanted to be able to use \pdfTeX{}, which did
not get along well with PK fonts, but was quite happy with \MP{}
pictures. So \MP{} support was added to \mfp{}. This got us a little bit
away from device independence, but many users were not much concerned
with that: they just wanted a convenient way to have text and pictures
described in the same document file.

With the extra capabilities of \PS{} (e.g., color) and the corresponding
abilities of \MP{}, there was a demand for some \mfp{} interface to
access them. Consequently, switches (options) have been added to access
some of them. When these are used, output files may no longer be
compatible with \MF{}.

\section{Who?}\label{author}

The original \mfp{} (and still the core of the current version) was written
primarily by Tom Leathrum during the late (northern hemisphere) spring
and summer of 1992, while at Dartmouth College. Different versions were
being written and tested for nearly two years after that, during which
time Tom finished his Ph.D. and took a job at Berry College, in Rome,
GA.  Between fall of 1992 and fall of 1993, much of the development was
carried out by others.  Those who helped most in this process are
credited in the Acknowledgements.

Somewhere in the mid 1990's the development passed to Geoffrey Tobin who
kept things going for several years.

The addition of \MP{} support was carried out by Dan Luecking around
1997--99. He is also responsible for all other additions and changes
since then, with help from Geoffrey and a few others mentioned in the
Acknowledgements.

\section{What?}\label{manifest}

See the \file{README} file for a list of files in the distribution and a
brief explanation of each. Only four are actually needed for full access
to \mfp{}'s capabilities: \file{mfpic.dtx}, \file{mfpic.ins},
\file{grafbase.dtx} and \file{mfppatch.tex}. Running \LaTeX{} on
\file{mfpic.ins} creates the only required files:
\begin{display}
    \file{mfpic.tex} and \file{mfpic.sty}, the latter required only for
        \LaTeX{}.\\
    \file{grafbase.mf}, required only if \MF{} will be processing
        figures.\\
    \file{grafbase.mp} and \file{dvipsnam.mp}, needed only if \MP{} will
        be the processor.\\
    \file{mfppatch.tex} is used to distribute simple bug fixes. It does
        nothing after a major update, but if it is not installed and a
        previous one is, a warning will be issued.
\end{display}
The README file also gives some guidence on the proper location for the
installation of these files.

\section{How?}\label{process}

Some guidance on writing files that contain \mfp{} figures can be found
in the accompanying file \file{mfpguide.pdf}. If you use \mfp{} to
produce \MP{} figures the process is straightforward: run \TeX{} (or
\LaTeX), then \MP{}, then \TeX{} again. If there are no errors, then
\prog{dvips} or other DVI-to-PS converter can be run to produce
viewable\slash printable output. You can also run \prog{dvipdfm(x)} to obtain PDF
output, or even use \pdfTeX{} instead of \TeX{} (or \pdfLaTeX{} instead
of \LaTeX{}) to get PDF output directly.

Here is an example of the process: for the sample file
\file{pictures.tex}, first run \TeX{} on it (or run \LaTeX{} on
\file{lapictures.tex}). You may see a message from \mfp{} that there is
no file \file{pics.1}, but \TeX{} will continue processing the file
anyway. When \TeX{} is finished, you will now have a file called
\file{pics.mp}. This is the \MP{} file containing the descriptions of
the pictures for \file{pictures.tex}. You need to run \MP{} on
\file{pics.mp} (Read your \MF{} manual to see how to do
this.%
    \footnote{The document \textit{Some experiences on running Metafont
    and MetaPost}, by Peter Wilson, can be useful for beginners. Fetch
    \file{CTAN/info/metafp.pdf}. `\file{CTAN}' means the Comprehensive
    \TeX{} Archive Network. You can access \file{CTAN} by pointing your
    web browser at \url{http://mirror.ctan.org/}\,.}) %
Typically, you just type
\begin{verbatim}
mpost pics.mp
\end{verbatim}
(or possibly ``\verb$mp pics.mp$'', but try ``\verb$mpost$'' first).

This produces files \file{pics.1}, \file{pics.2}, etc., the number of
files depending on the version of \file{pictures.tex}. You then
reprocess \file{pictures.tex} with \TeX{} to produce a DVI file. This
file can then be processed with \prog{dvips} (for example) to produce
\PS{} output which can be printed or viewed. One can also process the
DVI with \prog{dvipdfm(x)} to produce a PDF file.

If \pdfTeX{} is used instead of \TeX{} on the second run, you should be
able to view the resulting PDF file immediately, without any further
processing.

If instead you use \mfp{} to produce \MF{} figures, things are a little
less straightforward. The process is \TeX{}, then \MF{}, then
\prog{gftopk}, then \TeX{} again. After this, \TeX{}'s DVI output ought
to be viewable and printable by most DVI viewers or printer drivers. For
a few \TeX{} systems there may be some prior setup needed. One needs to
convince \TeX{} and its output drivers to find \MF{}'s output files. You
should do whatever is necessary (perhaps nothing!) to insure that \TeX{}
looks in the current directory for \file{.tfm} files, and that your DVI
drivers look in the current directory for \file{.pk} files. There may
also be some setup needed to ensure that the \file{.pk} files are
created at a resolution that matches that of your printer and of your
DVI viewer. See the discussion in \file{mfpguide.pdf}.

If you want to test this process on the supplied sample files, edit
\file{pictures.tex} removing the \cs{usemetapost} command (or edit
\file{lapictures.tex}, removing the \opt{metapost} option). After that,
run \TeX{} on \file{pictures.tex} (or run \LaTeX{} on
\file{lapictures.tex}). You may see a message from \mfp{} that there is
no file \file{pics.tfm}, but \TeX{} will continue processing the file.
When \TeX{} is finished, you will now have a file called \file{pics.mf}.
This is the \MF{} file containing the descriptions of the pictures for
\file{pictures.tex}. You need to run \MF{} on \file{pics.mf}, with
\texttt{mode:=localfont} set up.
(Read your \MF{} manual to see how to do this.%
    \footnote{If you are new to running \MF{}, the document
    \textit{Metafont for Beginners}, by Geoffrey~Tobin, is a good
    start. Fetch \file{CTAN/info/metafont-for-beginners.tex}.}) %
Typically, you just type
\begin{verbatim}
mf pics.mf
\end{verbatim}
or, to use a particular printer mode such as \texttt{ljfour}, possibly
something like
\begin{verbatim}
mf '\mode:=ljfour; input pics.mf'
\end{verbatim}
This produces a \file{pics.tfm} file and a GF file with a name something
like \file{pics.600gf}. The actual number may be different and the
extension may get truncated on some file systems. Then you run
\prog{gftopk} on the GF file to produce a PK font file. (Read your
\prog{gftopk} manual on how to do this.) Typically, you just run
\begin{verbatim}
gftopk pics.600gf
\end{verbatim}
(or possibly ``\verb$gftopk pics.600gf pics.600pk$'' or
``\verb$gftopk pics.600gf pics.pk$'').

Now that you have the font (the \file{.pk} file) and font metric file (the
\file{.tfm}) generated by \MF{}, reprocess the file \file{pictures.tex}
with \TeX{}. The resulting DVI file should now be complete, and you should
be able to print and view it at your computer (assuming your viewer and
print driver have been set up to be able to find the PK font generated
from \file{pics.mf}).

It is not advisable to rely on automatic font generation to create the
\file{.tfm} and \file{.pk} files. (Different systems do this in
different ways, so here I will try to give a generic explanation.) The
reason: later editing of a figure will require new files to be built,
and most automatic systems will \emph{not} remake the files once they
have been created. This is not so much a problem with the \file{.tfm},
because \mfp{} never tries to load the font if the \file{.tfm} is absent
and therefore no automatic \file{.tfm}-making should ever be triggered.
However, if you forget to run \prog{gftopk}, then try to view your
resulting file, you may have to search your system and delete some
automatically generated \file{.pk} file (they can turn up in far-away
places) before you can see any later changes. It might be wise to write
a shell script (batch file) that runs both \MF{} and \prog{gftopk}. It
should also do some error checking and delete the \file{.tfm} if the
\file{.pk} file is not produced. That way, if anything goes wrong, the
\file{.dvi} will not contain the font (\mfp{} will draw a rectangle and
the figure number in place of the figure).

These processing steps---processing with \TeX{}, processing with
\MF{}\slash\prog{gftopk} or \MP{}, and reprocessing with \TeX{}---may not always be
necessary. In particular, if you change the \TeX{} document without
making any changes at all to the pictures, then there will be no need to
repeat the \MF{} or \MP{} steps.

There are also somewhat subtle circumstance under which you can skip the
second \TeX{} step after editing a figure if the file  has already gone
through the above process. Delineating the exact cirumstances is rather
involved, so it is recommended that you always repeat the \TeX{} step if
you have made changes that affect any figure.

What makes \mfp{} work? When you run \TeX{} on the file
\file{pictures.tex}, the \mfp{} macros issue \TeX{} \cs{write} commands,
writing \MF{} (or \MP{}) commands to a file \file{pics.mf} (or
\file{pics.mp}).  The user should never have to read or change the file
\file{pics.mf} directly---the \mfp{} macros take care of it.

The enterprising user can determine by examining the \mfp{} source and
the resulting \file{.mf} or \file{.mp} file, that \mfp{} drawing macros
translate almost directly into similar \MF{}\slash\MP{} commands,
defined in one of the files \file{grafbase.mf} or \file{grafbase.mp}.
The labels and captions, however, are placed on the graph by \TeX{}
using box placement techniques similar to those used in \LaTeX{}'s
\env{picture} environment (except when option \opt{mplabels} is in
effect, in which case the labels are written to the \file{.mp} file and
handled by \MP{}).

\smallskip
\emph{Note}: In this manual, when describing \mfp{} operations, we will
often refer to ``\MF{}'' when we really mean ``\MF{} or \MP{}''. This
will especially be the case whenever we need to refer to commands in the
two languages which are substantially the same, but occasionally we will
even talk about ``running \MF{}'' when we mean running one or the other
program \texttt{mf} or \texttt{mpost} to process the figures. If we need
to discriminate between the two processors, (for example when they have
different behavior) we will make the difference explicit.

A similar shorthand is used when referring to ``\TeX{}''. It should not be
taken to mean ``plain \TeX{}'', but rather whatever version of \TeX{} is used
to process the source file: plain \TeX{}, \LaTeX{}, \pdfTeX{}, or
\pdfLaTeX{}. Also \AmSTeX{}, \prog{eplain} and some other variants. When
last tried, \mfp{} didn't work with \ConTeXt{}.

\clearpage
\chapter{Options.}\label{options}

There are several options to the \mfp{} package. These options can be
turned on with certain provided commands, but under \LaTeX{} they can
also be used in the standard \LaTeX{} \cs{usepackage} optional argument.
Some options can be switched off and on throughout the document. Here we
merely list them and provide a general description of their purpose.
More details may be found later in the discussion of the features
affected. The headings below give the option name, the alternative macro
and, if available, the command for turning off the option. Any option
in the \cs{usepackage} command not among those given below will be
passed on to the \prog{graphics} package, provided the \opt{metapost}
option has been used.

If the file \file{mfpic.cfg} exists, it will be input just before all
options are processed. You can create such a file containing an
\cs{ExecuteOptions} command to execute any options you would like to
have as default. Actual options to \cs{usepackage} will override these
defaults, of course. And so will any of the commands below.

If a file named \file{mfpic.usr} can be found, it will be input at the
end of the loading of \mfp{}. The user can create such a file containing
any of the commands of this section that he would like to have as
default.

Finally, if the file \file{mfppatch.tex} can be found, it will be input
slightly before the end of loading \mfp{}. It is part of the \mfp{}
distribution, and will be used to implement minor corrections when bugs
are found. The user should \emph{not} modify this file unless
he really knows what he is doing.

\section{\opt{metapost}, \opt{metafont}, \cs{usemetapost}, \cs{usemetafont}.}%
\label{metapost}\index{metapost@\opt{metapost}}\index{usemetapost@\cs{usemetapost}}%
\index{metafont@\opt{metafont}}\index{usemetafont@\cs{usemetafont}}

The option \opt{metapost} or the command \cs{usemetapost} selects \MP{}
as the figure processor and makes specific features available. It
changes the extension used on the output file to `\file{.mp}' to signal
that it can no longer be processed with \MF{}. There is also a
\opt{metafont} option (command \cs{usemetafont}), but it is redundant,
as \MF{} is the default (for backward compatibility of files written
before \MP{} existed). Either command must come before the
\cs{opengraphsfile} command (see section~\ref{files}). They should not
be used together in the same document. (Actually they can, but one needs
to close one output file and open another. Moreover, it hasn't ever been
seriously tested, and it wasn't taken into consideration in writing most
of the macros.) If the command form \cs{usemetapost} is used in a
\LaTeXe{} document, it must come in the preamble. Because of the timing
of actions by the \prog{babel} package and by older versions of
\file{supp-pdf.tex} (input by \file{pdftex.def} in the \prog{graphics}
package), when \pdfLaTeX{} is used, \mfp{} should be loaded and
\cs{usemetapost} (if used) declared before \prog{babel} is loaded.


\section{\opt{mplabels}, \cs{usemplabels},
\cs{nomplabels}.}\label{mplabels}
\index{mplabels@\opt{mplabels}}%
\index{usemplabels@\cs{usemplabels}}%
\index{nomplabels@\cs{nomplabels}}

Causes all label creation commands to write their contents to the output
file. It effects only labels on the figure, not a caption added by the
\cs{tcaption} command (see section~\ref{text}).  In this case labels are
handled by \MP{} and can be rotated. It requires \MP{}, and will be be
ignored without it (\MF{} cannot handle labels). Using this option
without the \opt{metapost} option may also produce an error message
either from \TeX{} or \MF{}. The command forms can be placed anywhere.
If used outside an mfpic environment, they affect all subsequent
\cs{tlabel} commands; inside an mfpic environment they affect all
\cs{tlabel} commands in that figure.

When this is in effect, the labels become part of the figure and, in the
default handling, they may be clipped off or covered up by later drawing
elements. But see the next section on the \opt{overlaylabels} option.
Labels added to a picture contribute to the bounding box even if
\opt{truebbox} is not in effect.

The user is responsible for adding the appropriate \mfc{verbatimtex}
header to the output file if necessary. For this purpose, there is the
\cs{mfpverbtex} command, see section~\ref{labels}. If the label text
contains only valid plain \TeX{} macros, there is generally no need for
a \mfc{verbatimtex} preamble at all. If you add a \mfc{verbatimtex}
preamble of \LaTeX{} code take care to make sure \MP{} calls \LaTeX{}
(for example, the \texttt{mpost} command may take an option for this
purpose, or an environmental variable named \texttt{TEX} may be set
equal to \texttt{latex} in the command shell of your operating system.).

\section{\opt{overlaylabels}, \cs{overlaylabels}, \cs{nooverlaylabels}.}
\label{overlaylabels}
\index{overlaylabels@\opt{overlaylabels}}%
\index{overlaylabels@\cs{overlaylabels}}%
\index{nooverlaylabels@\cs{nooverlaylabels}}

In the past, under \opt{mplabels} all text labels created by \cs{tlabel}
and its relatives were added to the picture by \MP{} \emph{as they
occurred}. This made them subject to later drawing commands: they could
be covered up, erased, or clipped. With this option (or after the
command \cs{overlaylabels}) text labels are saved in a separate place
from the rest of a picture. When a picture is completed, the labels that
were saved are added on top of it. This is the way labels always behave
under the \opt{metafont} option, because then \TeX{} must add the labels
and there is no possibility for special effects involving clipping or
erasing (at the \MF{} level).

With the \opt{metapost} option, but without \opt{mplabels} it has been
decided to keep the same behavior (and the same code) as under the
\opt{metafont} option. However, when \opt{mplabels} is used, there is
the possibility for special effects with text, and it has always been
the behavior before version 0.7 to simply place the labels as they
occurred. It turns out that placing the labels at the end is cleaner and
simpler to code, so I experimented with it and rejected it as a default,
but now offer it as an option. With this option, \mfp{} labels have
almost the same behavior with or without \opt{mplabels}.

The commands may be used anywhere. Outside a figure they affect all
subsequent figures, inside a figure they affect all subsequent text in
that figure. The commands and option are ignored under the metafont
option.

\section{\opt{truebbox}, \cs{usetruebbox},
\cs{notruebbox}.}\label{truebbox}
\index{truebbox@\opt{truebbox}}%
\index{usetruebbox@\cs{usetruebbox}}%
\index{notruebbox@\cs{notruebbox}}

Normally \MP{} outputs an EPS file with the actual bounding box of the
figure. By default, \mfp{} \emph{overrides} this and sets the bounding
box to the dimensions specified by the \cs{mfpic} command that produced
it. (This used to be needed for \TeX{} is to handle \cs{tlabel} commands
correctly. Now, it is just for backward compatability, and for
compatability with \MF{}'s behavior.) It is reasonable to let \MP{} have
its way, and that is what this option does. If one of the command forms
is used in an \env{mfpic} environment, it affects only that environment,
otherwise it affects all subsequent figures. This option currently has
no effect with \MF{}, but should cause no errors.

This option is almost mandatory if you wish to use \prog{dvipdfm(x)} to
convert \TeX{}'s DVI output to PDF. Both \prog{dvipdfm} and
\prog{dvipdfmx} have a tendency to clip \MP{} figures to the stated
bounding box. Thus, anything running outside those bounds is lost.


\section{\opt{clip}, \cs{clipmfpic}, \cs{noclipmfpic}.}\label{clip}
\index{clip@\opt{clip}}%
\index{clipmfpic@\cs{clipmfpic}}%
\index{noclipmfpic@\cs{noclipmfpic}}

Causes all parts of the figure outside the rectangle specified by the
\cs{mfpic} command to be removed. The commands can come anywhere. If
issued inside an \env{mfpic} environment they affect the current figure
only. Otherwise all subsequent figures are affected. Note: this is a
rather rudimentary option. It has an often unexpected interaction with
truebbox. When both are in effect, \MP{} will produce a bounding box
that is the intersection of two rectangles: the true one \emph{without
clipping}, and the clipping rectangle (i.e., the one specified in the
\cs{mfpic} command). It is possible for the actual figure to be much
smaller than this bounding box (even empty!). This is a property of the
\MP{} \gbc{clip} command and we know of no way to avoid it.


\section{\opt{centeredcaptions}, \cs{usecenteredcaptions},
\cs{nocenteredcaptions}.}\label{centeredcaptions}
\index{centeredcaptions@\opt{centeredcaptions}}%
\index{usecenteredcaptions@\cs{usecenteredcaptions}}%
\index{nocenteredcaptions@\cs{nocenteredcaptions}}

Causes multiline captions created by \cs{tcaption} to have all lines
centered. This has no effect on the normal \LaTeX{} \cs{caption}
command.%
    \footnote{This writer [DHL] feels that \cs{tcaption} is too limited
    and users ought to apply the caption by other means, such as
    \LaTeX{}'s \cs{caption} command, outside the \env{mfpic}
    environment.}%

The commands can be issued anywhere. If inside an \env{mfpic}
environment they should come before the \cs{tcaption} command and affect
only it, otherwise they affect all subsequent figures. They should not
be used in the argument of a \cs{tcaption} command.

\section{\opt{raggedcaptions}, \cs{useraggedcaptions},
\cs{noraggedcaptions}.}\label{raggedcaptions}
\index{raggedcaptions@\opt{raggedcaptions}}%
\index{useraggedcaptions@\cs{useraggedcaptions}}%
\index{noraggedcaptions@\cs{noraggedcaptions}}

Causes multiline captions created by \cs{tcaption} to have all lines
raggedright. If \opt{centeredcaptions} is on, both sides will be ragged.
The command \cs{noraggedcaptions} restores the default: all lines except
the last justified. The last is either centered or flush left according
to whether \opt{centeredcaptions} is on or off.

The commands can be issued anywhere. If inside an \env{mfpic}
environment they should come before the \cs{tcaption} command and affect
only it, otherwise they affect all subsequent figures. They should not
be used in the argument of a \cs{tcaption} command.

\section{\opt{debug}, \cs{mfpicdebugtrue},
\cs{mfpicdebugfalse}.}\label{debug}
\index{debug@\opt{debug}}%
\index{mfpicdebugtrue@\cs{mfpicdebugtrue}}%
\index{mfpicdebugfalse@\cs{mfpicdebugfalse}}

Causes \mfp{} to write a rather large amount of information to the
\file{.log} file and sometimes to the terminal. Debug information
generated by \file{mfpic.tex} \emph{while loading} is probably of
interest only to developers, but can be turned on by giving a definition
to the command \cs{mfpicdebug} prior to loading. Any definition will
work because \prog{mfpic} only checks whether it is defined.

\section{\opt{clearsymbols}, \cs{clearsymbols}, \cs{noclearsymbols}.}
\index{clearsymbols@\opt{clearsymbols}}%
\index{clearsymbols@\cs{clearsymbols}}%
\index{noclearsymbols@\cs{noclearsymbols}}

\Mfp{} has two commands, \cs{point} and \cs{plotsymbol} that place a
small symbol at each of a list of points. The first can place either a
small filled disk or an open disk, the choice being dictated by the
setting of the boolean \cs{pointfilltrue} or \cs{pointfillfalse}. The
behavior of \cs{point} in the case of \cs{pointfillfalse} is to erase the
interior of the disk in addition to drawing its circumference.

The second command \cs{plotsymbol} can place a variety of shapes, some
open, some not. Its behavior before version 0.7 was to always draw the
shape without erasing the interior. Two other commands that placed these
symbols, \cs{plotnodes} and \cs{plot}, had the same behavior. With this
option, two of these, \cs{plotsymbol} and \cs{plotnodes}, will erase the
interior of the open symbols before drawing them. Thus
\cs{plotsymbol}\marg{SolidCircle} still works just like
\cs{pointfilltrue}\cs{point}, and now with this option
\cs{plotsymbol}\marg{Circle} behaves the same as
\cs{pointfillfalse}\cs{point}. The \cs{plot} command is unaffected by
this option.


\section{\opt{draft}, \opt{final}, \opt{nowrite}, \cs{mfpicdraft},
    \cs{mfpicfinal}, \cs{mfpicnowrite}.}\label{draft}
\index{draft@\opt{draft}}%
\index{final@\opt{final}}%
\index{nowrite@\opt{nowrite}}%
\index{mfpicdraft@\cs{mfpicdraft}}%
\index{mfpicfinal@\cs{mfpicfinal}}%
\index{mfpicnowrite@\cs{mfpicnowrite}}

Under the \opt{metapost} option, the various macros that include the
\EPS{} files emit rather large amounts of confusing error messages when
the files don't exist (especially in \LaTeX{}). For this reason, before
each picture is placed, \mfp{} checks for the existence of the graphic
before trying to include it. However, on some systems checking for the
existence of a nonexistent file can be very slow because the entire
\TeX{} search path will need to be checked. Therefore, \mfp{} doesn't
even attempt any inclusion on the first run. The first run is detected
by the non-existence of \file{\meta{file}.1}, where \meta{file} is the
name given in the \cs{opengraphsfile} command (but see also
section~\ref{files}). These options can be used to override this
automatic detection. All the command versions \emph{should} come before
the \cs{opengraphsfile} command. The \cs{mfpicnowrite} command
\emph{must} come before it.

These options might be used if, for example, the first figure has an
error and is not created by \MP{}, but you would like \mfp{} to go
ahead and include the remaining figures. Then use \opt{final}. It can
also be used to override a \LaTeX{} global \opt{draft} option. Or if
\file{\meta{file}.1} exists, but other figures still have errors and you
would like several runs to be treated as first runs until \MP{} has
stopped issuing error messages, then use \opt{draft}.  These commands
also work under the \opt{metafont} option, but time and error messages
are less of an issue then. If all the figures have been created and
debugged, some time might be saved (with either \opt{metafont} or
\opt{metapost}) by not writing the output file again, then \opt{nowrite}
can be used.

\section{\opt{mfpreadlog}, \cs{mfpreadlog}.}\label{readlog}
\index{mfpreadlog@\opt{mfpreadlog}}%
\index{mfpreadlog@\cs{mfpreadlog}}

From version 0.8, there exists a scheme to allow \MF{} or \MP{} to pass
information back to the \file{.tex} file. This is done by writing code
to the figure file requesting \MF{} to place that information in the
\file{.log} file it produces. This option instructs \mfp{} to
read through that log file line-by-line looking for such information.
Since such log files can be potentially quite lengthy, this is made an
option. If the command form \cs{mfpreadlog} is used, it must come before
the \cs{opengraphsfile} command, since that is when the file will be
examined. At the present time, the only \mfp{} facility that requires
this two-way communication is \cs{assignmfvalue} (see
subsection~\ref{misc}). If this is used, the filename given to
\cs{opengraphsfile} should not be the same as the \TeX{} source file in
which this occurs, as then the wrong \file{.log} may be read.


\section{Scoping Rules.}\label{scoping}

Some of these options merely change \TeX{} behavior, others write
information to the output file for \MF{} or \MP{}. Changes in \TeX{}
behavior obey the normal \TeX{} grouping rules, the information written
to the output file obeys \MF{} grouping rules. Since each \env{mfpic}
environment is both a \TeX{} group and (corresponds to) a \MF{} group,
the following always holds: use of one of the command forms inside of an
\env{mfpic} environment makes the change local to that environment.

An effort has been made (as of version 0.7) to make this universal. That
is, any of the commands listed above for turning options on and off will
be global when issued outside an \env{mfpic} environment. The debug
commands are exceptions; they obey all \TeX{} scoping rules.

We have also tried to make all other \mfp{} commands for changing the
various parameters follow this rule: local inside \env{mfpic}
environment, global outside. If this is ever untrue, and I don't
document that fact, please let me know.

The following are special:
\begin{display}
\cs{usemetapost}\index{usemetapost@\cs{usemetapost}},
\cs{usemetafont}\index{usemetafont@\cs{usemetafont}},
\cs{mfpicdraft}\index{mfpicdraft@\cs{mfpicdraft}},
\cs{mfpicfinal}\index{mfpicfinal@\cs{mfpicfinal}},
\cs{mfpicnowrite}\index{mfpicnowrite@\cs{mfpicnowrite}},\\
and
\cs{mfpreadlog}\index{mfpreadlog@\cs{mfpreadlog}}.
\end{display}
\noindent Their effects are always global, partly because they should
occur prior to the initialization command \cs{opengraphsfile} (described
in section~\ref{files}). Note that \cs{usemetapost} may cause a file of
graphic inclusion macros to be input. If this command is issued inside a
group, some definitions in that file may be lost, breaking the graphic
inclusion code.


\clearpage
\chapter{\CMF{} and \CMP{} Data Types.}\label{types}

Since the arguments of most \mfp{} drawing commands are sent to \MF{} to
be interpreted, it's useful to know something about \MF{} concepts. In
this chapter we will discuss some of the data types \MF{} supports.
Even the casual user should know how coordinates and colors are treated
and so should at least skim the next two sections. The last
section can be read when the user wants to manipulate more complex
objects.

\CMF{} permits  several different data types, and we will mainly be
concerned with six of these: \kw{numeric}, \kw{pair}, \kw{color} (\MP{}
only), \kw{path}, \kw{picture} and \kw{boolean}.%
    \footnote{For the curious, there are a total of eight types (nine or
    ten for \MP{}). The other three are \kw{string}, \kw{transform} and
    \kw{pen}. \MF{} also permits expressions that produce nothing, which
    is sometimes called the vacuous type, but doesn't allow (or need)
    variables of this type.}
In \MP{} version 1.000, a tenth data type was added, \kw{cmykcolor}, and
the \kw{color} data type can be referred to as `\kw{rgbcolor}' when a
distinction is necessary.

A \emph{variable} is a symbolic name, which can be a single letter such
as \mfc{A}, or a descriptive name like \mfc{origin}. Any sequence of
letters and underscores is permitted as a variable name. Numeric indexes
are also allowed, provided all variables that differ only in the index
have the same type. Thus \mfc{A1}, \mfc{A2}, etc., might be variables
which are all of type \kw{pair}. Quite a lot more is permited for variable
names, but the rules are rather complex and easy to violate. \Mfp{} has
commands for creating both simple variables and indexed variables
(called \emph{arrays}) but the casual user can get quite a lot of use
out of \mfp{} without ever creating or using a \MF{} variable.

\CMF{} also has something akin to functions. For example, \mfc{sin(1.57)}
might represent a function named \mfc{sin} receiving the parameter
$1.57$ as input and returning the appropriate value. Functions
can take any number of parameters and return any of the data types that
\MF{} supports.%
    \footnote{Including the vacuous type.}


\section{Numerics and pairs.}\label{pairs}

\CMF{} has \kw{numeric} quantities. These include lengths, such as the
radius of a circle, as well as dimension units such as \mfc{in} (inches)
and \mfc{pt} (points). In fact it understands all the same units that
\TeX{} does. These \kw{numeric} quantities can be constants (explicit
numbers) or variables (symbolic names). In fact, \mfc{in} and \mfc{pt}
are symbolic names for \kw{numeric} quantities.

\CMF{} also has \kw{pair} objects, which may be constants or variables.
Constants of type \kw{pair} have the form \mfc{($x$,$y$)} where $x$ and
$y$ are numbers, for example \mfc{(0,0)}.  Pairs are two-dimensional
quantities used for representing either points or vectors in a
rectangular (Cartesian) coordinate system.

In this manual we often represent each pair by a brief name, such as
\meta{p} or \meta{v}, the meanings of which are usually obvious in the
context of the macro. These are intended to be replaced in actual use by
either a pair constant or variable. The succinctness of this notation
helps us to think geometrically rather than only of coordinates.


\section{Colors.}\label{MPcolors}

\CMP{} has the same concepts as \MF, but also has \kw{color} objects,
which may also be constants or variables. In recent MP{}, colors come in
two flavors: \kw{rgbcolor} and \kw{cmykcolor}. Constants of type
\kw{rgbcolor} have the form \mfc{($r$,$g$,$b$)} where $r$, $g$, and $b$
are numbers between $0$ and $1$ determining the relative proportions of
red, green and blue in the color (the `rgb' model). Constants of type
\kw{cmykcolor} have the form \mfc{($c$,$m$,$y$,$k$)} where $c$, $m$, $y$
and $k$ are numbers between $0$ and $1$ determining the relative
proportions of cyan, magenta, yellow and black in the color (the `cmyk'
model).

A color variable is a name, like \mfc{red}, \mfc{blue} (both predefined
rgb colors in \MP) or \gbc{magenta} (predefined by \mfp{} to be an rgb
color if \MP{} has version ${}<1.000$, a cmyk color if the version is at
least 1.000).


\section{Paths, pictures and booleans.}\label{paths}

Most of the things that \mfp{} is designed to draw are paths. Examples
of paths are circles, rectangles,  other polygons, graphs of
functions and splines. Because we tend to want to draw these (or fill
them, or render them in other ways) we call the \mfp{} commands that
produce them \emph{figure macros}. Although they are much more complex
than numerics, pairs, or colors, they can still be stored in symbolic
names.

Normally in \mfp{} we want to create a picture, usually by rendering one
or more paths. It is possible in \MF{} to store a picture in a symbolic
name without actually drawing it. However, because of their complexity,
objects of type \kw{picture} require somewhat more care than paths or
other data types. Do not expect to use stored pictures in the same way
as stored paths. In fact, one should use \kw{picture} variables only in
those command that are explicitely designed for them. In \mfp{} to date
these are only \cs{tile...}\cs{endtile} and \cs{mfpimage} to store
pictures, and \cs{putmfpimage} to draw copies of one. There is also
\cs{tess}, but it is used only to fill a region with copies of a picture
created by \cs{tile}.

The \kw{boolean} data type is one of the values \mfc{true} or
\mfc{false}. Variables of type \kw{boolean} are symbolic names that can
take either of these two values. Usually these are used to influence the
behavior of some command by setting a relevant \kw{boolean} variable to
one or the other value.


\clearpage
\chapter{The Macros.}\label{macros}

Many of the commands of \mfp{} have optional arguments. These are
denoted just as in \LaTeX{}, with square brackets. Thus, the command for
drawing a circle can be given
\begin{verbatim}
\circle{(0,0),1}
\end{verbatim}
having only the mandatory argument, or
\begin{verbatim}
\circle[p]{(0,0),1}
\end{verbatim}
Whenever an optional argument is omitted, the behavior is equivalent to
some choice of the optional argument. In this example, the two forms
have exactly the same behavior, drawing a circle centered at $(0,0)$
with radius $1$. In this case we will say ``\oarg{p} is the
\emph{default}''. Another example is \cs{point}\marg{(1,0)} versus
\cs{point}\oarg{3pt}\marg{(1,0)}. They both place a dot at the point
$(1,0)$. The second one explicitly requests that it have diameter
\dim{3pt}; the first will examine the length command \cs{pointsize},
which the user can change, but it is initialized to \dim{2pt}. In this
case we will say ``the default is the value of \cs{pointsize},
\emph{initially} \dim{2pt}''.

If an \mfp{} command that takes an optional argument finds only empty
brackets (completely empty, no spaces), then it will use the default
value. This is useful for commands that have two optional arguments and
one wants the default value in the first one and some nondefault value
in the second. An optional argument should normally not contain any
spaces. Even when the argument contains more than one piece of data,
spaces should not separate the parts. In some cases this will cause no
harm, but it would be better to avoid doing it altogether, because there
are cases where it will cause wrong results or error messages.


\section{Files and Environments.}\label{files}

\begin{cd}\pagelabel{opengraphsfile}
\cs{opengraphsfile}\marg{\meta{file}}\\
  \ $\ldots$\\
\cs{closegraphsfile}%
\index{opengraphsfile@\cs{opengraphsfile}}%
\index{closegraphsfile@\cs{closegraphsfile}}
\end{cd}

These macros open and close the \MF{} or \MP{} file which will contain
the pictures to be included in this document.  The name of the file will
be \file{\meta{file}.mf} (or \file{\meta{file}.mp}). Do \emph{not}
specify the extension, which is added automatically.

\emph{Note}: This command may cause \file{\meta{file}.mf} or
\file{\meta{file}.mp} to be overwritten if it already exists, so be sure
to consider that when selecting the name. Repeating the running of
\TeX{} will overwrite the file created on previous runs, but that should
be harmless. For if no changes are made to \env{mfpic} environments, the
identical file will be recreated, and if changes have been made, then
you want the file to be replaced with the new version.

It is possible (but \emph{has not} been seriously tested) to close one
file and open another, and even to change between \opt{metapost} and
\opt{metafont} in between. If anything goes wrong with this, contact the
maintainer and it might be fixed in some later version.

There may be limitations on what can be used as a filename. As of \mfp{}
version \mfpversion, we have tried to permit \cs{jobname} as part of
\meta{file}. Thus we permit \TeX{} macros, but they should expand to
non-special characters. Permitting macros makes it essentially
impossible for the filename to contain the backslash and brace
characters. Also spaces are problematic. However other special \TeX{}
characters (for example: tilde, underscore and percent) can be used,
though that is really not recommended.

\begin{cd}\pagelabel{mfpic}
\cs{mfpic}\oarg{\meta{xfactor}}\oarg{\meta{yfactor}}%
  \marg{\meta{xmin}}\marg{\meta{xmax}}\marg{\meta{ymin}}\marg{\meta{ymax}}\\
  \ $\ldots$\\
\cs{endmfpic}%
\index{mfpic@\cs{mfpic}}%
\index{endmfpic@\cs{endmfpic}}
\end{cd}

These macros open and close the \env{mfpic} environment%
    \footnote{We use the term `environment' loosely. However, in
    \LaTeX{} one may use an actual \env{mfpic} environment. See
    page~\pageref{envusage}.}
in which the drawing macros make sense. While many \mfp{} commands can
be used inside or outside this environment, those that actually produce
visible output are required to be inside. The \cs{mfpic} macro also sets
up the local coordinate system for the picture. The \meta{xfactor} and
\meta{yfactor} parameters establish the length of a coordinate system
unit, as a multiple of the \TeX{} dimension \cs{mfpicunit}. If neither
is specified, both are taken to be 1 and each coordinate system unit is
1 \cs{mfpicunit}. If only one is specified, then they are assumed to be
equal. Note that some drawing commands require equal scales to work as
expected: if you try to draw a circle with different scales you will get
an ellipse.

The \meta{xmin} and \meta{xmax} parameters establish the lower
and upper bounds for the $x$-axis coordinates; similarly, \meta{ymin}
and \meta{ymax} establish the bounds for the $y$-axis. These bounds are
expressed in local units---in other words, the actual width of the
picture will be $(\meta{xmax}-\meta{xmin})\cdot\meta{xfactor}$ times
\cs{mfpicunit}, its height $(\meta{ymax}-\meta{ymin})\cdot\meta{yfactor}$
times \cs{mfpicunit}, and its depth zero.

Most of \mfp{}'s drawing macros accept parameters which are
\emph{coordinate pairs}. A coordinate pair is a pair of numbers $(x,y)$
enclosed in parentheses, with $\meta{xmin} \le x \le \meta{xmax}$ and
$\meta{ymin} \le y \le \meta{ymax}$.%
    \footnote{These inequalities can be violated, usually causing
    something to be drawn outside the designated borders of the figure.}
We will call these \emph{graph coordinates} and refer to the numbers $x$
and $y$ as being \emph{in graph units}. Things like the thickness of
lines and the lengths of arrowheads are required to be expressed in
actual lengths such as \dim{1pt} or \dim{3mm}. These will be referred to as
\emph{absolute} units.

One can scale all pictures uniformly by changing \cs{mfpicunit}, and
scale an individual picture by changing \meta{xfactor} and \meta{yfactor}.
After loading \mfp{}, \cs{mfpicunit} has the value \dim{1pt}. One
\texttt{pt} is a \emph{printer's point}, which equals 1/72.27 inches or
0.35146 millimeters.

\emph{Note}: Changing \cs{mfpicunit} or the optional parameters will
scale the coordinate system, but not the values of parameters that are
defined in absolute units. If you wish, you can set these to multiples
of \cs{mfpicunit}, but it is difficult (and almost certainly unwise) to
get the thickness of lines (for example) to scale along with the scale
parameters.

In addition to establishing the coordinate system, these scales and
bounds are used to establish the metric for the \MF{} character or
bounding box for the \MP{} figure described within the environment. If
any of these parameters are changed, the \file{.tfm} file (\MF{}) or the
bounding box (\MP{}) will be affected, so you will have to be sure to
reprocess the \TeX{} file after processing the \file{.mf} or \file{.mp}
file, even if no other changes are made in the figure.

The value of these 6 parameters to \cs{mfpic} are available within the
environment as macros: \cs{xfactor}, \cs{yfactor}, \cs{xmin}, \cs{xmax},
\cs{ymin} and \cs{ymax}.

\begin{cd}\pagelabel{mfpicnumber}
\cs{mfpicnumber}\marg{\meta{num}}%
\index{mfpicnumber@\cs{mfpicnumber}}
\end{cd}

Normally, \cs{mfpic} assigns the number 1 to the first \env{mfpic}
environment, after which the number is increased by one for each new
\env{mfpic} environment. This number is used internally to include the
picture. It is also transmitted to the output file where it is used as
the argument to a \gbc{beginmfpic} command. In \MF{} this number becomes
the position of the character in the font file, while in \MP{} it is the
extension on the graphic file that is output. The above command tells
\mfp{} to ignore this sequence and number the next \env{mfpic}
figure with \meta{num} (and the one after that $\meta{num}+1$, etc.). It
is up to the user to make sure no number is repeated, as no checking is
done. Numbers greater than 255 may cause errors, as \TeX{} assumes that
characters are represented by 8-bit numbers. If the first figure is to
be numbered something other than $1$, then, under the \opt{metapost}
option, this command should come before \cs{opengraphsfile}, as that
command checks for the existence of the first numbered figure to
determine if there are figures to be included.

\begin{cd}\pagelabel{everymfpic}
\cs{everymfpic}\marg{\meta{commands}}\\
\cs{everyendmfpic}\marg{\meta{commands}}%
\index{everymfpic@\cs{everymfpic}}%
\index{everyendmfpic@\cs{everyendmfpic}}%
\end{cd}

These commands store the \meta{commands}. The first arranges for these
commands to be issued first thing in every \env{mfpic} environment and
the second arranges for its commands to be issued as the last thing in
every such environment. These could be any commands that make sense
inside that environment. Their purpose is mainly to save typing if there
is identical setup being performed in every picture.

\begin{cd}\pagelabel{envusage}
\cs{begin}\marg{mfpic}\texttt{...}\cs{end}\marg{mfpic}%
\index{begin@\cs{begin}\marg{mfpic}}
\end{cd}

In \LaTeX{} you may prefer to use \cs{begin}\marg{mfpic} and
\cs{end}\marg{mfpic} (instead of \cs{mfpic} and \cs{endmfpic}). This is
by no means required. The sample file \file{lapictures.tex} provided
with \mfp{} illustrates this use of an \env{mfpic} environment in
\LaTeX{}.

\medskip
A word about \TeX{} groups inside \env{mfpic} environments. These can be
useful to limit the scope of declarations or of changes to some
variables. However, they do not limit the scope of changes to the figure
file that is being written, so there is a danger that \TeX{} and \MF{}
will have different values. There are also some \mfp{} command that need
to be at the outermost level. Thus, grouping should generaly be avoided
except for those groups provided by \mfp{} commands.

\medskip
For the remainder of the macros, the numerical parameters are expressed
in graph units, the units of the local coordinate system specified by
\cs{mfpic}, unless otherwise indicated.

\section{Common objects.}\label{figures}

The \mfp{} macros that draw things can be roughly divided into two
classes.
\begin{enumerate}
  \item Those that simply cause something to be drawn. Examples of these
    are the \cs{point} command, which places a dot at a list of
    coordinates, and \cs{gridlines}, which draw coordinate lines with
    specified separation.
  \item Those that both \emph{define} and draw a \emph{path}. The macros
     \cs{circle}, \cs{rect}, and \cs{polyline} are examples of these.
\end{enumerate}

Macros of type 2 are referred to hereafter as \emph{figure macros}, for
lack of a better term. With them one can use \emph{prefix
macros}\index{prefix macro} to modify various aspects of the path and
how it is drawn. For example,
\begin{verbatim}
\polyline{(1,2),(3,4)}
\end{verbatim}
draws a line from $(1,2)$ to $(3,4)$, but
\begin{verbatim}
\dotted\polyline{(1,2),(3,4)}
\end{verbatim}
produces a dotted version, and
\begin{verbatim}
\arrow\polyline{(1,2),(3,4)}
\end{verbatim}
draws it with an arrowhead at the tip. This is not possible with
\cs{gridlines}, for example. As \mfp{} and the accompanying \MF{}
package \grafbase{} are currently written, prefix macros can only be
applied to single paths, and \cs{gridlines} produces a whole set of
lines. In this manual, as each macro is introduced, if it is a figure
macro, this will be explicitly stated.

Some commands depend on the value of separately defined parameters. all
these parameters are initialized when \mfp{} is loaded. In the following
descriptions we give the initial value of all the relevant parameters.
\Mfp{} provides commands to change any of these parameters. When \MP{}
output is selected, figures can be drawn in any color and several of the
above mentioned parameters are colors. For example, \gbc{drawcolor} is
the name of the default color used to draw curves, \gbc{headcolor} is
used when drawing arrowheads, etc.  To save repetition: all special
colors for figures are initialized to \mfc{black} except
\mfc{background}, which is \mfc{white}.


\subsection{Points, lines, and rectangles}\label{points}

\begin{cd}\pagelabel{point}
\cs{point}\oarg{\meta{size}}\marg{\meta{$p_0$},\meta{$p_1$},$\ldots$}%
\index{point@\cs{point}}
\end{cd}

Draws small disks centered at the points specified in the list of
ordered pairs. The optional argument \meta{size} is an absolute
dimension that determines the diameter of the disks. The default is the
\TeX{} dimension \cs{pointsize}, initially \dim{2pt}. The disks have a
filled interior if the command \cs{pointfilltrue} has been issued (the
initial behavior). After the command \cs{pointfillfalse}, \cs{point}
commands will produce outlined circles with the interiors erased. The
color of the circles is the value of the predefined variable
\gbc{pointcolor}, and the color inside of the open circles is the value
of the variable \mfc{background}.%
    \footnote{\MP{} cannot actually erase. The illusion of erasing is
    created by painting over with \mfc{background}.}

\begin{cd}\pagelabel{plotsymbol}
\cs{plotsymbol}\oarg{\meta{size}}\marg{\meta{symbol}}%
\marg{\meta{$p_0$},\meta{$p_1$},$\ldots$}%
\index{plotsymbol@\cs{plotsymbol}}
\end{cd}

Draws small symbols centered at the points \meta{$p_0$}, \meta{$p_1$},
and so on. The symbols must be given by name, and the available symbols
are:
\begin{display}
\gbc{Asterisk}\index{Asterisk@\gbc{Asterisk}},
\gbc{Circle}\index{Circle@\gbc{Circle}},
\gbc{Diamond}\index{Diamond@\gbc{Diamond}},
\gbc{Square}\index{Square@\gbc{Square}},
\gbc{Triangle}\index{Triangle@\gbc{Triangle}},
\gbc{Star}\index{Star@\gbc{Star}},
\gbc{SolidCircle}\index{SolidCircle@\gbc{SolidCircle}},\\
\gbc{SolidDiamond}\index{SolidDiamond@\gbc{SolidDiamond}},
\gbc{SolidSquare}\index{SolidSquare@\gbc{SolidSquare}},
\gbc{SolidTriangle}\index{SolidTriangle@\gbc{SolidTriangle}},
\gbc{SolidStar}\index{SolidStar@\gbc{SolidStar}},
\gbc{Cross}\index{Cross@\gbc{Cross}} and
\gbc{Plus}\index{Plus@\gbc{Plus}}.
\end{display}
The names should be self-explanatory, the `\gbc{Solid}' ones are filled
in, the others are outlines. Under \opt{metapost}, symbols are drawn in
\gbc{pointcolor}. The \meta{size} defaults to \cs{pointsize} as in
\cs{point} above. \gbc{Asterisk} consists of six line segments while
\gbc{Star} is the standard five-pointed star formed from ten straight
line segments. \gbc{Cross} is a $\times$ shape. The name
`\cs{plotsymbol}' comes from the fact that the \cs{plot} command (see
subsection~\ref{drawing}), which was written first, utilizes these same
symbols. The command \cs{symbol} was already taken (standard \LaTeX{}).

While one would rarely want to use them for this purpose, the following
symbols are also available:
\begin{display}
\gbc{Arrowhead}\index{Arrowhead@\gbc{Arrowhead}},
\gbc{Crossbar}\index{Crossbar@\gbc{Crossbar}},
\gbc{Leftbar}\index{Leftbar@\gbc{Leftbar}},
\gbc{Rightbar}\index{Rightbar@\gbc{Rightbar}},
\gbc{Lefthook}\index{Lefthook@\gbc{Lefthook}},
\gbc{Righthook}\index{Righthook@\gbc{Righthook}},
\gbc{Leftharpoon}\index{Leftharpoon@\gbc{Leftharpoon}},\\
\gbc{Rightharpoon}\index{Rightharpoon@\gbc{Rightharpoon}}.
\end{display}
These are mainly intended for making arrows. See subsection~\ref{arrows}
for a further description.

The difference between \cs{pointfillfalse}\cs{point}$\ldots$ and
\cs{plotsymbol}\marg{Circle}$\ldots$ is that the inside of the circle
will not be erased in the second version, so whatever else has already
been drawn in that area will remain visible. This is the default (for
backward compatibility), but that can be changed with the commands
below.

\begin{cd}\pagelabel{clearsymbols}
\cs{clearsymbols}\\
\cs{noclearsymbols}%
\index{clearsymbols@\cs{clearsymbols}}%
\index{noclearsymbols@\cs{noclearsymbols}}
\end{cd}

After the first of these two commands, subsequent \cs{plotsymbol}
commands will draw the open symbols with their interiors erased. After
the second, the default behavior (described above) will be restored.
These commands have no effect on \cs{point}. \cs{plotnodes} (see
subsection~\ref{drawing}) also responds to the settings made by these
commands. The \cs{plot} command (also in subsection~\ref{drawing}) does
not.

You can design your own `symbols'. See the discussion of arrowheads in
subsection~\ref{arrows}, and of storing paths in
subsection~\ref{transformation}.

\begin{cd}\pagelabel{pointdef}
\cs{pointdef}\marg{\meta{name}}\texttt{(\meta{xcoord},\meta{ycoord})}%
\index{pointdef@\cs{pointdef}}
\end{cd}

Defines a symbolic name for an ordered pair and the coordinates it
contains. \meta{name} is any legal \TeX{} command name \emph{without}
the backslash; \meta{xcoord} and \meta{ycoord} are any numbers. For
example, after the command \cs{pointdef}\marg{A}\texttt{(1,3)}, \cs{A}
expands to \texttt{(1,3)}, while \cs{Ax} and \cs{Ay} expand to
\texttt{1} and \texttt{3}, respectively. If \opt{mplabels} is in effect
one can use \cs{A} to specify where to place a text label, but if \TeX{}
is placing labels one must use \texttt{ (\cs{Ax},\cs{Ay})}. In most
other cases, one can use \cs{A} where a pair or point is required.

\begin{cd}\pagelabel{polyline}
\cs{polyline}\marg{\meta{$p_0$},\meta{$p_1$},$\ldots$}\\
\cs{lines}\marg{\meta{$p_0$},\meta{$p_1$},$\ldots$}%
\index{polyline@\cs{polyline}}%
\index{lines@\cs{lines}}
\end{cd}

The figure macro \cs{polyline} produces connected line segments from
\meta{$p_0$} to \meta{$p_1$}, and from there to \meta{$p_2$}, etc. The
result is an open polygonal path through the specified points, in the
specified order. The macro \cs{lines} is an alias for \cs{polyline}.

\begin{cd}\pagelabel{polygon}
\cs{polygon}\marg{\meta{$p_0$},\meta{$p_1$},$\ldots$}\\
\cs{closedpolyline}\marg{\meta{$p_0$},\meta{$p_1$},$\ldots$}%
\index{polygon@\cs{polygon}}%
\index{closedpolyline@\cs{closedpolyline}}
\end{cd}

The figure macro \cs{polygon} produces a closed polygon with vertices
at the specified points in the specified order. It works exactly like
\cs{polyline} except the last point in the list is also joined to the
first. The macro \cs{closedpolyline} is an alias for \cs{polygon}.

\begin{cd}\pagelabel{rect}
\cs{rect}\marg{\meta{$p_0$},\meta{$p_1$}}%
\index{rect@\cs{rect}}
\end{cd}

This figure macro produces the closed rectangle with horizontal and
vertical sides, having the points \meta{$p_0$} and \meta{$p_1$} as
diagonally opposite corners. The same rectangle can be specified in four
different ways: either pair of opposite corners in either order.

It is occasionally helpful to know that connected paths like those
produced by \cs{polyline} or \cs{rect} have a \emph{start} and an
\emph{end} as well as \emph{sense} (or direction). The path produced by
\cs{polyline} starts at the first listed point and ends at last, having
the direction determined by the order of the points. For \cs{rect} the
sense may be clockwise or anticlockwise depending on the corners used:
it starts by moving horizontally from the first listed point.
Several \mfp{} macros (such as those that add arrowheads) treat the
beginning and the end of a path differently, or adjust their behavior
according to the sense of the curve.

\begin{cd}\pagelabel{regpolygon}
\cs{regpolygon}\marg{\meta{num}}\marg{\meta{name}}%
  \marg{\meta{eqn$_1$}}\marg{\meta{eqn$_2$}}%
\index{regpolygon@\cs{regpolygon}}
\end{cd}

This figure macro produces a closed regular polygon with \meta{num}
sides. The second argument, \meta{name} is a symbolic name. It can be
used to refer to the vertices later. The last two arguments should be
equations that position two of the vertices or one vertex and the
center. The center is referred to by \meta{name}\gbc{0} and the vertices
by \meta{name}\gbc{1} \meta{name}\gbc{2}, etc., going anticlockwise
around the polygon. The \meta{name} itself (without a number suffixed)
will be a \MF{} variable assigned the value of \meta{num}. For example,
\begin{verbatim}
\regpolygon{5}{Kay}{Kay0=(0,1)}{Kay1=(2,0)}
\end{verbatim}
will produce a regular pentagon with its center at $(0,1)$ and its first
vertex at $(2,0)$. One could later draw a star inside it with
\begin{verbatim}
\polygon{Kay1,Kay3,Kay5,Kay2,Kay4}
\end{verbatim}
Moreover, \gbc{Kay} will equal $5$. The name given becomes a \MF{}
variable and care should be taken to make the name distinctive so as not
to redefine some internal variable.


\subsection{A word about list arguments}\label{list}

We have seen already four \mfp{} macros that take a mandatory argument
consisting of an arbitrary number of coordinate pairs, separated by
commas. There are many more, and some that take a comma-separated list
of items of other types. If the lists are long, especially if they are
generated by a program, it might be more convenient if one could simply
refer to an external file for the data. This is possible, and one does
it the following way: instead of \cs{polyline}\marg{\meta{list}}, one
can write\index{datafile@\cs{datafile}}
\begin{ex}
  \cs{polyline}\cs{datafile}\marg{\meta{filename}}
\end{ex}
where \meta{filename} is the full name of the file containing the data.
The required format of this file and the details of this usage can be
found in subsection~\ref{external}. This method is available for any
command that takes a comma-separated list of data (of arbitrary length)
as its last argument, \emph{with the exception of those commands that
add text to the picture}. Examples of the latter are \cs{plottext} and
\cs{axislabels} (subsection~\ref{text}).


\subsection{Axes, axis marks, and grids}\label{axesthings}

\begin{cd}\pagelabel{axes}
\cs{axes}\oarg{\meta{hlen}}\\
\cs{xaxis}\oarg{\meta{hlen}}\\
\cs{yaxis}\oarg{\meta{hlen}}%
\index{axes@\cs{axes}}%
\index{xaxis@\cs{xaxis}}%
\index{yaxis@\cs{yaxis}}
\end{cd}

These are retained for backward compatibility, but there are more
flexible alternatives below. They draw $x$- and  $y$-axes for the
coordinate system. The command \cs{axes} is equivalent to \cs{xaxis}
followed by \cs{yaxis} which produce the obvious. The $x$- and $y$-axes
extend the full width and height of the \env{mfpic} environment. The
optional \meta{hlen} sets the length of the arrowhead on each axis. The
default is the value of the \TeX{} dimension \cs{axisheadlen}, initially
\dim{5pt}. The shape of the arrowhead is determined as in the \cs{arrow}
macro (section~\ref{modifier}). The color of the head is the value of
\gbc{headcolor}, the shaft is \gbc{drawcolor}.

Unlike other commands that produce lines or curves, these do not respond
to prefix macros. They always draw a solid line (with an arrowhead
unless \cs{axisheadlen} is \dim{0pt}). They \emph{do} respond to changes
in the pen thickness (see \cs{penwd} in section~\ref{parameters}) but
that is pretty much the only possibility for variation.

\begin{cd}\pagelabel{axis}
\cs{axis}\oarg{\meta{hlen}}\marg{\meta{one-axis}}\\
\cs{doaxes}\oarg{\meta{hlen}}\marg{\meta{axis-list}}%
\index{axis@\cs{axis}}%
\index{doaxes@\cs{doaxes}}%
\end{cd}

These produce any of 6 different axes. The parameter \meta{one-axis} can
be \texttt{x} or \texttt{y}, to produce (almost) the equivalent of
\cs{xaxis} and \cs{yaxis}; or it can be \texttt{l}, \texttt{b},
\texttt{r}, or \texttt{t} to produce an axis on the border of the
picture (left, bottom, right or top, respectively). \cs{doaxes} takes a
list of any or all of the six letters (with either spaces or nothing in
between) and produces the appropriate axes. Example:
\cs{doaxes}\marg{lbrt}. The optional argument sets the length of the
arrowhead. In the case of axes on the edges, the default is the value of
\cs{sideheadlen}, which \mfp{} initializes to \dim{0pt}. For the $x$-
and $y$-axis the default is \cs{axisheadlen} as in \cs{xaxis} and
\cs{yaxis} above.

The commands \cs{axis}\marg{x}, \cs{axis}\marg{y}, and
\cs{doaxes}\marg{xy} differ from the old \cs{xaxis}, \cs{yaxis} and
\cs{axes} in that these new versions respond to prefix macros. The
\cs{arrow} prefix previously mentioned is an exception: these macros add
an arrowhead automatically. For example, the sequence
\cs{dotted}\cs{axis}\marg{x} draws a dotted $x$-axis, but
\cs{dotted}\cs{xaxis} produces a \MF{} error. A prefix macro applied to
\cs{doaxes} generates no error, but only the first axis in the list will
be affected.

\begin{cd}\pagelabel{axisline}
\cs{axisline}\marg{\meta{one-axis}}\\
\cs{border}%
\index{axisline@\cs{axisline}}%
\index{border@\cs{border}}%
\end{cd}

These are figure macros that draw the line or lines that an \cs{axis}
command would draw. An \cs{axis} command is almost the equivalent of
\begin{display}
\cs{arrow}\oarg{l\meta{hlen}}\cs{axisline}\marg{\meta{one-axis}}.
\end{display}
The \cs{axisline} command is provided as a figure macro for maximum
flexibility. For example, one can use the star-form of the \cs{arrow}
command if desired or decorate it with ones own choice of arrowhead (see
subsection~\ref{arrows}).

Also a figure macro, \cs{border} produces the rectangle which, if drawn,
is visibly the same as the four border \cs{axisline}\,s (without heads).
It is a closed path and could easily be drawn with a \cs{rect} command,
but the \cs{border} command automatically adjusts for the margins set by
the commands below.

The side axes are drawn by default with a pen stroke along the very edge
of the picture (as determined by the parameters to \cs{mfpic}). This can
be changed with the command \cs{axismargin} described below.

Axes on the edges are drawn so that they don't cross each other.
\cs{doaxes}\marg{lbrt}, for example, produces a perfect rectangle. If the
$x$- and $y$-axis are drawn with \cs{axis} or \cs{doaxes}, then they
will not cross the side axes. For this to work properly, all the
following margin settings have to be done before the axes are drawn.

\begin{cd}\pagelabel{axismargin}
\cs{axismargin}\marg{\meta{one-axis}}\marg{\meta{num}}\\
\cs{setaxismargins}%
  \marg{\meta{num}}\marg{\meta{num}}\marg{\meta{num}}\marg{\meta{num}}\\
\cs{setallaxismargins}\marg{\meta{num}}%
\index{axismargin@\cs{axismargin}}%
\index{setaxismargins@\cs{setaxismargins}}%
\index{setallaxismargins@\cs{setallaxismargins}}%
\end{cd}

The parameter \meta{one-axis} is one of the letters \texttt{l},
\texttt{b}, \texttt{r}, or \texttt{ t}, and \cs{axismargin} causes the
given axis to be shifted \emph{inward} by the \meta{num} specified (in
\emph{graph} units). The second command \cs{setaxismargins} takes
4 arguments, using them to set the margins starting with the left and
proceeding anticlockwise. The last command sets all the axis margins to
the same value.

A change to an axis margin affects not only the axis at that edge but
also the three axes perpendicular to it. For example, if the margins are
$M_{\mathrm{lft}}$, $M_{\mathrm{bot}}$, $M_{\mathrm{rt}}$ and
$M_{\mathrm{top}}$, then \cs{axis}\marg{b} draws a line starting
$M_{\mathrm{lft}}$ graph units from the left edge and ending
$M_{\mathrm{rt}}$ units from the right edge. Of course, the entire line
is $M_{\mathrm{bot}}$  units above the bottom edge. The margins are also
respected by the $x$- and $y$-axis, but only when drawn with \cs{axis}.
The old \cs{xaxis}, \cs{yaxis} and \cs{axes} ignore them.

Special effects can be achieved by lying to one axis about the other
margins. That is, axes can be draw in separate commands with changes to
the declared margins in between. Be aware that various other commands
are affected by the margin values. Examples are the already mentioned
\cs{border}, as well as \cs{grid} and \cs{gridlines}
(page~\pageref{grid} in this subsection).

\begin{cd}\pagelabel{axismarks}
\cs{xmarks}\oarg{\meta{len}}\marg{\meta{numberlist}}\\
\cs{ymarks}\oarg{\meta{len}}\marg{\meta{numberlist}}\\
\cs{lmarks}\oarg{\meta{len}}\marg{\meta{numberlist}}\\
\cs{bmarks}\oarg{\meta{len}}\marg{\meta{numberlist}}\\
\cs{rmarks}\oarg{\meta{len}}\marg{\meta{numberlist}}\\
\cs{tmarks}\oarg{\meta{len}}\marg{\meta{numberlist}}\\
\cs{axismarks}\marg{\meta{axis}}\oarg{\meta{len}}\marg{\meta{numberlist}}%
\index{xmarks@\cs{xmarks}}%
\index{tmarks@\cs{tmarks}}%
\index{bmarks@\cs{bmarks}}%
\index{ymarks@\cs{ymarks}}%
\index{lmarks@\cs{lmarks}}%
\index{rmarks@\cs{rmarks}}%
\index{axismarks@\cs{axismarks}}
\end{cd}

These macros place hash marks on the appropriate axes at the places
indicated by the values in the list. The optional \meta{len} gives the
length of the hash marks. If \meta{len} is not specified, the \TeX{}
dimension \cs{hashlen}, initially \dim{4pt}, is used. The marks on the
$x$- and $y$-axes are centered on the respective axis; the marks on the
border axes are drawn to the inside.  Both these behaviors can be
changed (see below). The commands may be repeated as often as desired.
(The timing of drawing commands can make a difference as outlined in
appendix~\ref{mpconsiderations}.) The command \cs{axismarks}\marg{x} is
equivalent to \cs{xmarks} and so on for each of the six axes. (I would
have used the shorter name \cs{marks}, but that name was already taken
by \eTeX{}.)

The \meta{numberlist} is normally a comma-separated list of numbers. In
place of this, one can give a starting number, an increment and an
ending number as in the following example:
\begin{verbatim}
\xmarks{-2 step 1 until 2}
\end{verbatim}
is the equivalent of
\begin{verbatim}
\xmarks{-2,-1,0,1,2}
\end{verbatim}

One must use exactly the words \mfc{step} and \mfc{until}. Spaces are
not needed unless a variable name is used in place of one of the
numbers (see subsection~\ref{variables}). The number of spaces is not
significant.%
    \footnote{Experienced \MF{} programmers may recognize that anything
    can be used that is permitted in \MF{}'s \meta{forloop} syntax. Thus
    the given example can also be reworded \cs{xmarks}\marg{-2 upto 2},
    or even \cs{xmarks}\marg{2 downto -2}. See subsection~\ref{loops}
    for more on for-loops in \mfp{}.} %
Users of this syntax should be aware that if any of the numbers is not
an integer then, because of natural round-off effects, the last value
might be overshot and a mark not printed there. For example, to ensure
that a mark is printed at the point $1.0$ on the $x$-axis, the second
line below is better than the first.
\begin{verbatim}
\xmarks{0 step .2 until 1.0}
\xmarks{0 step .2 until 1.1}
\end{verbatim}

\begin{cd}\pagelabel{setaxismarks}
\cs{setaxismarks}\marg{\meta{axis}}\marg{\meta{pos}}\\
\cs{setbordermarks}\marg{\meta{lpos}}\marg{\meta{bpos}}\marg{\meta{rpos}}\marg{\meta{tpos}}\\
\cs{setallbordermarks}\marg{\meta{pos}}\\
\cs{setxmarks}\marg{\meta{pos}}\\
\cs{setymarks}\marg{\meta{pos}}%
\index{setaxismarks@\cs{setaxismarks}}%
\index{setbordermarks@\cs{setbordermarks}}%
\index{setallbordermarks@\cs{setallbordermarks}}%
\index{setxmarks@\cs{setxmarks}}%
\index{setymarks@\cs{setymarks}}%
\end{cd}

These set the placement of the hash marks relative to the axis. The
parameter \meta{axis} is one of the letters \texttt{x}, \texttt{y}, \texttt{l},
\texttt{b}, \texttt{r}, or \texttt{t}, and \meta{pos} must be one of the literal
words \gbc{inside}, \gbc{outside}, \gbc{centered}, \gbc{onleft},
\gbc{onright}, \gbc{ontop} or \gbc{onbottom}. The second command takes
four arguments and sets the position of the marks on each border. The
third command sets the position on all four border axis to the same
value. The last two commands are abbreviations for
\cs{setaxismarks}\marg{x}\marg{\meta{pos}} and
\cs{setaxismarks}\marg{y}\marg{\meta{pos}}, respectively.

Not all combinations make sense (for example,
\cs{setaxismarks}\marg{r}\marg{ontop}). In these cases, no error message
is produced: \gbc{ontop} and \gbc{onleft} give the same results, as do
\gbc{onbottom} and \gbc{onright}. The parameters \gbc{inside} and
\gbc{outside} make no sense for the $x$- and $y$-axes, but if they are
used then \gbc{inside} means \gbc{ontop} for the $x$-axis and
\gbc{onright} for the $y$-axis. These words are actually \MF{} numeric
variables and the variables \gbc{ontop} and \gbc{onleft}, for example,
have the same value.

\begin{cd}\pagelabel{grid}
\cs{grid}\oarg{\meta{size}}\marg{\meta{xsep},\meta{ysep}}\\
\cs{gridpoints}\oarg{\meta{size}}\marg{\meta{xsep},\meta{ysep}}\\
\cs{lattice}\oarg{\meta{size}}\marg{\meta{xsep},\meta{ysep}}\\
\cs{hgridlines}\marg{\meta{ysep}}\\
\cs{vgridlines}\marg{\meta{xsep}}\\
\cs{gridlines}\marg{\meta{xsep},\meta{ysep}}%
\index{grid@\cs{grid}}%
\index{gridpoints@\cs{gridpoints}}%
\index{lattice@\cs{lattice}}%
\index{vgridlines@\cs{vgridlines}}%
\index{hgridlines@\cs{hgridlines}}%
\index{gridlines@\cs{gridlines}}%
\end{cd}

\cs{grid} draws a dot at every point for which the first coordinate is
an integer multiple of the \meta{xsep} and the second coordinate is an
integer multiple of \meta{ysep}. The diameter of the dot is determined
by \meta{size}. The default is the value of \cs{griddotsize},
initially \dim{0.5pt}. Under the \opt{metapost} option, the color of the
dot is \gbc{pointcolor}. The commands \cs{gridpoints and \cs{lattice}}
are synonyms for \cs{grid}.

\cs{hgridlines} draws the horizontal and \cs{vgridlines} the vertical
lines through these same points. \cs{gridlines} draws both sets of
lines. The thickness of the lines is set by \cs{penwd}. Authors are
recommended to either reduce the pen width or change \gbc{drawcolor} to
a lighter color for grid lines. Or omit them entirely: well-designed
graphs usually don't need them and almost never should both horizontals
and verticals be used.

The above commands draw their dots and lines within the margins set by the
axis margin commands on page~\pageref{axismargin}.

\begin{cd}\pagelabel{plrgrid}
\cs{plrgrid}\marg{\meta{rsep},\meta{anglesep}}\\
\cs{gridarcs}\marg{\meta{rsep}}\\
\cs{gridrays}\marg{\meta{anglesep}}\\
\cs{plrpatch}\marg{\meta{rmin},\meta{rmax},\meta{rsep},%
    \meta{tmin},\meta{tmax},\meta{tsep}}\\
\cs{plrgridpoints}\oarg{\meta{size}}\marg{\meta{rsep},\meta{anglesep}}%
\index{plrgrid@\cs{plrgrid}}%
\index{plrpatch@\cs{plrpatch}}%
\index{gridarcs@\cs{gridarcs}}%
\index{gridrays@\cs{gridrays}}%
\index{plrgridpoints@\cs{plrgridpoints}}%
\end{cd}

\cs{plrgrid} fills the graph with circular arcs and radial lines.
\cs{gridarcs} draws only the arcs, \cs{gridrays} only the radial lines.
\cs{plrgridpoints} places a dot (diameter \meta{size}) at all the places
the rays and arcs would intersect. It takes an optional argument for the
size of the dots, the default being \cs{griddotsize}, the same as the
\cs{grid} command.

The arcs lie on circles centered at $(0,0)$ and the rays would all meet
at $(0,0)$ if extended. The corresponding \MF{} commands actually draw
just enough to cover the graph area and then clip them to the graph
boundaries. If you don't want them clipped, use \cs{plrpatch}. Unlike
the rectangular coordinate grid commands, these do not respect the axis
margins (rectangular margins don't really belong with polar coordinates).

\cs{plrpatch} draws arcs with radii starting at \meta{rmin}, stepping by
\meta{rsep} and ending with \meta{rmax}. Each arc goes from angle
\meta{tmin} to \meta{tmax}. It also draws radial lines with angles
starting at \meta{tmin}, stepping by \meta{tsep} and ending with
\meta{tmax}. Each line goes from radius \meta{rmin} to \meta{rmax}. If
$\meta{rmax}-\meta{rmin}$ doesn't happen to be a multiple of
\meta{rsep}, the arc with radius \meta{rmax} is drawn anyway. The same
is true of the line at angle \meta{tmax}, so that the entire boundary is
always drawn.

If \meta{tsep} is larger than \meta{tmax}${}-{}$\meta{tmin}, then only
the boundary rays will be drawn. If \meta{rsep} is larger than
\meta{rmax}${}-{}$\meta{rmin}, then only the boundary arcs will be
drawn.

The color used for rays and arcs is \gbc{drawcolor}, and for dots
\gbc{pointcolor}. The advice about color and use of \cs{gridlines} holds
for \cs{plrgrid} and its relatives as well.

\begin{cd}\pagelabel{vectorfield}
\cs{vectorfield}\oarg{\meta{hlen}}\marg{\meta{xsp},\meta{ysp}}%
\marg{\meta{formula}}\marg{\meta{restriction}}\\
\cs{plrvectorfield}\oarg{\meta{hlen}}\marg{\meta{rsp},\meta{tsp}}%
\marg{\meta{formula}}\marg{\meta{restriction}}%
\index{vectorfield@\cs{vectorfield}}%
\index{plrvectorfield@\cs{plrvectorfield}}
\end{cd}

These commande draw a field of vectors (arrows). The optional argument
is the length of the arrowhead, the default being the dimension
\cs{headlen}, initially \dim{3pt}.

For \cs{vectorfield}, an arrow is drawn starting from each point $(x,y)$
where $x$ is an integer multiple of \meta{xsp} and $y$ is an integer
multiple of \meta{ysp}. The vector field is given by \meta{formula},
which should be a pair-valued expression in the literal variables
\mfc{x} and \mfc{y}. Typically that would be a pair of numeric
expressions enclosed in parentheses and separated by a comma. The last
argument is a boolean expression in the literal variables \mfc{x} and
\mfc{y}, used to restrict the domain. That is, if the expression is
false for some $(x,y)$, no arrow is drawn at that point. If you do not
wish to restrict the domain, type \texttt{true} for the restriction.

For \cs{plrvectorfield}, an arrow is drawn starting from each point with
polar coordinates $(r,\theta)$ if $r$ is an integer multiple of
\meta{rsp} and $\theta$ is an integer multiple of \meta{tsp}. In this
case, the \meta{formula} must be a pair-valued expression in the literal
variables \mfc{r} and \mfc{t}. This should be (or produce) a pair of $x$
and $y$ coorinates, not a polar coordinate pair. If you have formulas
$R(r,\theta)$ for the length of each vector and $T(r,\theta)$ for the
angle, then the following will convert to $(x,y)$ pairs:
\begin{verbatim}
{polar (R(r,t),T(r,t))}
\end{verbatim}
The last argument is as in \cs{vectorfield}, except it should depend on
the literal variables \mfc{r} and \mfc{t}.

In either case, the arrow is not drawn if the starting point would lie
outside the borders set with \cs{axismargins} and its relatives.

The following draws a rotational field, omitting the inside of the
circle of radius $1$, where the arrows would be excessively long, and
especially avoiding $(0,0)$ where the vector field is undefined.
\begin{verbatim}
\vectorfield[2.5pt]{.25,.25}{.5*(-y,x)/(x**2 + y**2)}{x**2 + y**2 >= 1}
\end{verbatim}
The following is the same field, represented by arrows whose locations
are regularly spaced in polar coordinates.
\begin{verbatim}
\plrvectorfield[2.5pt]{.25,20}{polar(.5/r,t+90)}{r >= 1}
\end{verbatim}


\subsection{Circles, arcs and ellipses}\label{circles}

\begin{cd}\pagelabel{circle}
\cs{circle}\oarg{\meta{format}}\marg{\meta{specification}}%
\index{circle@\cs{circle}}%
\end{cd}

This figure macro produces a circle. Starting with \mfp{} version 0.7,
there are more than one way to specify a circle. In version 0.8 and
later there are six ways, and one selects which one by giving
\cs{circle} an optional argument that signals what data will be
specified in the mandatory argument.

\begin{cd}
\cs{circle}\oarg{p}\marg{\meta{$c$},\meta{$r$}}\\
\cs{circle}\oarg{c}\marg{\meta{$c$},\meta{$p$}}\\
\cs{circle}\oarg{t}\marg{\meta{$p_1$},\meta{$p_2$},\meta{$p_3$}}\\
\cs{circle}\oarg{s}\marg{\meta{$p_1$},\meta{$p_2$},\meta{$\theta$}}\\
\cs{circle}\oarg{r}\marg{\meta{$p_1$},\meta{$p_2$},\meta{$r$}}\\
\cs{circle}\oarg{q}\marg{\meta{$p_1$},\meta{$p_2$},\meta{$r$}}%
\index{circle@\cs{circle}}%
\end{cd}

The optional arguments produce circles according to the following
descriptions.
%
\begin{description}
\item[\oarg{p}] The \textit{Polar form} is the default. The data in the
mandatory argument should then be the center \meta{c} and radius
\meta{r} of the circle. A negative radius is a mathematical error, but
it is accepted. It produces the same circle, with the same sense, but
the starting point (normally \meta{$r$} units to the right of the
center) is \meta{$r$} units \emph{left} of the center.

\item[\oarg{c}] The \textit{center-point form}. In this case the data
should be the center and one point on the circumference. The circle
starts at the point and has an anticlockwise sense.

\item[\oarg{t}] The \textit{three-point form}. The data are three points
that do not lie in a straight line. The circle starts at the first point
and has the sense determined by the order of the points.

\item[\oarg{s}] The \textit{point-sweep form}. The data are two points on the
circle, followed by the angle of arc between them. This circle starts at
the first point and has a sense determined by the angle: anticlockwise
for positive angles, clockwise for negative.

\item[\oarg{r}] The \textit{point-radius form}. The data are two points on the
circle, followed by the radius. There are two circles with this data.
The one that makes the angle from the first to the second point positive
and less than $180$ degrees is produced. The sense of the circle is
normally anticlockwise starting at the first point. Using a negative
radius is a mathematical error, but this command just produces the other
circle with the opposite sense.

\item[\oarg{q}] The \textit{alternative point-radius form}. The data are
the same as for the \oarg{r} case, except the other circle is produced.
That is, a circle starting at the first point, proceeding anticlockwise
through an angle greater than $180$ degrees to the second point, then
along the shorter arc to the first point. Again, a negative radius
produces the other circle with clockwise sense.
\end{description}
%
These optional arguments are also used in the \cs{arc} command (see
below). The \cs{circle} command draws the whole circle of which the
corresponding \cs{arc} command draws only a part.

\begin{cd}\pagelabel{arc}
\cs{arc}\oarg{\meta{format}}\marg{\meta{specification}}\\
\cs{arc*}\oarg{\meta{format}}\marg{\meta{specification}}%
\index{arc@\cs{arc}}%
\end{cd}

This figure macro produces a circular arc specified as determined by the
\meta{format} optional parameter. As with \cs{circle}, the optional
\meta{format} parameter determines the format of the other parameter, as
indicated below. The user is responsible for ensuring that the parameter
values make geometric sense. The starting point of each arc is at the
first specified angle or point and the ending point is at the last one.

The star-form produces the complementary arc. That is, instead of the
arc described below, it produces the rest of the circle from the ending
point to the starting point of the arc described.

\begin{cd}
\cs{arc}\oarg{s}\marg{\meta{$p_0$},\meta{$p_1$},\meta{$\theta$}}\\
\cs{arc}\oarg{p}\marg{\meta{$c$},\meta{$\theta_1$},\meta{$\theta_2$},\meta{$r$}}\\
\cs{arc}\oarg{a}\marg{\meta{$c$},\meta{$r$},\meta{$\theta_1$},\meta{$\theta_2$}}\\
\cs{arc}\oarg{c}\marg{\meta{$c$},\meta{$p_1$},\meta{$\theta$}}\\
\cs{arc}\oarg{t}\marg{\meta{$p_0$},\meta{$p_1$},\meta{$p_2$}}\\
\cs{arc}\oarg{r}\marg{\meta{$p_0$},\meta{$p_1$},\meta{$r$}}\\
\cs{arc}\oarg{q}\marg{\meta{$p_0$},\meta{$p_1$},\meta{$r$}}%
\index{arc@\cs{arc}}%
\end{cd}

The optional arguments produce arcs according to the following
descriptions.
\begin{description}
\item[\oarg{s}] The \textit{point-sweep form} is the default format. It
draws the circular arc starting from the point \meta{$p_0$}, ending at
the point \meta{$p_1$}, and covering an arc angle of \meta{$\theta$}
degrees, measured anticlockwise around the center of the circle. If,
for example, the points \meta{$p_0$} and \meta{$p_1$} lie on a
horizontal line with \meta{$p_0$} to the \emph{left}, and \meta{$\theta$}
is between $0$~and $360$ (degrees), then the arc will sweep \emph{below} the
horizontal line (in order for the arc to be anticlockwise). A
negative value of \meta{$\theta$} gives a clockwise arc from \meta{$p_0$}
to \meta{$p_1$}.

\item[\oarg{p}] The \textit{polar form} draws the arc of a circle with
center \meta{$c$} starting at the angle \meta{$\theta_1$} and ending at
the angle \meta{$\theta_2$}, with radius \meta{$r$}. Both angles are
measured anticlockwise from the positive $x$ axis. If the first angle is
less than the second, the arc has an anticlockwise sense, otherwise
clockwise. A negative radius is a mathematical error, but the result is
the arc on the opposite side of the circle, as if both angles were
increased by $180$ degrees

\item[\oarg{a}] The alternative polar form differs from the polar form
above only in the order of the arguments. This seems (to me) a more
reasonable order, and matches the order \cs{sector} requires (see below).
The p option is retained for backward compatibility.

\item[\oarg{c}] The \textit{center-point-angle form} draws the circular
arc with center \meta{$c$}, starting at the point \meta{$p_1$}, and
sweeping an angle of \meta{$\theta$} around the center from that point.
This is the fundamental method for drawing arcs. All other methods are
converted to this or the point-sweep method. Even the point sweep form
is converted to this one for angles greater than 90 degrees.

\item[\oarg{t}] The \textit{three-point form} draws the circular arc
which passes through all three points given, in the order given.
Internally, this is converted to two applications of the point-sweep
form.

\item[\oarg{r}] The \textit{point-radius form} draws an arc on the
circle that \cs{circle}\oarg{r} would produce. The arc starts at the
point \meta{$p_0$} and ends at \meta{$p_1$}. Of the two possible arcs on
that circle, it produces the shorter one that covers an angle $\theta$
from $0$ to $180$ degrees measured anticlockwise around the center
of the circle. A negative radius is a mathematical error, but the result
is the short arc on the other circle with a clockwise sense.

\item[\oarg{q}] The \textit{alternative point-radius form} is the same as
\oarg{r} except it produces the longer arc, that covers an angle
$\theta$ between $180$ and $360$ degrees measured anticlockwise around
the center of the circle. A negative radius is a mathematical error, but
the result is the longer arc on the other circle with a clockwise sense.
\end{description}

For both options \oarg{r} and \oarg{q} the angle is computed and then
the point-sweep method is used. If the absolute value of the radius is
less than half the distance between the points, then no such arc exists.
In this case, the angle is just set equal to $\pm180$ degrees (as if the
radius were changed to half the distance).

\begin{cd}\pagelabel{sector}
\cs{sector}\marg{\meta{$c$},\meta{$r$},\meta{$\theta_1$},\meta{$\theta_2$}}%
\index{sector@\cs{sector}}%
\end{cd}

This figure macro produces the sector of the circle with center at the
point \meta{$c$} and radius \meta{$r$}, from the angle \meta{$\theta_1$}
to the angle \meta{$\theta_2$}. Both angles are measured in degrees
anticlockwise from the direction parallel to the $x$ axis. The sector
forms a closed path. \emph{Note}: \cs{sector} and \cs{arc}\oarg{p} have
the same parameters, but \emph{in a different order}.%
    \footnote{This apparently was unintended, but we now have to live
    with it so as not to break existing \file{.tex} files.}


\begin{cd}\pagelabel{ellipse}
\cs{ellipse}\oarg{\meta{$\theta$}}\marg{\meta{$c$},\meta{$r_x$},\meta{$r_y$}}%
\index{ellipse@\cs{ellipse}}%
\end{cd}

This figure macro produces an ellipse with the $x$ radius \meta{$r_x$}
and $y$ radius \meta{$r_y$}, centered at the point \meta{$c$}. The
optional parameter \meta{$\theta$} provides a way of rotating the
ellipse by \meta{$\theta$} degrees anticlockwise around its center.
Ellipses may also be created by differentially scaling a circle and
perhaps rotating the result. See subsection~\ref{transformation}.


When dealing with arcs and circles, it is useful to work in polar
coordinates:

\begin{cd}\pagelabel{plr}
\cs{plr}\marg{(\meta{$r_0$},\meta{$\theta_0$}),%
  (\meta{$r_1$},\meta{$\theta_1$}), $\ldots$}%
\index{plr@\cs{plr}}%
\end{cd}

The macro \cs{plr} causes \MF{} to replace the specified list of polar
coordinate pairs by the equivalent list of rectangular (cartesian)
coordinate pairs. Through \cs{plr}, commands designed for rectangular
coordinates can be applied to data represented in polar coordinates. It
must be cautioned that this wholesale conversion of a list applies only
to commands that take a list consisting of an arbitrary number of
points, such as \cs{polyline}.

The effect of \cs{plr} is to apply a \MF{} command, \gbc{polar}, to each
point in the list, producing a new list. This \MF{} command can also be
used separately in any situation where a single \MF{} point is required. For
example, to connect the point $(2,3)$ to the point with polar
coordinates $(1, 135)$ write
\begin{verbatim}
\polyline{(2,3),polar(1,135)}
\end{verbatim}

This last circle-producing macro I wrote for my own use. It produces a
circle associated with the hyperbolic geometry of a disk or a
half-plane.

\begin{cd}\pagelabel{pshcircle}
\cs{pshcircle}\marg{\meta{center},\meta{radius}}\\
\cs{pshcircle}*\marg{\meta{center},\meta{radius}}%
\index{pshcircle@\cs{pshcircle}}%
\end{cd}

This produces the circle whose hyperbolic center is at \meta{center} and
whose pseudohyperbolic radius is \meta{radius}. This all takes place
inside the circle with center $(0,0)$ and radius $1$ (the \emph{unit
circle}). The \meta{center} is required to be inside the unit circle and
the \meta{radius} is required to be less than $1$.

The star-form is for the \emph{upper half-plane}, which is the set of
points with positive $y$-coordinate In this case, the \meta{center} must
be in the upper half-plane and the \meta{radius} must still be less than
$1$. If you are not versed in hyperbolic geometry, be warned that the
actual diameter of the resulting circle is on the order of $2y/(1-R)$,
where $R$ is the \meta{radius}. This can be quite large even for modest
values of $R$ and $y$.


\subsection{Curves}\label{curves}

\begin{cd}\pagelabel{curve}
\cs{curve}\oarg{\meta{tension}}\marg{\meta{$p_0$},\meta{$p_1$},$\ldots$}\\
\cs{cyclic}\oarg{\meta{tension}}\marg{\meta{$p_0$},\meta{$p_1$},$\ldots$}\\
\cs{closedcurve}\oarg{\meta{tension}}\marg{\meta{$p_0$},\meta{$p_1$},$\ldots$}%
\index{curve@\cs{curve}}%
\index{cyclic@\cs{cyclic}}
\index{closedcurve@\cs{closedcurve}}%
\end{cd}

These figure macros produce a smooth path through the specified points,
in the specified order. It is `smooth' in two ways: it never changes
direction abruptly (no `corners' or `cusps' on the curve), and it tries
to make turns that are not too sharp. This latter property is acheived
by specifying (to \MF{}) that the tangent to the curve at each listed
point is to be parallel to the line from that point's predecessor to its
successor. The \cs{cyclic} variant arranges for the last point to be
connected (smoothly) to the first, and produces a closed \MF{} B\'ezier
curve. The command \cs{closedcurve} is an alias for \cs{cyclic}.

The optional \meta{tension} influences \emph{how} smooth the curve is.
The special value \mfc{infinity} (in fact, usually anything greater than
about $10$), makes the curve not visibly different from a polyline. The
higher the value of tension, the sharper the corners on the curve and
the flatter the portions in between. \CMF{} requires the tension to be
larger than $0.75$. The default value of the tension is $1$ when \mfp{} is
loaded, but that can be changed with the following command.

\begin{cd}\pagelabel{settension}
\cs{settension}\marg{\meta{num}}%
\index{settension@\cs{settension}}
\end{cd}

This sets the default tension for all commands that take an optional
tension parameter.

Sometimes one would like a convex set of points to produce a convex
curve. This will not always be the case with \cs{curve} or \cs{cyclic}.
You can verify this with the following example, where the list of points
traces a rectangle:
\begin{verbatim}
\cyclic{(0,0),(0,1),(1,1),(2,1),(2,0),(0,0)}
\end{verbatim}
To produce a convex curve, use one of the following:

\begin{cd}\pagelabel{convexcurve}
\cs{convexcurve}\oarg{\meta{tension}}\marg{\meta{$p_0$},\meta{$p_1$},$\ldots$}\\
\cs{convexcyclic}\oarg{\meta{tension}}\marg{\meta{$p_0$},\meta{$p_1$},$\ldots$}\\
\cs{closedconvexcurve}\oarg{\meta{tension}}\marg{\meta{$p_0$},\meta{$p_1$},$\ldots$}%
\index{convexcurve@\cs{convexcurve}}%
\index{convexcyclic@\cs{convexcyclic}}%
\index{closedconvexcurve@\cs{closedconvexcurve}}%
\end{cd}

These figure macros can be used even if the list of points is not
convex, and the result will be convex where possible. The third one is
an alias for for the second one.

\medskip
Occasionally it is necessary to specify a sequence of points with
\emph{increasing} $x$-coordinates and draw a curve through them. One
would then like the resulting curve both to be smooth \textit{and} to
represent a function (that is, the curve always has increasing $x$
coordinate, never turning leftward). This cannot be guaranteed with the
\cs{curve} command unless the tension is \texttt{infinity}.

\begin{cd}\pagelabel{fcncurve}
\cs{fcncurve}\oarg{\meta{tension}}\marg{($x_0$,$y_0$),($x_1$,$y_1$),$\ldots$}%
\index{fcncurve@\cs{fcncurve}}%
\end{cd}

This figure macro produces a curve through the points specified. If the
points are listed with increasing (or decreasing) $x$ coordinates, the
curve will also have increasing (resp., decreasing) $x$ coordinates. The
\meta{tension} is a number greater than $1/3$ which controls how tightly
the curve is drawn. Generally, the larger it is, the closer the curve is
to the polyline through the points. The default tension is that set with
\cs{settension}, initially $1$. For those who know something about
\MF{}, this `tension' is not the same as the \MF{} notion of tension
(the tension in the \cs{curve} command), but it functions in a similar
fashion. In this case it can actually be any positive number, but only
values greater than $1/3$ guarantee the property of never doubling back.

\begin{cd}\pagelabel{turtle}
\cs{turtle}\marg{\meta{$p_0$},\meta{$v_1$},\meta{$v_2$},$\ldots$}%
\index{turtle@\cs{turtle}}%
\end{cd}

This figure macro produces a a sequence of line segments starting from
the point \meta{$p_0$}, and extending along the (2-dimen\-sional vector)
displacement \meta{$v_1$}. The next segment is from the previous
segment's endpoint, along displacement \meta{$v_2$}. This continues for
all listed displacements, a process similar to `turtle graphics'.


\subsection{Bar charts and pie charts}\label{charts}

\begin{cd}\pagelabel{barchart}
\cs{barchart}\oarg{\meta{start},\meta{sep},\meta{r}}%
  \marg{\meta{h-or-v}}\marg{\meta{list}}\\
\cs{bargraph}\dots\\
\cs{gantt}\dots\\
\cs{histogram}\dots\\
\cs{mfpbarchart}\dots\\
\cs{mfpbargraph}\dots\\
\cs{mfpgantt}\dots\\
\cs{mfphistogram}\dots
\index{barchart@\cs{barchart}}%
\index{bargraph@\cs{bargraph}}%
\index{histogram@\cs{histogram}}%
\index{gantt@\cs{gantt}}%
\end{cd}

The macro \cs{barchart} computes a bar chart or a Gantt chart. It does
not draw the bars, but only defines their rectangular paths which the
user may then draw or fill or both using the \cs{chartbar} macros (see
below). Since bar charts have many names, \cs{bargraph} and
\cs{histogram} are provided as synonyms. The macro \cs{gantt} is also a
synonym; whether a Gantt chart or bar chart is created depends on the
data.

Since \cs{barchart} never draws anything, there is no particular reason
it needs to be inside an \env{mfpic} environment. Starting with version
0.9 of \mfp{} this is no longer required, but the command name
\cs{mfpbarchart} must be used outside (in case some other package also
defines \cs{barchart}). One can use any of the four synonyms listed that
start with `\cs{mfp}'. The command to draw the bars is still required to
be inside an \env{mfpic} environment.

\meta{h-or-v} should be \texttt{v} if you want the ends of the bars to
be measured vertically from the $x$-axis, or \texttt{h} if they should
be measured horizontally from the $y$-axis. \meta{list} should be a
comma-separated list of numbers and ordered pairs giving the
end(s) of each bar. A number $c$ is interpreted as the pair $(0,c)$; a
pair $(a,b)$ is interpreted as an interval giving the ends of the bar
(for Gantt diagrams). The rest of this description refers to the
\texttt{h} case; the \texttt{v} case is analogous.

By default the bars are 1 graph unit high (thickness), from $y = n-1$ to
$y = n$. Their width and location are determined by the data. The
optional parameter consists of three numeric parameters separated by
commas. \meta{start} is the $y$-coordinate of the bottom edge of the
first bar, \meta{sep} is the distance between the  bottom edges of
successive bars, and \meta{r} is the fraction of \meta{sep} occupied by
each bar. The default behavior corresponds to \texttt{[0,1,1]}. In
general, bar number $n$ will be from $y = \meta{start} +
(n-1)*\meta{sep}$ to $y = \meta{start} + (n-1 + \meta{r})*\meta{sep}$

Notice the bars are numbered in order from bottom to top. You can
reverse them by making \meta{sep} negative, and making \meta{start} the
top edge of the first bar.

The fraction \meta{r}  should be between $-1$ and $1$. A negative value
reverses the direction from the `leading edge' of the bar to the
`trailing edge'. For example, if one bar chart is created with
\begin{ex}
  \cs{barchart}\oarg{1,1,-.4}\marg{h}\marg{$\ldots$}
\end{ex}
and another with
\begin{ex}
  \cs{barchart}\oarg{1,1,.4}\marg{h}\marg{$\ldots$}
\end{ex}
both having the same number of bars, then the first will have its first
bar from $y = 1$ to $y = 1 -.4 = .6$, while the second will have its
first bar on top of that one, from $1$ to $1 + .4$. Similarly the next
bars will be above and below $y=2$, etc. This makes it easy to draw bars
next to one another for comparison.

\begin{cd}\pagelabel{chartbar}
\cs{chartbar}\marg{\meta{num}}\\
\cs{graphbar}\marg{\meta{num}}\\
\cs{histobar}\marg{\meta{num}}\\
\cs{ganttbar}\marg{\meta{num}}%
\index{chartbar@\cs{chartbar}}%
\index{graphbar@\cs{graphbar}}%
\index{histobar@\cs{histobar}}%
\index{ganttbar@\cs{ganttbar}}%
\end{cd}

The figure macro \cs{chartbar} (synonyms \cs{graphbar}, \cs{ganttbar},
and \cs{histobar}) takes a number from $1$ to the number of elements in
the list of data of the most recent \cs{barchart} command and produces
the corresponding rectangular path computed by that command. This
behaves just like any other figure macro, and the prefix macros from
section~\ref{rendering} may be used to give adjacent bars contrasting
colors, fills, etc.


\begin{cd}\pagelabel{piechart}
\cs{piechart}\oarg{\meta{dir}\meta{angle}}\marg{\meta{$c$},\meta{$r$}}%
    \marg{\meta{list}}\\
\cs{mfppiechart}\dots
\index{piechart@\cs{piechart}}%
\end{cd}

The macro \cs{piechart} also does not draw anything, but computes the
\cs{piewedge} regions described below. The first part of the optional
parameter, \meta{dir}, is a single letter to indicate a direction:
`\texttt{c}' for \emph{clockwise} or `\texttt{a}' for \emph{anticlockwise}.
The \meta{angle} is the angle in degrees of the starting edge of the
first wedge. The defaults correspond to \oarg{c90}, which means the
first wedge starts at 12~o'clock and proceeds clockwise.

The first required argument contains the center \meta{$c$} and radius
\meta{$r$} of the chart. The second required argument is the list of
data: positive numbers separated by commas.

Since this command never actually draws anything, only defining the
wedges, it makes sense to heave it available outside the drawing
environment. Starting with version 0.9 of \mfp{} that is the case, but
the command name is \cs{mfppiechart} (to avoid a name clash with some
other package's \cs{piechart} command). The command to draw wedges
(\cs{piewedge}, see below) is still required to be inside an \env{mfpic}
environment.

\begin{cd}\pagelabel{piewedge}
\cs{piewedge}\oarg{\meta{spec}\meta{trans}}\marg{\meta{num}}%
\index{piewedge@\cs{piewedge}}%
\end{cd}

This figure macro takes a number from $1$ to the number of elements in
the list of data of the most recent \cs{piechart} command and produces
the corresponding wedge-shaped path computed by that command. By
default, the path is positioned as computed by that \cs{piechart}
command, but The optional argument to \cs{piewedge} can override this.
The parameter \meta{spec} is a single letter, which can be \texttt{x},
\texttt{s} or \texttt{m}. The \texttt{x} stands for \emph{exploded} and
it means the wedge is moved directly out from the center of the pie a
distance \meta{trans}. \meta{trans} should then be a pure number and is
interpreted as a distance in graph units. The \texttt{s} stands for
\emph{shifted} and in this case \meta{trans} should be a pair of the
form \texttt{(\meta{dx},\meta{dy})} indicating the wedge should be
shifted \meta{dx} horizontally and \meta{dy} vertically (in graph
units). The \texttt{m} stands for \emph{move to}, and \meta{trans} is
then the absolute coordinates \texttt{(\meta{x},\meta{y})} in the graph
where the point of the wedge should be placed.


\section{Colors in \mfp{}.}\label{colors}

\subsection{\CMP{} color functions}\label{mpcolors}

Because of changes to color handling with \MP{} 1.000, we will have to
give two descriptions of some operations. For brevity, we will refer to
\MP{} versions before the addition of the \kw{cmykcolor} data type as
`early' \MP{} and the versions afterward as `recent' \MP{}. Early \MP{}
actually ended with version 0.642. When development resumed, beta test
versions began with 0.900. Any version 0.900 or later qualifies as
`recent'.

In early \MP{}, the only \kw{color} data type is a triple of numbers
like \mfc{(1,.5,.5)}, with the components between 0 and 1, representing
red, green and blue levels, respectively. White is given by
\mfc{(1,1,1)} and black by \mfc{(0,0,0)}. Recent \MP{} has the
\kw{color} data type (refered to as either \kw{color} or \kw{rgbcolor})
as well as the \kw{cmykcolor} type. A \kw{cmykcolor} is a quadruple of
numbers like \mfc{(1,.2,0,.3)}, with components between 0 and 1
representing levels of cyan, magenta, yellow and black. White is
represented by \mfc{(0,0,0,0)}. While black can be obtained in several
ways,\mfc{(0,0,0,1)} is the simplest.

\CMP{} also has \kw{color} variables (and \kw{cmykcolor} variables) and
several have been predefined. The colors \mfc{red}, \mfc{green},
\mfc{blue}, \mfc{white} and \mfc{black} are built in to \MP{} and are of
type \kw{rgbcolor}. Colors \gbc{cyan}, \gbc{magenta} and \gbc{yellow}
are defined by \mfp{}'s \MP{} support macros to be \kw{cmykcolor}. In
addition, \mfp{} defines \gbc{cmykblack}, \gbc{cmykwhite},
\gbc{rgbblack}, \gbc{rgbwhite}, \gbc{grayscaleblack} and
\gbc{grayscalewhite}. These give black and white in the indicated data
type (grayscale being a numeric: $0$ for black, $1$ for white).

All the names in the \LaTeX{} \prog{color} package's \file{dvipsnam.def}
have also been predefined by \mfp{} as color variable names. Since \MP{}
allows color expressions, colors may be added (as long as they are the
same type) and multiplied by numerics. Multiplication by a number
between $0$ and $1$ darkens a \kw{rgbcolor}, but lightens a
\kw{cmykcolor}.

Moreover, several \MP{} color functions have been defined in
\file{grafbase.mp}. These have the same names as the color models.
Strictly speaking, it is never necessary to use these in recent \MP{}.
However, since \MF{} and early \MP{} don't have a data type consisting
of quadruples, and \MF{} doesn't have one for triples, these functions
allow the same \mfp{} code to be used for all three figure processors.
These functions are defined to convert to a usable data type, (which may
be ignored in \MF{}).

\begin{cd}
\mfc{cmyk($c$,$m$,$y$,$k$)}%
\index{cmyk@\mfc{cmyk($c$,$m$,$y$,$k$)}}
\end{cd}

In early \MP{}, this converts a cmyk color specification to \MP{}'s
native rgb. For example, the command \mfc{cmyk(1,0,0,0)} yields
\mfc{(0,1,1)}, which is the rgb equivalent of cyan. In recent \MP{} this
produces the \kw{cmykcolor} with the given components. That is,
\gbc{cmyk(1,0,0,0)} simply produces $(1,0,0,0)$, the cmyk coding for
cyan.

\begin{cd}
\mfc{gray($g$)}%
\index{gray@\mfc{gray($g$)}}
\end{cd}

In early \MP{}, this converts a numeric $g$ (designating a level of
gray) to the corresponding multiple of white: \mfc{($g$,$g$,$g$)}. In
recent \MP{}, commands to draw paths or pictures in a particular color
will accept a \kw{numeric} parameter instead of \kw{color} or
\kw{cmykcolor}, so in recent \MP{} this command simply returns the given
numeric $g$.

\begin{cd}
\mfc{named(\meta{name})},
\mfc{rgb($r$,$g$,$b$)}%
\index{named@\mfc{named(\meta{name})}}%
\index{rgb@\mfc{rgb($r$,$g$,$b$)}}
\end{cd}

These are essentially no-ops. However; \mfc{rgb()} will truncate the
arguments to the 0--1 range, and set an unknown argument to 0. An
unknown \meta{name} is converted to \mfc{black} (in the appropriate
color model if \meta{name} is an unknown color variable, otherwise rgb
black).

\begin{cd}
\mfc{RGB($R$,$G$,$B$)}%
\index{RGB@\mfc{RGB($R$,$G$,$B$)}}
\end{cd}

Converts an RGB color specification to rgb. It divides each component by
255, and performs the same truncations as \gbc{rgb()}. The RGB model
consists of a triple of numbers between 0 and 255. Originally, the model
required they be integers. However, since they are converted to
fractions anyway, it doesn't matter in this command.

\medskip
As an example of the use of these functions, in early \MP{} one could
conceivable write:
%
\begin{verbatim}
\draw[0.5*RGB(255,0,0)+0.5*cmyk(1,0,0,0)]\circle{(0,0),1}
\end{verbatim}
%
to have a circle drawn in a color halfway between red and cyan (which
turns out to be the same as \gbc{gray(0.5)}). In recent \MP{}, however,
this would be an error, as one cannot add two different data types
(\kw{rgbcolor} and \kw{cmykcolor}). So \mfp{} supplies conversion functions.

\begin{cd}
\mfc{makecmyk \meta{clr}}\\
\mfc{makergb  \meta{clr}}\\
\mfc{makegray \meta{clr}}%
\index{makecmyk@\mfc{makecmyk}}%
\index{makergb@\mfc{makergb}}%
\index{makegray@\mfc{makegray}}
\end{cd}

In recent \MP{}, the \meta{clr} can be a known color name, a constant
of type \kw{numeric}, \kw{rgbcolor}, or \kw{cmykcolor}, or the result of
a color function. Then \mfc{makecmyk} returns the \kw{cmykcolor}
equivalent, and \mfc{makergb} returns the \kw{rgbcolor} equivalent (a
\kw{numeric} \meta{clr} is interpreted as a grayscale color). Unknown
colors produce a black in the appropriate model. Then one can use
%
\begin{verbatim}
\draw{.5*RGB(255,0,0) + .5*makergb cmyk(1,0,0,0)}\circle{(0,0),1}
\end{verbatim}
%
If one didn't know that \mfc{RGB} returned an \kw{rgbcolor}, one
could write \verb$makergb RGB(255,0,0)$ to be sure to get an
\kw{rgbcolor}.

The first two commands are never necessary in early \MP{}, but they are
still defined: they simply return the given color if it is a known
argument of type \kw{color}, or apply the function \gbc{gray()} is it is
\kw{numeric}, and return black for an unknown name.

The last one \gbc{makegray} converts any color to a numeric, and then
returns either that number (recent \MP) or that multiple of \mfc{white}
(early \MP). In \MF{}, all three pass the (presumably numeric) argument
\meta{clr} unchanged.

All three functions return some kind of black if \meta{clr} is not some
kind of color, or has an unknown value.


\subsection{Establishing \mfp{} default colors}\label{defaultcolors}

\begin{cd}\pagelabel{drawcolor}
\cs{drawcolor}\oarg{\meta{model}}\marg{\meta{colorspec}}\\
\cs{fillcolor}\oarg{\meta{model}}\marg{\meta{colorspec}}\\
\cs{hatchcolor}\oarg{\meta{model}}\marg{\meta{colorspec}}\\
\cs{pointcolor}\oarg{\meta{model}}\marg{\meta{colorspec}}\\
\cs{headcolor}\oarg{\meta{model}}\marg{\meta{colorspec}}\\
\cs{tlabelcolor}\oarg{\meta{model}}\marg{\meta{colorspec}}\\
\cs{backgroundcolor}\oarg{\meta{model}}\marg{\meta{colorspec}}%
\index{drawcolor@\cs{drawcolor}}%
\index{fillcolor@\cs{fillcolor}}%
\index{hatchcolor@\cs{hatchcolor}}%
\index{pointcolor@\cs{pointcolor}}%
\index{headcolor@\cs{headcolor}}%
\index{tlabelcolor@\cs{tlabelcolor}}%
\index{backgroundcolor@\cs{backgroundcolor}}%
\end{cd}

These macros set the default color for various drawing elements. Any
curve (with one exception, those drawn by \cs{plotdata}), whether solid,
dashed, dotted, or plotted in symbols, will be in the color set by
\cs{drawcolor}. Set the color used by \cs{gfill} with \cs{fillcolor}.
For all the hatching commands use \cs{hatchcolor}. For the \cs{point},
\cs{plotsymbol} and \cs{gridpoints} commands use \cs{pointcolor}, and
for arrowheads, \cs{headcolor}. One can set the color used by
\cs{gclear} with \cs{backgroundcolor} (the same color will also be used
in the interior of unfilled points that are drawn with \cs{point}) and,
when \opt{mplabels} is in effect, the color of labels can be set with
\cs{tlabelcolor}.

The optional \meta{model} may be one of \texttt{rgb}, \texttt{RGB},
\texttt{cmyk}, \texttt{gray}, and \texttt{named}. The \meta{colorspec}
depends on the model, as outlined below. Each of these commands sets a
corresponding \MP{} color variable with the same name (except
\cs{backgroundcolor} sets the color named \mfc{background}). Thus, after
\texttt{drawcolor} has been set, one can issue the command
\cs{fillcolor}\marg{drawcolor} to fill with the same color.

As previously discussed, all these colors are initially set to
\mfc{black} except \mfc{background} is set to \mfc{white}.

If the optional \meta{model} argument is omitted, the color
specification may be any expression recognized as a color by \MP{}. It
is highly recommended (for portability) that one use either a predefined
name or one of the color functions of the previous section.

When the optional \meta{model} is specified in the color setting
commands, it determines the format of the color specification as in
figure~\ref{fig:colorspecs}.

\begin{figure}[hbt]
\halign{\quad\texttt{#}\hfil\quad&#\hfil\cr
\omit\quad{\slshape Model:}\hfil& {\slshape Specification:}\cr
rgb     & Three numbers in the range 0 to 1 separated by commas.\cr
RGB     & Three numbers in the range 0 to 255 separated by commas.\cr
cmyk    & Four numbers in the range 0 to 1 separated by commas.\cr
gray    & One number in the range 0 to 1, with 0 indicating
                  black, 1 white.\cr
named   & A \MP{} color variable name either predefined by
                   \mfp{} or by the user.\cr}
\caption{Color specifications}
\label{fig:colorspecs}
\end{figure}

\pagebreak[3]
\Mfp{} translates the command:
\begin{verbatim}
\fillcolor[cmyk]{1,.3,0,.2}
\end{verbatim}
into the equivalent of:
\begin{verbatim}
\fillcolor{cmyk(1,.3,0,.2)}.
\end{verbatim}
Note that when the optional model is specified, the color specification
must \emph{not} be enclosed in parentheses. Note also that each model
name is the name of a color function described in the previous
subsection. That is how the models are implemented internally. One sees
from this that the optional argument is never necessary. It's there only
to make the \LaTeX{} user comfortable.


\subsection{Defining a color name}\label{colorname}

\begin{cd}\pagelabel{mfpdefinecolor}
\cs{mfpdefinecolor}\marg{\meta{name}}\marg{\meta{model}}\marg{\meta{colorspec}}%
\index{mfpdefinecolor@\cs{mfpdefinecolor}}
\end{cd}

This defines a color variable \meta{name} for later use, either in the
commands \cs{drawcolor}, etc., or in the optional parameters to
\cs{draw}, etc. The name can be used alone or in the \texttt{named} model.
The mandatory \meta{model} and \meta{colorspec} are as above.

\medskip
A final caution, the colors of an \mfp{} figure are stored in the
\file{.mp} output file, and are not related to colors used or defined by
the \LaTeX{} \prog{color} package. In particular a color defined only by
\LaTeX{}'s \cs{definecolor} command will remain unknown to \mfp{}.
Conversely, \LaTeX{} commands will not recognize any color defined only
by \cs{mfpdefinecolor}.


\subsection{\CMF{} colors}\label{MFcolor}

\CMF{} was never meant to understand colors, but it certainly can be
taught the difference between black and white and, to a limited extent,
various grays. Starting with version 0.7, \mfp{} will not generate
an error when a color-changing command is used under the \opt{metafont}
option. Instead, when possible, the variables that represent colors in
\MP{} will be converted to a numeric value between 0 and 1 in \MF{}. When
possible (for example, when a region is filled) the numeric will be
interpreted as a gray level and shading (see subsection~\ref{filling})
will be used to approximate the gray. In other cases (drawing or dashing
of curves, placing of points or symbols, filling with a pattern of hatch
lines) the number will be interpreted as black or white: a value less
than 1 will cause the figure to be rendered in black, while a value
equal to 1 (white) will cause pixels corresponding to the figure to be
erased.

This depends on adhering to certain restrictions. \CMF{}'s syntax
does not recognize a triple of numbers as any sort of data structure,
but it does allow \emph{commands} to have any number of parameters in
parentheses. So colors must be specified using the color commands such
as \gbc{rgb(1,1,0)} or color names such as \gbc{yellow}, and never as a
bare triple. Also, as currently written, the color names defined in
\file{dvipsnam.mp} are not defined in \MF{}. With these provisions the
same \mfp{} code can often produce either gray scale \MF{} pictures or
\MP{} color pictures depending only on the \opt{metapost} option.

The commands \cs{shade} and \cs{gfill}\oarg{gray(.75)} (see
subsection~\ref{filling} for their meaning) will produce a similar shade
of gray, but there is a difference. The first simply adds small dots on
top of whatever is already drawn. The second, however, tries to simulate
the \MP{} effect, which is to cover up whatever is previously drawn.
Therefore, it first erases all affected pixels before adding the dots to
simulate gray. In particular, \cs{gfill}\oarg{white} should have the
same effect as \cs{gclear}.


\section{Modifying the figures.}\label{modifier}

Some \mfp{} macros operate by \emph{modifying} a figure macro: if you
want to turn an open arc into a closed figure by adding a straight line,
you can write: \cs{lclosed}\cs{arc}\marg{(0,0),(1,0),45}. These are
always prefixed to some figure drawing command, and apply only to the
next following figure macro provided that only other prefix commands
intervene. This is a rather long section, but even more modification
prefixes are documented in subsection~\ref{transformation}.

The combination of a modifying macro, followed by a figure macro, can
usually be thought of as a new figure macro, to which further prefixes
might be prepended.

More precisely: all prefix macros have an \emph{input} path, an
\emph{output} path, and a \emph{side effect}. The input is the path that
is output by the \emph{following} prefix or figure macro. The output is
either the same as the input or a modification of it. The side effect
might be a drawing or filling of the path or the addition of an
arrowhead.

We list here a classifications of prefix and figure macros that is
useful for understanding the \mfp{} system.

\begin{description}

\item[Figure macros.] These\index{figure macro} take no input path; they
    must come last in a sequence. They output the path they were
    designed to produce. Examples are \cs{circle}, \cs{rect} and
    \cs{polygon}. If they have no prefixes, or are preceded only by
    appending macros (see next), they invoke a default rendering of the
    path (usually a drawing as a solid stroke) as the side effect.

\item[Apending macros] These\index{prefix macro} pass their input
    unchanged as their output. Their side effect is the appending of
    some object such as an arrow head or tail. Currently only the
    various prefix macros whose names begin with \texttt{arrow} are
    appending macros (see subsection~\ref{arrows}). But \cs{reverse},
    which technically modifies a path and has no side effect, is coded
    as an appending macro so that it will work correctly with arrows.
    Think of it as `appending' a new direction.

\item[Rendering macros] These\index{prefix macro} pass their
    input unchanged as their output. They have the side effect of adding
    or subtracting ink from a picture in the shape of the input path.
    Examples are \cs{draw}, \cs{dotted}, \cs{gfill} and \cs{gclip}.

\item[Modifying macros] These\index{prefix macro} output the result of
    applying their intended modification to the input path. Examples are
    macros that close the path if it was open, macros that apply a
    transformation such as a rotation, and macros that return only a
    part of a path. If they have no prefixes, or are preceded only by
    appending macros (see above), they also invoke a default rendering
    of the output path (usually a drawing as a solid stroke of the
    modified path) as the side effect.

\end{description}

\subsection{Closure of paths}\label{closure}

It should be pointed out that the closure macros will leave already
closed paths unchanged, so it is always safe to add one when uncertain.
Moreover, if the path is not closed but the endpoints are identical,
\cs{lclosed} and \cs{bclosed} will close it without adding any path segment.


\begin{cd}\pagelabel{lclosed}
\cs{lclosed}$\ldots$\\
\cs{bclosed}\oarg{\meta{tens}}$\ldots$\\
\cs{sclosed}\oarg{\meta{tens}}$\ldots$%
\index{lclosed@\cs{lclosed}}
\index{bclosed@\cs{bclosed}}%
\index{sclosed@\cs{sclosed}}
\end{cd}

These modifying macros all turn an open path into a closed one. If the
path is already closed, they do nothing.

\cs{lclosed} makes an open path into a closed path by adding a line
segment between the endpoints of the path. In the special case where
the path ends exactly where it begins, all \cs{lclosed} does is change
the type of the path from open to closed.

The \cs{bclosed} macro is similar to \cs{lclosed}, except that it closes
an open path smoothly by drawing a B\'ezier curve. A B\'ezier is \MF{}'s
natural way of connecting points into a curve, and \cs{bclosed} is the
simplest and most efficient closure next to \cs{lclosed}. Moreover it
usually gives a reasonably aesthetic result. Sometimes, however, one
might wish a tighter connection. If that is the case, use the optional
argument with a value of the tension \meta{tens} greater than $1$, the
default. The command \cs{settension} (see subsection~\ref{curves}) can
be used to change the default.

\cs{sclosed} closes the curve by mimicking the definition of the
\cs{curve} command. That command tries to force the curve to pass
through the $n$th point in a direction parallel to the line from point
$(n-1)$ to point $(n+1)$. In order to close a curve in this way, the
direction at the two endpoints often has to be changed, and this changes
the shape of the first and last segments of the curve. Use \cs{bclosed}
if you don't wish this to happen. However, \cs{sclosed}\cs{curve}
produces a result almost identical to \cs{cyclic} given the same points
and tension values. The optional tension argument is as in the
\cs{bclosed} command.

There are two other closure commands but, because they are associated
with particular types of paths (splines), we delay their discussion
until those are discussed (subsection~\ref{splines}).

\begin{cd}\pagelabel{makesector}
\cs{makesector}\cs{arc}[\meta{fmt}]\marg{\meta{spec}}%
\index{makesector@\cs{makesector}}
\end{cd}

The modifying macro \cs{makesector} can be applied to any path, but its
name makes sense (and its action is predictable) only if that path is an
arc. It appends line segments from the center of the arc's circle to the
ends of the arc, producing a closed path. It is useful if one doesn't
know where the center of the arc is (a required parameter of
\cs{sector}). It works by selecting the first point, a middle point, and
the last point of the following path, then calculates the center of the
circle through those three points.


\subsection{Reversal, connection and other path modifications}%
\label{reversal}

\begin{cd}\pagelabel{reverse}
\cs{reverse}$\ldots$%
\index{reverse@\cs{reverse}}
\end{cd}

This modifies the following path by reversing its sense. This will
affect the direction of arrows: bi-directional arrows can be coded with
\cs{arrow}\cs{reverse}\cs{arrow}$\ldots$, where the leftmost \cs{arrow}
prefix applies to the \emph{reversed} path. The order of endpoints for
the following \env{connect} environment will also be affected.


\begin{cd}\pagelabel{connect}
\cs{connect} $\ldots$ \cs{endconnect}%
\index{connect@\cs{connect}}%
\index{endconnect@\cs{endconnect}}
\end{cd}

The macro \cs{connect} produces a connected path by joining all the
paths following it up to the matching \cs{endconnect} command. Line
segments are added from the end of one path to the start of the next.
The whole group acts as one figure macro, permitting any prefix macros
to come before.

In \LaTeX{}, instead of this pair of macros, an environment named
\env{connect} may be used. For example
\begin{verbatim}
\lclosed
\begin{connect}
  \curve{(2,1),(1,2),(0,1)}
  \polyline{(0,0),(2,0)}
\end{connect}
\end{verbatim}
produces a closed figure consisting of one smooth curve and three line
segments: the segment produced by \cs{polyline}, the segment added by
the \env{connect} environment, and the segment added by \cs{lclosed}.


\begin{cd}\pagelabel{partpath}
\cs{partpath}\marg{\meta{frac1},\meta{frac2}}\dots\\
\cs{subpath}\marg{\meta{num1},\meta{num2}}\dots\\
\cs{trimpath}\marg{\meta{dim$_1$},\meta{dim$_2$}}\dots\\
\cs{trimpath}\marg{\meta{dim}}\dots%
\index{partpath@\cs{partpath}}%
\index{subpath@\cs{subpath}}%
\index{trimpath@\cs{trimpath}}%
\end{cd}

These macros modify the following path by producing only a part of it. In
\cs{partpath} the parameters \meta{frac1} and \meta{frac2} should be
numbers between 0 and 1. The path produced travels the same course as
the path that follows, but starts at the point that is the fraction
\meta{frac1} of the original length along it, and ends at the point
\meta{frac2} of its original length. If \meta{frac1} is greater than
\meta{frac2}, the sense of the path is reversed. In \cs{subpath}, the
two numbers should be between 0 and the number of B\'ezier segments in
the path. This is mainly for experienced \MF{}ers and provides an \mfp{}
interface to \MF{}'s `\mfc{subpath}' operation.

The \cs{trimpath} macro takes two dimensions separated by commas and
trims those lengths off the initial and terminal ends of the following path.
Alternatively, it takes one dimension and and trims that length off of
both ends. If any of \meta{dim$_1$}, \meta{dim$_2$} or \meta{dim} is
missing, it is taken to be \dim{0pt}. This works by finding the points of
intersection between the path and circles around the endpoints with the given
dimensions as radii. If the path is shorter than either dimension, it
will not intersect either circle and nothing will be trimmed. Similar
problems can occur, at one end or the other, if the path is shorter than
the sum of the dimensions.

\begin{cd}\pagelabel{parallelpath}
\cs{parallelpath}\marg{\meta{dist}}$\ldots$
\index{parallelpath@\cs{parallelpath}}
\end{cd}

This modifying macro takes the following path and returns a path that
follows beside it, keeping a fixed distance \meta{dist} to the left. If
\meta{dist} is negative, it keeps to the right. Left or right is from
the point of view of a traveller following the given path from start to
finish. The distance is a pure number in \emph{graph} coordinates. Note:
this should be compared to the first optional argument of
\cs{doubledraw} (see subsection~\ref{drawing}), which requires an
absolute dimension like \dim{2pt}, even though it is implemented using
the internal code of \cs{parallelpath}.

The calculation of the parallel path is approximate and rather
inefficient. It is likely to produce inexplicable small loops where it
tries to follow the inside of tight turns (radius less than
\meta{dist}). Actual corners, (which might be thought of as turns of
radius $0$) are usually detected and dealt with in a reasonable manner.
However, if the path is made up of segments of length \meta{dist} or
less, this is unlikely to work correctly at all.

\begin{cd}\pagelabel{arccomplement}
\cs{arccomplement}\dots%
\index{arccomplement@\cs{arccomplement}}
\end{cd}

This macro, to work properly, must be followed by an arc of a circle. It
produces the complementary arc. That is, it produces the circular arc,
which would, if appended to the following arc, complete the circle. The
complementary arc will have the same direction, clockwise or
anticlockwise, as the original. The arc that follows doesn't have to be
produced by \cs{arc}, as in the following example:
\begin{ex}
\cs{draw}\oarg{blue}\cs{arccomplement}\\
\ \cs{draw}\oarg{red}\cs{partpath}\marg{0,.333}\\
\ \cs{circle}\marg{(0,0),1}
\end{ex}
This will draw 1/3 of the circle in red and the rest in blue.

\CMF{} cannot check if a path is really a circular arc. The \MF{} code,
like that of \cs{makesector} (see subsection~\ref{closure}), selects
three key points on the arc, then it produces the rest of the circle
much the same way as the internal code of \cs{arc}\oarg{t} (the three
point option for \cs{arc}). Thus, it will produce \emph{some} arc from
the end of any following path to its beginning (or a straight line if
the three chosen points happen to lie in a straight line). However, the
result needn't bear any significant relation to the original path.


\subsection{Arrows}\label{arrows}

\begin{cd}\pagelabel{arrow}
\cs{arrow}\oarg{l\meta{headlen}}\oarg{r\meta{rotate}}%
    \oarg{b\meta{backset}}\oarg{c\meta{color}}$\ldots$\\
\cs{arrow*}\oarg{l\meta{headlen}}\oarg{r\meta{rotate}}%
    \oarg{b\meta{backset}}\oarg{c\meta{color}}$\ldots$%
\index{arrow@\cs{arrow}}
\end{cd}

This macro adds an arrowhead at the endpoint of the open path (or at the
last key point of the closed path) that follows. The optional parameter
\meta{headlen} determines the length of the arrowhead.  The default is
the value of the \TeX{} dimension \cs{headlen}, initially \dim{3pt}. The
optional parameter \meta{rotate} allows the arrowhead to be rotated
anticlockwise around its point an angle of \meta{rotate} degrees. The
default is 0. The optional parameter \meta{backset} allows the arrowhead
to be `set back' from its original point, thus allowing (for example)
double arrowheads. This parameter is in the form of a \TeX{}
dimension---its default value is \dim{0pt}. If an arrowhead is both
rotated and set back, it is set back in the direction after the
rotation. The optional \meta{color} defaults to \gbc{headcolor},
initially black. The optional parameters may appear in any order, the
indicated key character determining the meaning of a parameter. The key
letter \texttt{l} for `length' can be replaced by \texttt{s} for `size'.

There is also a star-form: If \cs{arrow} is called as \cs{arrow*}, then
any part of the tip of the following curve that lies outside the
arrowhead shape is clipped off. Imagine a rectangle with one side
connecting the ends of the barbs and the opposite  side passing through
the tip. Everything in that rectangle outside the arrowhead is erased,
so be careful using this (also see comments about \MP{}'s method of
`erasing' in the description of \cs{gclear} in
\cs{}subsection~\ref{filling}). One use of this is adding an arrowhead
to a figure rendered with \cs{doubledraw} (see the next section) or with
a rather large pen diameter (see section~\ref{parameters}).

For the star-form to work, the head has to be added after the path is
drawn. What this means in practice is that the \cs{arrow*} command must
come before any drawing command in the list of prefixes. This is because
prefix macros add their elements to the result of everything that
follows. If you \cs{store} a curve in a path variable (see
subsection~\ref{transformation}), and draw the path and the arrowhead in
separate commands, then the arrow command must come \emph{after} the
drawing command.

\begin{cd}\pagelabel{arrowhead}
\cs{arrowhead}\marg{\meta{symbol}}\oarg{l\meta{length}}\oarg{r\meta{rotate}}%
    \oarg{b\meta{backset}}\oarg{c\meta{color}}$\ldots$\\
\cs{arrowmid}\marg{\meta{symbol}}\oarg{l\meta{length}}\oarg{r\meta{rotate}}%
    \oarg{f\meta{fraction}}\oarg{c\meta{color}}$\ldots$\\
\cs{arrowtail}\marg{\meta{symbol}}\oarg{l\meta{length}}\oarg{r\meta{rotate}}%
    \oarg{f\meta{forward}}\oarg{c\meta{color}}$\ldots$%
\index{arrowhead@\cs{arrowhead}}%
\index{arrowmid@\cs{arrowmid}}%
\index{arrowtail@\cs{arrowtail}}
\end{cd}

These macros add some sort of symbol at different locations along a
path. The first adds an arrowhead, but the head can be any appropriately
designed symbol. It has been arranged that any of the symbols usable in
\cs{plotsymbol} (see subsection~\ref{points}) can be used: you can have
\gbc{Diamond}- or \gbc{Asterisk}-tipped arrows. The special symbol
\gbc{Arrowhead} produces the same shape as the head in the \cs{arrow}
command. In total eight special \meta{symbols} have been made available,
intended for use with \cs{arrowhead}, \cs{arrowmid} and \cs{arrowtail}.
Here is a list and description of all these symbols.
\begin{description}
  \item[\gbc{Arrowhead}] The\index{Arrowhead@\gbc{Arrowhead}} shape that
    would be drawn at the end of a path by \cs{arrow}.
  \item[\gbc{Leftharpoon}] The\index{Leftharpoon@\gbc{Leftharpoon}} left
    half of \gbc{Arrowhead}.
  \item[\gbc{Rightharpoon}] The\index{Rightharpoon@\gbc{Rightharpoon}}
    right half of \gbc{Arrowhead}.
  \item[\gbc{Crossbar}] A\index{Crossbar@\gbc{Crossbar}} short
    line crossing the path perpendicularly unless rotated.
  \item[\gbc{Leftbar}] Essentially\index{Leftbar@\gbc{Leftbar}} the left
    half of \gbc{Crossbar}.
  \item[\gbc{Rightbar}] The\index{Rightbar@\gbc{Rightbar}} right half.
  \item[\gbc{Lefthook}] An\index{Lefthook@\gbc{Lefthook}} open
    semicircle with its open face in the direction of the path, added to
   the left side of the path.
  \item[\gbc{Righthook}] Like\index{Righthook@\gbc{Righthook}}
    \gbc{Lefthook} but on the right side.
\end{description}
Here `left' and `right' are from the point of view of an observer facing
in the direction of the path.

If the symbol is a closed path (see subsection~\ref{closure} for the
difference between a closed path and one that merely looks closed), the
head will be filled, otherwise its outline will be drawn. Thus
\cs{arrowhead}\marg{Diamond} draws an outline, and
\cs{arrowhead}\marg{SolidDiamond} draws a filled shape because
\gbc{Diamond} has been left open, while \gbc{SolidDiamond} has been
defined to be closed.

It is possible, to get an outline drawn with the inside erased: just
place the solid version with color \mfc{background} (usually the same as
\mfc{white}) and then the outline version. This can produce a pleasing
result. But recall that the prefix macro nearest the figure macro is
executed first. For example:
%
\begin{verbatim}
\arrowmid{Circle}\arrowmid{SolidCircle}[cwhite]\polyline{(0,0),(1,1)}
\end{verbatim}

The symbol is always rotated so that it points in the direction of the
path (for this purpose, all symbols are initially assumed to point
straight upward) before the \oarg{r\meta{rotate}} parameter is applied.

There is a star-form \cs{arrowhead*} that behaves like \cs{arrow*} (when
possible). The optional arguments are exactly as in \cs{arrow}, with the
same defaults for all of them.

The second command, \cs{arrowmid}, places the symbol somewhere between
the start and the end of the path. In this case the optional parameter
\oarg{f\meta{fraction}} gives the location of the symbol as a fraction
of the length of the path. The default is \oarg{f0.5}, which places it
approximately in the middle. The other optional arguments have the same
meaning as for \cs{arrowhead}. As with \cs{arrowhead}, the symbol is
rotated to `point' in the direction of the path before the
\oarg{r\meta{rotate}} is applied.

The third command \cs{arrowtail} places the symbol at the start of the
path. Otherwise it behaves as the other two commands, except the option
\oarg{f\meta{forward}} is an amount to shift the symbol forward from
that first point.

One might be tempted to use \cs{arrowmid} with the \meta{fraction} equal
to $1$ or $0$ to get arrowheads or tails. This will work sometimes.
However, some shapes have a `tip', that is, a particular point
designated as the tip of the arrowhead. The \cs{arrowhead} and
\cs{arrowtail} commands pay attention to this, while \cs{arrowmid} does
not. Also, \cs{arrowmid} has no star-form.

You can design your own \meta{symbol} for these commands: use \cs{store}
to store a path in a path variable (see
subsection~\ref{transformation}). These commands assume that the length
is $1$, that the symbol `points' up and that the `tip' (the `pointy
end') is at $(0,0)$ (unless the pair variable \meta{symbol}\gbc{.tip} is
defined, in which case that is taken to be the tip). So draw your symbol
pointing up with its tip at $(0,0)$ and its length equal to $1$ (graph
unit). For example the following produces a solid head with a common
shape:
\begin{verbatim}
\store{myAH}\polygon{(-.5,-1)(0,0),(0.5,-1),(0,-.7)}
\arrowhead{myAH}\arc{(-10,0),(10,0),90}
\end{verbatim}
If you replace the \cs{polygon} above with \cs{polyline}:
\begin{verbatim}
\store{myAH}\polyline{(-.5,-1)(0,0),(0.5,-1),(0,-.7),(-.5,-1)}
\end{verbatim}
the path will not be closed and so the arrowhead will not be filled in.

To make the star-form work with such self-defined symbols, one must also
define a closed path \gbc{myAH.clear} that gives the region to be
erased. In the above example:
\begin{verbatim}
\store{myAH.clear}\polygon{(-.5,-1),(-.5,0),(.5,0),(.5,-1),(0,-.7)}
\end{verbatim}


\section{Rendering figures.}\label{rendering}

When \mfp{} is loaded, the initial way in which figures are drawn is
with a solid outline. That is, \cs{polyline}\marg{(1,0),(1,1),(0,0)} will
draw two solid lines connecting the points. It is possible to establish
a different default (see \cs{setrender} in subsection~\ref{default}),
however that default is used only when no explicit rendering prefix is
present. That is, when the macros in this section are used, any
previously established default is overridden.

\begin{cd}\pagelabel{norender}
\cs{norender}$\ldots$%
\index{norender@\cs{norender}}%
\end{cd}

This causes the following path not to be rendered at all. This can be
used to override \mfp{}'s automatic rendering rules. See
section~\ref{transformation}, page~\pageref{norenderexample} for an
example where one might need to do this.

\subsection{Drawing}\label{drawing}

\begin{cd}\pagelabel{draw}
\cs{draw}\oarg{\meta{color}}$\ldots$%
\index{draw@\cs{draw}}
\end{cd}

Draws the subsequent path using a solid outline. For an example: to both
draw a curve and hatch its interior, \cs{draw}\cs{hatch} must be used.
The default for \meta{color} is \gbc{drawcolor}.

To save repetition, the color used for the following commands is also
\gbc{drawcolor}: \cs{dashed}, \cs{dotted}, \cs{doubledraw}, \cs{plot},
\cs{plotnodes}, and \cs{gendashed},

\begin{cd}\pagelabel{doubledraw}
\cs{doubledraw}\oarg{\meta{sep}}\oarg{\meta{color}}$\ldots$
\index{doubledraw@\cs{doubledraw}}
\end{cd}

This rendering macro draws the path with a double line. The default
separation (distance between centers of the two penstrokes) is twice the
pen diameter. This normally leaves one line thickness of white space
between. You can change this with the \oarg{\meta{sep}} argument. In
order to make the space between the lines transparent, this command is
implemented by calculating two curves that parallel the given curve and
drawing those. For technical reasons, that calculation is rather lengthy
so this is somewhat inefficient and users of slow machines might want to
avoid it. See also comments at \cs{parallelpath} in
subsection~\ref{reversal}.

\begin{cd}\pagelabel{dashed}
\cs{dashed}\oarg{\meta{length},\meta{space}}$\ldots$%
\index{dashed@\cs{dashed}}
\end{cd}

This rendering macro draws dashed segments along the path specified.
The default length of the dashes is the value of the \TeX{} dimension
\cs{dashlen}, initially \dim{4pt}. The default space between the dashes
is the value of the \TeX{} dimension \cs{dashspace}, initially
\dim{4pt}. The dashes and the spaces between may be increased or
decreased by as much as $1/n$ of their value, where $n$ is the number of
spaces appearing in the curve, in order to have the proper dashes at the
ends. The dashes at the ends are half of \cs{dashlen} long.

\begin{cd}\pagelabel{dotted}
\cs{dotted}\oarg{\meta{size},\meta{space}}$\ldots$%
\index{dotted@\cs{dotted}}
\end{cd}

This rendering macro draws dots along the specified path. The default
size of the dots is the value of the \TeX{} dimension \cs{dotsize},
initially \dim{0.5pt}. The default space between the dots is the value
of the \TeX{} dimension \cs{dotspace}, initially \dim{3pt}. The size of
the spaces may be adjusted as in \cs{dashed}.

\begin{cd}\pagelabel{plot}
\cs{plot}\oarg{\meta{size},\meta{space}}\marg{\meta{symbol}}$\ldots$%
\index{plot@\cs{plot}}
\end{cd}

Similar to \cs{dotted}, this rendering macro draws copies of
\meta{symbol} along the path. Possible symbols are those listed under
\cs{plotsymbol} in subsection~\ref{points}. The default \meta{size} is
\cs{pointsize} (initially \dim{2pt}) and the default \meta{space} is
\cs{symbolspace} (initially \dim{5pt}).

\begin{cd}\pagelabel{plotnodes}
\cs{plotnodes}\oarg{\meta{size}}\marg{\meta{symbol}}$\ldots$%
\index{plotnodes@\cs{plotnodes}}
\end{cd}

This rendering macro places a symbol at each \emph{node} of the path
that follows. Possible symbols are those listed under \cs{plotsymbol} in
subsection~\ref{points}. A node is one of the points through which \MF{}
draws its curve.  If one of the macros \cs{polyline}\marg{$\ldots$} or
\cs{curve}\marg{$\ldots$} follows, each of the points listed is a node.
In the \cs{datafile} command (subsection~\ref{external}), each of the
data points in the file is a node. In the function macros
(subsection~\ref{plotting}) the points corresponding to \meta{min},
\meta{max} and each step in between are nodes. The optional \meta{size}
defaults to \cs{pointsize}. If the command \cs{clearsymbols} has been
issued then the interiors of the open symbols are erased. The effect of
something like the following is rather nice:
\begin{verbatim}
\clearsymbols
\plotnodes{Circle}\draw\polyline{...}
\end{verbatim}
This will first draw the polyline with solid lines, and then the points
listed will be plotted as open circles with the portion of the lines
inside the circles erased. One sees a series of open circles connected
one to the next by line segments


\begin{cd}\pagelabel{dashpattern}
\cs{dashpattern}\marg{\meta{name}}%
    \marg{\meta{len1},\meta{len2},$\ldots$,\meta{len2k}}%
\index{dashpattern@\cs{dashpattern}}
\end{cd}

For more general dash patterns than \cs{dashed} and \cs{dotted} provide,
\mfp{} offers a generalized dashing command. Before using it, one must
first establish a named dashing pattern with the above command. The
\meta{name} can be any sequence of letters and underscores. Try to make
it distinctive to avoid undoing some internal variable. \meta{len1}
through \meta{len2k} are an even number of lengths. The odd ones
determine the lengths of dashes, the even ones the lengths of spaces. A
dash of length \texttt{0pt} means a dot. An alternating dot-dash pattern
can be specified with
\begin{verbatim}
\dashpattern{dotdash}{0pt,4pt,3pt,4pt}
\end{verbatim}
\emph{Note}: Since pens have some thickness, dashes look a little
longer, and spaces a little shorter, than the numbers suggest. If one
wants dashes and spaces with the same length, one needs to take the size
desired and increase the spaces by the thickness of the drawing pen
(normally) \dim{0.5pt}) and decrease the dashes by the same amount.%
    \footnote{Experienced \MP{} users could also set the \mfc{linecap}
    variable to \mfc{butt}.}

If \cs{dashpattern} is used with an odd number of entries, a space of
length \dim{0pt} is appended. This makes the last dash in one copy of
the pattern abut the first dash in the next copy.

\begin{cd}\pagelabel{gendashed}
\cs{gendashed}\marg{\meta{name}}$\ldots$%
\index{gendashed@\cs{gendashed}}
\end{cd}

Once a dashing pattern name has been defined, it can be used in this
figure macro to draw the curve that follows it. Using a name not
previously defined will cause the curve to be drawn with a solid line,
and generate a \MF{} warning, but \TeX{} will not complain. If all the
dimensions in a dash pattern are 0, \cs{gendashed} responds by drawing a
solid curve. The same is true if the pattern has only one entry.

\begin{cd}\pagelabel{zigzag}
\cs{zigzag}\marg{\meta{start},\meta{end},\meta{wl},\meta{amp}}\dots\\
\cs{sinewave}\oarg{\meta{tens}}%
    \marg{\meta{start},\meta{end},\meta{wl},\meta{amp}}\dots%
\index{zigzag@\cs{zigzag}}%
\index{sinewave@\cs{sinewave}}
\end{cd}

These figure macros both draw a solid line that crosses from one
side of the path to the other. The \cs{zigzag} makes a jagged result
while the \cs{sinewave} makes a smooth one. The optional argument of
\cs{sinewave} is a `tension' and controls how smooth the result is. The
default tension is $1$. Higher values make a less smooth path, and
values of 10 or so produce a result almost indistinguishable from
\cs{zigzag}. Tension is required to be greater than $3/4$.

The mandatory arguments consists of four dimensions separated by a comma.
The rendering produced by these macros actually follow the path a little
way at the start and end of the path. This is controlled by the
dimensions \meta{start} and \meta{end}.

The third dimension, \meta{wl}, is the distance from one `peak' to the
next (the `wavelength'). The second, \meta{amp}, is the maximum distance
to either side of the true path (the `amplitude'). Reasonable values of
\meta{wl} and \meta{amp} are \dim{8pt} and \dim{2pt}, respectively.
These proportions (4 to 1) causes the zigzag and the sinewave to cross
the path at an angle of about 45 degrees, a rather pleasant result.
Those sizes are close to optimal: too much smaller and the rendering
just looks like a fuzzy line, too much larger, and bends in the path
will distort the zigzagging.

The zigzags zig to the left first if \meta{amp} is positive, to the
right if it is negative. For closed curves, the beginning and end are
constructed to meet smoothly. It is always arranged that there are an
equal number of left zigs and right zags, so the \meta{wl} is only
approximate.


\begin{cd}\pagelabel{corkscrew}
\cs{corkscrew}\oarg{\meta{tens}}%
    \marg{\meta{start},\meta{end},\meta{wl},\meta{amp}}\dots\\%
\cs{coil}\oarg{\meta{tens}}%
    \marg{\meta{start},\meta{end},\meta{wl},\meta{amp}}\dots%
\index{corkscrew@\cs{corkscrew}}%
\index{coil@\cs{coil}}
\end{cd}

This rendering macro draws a coil or corkscrew that coils around a
given path, something like this: \includegraphics{coil.mps} (the red
dots show the actual path). The \meta{tens} is a tension option that
controls how `loopy' the result will be (the higher the number the more
jagged). The mandatory argument contains four explicit dimensions. The
first two, \meta{start} and \meta{end} are as in \cs{zigzag}. The
\meta{wl} is the distance from one loop to the next, and \meta{amp} is
the distance from the true path to the tops (or bottoms) of the
loops. If \meta{amp} is positive, the tip of the loop is to the left
of the path, if negative it is to the right. The example at the start of
this paragraph was drawn using the following code:
\begin{verbatim}
\mfpic{0}{33}{0}{6.4}
\dotsize=1pt
\drawcolor{red}
  \dotted\polyline{(0,3.2),(33,3.2)}
\drawcolor{black}
  \coil[1.5]{3pt,3pt,4.8pt,3.2pt}\polyline{(0,3.2),(33,3.2)}
\endmfpic
\end{verbatim}


\subsection{Shading, filling, erasing, clipping, hatching}\label{filling}

For the purposes of this section, a distinction must be made in the
figure macros between `open' and `closed' paths. A path that merely
returns to its starting point is \emph{not} automatically closed; such a
path might be open and may need to be explicitly closed, for example by
\cs{lclosed}. The (already) closed paths are those that have
`\texttt{closed}' or `\texttt{cyclic}' in their name plus:
\begin{display}
  \cs{belowfcn}, \cs{border}, \cs{btwnfcn}, \cs{btwnplrfcn},
    \cs{chartbar} (and its aliases),\\
  \cs{circle}, \cs{ellipse}, \cs{levelcurve}, \cs{makesector},
    \cs{piewedge}, \cs{plrregion},\\
  \cs{polygon}, \cs{pshcircle}. \cs{rect}, \cs{regpolygon},
    \cs{sector}, \cs{tlabelcircle},\\
  \cs{tlabelellipse}, \cs{tlabeloval}, and \cs{tlabelrect}.
\end{display}

The macros of this section can all be used to fill (or unfill) the
interior of closed paths, even if the paths cross themselves. Filling an
open curve is technically an error, but the \MF{} code responds by
drawing the path and not doing any filling. Note that these macros
override the default rendering, so if you want some sort of fill pattern
\emph{and} an outline drawn, you need an explicit prefix for both.

\begin{cd}\pagelabel{gfill}
\cs{gfill}\oarg{\meta{color}}$\ldots$%
\index{gfill@\cs{gfill}}
\end{cd}

This rendering macro fills in the subsequent closed path. Under \MP{} it
fills with \meta{color}, which defaults to
\index{fillcolor@\gbc{fillcolor}}\gbc{fillcolor}. Under \MF{} it
approximates the color with a shade of gray, clears the interior, and
then fills with a pattern of black and white pixels simulating gray.

\begin{cd}\pagelabel{gclear}
\cs{gclear}$\ldots$%
\index{gclear@\cs{gclear}}
\end{cd}

This rendering macro erases everything \emph{inside} the subsequent
closed path (except text labels under some circumstances, see
section~\ref{mplabels} and \ref{overlaylabels}). Under \MP{} it actually
fills with the predefined color named \mfc{background}. Since
\mfc{background} is normally \mfc{white}, and so are most actual
backgrounds, this is usually indistinguishable from clearing. However,
if an \env{mfpic} environment utilzes \emph{background text} (see
subsection~\ref{text}), part of the background text may appear to be
`erased'. Unfortunately, there is little that can be done about this.

\begin{cd}\pagelabel{gclip}
\cs{gclip}$\ldots$%
\index{gclip@\cs{gclip}}
\end{cd}

This rendering macro erases everything \emph{outside} the subsequent
closed path from the picture (except text labels under some
circumstances, see section~\ref{mplabels} and \ref{overlaylabels}). Note
that this is a true erasing, even in \MP{}.

\begin{cd}\pagelabel{shade}
\cs{shade}\oarg{\meta{shadesp}}$\ldots$%
\index{shade@\cs{shade}}
\end{cd}

This rendering macro shades the interior of the subsequent closed path
with dots. The diameter of the dots is the \MF{} variable \mfc{shadewd},
set by the macro \cs{shadewd}\marg{\meta{size}}. Normally this is
\dim{0.5bp}. The optional argument specifies the spacing between (the
centers of) the dots, which defaults to the \TeX{} dimension
\cs{shadespace}, initially \dim{1pt}. If \cs{shadespace} is less than
\mfc{shadewd}, the closed path is filled with black, as if with
\cs{gfill}. Under \MP{} this macro actually fills the path's interior
with a shade of gray. The shade to use is computed based on
\cs{shadespace} and \mfc{shadewd}. The default values of these
parameters correspond to a gray level of about 78\% of white.%
    \footnote{If \cs{shadewd} is $w$ and \cs{shadespace} is $s$, then
    the level of gray is $1 - (.88w/s)^2$, where $0$ denotes black and
    $1$ white.} %
The \MF{} version attempts to optimize the dots to the pixel grid
corresponding to the printers resolution (to avoid generating dither
lines). Because this involves rounding, it will happen that values of
\cs{shadespace} that are relatively close and at the same time close to
\mfc{shadewd} produce exactly the same shade. Most of the time, however,
values of \cs{shadespace} that differ by at least 20\% will produce
different patterns. The actual behavior for particular values of the
parameters and particular printer resolutions cannot be predicted, and
we even make no guarantee it will not change from one version of \mfp{}
to another.

\begin{cd}\pagelabel{polkadot}
\cs{polkadot}\oarg{\meta{space}}$\ldots$%
\index{polkadot@\cs{polkadot}}
\end{cd}

This rendering macro fills the interior of a closed path with large
dots. This is almost what \cs{shade} does, but there are several
differences. \cs{shade} is intended solely to simulate a gray fill in
\MF{} where the only color is black. So it is optimized for small dots
aligned to the pixel grid (in \MF{}). In \MP{} \cs{shade} only fills with
gray and is intended merely for compatibility. The macro \cs{polkadot}
is intended for large dots in any color, and so it optimizes spacing (a
nice hexagonal array) and makes no attempt to align at the pixel level.
The \meta{space} defaults to the \TeX{} dimension \cs{polkadotspace},
initially \dim{10pt}. The diameter of the dots is the value of the \MF{}
variable \mfc{polkadotwd}, which can be set with
\cs{polkadotwd}\marg{\meta{size}}, and is initially \dim{5bp}. The dots
are colored with \index{fillcolor@\gbc{fillcolor}}\gbc{fillcolor}. In
\MF{}, nonblack values of \gbc{fillcolor} will produce shaded dots.

\begin{cd}\pagelabel{thatch}
\cs{thatch}\oarg{\meta{hatchsp},\meta{angle}}\oarg{\meta{color}}$\ldots$%
\index{thatch@\cs{thatch}}
\end{cd}

This rendering macro fills a closed path with equally spaced parallel
lines at the specified angle. The thickness of the lines is set by the
macro \cs{hatchwd}. In the optional argument, \meta{hatchsp} specifies
the space between lines, which defaults to the \TeX{} dimension
\cs{hatchspace}, initially \dim{3pt}. The \meta{angle} defaults to 0.
The \meta{color} defaults to \gbc{hatchcolor}. If \cs{hatchspace} is
less than the line thickness, the closed path is filled with
\meta{color}, as if with \cs{gfill}. If the first optional argument
appears, both parts must be present, separated by a comma. For the color
argument to be present, the other optional argument must also be
present. However, if one wishes only to override the default color one
can use an empty first optional argument (completely empty, no spaces or
comma).

\begin{cd}\pagelabel{hatch}
\cs{lhatch}\oarg{\meta{hatchsp}}\oarg{\meta{color}}$\ldots$\\
\cs{rhatch}\oarg{\meta{hatchsp}}\oarg{\meta{color}}$\ldots$\\
\cs{hatch}\oarg{\meta{hatchsp}}\oarg{\meta{color}}$\ldots$\\
\cs{xhatch}\oarg{\meta{hatchsp}}\oarg{\meta{color}}$\ldots$%
\index{lhatch@\cs{lhatch}}%
\index{rhatch@\cs{rhatch}}%
\index{hatch@\cs{hatch}}%
\index{xhatch@\cs{xhatch}}%
\end{cd}

These rendering macros are just \cs{thatch} with predefined values of
the angle. \cs{lhatch} fills the region with left slanted lines (from
upper left to lower right). It is exactly the same as
\begin{display}
\cs{thatch}\oarg{\meta{hatchsp},-45}\oarg{\meta{color}}$\ldots$
\end{display}

\cs{rhatch} draws right slanted lines (lower left to upper right). It is
exactly the same as
\begin{display}
\cs{thatch}\oarg{\meta{hatchsp},45}\oarg{\meta{color}}$\ldots$
\end{display}

\cs{hatch} (\cs{xhatch} is a synonym) draws lines in a cross-hatched
pattern. It is exactly the same as \cs{rhatch} followed by \cs{lhatch}
using the same \meta{hatchsp} and \meta{color}.

Hatching should normally be used very sparingly, or never if
alternatives are available (color, shading). However, hatching or
polkadotting on top of another filling macro is almost the only way to
fill in two regions that \emph{automatically} shows the overlap area.
Hatching is at least less garish than polkadots.


\subsection{Changing the default rendering}\label{default}

\emph{Rendering} is the process of converting a geometric description
into a drawing.  In \MF{}, this means producing a bitmap (\MF{} stores
these in \mfc{picture} variables), either by stroking (drawing) a path
using a particular pen), or by filling a closed path. In \MP{} it means
producing a \PS{} description of penstrokes and fills (with possible
clipping).

\begin{cd}\pagelabel{setrender}
\cs{setrender}\marg{\meta{\TeX{} commands}}%
\index{setrender@\cs{setrender}}
\end{cd}

Initially, \mfp{} uses the \cs{draw} command (stroking) as the default
operation when a figure is to be rendered.  However, this can be changed
to any combination of \mfp{} rendering commands or indeed any \TeX{}
commands, by using the \cs{setrender} command.  This redefinition is
local inside an \env{mfpic} environment, so it can be enclosed in braces
to restrict its range. Outside an \env{mfpic} environment it is a global
redefinition.

For example, after \cs{setrender}\marg{\cs{dashed}\cs{shade}} the
command \cs{circle}\marg{(0,0),1} produces a shaded circle with a dashed
outline. Any explicit rendering prefix overrides this default.

\subsection{Examples}\label{examples}

It may be instructive, for the purpose of understanding the syntax of
\emph{shape-modifier and rendering prefixes}, to consider two examples:
\begin{ex}
    \cs{draw}\cs{gfill}\oarg{red}\cs{lclosed}\cs{polyline}\marg{$\ldots$}
\end{ex}
which fills inside a polygon and draws its outline; and
\begin{ex}
    \cs{gfill}[red]\cs{lclosed}\cs{draw}\cs{polyline}\marg{$\ldots$}
\end{ex}
which draws all of the outline \emph{except} the line segment supplied
by \cs{lclosed}, then fills the interior. Thus, in the first case the
path is first defined (by \cs{polyline}), then closed, then the resulting closed
path is filled, and finally drawn. In the second case the order is:
defined, drawn, closed, filled. In particular, what is drawn in the
second case is the path not yet closed. It should also be pointed out
that in the last case, the fill is placed last and will cover half
the thickness of the previously drawn outline.


\section{Functions and Plotting.}\label{functions}

In the following macros, expressions like $f(\mathtt{x})$ or $g(\mathtt{t})$
stand for any legal \MF{} expression, in which the only unknown
variables are those indicated (\texttt{x} in the first case, and \texttt{t} in
the second).

\subsection{Defining functions}\label{defining}

\begin{cd}\pagelabel{fdef}
\cs{fdef}\marg{\meta{fcn}}\marg{\meta{param1},\meta{param2},$\ldots$}%
    \marg{\meta{mf-expr}}%
\index{fdef@\cs{fdef}}
\end{cd}

Defines a \MF{} function \meta{fcn} of the parameters \meta{param1},
\meta{param2}, $\ldots$, by the \MF{} expression \meta{mf-expr} in which
the only free parameters are those named.  The return type of the
function is the same as the type of the expression. What is allowed for
the function name \meta{fcn} is more restrictive than \MF{}'s rule for
variable names. Roughly speaking, it should consist of letters and
underscore characters only. (In particular, for those that know what
this means, the name should have no suffixes.) Try to make the name
distinctive to avoid redefining internal \MF{} commands.

The expression \meta{mf-expr} is passed directly into the corresponding
\MF{} macro and interpreted there, so \MF{}'s rules for algebraic
expressions apply. If \cs{fdef} occurs inside an \env{mfpic}
environment, it is local to that environment, otherwise it is available
to all subsequent \env{mfpic} environments.

As an example, after \cs{fdef}\marg{myfcn}\marg{s,t}\marg{s*t-t}, any
place below where a \MF{} expression is required, you can use
\mfc{myfcn(2,3)} to mean \mfc{2*3-3} and \mfc{myfcn(x,x)} to mean
\mfc{x*x-x}.

Operations available include \mfc{+}, \mfc{-}, \mfc{*}, \mfc{/}, and
\mfc{**} (\mfc{x**y}$=x^y$), with `\mfc{(}' and `\mfc{)}' for grouping.
Functions already available include the standard \MF{} functions
\mfc{round}, \mfc{floor}, \mfc{ceiling}, \mfc{abs}, \mfc{sqrt},
\mfc{sind}, \mfc{cosd}, \mfc{mlog}, and \mfc{mexp}. Note that in \MF{}
the operations \mfc{*} and \mfc{**} have the same level of precedence,
so \mfc{x*y**z} means $(xy)^z$. Use parentheses liberally!

(\textit{Notes:} The \MF{} trigonometric functions \mfc{sind} and
\mfc{cosd} take arguments in degrees; \mfc{mlog(x)}$=256\ln x$, and
\mfc{mexp} is its inverse.) You can also define the function \meta{fcn}
by cases, using the \MF{} conditional expression
\begin{ex}
  \mfc{if~\meta{boolean}:~\meta{expr}~elseif
  \meta{boolean}:~$\ldots$~else:~\meta{expr}~fi}.
\end{ex}
Relations available for the \meta{boolean} part of the expression
include \mfc{=}, \mfc{<}, \mfc{>}, \mfc{<=}, \mfc{<>} and \mfc{>=}.

Complicated functions can be defined by a compound expression, which is
a series of \MF{} statements, followed by an expression, all enclosed
between \mfc{begingroup} and \mfc{endgroup}. The \cs{fdef} command
automatically supplies these grouping commands around the definition so
if the entire \meta{mf-expr} is one such compound expression the user
need not type them. \CMF{} functions can call \MF{} functions, even
recursively.

Many common functions have been predefined in \file{grafbase}, which is
a package of \MF{} macros that implement \prog{mfpic}'s drawing. These
include the rest of the trig functions \mfc{tand}, \mfc{cotd}, \mfc{secd},
\mfc{cscd}, which take angles in degrees, plus variants \mfc{sin},
\mfc{cos}, \mfc{tan}, \mfc{cot}, \mfc{sec}, and \mfc{csc}, which take
angles in radians. Some inverse trig functions are also available, the
following produce angles in degrees: \mfc{asin}, \mfc{acos}, and
\mfc{atan}, and the following in radians: \mfc{invsin}, \mfc{invcos},
\mfc{invtan}. The exponential and hyperbolic functions: \mfc{exp},
\mfc{sinh}, \mfc{cosh}, \mfc{tanh}, \mfc{coth}, \mfc{sech}, and
\mfc{csch}; and some of their inverses: \mfc{ln} (or \mfc{log}),
\mfc{asinh}, \mfc{acosh}, and \mfc{atanh} are also defined.

There are also two conversion functions: \gbc{radians(t)} produces the
number of radians in \mfc{t} degrees and \gbc{degrees(t)} produces the
number of degrees in \mfc{t} radians. In these expressions the special
variable \gbc{pi} produces $\pi$, accurate to roughly 5 decimals.
(\CMF{} and \MP{} provide accuracy only to $\pm2^{-17} = \pm
.76\times10^{-5}$.)

The integer functions \gbc{gcd(m,n)} and \gbc{lcm(m,n)} produce the
greatest common divisor and least common multiple of two integers
\gbc{m} and \gbc{n}.


\subsection{Plotting functions}\label{plotting}

The plotting macros take two or more arguments. They have an optional
first argument, \meta{spec}, which determines whether a function is
drawn smooth (as a \MF{} B\'ezier curve), or polygonal (as line
segments)---if \meta{spec} is \texttt{p}, the function will be
polygonal. Otherwise the \meta{spec} should be \texttt{s}, followed by
an optional positive number no smaller than 0.75. In this case the
function will be smooth with a tension equal to the number. See the
\cs{curve} command (subsection~\ref{curves}) for an explanation of
tension. The default \meta{spec} depends on the purpose of the macro.

One compulsory argument contains three values \meta{min}, \meta{max} and
\meta{step} separated by commas.  The independent variable of a function
starts at the value \meta{min} and steps by \meta{step} until reaching
\meta{max}. If \meta{max}${}-{}$\meta{min} is not a whole number of
steps, then round$((\meta{max}-\meta{min})/\meta{step})$ equal steps are
used. One may have to experiment with the size of \meta{step}, since
\MF{} merely connects the points corresponding to these steps with what
\emph{it} considers to be a smooth curve. Smaller \meta{step} gives
better accuracy, but too small may cause the curve to exceed \MF{}'s
capacity or slow down its processing. Increasing the tension may help
keep the curve in line, but at the expense of reduced smoothness.

There are one or more subsequent arguments, each of which is a \MF{}
function or expression as described above. All the macros are figure
macros, defining a path to which prefixes may be applied.

\begin{cd}\pagelabel{function}
\cs{function}\oarg{\meta{spec}}\marg{\meta{$x_{\mathrm{min}}$},%
    \meta{$x_{\mathrm{max}}$},\meta{$\Delta x$}}%
    \marg{$f(\mathtt{x})$}%
\index{function@\cs{function}}
\end{cd}

This figure macro produces the graph of $y = f(x)$, where $f$ is a \MF{} numeric
function or expression of one numeric argument, which must be denoted by
a literal \texttt{x}. The default \meta{spec} is \texttt{s}. For example
\begin{verbatim}
\function{0,pi,pi/10}{sin x}
\end{verbatim}
draws the graph of $\sin x$ between 0 and $\pi$.

\begin{cd}\pagelabel{parafcn}
\cs{parafcn}\oarg{\meta{spec}}\marg{\meta{$t_{\mathrm{min}}$},%
    \meta{$t_{\mathrm{max}}$},\meta{$\Delta t$}}%
    \marg{($x(\mathtt{t}), y(\mathtt{t})$)}\\
\cs{parafcn}\oarg{\meta{spec}}\marg{\meta{$t_{\mathrm{min}}$},%
    \meta{$t_{\mathrm{max}}$},\meta{$\Delta t$}}%
    \marg{\meta{pair-fcn}}%
\index{parafcn@\cs{parafcn}}
\end{cd}

This figure macro produces the parametric path determined by the last
argument. This can be a pair of expressions $x(\mathtt{t})$ and
$y(\mathtt{t})$ enclosed in parentheses and separated by a comma, with
the literal variable \texttt{t}. Alternatively, the last argument can be
a \MF{} function or expression in \texttt{t} that returns a pair.%
    \footnote{There are very few of these. \CMF{} provides \mfc{dir t},
    which is essentially \mfc{(cosd t, sind t)}. \Mfp{} adds \gbc{cis t}
    which is \gbc{(cos t, sin t)}.}
The default \meta{spec} is \texttt{s}. For example
\begin{verbatim}
\parafcn{0,1,.1}{(2t, t+t*t)}
\end{verbatim}
plots a smooth parabola from $(0,0)$ to $(2,2)$.

\begin{cd}\pagelabel{plrfcn}
\cs{plrfcn}\oarg{\meta{spec}}\marg{\meta{$\theta_{\mathrm{min}}$},%
    \meta{$\theta_{\mathrm{max}}$},\meta{$\Delta\theta$}}%
    \marg{$f(\mathtt{t})$}%
\index{plrfcn@\cs{plrfcn}}
\end{cd}

This figure macro produces the graph of the polar coordinate equation
$r=f(\theta)$, where $f$ is a \MF{} numeric function or expression of
one numeric argument, and $\theta$ varies from
\meta{$\theta_{\mathrm{min}}$} to \meta{$\theta_{\mathrm{max}}$} in
steps of \meta{$\Delta\theta$}. Each $\theta$ value is interpreted as an
angle measured in \emph{degrees}. In the expression $f(\mathtt{t})$, the
unknown \texttt{t} stands for $\theta$. The default \meta{spec} is
\texttt{s}. For example
\begin{verbatim}
\plrfcn{0,90,5}{sind (2t)}
\end{verbatim}
draws one loop of a 4-petal rosette. Note that this function demands the
variable \mfc{t} be in degrees. The range and step size must be in
degrees and the function must operate on the numeric variable \gbc{t} in
degrees. If one needs to measure angles in radians, use the conversion
functions \gbc{degrees()} and \gbc{radians()}, as follows:
\begin{verbatim}
\plrfcn{0,degrees(pi/2),degrees(pi/36)}{sin (radians(2t))}
\end{verbatim}


\begin{cd}\pagelabel{btwnfcn}
\cs{btwnfcn}\oarg{\meta{spec}}\marg{\meta{$x_{\mathrm{min}}$},%
    \meta{$x_{\mathrm{max}}$},\meta{$\Delta x$}}%
    \marg{$f(\mathtt{x})$}\marg{$g(\mathtt{x})$}\\
\cs{btwnplrfcn}\oarg{\meta{spec}}\marg{\meta{$\theta_{\mathrm{min}}$},%
    \meta{$\theta_{\mathrm{max}}$},\meta{$\Delta \theta$}}%
    \marg{$f(\mathtt{t})$}\marg{$g(\mathtt{t})$}%
\index{btwnfcn@\cs{btwnfcn}}%
\index{btwnplrfcn@\cs{btwnplrfcn}}
\end{cd}

These are figure macros. The first one produces a closed path
surrounding the region between the graphs of the two functions. The
second one does the same for two polar functions. That is (in both
cases), the path follows the first function (in order or increasing $x$
or $\theta$), thence along the straight line to the \emph{end} of the
second one, thence backwards along the second function (decreasing $x$
or $\theta$) and finally along the straight line to the start. The last
two mandatory arguments, the functions, are specified exactly as in
\cs{function} and \cs{plrfcn}, being numeric functions of one numeric
argument \texttt{x} or \texttt{t}. Unlike the previous function macros,
the default \meta{spec} is \texttt{p}---these macros are intended to be
used for shading between drawn functions, a task for which smoothness is
usually unnecessary. For example, the first line below
\begin{verbatim}
\shade\btwnfcn{0,1,.1}{0}{x - x**2}
\btwnplrfcn[s]{-30,30,5}{1}{2*cosd 2t}
\end{verbatim}
shades the area between the $x$-axis and the given parabola. The second
draws the boundary of the region between the circle $r = 1$ and one loop
of the rosette $r = 2\cos 2\theta$.

Note: the effect of \cs{btwnfcn} could also be accomplished with
\begin{ex}
  \cs{lclosed}\cs{connect}\\
    \cs{function}\marg{\meta{$x_{\mathrm{min}}$},%
    \meta{$x_{\mathrm{max}}$},\meta{$\Delta x$}}\marg{$f(\mathtt{x})$}\\
    \cs{reverse}\cs{function}\marg{\meta{$x_{\mathrm{min}}$},%
    \meta{$x_{\mathrm{max}}$},\meta{$\Delta x$}}\marg{$g(\mathtt{x})$}\\
  \cs{endconnect}
\end{ex}
\cs{lclosed} was described in subsection~\ref{closure} and the
\cs{connect}\dots\cs{endconnect} pair was described in
subsection~\ref{reversal}.

\begin{cd}\pagelabel{belowfcn}
\cs{belowfcn}\oarg{\meta{spec}}%
    \marg{\meta{$x_{\mathrm{min}}$},\meta{$x_{\mathrm{max}}$},%
    \meta{$\Delta x$}}\marg{$f(\mathtt{x})$}\\
\cs{plrregion}\oarg{\meta{spec}}%
    \marg{\meta{$\theta_{\mathrm{min}}$},\meta{$\theta_{\mathrm{max}}$},%
    \meta{$\Delta\theta$}}\marg{$f(\mathtt{t})$}%
\index{belowfcn@\cs{belowfcn}}%
\index{plrregion@\cs{plrregion}}
\end{cd}

These figure macros produce identical results to \cs{btwnfcn} and
\cs{btwnplrfcn} when the first function is just $0$. They are, however,
much more efficient.  The first of these, \cs{belowfcn}, produces the
path surrounding the region bounded by the $x$-axis, the graph of
$y=f(x)$ and the two vertical lines $x=x_{\mathrm{min}}$ and $x =
x_{\mathrm{max}}$. (The region is not actually \emph{below} $y = f(x)$
unless $f(x) \ge 0$ throughout the interval.) The second produces the
path surrounding the region bounded by the polar function $r =
f(\theta)$ and the two rays $\theta=\theta_{\mathrm{min}}$ and
$\theta=\theta_{\mathrm{max}}$.

The arguments of these command are the same as the nonclosed versions,
\cs{function} and \cs{plrfcn}, except the default for the optional
agument is \texttt{[p]}. Again, this is because it is mainly for
shading. However, drawing the boundary is often needed:
\begin{verbatim}
\shade\plrregion{0,90,5}{sind (2t)}
\plrregion[s]{0,90,5}{sind (2t)}
\end{verbatim}
shades one loop of the 4-petal rosette, and then draws it.

\begin{cd}\pagelabel{levelcurve}
\cs{levelcurve}\oarg{\meta{spec}}\marg{\meta{seed},\meta{step}}
    \marg{\meta{inequality}}%
\index{levelcurve@\cs{levelcurve}}
\end{cd}

This figure macro produces a level curve of some function $F(x,y)$.
There are three requirements on the parameters for this to work
correctly. First, in order to obtain the curve satisfying $F(x,y) = C$,
the \marg{\meta{inequality}} must be either \verb${F(x,y) > C}$ or
\verb${F(x,y) < C}$.%
   \footnote{A non-strict inequality such as \mfc{>=} can be
    used, but the result will not be significantly different.}
Second, the level curve must surround the point given by the \meta{seed}
paramter, and third, the inequality must be true at this seed point.

The command works by searching rightward from \meta{seed} until it
encounters the first point on the level curve. It then tries to find a
nearby point on the level curve and joins it to the first one, and
continues similarly until it finds it has returned near the starting
point. The meaning of ``nearby point on the level curve'' is the
intersection of the level curve with a circle of radius \meta{step}
centered at the previously found point. If the region defined by the
inequality extends beyond the bounds of the picture (as set by the
\cs{mfpic} command), the region is truncated and the resulting curve
will follow along the picture's border.

Since the algorithm only approximates the level set, a tolerance (how
close the points are to actually being \emph{on} the level curve) is
chosen which gives two decimal places more accuracy than \meta{step}.
The value of \meta{step} is interpreted in \emph{graph} units and
so should be a pure number. The \oarg{\meta{spec}} is either \oarg{p},
in which case the calculated points are joined with straight lines, or
\oarg{s\meta{tension}} as in \cs{function}. The default is
\oarg{s}: a smooth curve with the current default tension.

In general, choosing a \meta{step} that corresponds to a few millimeters
works reasonably well. For example, if the graph unit is 1cm (for
example, \cs{mfpicunit=1cm} and no scaling is used), then
\meta{step}${}= 0.5$ might be a reasonable first choice. If the level
set is reasonably smooth and \oarg{s} is used, then the result will
match the actual curve to within .005cm, which is approximately .14pt,
which is less than half the thickness of the standard pen used to draw
it.

Be warned that there is a limit: there should not be more than 2000
steps in the completed curve. In a figure which is 10-by-10 graph units,
a level curve without too much oscillation would probably be less than
80 units in length and a step size of .04 would probably produce under
2000 steps. This should be accurate enough for most purposes. If you
\emph{really} need more, the value of the \MF{} variable
\verb$max_points$ must be changed. This can be done with
\cs{setmfvariable} (see section~\ref{variables}).

As a special case, if \meta{step} is 0, the maximum of width and height
of the figure (as given by the arguments to the \env{mfpic} environment)
is divided by 100. For example, in a 5-by-10 graph, giving a step size
of $0$ will actually select \meta{step}${}= 10/100 = 0.1$.

The algorithm used will produce imprecise results if there are two
points on the curve closer than \meta{step} in straight-line distance,
but much further apart when measured along the curve.

\subsection{Plotting external data files}\label{external}

\begin{cd}\pagelabel{datafile}
\cs{datafile}\oarg{\meta{spec}}\marg{\meta{file}}\\
\cs{smoothdata}\oarg{\meta{tension}}\\
\cs{unsmoothdata}%
\index{datafile@\cs{datafile}}%
\index{smoothdata@\cs{smoothdata}}%
\index{unsmoothdata@\cs{unsmoothdata}}
\end{cd}

The figure macro \cs{datafile} produces a curve connecting the points
listed in the file \meta{file}. (The context makes it clear whether this
meaning of \cs{datafile} or that of subsection~\ref{list} is meant.) The
\meta{spec} may be \texttt{p} to produce a polygonal path, or \texttt{s}
followed by a tension value (as in \cs{curve}) to produce a smooth path.
If no \meta{spec} is given, the default is initially \texttt{p}, but
\cs{smoothdata} may be used to change this. Thus, after the command
\cs{smoothdata}\oarg{\meta{tension}} the default \oarg{\meta{spec}} is
changed to \oarg{s\meta{tension}}. If the tension parameter is not
supplied it defaults to \mfc{1.0} (or the value set by the
\cs{settension} command if one has been used).

The command \cs{unsmoothdata} restores the default \oarg{\meta{spec}} to
\oarg{p}.

By default, each non-blank line in the file is assumed to contain at
least two numbers, separated by whitespace (blanks or tabs). The first
two numbers on each line are assumed to represent the $x$- and
$y$-coordinates of a point. Initial blank lines in the file are ignored,
as are comments. The comment character in the data file is assumed to be
\texttt{\%}, but it can be reset using \cs{mfpdatacomment} (below). Any
blank line other than at the start of the file causes the curve to
terminate. The \cs{datafile} command may be preceded by any of the
prefix commands, so that, for example, a closed curve could be formed
with \cs{lclosed}\cs{datafile}\marg{data.dat}.

The \index{datafile@\cs{datafile}}\cs{datafile} command has another use,
independent of the above description. We saw in subsection~\ref{list}
that any \mfp{} command (other than one that prints text labels) that
takes as its last argument a list of points (or numerical values)
separated by commas, can have that list replaced with a reference to an
external data file. For example, if a file \file{ptlist.dat} contains
two or more numerical values per line separated by whitespace, then one
can draw a dot at each of the points corresponding to the first pair of
numbers on each line with the following.
\begin{verbatim}
\point\datafile{ptlist.dat}
\end{verbatim}
In fact there is no essential difference between `\cs{datafile}\oarg{p}'
and `\cs{polyline}\cs{datafile}', and no difference between
`\cs{datafile}\oarg{s}' and `\cs{curve}\cs{datafile}'.
Here is the full list (omitting aliases) of \mfp{} macros that allow
this usage of \cs{datafile}\index{datafile@\cs{datafile}}:

\begin{itemize}
\raggedright
\item Numeric data:
    \cs{barchart}, \cs{dashpattern}, \cs{numericarray},
    \cs{piechart}, and all the axis marks commands.
\item Point or vector data:
    \cs{cbeziers}, \cs{closedcbeziers}, \cs{closedcomputedspline},
    \cs{closedcspline}, \cs{closedmfbezier}, \cs{closedqbeziers},
    \cs{closedqspline}, \cs{computedspline}, \cs{convexcurve},
    \cs{convexcyclic}, \cs{cspline}, \cs{curve}, \cs{cyclic},
    \cs{fcncurve}, \cs{fcnspline}, \cs{mfbezier},
    \cs{periodicfcnspline}, \cs{plotsymbol}, \cs{point}, \cs{polygon},
    \cs{polyline}, \cs{putmfpimage}, \cs{qbeziers}, \cs{qspline},
    \cs{turtle}, and \cs{pairarray}.
\end{itemize}
In addition \cs{setarray} and \cs{globalsetarray} (with the numeric or
pair data type) allow this usage.

\begin{cd}\pagelabel{mfpdatacomment}
\cs{mfpdatacomment}\cs{}\meta{char}%
\index{mfpdatacomment@\cs{mfpdatacomment}}
\end{cd}

Changes \meta{char} to a comment character and changes the usual \TeX{}
comment character \texttt{\%} to an ordinary character \emph{while reading a
datafile for drawing}.

\begin{cd}\pagelabel{using}
\cs{using}\marg{\meta{in-pattern}}\marg{\meta{out-pattern}}%
\index{using@\cs{using}}
\end{cd}

Used to change the assumptions about the format of the data file. For
example, if there are four numbers on each line separated by commas, to
plot the third against the second (in that order) you can say
\cs{using}\marg{\#1,\#2,\#3,\#4}\marg{(\#3,\#2)}. This means the
following: Everything on a line up to the first comma is assigned to
parameter \texttt{\#1}, everything from there up to the second comma is
assigned to parameter \texttt{\#2}, etc. Everything from the third comma
to the end of line is assigned to \texttt{\#4}. When the line is
processed by \TeX{} a \MF{} pair is produced representing a point on the
curve. \CMF{} pair expressions can be used in the output portion of
\cs{using}. For example \cs{using}\marg{\#1,\#2,\#3}\marg{(\#2,\#1)/10}
or even \cs{using}\marg{\#1 \#2 \#3}\marg{polar(\#1,\#2)} if the data
are polar coordinates. The default assumptions of the \cs{datafile}
command (numbers separated by spaces, with the first two determining the
$(x,y)$ pair) corresponds to the following setting.
\begin{verbatim}
\using{#1 #2 #3}{(#1,#2)}
\end{verbatim}
The \cs{using} command cannot normally be used in the replacement text
of another command. Or rather, it can be so used, but then each
\texttt{\#} has to be doubled. If a \cs{using} declaration occurs in an
\env{mfpic} environment it is local to that environment. Otherwise it
affects all subsequent ones.

\begin{cd}\pagelabel{sequence}
\cs{sequence}%
\index{sequence@\cs{sequence}}
\end{cd}

As a special case, you can plot any number against its sequence
position, with something like
\cs{using}\marg{\#1 \#2}\marg{(\cs{sequence},\#1)}. Here, the macro
\cs{sequence} will take on the values \texttt{1}, \texttt{2}, etc. as
lines are read from the file.

\begin{cd}\pagelabel{usingpairdefault}
\cs{usingpairdefault}\\
\cs{usingnumericdefault}%
\index{usingpairdefault@\cs{usingpairdefault}}%
\index{usingnumericdefault@\cs{usingnumericdefault}}
\end{cd}

The command \cs{usingpairdefault} restores the above described default
for pair data. The command \cs{usingnumericdefault} is the equivalent of
\cs{using}\marg{\#1 \#2}\marg{\#1}, a useful default for numeric data.

Note that the default value of \cs{using} appears to reference three
arguments. If there are only two numbers on a line separated by
whitespace, this will still work because of \TeX{}'s argument matching
rules. \TeX{}'s file reading mechanism normally converts the EOL to a
space, but there are exceptions so \mfp{} internally adds a space at
the end of each line read in to be on the safe side. Then the default
definition of \cs{using} reads everything up to the first space as
\texttt{\#1} (whitespace is normally compressed to a single space by
\TeX{}'s reading mechanism), then everything to the second space (the one
added at the end of the line, perhaps) is \texttt{\#2}, then everything
to the EOL is \texttt{\#3}. This might assign an empty argument to
\texttt{\#3}, but it is discarded anyway.

If the numerical data contain percentages with explicit \texttt{\%} signs,
then choose another comment character with \cs{mfpdatacomment}. This
will change \texttt{\%} to an ordinary character \emph{in the data file}.
However, in your \cs{using} command it would still be read as a comment.
The following allows one to overcome this.

\begin{cd}\pagelabel{makepercentother}
\cs{makepercentother}\\
\cs{makepercentcomment}%
\index{makepercentother@\cs{makepercentother}}%
\index{makepercentcomment@\cs{makepercentcomment}}
\end{cd}

Here is an example or their use:
\begin{verbatim}
\makepercentother
\using{#1% #2 #3}{(#1/100,#2)}
\makepercentcomment
\end{verbatim}

Here is an analysis of the meaning of this example: everything in a
line, up to the first percent followed by a space is assigned to
parameter \texttt{\#1}, everything from there to the next space is
assigned to \texttt{\#2} and the rest of the line (which may be empty)
is \texttt{\#3}. On the output side in the above example, the percentage
is divided by 100 to convert it to a fraction, and plotted against the
second parameter. Note: normal comments should not be used between
\cs{makepercentother} and \cs{makepercentcomment}, for obvious reasons.
Moreover, the above construction will fail inside the argument of
another command.

\begin{cd}\pagelabel{plotdata}
\cs{plotdata}\oarg{\meta{spec}}\marg{\meta{file}}%
\index{plotdata@\cs{plotdata}}
\end{cd}

This plots several curves from a single file. The \meta{spec} and the
command \cs{smoothdata} have the same effect on each curve as in the
\cs{datafile} command. The data for each curve is a succession of
nonblank lines separated from the data for the next curve by a single
blank line. A \emph{pair} of successive blank lines is treated as the
end of the data. No prefix macros are permitted in front of
\cs{plotdata}.

Each successive curve in the data file is drawn differently. By default,
the first is drawn as a solid line the next dashed, the third dotted,
etc., through a total of six different line types. A \cs{gendashed}
command is used with predefined dash patterns named \mfc{dashtype0}
through \mfc{dashtype5}. This behavior can be changed with:

\begin{cd}\pagelabel{coloredlines}
\cs{coloredlines}\\
\cs{pointedlines}\\
\cs{datapointsonly}\\
\cs{dashedlines}%
\index{coloredlines@\cs{coloredlines}}%
\index{pointedlines@\cs{pointedlines}}%
\index{datapointsonly@\cs{datapointsonly}}%
\index{dashedlines@\cs{dashedlines}}
\end{cd}

The command \cs{coloredlines} causes \cs{plotdata} to use the rendering
command \cs{draw} with a color option that cycles through eight
different colors starting with black (hey! black is a color too). The
command \cs{pointedlines} causes \cs{plotdata} to use the rendering
command \cs{plot}, cycling through nine symbols. The command
\cs{datapointsonly} causes \cs{plotdata} to use the rendering command
\cs{plotnodes}, cycling through the same nine symbols. The data points
become the nodes of the paths created and so only the data points are
plotted. The command \cs{dashedlines} restores the default. See
appendix~\ref{styles} for the details on the actual dash patterns,
colors and symbols used.

The command \cs{coloredlines} will produce a warning under the
\opt{metafont} option and substitute \cs{dashedlines}. Under the
\opt{metapost} option, this is the sole exception to the general rule
that all curves are drawn in \gbc{drawcolor} by default: the
\cs{plotdata} command after \cs{coloredlines} has been issued.

If, for some reason, you do not like the default starting line style
(say you want to start with a color other than black), you can use one
of the following commands.

\begin{cd}\pagelabel{mfplinetype}
\cs{mfplinetype}\marg{\meta{num}}, or\\
\cs{mfplinestyle}\marg{\meta{num}}%
\index{mfplinetype@\cs{mfplinetype}}%
\index{mfplinestyle@\cs{mfplinestyle}}
\end{cd}

Here \meta{num} is a non-negative number, less than the number of
different drawing types available. The four previous commands reset the
number to 0, so if you use one of them, issue \cs{mfplinetype}
\emph{after} it. The different line styles are numbered starting from
$0$. If two or more \cs{plotdata} commands are used in the same
\env{mfpic} environment, the numbering in each continues where the one
before left off (unless you issue one of the commands above in between).
\cs{mfplinestyle} means the same as \cs{mfplinetype}, and is included
for compatibility. See appendix~\ref{styles} to find out what dash pattern,
color or symbol corresponds to each number by default. The commands
below can be used to change the default dashes,  colors, or symbols.

\begin{cd}\pagelabel{reconfigureplot}
\cs{reconfigureplot}\marg{dashes}\marg{\meta{pat$_1$},\dots,\meta{pat$_n$}}\\
\cs{reconfigureplot}\marg{colors}\marg{\meta{clr$_1$},\dots,\meta{clr$_n$}}\\
\cs{reconfigureplot}\marg{symbols}\marg{\meta{symb$_1$},\dots,\meta{symb$_n$}}%
\index{reconfigureplot@\cs{reconfigureplot}}
\end{cd}

The first argument of \cs{reconfigureplot} is the rendering method to be
changed: \texttt{dashes}, \texttt{colors} or \texttt{symbols}. The
second argument is a list of dash patterns, colors, or symbols. The dash
patterns should be names previously defined through the use of
\cs{dashpattern}. The colors can be any names already known to \MP{}, or
defined through \cs{mfpdefinecolor}. The symbols can be any of those
listed with the \cs{plotsymbol} command (subsection~\ref{points}),
or any known \MF{} path variable. The colors can also be \MP{} color
constants or expressions, and the symbols can be expressions of type
path. In recent \MP{} these `colors' can be \kw{numeric} (selecting
gray), \kw{rgbcolor} or \kw{cmykcolor}. Within a \env{mfpic}
environment, the changes made are local to that environment. Outside,
they affect all subsequent environments.

Using \cs{reconfigureplot}\marg{colors} under the \opt{metafont} option
will have no effect, but may produce an error from \MF{} unless the
colors used conform to the guidelines in subsection~\ref{MFcolor}.
This also holds for \cs{defaultplot}\marg{colors} (below).

\begin{cd}\pagelabel{defaultplot}
\cs{defaultplot}\marg{dashes}\\
\cs{defaultplot}\marg{colors}\\
\cs{defaultplot}\marg{symbols}
\index{defaultplot@\cs{defaultplot}}
\end{cd}

The command \cs{defaultplot} restores the built-in defaults for the
indicated method of rendering in \cs{plotdata}.

The commands \cs{using}, \cs{mfpdatacomment} and \cs{sequence} have the
same meaning here (for \cs{plotdata}) as they do for \cs{datafile}
(above). The sequence numbering for \cs{sequence} starts over with each
new curve.

\section{Labels and Captions.}\label{labels}

\subsection{Setting text}\label{text}

If option \opt{metafont} is in effect macros \cs{tlabel}, \cs{tlabels},
\cs{axislabels} and \cs{tcaption} do not affect the \MF{} file
(\file{\meta{file}.mf}) at all, but are added to the picture by \TeX{}. If
\opt{metapost} is in effect but \opt{mplabels} is not, they do not
affect the \MP{} file. In these cases, if these macros are the only
changes or additions to your document, there is no need to repeat the
processing with \MF{} or \MP{} nor the reprocessing with \TeX{} in order
to complete your \TeX{} document.

\begin{cd}\pagelabel{tlabel}
\cs{tlabel}\oarg{\meta{just}}\parg{\meta{x},\meta{y}}\marg{\meta{labeltext}}\\
\cs{tlabel}\oarg{\meta{just}}\marg{\meta{pair-list}}\marg{\meta{label text}}\\
\cs{tlabels}\marg{\meta{params$_1$} \meta{params$_2$} $\ldots$}%
\index{tlabel@\cs{tlabel}}%
\index{tlabels@\cs{tlabels}}
\end{cd}

These place \TeX{} text or math on the graph. The special form
\cs{tlabels} (note the plural) essentially just applies \cs{tlabel} to
each set of parameters listed in its argument. That is, each
\meta{params$_k$} is a valid set of parameters for a \cs{tlabel}
command. These can be separated by spaces, newlines, or nothing at all.
They should \emph{not} be separated by blank lines.

The last required parameter is ordinary \TeX{} text. The pair
\parg{\meta{x},\meta{y}} gives the coordinates of a point in the graph
where the text will be placed. It may optionally be enclosed in braces,
\verb${$ and \verb$}$. If braces are used, any number of coordinate
pairs may be listed, separated by commas. This is what is meant by
\meta{pair-list} in the above syntax. If \opt{mplabels} is in effect,
the \meta{pair-list} can be any list of expressions recognized as a pair
by \MP{}.

The optional parameter \oarg{\meta{just}} specifies the
\emph{justification}, the relative placement of the label with respect
to the point with coordinates \parg{\meta{x},\meta{y}}. It is a
two-character sequence in which the first character is one of \texttt{t}
(top), \texttt{c} (center), \texttt{b} (bottom), or \texttt{B}
(Baseline), to specify vertical placement, and the second character is
one of \texttt{l} (left), \texttt{c} (center), or \texttt{r} (right), to
specify horizontal placement. These letters specify what part of the
\emph{text} is to be placed at the given point, so \texttt{r} puts the
right end of the text there---which means the text will be left of the
point. The default justification is \oarg{Bl}: the left end of the
baseline of the text is placed at the coordinates.

When \opt{mplabels} is in effect, the two characters may optionally be
followed by a number, specifying an angle in degrees to rotate the text
about the point \parg{\meta{x},\meta{y}}. If the angle is supplied
without \opt{mplabels} it is ignored after a warning. If the angle is
absent, there is no rotation. Note that the rotation takes place after
the placement and uses the given point as the center of rotation. For
example, \oarg{cr} will place the text left of the point, while
\oarg{cr180} will rotate it around to the right side of the point (and
upsidedown, of course).

There should be no spaces before, between, or after the first two
characters. However the number, if present, is only required to be a
valid \MP{} numerical expression containing no bracket characters; as
such, it may contain some spaces (e.g., around operations as in
\texttt{45 + 30}).

A multiline \cs{tlabel} may be specified by explicit line breaks, which
are indicated by the \bbsl{} command or the \cs{cr} command. This is a
very rudimentary feature. By default it left justifies the lines and
causes \cs{tlabel} to redefine \bbsl. One can center a line by putting
\cs{hfil} as the first thing in the line, and right justify by putting
\cs{hfill} there (these are \TeX{} primitives). Redefining \bbsl{} can
interfere with \LaTeX{}'s definition. For better control in \LaTeX{} use
\cs{shortstack} inside the label (or a \env{tabular} environment or some
other environment which always initializes \bbsl{} with its own
definition).

If the label goes beyond the bounds of the graph in any direction, the
space reserved for the graph is expanded to make room for it. (Note:
this behavior is very much different from that of the \LaTeX{}
\env{picture} environment.)

If the \opt{mplabels} option is in effect, \cs{tlabel} will write a
\mfc{btex $\ldots$ etex} group to the output file, allowing \MP{} to
arrange for typesetting the label. Normally, the label becomes part of
the picture, rather than being laid on top of it, and can be covered up
by any filling macros that follow, or clipped off by \cs{gclip}.
However, under the \opt{overlaylabels} option (or after the command
\cs{overlaylabels}), labels are saved and added to the picture at the
very end. This may prevent some special effects, but it makes the
behavior of labels much more consistent through all the 12 permissable
settings of the options \opt{metapost}, \opt{mplabels}, \opt{clip}, and
\opt{truebbox}.

There is another command, \cs{startbacktext}, which also save the labels
and adds them later, but \emph{under} the rest of the picture as
background text. Thus, they will not be clipped, but may be covered up.
Since erasing regions with \cs{gclear} actually covers up those regions
with white, labels saved as background text may appear to have portions
erased.

\begin{cd}\pagelabel{everytlabel}
\cs{everytlabel}\marg{\meta{\TeX{}-code}}%
\index{everytlabel@\cs{everytlabel}}
\end{cd}

One problem with multiline \cs{tlabel}s is that each line of their
contents constitutes a separate group. This makes it difficult to change
the \cs{baselineskip} (for example) inside a label. The command
\cs{everytlabel} saves it's contents in a token register and the code is
issued in each \cs{tlabel}, as the last thing before the actual line(s)
of text. Any switch you want to apply to every line can be supplied. For
example
\begin{verbatim}
\everytlabel{\bf\baselineskip 10pt}
\end{verbatim}
will make every line of every \cs{tlabel}'s text come out bold with 10
point baselines. The effect of \cs{everytlabel} is local to the
\env{mfpic} environment, if it is issued inside one. Note that each line
of a tlabel is wrapped in a box, but the commands of \cs{everytlabel}
are outside all of them, so no actual text should be produced by the
contents of \cs{everytlabel}.

Using \cs{tlabel} without an optional argument is equivalent to
specifying \oarg{Bl}. Use the following command to change this
behavior.

\begin{cd}\pagelabel{tlabeljustify}
\cs{tlabeljustify}\marg{\meta{just}}%
\index{tlabeljustify@\cs{tlabeljustify}}
\end{cd}

After this command the placement of all subsequent labels without
optional argument will be as specified in this command. For example,
\cs{tlabeljustify}\marg{cr45} would cause all subsequent \cs{tlabel}
commands lacking an optional argument to be placed as if the argument
\oarg{cr45} were used in each. If \opt{mplabels} is not in effect at the
time of this command, the rotation part will be saved in case that
option is turned on later, but a warning message will be issued. If
\opt{mplabels} is not turned on later, that rotation will be ignored by
\cs{tlabel}.

\begin{cd}\pagelabel{tlabeloffset}
\cs{tlabeloffset}\marg{\meta{hlen}}\marg{\meta{vlen}}\\
\cs{tlpointsep}\marg{\meta{len}}\\
\cs{tlpathsep}\marg{\meta{len}}\\
\cs{tlabelsep}\marg{\meta{len}}%
\index{tlabeloffset@\cs{tlabeloffset}}%
\index{tlabelsep@\cs{tlabelsep}}%
\index{tlpathsep@\cs{tlpathsep}}%
\index{tlpointsep@\cs{tlpointsep}}
\end{cd}

The first command causes all subsequent \cs{tlabel} commands to shift
the label right by \meta{hlen} and up by \meta{vlen} (negative lengths
cause it to be shifted left and down, respectively).

The \cs{tlpointsep} command causes labels to be shifted by the given
amount in a direction that depends on the optional positioning
parameter. For example, if the first letter is \texttt{t} the label is
shifted down by the amount \meta{len} and if the second letter is
\texttt{l} it is also shifted right. In all cases it is shifted
\emph{away} from the point of placement (unless the dimension is
negative). If \texttt{c} or \texttt{B} is the first parameter, no
vertical shift takes place, and if \texttt{c} is the second, there is no
horizontal shift. This is intended to be used in cases where something
has been drawn at that particular point, in order to separate the text
from the drawing.

Prior to version 0.8, this separation also defined the separation
between the label and those curves designed to frame the label
such as \cs{tlabelrect} (subsection~\ref{surrounding}). Now the two
separations are independent and \cs{tlpathsep} is used to set the
separation between the label and such paths.

For backward compatability, the command \cs{tlabelsep} is still
available and sets both separations to the same value.

\begin{cd}\pagelabel{axislabels}
\cs{axislabels}\marg{\meta{axis}}\oarg{\meta{just}}%
    \marg{\marg{\meta{text$_1$}}\meta{$n_1$},%
    \marg{\meta{text$_2$}}\meta{$n_2$},$\ldots$}%
\index{axislabels@\cs{axislabels}}
\end{cd}

This command places the given \TeX{} text (\meta{text$_k$}) at the given
positions (\meta{$n_k$}) on the given axis, \meta{axis}, which must be a
single letter and one of \texttt{l}, \texttt{b}, \texttt{r}, \texttt{t},
\texttt{x}, or \texttt{y}. The text is placed as in \cs{tlabels}
(including the taking into account of \cs{tlpointsep} and
\cs{tlableoffset}), except that the default justification depends on the
axis (the settings of \cs{tlabeljustify} are ignored). In the case of
the border axes, the default is to place the label outside the axis and
centered. So, for example, for the bottom axis it is \oarg{tc}. The
defaults for the $x$- and $y$-axis are below and left, respectively. The
optional \meta{just} can be used to change this. For example, to place
the labels \emph{inside} the left border axis, use \oarg{cl}. If
\opt{mplabels} is in effect, rotations can be included in the
justification parameter. For example, to place the text strings
`first', `second' and `third' just below the positions 1, 2 and 3
on the $x$-axis, rotated so they read upwards at a 90 degree angle, one
can use \cs{axislabels}\marg{x}\oarg{cr90}\marg{\marg{first}1,
\marg{second}2, \marg{third}3}.

\begin{cd}\pagelabel{plottext}
\cs{plottext}\oarg{\meta{just}}\marg{\meta{text}}\marg{($x_0$,$y_0$),
 ($x_1$,$y_1$), $\ldots$}%
\index{plottext@\cs{plottext}}
\end{cd}

Similar in effect to \cs{point} and \cs{plotsymbol}, \cs{plottext}
places a copy of \meta{text} at each of the listed points. Since \mfp{}
version 0.9, when \cs{tlabel} was enhanced to allow lists of points, it
is implemented by an equivalent \cs{tlabel} command and is only kept for
backward compatibility. It differs from \cs{tlabel} when the optional
argument is absent: the default justification is \oarg{cc} regardless of
the setting of \cs{tlabeljustify}.

\begin{cd}\pagelabel{mfpverbtex}
\cs{mfpverbtex}\marg{\meta{\TeX{}-cmds}}%
\index{mfpverbtex@\cs{mfpverbtex}}
\end{cd}

This writes a \mfc{verbatimtex} block to the \file{.mp} file. It makes
sense only if the \opt{mplabels} option is used and so only for \MP{}.
The \meta{\TeX{}-cmds} in the argument are written to the \file{.mp}
file, preceded by the \MP{} command \mfc{verbatimtex} and followed by
\mfc{etex}. Line breaks within the \meta{\TeX{}-cmd} are preserved.
There is also a linebreak between the end of \meta{\TeX-cmds} and the
\mfc{etex}. The \cs{mfpverbtex} command must come before any \cs{tlabel}
that is to be affected by it. Any settings common to all \env{mfpic}
environments should be in a \cs{mfpverbtex} command preceding all such
environments.

It may be issued at any point after \mfp{} is loaded, and any number of
times. If it is issued after \cs{opengraphsfile}, its contents are
immediately written to the \file{.mp} file. If it is issued before
\cs{opengraphsfile}, its contents are saved and written when the file is
opened (successive uses being cummulative). In this case its contents
will precede the boilerplate \TeX{} code that \mfp{} writes. If you wish
to redefine some of that code, you need to use \cs{mfpverbtex} after
\cs{opengraphsfile}.

Because of the way \MP{} handles \mfc{verbatimtex} material, the effects
cannot be constrained by any grouping unless one places \TeX{} grouping
commands within \meta{\TeX{}-cmds}. However, \mfp{} itself places
grouping commands into the output file at the beginning and end of each
picture, so definitions written by a \cs{mfpverbtex} are local to any
picture in which it occurs.  Prior to version 0.8, \mfp{} did not write
comments that occured within the \meta{\TeX{}-cmds}. Now they will be
preserved, and can be used to place the `\verb$%&latex$' line that some
\TeX{} distributions permit as a signal that latex should be run to
produce the labels.

This command attempts a near-verbatim writing of the \meta{\TeX{}-cmds}
and, as with all verbatim-like commands, it should not be used in the
argument of another command.

\begin{cd}\pagelabel{backtext}
\cs{startbacktext} \dots \cs{stopbacktext}%
\index{startbacktext@\cs{startbacktext}}%
\index{stopbacktext@\cs{stopbacktext}}
\end{cd}

When \TeX{} adds labels (\cs{nomplabels}) they have to be positioned
either on top of a complete figure, or placed under a complete figure.
The most reasonable choice (and happily the easiest to implement) is to
put them on top. When \MP{} is placing labels (option \opt{mplabel}) the
same can be forced with the option \opt{overlaylabels}, but otherwise
they are placed as they occur, with later drawing commands perhaps
putting their results on top of the labels or clipping parts of them off.

Sometimes it is useful to place some label as a background (not on top),
and yet not have it clipped by later commands. The effect of the command
\cs{startbacktext} is that \cs{tlabel} commands are saved in a special
place until the command \cs{stopbacktext}. Then, at \cs{endmfpic} the
rest of the figure is simply place on top of them. Since labels in \MP{}
files can only consist of characters from some font, if one wants to
include a graphic in the background (for example, via
\cs{includegraphics}), one needs to switch off \opt{mplabels}:
\begin{verbatim}
\nomplabels
\startbacktext
  \tlabel[cc](0,0){\includegraphics{mygraph}}
\stopbacktext
\usemplabels
\end{verbatim}
As with other labels, it is permitted to switch \opt{mplabels} off and
on while creating background text. If there are both kinds of labels
within the background text area the ones handled by \TeX{} will be
further back than the ones handled by \MP{}. Within a given type,
earlier ones are further back than later ones.

\Mfp{} normally uses a naming scheme like \cs{cmd} \dots \cs{endcmd} and
tries to arrange that \env{cmd} can be used as an environment. As
currently written, the extra grouping added by \cs{begin}\marg{cmd} and
\cs{end}\marg{cmd} would break the code that implements background text,
so we have named these in a different way to avoid suggesting this
possiblity. There should be at most one of these pairs in any
\env{mfpic} environment. It can occur anywhere in the environment, but
the two commands must not be inside any grouping.

Under the \opt{metapost} option, the \cs{gclear} command doesn't really
clear a space, but rather paints the space over with white. Any
background text will not be visible through such `holes'. This is a
limitation of \MP{}.

\begin{cd}\pagelabel{tcaption}
\cs{tcaption}\oarg{\meta{maxwd},\meta{linewd}}\marg{\meta{caption text}}%
\index{tcaption@\cs{tcaption}}
\end{cd}

Places a \TeX{} caption at the bottom of the graph. (Not to be confused
with \LaTeX{}'s similar \cs{caption} command.) The macro will
automatically break lines which are too much wider than the graph---if
the \cs{tcaption} line exceeds \meta{maxwd} times the width of the
graph, then lines will be broken to form lines at most \meta{linewd}
times the width of the graph. The default settings for \meta{maxwd} and
\meta{linewd} are 1.2 and 1.0, respectively. \cs{tcaption} may typeset
its argument twice (as might \LaTeX{}'s \cs{caption}), the first time as
a single line to test its width, then again if that was too wide.
Therefore, the user is advised \emph{not} to include any global
assignments in the caption text.

If the \cs{tcaption} and graph have different widths, the two are
centered relative to each other. If the \cs{tcaption} takes multiple
lines, then the default is to set lines both left- and right-justified
(except for the last line) with no indentation on the first line. If the
option \opt{raggedcaptions} is in effect, the lines are only
left-justified and ragged on the right. Finally, if the option
\opt{centeredcaptions} is in effect, each line of the caption will be
centered (under \opt{raggedcaptions} they will be ragged on both sides).

In a \cs{tcaption}, explicit line breaks may be specified by using the
\bbsl{} command. The separation between the bottom of the picture and
the caption can be changed by increasing or decreasing the skip
\cs{mfpiccaptionskip}\index{mfpiccaptionskip@\cs{mfpiccaptionskip}}
(a `rubber' length in Lamport's terminology).

Many \mfp{} users find the \cs{tcaption} command too limiting (one
cannot, for example, place the caption to the side of the figure). It is
common to use some other method (such as \LaTeX{}'s \cs{caption} command
in a \env{figure} environment). The dimensions \cs{mfpicheight} and
\cs{mfpicwidth} (see section~\ref{parameters}) might be a convenience
for plain \TeX{} users who want to roll their own caption macros.


\subsection{Curves surrounding text}\label{surrounding}

\begin{cd}\pagelabel{tlabelrect}
\cs{tlabelrect}\oarg{\meta{rad}}\oarg{\meta{just}}%
    \meta{pair}\marg{\meta{text}}\\
\cs{tlabelrect*...}%
\index{tlabelrect@\cs{tlabelrect}}
\end{cd}

This figure macro and the following two methods of surounding a bit of
text with a curve share some common characteristics which will be
described here. The commands all take an optional argument that can
modify the shape of the curve. After that come arguments exactly as for
the \cs{tlabel} command except that only a single point is permitted,
not a list. (So \meta{pair} is either of the form
\parg{\meta{x},\meta{y}} or the same enclosed in braces, or for
\opt{mplabels} a pair expression in braces.) After processing the
surrounding curve, a \cs{tlabel} is applied to those arguments unless a
\texttt{*} is present. In order for the second optional argument (the
optional justification argument for the \cs{tlabel} command) to be
recognized as the second, the first optional argument must also be
present. An empty first optional argument is permitted, causing the
default value to be used. The default for the justification argument is
\texttt{cc}, for compatibility with past \mfp{} versions, in which these
commands all centered the figure around the point and no justification
parameter existed. This default can be changed with the
\cs{tlpathjustify} command below.

The plain rectangle version produces a frame separated from the text on
all sides by the amount defined with \cs{tlpathsep}. All other versions
produce the smallest described curve that contains this rectangle.

These commands may be preceded by prefix macros (see the sections
\ref{modifier}~and \ref{rendering}, above). They all have a
`star-form' which produces the curve but omits placing the text.
All have the effect of rendering the path \emph{before} placing any
text. For example, \cs{gclear}\cs{tlabelrect}\dots\ will clear the
rectangle and then place the following text in the cleared space.

The optional argument of \cs{tlabelrect}, \meta{rad}, is a dimension,
defaulting to \dim{0pt}, that produces rounded corners made from
quarter-circles of the given radius. If the corners are rounded, the
sides are expanded slightly so the resulting shape still encompasses the
rectangle mentioned above. There is one special case for the optional
argument \meta{rad}: if the keyword `\texttt{roundends}' is used instead
of a dimension, the radius will be chosen to make the nearest quarter
circles just meet, so the narrow side of the rectangle is a half circle.

\begin{cd}\pagelabel{tlabeloval}
\cs{tlabeloval}\oarg{\meta{mult}}\oarg{\meta{just}}%
    \meta{pair}\marg{\meta{text}}\\
\cs{tlabeloval*...}%
\index{tlabeloval@\cs{tlabeloval}}
\end{cd}

This figure macro is similar to \cs{tlabelrect}, except it produces an
ellipse. The ellipse is calculated to have the same ratio of width to
height as the rectangle mentioned above. The optional \meta{mult} is a
multiplier that increases or decreases this ratio. Values of \meta{mult}
larger than 1 increase the width and decrease the height.

\begin{cd}\pagelabel{tlabelellipse}
\cs{tlabelellipse}\oarg{\meta{ratio}}\oarg{\meta{just}}%
    \meta{pair}\marg{\meta{text}}\\
\cs{tlabelellipse*...}\\
\cs{tlabelcircle}\oarg{\meta{just}}\meta{pair}\marg{\meta{text}}\\
\cs{tlabelcircle*...}%
\index{tlabelellipse@\cs{tlabelellipse}}%
\index{tlabelcircle@\cs{tlabelcircle}}
\end{cd}

This figure macro produces the smallest ellipse centered at the point
that encompasses the rectangle defined above, and that has a ratio of
width to height equal to \meta{ratio}, then places the text. The default
ratio is $1$, which produces a circle. We also provide the command
\cs{tlabelcircle}, which takes only the \oarg{\meta{just}} optional
argument. Internally, it just processes any \texttt{*} and calls
\cs{tlabelellipse} with parameter 1.

In the above \cs{tlabel...} curves, the optional parameter should be
positive. If it is zero, all the curves silently revert to
\cs{tlabelrect}. If it is negative, it is silently accepted. In the case
of \cs{tlabelrect} this causes the quarter-circles at the corners to be
indented rather than convex. In the other cases, there is no visible
effect, but in all cases the sense of the curve is reversed.

\begin{cd}\pagelabel{tlpathjustify}
\cs{tlpathjustify}\marg{\meta{just}}%
\index{tlpathjustify@\cs{tlpathjustify}}
\end{cd}

This can be used to change the default justification for \cs{tlabelrect}
and friends. The \meta{just} parameter is exactly as in
\cs{tlabeljustify} in subsection~\ref{text}.


\section{Saving and Reusing an \mfp{} Picture.}\label{saving}

These commands have been changed from versions prior to 0.3.14 in order
to behave more like the \LaTeX{}'s \cs{savebox}, and also to allow the
reuse of an allocated box. Past files that use \cs{savepic} will have to
be edited to add \cs{newsavepic} commands that allocate the \TeX{}
boxes.

\begin{cd}\pagelabel{newsavepic}
\cs{newsavepic}\marg{\meta{picname}}\\
\cs{savepic}\marg{\meta{picname}}\\
\cs{usepic}\marg{\meta{picname}}%
\index{newsavepic@\cs{newsavepic}}%
\index{savepic@\cs{savepic}}%
\index{usepic@\cs{usepic}}
\end{cd}

\cs{newsavepic} allocates a box (like \LaTeX{}'s \cs{newsavebox}) in which
to save a picture. As in \cs{newsavebox}, \meta{picname} is a control
sequence. Example: \cs{newsavepic}\marg{\cs{foo}}. In a \LaTeX{}
document, \cs{newsavepic} is actually defined to be \cs{newsavebox}.

\cs{savepic} saves the \emph{next} \cs{mfpic} picture in the named box,
which should have been previously allocated with \cs{newsavepic}. (This
command should not be used \emph{inside} an \env{mfpic} environment.)
The next picture will not be placed, but saved in the box for later use.
This is primarily intended as a convenience. One \emph{could} use
\begin{ex}
  \cs{savebox}\marg{\meta{picname}}\marg{\meta{entire
  \env{mfpic} environment}},
\end{ex}
but \cs{savepic} avoids having to place the \env{mfpic} environment in
braces, and avoids one extra level of \TeX{} grouping. It also avoids
reading the entire \env{mfpic} environment as a parameter, which would
nullify \mfp{}'s efforts to preserve line breaks in parameters
written to the \MF{} output file. If you repeat \cs{savepic} with the
same \meta{picname}, the old contents are replaced with the next
picture.

\cs{usepic} copies the picture that had been saved in the named box.
This may be repeated as often as liked to create multiple copies of one
picture. The \cs{usepic} command is essentially a clone of the \LaTeX{}
\cs{usebox} command. Since the contents of the saved picture are only
defined during the \TeX{} run, \cs{usebox} cannot be used in the
\TeX-commands argument of the \cs{tlabel} command while \opt{mplabels}
is in effect.


\section{Picture Frames.}\label{frames}

When \TeX{} is run but before \MF{} or \MP{} has been run on the output
file, \mfp{} detects that the \file{.tfm} file is missing or that
the first \MP{} figure file \file{\meta{file}.1} is missing. In these
cases, the \env{mfpic} environment draws only a rectangular frame with
dimensions equal to the nominal size of the picture, containing the
figure number (and any text placed by \cs{tlabel} and its relatives
without \opt{mplabels} in effect). The command(s) used internally to do
this are made available to the user.

\begin{cd}\pagelabel{mfpframe}
\cs{mfpframe}\oarg{\meta{fsep}}%
    \meta{ material-to-be-framed }%
\cs{endmfpframe}\\
\cs{mfpframed}\oarg{\meta{fsep}}\marg{\meta{material-to-be-framed}}%
\index{mfpframed@\cs{mfpframed}}%
\index{endmfpframe@\cs{endmfpframe}}%
\index{mfpframe@\cs{mfpframe}}
\end{cd}

These commands surround their contents with a rectangular frame
consisting of lines with thickness \cs{mfpframethickness} separated from
the contents by the \meta{fsep} if specified, otherwise by the value of
the dimension \cs{mfpframesep}. The default value of the \TeX{}
dimensions \cs{mfpframesep} and \cs{mfpframethickness} are \dim{2pt} and
\dim{0.4pt}, respectively. The \cs{mfpframe} $\ldots$ \cs{endmfpframe}
version is preferred around \env{mfpic} environments or verbatim
material since it avoids reading the enclosed material before
appropriate \cs{catcode} changes go into effect. In \LaTeX{}, one can
also use environment syntax: \cs{begin}\marg{mfpframe} $\ldots$
\cs{end}\marg{mfpframe}.

An alternative way to frame \env{mfpic} pictures is to save them with
\cs{savepic} (see previous section) and issue a corresponding
\cs{usepic} command inside any framing environment or command of the user's
choice or devising.


\section{Affine Transforms.}\label{transforms}

Coordinate transformations that keep parallel lines in parallel are
called \emph{affine transforms}.  These include translation, rotation,
reflection, scaling and skewing (slanting). For the \MF{} coordinate
system only (that is, for paths, but not for \cs{tlabel} nor
\cs{tcaption}) \mfp{} provides the ability to apply \MF{} affine
transforms.


\subsection{Transforming the \MF{} coordinate system}\label{affine}

\begin{cd}\pagelabel{coords}
\cs{coords} \dots \cs{endcoords}%
\index{coords@\cs{coords}}%
\index{endcoords@\cs{endcoords}}
\end{cd}

All affine transforms are restricted to the innermost enclosing
\cs{coords}$\ldots$\cs{endcoords} pair.  If there is \emph{no} such
enclosure, then the transforms will apply to the rest of the \env{mfpic}
environment. In \LaTeX{}, one can use the environment named
\env{coords}.

\medskip
\noindent Transforms provided by \mfp{}:

\nobreak
\begin{cd}\pagelabel{applyT}%
\begin{tabular}[b]{@{}ll@{}}
  \cs{rotate}\marg{\meta{$\theta$}}%
    \index{rotate@\cs{rotate}}%
    & Rotate around origin by \meta{$\theta$} degrees.\\
  \cs{rotatearound}\marg{\meta{$p$}}\marg{\meta{$\theta$}}%
    \index{rotatearound@\cs{rotatearound}}%
    & Rotate around point \meta{$p$} by \meta{$\theta$} degrees.\\
  \cs{turn}\oarg{\meta{p}}\marg{\meta{$\theta$}}%
    \index{turn@\cs{turn}}%
    & Rotate around point \meta{p} (origin is default) by
    \meta{$\theta$}.\\
  \cs{reflectabout}\marg{\meta{$p_1$}}\marg{\meta{$p_1$}}%
    \index{reflectabout@\cs{reflectabout}}%
    & Reflect in the line through points \meta{$p_1$} and \meta{$p_2$}.\\
  \cs{mirror}\marg{\meta{$p_1$}}\marg{\meta{$p_2$}}%
    \index{mirror@\cs{mirror}}%
    & Same as \cs{reflectabout}.\\
  \cs{shift}\marg{\meta{v}}%
    \index{shift@\cs{shift}}%
    & Shift origin by the vector \meta{v}.\\
  \cs{scale}\marg{\meta{s}}%
    \index{scale@\cs{scale}}%
    & Scale uniformly by a factor of \meta{s}.\\
  \cs{xscale}\marg{\meta{s}}%
    \index{xscale@\cs{xscale}}%
    & Scale only the $x$ coordinates by a factor of \meta{s}.\\
  \cs{yscale}\marg{\meta{s}}%
    \index{yscale@\cs{yscale}}%
    & Scale only the $y$ coordinates by a factor of \meta{s}.\\
  \cs{zscale}\marg{\meta{pair}}%
    \index{zscale@\cs{zscale}}%
    & Scale by the length of vector \meta{v}, and rotate by its
      angle.\\
  \cs{xslant}\marg{\meta{s}}%
    \index{xslant@\cs{xslant}}%
    & Skew in $x$ direction by the multiple \meta{s} of $y$.\\
  \cs{yslant}\marg{\meta{s}}%
    \index{yslant@\cs{yslant}}%
    & Skew in $y$ direction by the multiple \meta{s} of $x$.\\
  \cs{zslant}\marg{\meta{pair}}%
    \index{zslant@\cs{zslant}}%
    & See \mfc{zslanted} in \file{grafbase.dtx}.\\
  \cs{boost}\marg{\meta{$\chi$}}%
    \index{boost@\cs{boost}}%
    & Special relativity boost by $\chi$, see \mfc{boost} in
        \file{grafbase.dtx}.\\
  \cs{xyswap}%
    \index{xyswap@\cs{xyswap}}%
    & Exchange the values of $x$ and $y$.\\
  \cs{applyT}\marg{\meta{transformer}}%
    \index{applyT@\cs{applyT}}%
    & Apply the \meta{transformer}.
\end{tabular}
\end{cd}

\cs{applyT} is for \MF{} hackers. Any code is permitted that satisfies
\MF{}'s syntax for a \meta{transformer} (see D.~E.~Knuth, ``The
\MF{}book'', page~73), although no effort is made to correctly write
\TeX{} special characters nor to preserve linebreaks in the code.

When any of these commands is issued, the effect is to transform all
subsequent figures (within the enclosing \env{coords} or \env{mfpic}
environment). In particular, attention may need to be paid to whether
these transformations move (part of) the figure outside the space
allotted by the \cs{mfpic} command parameters.

A not-so-obvious point is that if several of these transformations are
applied in succession, then the most recent is applied first, so that
figures are transformed as if the transformations were applied in the
reverse order of their occurrence. This is similar to the application of
prefix macros (as well as application of transformations in mathematics:
$S T z$ usually means to apply $S$ to the result of $T z$).

Finally, some of these may not produce what the unwary user might expect
if the \env{mfpic} environment was started with unequal scaling. For
example, in such a case a rotated rectangle will not have right angles
unless the rotation is by a multiple of 90 degrees. The reason for this:
the scaling given by the \cs{mfpic} command is applied last and
slanted lines subjected to unequal horizontal and verical scaling will
change have their angles changed.


\subsection{Transforming paths}\label{transformation}

In the previous section we discussed transformations of the \MF{}
coordinate system. Those macros affect the \emph{drawing} of paths and
other figures, but do not change the actual paths. We will explain the
distinction after introducing two macros for storing and reusing
figures.

\begin{cd}\pagelabel{store}
\cs{store}\marg{\meta{path variable}}\marg{\meta{path}}\\
\cs{store}\marg{\meta{path variable}}\meta{path}%
\index{store@\cs{store}}
\end{cd}

This stores the following \meta{path} in the specified \MF{} \meta{path
variable}. Any valid \MF{} symbolic token will do, in particular, any
sequence of letters and underscores. You should be careful to make the
name distinctive to avoid overwriting the definition of some internal
variable. The stored path may later be used as a figure macro
using \cs{mfobj} (below). The \meta{path} may be any of the figure
macros (such as \cs{curve}\marg{(0,0),(1,0),(1,1)}) or the result of
modifying it. For example:
\begin{verbatim}
\store{pth}\lclosed\reverse\curve{(0,0),(1,0),(1,1)}
\end{verbatim}

In fact, \cs{store} is a prefix macro that does nothing to the following
curve except store it. It acts as a rendering macro with a null
rendering, so the curve is not made visible unless other rendering
macros appear before or after it. It allows the following path to be an
argument, that is, enclosed in braces. This is solely to support files
written for past \mfp{} versions in which \cs{store} was \emph{not}
defined as a prefix macro.

One use of \cs{store} is to create a shorthand for a path that is
otherwise long and tedious to type. Another is to create `symbols' or
`arrowheads' for use in \cs{plotsymbol}, \cs{arrowhead} and related
commands.

\begin{cd}\pagelabel{mfobj}
\cs{mfobj}\marg{\meta{path expression}}\\
\cs{mpobj}\marg{\meta{path expression}}%
\index{mfobj@\cs{mfobj}}%
\index{mpobj@\cs{mpobj}}
\end{cd}

This figure macro produces the path represented by \meta{path
expression}, which is either a path variable in which a path was
previously stored, or a valid \MF{} expression combining such variables
and constant paths. This allows the use of path variables or
expressions as figure macros, permitting all prefix operations, etc..
Here are some examples of the use of \cs{store} and \cs{mfobj}.

\nobreak
\begin{verbatim}
\store{my_f}{\cyclic{...}}             % Store a closed curve.
\dotted\mfobj{my_f}                    % Now draw it dotted,
\hatch\mfobj{my_f}                     % and hatch its interior
% Create two symbols
%   one outline:
\store{MyTriang}{\polyline{(-.5,-.5),(.5,-.5),(0,.5),(-.5,-.5)}
%   one solid:
\store{MySolidTriang}{\polygon{(-.5,-.5),(.5,-.5),(0,.5)}
% Use them as symbols:
\plotsymbols{MyTriang}{(0,0),(2,2)}
\arrowmid{MySolidTriang}\polyline{(1,1),(0,2)}
\end{verbatim}

\emph{Note}: If a stored path has the same starting point as ending
point, but is \emph{not} closed then it will behave like \texttt{Circle}
(for example) when used in \cs{plotsymbol}: only its outline is drawn,
and its interior is erased when \opt{clearsymbols} is in effect. If a
closed path is stored, it behaves like \texttt{SolidCircle}: it is not
drawn, but rather filled. If a path is stored that satisfies neither, it
behaves like \texttt{Asterisk}, being simply drawn in all circumstances.

The two forms \cs{mfobj} and \cs{mpobj} are absolutely equivalent; they
differ only in spelling.

It should be noted that every \mfp{} figure is implicitly stored in the
object \gbc{curpath}. So you can use \cs{mfobj}\marg{curpath} and get
the path defined by the most recently completed figure macro (possibly
modified by prefixes).

Getting back to coordinate transforms, if one changes the coordinate
system and then stores and draws a curve, say by
\begin{verbatim}
\coords
  \rotate{45 deg}
  \store{xx}{\rect{(0,0),(1,1)}}
  \dashed\mfobj{xx}
\endcoords
\end{verbatim}
one will get a transformed picture, but the object \cs{mfobj}\marg{xx}
will contain the simple, unrotated rectangular path and drawing it later
(outside the \env{coords} environment) will prove that. This is because
the \texttt{coords} environment works at the drawing level, not at the
definition level.

In oversimplified terms, \cs{dashed} invokes the transformation, but not
\cs{store}. More precisely, the rendering macros have the side effect of
adding ink to the page (or subtracting it). To know where to place this
ink, a calculation is performed that translates graph coordinates to
actual positions. The above transforms work by modify the parameters
used in that calculation.  On the other hand, \cs{store} merely stores
the output of the immediately following prefix or figure macro. See the
beginning of section~\ref{modifier} for a discussion of input, output
and side effects of \mfp{} prefix and figure macros.

The following transformation prefixes provide a means of actually
creating and storing a transformed path. In the terms just discussed,
their input is a path, their output is the transformed path, and they
have no side effects.

\begin{cd}\pagelabel{shiftpath}
\cs{rotatepath}\marg{\meta{$p$},\meta{$\theta$}}$\ldots$\\
\cs{shiftpath}\marg{\meta{v}}$\ldots$\\
\cs{scalepath}\marg{\meta{$p$},\meta{s}}$\ldots$\\
\cs{xscalepath}\marg{\meta{x},\meta{s}}$\ldots$\\
\cs{yscalepath}\marg{\meta{y},\meta{s}}$\ldots$\\
\cs{slantpath}\marg{\meta{y},\meta{s}}$\ldots$\\
\cs{xslantpath}\marg{\meta{y},\meta{s}}$\ldots$\\
\cs{yslantpath}\marg{\meta{x},\meta{s}}$\ldots$\\
\cs{reflectpath}\marg{\meta{$p_1$},\meta{$p_2$}}$\ldots$\\
\cs{xyswappath}$\ldots$\\
\cs{transformpath}\marg{\meta{transformer}}$\ldots$%
\index{rotatepath@\cs{rotatepath}}%
\index{shiftpath@\cs{shiftpath}}%
\index{scalepath@\cs{scalepath}}%
\index{xscalepath@\cs{xscalepath}}%
\index{yscalepath@\cs{yscalepath}}%
\index{slantpath@\cs{slantpath}}%
\index{xslantpath@\cs{xslantpath}}%
\index{yslantpath@\cs{yslantpath}}%
\index{reflectpath@\cs{reflectpath}}%
\index{xyswappath@\cs{xyswappath}}%
\index{transformpath@\cs{transformpath}}
\end{cd}

These are modifying macros that all return the result of applying an
affine transformation to the following path. They differ in the
transformation applied and the data needed in the mandatory argument. I
have found them extremely useful, and better than \env{coords}
environments when I need to draw a figure, together with several
slightly different versions of it. If \cs{store} is used just before one
of these prefixes, it stores the transformed path rather than the
original.

\cs{rotatepath} rotates the following path by \meta{$\theta$} degrees
about point \meta{$p$}.

\cs{shiftpath} shifts the following path by the vector \meta{v}.

\cs{scalepath} scales (magnifies or shrinks) the following path by the
factor \meta{s}, in such a way that the point \meta{$p$} is kept fixed.
That is
\begin{verbatim}
\scalepath{(0,0),2}\rect{(0,0),(1,1)}
\end{verbatim}
is essentially the same as \cs{rect}\marg{(0,0),(2,2)}, while
\begin{verbatim}
\scalepath{(1,1),2}\rect{(0,0),(1,1)}
\end{verbatim}
is the same as \cs{rect}\marg{(-1,-1),(1,1)}. In both cases the
rectangle is doubled in size. In the first case the lower left corner
stays the same, while in the second case the the upper right corner
stays the same.

\cs{xscalepath} is similar to \cs{scalepath}, but only the $x$-direction
is scaled, and all points with first coordinate equal to \meta{x} remain
fixed. \cs{yscalepath} is similar, except the $y$-direction is affected.

\cs{slantpath} applies a slant transformation to the following path,
keeping points with second coordinate equal to \meta{y} fixed. That is,
a point $p$ on the path is moved right by an amount proportional to the
height of $p$ above the line $y={}$\meta{y}, with $s$ being the
proportionality factor. Points below that line move left. Vertical lines
in the path will acquire a slope of $1/s$, while horizontal lines stay
horizontal.

\cs{xslantpath} is an alias for \cs{slantpath}

\cs{yslantpath} is similar to \cs{xslantpath}, but exchanges the roles
of $x$ and $y$ coordinates.

\cs{reflectpath} returns the mirror image of the following path, where
the line determined by the points \meta{$p_1$} and \meta{$p_2$} is the
mirror.

\cs{xyswappath} returns the path with the roles of $x$ and $y$
exchanged. This is similar in some respects to
\cs{reflectpath}\marg{(0,0),(1,1)}, and produces the same result if the
$x$ and $y$ scales of the picture are the same. However,
\cs{reflectpath} compensates for such different scales (so the path
shape remains the same), while \cs{xyswappath} does not. However, after
a swap, verticals become horizontal and horizontals become vertical.
(It is impossible, when the scales are different, for an affine transform
to both preserve shape and exchange horizontal and vertical lines.)

This compensation for different scales is also done for \cs{rotatepath},
so the resulting path always has the same shape after the rotation as
before. None of the other path transformation prefixes compensate for
different scales, and none of the coordinate system transformations of
the previous subsection do it.

For \MF{} or \MP{} power users, \cs{transformpath} can take any
`transformer' and transform the following path with it. Here, a
\emph{transformer} is the same as in the previous section. Examples are
\mfc{scaled}, \mfc{shifted(1,1)}, and \mfc{rotatedabout(0,1)}. Note that
using this last transformer with \cs{transformpath} is almost like
\cs{rotatepath}\marg{(0,1)}, but it does not compensate for different
scales.

All these prefixes change only the path that follows, not any rendering
of it that follows. For example:
\begin{verbatim}
\gfill\rotatepath{(0,0),90}\dashed\rect{(0,0),(1,1)}
\end{verbatim}
will not produce a rotated dashed rectangle. Rather the original
rectangle will be dashed, and the rotated rectangle will be filled.

One complication is the handling of the default rendering. One expects
\begin{verbatim}
\rect{(0,0),(1,1)}
\end{verbatim}
to draw a rectangle, and
\begin{verbatim}
\rotatepath{(0,0),45}\rect{(0,0),(1,1)}
\end{verbatim}
to draw a rotated rectangle (but not the original). That is, a
transformation + figure is treated as if it were a single figure. But
what would one expect in the following?
\begin{verbatim}
\rotatepath{(0,0),45}\dashed\rect{(0,0),(1,1)}
\end{verbatim}
What one will get is the original dashed and the rotated one with the
default rendering (typically drawn with solid lines). That is, these
prefixes cannot see the renderings that occur later in the sequence.
They add the default rendering as if those didn't exist. If something
other than this is desired, one can either rearrange the prefixes or add
a \phantomsection\label{norenderexample}\cs{norender} in appropriate
places. For example, to add a shifted arrowhead without drawing the
shifted path:
\begin{verbatim}
\arrow\norender\shiftpath{(0,1)}\arrow\draw\lines{(0,0),(8,8)}
\end{verbatim}

\section{Parameters.}\label{parameters}

There are many parameters in \mfp{} which the user can modify to
obtain different effects, such as different arrowhead size or shape.
Most of these parameters have been described already in the context of
macros they modify, but they are all described together here.

Many of the parameters are stored by \TeX{} as dimensions, and so are
available even if there is no \MF{} file open; changes to them are not
subject to the usual \TeX{} rules of scope however: they are local
only to \env{mfpic} environments if set inside one, otherwise
they are global. This is for consistency: other parameters are stored by
\MF{} (so the macros to change them will have no effect unless a \MF{}
file is open) and the changes are subject to \MF{}'s rules of scope---to
the \mfp{} user, this means that changes inside the \cs{mfpic} $\ldots$
\cs{endmfpic} environment are local to that environment, but other
\TeX{} groupings have no effect on scope. Some commands (notably those
that set the axismargins and \cs{tlabel} parameters) change both \TeX{}
parameters and \MF{} parameters, and it is important to keep them
consistent.

There are a few parameters that do obey \TeX{} grouping, but only inside
\env{mfpic} environments. These are noted where the parameter is
described.

All parameters are initialized when \prog{mfpic} is loaded. We give the
initial value or state in each of these descriptions.

\begin{cd}\pagelabel{mfpicunit}
\cs{mfpicunit}%
\index{mfpicunit@\cs{mfpicunit}}
\end{cd}

This dimension stores the basic unit length for \mfp{} pictures. The $x$
and $y$ scales in the \cs{mfpic} macro are multiples of this unit. The
initial value is \dim{1pt}. It is global outside an \env{mfpic}
environment. Changes made to it inside an \env{mfpic} environment have
no effect and are lost at the end of the environment.

\begin{cd}\pagelabel{pointsize}
\cs{pointsize}%
\index{pointsize@\cs{pointsize}}
\end{cd}

This dimension stores the diameter of the circle drawn by the
\cs{point} macro and the diameter of the symbols drawn by \cs{plot},
\cs{plotsymbol} and \cs{plotnodes}. The initial value is \dim{2pt}.

\begin{cd}\pagelabel{pointfilltrue}
\cs{pointfilltrue}, \cs{pointfillfalse}%
\index{pointfilltrue@\cs{pointfilltrue}}%
\index{pointfillfalse@\cs{pointfillfalse}}
\end{cd}

This \TeX{} boolean switch determines whether the circle drawn by
\cs{point} will be filled or open (outline drawn, inside erased).
The initial state is \texttt{true}: filled. This value is local to any \TeX{}
group inside an \env{mfpic} environment. Outside such it is global.

\begin{cd}\pagelabel{drawpen}
\cs{pen}\marg{\meta{size}}\\
\cs{drawpen}\marg{\meta{size}}\\
\cs{penwd}\marg{\meta{size}}%
\index{pen@\cs{pen}}%
\index{drawpen@\cs{drawpen}}%
\index{penwd@\cs{penwd}}
\end{cd}

These commands establishes the width of the normal drawing pen (that is,
the thickness of lines, whether solid or dashed). The initial value is
\dim{0.5bp}. This width is stored by \MF{}. This has no effect on the
size of dots for \cs{dotted}, \cs{shade}, \cs{grid}, etc. It also has no
effect on the lines drawn for hatching. There exist three aliases for
this command, the first two to maintain backward compatibility, the last
one for consistency with other dimension changing commands. Publishers
generally recommended authors to use at least a width of one-half point
for drawings submitted for publication.

\begin{cd}\pagelabel{shadewd}
\cs{shadewd}\marg{\meta{diam}}%
\index{shadewd@\cs{shadewd}}
\end{cd}

This command sets the diameter of the dots used in the shading macro.
The drawing and hatching pens are unaffected by this. The initial value
is \dim{0.5bp}, and the value is stored by \MF{}.

\begin{cd}\pagelabel{hatchwd}
\cs{hatchwd}\marg{\meta{size}}%
\index{hatchwd@\cs{hatchwd}}
\end{cd}

This sets the line thickness used in the hatching macros. The drawing
pen and shading dots are unaffected by this. The initial value is
\dim{0.5bp}, and the value is stored by \MF{}.

\begin{cd}\pagelabel{polkadotwd}
\cs{polkadotwd}\marg{\meta{diam}}%
\index{polkadotwd@\cs{polkadotwd}}
\end{cd}

This sets the diameter of the dots used in the \cs{polkadot} macro. The
initial value is \dim{5bp}, and the value is stored by \MF{}.

\begin{cd}\pagelabel{headlen}
\cs{headlen}%
\index{headlen@\cs{headlen}}
\end{cd}

This dimension stores the length of the arrowhead drawn by the
\cs{arrow} macro. The initial value is \dim{3pt}.

\begin{cd}\pagelabel{axisheadlen}
\cs{axisheadlen}%
\index{axisheadlen@\cs{axisheadlen}}
\end{cd}

This dimension stores the length of the arrowhead drawn by the
\cs{axes}, \cs{xaxis} and \cs{yaxis} macros, and by the macros \cs{axis}
and \cs{doaxes} when applied to the parameters \texttt{x} and
\texttt{y}. The initial value is \dim{5pt}.

\begin{cd}\pagelabel{sideheadlen}
\cs{sideheadlen}%
\index{sideheadlen@\cs{sideheadlen}}
\end{cd}

This dimension stores the length of the arrowhead drawn by the \cs{axis}
and \cs{doaxes} macros when applied to \texttt{l}, \texttt{b},
\texttt{r} or \texttt{t}. The initial value is \dim{0pt} (that is, the
default is not to put arrowheads on border axes).

\begin{cd}\pagelabel{headshape}
\cs{headshape}\marg{\meta{ratio}}\marg{\meta{tension}}\marg{\meta{filled}}%
\index{headshape@\cs{headshape}}
\end{cd}

This establishes the shape of the \gbc{Arrowhead} drawn by the
\cs{arrow...} and \cs{axes} macros. It also establishes the shape of
\gbc{Leftharpoon} and \gbc{Rightharpoon}. The value of \meta{ratio} is
the ratio of the width of the arrowhead to its length; \meta{tension} is
the tension of the B\'ezier curves; and \meta{filled} is a \MF{} boolean
value indicating whether the arrowheads are to be filled (if \mfc{true})
or open. The initial values are $1$, $1$, and \mfc{false}, respectively.
Setting \meta{tension} to the literal keyword `\mfc{infinity}' will make
the sides of the arrowheads straight lines. The harpoon heads are
arranged to be exactly half of the full arrowhead. The \meta{ratio},
\meta{tension} and \meta{filled} values are stored by \MF{}.

After \cs{headshape} is used, the symbols \gbc{Arrowhead},
\gbc{Leftharpoon}, and \gbc{Rightharpoon} take on the new shape if used
in one of the \cs{plot...} commands.

\begin{cd}\pagelabel{dashlen}
\cs{dashlen}, \cs{dashspace}%
\index{dashlen@\cs{dashlen}}
\end{cd}

These dimensions store, respectively, the length of dashes and the
length of spaces between dashes, for lines drawn by the \cs{dashed}
macro. The \cs{dashed} macro may adjust the dashes and the spaces
between by as much as $1/n$ of their value, where $n$ is the number of
spaces appearing in the curve, in order not to have partial dashes at
the ends. The initial values are both \dim{4pt}. The dashes will
actually be longer (and the spaces shorter) by the thickness of the pen
used when they are drawn.

\begin{cd}\pagelabel{dashlineset}
\cs{dashlineset}, \cs{dotlineset}%
\index{dashlineset@\cs{dashlineset}}%
\index{dotlineset@\cs{dotlineset}}
\end{cd}

These macros provide shorthands for certain settings of the \cs{dashlen}
and \cs{dashspace} dimensions. The macro \cs{dashlineset} sets both
values to \dim{4pt}, while \cs{dotlineset} sets \cs{dashlen} to
\dim{1pt} and \cs{dashspace} to \dim{2pt}. They are kept mainly for
backward compatibility.

\begin{cd}\pagelabel{hashlen}
\cs{hashlen}%
\index{hashlen@\cs{hashlen}}
\end{cd}

This dimension stores the length of the axis hash marks drawn by the
\cs{xmarks} and \cs{ymarks} macros. The initial value is \dim{4pt}.

\begin{cd}\pagelabel{shadespace}
\cs{shadespace}%
\index{shadespace@\cs{shadespace}}
\end{cd}

This dimension establishes the spacing between dots drawn by the
\cs{shade} macro. The initial value is \dim{1pt}.

\begin{cd}\pagelabel{darkershade}
\cs{darkershade}, \cs{lightershade}%
\index{darkershade@\cs{darkershade}}%
\index{lightershade@\cs{lightershade}}
\end{cd}

These macros both multiply the \cs{shadespace} dimension by constant
factors, $5/6=.833333$ and $6/5=1.2$ respectively, to provide convenient
standard settings for several levels of shading. Under \MF{} it is
possible that using one of these macros can have no visible effect. See
the discussion of the \cs{shade} macro in subsection~\ref{filling}.

\begin{cd}\pagelabel{polkadotspace}
\cs{polkadotspace}%
\index{polkadotspace@\cs{polkadotspace}}
\end{cd}

This dimension establishes the spacing between the centers of the dots
used for the macro \cs{polkadot}. The initial value is \dim{10pt}.

\begin{cd}\pagelabel{dotsize}
\cs{dotsize}, \cs{dotspace}%
\index{dotsize@\cs{dotsize}}\index{dotspace@\cs{dotspace}}%
\end{cd}

These \TeX{} dimensions establishes the size and spacing between the
centers of the dots used in the \cs{dotted} macro. The initial values
are \dim{0.5pt} and \dim{3pt}.

\begin{cd}\pagelabel{griddotsize}
\cs{griddotsize}%
\index{griddotsize@\cs{griddotsize}}%
\end{cd}

This dimension gives the default for the sizes of dots in the \cs{grid}
and \cs{plrgridpoints} commands. The initial value is \dim{0.5pt}.

\begin{cd}\pagelabel{symbolspace}
\cs{symbolspace}%
\index{symbolspace@\cs{symbolspace}}
\end{cd}

Similar to \cs{dotspace}, this \TeX{} dimension establishes the space
between the centers of symbols placed by the macro
\cs{plot}\marg{\meta{symbol}}$\ldots\,$. Its initial value is \dim{5pt}.

\begin{cd}\pagelabel{hatchspace}
\cs{hatchspace}%
\index{hatchspace@\cs{hatchspace}}
\end{cd}

This dimension establishes the spacing between lines drawn by the
\cs{hatch} macro. The initial value is \dim{3pt}.

\begin{cd}
\cs{tlpointsep}\marg{\meta{separation}}\\
\cs{tlpathsep}\marg{\meta{separation}}\\
\cs{tlabelsep}\marg{\meta{separation}}%
\index{tlpointsep@\cs{tlpointsep}}%
\index{tlpathsep@\cs{tlpathsep}}%
\index{tlabelsep@\cs{tlabelsep}}
\end{cd}

The first macro establishes the separation between a label and its
nominal position. It affects text written with any of the commands
\cs{tlabel}, \cs{tlabels}, \cs{axislabels} or \cs{plottext}. The second
sets the separation between the text and the curve defined by the
commands \cs{tlabelrect}, \cs{tlabeloval} or \cs{tlabelellipse}. The
third sets both of these separations to the same value. It is for
backward compatibility: in the past there was only one dimension used
for both purposes. The initial value of each is \dim{0pt}. The values
are stored by both \TeX{} and \MF{}.

\begin{cd}
\cs{tlabeloffset}\marg{\meta{hlen}}\marg{\meta{vlen}}%
\index{tlabeloffset@\cs{tlabeloffset}}
\end{cd}

This macro establishes a uniform offset that applies to all labels. It
affects text written with any of the commands \cs{tlabel}, \cs{tlabels},
\cs{axislabels} or \cs{plottext}. The initial state is to have both
horizontal and vertical offsets of \dim{0pt}. The values are stored by
both \TeX{} and \MF{}.

\begin{cd}\pagelabel{mfpdataperline}
\cs{mfpdataperline}%
\index{mfpdataperline@\cs{mfpdataperline}}
\end{cd}

When \mfp{} is reading from data files and writing to the output file,
this macro stores the maximum number of data points that will be written
on a single line in the output file. Its initial definition is
\cs{def}\cs{mfpdataperline}\marg{5}. Any such definition (or
redefinition) obeys \emph{all} \TeX{} groupings.

\begin{cd}\pagelabel{mfpicheight}
\cs{mfpicheight}, \cs{mfpicwidth}%
\index{mfpicheight@\cs{mfpicheight}}%
\index{mfpicwidth@\cs{mfpicwidth}}
\end{cd}

These dimensions store the height and width of the figure created by the
most recently completed \env{mfpic} environment. This might perhaps be
of interest to hackers or to aid in precise positioning of the graphics.
They are meant to be read-only: the \cs{endmfpic} command globally sets
them equal to the height and width of the picture, but \mfp{} does not
otherwise make any use of them. As they are not to be changed, grouping
is irrelevent, but when \mfp{} sets them, it does so globally. These are
set even if the picture is saved with \cs{savepic}. If they are needed
for the corresponding \cs{usepic}, and that occurs after another
\env{mfpic} environment, they should be copied to other length commands
right after the \env{mfpic} environment that set them.

\begin{cd}\pagelabel{mfpiccaptionskip}
\cs{mfpiccaptionskip}%
\index{mfpiccaptionskip@\cs{mfpiccaptionskip}}
\end{cd}

This skip register (`rubber length' in \LaTeX) stores the space between
a picture and the caption produced with \cs{tcaption}. It is local to
all \TeX{} groups. If changed inside an \env{mfpic} environment it will
affect only the \cs{tcaption} command in that picture. It's initial
setting is \cs{medskipamount}, producing the same space as a
\cs{medskip}.


\section{For Advanced Users.}\label{advanced}

\subsection{Splines}\label{splines}

\begin{cd}\pagelabel{qspline}
\cs{qspline}\marg{\meta{list}}\\
\cs{closedqspline}\marg{\meta{list}}\\
\cs{cspline}\marg{\meta{list}}\\
\cs{closedcspline}\marg{\meta{list}}%
\index{qspline@\cs{qspline}}%
\index{closedqspline@\cs{closedqspline}}%
\index{cspline@\cs{cspline}}%
\index{closedcspline@\cs{closedcspline}}%
\end{cd}

These figure macros use alternative ways of defining curves. In each case,
\meta{list} is a comma separated list of ordered pairs. These represent
not the points the curve passes through, but the \emph{control points}.
The first two produce quadratic B-splines and the last two produce cubic
B-splines. If you don't know what B-splines are, or don't know what
control points are, it is recommended you not use these commands.

For \cs{qspline}, the curve will pass through the midpoints of the line
segments joining the points in the list, tangent to that line segment.

For the \cs{cspline}, the list also defines line segments. Divide these
into equal thirds at two points on each segment. Connect these
\emph{division points only} to obtain line segments. Each \emph{odd
numbered} segment is the middle third of one of the original line
segments. The \cs{cspline} curve passes through the midpoint of each
\emph{even numbered} line segment, tangent to it.

\begin{cd}\pagelabel{computedspline}
\cs{computedspline}\marg{\meta{list}}\\
\cs{closedcomputedspline}\marg{\meta{list}}%
\index{computedspline@\cs{computedspline}}%
\index{closedcomputedspline@\cs{closedcomputedspline}}%
\end{cd}

These figure macros both produce cubic splines. For these you \emph{do}
provide the list of points the curves are to pass through. They become
the nodes, and then the control points are computed from them. The nodes
do not uniquely determine the control points so extra equations are
required. For the first version, the extra equations give the path zero
curvature at the endpoints (a \emph{relaxed} spline). For the closed
version, the extra equations are those that close the curve smoothly.
The portions of the spline that connect one node to the next are
parametrized cubic B/'eziers, they are computed so that the first and
second derivatives (with respect to the parameter) of adjacent curves
match at the common node.

\begin{cd}\pagelabel{fcnspline}
\cs{fcnspline}\marg{\meta{list}}\\
\cs{periodicfcnspline}\marg{\meta{list}}%
\index{fcnspline@\cs{fcnspline}}%
\index{periodicfcnspline@\cs{periodicfcnspline}}%
\end{cd}

These figure macros use cubic spline equations (as in
\cs{computedspline} above) to produce a smooth graph of a function based
on a list of points with increasing $x$-values. See \cs{fcncurve} in
section~\ref{curves} for another way to do this. As in the computed
splines, above, the spline equations at the nodes do not provide
sufficient information to compute all control points. In the basic
version, \cs{fcnspline}, extra equations produce a graph with zero
curvature at the endpoints (a relaxed spline), while the periodic
version uses equations that make the first and second derivatives at the
last point match those at the first point.

\begin{cd}\pagelabel{cbclosed}
\cs{cbclosed}$\ldots$\\
\cs{qbclosed}$\ldots$%
\index{cbclosed@\cs{cbclosed}}%
\index{qbclosed@\cs{qbclosed}}
\end{cd}

These are modifying macros that close the following path. The first
closes with a cubic B-spline, the second with a quadratic B-spline. They
will close any given curve, but the command \cs{cbclosed} is meant to
close a cubic B-spline (see above). That is, \cs{cbclosed}\cs{cspline}
should produce the same result as \cs{closedcspline} with the same
argument. The corresponding statements are true of \cs{qbclosed}: it is
meant to close a quadratic B-spline and \cs{qbclosed}\cs{qspline} should
produce the same result as \cs{closedqspline} with the same argument.

\subsection{B\'eziers}

The power user, having noticed that \cs{curve} and \cs{cyclic} insert
some direction modifiers into the path created, may have decided that
there is no \mfp{} command to create a simple \MF{} default style path,
for example \mfc{(1,1)..(0,1)..(0,0)..cycle}. If so, he or she has forgotten
about \cs{mfobj}: the command
\begin{verbatim}
\mfobj{(1,1)..(0,1)..(0,0)..cycle}
\end{verbatim}
will produce, in the \file{.mf} file, exactly this path, but surround it
with the \TeX{} wrapping needed to make \mfp{}'s prefix macro system work.
However, the syntax of more complicated paths can be extremely lengthy,
so we offer this interface:

\begin{cd}\pagelabel{mfbezier}
  \cs{mfbezier}\oarg{\meta{tens}}\marg{\meta{list}}\\
  \cs{closedmfbezier}\oarg{\meta{tens}}\marg{\meta{list}}%
\index{mfbezier@\cs{mfbezier}}%
\index{closedmfbezier@\cs{closedmfbezier}}
\end{cd}

These figure macros uses the \MF{} path join operator `\mfc{..tension
\meta{tens}..}' to connect the points in the list. If the tension option
\oarg{\meta{tens}} is omitted, the value set by \cs{settension}
(initially 1) is used. One can get a cyclic path by prepending
\cs{bclosed} (with matching tension option), but it will not produce the
same result as \cs{closedmfbezier}. These are cubic B\'ezier's (but you
know that if you are a power user). Quadratic B\'eziers (as in
\LaTeX{}'s picture environment) can be obtained with the following:

\begin{cd}\pagelabel{qbeziers}
  \cs{qbeziers}\marg{\meta{list}}\\
  \cs{closedqbeziers}\marg{\meta{list}}%
\index{qbeziers@\cs{qbeziers}}%
\index{closedqbeziers@\cs{closedqbeziers}}
\end{cd}

These figure macros produce \emph{quadratic} B\'ezier curves, the
equivalent of a sequence of \LaTeX{} \cs{qbezier} commands. Note the
plural forms, to distinguish the first from the \LaTeX{} command, and to
indicate that they can draw a \emph{series} of quadratic B\'eziers.

In the \meta{list}, the first, third, fifth, etc., are the points to
connect, while the second, fourth, etc., are the control points. The
open version requires an ending point, and so needs an odd number of
points in the list. The closed version assumes the first point is the
ending, and so requires an even number in the list. If the number of
ponts is wrong, no error is produced: the last point is simply repeated
to get the required number.

The curve will not automatically be smooth; that depends on the choice
of the control points.

\begin{cd}\pagelabel{cbeziers}
  \cs{cbeziers}\marg{\meta{list}}\\
  \cs{closedcbeziers}\marg{\meta{list}}%
\index{cbeziers@\cs{cbeziers}}%
\index{closedcbeziers@\cs{closedcbeziers}}
\end{cd}

These figure macros produce a series of \emph{cubic} B\'ezier curves. In
the \meta{list}, the first, fourth, seventh, etc., are the points to
connect, while the second and third, fifth and sixth, etc., are pairs of
control points. The closed version uses the starting point as the ending
point, and so needs a number of points divisible by $3$ ($n=3k$). The
open version requires an explicitly given ending node (so $n=3k+1$).
If the number of ponts is wrong, no error is produced: the last point
or last two points are simply repeated to get the required number.

The curves will not automatically be smooth; that depends on the choice
of the control points. Cubic B\'eziers are how curves are represented in
PostScript files, and how a number of vector drawing programs represent
curves.

\subsection{Raw \MF{} code}\label{mfcode}

\begin{cd}\pagelabel{mfsrc}
\cs{mfsrc}\marg{\meta{metafont code}}\\
\cs{mfcmd}\marg{\meta{metafont code}}\\
\cs{mflist}\marg{\meta{metafont code}}%
\index{mfsrc@\cs{mfsrc}}%
\index{mfcmd@\cs{mfcmd}}%
\index{mflist@\cs{mflist}}%
\end{cd}

These all write the \meta{metafont code} directly to the \MF{} file,
using a \TeX{} \cs{write} command. Line breaks within \meta{metafont
code} are preserved.%
    \footnote{Under most circumstances, but not if the command (plus its
    argument) is part of the argument of another macro.} %
Almost all the \mfp{} drawing macros invoke one of these. Because of the
way \TeX{} reads and processes macro arguments, not all drawing macros
preserve line breaks (nor do they all need to). However, the ones that
operate on long lists of pair or numeric data (for example, \cs{point},
\cs{curve}, etc.), do preserve line breaks in that data. The difference
in these is minor: \cs{mfsrc} writes its argument without change,
\cs{mfcmd} appends a semicolon (`\mfc{;}') to the code, while
\cs{mflist} surrounds its argument with parentheses and then appends a
semicolon.

Using these can have some rather bizarre consequences, though, so it is
not recommended to the unwary. It is, however, currently the only way to
make use of \MF{}'s equation solving ability. Here's an oversimplified
example:
\begin{verbatim}
\mfpic[20]{-0.5}{1.5}{0}{1.5}
\mfsrc{z1=(0,0);
  z2-z3=(1,2);
  z2+2z3=(1,-1);}       % z2=(1,1), z3=(0,-1)
\arc[t]{z1,z2,z3}
\endmfpic
\end{verbatim}

Check out the sample \file{forfun.tex} for a more extensive example. It
should produce the word `\textsf{mfpic}' in blue, outlined in green in a
box with yellow background.

\subsection{Creating \MF{} variables}\label{variables}

\begin{cd}\pagelabel{setmfvariable}
\cs{setmfvariable}\marg{\meta{type}}\marg{\meta{name}}\marg{\meta{value}}\\
\cs{setmpvariable}\marg{\meta{type}}\marg{\meta{name}}\marg{\meta{value}}\\
\cs{globalsetmfvariable}\marg{\meta{type}}\marg{\meta{name}}\marg{\meta{value}}\\
\cs{globalsetmpvariable}\marg{\meta{type}}\marg{\meta{name}}\marg{\meta{value}}\\
\cs{setmfnumeric}\marg{\meta{name}}\marg{\meta{value}}\\
\cs{setmfpair}   \marg{\meta{name}}\marg{\meta{value}}\\
\cs{setmfboolean}\marg{\meta{name}}\marg{\meta{value}}\\
\cs{setmfcolor}  \marg{\meta{name}}\marg{\meta{value}}%
\index{setmfvariable@\cs{setmfvariable}}%
\index{setmpvariable@\cs{setmpvariable}}%
\index{globalsetmfvariable@\cs{globalsetmfvariable}}%
\index{globalsetmpvariable@\cs{globalsetmpvariable}}%
\index{setmfnumeric@\cs{setmfnumeric}}%
\index{setmfpair@\cs{setmfpair}}%
\index{setmfboolean@\cs{setmfboolean}}%
\index{setmfcolor@\cs{setmfcolor}}%
\end{cd}

These formerly internal \mfp{} macros can be use to define symbolic
names for any \MF{} or \MP{} variable type. The last four are
abbreviations for the first used with an appropriate value for
\meta{type}. For example,
\cs{setmfvariable}\marg{pair}\marg{X}\marg{(2,0)} can be abbreviated
\cs{setmfpair}\marg{X}\marg{(2,0)}. Note that these overwrite any
variable with the specified \meta{name}. For certain internal names,
\MF{} will issue an error, but usually the variable is silently
redefined.

The commands \cs{setmpvariable} and \cs{globalsetmpvariable} (note the
\texttt{mp} instead of \texttt{mf}) are just alternative spellings . You
can use either spelling with either the \opt{metafont} or \opt{metapost}
option.

The \meta{value} must be a constant of the appropriate type or a \MF{}
expression returning the appropriate type. It can also be (or include)
other variables previously defined. The \cs{setmfcolor} command has been
enhanced so that in recent \MP{} the \meta{value} can be any of the
three types of colors \MP{} allows: \kw{numeric} (for grayscale color),
\kw{rgbcolor} or \kw{cmykcolor}. The data type of \meta{value} will be
examined, and the variable \meta{name} will be declared to be a variable
of the appropriate type. The same is true of
\cs{setmfvariable}\marg{color}.

As an example of their use, since dimensions are numeric data types in
\MF{}, the command
\begin{verbatim}
\setmfnumeric{my_spc}{5pt}
\setmfnumeric{my_dia}{.8pt}
\end{verbatim}
would set the \MF{} variables \verb$my_spc$ and \verb$my_dia$ to the
values \texttt{5pt} and \texttt{.8pt}, respectively. After that, these
variables can be used in any \emph{drawing} command where a dimension is
required:
\begin{verbatim}
\plot[my_dia,my_spc]{Triangle}\rect{(0,0),(1,1)}
\end{verbatim}
will plot the rectangle with small triangles of diameter \dim{.8pt},
spaced \dim{5pt} apart.

The knowledgeable user may realize that \mfc{path} and \gbc{picture} are
\MF{} data types, and may want use them in \cs{setmfvariable}. It is
also true that at some level, \mfp{} figure macros produce a path and
\cs{mfpimage} produces a picture. However, \mfp{} commands cannot be
used in the value portion of \cs{setmfvariable}. The \TeX{} code that
most \mfp{} commands produce would be meaningless to \MF{}. You can
store the path produced by figure macros with \cs{store}, and store
pictures in variables with \cs{mfpimage} or even \cs{tile}.

With the obvious exception of the \cs{globalsetmfvariable} command, these
commands define the variable locally. That is, the variable will revert
to any previous definition (or become undefined) at the end of the
\env{mfpic} environment it is defined in. It is in fact local to any
\MF{} group. In \mfp{}, only \cs{connect} {\dots} \cs{endconnect},
\cs{mfpimage} {\dots} \cs{endmfpimage}, and \cs{mfpic} {\dots}
\cs{endmfpic} create \MF{} groups in the graph file.

A warning about variable names. \CMF{} and \MP{} allow multi-part
variable names like `\mfc{arrowhead length}' or `\mfc{X.r}' The part
after the first space or `.' is called a \mfc{suffix}. In \MF{},
variable settings are global unless explicitly made local. The code of
the \cs{set...} commands does make the variable setting local. However,
\MF{} syntax forbids this localization when a variable name has a
suffix. Moreover, if you localize a variable, \MF{} will localize all
variables with that name plus any suffix. Even more, localizing a
variable renders all variables with the same name plus suffix locally
undefined. The command \cs{globalsetmfvariable} simply omits the
localization part, so suffixes are permitted, but it cannot `globalize'
something that has previously been localized within the same group.

For example, suppose you use the example code in subsection~\ref{arrows}
and define a custom arrowhead path \gbc{myAH} and the corresponding
clearing path \gbc{myAH.clear}. Suppose now you try to make this head
the default for the \cs{arrow} command by doing the following.
\begin{verbatim}
\setmfvariable{path}{Arrowhead}{myAH}
\end{verbatim}
Then this assignments is local and makes \gbc{Arrowhead.clear} undefined
(locally). You cannot use \cs{setmfvariable} to define
\gbc{Arrowhead.clear}; that will produce an error from \MF{}. You need
to do
\begin{verbatim}
\setmfvariable{path}{Arrowhead}{myAH}
\globalsetmfvariable{path}{Arrowhead.clear}{myAH.clear}
\end{verbatim}
and \emph{both} assignments will be local. To make both assignments
global, use the global version in both.


\begin{cd}\pagelabel{patharr}
\cs{patharr}\marg{\meta{name}}$\ldots$\cs{endpatharr}%
\index{patharr@\cs{patharr}}%
\index{endpatharr@\cs{endpatharr}}
\end{cd}

This pair of macros, acting as an environment, accumulate all enclosing
paths, in order, into a path array named \meta{name}. A path array is a
collection of paths with a common base name indexed by integers from 1
to the number of paths.  Any path in the array can be accessed by means
of \cs{mfobj}. For example, after
\begin{verbatim}
\patharr{pa}
  \rect{(0,0),(1,1)} \circle{(.5,.5), .5}
\endpatharr
\end{verbatim}
then \cs{mfobj}\marg{pa[1]} refers to the rectangle and
\cs{mfobj}\marg{pa[2]} refers to the circle. In case explicit numbers
are used, \MF{} allows \gbc{pa1} as an abbreviation for \gbc{pa[1]}.
However, if a numeric variable or some expression is used (e.g.,
\gbc{pa[n+1]}) the square brackets are required.

This command can only be used in an \env{mfpic} environment. For this
reason, the definitions it makes are global.

\emph{Note}: In \LaTeX{}, this pair of macros can be used in the form of a
\LaTeX{}-style environment called \env{patharr}---as in
\cs{begin}\marg{patharr}$\ldots$\cs{end}\marg{patharr}.

\begin{cd}\pagelabel{setarray}
\cs{setarray}\marg{\meta{type}}\marg{\meta{var}}\marg{\meta{list}}\\
\cs{globalsetarray}\marg{\meta{type}}\marg{\meta{var}}\marg{\meta{list}}\\
\cs{pairarray}\marg{\meta{var}}\marg{\meta{list-of-points}}\\
\cs{numericarray}\marg{\meta{var}}\marg{\meta{list-of-numbers}}\\
\cs{colorarray}\marg{\meta{var}}\marg{\meta{list-of-colors}}\\
\cs{rgbcolorarray}\marg{\meta{var}}\marg{\meta{list-of-rgbcolors}}\\
\cs{cmykcolorarray}\marg{\meta{var}}\marg{\meta{list-of-cmykcolors}}%
\index{setarray@\cs{setarray}}%
\index{globalsetarray@\cs{globalsetarray}}%
\index{pairarray@\cs{pairarray}}%
\index{numericarray@\cs{numericarray}}%
\index{colorarray@\cs{colorarray}}%
\index{rgbcolorarray@\cs{rgbcolorarray}}%
\index{cmykcolorarray@\cs{cmykcolorarray}}
\end{cd}

These enable the simultaneous definition of variables.
For example, after
\begin{verbatim}
\pairarray{X}{(0,1),(1,1),(0,0),(1,0)}
\end{verbatim}
the variables \mfc{X1}, \mfc{X2}, \mfc{X3}, and \mfc{X4} are equal to
the given points in that order. And then
\begin{verbatim}
\polyline{X1,X2,X3,X4}
\end{verbatim}
will draw the lines connecting these four points. The index may
optionally be put in square brackets and may be separated from the name
by any number of spaces. That is, \verb$\polyline{X[1],X[2]}$
and \verb$\polyline{X 1,X  2}$ are the same as \verb$\polyline{X1,X2}$
to \MF{}. If a numeric \emph{expression} is used instead of an explicit
number, square brackets \emph{must} surround it: \gbc{X[1+1]},
\gbc{X[2]}, \gbc{X2} and \gbc{X 2} are all the same. For all these array
commands, the variable \gbc{X} by itself (not followed by any digit or
brackets) becomes a numeric variable equal to the number of elements in
the array. Except for \cs{globalsetarray}, the arrays are defined
locally if these commands occur in an \env{mfpic} environment, global
otherwise.

Array variables may be used only where the values are processed only by
\MF{} or \MP{}, they are unknown to \TeX{}. In particular, they cannot be
used in commands that position text unless \opt{mplabels} is in effect.
Variables may be used in the \meta{list} parameters of commands,
but they must have been previously defined or otherwise known to \MF{}.

Since arrays must all be variables of the same type, one cannot mix rgb
and cmyk colors. The \verb$\colorarray$ command requires rgb colors (for
compatibility with early \MP{}).

Several commands in \mfp{} define arrays of objects that can be used in
other commands. The main ones are \cs{regpolygon}, \cs{piechart} and
\cs{barchart}. These arrays are always global (either because their
use is restricted to an \env{mfpic} environment or for backward
compatibility with the time when they were so restricted).

Using \cs{regpolygon}\marg{\meta{num}}\marg{X}\marg{...}\marg{...} causes a
pair array named \gbc{X} to be defined having \meta{num} elements (and
the additional pair \gbc{X0} for the center). This is in addition to
creating the actual figure. The variable \gbc{X} alone becomes a numeric
equated to \meta{num}.

Using \cs{piechart} (or \cs{mfppiechart}) causes the following arrays to
become defined (or redefined):
\begin{itemize}
  \item \gbc{piewedge}, a path array describing the wedges of the chart.
        To access \gbc{piewedge[1]}, for example, one could use
        \cs{mfobj}\marg{piewedge[1]}. This is almost exactly the
        same as the \mfp{} command \cs{piewdge}\marg{1} without
        optional arguments.
  \item \gbc{pieangle}, a numeric array, gives the starting angles of
        the wedges.
  \item \gbc{piedirection}, a pair array, gives the unit vectors
        pointing from the center of the piechart through middles of the
        wedges. For example, if \cs{pieangle1} is 0 and \gbc{pieangle2}
        is 90 degrees, then \gbc{piedirection1} is $(\cos 45,\sin 45)$,
        the unit vector whose angle is $45$ degrees.
\end{itemize}

Using \cs{barchart} (or \cs{mfpbarchart} or any of its aliases) causes
the following arrays to become defined (or redefined). The exact meaning
depends on whether bars are horizontal or vertical. The following
describes horizontal bars; exchange the roles of $x$ and $y$ if they are
vertical (also change `right' to `top', etc.):
\begin{itemize}
  \item \gbc{barstart}, a numeric array, gives the position on the
        $y$-axis of the leading edge of the bars.
  \item \gbc{barbegin}, numeric, gives the $x$-coordinate of the
        leftmost end of the bars.
  \item \gbc{barend}, numeric, gives the $x$-coordinate of the rightmost
        end of the bars.
  \item \gbc{chartbar}, a path array, gives the actual bars. For
        example, \gbc{chartbar2} is the rectangle with opposite corners
        \gbc{(barbegin2,barstart2)} and \gbc{(barend2,barstart2+barwd)},
        where the numeric variable \gbc{barwd} is the thickness of the
        bar (which is a height for horizontal bars).
  \item \gbc{barlength}, the same as \gbc{barend}. This is for backward
        compatibility; the name was chosen at a time when all the bars
        had one side on an axis.
\end{itemize}

\subsection{Miscelaneous pair expressions}\label{pairexpressions}

A useful \MF{} operator that produces points is the intermediation
operator, whose syntax is
\begin{cd}
\texttt{(\meta{num})[\meta{$p_1$},\meta{$p_2$}]}
\end{cd}
That is, a number or numeric expression in parentheses followed by
literal brackets (this is \emph{not} an optional argument) containing two points
or pair expressions separated by a comma. It returns an intermediate point on
the line through \meta{$p_1$} and \meta{$p_2$}. The formula for the
returned value is $p_1 + \mbox{\meta{num}}(p_2 - p_1)$. The midpoint is
obtained with $\mbox{\meta{num}} = .5$. If the \meta{num} is a pure
number, the parentheses can be omitted, but they are required if it is
any other numeric expression. Values of \meta{num} larger than 1 or
less than zero produce points on the line that lie outside the segment
from $p_1$ to $p_2$. This operator can also be applied to numbers or (in
\MP{}) to colors (of the same type). So that \mfc{(2/3)[3,6] = 5} and
\mfc{.7[green,blue] = (0,.3,.7)}.  See section~\ref{colors} for a
description of colors in \MP{} and \MF{}.

\begin{cd}
  \gbc{pathpoint(\meta{frac},\meta{name})}
\end{cd}
This is another useful \MF{} command. It requires a number, \meta{frac},
and the \emph{name} of a previously defined \MF{} path variable.
(Defined, for example, using \cs{store}; see
subsection~\ref{transformation}). It returns the point on the path that
is approximately that fraction of the path's length from the start of
the path. For example to draw a line from $(0,0)$ to the midpoint of an
arc, do the following:
\begin{verbatim}
\store{myarc}\draw\arc{(1,0),(0,2),90}
\polyline{(0,0), pathpoint(.5,myarc)}
\end{verbatim}
\CMF{} has no general command for calculating the lengths of paths;
\CMP{} does, but it is quite slow. Thus neither program has an efficient
method for finding the described point, so \mfp{} uses \MF\slash\MP{} macros
that are faster, but less accurate than they could be. Still, the
results should (except in pathological cases) be accurate to within a
couple of percent of the length of the path. If they are not, adjust the
value of the fraction. These remarks about accuracy also hold for any
other command (such as \cs{partpath} in subsection~\ref{reversal}) that
take the fraction of a path length as a parameter.

The \gbc{pathpoint} command is not a basic \MF{} command, but is defined
by the \prog{grafbase} macros that accompany \mfp{}.

\MF{} pairs can conveniently be viewed as complex numbers. So
\file{grafbase} also contains some functions useful in complex analysis
(my research field). In what follows \gbc{a}, \gbc{z} and \gbc{w} denote
pair variables or constants, and each function interprets them as
complex numbers. Also \gbc{t} denotes an angle in radians. There are
both numeric and pair valued functions, the type of each is noted after
the description:

\noindent
\begin{tabular}{@{}lp{4.2in}}
\gbc{Arg z}         & The principle argument of $z$ in radians (numeric).\\
\gbc{Log z}         & The principle logarithm of $z$ (pair).\\
\gbc{cis t}         & $(\cos t, \sin t)$, same as \gbc{dir degrees(t)} (pair).\\
\gbc{zexp w}        & The complex exponential, $e^w$ (pair).\\
\gbc{sgn z}         & The signum, $\sgn (0,0) = (0,0)$ otherwise $\sgn z = z/|z|$ (pair).\\
\gbc{conj z}        & The complex conjugate, $\bar z$ (pair).\\
\gbc{Moebius(a) z}  & The M\"obius transformation $(z+a)/(1+\bar{a}z)$ (pair)\\
\gbc{pshdist(z,w)}  & The pseudohyperbolic distance between $z$ and
        $w$: $|z-w| / |1-\bar{w}z|$ (numeric).
\end{tabular}


\subsection{Manipulating \MF{} picture variables}

\begin{cd}\pagelabel{tile}
\cs{tile}\marg{\meta{tilename},\meta{unit},\meta{wd},\meta{ht},\meta{clip}}\\
  \ \meta{\mfp{} drawing commands}\\
\cs{endtile}%
\index{tile@\cs{tile}}%
\index{endtile@\cs{endtile}}
\end{cd}

In this environment, all drawing commands contribute to a \emph{tile}. A
\emph{tile} is a rectangular picture which may be used to fill the
interior of closed paths. Actually, a tile is a composite object. After
\cs{tile}\marg{Nick, ... } $\ldots$ \cs{endtile} a picture variable
\gbc{Nick.pic} is created as well as numeric variable \gbc{Nick.wd} and
\gbc{Nick.ht}. These are needed by the \cs{tess} command, below.

The units of drawing are given by \meta{unit}, which should be an
explicit dimension (like \dim{1pt} or \dim{.2in}). The tile's horizontal
dimensions are $0$ to $\meta{wd}\cdot\meta{unit}$ and its vertical
dimensions $0$ to $\meta{ht}\cdot\meta{unit}$, so \meta{wd} and
\meta{ht} should be pure numbers. If \meta{clip} is \mfc{true} then the
drawing is clipped to be within the tile's boundary.

By using this macro, you can design your own fill patterns (to use them,
see the \cs{tess} macro below), but see the warning about memory use by
the \cs{tess} command. The \meta{tilename} is globally defined by this
command.

\begin{cd}\pagelabel{tess}
\cs{tess}\marg{\meta{tilename}}$\ldots$%
\index{tess@\cs{tess}}
\end{cd}

This rendering macro tiles the interior of a closed path with a
tessellation comprised of copies of the \emph{tile} specified by
\meta{tilename}. The tile must have been previously created by
\cs{tile}\marg{\meta{tilename}, ... }. Tiling an open curve is
technically an error, but the \MF{} code responds by drawing the path
and not doing any tiling. The \MF{} code places shifted copies of the
tile picture in a rectangular grid sufficient to cover the region, then
clips it to the closed path before drawing it.

Tiling large regions with complicated tiles can exceed the capacity of
some versions of \MP{}. There is less of a problem with \MF{}. This is not
because \MF{} has greater capacity, but because of the natural
difference between bitmaps and vector graphics.

In \MP{}, the tiles are copied with whatever color they are given when
they are defined. They can be multicolored.

Before version 0.8, \cs{tile} was the only way to create a picture
variable, and the only way to draw this picture was with the \cs{tess}
command. Now we have the following command to place multiple copies of
a picture:

\begin{cd}\pagelabel{putmfpimage}
  \cs{putmfpimage}\marg{\meta{name}}\marg{\meta{list}}%
\index{putmfpimage@\cs{putmfpimage}}
\end{cd}

This take the name of a picture variable and copies the picture at each
location in the \meta{list}, which should be a comma-separated list of
coordinate pairs in graph coordinates. The picture is copied so that its
\emph{reference point} is placed at each of the locations. The reference
point of a picture created with \cs{tile} is its lower left corner.

\begin{cd}\pagelabel{mfpimage}
  \cs{mfpimage}\oarg{\meta{refpt}}\marg{\meta{picname}}\\
  \ \meta{\mfp{} drawing commands}\\
  \cs{endmfpimage}%
\index{mfpimage@\cs{mfpimage}}%
\index{endmfpimage@\cs{endmfpimage}}
\end{cd}

This is another way to create a picture variable. The drawing commands
within the \env{mfpimage} environment contribute not to the current
\mfp{} picture, but rather to the picture variable named in \meta{picname}.
Otherwise, they operate exactly as they would outside this environment,
using the same coordinate system and the same default values of all
parameters, etc. (unlike the \env{tile} environment, which defines its
own coordinate system). The picture is created with its reference point
at the point \meta{refpt} given in the optional argument. The default is
\texttt{(0,0)}. For example:
\begin{verbatim}
\mfpimage[(1,1)]{Jan}
  \fill\rect{(0,0),(1,1)}
  \fill\rect{(1,1),(2,2)}
  \rect{(0,0),(2,2)}
\endmfpimage
\end{verbatim}
produces a simple 2-by-2 chessboard with its reference point at the
center point $(1,1)$. One can then write something like
\begin{verbatim}
\putmfpimage{Jan}{(1,1),(3,1),(1,3),(3,3)}
\end{verbatim}
to get a 4-by-4 chessboard: the picture \mfc{Jan} copied with its
center at each of the listed points.

The behavior of \cs{tlabel} in an \env{mfpimage} environment depends on
the setting.  If \opt{mplabels} is turned off, then labels are added by
\TeX{} and are \emph{not} included as part of the named \MF{} or \MP{}
picture variable. They are placed on the current picture as if the
\env{mfpimage} environment were not there at all. If \opt{mplabels} is
turned on and \opt{overlaylabels} is also turned on, or if the
\env{mfpimage} environment is between \cs{startbacktext} and
\cs{stopbacktext}, then the labels will be saved and placed when the
\env{mfpic} environment ends and \emph{not} added to the named picture
variable. Thus, to include text labels in the named picture variable,
you must have \opt{mplabels} on, \opt{overlaylabels} off, and
\env{mfpimage} outside any \cs{startbacktext}\slash\cs{stopbacktext}.

The picture created by \cs{mfpimage} is locally defined. That is, it
becomes undefined at the end of the current \env{mfpic} environment. If
one needs it to be global, one can use \cs{globalsetmfvariable} (see
subsection~\ref{variables}) to copy it to another variable. For example.
the command
\begin{verbatim}
  \globalsetmfvariable{picture}{Dan}{Jan}
\end{verbatim}
would make \gbc{Dan} globally defined to be equal to the current value
of the picture \gbc{Jan}. Note that picture variables can consume a lot
of \MF{}'s memory. Copying one variable to another doubles the amount of
memory, at least until the end of the \env{mfpic} environment.

You can use \cs{putmfpimage} inside a \env{mfpimage} environment,
provided the picture being placed has been previously defined. Nesting a
\env{mfpimage} inside another has not been tested at all and so is not
recommended. But if it works, the inner image would be local to the
environment created by the outer one, and so would be of limited use.
One can use the \LaTeX{} environment construct \cs{begin}\marg{mfpimage}
$\ldots$ \cs{end}\marg{mfpimage} in a LaTeX document instead of
\cs{mfpimage} $\ldots$ \cs{endmfpimage}.

\subsection{\CMF{} loops}\label{loops}

All the \mfp{} loop commands create a loop (in the \MF{} language) in
the output file. The \MF{} commands in that loop are executed repeatedly
by \MF{} or \MP{}. From the point of view of \TeX{}, however each
command occurs only once. Starting with version 0.9, these loops can be
created inside or outside the \env{mfpic} drawing environment. If
outside, they must not contain any drawing commands, but can contain
commands that set variables, perform computations, etc.

\begin{cd}\pagelabel{mfpfor}
\cs{mfpfor}\marg{\meta{for-loop header}}\\
 \ \meta{\mfp{} commands}\\
\cs{endmfpfor}%
\index{mfpfor@\cs{mfpfor}}%
\index{endmfpfor@\cs{endmfpfor}}
\end{cd}

This creates a for-loop in the \MF{} output file. The \cs{mfpfor} writes
the start of the loop and \cs{endmfpfor} writes the end. Any code
written in the output file between them is executed repeatedly by
\MF{}, according to the information in \meta{for-loop header}. There are
two types of headers possible, illustrated by the following examples.
\begin{verbatim}
\mfpfor{center = (0,0), (1,0), (0,1)}
  \gfill\circle{center,1}
\endmfpfor
\end{verbatim}
This example will fill three circles of radius 1 with centers at the
three given points. This type of header has the format
\begin{display}
    \mfc{\meta{variable} = \meta{list}}
\end{display}
where \meta{variable} should be a simple variable name and \meta{list}
is a comma separated list of items of the appropriate data type. In the
above, \gbc{center} is equated to pairs, but in the following
\begin{verbatim}
\mfpfor{radius = 1,3,4}
  \dotted\circle{(0,0),radius}
\endmfpfor
\end{verbatim}
\gbc{radius} gets numeric values.

The other type of header uses a stepped variable:
\begin{verbatim}
\mfpfor{level = 3 step 2 until 9}
  \circle{(0,0),sqrt(level)}
\endmfpfor
\end{verbatim}
This will cause the \MF{} variable \gbc{level} to step through the
values 3, 5, 7 and 9 and the circles with radius $\sqrt{3}$, $\sqrt{5}$,
etc. will be drawn. This type of header has the format
\begin{display}
  \mfc{\meta{variable} = \meta{start} step \meta{delta} until \meta{stop}}
\end{display}
where \meta{variable} is as before, while \meta{start}, \meta{delta} and
\meta{stop} are numeric values. If \meta{delta} is positive the loop is
skipped entirely if \meta{stop} is less than \meta{start}. Otherwise the
loop is executed successively with the variable equal to \meta{start},
then $\meta{start} + \meta{delta}$ then $\meta{start} + 2\meta{delta}$,
etc., as long as the variable is not greater than \meta{stop}. The
behavior is similar if \meta{delta} is negative, except the loop is
repeated only as long as the variable is not less than \meta{stop}. If
\meta{delta} is \mfc{0}, then the \MF{} run will generate an error.

Note that the index variable (\gbc{center} and \gbc{radius} in the above
two examples) is a temporary \MF{} variable. If \opt{mplabels} is turned
on, this variable will work as expected in the \emph{location} parameter
of a \cs{tlabel} command, but if it is used in the \emph{label} part, it
will be interpreted as \TeX{} code and printed as is. The index variable
reverts to its previous state outside the loop. That is, if it existed
before the loop, it regains its previous value after the loop, and if it
was undefined before the loop, it is again undefined after.

The single word ``\mfc{upto}'' can be used as an abbreviation for
``\mfc{step 1 until}'' and ``\mfc{downto}'' for ``\mfc{step -1 until}''
in for-loop headers. Spaces are not significant in for-loop headers,
except to distinguish the keywords (e.g. \mfc{step}) from variable names
that might be use (e.g., for \meta{start}).

\begin{cd}\pagelabel{mfpwhile}
  \cs{mfpwhile}\marg{\meta{condition}}\\
  \ \meta{\mfp{} commands}\\
  \cs{endmfpwhile}%
\index{mfpwhile@\cs{mfpwhile}}%
\index{endmfpwhile@\cs{endmfpwhile}}
\end{cd}

The \meta{condition} should be an expression that can be either true or
false about a \MF{} variable that changes at some time during the loop
body. The loop body is executed (by \MF) as long as the condition is
true. Example:
\begin{verbatim}
\setmfvariable{numeric}{R}{20}
\mfpwhile{R > 1}
  \rect{(0,0), (R,3R)}
  \mfcmd{R:=R/2}
\endmfpwhile
\end{verbatim}
There are no \mfp{} command to \emph{systematically} change a variable,
so in this example we have resorted to directly writing a \MF{} assignment
command via \cs{mfcmd} (see subsection~\ref{mfcode} above) that reduces
\mfc{R} by half. The loop will be executed with \mfc{R} having the
successive values $20$, $10$, $5$, $2.5$, and $1.25$. The resulting
picture could have been achieved with \cs{mfpfor} using this list of
values.

\begin{cd}\pagelabel{mfploop}
\cs{mfploop}\\
 \ \meta{\mfp{} commands}\\
\cs{mfpuntil}\marg{\meta{condition}}\\
 \ \meta{\mfp{} commands}\\
\cs{endmfploop}%
\index{mfploop@\cs{mfploop}}%
\index{mfpuntil@\cs{mfpuntil}}%
\index{endmfploop@\cs{endmfploop}}%
\end{cd}

The body of this loop will be repeated until the \meta{condition}
becomes true. The condition should be some expression that can be either
true or false about a variable that changes during the loop execution.
It should eventually become true. If an \env{mfploop} environment does
not contain an \cs{mfpuntil} command, then the \cs{endmfploop} command
will generate a warning message. If the warning is ignored, and the user
has not otherwise arranged for loop termination,%
    \footnote{Perhaps by means of \cs{mfsrc} commands. It is because of
    this possibility that only a warning is produced and not an error.}
the \file{.mf} file will contain an infinite loop. The \cs{mfpuntil}
command will break the loop at whatever point it occurs. Example:
\begin{verbatim}
\setmfvariable{numeric}{R}{20}
\mfploop
  \mfcmd{R:=R/2}
  \mfpuntil{R <= 1}
  \rect{(0,0), (R,3R)}
\endmfploop
\end{verbatim}
This will draw rectangles with $R$ equal to $10$, $5$, $2.5$, and
$1.25$. On the next execution of the loop the condition \mfc{R<=1} is
true, and the break occurs before the next rectangle is drawn. Note that
any \cs{mfpwhile} could be encoded with \cs{mfploop}. In fact, the code
written to the output file by
\begin{display}
\cs{mfpwhile}\marg{\meta{condition}}
\end{display}
is identical to that written by
\begin{display}
\cs{mfploop}\cs{mfpuntil}\marg{not \meta{condition}}
\end{display}

The command \cs{mfpuntil} can also be used in \env{mfpfor} and
\env{mfpwhile} environments to break the loop prematurely when the given
condition becomes true.

All three of these loop structures bracket the inner code in a \TeX{}
group. In a \LaTeX{} document, the usual \cs{begin}\slash\cs{end} style can
be used. For example,
\begin{verbatim}
\begin{mfpfor}{radius = 1,3,4}
  \circle{(0,0),radius}
\end{mfpfor}
\end{verbatim}

Just to be clear: in all the examples, what is written to the figure
file is a \emph{single} circle or rectangle drawing command, bracketed
by code that causes \MF{} to execute it several times with different
values for the variable. From \TeX{}'s point of view, there is only one
\mfp{} drawing command.

\subsection{Miscellaneous}\label{misc}

\begin{cd}\pagelabel{mfmode}
\cs{mfmode}\marg{\meta{mode-name}}\\
\cs{mfresolution}\marg{\meta{DPI}}%
\end{cd}

When working with \MF{}, the code in \file{grafbase.mf} needs to know
the resolution at which to make the font with all the figures. If the
wrong resolution is assumed, the figure may end up appearing wrongly
scaled or have other problems (especially with shading). If your DVI
viewing/printing program and the file \file{modes.mf} are correctly
configured, nothing may need to be done. If not, as a last resort, you
can set the \MF{} mode or the \MF{} resolution in your \file{.tex} file
with these commands. If you don't know what that means, ask a guru, but
then you should probably be using \MP{} and not \MF{}.

Note that this is a \emph{last resort}. The code in \file{grafbase.mf}
first checks if \mfc{mode} has been defined, then checks if
\mfc{localfont} is defined and only then checks if the resolution has
been set by this method (if all three fail, it uses a value of 600 DPI).

\begin{cd}\pagelabel{noship}
\cs{noship}\\%
\cs{stopshipping}\\%
\cs{resumeshipping}%
\index{noship@\cs{noship}}%
\index{stopshipping@\cs{stopshipping}}%
\index{resumeshipping@\cs{resumeshipping}}
\end{cd}

\cs{stopshipping} turns off character shipping (by \MF{} to the TFM and
GF files, or by \MP{} to appropriate \EPS{} output file) until
\cs{resumeshipping} occurs. If you want just one character not shipped,
just use \cs{noship} inside that \env{mfpic} environment. This is useful
if all one wishes to do in the current \env{mfpic} environment is to
make tiles (see above) or define picture variables with \cs{mfpimage} or
path arrays with \cs{patharr}. While \cs{mfpimage} defines the picture
locally, one can globally copy it to another variable with
\cs{globalsetmfvariable} (see subsection~\ref{variables}).

\begin{cd}\pagelabel{assignmfvalue}
\cs{assignmfvalue}\marg{\meta{\TeX{}-macro}}\marg{\meta{MF-expr}}\\
\cs{assignmpvalue}\marg{\meta{\TeX{}-macro}}\marg{\meta{MF-expr}}\\
\cs{globalassignmfvalue}\marg{\meta{\TeX{}-macro}}\marg{\meta{MF-expr}}\\
\cs{globalassignmpvalue}\marg{\meta{\TeX{}-macro}}\marg{\meta{MF-expr}}%
\index{assignmfvalue@\cs{assignmfvalue}}%
\index{assignmpvalue@\cs{assignmpvalue}}%
\index{globalassignmfvalue@\cs{globalassignmfvalue}}%
\index{globalassignmpvalue@\cs{globalassignmpvalue}}%
\end{cd}

The command names spelled with ``\texttt{mp}'' are no different than the
ones spelled with ``\texttt{mf}''. You can use either spelling with
either the \opt{metafont} or \opt{metapost} option.

These commands causes the \meta{MF-expr} to be written to the output
file for \MF{} to evaluate. The resulting value is then written to the
\file{.log} file of that \MF{} run. On the next \TeX{} run, if
\opt{mfpreadlog} (see section~\ref{readlog}) is in effect, the macro
\meta{\TeX{}-macro} will be defined to produce the resulting value. For
example:
\begin{verbatim}
\setmfnumeric{s}{2}
\assignmfvalue{\val}{exp s}
\tlabel(1,2){$e^s = \val$}
\end{verbatim}
After \MF{} is run and then \TeX{} run a second time, \cs{val} will
acquire the definition `7.38905', the value of \gbc{exp s} when
\gbc{s=2} (i.e., $e^2$, correct to at least the fourth decimal place).
If \opt{mplabels} is in effect, the correct label is written to the
figure file only during this second run, and a second \MP{} run will be
required. In many cases (when using \pdfTeX{}, for example, or when the
label changes the figure dimensions), a third \TeX{} run will be
required to make the figure correct when it is included in the document.

Before \MF{} is run to evaluate the expression, the macro produces
`???'. Thus, it cannot be used in places where a number is needed (as in
the position arguments of a \cs{tlabel} command). Note also that if a
command defined by \cs{assignmfvalue} is used in a tlabel with
\opt{mplabels} in effect, then \opt{mplabels} must be in effect during
the \cs{assignmfvalue} command as well.

The `\texttt{global}' version makes the definition of the
\meta{\TeX{}-macro} global, surviving the current group. In particular,
it can be used in other pictures. The plain versions create commands
that are only locally defined. Past versions of this manual stated that
you can say
\begin{display}
   \cs{global}\cs{assignmfvalue}
\end{display}
to define the macro globally. This turns out not to be true in all
cases. If a global definition is needed, use the global versions above.

Because of the asynchronous nature of the definition process, using
\cs{assignmfvalue} with the same macro name more than once in the same
\env{mfpic} environment will not work. The macro becomes defined
upon reading the logfile during the execution of \cs{opengraphsfile},
and it will end up with the last definition encountered. (The same is
true for uses outside \env{mfpic} environments: the macro acquires the
last such definition.)  Moreover, the definition is associated to a
picture by number. Which means that reordering the environments or
changing the numbering by any means will require the \TeX{}-\MF{}-\TeX{}
sequence (or more) to be repeated.

If the \meta{\TeX{}-macro} is already defined, no warning will be issued
and the command will be redefined, so be careful in the name chosen.
If \opt{mplabels} is turned off when \cs{assignmfvalue} is used, but
turned on before the \meta{\TeX{}-macro} is used in a \cs{tlabel}
command, the macro definition will not be written to the \file{.mp}
file, and either an error message, or incorrect label will result
when \MP{} tries to make the tlabel.

The concept and much of the code for \cs{assignmfvalue} came from Werner
Lemberg. However, I have rewritten it substantially to conform to \mfp{}
conventions and so any errors are my responsibility.

\begin{cd}\pagelabel{cutoffafter}
\cs{cutoffafter}\marg{\meta{obj}}\dots\\
\cs{cutoffbefore}\marg{\meta{obj}}\dots\\
\index{cutoffafter@\cs{cutoffafter}}%
\index{cutoffbefore@\cs{cutoffbefore}}%
\end{cd}

These prefix macros modify the following path by cutting part of it off.
They take an `object' (a variable in which a path was
previously stored using \cs{store}) and uses it to trim off one end of the
following path. \cs{cutoffbefore} cuts off the part of the path
\emph{before} its first intersection with the object, while
\cs{cutoffafter} cuts off the part \emph{after} the last intersection.
If the path does not intersect the object, nothing is cut off. If the
object and the path intersect in more than one point, as little as
possible (usually%
    \footnote{\MF{}'s methods for finding the `first' point of
    intersection do not always find the actual first one.}) %
is cut off. This is reliable only when there is only one point of
intersection.

These macros can be used to create a curve that starts or ends
right at another figure without having to know the point where the two
curves intersect.

\begin{cd}\pagelabel{random}
\cs{randomlines}\marg{\meta{maxshift}}\dots\\
\cs{randomizepath}\marg{\meta{maxshift}, \meta{weirdness}}\dots
\index{randomizepath@\cs{randomizepath}}%
\index{randomlines@\cs{randomlines}}
\end{cd}

These modify the following path by applying random shifts to the nodes
of a path. The first one, \cs{randomlines} then simply connects those
new points by straight lines, while the second one also applies
randomization to the control vectors. The \meta{maxshift} argument is
either a positive number (in graph units) that limits the distance a
node can be moved, or it is an ordered pair of positive numbers, in
which case the first limits the horizontal distance and the second
limits the vertical. If \meta{maxshift} is larger than the distance
between nodes, cusps or loops are likely in the result.

For \cs{randomizepath} the \meta{weirdness} parameter controls how the
control vectors are modified. Roughly speaking the control vectors are
randomly rotated up to $30\langle\mathit{weirdness}\rangle$ degrees and
randomly scaled up or down by a factor of
$2^{\langle\mathit{weirdness}\rangle}$. (A `control vector' is a vector
pointing from a node to one of its control points.) However, this is
done in a way that preserves smoothness at each node where the path is
smooth. Values of \meta{weirdness} greater than 1 are probably too
weird.

\begin{cd}\pagelabel{brownianmotion}
\cs{brownianmotion}\marg{\meta{start},\meta{num},\meta{scale}}%
\index{brownianmotion@\cs{brownianmotion}}
\end{cd}

This figure macro uses another kind of randomness. The path starts at
the point \meta{start}, then proceeds in a straight line in a random
direction a random distance. The random process used is a normaldeviate
in each coordinate, scaled by \meta{scale}. This is repeated \meta{num}
times. Thus, \meta{start} is a coordinate pair in graph coordinates,
\meta{num} is a positive whole number and \meta{scale} is a positive
real number. In rare cases, the random distance can be quite large, but
on average it will be about $0.56\times{}$\meta{scale}. The size
(bounding box) of the resultant path can also be, in rare cases,
quite large, but it is usually on the order of $\sqrt{\meta{num}}$ times
\meta{scale}.

The path produced is technically not Brownian motion, but rather a
`random walk'. However, for small \meta{scale} and large \meta{num} it
approximates Brownian motion.

\begin{cd}\pagelabel{mftitle}
\cs{mftitle}\marg{\meta{title}}%
\index{mftitle@\cs{mftitle}}
\end{cd}

Write the string \meta{title} to the \MF{} file, and use it as a \MF{}
message. (See \textit{The \MF{}book}, chapter 22, page 187, for two uses of
this.)

\begin{cd}\pagelabel{tmtitle}
\cs{tmtitle}\marg{\meta{title}}%
\index{tmtitle@\cs{tmtitle}}
\end{cd}

Write the text \meta{title} to the \TeX{} document, and to the log file,
and use it implicitly in \cs{mftitle}. This macro forms a local group
around its argument.

\medskip

Since \TeX{} is limited to 256 dimension registers, and since dimensions
are so important to typesetting and drawing, it is common to use up all
256 when drawing packages are loaded. Therefore \mfp{} uses font
dimensions to store dimension values. The following is the command that
handles the allocation of these dimensions.

\begin{cd}\pagelabel{newfdim}
\cs{newfdim}\marg{\meta{fdim}}%
\index{newfdim@\cs{newfdim}}
\end{cd}

This create a new global font dimension named \meta{fdim}, which is a
\TeX{} control sequence (with backslash). It can be used almost like
an ordinary \TeX{} dimension.  One exception is that the \TeX{} commands
\cs{advance}, \cs{multiply} and \cs{divide} cannot be applied directly
to font dimensions (nor \LaTeX{}'s \cs{addtolength}); however, the font
dimension can be copied to a temporary \TeX{} dimension register, which
can then be manipulated and copied back (using \cs{setlength} in \LaTeX{},
if desired). Another exception is that all changes to a font dimension
are global in scope. Also beware that \cs{newfdim} uses font dimensions
from a single font, the \file{dummy} font, which most \TeX{} systems
ought to have. (You'll know if yours doesn't, because \mfp{} will fail
upon loading!) Also, implementations of \TeX{} differ in the number of
font dimensions allowed per font. \Mfp{} currently uses font dimensions
23 through 52, which should be OK.

Almost all of \mfp{}'s basic dimension parameters are font dimensions.
We arrange for them to be local to \env{mfpic} environments by saving
their values at the start and restoring them at the end.

\begin{cd}\pagelabel{setmfpicgraphic}
\cs{setmfpicgraphic}\marg{\meta{filename}}%
\index{setmfpicgraphic@\cs{setmfpicgraphic}}
\end{cd}

This is the command that is invoked to place the graphic created. See
appendix~\ref{graphics} for a discussion of its use and its default
definition. It is a user-level macro so that it can be redefined in
unusual cases. It operates on the output of the following macro:

\begin{cd}\pagelabel{setfilename}
\cs{setfilename}\marg{\meta{file}}\marg{\meta{num}}%
\index{setfilename@\cs{setfilename}}
\end{cd}

\Mfp's figure inclusion code ultimately executes
\cs{setmfpicgraphic} on the result of applying \cs{setfilename} to two
arguments: the file name specified in the \cs{opengraphsfile} command
and the number of the current picture. Normally \cs{setfilename} just
puts them together with the `\texttt{.}' separator (because that is
usually the way \MP{} names its output), but this can be redefined if
the \MP{} output undergoes further processing or conversion to another
format in which the name is changed. Any redefinition of
\cs{setfilename} must come before \cs{opengraphsfile} because that
command tests for the existence of the first figure. After any
redefinition, \cs{setfilename} must be a macro with two arguments that
creates the actual filename from the above two parts. It should also be
completely expandable. See the appendices, subsection~\ref{graphics} for
further dicussion.

\begin{cd}\pagelabel{setfilenametemplate}
\cs{setfilenametemplate}\marg{\meta{template}}%
\index{setfilenametemplate@\cs{setfilenametemplate}}
\end{cd}

With the \opt{metapost} option, when you write
\cs{opengraphsfile}\marg{figs}, a file \file{figs.mp} is created. By 
default, running \MP{} on it results in files named \file{figs.1}, 
\file{figs.2}, etc. Recent \MP{} allows the output filenames to be 
modified. As of \mfp{} version 1.00, you can do this to some extent from 
your \file{.tex} file. One needs to define a template that tells \MP{} 
how to construct the output file name from the `jobname' and the figure 
number. This is done with the above command. In \meta{template} you can 
put any plain characters, plus the two special tokens: \verb$\_$ and 
\verb$\#$. Each figure's filename is constructed by replacing these 
tokens with the \MP{} jobname and the figure number, respectively. For 
example, with the jobname \file{figs},
\begin{verbatim}
\setfilenametemplate{my\_-\#.mps}
\end{verbatim}
will cause the figure files to have names \file{myfigs-1.mps},
\file{myfigs-2.mps}, etc., instead of the defaults. \Mfp{} adjusts the
definition of \cs{setfilename} accordingly, so that the correct
filenames are used.

Do not use this command unless you know your version of \MP{} is recent
enough to have this capability. Under the \opt{metafont} option, this
command is simply ignored, but \mfp{} has no way of checking the \MP{}
version on its own.


\begin{cd}\pagelabel{preparemfpicgraphic}
\cs{preparemfpicgraphic}\marg{\meta{filename}}%
\index{preparemfpicgraphic@\cs{preparemfpicgraphic}}
\end{cd}

This command is automatically invoked before \cs{setmfpicgraphic} to
make any preparations needed. The default definition is to do nothing
except when the \prog{graphics} package is used. That package provides
no clean way to determine the bounding box of the graphic after it is
included. Since \mfp{} needs this information, this command redefines an
internal command of the \prog{graphics} package to make the data
available. If \cs{setmfpicgraphic} is redefined then this may also have
to be redefined.

\begin{cd}\pagelabel{getmfpicoffset}
\cs{getmfpicoffset}\marg{\meta{filename}}%
\index{getmfpicoffset@\cs{getmfpicoffset}}
\end{cd}

This command is automatically invoked after \cs{setmfpicgraphic} to
store the offset of the lower left corner of the figure in the macros
\cs{mfpicllx} and \cs{mfpiclly}. If \cs{setmfpicgraphic} is redefined
then this may also have to be redefined.

\begin{cd}\pagelabel{ifmfpmpost}
\cs{ifmfpmpost}%
\index{ifmfpmpost@\cs{ifmfpmpost}}
\end{cd}

Users wishing to write code that adjusts its behavior to the graph file
processor can use this to test which option is in effect. The macro
\cs{usemetapost} sets it true and \cs{usemetafont} sets it false. There
are no commands \cs{mfpmposttrue} nor \cs{mfpmpostfalse}, since the user
should not be changing the setting once it is set: a great deal of
\mfp{} internal code depends on them, and on keeping them consistent
with the \cs{opengraphsfile} commands reading of these booleans.

\begin{cd}\pagelabel{mfpicversion}
\cs{mfpicversion}%
\index{mfpicversion@\cs{mfpicversion}}
\end{cd}

This expands to the current \mfp{} version multiplied by 100. At this
writing, it produces `\texttt{105}' because the version is 1.05. It can
be used to test the version:
\begin{verbatim}
\ifx\mfpicversion\undefined \def\mfpicversion{0}\fi
\ifnum\mfpicversion<70 ... \else ... \fi
\end{verbatim}
\cs{mfpicversion} was added in version 0.7.


Most of \mfp{}'s commands have arguments with parts delimited by commas
and parentheses. In most cases this is no problem because they are
written unchanged to the \file{.mf} and there they are parsed just fine.
Some commands' arguments, however, have to be parsed by both \TeX{} and
\MF{}. Examples are \cs{tlabel} (sometimes, under \opt{mplabels}), and
\cs{pointdef}. One might be tempted to use \MP{} expressions there and
that works fine as long as they do not contain commas or parentheses. In
such cases, they can sometimes be enclosed in braces to prevent \TeX{}
seeing these elements as delimiters, but sometimes these braces might
get written to the \file{.mf} (or \file{.mp}) output and cause a \MF{}
(\MP{}) error. In such cases the following work-around might be possible:
\begin{verbatim}
\def\identity#1{#1}
\pointdef{A}(\identity{angle (1,2)},3)
\rect{(0,0),\A}
\end{verbatim}

The braces prevent \TeX{}'s argument parsing from seeing the first comma
as a delimiter, but upon writing to the \file{.mf}, the \cs{identity}
commands are expanded and only the contents appear in the output. (\TeX{}
parses the argument to assign meanings to \cs{Ax} and \cs{Ay}.)

If the \prog{babel} package is loaded with certain options, the comma
may become a special character. In that case, one may need to deactivate
babel shorthands before some \mfp{} code. One might use \cs{everymfpic}
to do this in every \env{mfpic} environment. In some cases, one may need
to reactivate babel shorthands insided \cs{tlabel}, and one might use
\cs{everytlabel} for this purpose. See your \prog{babel} documentation
for the commands to do these things.

\clearpage

\def\sectionmark#1{\markright {\thesection\quad#1}}%
\def\subsectionmark#1{}
\def\subsubsectionmark#1{}
\thispagestyle{plain}
\chapter{Appendices}\label{appendices}

\section{Acknowledgements.}\label{acknowledgements}

Tom would like to thank all of the people at Dartmouth as well as out in
the network world for testing \mfp{} and sending him back
comments. He would particularly like to thank:

Geoffrey Tobin
for his many suggestions, especially about cleaning up the \MF{} code,
enforcing dimensions, fixing the dotted line computations, and speeding
up the shading routines (through this process, Geoffrey and Tom managed
to teach each other many of the subtleties of \MF{}), and for keeping
track of \mfp{} for nearly a year while Tom finished his thesis;

Bryan Green
for his many suggestions, some of which (including his rewriting the
\cs{tcaption} macro) ultimately led to the current version's ability to
put graphs in-line or side-by-side; and

Uwe Bonnes and
Jarom\'\i r Kuben,
who worked out rewrites of \mfp{} during Tom's working hiatus and who
each contributed several valuable ideas.

Some credit also belongs to
Anthony Stark,
whose work on a FIG to \MF{} converter has had a serious impact on the
development of many of \mfp{}'s capabilities.

Finally, Tom would like to thank
Alan Vlach,
the other \TeX{}nician at Berry College, for helping him decide on the
format of many of the macros, and for helping with testing.

\medskip
Dan Luecking would like to echo Tom's thanks to all of the above,
especially Geoffrey Tobin and Jarom\'\i r Kuben. And to add the names
Taco Hoekwater, for comments, advice and suggestions, Werner Lemberg,
for the \cs{assignmfvalue} command, and Zaimi Sami~Alex for suggestions.

But mostly, he'd like to thank Tom Leathrum for starting it all.

\section{Changes History.}\label{changes}

See the file \file{changes.txt} for a somewhat sporadic history of
changes to \mfp{}. See the file \file{README} for changes added since
the previous version, and for any known problems.

%\clearpage
\section{Summary of Options.}\label{summary}

Unless otherwise stated, any of the command forms will be local to the
current \env{mfpic} environment if used inside. Otherwise it will affect
all later environments.

\medskip
% \shortstack doesn't allow [t] aligment:
\def\stack#1{{\tabular[t]{@{}l@{}}#1\endtabular}}

% Use halign so it will break over 2 pages
{\openup\jot
\halign to \textwidth
    {#\hfil\quad\tabskip0ptplus 1fil&
        #\hfil\quad&
                \vtop{\parindent0pt\rightskip0pt plus 1fil\relax
                  \hsize.5\hsize\normalbaselines
                  \strut#\strut}\tabskip0pt             \cr
\textsc{Option}:& \textsc{Command form(s)}:& \textsc{Restrictions}:\cr
\noalign{\smallskip\hrule\smallskip}
\opt{metapost}&
        \cs{usemetapost}&
                Command must come before \cs{opengraphsfile}.
                Incompatible with \opt{metafont} option.\cr
\opt{metafont}&
        \cs{usemetafont}&
                The default. Command must come before
                \cs{opengraphsfile}. Incompatible with \opt{metapost}
                option.                                 \cr
\opt{mplabels}&
        \stack{\cs{usemplabels},\\ \cs{nomplabels}}&
                Requires \opt{metapost}. If command is used inside an
                \env{mfpic} environment, it should come before
                \cs{tlabel} commands to be affected.    \cr
\opt{overlaylabels}&
        \stack{\cs{overlaylabels},\\ \cs{nooverlaylabels}}&
                Has no effect without \opt{metapost}.   \cr
\opt{truebbox}&
        \stack{\cs{usetruebbox},\\ \cs{notruebbox}}&
                Has no effect without \opt{metapost}.   \cr
\opt{clip}&
        \stack{\cs{clipmfpic},\\ \cs{noclipmfpic}}&
                No restrictions.                        \cr
\opt{clearsymbols}&
        \stack{\cs{clearsymbols},\\ \cs{noclearsymbols}}&
                No restrictions.                        \cr
\stack{\opt{centeredcaptions}\\ \opt{raggedcaptions}}&
        \stack{\cs{usecenteredcaptions},\\ \cs{nocenteredcaptions}\\
        \cs{useraggedcaptions},\\ \cs{noraggedcaptions}}&
                If command is used inside an \env{mfpic} environment, it
                should come before the \cs{tcaption} command.\cr
\opt{debug}&
        \stack{\cs{mfpicdebugtrue},\\\cs{mfpicdebugfalse}}&
                To turn on debugging while \file{mfpic.tex} is loading,
                issue \cs{def}\cs{mfpicdebug}\marg{true}.\cr
\stack{\opt{draft}\\ \opt{final}\\ \opt{nowrite}}&
        \stack{\cs{mfpicdraft}\\ \cs{mfpicfinal}\\ \cs{mfpicnowrite}}&
                Should not be used together. Command forms should come
                before \cs{opengraphsfile}              \cr
\opt{mfpreadlog}&
        \cs{mfpreadlog}&
                Needed for \cs{assignmfvalue}. Must occur before
                \cs{opengraphsfile}.                    \cr
}}

%\clearpage
\section{Plotting Styles for \cs{plotdata}.}\label{styles}

When \cs{plotdata} passes from one curve to the next, it increments a
counter and uses that counter to select a dash pattern, color, or
symbol. It uses predefined dash patterns named \mfc{dashtype0} through 
\mfc{dashtype5}, or predefined colors named \mfc{colortype0} through 
\mfc{colortype7}, or predefined symbols named \mfc{pointtype0} through 
\mfc{pointtype8}. Here follows a description of each of these variables. 
These variables must not be used in the second argument of 
\cs{reconfigureplot}, whose purpose is to redefine these variables.

\medskip
Under \cs{dashedlines}, we have the following dash patterns:

\medskip
\begin{tabular}{@{}lll}
\textsc{Name}&\textsc{Pattern}&\textsc{Meaning}\cr
\hline
\vbox to 10pt{}%
\mfc{dashtype0}&    \dim{0bp}                    & solid line   \\
\mfc{dashtype1}&    \dim{3bp,4bp}                & dashes       \\
\mfc{dashtype2}&    \dim{0bp,4bp}                & dots         \\
\mfc{dashtype3}&    \dim{0bp,4bp,3bp,4bp}        & dot-dash     \\
\mfc{dashtype4}&    \dim{0bp,4bp,3bp,4bp,0bp,4bp}& dot-dash-dot \\
\mfc{dashtype5}&    \dim{0bp,4bp,3bp,4bp,3bp,4bp}& dot-dash-dash
\end{tabular}

\medskip
Under \cs{coloredlines}, we have the following colors. Except for
\mfc{black} and \mfc{red}, each color is altered as indicated. This is
an attempt to make the colors more equal in visibility against a white
background. (The success of this attempt varies greatly with the output
or display device.) Four of the eight colors use the cmyk model
when the \MP{} version is at least $1.000$.

\medskip
\begin{tabular}{@{}llll}
\textsc{Name}&\textsc{Color}&\textsc{(r,g,b)}&\textsc{(c,m,y,k)}\\
\hline
\vbox to 10pt{}%
\mfc{colortype0}&   black  &  $(  0,  0,  0)$&(0,0,0,1)\\
\mfc{colortype1}&   red    &  $(  1,  0,  0)$&\\
\mfc{colortype2}&   blue   &  $( .2, .2,  1)$&\\
\mfc{colortype3}&   orange &  $(.66,.34,  0)$&\\
\mfc{colortype4}&   green  &  $(  0, .8,  0)$&\\
\mfc{colortype5}&   magenta&  $(.85,  0,.85)$&(0,.85,0,.15)\\
\mfc{colortype6}&   cyan   &  $(  0,.85,.85)$&(.85,0,0,.15)\\
\mfc{colortype7}&   yellow &  $(.85,.85,  0)$&(0,0,.85,.15)\\
\end{tabular}

\medskip

Under \cs{pointedlines} and \cs{datapointsonly}, the following symbols
are used. Internally each is referred to by the numeric name, but they
are identical to the more descriptive name. Syntactically, all are \MF{}
path variables. (The order changed between versions 0.6 and 0.7.)

\medskip
\begin{tabular}{@{}ll}
\textsc{Name}&\textsc{Description}\\
\hline
\vbox to 10pt{}%
\mfc{pointtype0}&   \mfc{Circle}       \\
\mfc{pointtype1}&   \mfc{Cross}        \\
\mfc{pointtype2}&   \mfc{SolidDiamond} \\
\mfc{pointtype3}&   \mfc{Square}       \\
\mfc{pointtype4}&   \mfc{Plus}         \\
\mfc{pointtype5}&   \mfc{Triangle}     \\
\mfc{pointtype6}&   \mfc{SolidCircle}  \\
\mfc{pointtype7}&   \mfc{Star}         \\
\mfc{pointtype8}&   \mfc{SolidTriangle}
\end{tabular}

\section{Special Considerations When Using \CMF{}.}\label{mfconsiderations}

The most important restriction in \MF{} is on the size of a picture.
Coordinates in \MF{} ultimately refer to pixel units in the font that is
output. These are required to be less than 4096, so an absolute limit on
the size of a picture is whatever length a row of 4095 pixels is. In
fonts prepared for a LaserJet4 (600 DPI), this means 6.825 inches
(17.3355cm). For a 1200 DPI pronter, the limit is 3.4125 inches.

A similar limit holds for numbers input, and the values of variables:
\MF{} will return an error for ``\mfc{sin 4096}''. Intermediate values 
can be greater (\mfc{sin (2*2048)} will cause no error), but final, 
stored results are subject to the limit. An \mfp{} example that 
generated an error recently was:
\begin{verbatim}
\mfpicunit 1mm
\mfpic[10]{-3}{7}{-3.5}{5}
  \function{-4.5,4,.1}{x*x}
\endmfpic
\end{verbatim}
The problem was the value of $4.5*4.5 = 20.25$: after multiplying by the
\cs{mfpic} scaling factor, the \cs{mfpicunit} in inches, and the DPI
value, this produces $20.25\times10\times0.03937\times600 > 4783$ pixel
units. The error did not occur at the point of creating the font, but
merely at the point of storing the path in an internal variable for
manipulation and drawing. Thus, the fact that this particular picture
was clipped to a much smaller size for printing did not help.

In \MP{}, the limit on numeric values is only 8 times as high: $32768$.
However, that is independent of printer resolution and is interpreted as
\PS{} points (\TeX{}'s `big points'). At $72$ points to the inch, this
allows figures to be about 12.64 yards (11.56$\,$m).


\section{Special Considerations When Using \CMP{}.}\label{mpconsiderations}

\subsection{Required support}

To use \mfp{} with \MP{}, the following support is needed (besides a
working \MP{} installation):

\medskip\noindent
\begin{tabular}{@{}lp{4.2in}}
plain \TeX{}        &The file \file{epsf.tex} or \file{epsf.sty}\\
\LaTeX{}209         &(No longer supported, but plain \TeX{} methods
                        might work)\\
\LaTeX{}            &The package \prog{graphics} or \prog{graphicx}\\
\pdfLaTeX{}         &The package \prog{graphics} or
                        \prog{graphicx} with option \opt{pdftex}\\
plain \pdfTeX{}     &\raggedright The files \file{supp-pdf.mkii} or
                        \file{supp-pdf.tex} and (possibly)
                        \file{supp-mis.tex}\tabularnewline
In all cases        &\raggedright The files \file{grafbase.mp} and
                        \file{dvipsnam.mp} plus, of course,
                        \file{mfpic.tex} (and \file{mfpic.sty} for
                        \LaTeX{})
\end{tabular}

\medskip
The files \file{grafbase.mp} and \file{dvipsnam.mp} should be in a
directory searched by \MP{}. The remaining files should be in directories
searched by the appropriate \TeX{} variant. If \MP{} cannot find the
file \file{grafbase.mp}, then by default it will try to input
\file{grafbase.mf}, which is generally futile (or fatal).

In case \pdfLaTeX{} is used, the \prog{graphics} package is given the
\opt{pdftex} option. This option requires the file \file{pdftex.def}
which currently inputs one of the \file{supp-pdf} files. Early versions
of \file{supp-pdf.tex} will input \file{supp-mis.tex}. These three files
should be supplied with most \TeX{} installations.%
    \footnote{At this writing, these files, plus a few others,
    can found at \file{CTAN/macros/pdftex/graphics/}.} %
Older versions had some bugs in connection with the \prog{babel}
package. One workaround was to load the \prog{graphics} package and
\mfp{} before \prog{babel}.

If the user loads one of the above required files or packages before the
\mfp{} macros are loaded then \mfp{} will not reload them. \Mfp{} will
load whichever one it decides is required. In the \LaTeXe{} case, \mfp{}
will load the \prog{graphics} package. If the user wishes
\prog{graphicx}, then that package must be loaded before \mfp{}.

\subsection{\CMP{} is not \MF{}}

\PS{} is not a pixel oriented language and so neither is \MP{}. The model
for drawing objects is completely different between \MF{} and \MP{}, and
so one cannot always expect the same results. \CMP{} support in
\mfp{} was carefully written so that files successfully printed
with \mfp{} using \MF{} would be just as successfully printed
using \MP{}. Nevertheless, it frequently chokes on files that make use of
the \cs{mfsrc} command for writing code directly to the \file{.mf} file.
While \file{grafbase.mp} is closely based on \file{grafbase.mf}, some of
the code had to be completely rewritten.

Pictures in \MP{} are stored as (possibly nested) sequences of objects,
where objects are things like points, paths, contours, sub-pictures,
etc. In \MF{}, pictures are stored as a grid of pixels. Pictures that are
relatively simple in one program might be very complex in the other and
even exceed memory allocated for their storage. Two examples are the
\cs{polkadot} and \cs{hatch} commands. When the polkadot space and size
are both too small, a \cs{polkadot}-ed region has been known to exceed
\MP{} capacity, while being well within \MF{} capacity. In \MP{} the
memory consumed by \cs{hatch} goes up in direct proportion to the linear
dimensions of the figure being hatched, while in \MF{} it goes up in
proportion to the area (except in horizontal hatching), and then the
reverse can happen, with \MF{}'s capacity exeeded far sooner that \MP{}'s.

In \MP{} it is important to note that each
prefix modifies the result of the entire following sequence. In essence
prefixes can be viewed as being applied in the opposite order to their
occurrence. Example:
\begin{verbatim}
\dashed\gfill\rect{(0,0),(1,1)}
\end{verbatim}
This adds the dashed outline to the filled rectangle. That is, first the
rectangle is defined, then it is filled, then the outline is drawn in
dashed lines. This makes a difference when colors other than black are
used. Drawing is done with the center of the virtual pen stroked down
the middle of the boundary, so half of its width falls inside the
rectangle. On the other hand, filling is done right up to the boundary.
In this example, the dashed lines are drawn on top of part of the fill.
In the reverse order, the fill would cover part of the dashed outline.

\subsection{Graphic inclusion}\label{graphics}

It may be impossible to completely cater to all possible methods of
graphic inclusions with automatic tests. The macro that is invoked
to include the \PS{} graphic is \cs{setmfpicgraphic} and the user
may (carefully!) redefine this to suit special circumstances. Actually,
\mfp{} runs the following sequence:

\begin{ex}
  \cs{preparemfpicgraphic}\marg{\meta{filename}}\\
  \cs{setmfpicgraphic}\marg{\meta{filename}}\\
  \cs{getmfpicoffset}\marg{\meta{filename}}%
\index{preparemfpicgraphic@\cs{preparemfpicgraphic}}%
\index{setmfpicgraphic@\cs{setmfpicgraphic}}%
\index{getmfpicoffset@\cs{getmfpicoffset}}
\end{ex}

The following are the default definitions for \cs{setmfpicgraphic}:

\medskip\noindent
\begin{tabular}{@{}ll}
plain \TeX{}&
    \cs{def}\cs{setmfpicgraphic}\texttt{\#1}\marg{\cs{epsfbox}%
    \marg{\#1}}\\
\LaTeX{}209& (No longer supported, but likely the plain \TeX{}
definition will be selected.)\\
\LaTeX{}&
    \cs{def}\cs{setmfpicgraphic}\texttt{\#1}\marg{\cs{includegraphics}%
    \marg{\#1}}\\
\pdfLaTeX{}&
    \cs{def}\cs{setmfpicgraphic}\texttt{\#1}\marg{\cs{includegraphics}%
    \marg{\#1}}\\
\pdfTeX{}&
    \cs{def}\cs{setmfpicgraphic}\texttt{\#1}\marg{\cs{convertMPtoPDF}%
    \marg{\#1}\marg{1}\marg{1}}
\end{tabular}

\medskip
Moreover, since \MP{} by default writes files with numeric extensions,
we add code to each figure, so that these graphics are correctly
recognized as \EPS{} or \prog{MPS}. For example, to the figure with
extension \file{.1}, we add the equivalent of one of the following
\begin{itemize}
\item[] \cs{DeclareGraphicsRule}\marg{.1}\marg{eps}\marg{.1}\marg{} in \LaTeXe{}.
\item[] \cs{DeclareGraphicsRule}\marg{.1}\marg{mps}\marg{.1}\marg{} in
\pdfLaTeX{}.
\end{itemize}

After running the command \cs{setmfpicgraphic}, \mfp{} runs
\cs{getmfpicoffset} to store the lower left corner of the bounding box
of the figure in two macros \cs{mfpicllx} and \cs{mfpiclly}. All the
above versions of \cs{setmfpicgraphic} (except \cs{includegraphics})
make this information available; the definition of \cs{getmfpicoffset}
merely copies it into these two macros. What \mfp{} does in the
case of \cs{includegraphics} is to modify (locally) the definition of an
internal command of the \prog{graphics} package so that it copies the
information to those macros, and then \cs{getmfpicoffset} does nothing.
This internal modification is accomplished by the macro
\cs{preparemfpicgraphic}. Changes to \cs{setmfpicgraphic} might require
changing either or both of \cs{preparemfpicgraphic} and
\cs{getmfpicoffset}. All three of these commands are fed the graphic's
file name as the only argument, although only \cs{setmfpicgraphic}
currently does anything with it.

One possible reason for wanting to redefine \cs{setmfpicgraphic} might
be to rescale all pictures. This is \emph{definitely not} a good idea. A
good deal of \mfp{}'s figure placemant code assumes that the size of the
figure is consistent with the coordinate system set up by the \cs{mfpic}
command. With \opt{mplabels} plus \opt{truebbox} it might work, but
(i)~it has \emph{not} been considered in writing the \mfp{} code,
(ii)~it will then scale all the text as well as the figure, and (iii)~it
will scale all line thickness, which should normally be a design choice
independent of the size of a picture. To rescale all pictures, one need
only change \cs{mfpicunit} and rerun \TeX{} and \MP{}.

A better reason might be to allow the conversion of your \MP{} figures
to some other format. Then redefining \cs{setmfpicgraphic}
could enable including the appropriate file in the appropriate format.

The filename argument mentioned above is actually the result obtained by
running the macro \cs{setfilename}\index{setfilename@\cs{setfilename}}.
The command \cs{setfilename} gets two arguments: the name of the \MP{}
output file (set in the \cs{opengraphsfile} command) without extension,
and the number of the picture. The default definition of
\cs{setfilename} merely inserts a dot between the two arguments. That is
\cs{setfilename}\marg{fig}\marg{1} produces \file{fig.1}. You can
redefine this behavior also. Any changes to \cs{setfilename} must come
after the \mfp{} macros are input and before the
\cs{opengraphsfile} command. Any changes to \cs{setmfpicgraphic} must
come after the \mfp{} macros are input and before any \cs{mfpic}
commands, but it is best to place it before the \cs{opengraphsfile}
command.

As \mfp{} is currently written, \cs{setfilename} must be
\emph{completely expandable}, which means it should contain no
definitions, no assignments such as \cs{setcounter}, and no
calculations.%
    \footnote{But appropriate use of \cs{numexpr} (in \eTeX) for
    calculations is probably OK.}
To test whether a proposed definition is completely
expandable, put
\begin{verbatim}
\message{***\setfilename{file}{1}***}
\end{verbatim}
after the definition in a \file{.tex} file and view the result on the
terminal or in the \file{.log} file. You should see only your expected
filename between the asterisks.


\section{\prog{Mfpic} and the Rest of the World.}

\subsection{The literature}

This author has personal knowledge of only one mathematical article
which definitely uses \mfp{} to create diagrams, and that is this
author's joint paper with J.~Duncan and C.~M.~McGregor: \textit{On the
value of pi for norms in $\mathbf{R}^2$} in the College Mathematics
Journal, vol.~35, pages 84--92. Oddly enough, it was McGregor and not me
who chose to use \mfp{} for the illustrations.

There are at least two major publications where \mfp{} has garnered more
than a cursory mention. The most up-to-date is a section in \textit{The
\LaTeX{} Graphics Companion} by Michel Goossens, Sebastian Rahtz and
Frank Mittelbach. It describes a version prior to the introduction of
\MP{} support, but it correctly describes a subset of its current
commands and abilities. \textit{The \LaTeX{} Companion} (Second Edition)
mentions \mfp{}, but only in its annotation of the bibliography entry
for \textit{\TeX{} Unbound} (see below).

The other is \textit{\TeX{} Unbound} by Alan Hoenig, which contains a
chapter on \mfp{}. Unfortunately, it describes a version that was
replaced in 1996 with version 0.2.10.9.  The following summarizes the
differences between the description%
    \footnote{While I'm at it: \textit{\TeX{} Unbound} occasionally
    refers to \mfp{} using a logo-like formatting in which the `MF' is
    in a special font and the `I' is lowered. This `logo' may suggest
    a relationship between \mfp{} and \PiCTeX{}. There is no such
    relationship, and there is no official logo-like designation for
    \mfp{}.} %
found in Chapter 15 and \mfp{} versions 0.2.10.9 through the
current one:

\cs{wedge} is now renamed \cs{sector} to avoid conflict with the \TeX{}
command of the same name. The syntax is slightly different from that
given for \cs{wedge}:
\begin{ex}
  \cs{sector}\marg{(\meta{x},\meta{y}), \meta{radius}, \meta{angle1},
  \meta{angle2}}
\end{ex}

The macro \cs{plr}\marg{(\meta{$r_0$},\meta{$\theta_0$}),%
(\meta{$r_1$},\meta{$\theta_1$}),$\ldots$} is used to convert polar
coordinate pairs to rectangular coordinates, so commands
\cs{plrcurve}, \cs{plrcyclic}, \cs{plrlines} and \cs{plrpoint} were
dropped from \mfp{}. Now use
\begin{ex}
  \cs{curve}\marg{\cs{plr}\marg{(\meta{$r_0$},\meta{$\theta_0$}),%
    (\meta{$r_1$},\meta{$\theta_1$}),$\ldots$}}
\end{ex}
instead of
\begin{ex}
  \cs{plrcurve}\marg{(\meta{$r_0$},\meta{$\theta_0$}),%
    (\meta{$r_1$},\meta{$\theta_1$}),$\ldots$}
\end{ex}
and similarly for \cs{plrcyclic}, \cs{plrlines} and \cs{plrpoint}.

\cs{fill} is now renamed \cs{gfill} to avoid conflict with the \LaTeX{}
command of the same name.

\cs{rotate},  which rotates a following figure about a point, is now
renamed \cs{rotatepath} to avoid confusion with a similar name for a
transformation (see below).

\cs{white} is now renamed \cs{gclear} because \cs{white} is too likely
to be chosen for, or confused with, a color command.

\smallskip
The following affine transform commands were changed from a third person
indicative form (which could be confused with a plural noun) to an
imperative form:

\begin{ex}
\begin{tabular}{@{}ll}
    Old name:           & New name:\\
    \cs{boosts}         & \cs{boost}\\
    \cs{reflectsabout}  & \cs{reflectabout}\\
    \cs{rotatesaround}  & \cs{rotatearound}\\
    \cs{rotates}        & \cs{rotate}\\
    \cs{scales}         & \cs{scale}\\
    \cs{shifts}         & \cs{shift}\\
    \cs{xscales}        & \cs{xscale}\\
    \cs{xslants}        & \cs{xslant}\\
    \cs{xyswaps}        & \cs{xyswap}\\
    \cs{yscales}        & \cs{yscale}\\
    \cs{yslants}        & \cs{yslant}\\
    \cs{zscales}        & \cs{zscale}\\
    \cs{zslants}        & \cs{zslant}
\end{tabular}
\end{ex}

\cs{caption} and \cs{label} are now renamed \cs{tcaption} and
\cs{tlabel} to avoid conflict with the \LaTeX{} commands.

\cs{mfcmd} was renamed \cs{mfsrc} for clarity, and (in version 0.7) a
new \cs{mfcmd} was defined, which is pretty much the same except it appends
a semicolon to its argument.

\smallskip
There is a misprint: \cs{axisheadlin} should be \cs{axisheadlen}.

Finally, the \LaTeX{} template on page 496 is no longer the only
possiblity: recent \mfp{} may be loaded with \cs{usepackage}.

\subsection{Other programs}

There exists a program, \prog{fig2mfpic} that produces \mfp{} code as
output. The code produced (as of this writing) is somewhat old and
mostly incompatible with the description in this manual. Fortunately, it
is accompanied by the appropriate versions of files \file{mfpic.tex} and
\file{grafbase.mf}. Unfortunately, the names conflict with the current
filenames and so they should only be used in circumstances where no
substitution will occur, say in a local directory with the other sources
for the document being produced. Moreover, the documentation in this
manual may not apply to the code produced. However the information in
\textit{\TeX{} Unbound} may apply.

There exist a package, \prog{circuit\_macros}, that can produce a
variety of output formats, one of which is \mfp{} code. One writes a
file (don't ask me what it consists of) and apparently processes it with
\prog{m4} and then \prog{dpic} to produce the output. The \mfp{} code
produced appears to be compatible with the current \mfp{}.

\renewcommand\mfpindexheading{%
  \section{Index of commands, options and parameters.}}
\let\oldcs\cs
\renewcommand{\cs}[1]{\leavevmode\mytt{\llap{\char`\\}#1}}

\InputIfFileExists{mfpic-doc.ind}{}{\mfpindexheading}

\let\cs\oldcs

\columnseprule 0pt
\columnsep 35pt
\twocolumn[\section{List of commands by type.}]
\parindent0pt
\parskip0pt plus .3pt\relax
\makeatletter
\renewcommand\@idxitem{\par\hangindent 10\p@}
\let\item\@idxitem
\makeatother

\subsection{Figures}

  \item \cs{arc}, \pageref{arc}
  \item \cs{axis}, \pageref{axis}
  \item \cs{axisline}, \pageref{axisline}
  \item \cs{belowfcn}, \pageref{belowfcn}
  \item \cs{border}, \pageref{axisline}
  \item \cs{brownianmotion}, \pageref{brownianmotion}
  \item \cs{btwnfcn}, \pageref{btwnfcn}
  \item \cs{btwnplrfcn}, \pageref{btwnfcn}
  \item \cs{cbeziers}, \cs{closedcbeziers}, \pageref{cbeziers}
  \item \cs{chartbar}, \pageref{chartbar}
  \item \cs{circle}, \pageref{circle}
  \item \cs{computedspline},\\
        \cs{closedcomputedspline}, \pageref{computedspline}
  \item \cs{convexcurve}, \cs{closedconvexcurve}, \pageref{convexcurve}
  \item \cs{convexcyclic}, \pageref{convexcurve}
  \item \cs{cspline}, \cs{closedcspline}, \pageref{qspline}
  \item \cs{curve}, \cs{closedcurve}, \pageref{curve}
  \item \cs{cyclic}, \pageref{curve}
  \item \cs{datafile}, \pageref{datafile}
  \item \cs{ellipse}, \pageref{ellipse}
  \item \cs{fcncurve}, \pageref{fcncurve}
  \item \cs{fcnspline}, \pageref{fcnspline}
  \item \cs{function}, \pageref{function}
  \item \cs{ganttbar}, \pageref{chartbar}
  \item \cs{graphbar}, \pageref{chartbar}
  \item \cs{histobar}, \pageref{chartbar}
  \item \cs{levelcurve}, \pageref{levelcurve}
  \item \cs{lines}, \pageref{polyline}
  \item \cs{mfbezier}, \cs{closedmfbezier}, \pageref{mfbezier}
  \item \cs{mfobj}, \cs{mpobj}, \pageref{mfobj}
  \item \cs{parafcn}, \pageref{parafcn}
  \item \cs{periodicfcnspline}, \pageref{fcnspline}
  \item \cs{piewedge}, \pageref{piewedge}
  \item \cs{plrfcn}, \pageref{plrfcn}
  \item \cs{plrregion}, \pageref{belowfcn}
  \item \cs{polygon}, \pageref{polygon}
  \item \cs{polyline}, \pageref{polyline}
  \item \cs{pshcircle}, \pageref{pshcircle}
  \item \cs{qbeziers}, \cs{closedqbeziers}, \pageref{qbeziers}
  \item \cs{qspline}, \cs{closedqspline}, \pageref{qspline}
  \item \cs{rect}, \pageref{rect}
  \item \cs{regpolygon}, \pageref{regpolygon}
  \item \cs{sector}, \pageref{sector}
  \item \cs{tlabelcircle}, \pageref{tlabelellipse}
  \item \cs{tlabelellipse}, \pageref{tlabelellipse}
  \item \cs{tlabeloval}, \pageref{tlabeloval}
  \item \cs{tlabelrect}, \pageref{tlabelrect}
  \item \cs{turtle}, \pageref{turtle}

\subsection{Renderings}

  \item \cs{corkscrew}, \pageref{corkscrew}
  \item \cs{dashed}, \pageref{dashed}
  \item \cs{dotted}, \pageref{dotted}
  \item \cs{doubledraw}, \pageref{doubledraw}
  \item \cs{draw}, \pageref{draw}
  \item \cs{gclear}, \pageref{gclear}
  \item \cs{gclip}, \pageref{gclip}
  \item \cs{gendashed}, \pageref{gendashed}
  \item \cs{gfill}, \pageref{gfill}
  \item \cs{hatch}, \pageref{hatch}
  \item \cs{lhatch}, \pageref{hatch}
  \item \cs{plot}, \pageref{plot}
  \item \cs{plotdata}, \pageref{plotdata}
  \item \cs{plotnodes}, \pageref{plotnodes}
  \item \cs{polkadot}, \pageref{polkadot}
  \item \cs{rhatch}, \pageref{hatch}
  \item \cs{sinewave}, \pageref{zigzag}
  \item \cs{shade}, \pageref{shade}
  \item \cs{tess}, \pageref{tess}
  \item \cs{thatch}, \pageref{thatch}
  \item \cs{xhatch}, \pageref{hatch}
  \item \cs{zigzag}, \pageref{zigzag}

\subsection{Arrows}

  \item \cs{arrow}, \pageref{arrow}
  \item \cs{arrowhead}, \pageref{arrowhead}
  \item \cs{arrowmid}, \pageref{arrowhead}
  \item \cs{arrowtail}, \pageref{arrowhead}

\subsection{Modifying figures}

  \item \cs{bclosed}, \pageref{lclosed}
  \item \cs{cbclosed}, \pageref{cbclosed}
  \item \cs{connect}, \cs{endconnect}, \pageref{connect}
  \item \cs{cutoffafter}, \pageref{cutoffafter}
  \item \cs{cutoffbefore}, \pageref{cutoffafter}
  \item \cs{lclosed}, \pageref{lclosed}
  \item \cs{makesector}, \pageref{makesector}
  \item \cs{parallelpath}, \pageref{parallelpath}
  \item \cs{partpath}, \pageref{partpath}
  \item \cs{qbclosed}, \pageref{cbclosed}
  \item \cs{randomizepath}, \pageref{random}
  \item \cs{randomlines}, \pageref{random}
  \item \cs{reflectpath}, \pageref{shiftpath}
  \item \cs{reverse}, \pageref{reverse}
  \item \cs{rotatepath}, \pageref{shiftpath}
  \item \cs{scalepath}, \pageref{shiftpath}
  \item \cs{sclosed}, \pageref{lclosed}
  \item \cs{shiftpath}, \pageref{shiftpath}
  \item \cs{slantpath}, \pageref{shiftpath}
  \item \cs{subpath}, \pageref{partpath}
  \item \cs{transformpath}, \pageref{shiftpath}
  \item \cs{trimpath}, \pageref{partpath}
  \item \cs{xscalepath}, \pageref{shiftpath}
  \item \cs{xslantpath}, \pageref{shiftpath}
  \item \cs{xyswappath}, \pageref{shiftpath}
  \item \cs{yscalepath}, \pageref{shiftpath}
  \item \cs{yslantpath}, \pageref{shiftpath}

\subsection{Lengths}

  \item \cs{axisheadlen}, \pageref{axisheadlen}
  \item \cs{dashlen}, \pageref{dashlen}
  \item \cs{dotsize}, \pageref{dotsize}
  \item \cs{dotspace}, \pageref{dotsize}
  \item \cs{griddotsize}, \pageref{griddotsize}
  \item \cs{hashlen}, \pageref{hashlen}
  \item \cs{hatchspace}, \pageref{hatchspace}
  \item \cs{headlen}, \pageref{headlen}
  \item \cs{mfpiccaptionskip}, \pageref{mfpiccaptionskip}
  \item \cs{mfpicheight}, \pageref{mfpicheight}
  \item \cs{mfpicunit}, \pageref{mfpicunit}
  \item \cs{mfpicwidth}, \pageref{mfpicheight}
  \item \cs{pointsize}, \pageref{pointsize}
  \item \cs{polkadotspace}, \pageref{polkadotspace}
  \item \cs{shadespace}, \pageref{shadespace}
  \item \cs{sideheadlen}, \pageref{sideheadlen}
  \item \cs{symbolspace}, \pageref{symbolspace}

\subsection{Coordinate transformation}

  \item \cs{applyT}, \pageref{applyT}
  \item \cs{boost}, \pageref{applyT}
  \item \cs{coords}, \cs{endcoords}, \pageref{coords}
  \item \cs{mirror}, \pageref{applyT}
  \item \cs{reflectabout}, \pageref{applyT}
  \item \cs{rotate}, \pageref{applyT}
  \item \cs{rotatearound}, \pageref{applyT}
  \item \cs{scale}, \pageref{applyT}
  \item \cs{shift}, \pageref{applyT}
  \item \cs{turn}, \pageref{applyT}
  \item \cs{xscale}, \pageref{applyT}
  \item \cs{xslant}, \pageref{applyT}
  \item \cs{xyswap}, \pageref{applyT}
  \item \cs{yscale}, \pageref{applyT}
  \item \cs{yslant}, \pageref{applyT}
  \item \cs{zscale}, \pageref{applyT}
  \item \cs{zslant}, \pageref{applyT}

\subsection{Symbols, axes, grids, marks}

  \item \cs{axes}, \pageref{axes}
  \item \cs{axis}, \pageref{axis}
  \item \cs{axismarks}, \pageref{axismarks}
  \item \cs{bmarks}, \pageref{axismarks}
  \item \cs{doaxes}, \pageref{axis}
  \item \cs{grid}, \pageref{grid}
  \item \cs{gridarcs}, \pageref{plrgrid}
  \item \cs{gridlines}, \pageref{grid}
  \item \cs{gridpoints}, \pageref{grid}
  \item \cs{gridrays}, \pageref{plrgrid}
  \item \cs{hgridlines}, \pageref{grid}
  \item \cs{lattice}, \pageref{grid}
  \item \cs{lmarks}, \pageref{axismarks}
  \item \cs{plotsymbol}, \pageref{plotsymbol}
  \item \cs{plrgridpoints}, \pageref{plrgrid}
  \item \cs{plrgrid}, \pageref{plrgrid}
  \item \cs{plrpatch}, \pageref{plrgrid}
  \item \cs{plrvectorfield}, \pageref{vectorfield}
  \item \cs{point}, \pageref{point}
  \item \cs{putmfpimage}, \pageref{putmfpimage}
  \item \cs{rmarks}, \pageref{axismarks}
  \item \cs{tmarks}, \pageref{axismarks}
  \item \cs{vectorfield}, \pageref{vectorfield}
  \item \cs{vgridlines}, \pageref{grid}
  \item \cs{xaxis}, \pageref{axes}
  \item \cs{xmarks}, \pageref{axismarks}
  \item \cs{yaxis}, \pageref{axes}
  \item \cs{ymarks}, \pageref{axismarks}

\subsection{Symbol names}

  \item \gbc{Arrowhead}, \pageref{arrowhead}
  \item \gbc{Asterisk}, \pageref{plotsymbol}
  \item \gbc{Circle}, \pageref{plotsymbol}
  \item \gbc{Crossbar}, \pageref{arrowhead}
  \item \gbc{Cross}, \pageref{plotsymbol}
  \item \gbc{Diamond}, \pageref{plotsymbol}
  \item \gbc{Leftbar}, \pageref{arrowhead}
  \item \gbc{Leftharpoon}, \pageref{arrowhead}
  \item \gbc{Lefthook}, \pageref{arrowhead}
  \item \gbc{Plus}, \pageref{plotsymbol}
  \item \gbc{Rightbar}, \pageref{arrowhead}
  \item \gbc{Rightharpoon}, \pageref{arrowhead}
  \item \gbc{Righthook}, \pageref{arrowhead}
  \item \gbc{SolidCircle}, \pageref{plotsymbol}
  \item \gbc{SolidDiamond}, \pageref{plotsymbol}
  \item \gbc{SolidSquare}, \pageref{plotsymbol}
  \item \gbc{SolidStar}, \pageref{plotsymbol}
  \item \gbc{SolidTriangle}, \pageref{plotsymbol}
  \item \gbc{Square}, \pageref{plotsymbol}
  \item \gbc{Star}, \pageref{plotsymbol}
  \item \gbc{Triangle}, \pageref{plotsymbol}

\subsection{Setting options}

  \item \cs{clearsymbols}, \pageref{clearsymbols}
  \item \cs{clipmfpic}, \pageref{clip}
  \item \cs{mfpicdebugfalse}, \pageref{debug}
  \item \cs{mfpicdebugtrue}, \pageref{debug}
  \item \cs{mfpicdraft}, \pageref{draft}
  \item \cs{mfpicfinal}, \pageref{draft}
  \item \cs{mfpicnowrite}, \pageref{draft}
  \item \cs{mfpreadlog}, \pageref{readlog}
  \item \cs{nocenteredcaptions}, \pageref{centeredcaptions}
  \item \cs{noclearsymbols}, \pageref{clearsymbols}
  \item \cs{noclipmfpic}, \pageref{clip}
  \item \cs{nomplabels}, \pageref{mplabels}
  \item \cs{nooverlaylabels}, \pageref{overlaylabels}
  \item \cs{noraggedcaptions}, \pageref{raggedcaptions}
  \item \cs{notruebbox}, \pageref{truebbox}
  \item \cs{overlaylabels}, \pageref{overlaylabels}
  \item \cs{usecenteredcaptions}, \pageref{centeredcaptions}
  \item \cs{usemetafont}, \pageref{metapost}
  \item \cs{usemetapost}, \pageref{metapost}
  \item \cs{usemplabels}, \pageref{mplabels}
  \item \cs{useraggedcaptions}, \pageref{raggedcaptions}
  \item \cs{usetruebbox}, \pageref{truebbox}

\subsection{Setting values}

  \item \cs{axismargin}, \pageref{axismargin}
  \item \cs{darkershade}, \pageref{darkershade}
  \item \cs{dashlineset}, \pageref{dashlineset}
  \item \cs{dashpattern}, \pageref{dashpattern}
  \item \cs{dotlineset}, \pageref{dashlineset}
  \item \cs{drawpen}, \pageref{drawpen}
  \item \cs{globalsetmfvariable}, \pageref{setmfvariable}
  \item \cs{hatchwd}, \pageref{hatchwd}
  \item \cs{headshape}, \pageref{headshape}
  \item \cs{lightershade}, \pageref{darkershade}
  \item \cs{mfpicnumber}, \pageref{mfpicnumber}
  \item \cs{mfplinestyle}, \pageref{mfplinetype}
  \item \cs{mfplinetype}, \pageref{mfplinetype}
  \item \cs{pen}, \pageref{drawpen}
  \item \cs{penwd}, \pageref{drawpen}
  \item \cs{polkadotwd}, \pageref{polkadotwd}
  \item \cs{setallaxismargins}, \pageref{axismargin}
  \item \cs{setallbordermarks}, \pageref{setaxismarks}
  \item \cs{setaxismargins}, \pageref{axismargin}
  \item \cs{setaxismarks}, \pageref{setaxismarks}
  \item \cs{setbordermarks}, \pageref{setaxismarks}
  \item \cs{setmfboolean}, \pageref{setmfvariable}
  \item \cs{setmfcolor}, \pageref{setmfvariable}
  \item \cs{setmfnumeric}, \pageref{setmfvariable}
  \item \cs{setmfpair}, \pageref{setmfvariable}
  \item \cs{setmfvariable}, \pageref{setmfvariable}
  \item \cs{settension}, \pageref{settension}
  \item \cs{setxmarks}, \pageref{setaxismarks}
  \item \cs{setymarks}, \pageref{setaxismarks}
  \item \cs{shadewd}, \pageref{shadewd}

\subsection{Setting colors}

  \item \cs{backgroundcolor}, \pageref{drawcolor}
  \item \cs{drawcolor}, \pageref{drawcolor}
  \item \cs{fillcolor}, \pageref{drawcolor}
  \item \cs{hatchcolor}, \pageref{drawcolor}
  \item \cs{headcolor}, \pageref{drawcolor}
  \item \cs{mfpdefinecolor}, \pageref{mfpdefinecolor}
  \item \cs{pointcolor}, \pageref{drawcolor}
  \item \cs{tlabelcolor}, \pageref{drawcolor}

\subsection{Defining arrays}

  \item \cs{barchart}, \pageref{barchart}
  \item \cs{bargraph}, \pageref{barchart}
  \item \cs{colorarray}, \pageref{setarray}
  \item \cs{gantt}, \pageref{barchart}
  \item \cs{globalsetarray}, \pageref{setarray}
  \item \cs{histogram}, \pageref{barchart}
  \item \cs{mfpbarchart}, \pageref{barchart}
  \item \cs{mfpbargraph}, \pageref{barchart}
  \item \cs{mfpgantt}, \pageref{barchart}
  \item \cs{mfphistogram}, \pageref{barchart}
  \item \cs{mfppiechart}, \pageref{piechart}
  \item \cs{numericarray}, \pageref{setarray}
  \item \cs{pairarray}, \pageref{setarray}
  \item \cs{patharr}, \cs{endpatharr}, \pageref{patharr}
  \item \cs{piechart}, \pageref{piechart}
  \item \cs{setarray}, \pageref{setarray}

\subsection{Changing behavior}

  \item \cs{coloredlines}, \pageref{coloredlines}
  \item \cs{dashedlines}, \pageref{coloredlines}
  \item \cs{datapointsonly}, \pageref{coloredlines}
  \item \cs{defaultplot}, \pageref{defaultplot}
  \item \cs{everytlabel}, \pageref{everytlabel}
  \item \cs{everymfpic}, \cs{everyendmfpic}, \pageref{everymfpic}
  \item \cs{makepercentcomment}, \pageref{makepercentother}
  \item \cs{makepercentother}, \pageref{makepercentother}
  \item \cs{mfpdatacomment}, \pageref{mfpdatacomment}
  \item \cs{mfpdataperline}, \pageref{mfpdataperline}
  \item \cs{mfpverbtex}, \pageref{mfpverbtex}
  \item \cs{noship}, \pageref{noship}
  \item \cs{pointedlines}, \pageref{coloredlines}
  \item \cs{pointfillfalse}, \cs{pointfilltrue}, \pageref{pointfilltrue}
  \item \cs{reconfigureplot}, \pageref{reconfigureplot}
  \item \cs{resumeshipping}, \pageref{noship}
  \item \cs{setrender}, \pageref{setrender}
  \item \cs{smoothdata}, \pageref{datafile}
  \item \cs{stopshipping}, \pageref{noship}
  \item \cs{tlabeljustify}, \pageref{tlabeljustify}
  \item \cs{tlabeloffset}, \pageref{tlabeloffset}
  \item \cs{tlabelsep}, \pageref{tlabeloffset}
  \item \cs{tlpathjustify}, \pageref{tlpathjustify}
  \item \cs{tlpathsep}, \pageref{tlabeloffset}
  \item \cs{tlpointsep}, \pageref{tlabeloffset}
  \item \cs{unsmoothdata}, \pageref{datafile}
  \item \cs{using}, \pageref{using}
  \item \cs{usingnumericdefault}, \pageref{usingpairdefault}
  \item \cs{usingpairdefault}, \pageref{usingpairdefault}

\subsection{Files and environments}

  \item \cs{closegraphsfile}, \pageref{opengraphsfile}
  \item \cs{mfpframe},  \cs{endmfpframe}, \pageref{mfpframe}
  \item \cs{mfpic}, \cs{endmfpic}, \pageref{mfpic}
  \item \cs{opengraphsfile}, \pageref{opengraphsfile}
  \item \cs{setfilename}, \pageref{setfilename}
  \item \cs{setfilenametemplate}, \pageref{setfilenametemplate}

\subsection{Text}

  \item \cs{axislabels}, \pageref{axislabels}
  \item \cs{plottext}, \pageref{plottext}
  \item \cs{startbacktext}, \pageref{backtext}
  \item \cs{stopbacktext}, \pageref{backtext}
  \item \cs{tcaption}, \pageref{tcaption}
  \item \cs{tlabel}, \pageref{tlabel}
  \item \cs{tlabels}, \pageref{tlabel}

\subsection{Miscellaneous}

  \item \cs{assignmfvalue}, \cs{assignmpvalue}, \pageref{assignmfvalue}
  \item \cs{fdef}, \pageref{fdef}
  \item \cs{getmfpicoffset}, \pageref{getmfpicoffset}
  \item \cs{globalassignmfvalue},\\
        \cs{globalassignmpvalue}, \pageref{assignmfvalue}
  \item \cs{ifmfpmpost}, \pageref{ifmfpmpost}
  \item \cs{mfcmd}, \pageref{mfsrc}
  \item \cs{mflist}, \pageref{mfsrc}
  \item \cs{mfmode}, \pageref{mfmode}
  \item \cs{mfpfor}, \cs{endmfpfor}, \pageref{mfpfor}
  \item \cs{mfpframed}, \pageref{mfpframe}
  \item \cs{mfpicversion}, \pageref{mfpicversion}
  \item \cs{mfpimage}, \cs{endmfpimage}, \pageref{mfpimage}
  \item \cs{mfploop}, \cs{endmfploop}, \pageref{mfploop}
  \item \cs{mfpuntil}, \pageref{mfploop}
  \item \cs{mfpwhile}, \cs{endmfpwhile}, \pageref{mfpwhile}
  \item \cs{mfresolution}, \pageref{mfmode}
  \item \cs{mfsrc}, \pageref{mfsrc}
  \item \cs{mftitle}, \pageref{mftitle}
  \item \cs{newfdim}, \pageref{newfdim}
  \item \cs{newsavepic}, \pageref{newsavepic}
  \item \cs{plr}, \pageref{plr}
  \item \cs{pointdef}, \pageref{pointdef}
  \item \cs{preparemfpicgraphic}, \pageref{preparemfpicgraphic}
  \item \cs{savepic}, \pageref{newsavepic}
  \item \cs{sequence}, \pageref{sequence}
  \item \cs{setmfpicgraphic}, \pageref{setmfpicgraphic}
  \item \cs{store}, \pageref{store}
  \item \cs{tile}, \cs{endtile}, \pageref{tile}
  \item \cs{tmtitle}, \pageref{tmtitle}
  \item \cs{usepic}, \pageref{newsavepic}


\end{document}

