Dear music lover:

Here are the tools to write music with MusicTeX under TeX.
The useful macros are MUSIC*.TEX and the *.STY for LaTeX.
The MUSICTEX.ZIP contains a complete distribution to be un`zipped'
on a PC (with pkunzip). 

Starting from November 1st, 1992, examples are stored in a distinct ZIP file,
namely MUSICEXA.ZIP. Therefore, if you only need the new release, you can take
only MUSICTEX.ZIP, but trying the examples is still strongly recommended to
MusicTeX beginners...

The MUSICTEX.BCK file is a VMS saveset.

Fonts are readily given in 300 dpi as *.PK and *.TFM. The MUSICPK.ZIP
file contains other dpi implementations.

If you have problems printing or viewing the MUSICDOC.DVI file, think of:
 --- transferring all the given *.TFM in the right directory, whose name
     depends on your installation of TeX,
 --- transferring all the given *.PK in the "pk" right directory, whose name
     depends on your installation of TeX,
 --- check whether your installation does not need *.PK files named *.300pk
     rather than *.pk (often in unix systems...)

Other *.TEX files are examples. 

IMPORTANT: use the ftp "ASCII" command to transfer the *.TEX,
the *.STY, the *.PS and the *.MF files. But use the ftp "BINARY" command
to transfer the *.ZIP, *.DVI, *.TFM and *.PK files. If you have a VMS,
use "IMAGE" to transfer the *.BCK file. Transferring the *.ZIP
or the *.BCK provides you with all the needed files, then you
do not need to transfer other files (unless some error happens).

My suggestion is that you tranfer all MUSI*.* files and
the needed fonts (the *.PK and the *.TFM), and
then try to run TeX upon HADAGIO.TEX and to LaTeX MUSICDOC.TEX

MUSICDOC gives a supposedly (LaTeX) complete notice about how to use MusicTeX.
                             Daniel TAUPIN
                             33 (1) 69 41 60 79
                             Physique des Solides
                             F-91405 ORSAY
