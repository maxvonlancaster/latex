%
\def\text#1{\hbox{\rm #1}}%
%\check
\def\ital#1{{\sl #1\/}}%
\def\mutex{M\raise 2pt\hbox{\kern -1pt u\kern -1pt}\TeX}
\def\bslash{{\tt\char'134}}%
\def\|{{\tt\char'174}}%
\def\#{{\tt\char'043}}%
\def\{{{\char'173}}%
\def\}{{\char'175}}%
\def\@{{\char'100}}%
\def\musictex{Music\TeX}%
\def\musicxx{\musictwenty}
\def\keyindex#1{{\tt\bslash #1}\index{#1 \&\&\protect\tt\protect\bslash #1}}
\def\zkeyindex#1{\index{#1 \&\&\protect\tt\protect\bslash #1}}
\def\ixem#1{#1\index{#1}}
\def\itxem#1{\ital{#1}\index{#1}}
\def\aujourdhui{\today}
\def\vboxit#1{\vbox{\hrule\hbox{\vrule\kern3pt
    \vbox{\kern3pt#1\kern3pt}\kern3pt\vrule}\hrule}}
\def\hboxit#1{\hbox{\vrule\vbox{\hrule\kern3pt
    \hbox{\kern3pt\small\it #1\kern3pt}\kern3pt\hrule}\vrule}}
%
%\check
\tracingstats=1
%
\input musicext
\documentstyle[musictex,musictrp,euro92]{article}
%% local commands
\tolerance 10000
\font\mf=logo10 \hyphenchar\mf=-1 \newcommand{\METAFONT}{{\mf METAFONT}}
\begin{document}

\title{\musictex : using \TeX\ to write\\polyphonic or
instrumental music}
\titlehead{\musictex -- Euro\TeX{}~92}

\author{Daniel {\sc Taupin}$^{\dag}$}
\authorhead{Taupin}

\affiliation{$\dag$ Universit\'e de Paris-Sud\\
Laboratoire de Physique des Solides\\
b\^atiment 510\\
F-91405 ORSAY Cedex, France}

\maketitle

\begin{abstract} \musictex\ is a set of \TeX\ or La\TeX\ macros which are
fit to typeset polyphonic, instrumental or orchestral music. It is able to
handle an important number of instruments or voices (up to nine) and staffs
(up to four for each instrument). Many usual ornaments have been provided, including
several note sizes which can handle grace notes or extra music like cadenzas.

Except the risk of typing errors due to a sophisticated set of macros,
the major difficulty still resides in glue and line breaking in the case of
irregular music and slurs.
 \end{abstract}

\begin{resume}{R\'esum\'e}
\musictex\ est un jeu de macro-instructions \TeX\ ou La\TeX\ adapt\'ees \`a
l'\'ecriture de partitions polyphoniques, instrumentales ou orchestrales.
\musictex\ est capable de g\'erer un nombre important d'instruments ou de voix
(jusqu'\`a neuf) et de port\'ees (jusqu'\`a quatre par instrument).
En outre, beaucoup d'ornements sont maintenant disponibles, notamment les notes
de petite taille our indiquer les ornements explicites ou les cadences.

\`A part les risques d'erreurs li\'ees \`a la difficult\'e d'usage de
certaines proc\'edures, la principale difficult\'e est celle des coupures de
lignes dans le cas o\`u d'une musique irr\'eguli\`ere et la r\'alisation de
liaisons ou de phras\'es esth\'etiques. \end{resume}

\begin{keywords}
music typesetting, orchestral music, polyphonic music.
\end{keywords}

\section{What is \musictex\ ?}
 \musictex\ is a set of \TeX\ macros to typeset polyphonic, orchestral or
polyphonic music. Therefore, it is mainly supposed to be used to type wide
scores -- just because true musicians seldom like to have to frequently turn
pages -- and this is not really compatible with La\TeX's standard page
formats, even the {\tt A4.sty} whose {\tt\bslash textheight} and  {\tt\bslash
textwidth} are too small for musician needs.

  However, a La\TeX\ style has been also provided (and it is used for the
typing of the present paper) but this {\tt musictex} style is fit for
musicographic books  rather than for normal scores to be actually played.

 It is to be emphasized that \musictex\ is not intended to be a
compiler which would translate into \TeX\ some \ixem{standard musical
notation}s, nor to decide by itself about aesthetic problems in music typing.
\musictex\ only typesets staves, notes, chords, beams, slurs and ornaments as
requested by the composer. Since it makes very few typesetting decisions,
\musictex\ appears to be a versatile and rather powerful tool. However, due
to the important amount of informations to be provided to the typesetting
process, coding \musictex\ might appear to be awfully complicated, just as
the real keyboard or orchestral music. It should be interfaced therefore by
some pre-compiler in the case of the composer/typesetter wanting aesthetic
decisions to be automatically made by somebody (or something) else.

\section{\musictex\ pricipal features}
\subsection{Music typesetting is two-dimensional}

 Most of the people who just learnt a bit of music at college probably think
that music is a linear sequence of symbols, just as literary texts to be
\TeX-ed. In fact, with the exception of strictly monodic instruments like
most orchestral wind instruments and solo voices, one should be aware that
reading music actually is a matricial operation~: the non-soloist musician
successively reads \ital{columns} of simultaneous notes which he actually plays
if he is a pianist, clavichordist or organist, which he actually reads and
watches if he conducts an orchertra, and which he is supposed to check and
partially play when he is a soloist who wants to play in time with the
accompanying instrument or choir.

 In fact, our personal experience of playing piano and organ as well as
sometimes helping as an alternate Kapellmeister leads us to think that actual
music reading and composing is a slightly more complicated intellectual
process~: music reading, music composing and music thinking seems to be a
three layer process. The musician usually reads or thinks several consecutive
notes (typically a long \ital{beat} or a group of logically connected notes), then
he goes down to the next instrument or voice and finally assembles the whole
to build a part of music lasting roughly a few seconds. Then he handles the
next \ital{beat} or \ital{bar} of his score.

  Thus, it appears that the humanly logical way of coding music consists
in horizontaly accumulating a set of \ital{vertical combs} with
\ital{horizontal teeth} as described in Table \ref{readtable}.

 \begin{table}
 \begin{center}
 \begin{tabular}{|ll|ll|}\hboxit{note sequence one}
  &\hboxit{note seq.\ four}
  &\hboxit{note seq.\ seven}
  &\hboxit{note seq.\ ten}\\\hboxit{note sequence two}
  &\hboxit{note seq.\ five}
  &\hboxit{note seq.\ eight}
  &\hboxit{note seq.\ eleven}\\\hboxit{note seqence three}
  &\hboxit{note seq.\  six}
  &\hboxit{note seq.\  nine}
  &\hboxit{note seq.\  twelve}\\
 \end{tabular}
 \end{center}
 \caption{The order in which a musician reads music}\label{readtable}
 \end{table}

This is the reason
why, in {\bf \musictex} the fundamental \ital{macro} is of the form

\begin{center}
{\tt \bslash notes \dots\ \& \dots\ \& \dots\ \bslash enotes}
\end{center}

\noindent where the character {\tt\&} is used to separate the notes to be
typeset on respective staffs of the various instruments, starting from the
bottom.

In the case of an instrument whose score has to be written with
several staffs, these staffs are separated by the
character \|. Thus, a score written for a keyboard instrument and
a monodic instrument (for example piano and violin)
will be coded as follows~:

\begin{center}
{\tt \bslash notes \dots\ \| \dots\ \& \dots\ \bslash enotes}
\end{center}
\noindent for each column of simultaneous \ital{groups of notes}.
It is worth emphasizing that we actually said \ital{``groups of notes''}:
this means that in each section of the previous macro, the music typesetter is
welcome to insert, not only chord notes to be played at once, but small
sequences of consecutive notes which build something he understands as a
musical phrase. This is why note typing macros are of two kinds in \musictex,
namely the note macros which are not followed by spacing afterwards, and those
which induce horizontal spacing afterwards.

\subsection{The spacing of the notes}

It seems that many books have dealt with this problem.
Although it can lead to interesting algorithms, we think it is
in practice a rather minor one.

In fact, each column of notes has not necessarily the same spacing
and, in principle, this \itxem{spacing} should depend on the shortest
duration of the simultaneous notes. But this cannot be established as a rule,
for at least two reasons~:

\begin{enumerate}
 \item spacing does not depend only of the local notes,
but also on the context, at least in the same bar.
\item in the case of polyphonic music, exceptions can easily be found.
Here is an example~:


 \begin{music}
\computewidths
\def\nbinstruments{1}\relax
\generalmeter{\meterfrac{4}{4}}\relax
\debutextrait
\normal\elemskip=0.110\hsize
\temps\Notes\qsk\qsk\rlap{\hu j}\ql h\enotes
\temps\Notes\hl g\enotes
\temps\Notes\hu k\enotes
\temps\Notes\ql f\enotes
\finextrait
 \end{music}

%\check

\noindent where it can be clearly seen that the half notes at beats 2 and 3
must be spaced as if they were quarter notes since they overlap, which is
obvious only because of the presence of the indication of the \itxem{meter} 4/4.
\end{enumerate}
%

Therefore, we prefered to provide the composer/typesetter with a
set of macros, the spacing of which increases by a factor of $\sqrt 2$
(incidentally, this can be adjusted)~:

\begin{center}
{\tt \bslash notes \dots\ \& \dots\ \& \dots\ \bslash enotes\ \ \%
\hbox to 2.5cm{\rm 1 spatial unit\hss}}
\\
{\tt \bslash Notes \dots\ \& \dots\ \& \dots\ \bslash enotes\ \ \%
\hbox to 2.5cm{\rm 1.4 spacial unit\hss}}
\\
{\tt \bslash NOtes \dots\ \& \dots\ \& \dots\ \bslash enotes\ \ \%
\hbox to 2.5cm{\rm 2 spatial units\hss}}
\\{\tt \bslash NOTes \dots\ \& \dots\ \& \dots\ \bslash enotes\ \ \%
\hbox to 2.5cm{\rm 2.8 spatial units\hss}}
\\{\tt \bslash NOTEs \dots\ \& \dots\ \& \dots\ \bslash enotes\ \ \%
\hbox to 2.5cm{\rm 4 spatial units\hss}}
\\{\tt \bslash NOTES \dots\ \& \dots\ \& \dots\ \bslash enotes\ \ \%
\hbox to 2.5cm{\rm 5.6 spatial units\hss}}
\end{center}

%\check
 The size of the spatial unit (\keyindex{elemskip}) can be freely adjusted.
In addition, \musictex\ provides a means of adjusting the note spacing
according to an average number of elementary spaces within a line (macro
\keyindex{autolines}).
 
\subsection{Music tokens, rather than a readymade generator}
%\check

The tokens provided by \musictex\ are~:

\begin{itemize}
\item the note symbols \ital{without stem}~;
\item the note symbols \ital{with stems, and hooks for eighth notes and
beyond}~;
\item the indications of beam beginnings and beam ends~;
\item the indications of beginnings and ends of ties and slurs~;
\item the indications of accidentals~;
\item the ornaments~: arpeggios, trills, mordents, pinc\'es, turns, staccatos
and pizzicatos, fermatas~;
\item the bars, the meter and signature changes, etc.
\end{itemize}

 Thus, {\tt\bslash wh g} produces an \ital{A (445~Hz)} whose
duration is a \ital{whole note}. In the same way, {\tt\bslash qu~c} produces
a \ital{C (250~Hz approx.)} whose value is a \ital{quarter note with stem
up}, {\tt \bslash cl J} produces a \ital{C (125~Hz approx.)} whose duration
is an \ital{eighth note with stem down}, etc.

 To generate quarter, eighth, sixteenth, etc. chords, the macro
\keyindex{zq} can be used~: it produces a quarter note head whose position is
memorized and recalled when another stemmed note (possibly with a hook) is
coded~; then the stem is adjusted to link all simultaneous notes. Thus, the
perfect C-major chord, i.e.

%\check
{\def\nbinstruments{1}\relax
\cleftoksi={{0}{0}{0}{0}}\generalmeter{}\relax
\debutextrait
\normal
\NOTes\qsk\zq c\zq e\zq g\qu j\enotes
\finextrait}

%\check
\noindent is coded
 \hbox{\tt\bslash zq~c\bslash zq~e\bslash zq~g\bslash qu~j}
or, in a more concise way, \hbox{\tt\bslash zq\{ceg\}\bslash qu~j} (stem up)~:
in fact, single notes are treated\dots\ like one-note chords.

\subsection{Beams}
 \ital{Beams}\index{beams} are generated using macros which define their
beginning (at the current horizontal position), together with their altitude,
their sense (upper of lower), their multiplicity, their slope and their
reference number. This latter feature -- the reference number -- appears to
be necessary, since one may want to write beams whose horizontal extents
overlap~: therefore, it is necessary to specify which beam the notes hang on
and which beam is terminated at a given position.

%\check
\subsection{Setting anything on the score}

A general macro (\keyindex{zcharnote}) provides a means of
putting any sequence of symbols (in fact, some {\tt \bslash hbox\{...\}}) at any
pitch of any staff of any instrument. Thus, any symbol defined in a font
(letters, math symbols, etc.) can be used to typeset music.

\subsection{A simple example}

 Before entering other details, we give below an example of the two first
bars of the sonata in C-major K545 by {\sc\ixem{Mozart}}~:

\begin{music}
\parindent 1cm
\def\nbinstruments{1}\relax
\def\instrumenti{Piano}%
\nbporteesi=2\relax
\generalmeter{\meterfrac{4}{4}}\relax
\debutextrait
\normal
\temps\Notes\ibu0f0\qh0{cge}\tbu0\qh0g|\hl j\enotes
\temps\Notes\ibu0f0\qh0{cge}\tbu0\qh0g|\ql l\sk\ql n\enotes
\barre
\Notes\ibu0f0\qh0{dgf}|\qlp i\enotes
\notes\tbu0\qh0g|\ibbl1j3\qb1j\tbl1\qb1k\enotes
\temps\Notes\ibu0f0\qh0{cge}\tbu0\qh0g|\hl j\enotes
\finextrait
\end{music}
%\check

 The \ital{coding} is set as follows~:

\begin{quote}\begin{verbatim}
\begin{music}
\parindent 1cm
\def\nbinstruments{1}\relax % a single instrument
\def\instrumenti{Piano}%    % whose name is Piano
\nbporteesi=2\relax         % with two staffs
\generalmeter{\meterfrac{4}{4}}\relax % 4/4 meter choosen
\debutextrait               % starting real score
\normal                     % normal 12 pt note spacing
\temps\Notes\ibu0f0\qh0{cge}\tbu0\qh0g|\hl j\enotes
\temps\Notes\ibu0f0\qh0{cge}\tbu0\qh0g|\ql l\sk\ql n\enotes
\barre                      % bar
\Notes\ibu0f0\qh0{dgf}|\qlp i\enotes
\notes\tbu0\qh0g|\ibbl1j3\qb1j\tbl1\qb1k\enotes
\temps\Notes\ibu0f0\qh0{cge}\tbu0\qh0g|\hl j\enotes
\finextrait                 % terminate excerpt
\end{music}
 \end{verbatim}\end{quote}

\begin{itemize}
\item {\tt\bslash ibu0f0} begins an upper beam, aligned on the
\ital{f}, reference number 0, slope 0
\item {\tt\bslash tbu0} terminates this beam before writing
the second \ital{g} by means of {\tt\bslash qh0g}
\item {\tt\bslash qh..} indicates a note hanging on a beam.
 \item {\tt\bslash sk } sets a space between the two quarters at the right
hand, so that the second is aligned with the third eighth of the left hand.
 \item{\tt\bslash qlp} is a  quarter with a point.
 \item{\tt\bslash ibbl1j3} begins
a double beam, aligned on the \ital{C} ({\tt j} at this pitch)
of slope 0.15.
\end{itemize}

\subsection{Signatures}

Signatures are stated either for all instruments, such as~:
\keyindex{generalsignature}{\tt=-2} which sets two flats on each staff, but
this can be partly overrided by \keyindex{signatureii}{\tt=1} which puts one
sharp on the staffs of \itxem{instrument number} 2 (ii). Of course, the
current signature may change at any time as well as meters and clefs.

\subsection{Transposition}

 An important question  is~: \ital{``can \musictex\ transpose a score~?''}.
The answer is now 99.5~\% \ital{yes}. If fact, there is an internal register
named \keyindex{transpose} the default value of which is zero, but it may be
set to any positive of negative reasonable value. Then, it offsets all
symbols pitched with letter symbols by that number of pitch steps. However,
it will neither change the signature nor the local accidentals, and if -- for
example -- you transpose by 1 pitch a piece written in $C$, \musictex\ will
not know whether you want it in $D\flat$, in $D$ or in $D\sharp$. This might
become tricky if accidentals occur within the piece, which might have to be
converted into flats, naturals, sharps or double sharps, depending on the new
choosen signature. To circumvent this trouble, \ital{relative} accidentals
have been implemented, the actual output of which depends of the pitch of
this accidental and of the current signature.\index{relative accidentals}

\subsection{Grace notes and cadenzas}

In addition to its facility of generating either sixteen point or twenty point
staffs with note heads of corresponding size, \musictex\ also allows the user
to type smaller notes, in order to represent either \ital{grace notes},
\ital{cadenzas} or a proposed realization of a \ital{figured bass}. This may
give something like~:

\begin{music}
\def\DS{\hbox{\ds}}
\def\FS{\hbox{\kern 0.3\noteskip\soupir}\kern -0.3\noteskip}
\def\qbl#1#2#3{\ibl{#1}{#2}{#3}\qb{#1}{#2}}%
\def\qbu#1#2#3{\ibu{#1}{#2}{#3}\qh{#1}{#2}}%
\def\nbinstruments{2}%
\generalmeter{\meterfrac{4}{4}}%
\signaturegenerale{0}%
\nbporteesii=1\relax
\nbporteesii=2\relax
\cleftoksi={{6}{0}{0}{0}}
\cleftoksii={{6}{0}{0}{0}}
\etroit     
\debutextrait
\NOTes\soupir&\soupir|\qu g\enotes
% mesure 1
\advance\barno by -1\relax
\barre\NOtes\itenu2J\wh J&\zw N\ibl0c0\qb0e|\qu j\enotes
\notes&\qbl0c0|\noteskip=0.6\elemskip\tinynotesize
\Ibbu1ki2\qh1{kj}\tqh1i\qsk\enotes
\Notes&\qb0e\tbl0\qb0c|\qu j\enotes
\temps\Notes&\ibl0c0\qb0{ece}\tbl0\qb0c|\ql l\sk\ql j\enotes
% mesure 2
\barre\Notes\tten2\wh J&\ql J\sk\ql L|\ppt g\rlap{\qu g}\qbl1e0\relax
       \zq c\qb1e\zq c\qb1e\relax
       \zq c\tbl1\rlap{\qb1e}\ \ \ccu h\enotes
\temps\Notes&\ql N\sk\pt L\ibl0L{-4}\qb0L|\ibl1e0\zq c\rlap{\qb1e}\cu g\relax
       \zq c\rlap{\qb1e}\raise\Interligne\DS \rlap{\qu g}\qb1g\enotes
\notes&\sk\tbbl0\tbl0\qb0J|\tbl1\zq c\qb1e\enotes
\finextrait
\end{music}

\subsection{Selecting special instrument scores}

Another question is~: \ital{``can I write an orchestral score and extract the
separate scores for individual instruments~?''} The answer is 95~\%
\ital{yes}~: in fact, you can define your own macros {\tt \bslash
mynotes...\bslash enotes}, {\tt \bslash myNotes...\bslash enotes} with as
many arguments as there are in the orchestral score (hope this is less or
equal to 9, but \TeX perts know how to work around) and change its definition
depending on the selected instrument (or insert a test on the value of some
selection register). But the limitation is that the numbering of instruments
may change, so that {\tt \bslash signatureiii} may have to become {\tt
\bslash signaturei} if instrument $iii$ is alone. But, in turn, this is not a
serious problem for average \TeX\ wizard apprentices.

 \section{How to get it}

 The whole \itxem{distribution} fits into a single 1.2Mbyte or
1.44Mbyte diskette. It can also be obtained through an \ital{anonymous ftp}
at {\tt rsovax.ups.circe.fr} (130.84.128.100), after selecting the
subdirectory {\tt [.musictex]}. All sources (including fonts) are provided,
either separately or ``zipped'' or as VMS `savesets''.

%\check
\section{Implementation, restrictions and enhancement possibilities}

The macroinstruction file \musictex\ contains approximately 2500 lines of
code, that is 80~000 bytes approximately. This requires your score to be
compiled by the most extended versions of \TeX\ (65~000 words of working
memory). In desperate situations, we recommend using the
``Big\TeX''\index{Big\protect\TeX} processors which, unfortunately, perform
a great deal of disk input/outputs (on PCs) which make them awfully slow.

In particular, the number of registers it uses can hinder its compatibility
with some La\TeX\ styles or with La\TeX\ itself in case of restricted memory
availability\index{La\protect\TeX}.

\subsection{Recent easy enhancements}

Many enhancements have been asked for, and this is a proof thet \musictex\ is
considered as useful by many people. Some of these enhancements which seemed
hard were in fact rather easy to implement, for example small notes to represent
grace notes and cadenzas. But others may induce heavy problems, for example
the need of having \ital{nice} slurs and ties.

\subsection{The tie/slur problem}

While typesetting notes and even beams is a rather simple problem because is
is a \ital{local typesetting}, ties and slurs are much more difficult to handle.

Or course there is little problem in case of a typesetter wanting a slur or a
tie binding two consecutive notes, not separated by a bar. In practice this
\ital{very restricted} use of slurs or ties can easily be solved by putting
some symbols extracted from the {\tt slur16} or {\tt slurn16}/{\tt slurn20}
fonts somewhere on the staffs using the general use \keyindex{zcharnote}
macro.

 But serious music typesetters or composers know that many ties are supposed
to link notes which are on both sides of a bar, which is a likely place to
insert line breakings, so that the \itxem{tie} coding must have various
versions and sizes to resist that possible line breaking. What has been said
about ties is still more serious in the case of \itxem{phrasing slurs} which
may extend over several bars, lines and sometimes pages. In this case, their
shape is not only a question of producing a long curved symbol of nice
looking shape, it also has to cope with \itxem{glue}. An then the worst is
that music way of typing does not accept \itxem{ragged lines} but equal
length lines, even for the last line of a music piece. Thus, long distance
slurs and ties need to be cut into separate parts (beginning, continuing(s),
endings) which \TeX\ can only link using \ital{horizontal line overlaps}
or \keyindex{leaders} to insure slur continuity over this unavoidable glue.



Therefore and up to now, ties and slurs have been implemented in a way which
may look rather ugly, but we think it is the only way of implementing
\ital{in one pass} ties and slurs which run \ital{across glue}. The principle
is to have tie/slur symbols with a rather long part of horizontal stuff.
Then, at each time a glue occurs and at each time a group of notes is coded
while a slur or tie is pending, an \keyindex{hrule} is issued which overlaps
the preceeding tie/slur symbol so that the final output seems to contain a
continuous line. Unfortunately, this is possible only in the glue expansion
direction, namely in the horizontal direction.

  It could be requested -- and this is thought of -- to have variable size
initial and final curved slur symbols which the user would choose according
to his intention to have short or long range slur symbols. This has not been
\ital{yet} implemented for several reasons~:

 \begin{itemize}

   \item The author's natural lazyness (!)

   \item More seriously~: this would require using some more registers (at
least a dozen) to record the initial sizes (horizontal and vertical) of this
symbol \ital{for each slur/tie} in order to make adequate links over glue and
to close it with the symmetrical symbol. Unfortunately, \TeX\ registers are
not numerous (256 of each kind) and we are afraid such a feature would make
\TeX\ stupidly crash even when typesetting reasonable scores.

  \item We do not think it wise to introduce in \musictex\ itself a great
number of macros which would be pooly used by most users~: the reason is that
\TeX\ memory is hardly limited and that unused macros may occupy some \TeX\
storage which could make things crash because of {\tt TeX capacity
exceeded}...

 \end{itemize}  

\section{Acknowledgements}

The idea of implementing the present package is due to the previous work
(\mutex)\index{mutex\&\%\protect\mutex} of Andrea {\sc Steinbach} and Angelika
{\sc Schofer}[1]. This work
provided the basis of the Metafont codes and some line breaking procedures,
which both are still used here... with 99\% corrections and
updates.\index{Schofer, A.}\index{Steinbach, A.}


\section{Some examples}

Many examples can be typeset from the \musictex\ distribution, but they are
not included here because of their need to be compiled with \TeX\ rather than
La\TeX\ and because they need output paper of the extended
291$\times$297~mm$^2$ european size. However we chose to produce a small type
size version of {\sc Haydn}'s \ital{aria} from the Creation and a piece of a
personal composition heavily using beams.

\vspace{.5cm}
\begin{thebibliography}{Taupin 92}
\bibitem[1]{} Andrea {\sc Steinbach} and Angelika
{\sc Schofer}, {\em Theses} (1987, 1988),
Rheinische Friedrich-Wilhelms Universit\"at, Bonn, Germany.

\end{thebibliography}

\newpage

\begin{center}
{\Large\bf Aria No. 24}
\\{\large\bf (The Creation)}
\\{\large Joseph {\sc Haydn}}
\\Transcription for organ and tenor, D. {\sc Taupin} (1990)
\end{center}

\begin{music}
\musicsize=16\relax
\def\nbinstruments{2}%
\generalmeter{\meterfrac{4}{4}}%
\signaturegenerale{0}%
\nbporteesii=2\relax
\def\qbl#1#2#3{\ibl{#1}{#2}{#3}\qb{#1}{#2}}%
\def\qbu#1#2#3{\ibu{#1}{#2}{#3}\qh{#1}{#2}}%
\def\DS{\hbox{\ds}}\def\FS{\hbox{\kern 0.3\noteskip\soupir}\kern -0.3\noteskip}
\cleftoksi={{6}{0}{0}{0}}
\cleftoksii={{6}{0}{0}{0}}
\debutmorceau
\autolines {16}24\relax
\NOTes\soupir&\rlap{\rmidtwotext{\bf II}}\relax
\soupir\relax|\qu g\enotes
% mesure 1
\advance\barno by -1\relax
\advance\barsinline by -1\relax
\barre\NOtes\itenu2J\wh J&\zw N\ibl0c0\qb0e|\itenl0j\ibu1l0\qh1j\enotes
\notes&\qbl0c0|\nbbu1\nbbbu1\tten0\qh1{jkj}\tbu1\qh1i\enotes
\Notes&\qb0e\tbl0\qb0c|\qu j\enotes
\temps\Notes&\ibl0c0\qb0{ece}\tbl0\qb0c|\ql l\sk\ql j\enotes
% mesure 2
\barre\Notes\tten2\wh J&\ql J\sk\ql L|\ppt g\rlap{\qu g}\qbl1e0\relax
       \zq c\qb1e\zq c\qb1e\relax
       \zq c\tbl1\rlap{\qb1e}\ \ \ccu h\enotes
\temps\Notes&\ql N\sk\pt L\ibl0L{-4}\qb0L|\ibl1e0\zq c\rlap{\qb1e}\cu g\relax
       \zq c\rlap{\qb1e}\raise\Interligne\DS \rlap{\qu g}\qb1g\enotes
\notes&\sk\tbbl0\tbl0\qb0J|\tbl1\zq c\qb1e\enotes
% mesure 3
\autolines {14}24\relax
\barre\NOtes\itenl2G\wh G&\zw N\raise 3.5\Interligne\ds
|\rlap{\cl f}\itenl0k\ibu1m0\qh1k\enotes
\notes&\qbl0b0|\nbbu1\nbbbu1\tten0\qh1{klk}\tbu1\qh1{^j}\enotes
\Notes&\zq d\qb0f\tbl0\qb0b|\qu k\enotes
\temps\Notes&\ibl0d0\zq d\qb0{fb}\zq d\qb0f|\qu m\sk\pt k\qbu1k{-4}\enotes
\notes&\tbl0\qb0b|\sk\tbbu1\tbu1\qh1i\enotes
% mesure 4
\barre\Notes\tten2\wh G&\ql G\sk\ql I|\rlap{\qupp g}\ibl1c0\qb1g\relax
       \zq{bd}\qb1f\zq{bd}\qb1f\relax
       \zq{bd}\tbl1\rlap{\qb1f}\ \ \ccu h\enotes
\temps\Notes&\ql K\sk\pt I\qbl0I{-4}|\ibl1d0\rlap{\qb1b}\cu g\relax
       \zq{bd}\rlap{\qb1f}\raise\Interligne\DS \rlap{\qu g}\qb1g\enotes
\notes&\sk\tbbl0\tbl0\qb0G|\tbl1\zq{bd}\qb1f\enotes
% mesure 5
\barre\Notes\hu J&\rlap{\lhu J}\ibl0M0\qb0J\zq N\qb0c\zq N\qb0c\tbl0\zq N\qb0c\relax
      |\rlap{\hl e}\qu j\sk\qbu1l{-4}\tbu1\qh1j\enotes
\temps\Notes\hu K&\rlap{\lhu K}\ibl0M0\qb0K\zq N\qb0b\zq N\qb0b\tbl0\zq N\qb0b\relax
      |\rlap{\hl f}\ibu1k0\qh1{ikm}\tbu1\qh1k\enotes
%\check
% mesure 6
\barre\Notes\wh L&\zw N\raise 3.5\Interligne\ds\qbl0c0\qb0e\relax
      |\rlap{\hl g}\ppt j\qu j\enotes
\notes&\tbl0\qb0c|\sk\ccu l\enotes
\temps\Notes&\ibl0c0\qb0{ece}|\rlap{\hl g}\qu n\sk\raise 2\Interligne\DS\enotes
\notes&\tbl0\qb0c|\ibbu1m{-3}\qh1m\tbu1\qh1l\enotes
%\check
% mesure 7
\barre\Notes\wh M&\zw a\raise 3.5\Interligne\ds\qbl0d0\qb0f\tbl0\qb0d\relax
      |\rlap{\hlp h}\qu k\sk\qu m\enotes
\temps\Notes&\ibl0d0\qb0f|\pt o\qbu1o{-3}\enotes
\notes&\zq d\qb0f\sk
\zq d\qb0f\sk\tbl0\zq d\qb0f|\sk\tbbu1\tbu1\qh1n\relax
      \ibbu1m{-3}\rlap{\raise -\Interligne\qp}\qh1{mlk}\tbu1\qh1j\enotes
%\check
% mesure 8
\barre\Notes\hu G&\lcharnote c{\bf I}\relax
\pz b\ibl0M3\qb0N\pz d\qb0b\pz f\qb0d\pz d\tqb0b|\ql i\rlap{\uptext{\bf I}}\sk\ds\ppz p\zq{km}\cl p\enotes
\temps\Notes\hu G&\ibl0M3\pz b\qb0N\pz e\qb0c\pz g\qb0e\pz e\tqb0c\relax
        |\zq{jl}\ql q\sk\ds\ppz l\zq{gj}\cl l\enotes
%\check
% mesure 9
\barre\Notes\hu G&\pz b\ibl0M3\qb0N|\zq{gi}\ql k\enotes
\notes&\nbbl0\qb0b\tqb0c\enotes
\zglu\Notes&\pz f\ibl0d{-4}\qb0d\pz d\tqb0b|\ds\ppz p\zq{km}\cl p\enotes
\temps\Notes\hu G&\pz b\ibl0M3\qb0N|\ppz q\ibl1m3\zq l\qb1q\enotes
\notes&\nbbl0\qb0c\tqb0d|\nbbl1\qb1q\tqb1p\enotes
\temps\notes&\pz g\ibl0e{-4}\qb0e\sk\pz e\tqb0c|\ibbl1p0\qb1{qpq}\tqb1s\enotes
%\check
% mesure 10
\barre\Notes\hu G&\zq{Nb}\ql d|\pz t\ibl1o{-3}\qb1r\enotes
\notes&|\nbbl1\qb1n\tqb1{^m}\enotes
\zglu\notes&\soupir|\ibbl1m0\qb1{nmn}\tqb1m\enotes
\temps\Notes\hpause&\ibl0I6\pz J\qb0G|\pz p\cl n\enotes
\notes&\nbbl0\qb0N\tqb0{^M}|\ibbu1g{-3}\qh1g\tqh1{^f}\enotes
\notes&\ibbl0N0\qb0{NMN}\tqb0M|\ibbu1g0\qh1{gfg}\tqh1f\enotes
%\check
\suspmorceau
\def\nbinstruments{4}%
\nbporteesiii=0\relax
% mesure 11
\reprmorceau
\autolines {11}23\relax
\NOtes\qu G&\ql N|\st n\qu g&&\hpause\enotes
\zglu\NOtes\soupir&\rlap{\rmidtwotext{\bf II}}\soupir|\zq d\qu{=f}&&\enotes
\temps\NOtes\hpause&\hpause|\zq c\qu e&&\soupir\enotes
\zglu\NOtes&|\zq b\qu d&Mit~&\ilegu0p\qu g\enotes
% mesure 12
\barre\Notes&\qu J|\zw N\zq c\ibu0e0\qh0e\zq c\qh0e&W\"urd~&\ql j\enotes
\Notes&\soupir|\zq c\qh0e\zq c\tqh0e&und~&\ql j\enotes
\temps\Notes\pause&\hpause|\zq c\ibu0e0\qh0e\zq c\qh0e&Ho-&\ql l\enotes
\Notes&\zcharnote C{\bf I}|\zq c\qh0e\zq c\tqh0e&heit~&\ql j\enotes
% mesure 13
\autolines {12}23\relax
\barre\Notes&\qu C|\zw N\zq c\ibu0e0\qh0e\zq c\qh0e&an-&\qup g\enotes
\Notes&\qu E|\zq c\qh0e\zq c\tqh0e&\sk ge-&\sk\cu h\enotes
\temps\Notes\pause&\qu G|\zq c\ibu0e0\qh0e\zq c\qh0e&tan,~&\tleg0\qu g\enotes
\Notes&\ibu1E{-3}\qhp1E|\zq c\qh0e&mit~&\ilegu0p\qu g\enotes
\notes&\sk\tbbu1\tqh1C|\zq c\tqh0e&&\enotes
% mesure 14
\barre\Notes&\qu G|\zw N\ibu0e0\zq{bd}\qh0f\zq{bd}\qh0f\relax
    &Sch\"on-&\ql k\enotes
\Notes&\soupir|\zq{bd}\qh0f\zq{bd}\tqh0f&heit,~&\ql k\enotes
\temps\Notes\pause&\hpause|\ibu0e0\zq{bd}\qh0f\zq{bd}\qh0f&St\"ark~&\ql m\enotes
\Notes&|\zq{bd}\qh0f\zq{bd}\tqh0f&und~&\ibu2k{-4}\qh2k\tqh2i\enotes
\suspmorceau
\end{music}
%\removelastskip\rightline{\sl\aujourdhui}
\newpage

\def\testpgs{4}%
\begin{music}
%
\resettens
\musicsize=16\relax
\nbporteesi=2\relax
\def\nbinstruments{1}%
\global\cleftoksi={{6}{0}{0}{0}}%
%
\signaturegenerale{-3}\relax % trois bemols a la clef
\def\quatretemps{\generalmeter{\meterfrac{4}{2}}}%
%\check
\def\instrumenti{Piano}
\def\fff{{f}\kern -1pt{f}\kern -1pt{f}}%
\def\ff{{f}\kern -1pt{f}}%
% formes frequentes
%
%\check
%
\def\Tqbbh{\Tqbbu}%
\def\Tqbh{\Tqbu}%
%
%\check
% octave basse en blanche
\def\bohl#1{\zh{`#1}\hl{'#1}}%
\def\bohu#1{\zh{`#1}\hu{'#1}}%
\def\sohl#1{\zh{'#1}\hl{`#1}}%
\def\sohu#1{\zh{'#1}\hu{`#1}}%
%
\def\bozh#1{\zh{`#1'#1}}%
\def\bozq#1{\zq{`#1'#1}}%
\def\boql#1{\zq{`#1}\ql{'#1}}%
\def\boqu#1{\zq{`#1}\qu{'#1}}%
% octave sup. en croches u
\def\soqh#1{\zq{'#1}\qh0{`#1}}%
\def\soqb#1{\zq{'#1}\qb0{`#1}}%
\def\soqu#1{\zq{'#1}\qu{`#1}}%
\def\soql#1{\zq{'#1}\ql{`#1}}%
%\check
%
\def\interfacteur{11}%
\computewidths
\staffbotmarg=5\Interligne %
\def\sfhu#1{\lsf #1\hu #1}%
\def\sfwu#1{\lsf #1\wh #1}%
\def\sfhl#1{\usf #1\hl #1}%
\def\sfwl#1{\usf #1\wh #1}%
\def\Sfwu#1{\zw{'#1}\usf{#1}\wh{!#1}}%
\def\Sfwp#1{\pt{'#1}\zw{#1}\pt{!#1}\usf{#1}\wh{#1}}%
%
\parindent 1cm\debutmorceau
%
\autolines{28}14\relax
\znotes\zmidtwotext{\kern -3.5mm\ppff ff}|\raise 9\Interligne
\llap{\moyen
   appassionnato\kern -1.5cm}\uptext{\kern -1mm\rm(\metron{\qu}{80})}\enotes 
% 
%\check
\newif\ifsupa
\supafalse
\def\accord#1#2#3#4#5#6#7{\notes
\raise -2\Interligne\llap{\PED\sk}\Ibbu0{#1}{#3}3\qh0{!`#1!#1#2}\tqh0{#3}\relax
|\pt{#5}\pt{#6}\pt{#7}\zw{#5}\zw{#6}\zw{#7}\relax
\usf{!'#5}\pt{!'#5}\wh{!'#5}\Ibbu0{!#5}{!#7}3\qh0{!#5#6}\tqh0{#7}\enotes
%\check 1
\temps\notes
\Qqbbu{!#4}{!'#2}{!'#3}{!'#4}|\Qqbbu{!'#5}{!'#5}{!'#6}{!'#7}\enotes
%\check 1
\temps\notes
|\rlap{\Qqbbl{!''#2}{!''#3}{!''#4}{!'''#2}}\relax
\Qqbbu{!''#5}{!''#5}{!''#6}{!''#7}\enotes
%\check 1
\temps\notes|\rlap{\Qqbbl{!'''#3}{!'''#2}{!''#4}{!''#3}}\relax
\Qqbbu{!'''#5}{!''#7}{!''#6}{!''#5}\enotes
%\check 1
\temps\notes
\ifsupa|\rlap{\Qqbbl{!'#4}{!'#4}{!'#3}{!'#2}}\relax
\else\ibbu0{!''#3}{-9}\qh0{!''#3#2`#4}\tqh0{#3}|\fi
\Qqbbu{!''#5}{!'#7}{!'#6}{!'#5}\enotes
%\check 1
\temps\notes\raise -2\Interligne
\rlap{\sk\sk\DEP}\Ibbu0{!#4}{!`#4}3\qh0{!#4#3#2}\tqh0{!`#4}\relax
|\Ibbu0{!'#5}{!#5}3\qh0{!'#5`#7#6}\tqh0{#5}\enotes
}%
%
%\check
\accord CEGJNce\relax
% mesure 2
\autolines{25}1{\testpgs}\relax
\barre\accord CFHJacf\relax
% mesure 3
\barre\supatrue\accord CFHJcfh\relax
% mesure 4
\barre
\accord CFHJdfh\relax
% mesure 5
\def\Accord#1#2#3#4#5#6#7{\notes
\raise -2\Interligne\llap{\PED\sk}\Ibbu0{#1}{#3}3\qh0{`#1!#1#2}\tqh0{#3}\relax
|\pt{#5}\pt{#6}\pt{#7}\zw{#5#6#7}\relax
\usf{!'#5}\whp{!'#5}\Tqbbu{!#5}{!#6}{!#7}\enotes
%\check 1
\temps\notes\Qqbbu{!#4}{!'#2}{!'#3}{!'#4}\relax
|\Qqbbu{!'#5}{!'#5}{!'#6}{!'#7}\enotes
%\check 1
\temps\notes|\rlap{\Qqbbl{!''#2}{!''#3}{!''#4}{!'''#2}}\relax
\Qqbbu{!''#5}{!''#5}{!''#6}{!''#7}\enotes}\relax
%
%\check
\barre\Accord CEGJNce\relax
\temps\notes|\qsk\octfin{22}{12}\rlap{\Qqbbl gjln}\Qqbbu nnqs\enotes
\temps\notes|\rlap{\Qqbbl qnlj}\Qqbbu usqn\enotes
\temps\notes|\rlap{\Qqbbl gecN}\Qqbbu nljg\enotes
% mesure 6
\barre\notes\Qqbbu ecNL|\Qqbbu nljg\enotes
\temps\notes\Qqbbu cNLJ|\Qqbbu ljge\enotes
\temps\notes\Qqbbu NLJG|\zcharnote{a}{\DIMin}\Qqbbu jgec\enotes
\temps\notes\Qqbbu LJGE|\Qqbbu gecN\enotes
\temps\notes\Qqbbu JGEC|\Qqbbu ecNL\enotes
\temps\notes\boqu C\DEP\qsk|\qu J\enotes
% mesure 7
\barre\supafalse\znotes\zmidtwotext{\llap{\ppff ff}}\enotes
\accord CEGJNce\relax
% mesure 8
\barre\accord CFHJacf\relax
\suspmorceau



\end{music}
\end{document}

