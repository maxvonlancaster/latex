% E  Sample TeX file demonstrating how to separate colors using CMYK-HAX
%    macros -- outline graphics + text
% P  Przyk/ladowy plik TeX-owy demonstruj/acy separacj/e kolor/ow za pomoc/a
%    makr pakietu CMYK-HAX -- grafika obwiedniowa + tekst
%
\input epsf     % DVIPS standard distribution
\input colordvi % DVIPS standard distribution
\input cmyk-hax

\newif\ifPolish
%\Polishtrue     % true for Polish text,
\Polishfalse  % else English text
%%
\ifx\unknown\separate\def\separate{\COLOR}\fi
\ifx\unknown\separate\def\separate{\CYAN}\fi
\ifx\unknown\separate\def\separate{\MAGENTA}\fi
\ifx\unknown\separate\def\separate{\YELLOW}\fi
\ifx\unknown\separate\def\separate{\BLACK}\fi

\def\COLOR{}
\def\CYAN{\projectCMYK\cyan}
\def\MAGENTA{\projectCMYK\magenta}
\def\YELLOW{\projectCMYK\yellow}
\def\BLACK{\projectCMYK\black}

\delblacktrue\separate

\nopagenumbers
\ifPolish
\font\bf plssdc10 at 20pt \bf
\font\rm plss8
\prefixing
\else
\font\bf cmssdc10 at 20pt \bf
\font\rm cmss8
\fi

\baselineskip 28pt
\hsize 100mm

\setbox1 \hbox{\epsfxsize\hsize \epsffile{fountain.eps}}

\hbox{\rlap{\copy1}%
\vbox to \ht1{\vss
\textBlack
\ifPolish
\centerline{Pan Sobieski mia/l trzy pieski:}
\centerline{\Red{czerwony}, \Green{zielony}, \Blue{niebieski}.}
\else
\centerline{All words tell the genuine true:}
\centerline{\Red{red}, and \Green{green}, and even \Blue{blue}.}
\fi\vss}}
\nointerlineskip\vskip 5mm
\centerline{\CMYKlabels\rm}

\end
