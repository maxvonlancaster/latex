%
%    manual.tex v2.14
%
%    LaTeX source file for the manual for the Devanagari for TeX" package.
%    Copyright (C) 1991-2003 University of Groningen, The Netherlands
%
%    Author     : Anshuman Pandey <apandey@u.washington.edu>
%    $Id: manual.tex,v 1.6 2006/05/29 09:00:12 icebearsoft Exp $
%
%    This program is free software; you can redistribute it and/or modify
%    it under the terms of the GNU General Public License as published by
%    the Free Software Foundation.
%
%    This program is distributed in the hope that it will be useful,
%    but without any warranty; without even the implied warranty of
%    merchantability or fitness for a particular purpose. See the
%    GNU General Public License for more details.
%

\documentclass[10pt]{article}
\def\DevnagVersion{new}
\usepackage{geometry,fancyhdr,multicol,mflogo,textcomp,parskip,array,color}
\usepackage{devanagari}
\renewcommand{\rmdefault}{ptm}

\geometry{paperwidth=8.5in,
          paperheight=11in,
          top=.875in,
          bottom=1in,
          left=1.25in,
          right=1.25in,
          headheight=0pt,
          headsep=0pt,
          }

\usepackage{hyperref}

% High resolution pk fonts for pdfTeX
\ifx\pdfpkresolution\undefined \else
    \pdfpkresolution 600
\fi

\newcommand{\moddate}{27 May 2006}
\newcommand{\version}{2.14}
\renewcommand{\headrulewidth}{0pt}
\setcounter{secnumdepth}{3}
\setcounter{tocdepth}{3}
\setlength{\columnseprule}{0pt}

\newcommand{\devnag}{Devan\=agar{\=\i}}

\def\diatop[#1|#2]{{\leavevmode\setbox1=\hbox{{#1{}}}\setbox2=\hbox{{#2{}}}%
    \dimen0=\ifdim\wd1>\wd2\wd1\else\wd2\fi%
    \dimen1=\ht2\advance\dimen1by-1ex%
    \setbox1=\hbox to1\dimen0{\hss#1\hss}%
    \rlap{\raise1\dimen1\box1}%
    \hbox to1\dimen0{\hss#2\hss}}}%

\def\underrng #1{\oalign{#1\crcr\hidewidth
     \vbox to.2ex{\hbox{\char"17}\vss}\hidewidth}}

\clubpenalty 10000
\widowpenalty 10000

\begin{document}
\pagestyle{fancy}
\fancyhf{}
\fancyfoot[C]{\thepage}

\fontsize{10}{13}\selectfont

\title{{\LARGE \bfseries \devnag{} for \TeX{}} \\ Version \version{}}
\author{Anshuman Pandey}
\date{\large \moddate{}}
\maketitle

\tableofcontents
%\newpage

\listoftables

\section{Introduction}

The \textit{\devnag{} for \TeX{}} (\textsf{devnag}) package provides a
way to typeset high-quality \devnag{} text with \TeX{}.
\devnag{} is a script used for writing and printing Sanskrit and
a number of languages in Northern and Central India such as Hindi and
Marathi, as well as Nepali. The \textsf{devnag} package was originally
developed in May 1991 by Frans Velthuis for the University of Groningen,
The Netherlands, and it was the first system to provide support for
the \devnag{} script for \TeX{}.

Several individuals have contributed to the \textsf{devnag} package
over the years. Kevin Carmody proposed a method for managing variant
glyphs. Marc Csernel revised the preprocessor to handle standard
\LaTeX{} commands. Richard Mahoney generated PostScript Type~1
versions of the \devnag{} fonts. These fonts were later (in version~2.14) replaced with optimized 
Type~1 fonts created by Karel P\'{\i}\v{s}ka. Fran\c{c}ois Patte enhanced the \LaTeX{} package by 
introducing
a feature to produce citations in \devnag{}. Zden\v{e}k Wagner greatly improved the \LaTeX{}
package by revising macro definitions and catcodes, eliminating
conflicts with other packages, and by introducing support for
section headings and captions in \devnag{}. Rob Adriaanse, Hans Bakker,
Roelf Barkhuis, and Henk van Linde provided advice and support
to Frans Velthuis when this package was being developed.

The \textsf{devnag} package is presently maintained by the following
individuals:

\begin{quote}
\begin{tabular}{ll}
% Frans Velthuis   & \verb+velthuis@rc.rug.nl+ \\
John Smith         & \verb+jds10@cam.ac.uk+ \\
Anshuman Pandey    & \verb+apandey@u.washington.edu+ \\
Dominik Wujastyk   & \verb+d.wujastyk@ucl.ac.uk+ \\
%Fran\c{c}ois Patte & \verb+patte@math-info.univ-paris5.fr+ \\
Zden\v{e}k Wagner  & \verb+wagner@cesnet.cz+ \\
Kevin Carmody      & \verb+i@kevincarmody.org+
\end{tabular}
\end{quote}


\section{Project Information}

The \textit{\devnag{} for \TeX{}} package is now a project
officially housed at Sarovar. The homepage is

\begin{quote}
\texttt{http://devnag.sarovar.org/}
\end{quote}

This package is available from the project homepage at Sarovar and
from the Comprehensive \TeX{} Archive Network (CTAN). The CTAN path
for the package at the primary UK TUG site is

\begin{quote}
\texttt{ftp://ftp.tex.ac.uk/tex-archive/language/devanagari/velthuis/}
\end{quote}

Please use the tracking system at the project homepage at Sarovar to
submit feature requests, bugs, comments, and questions to the development
team.


\section{Producing \devnag{} Text with \TeX{}}

\devnag{} text may be included in any \TeX{} document. There are three steps
to producing \devnag{} text with \TeX{}. First, since \TeX{} does
not support \devnag{} natively, it is necessary to type \devnag{} using
\hbox{7-bit} (ASCII) roman characters that represent \devnag{} characters.
Secondly, transliterated \devnag{} text must be entered within
\textsf{devnag}-specific delimiters. These delimiters allow the preprocessor
to recognize the \devnag{} sections of the \TeX{} document. Third, the
transliterated input must be converted by the preprocessor into a format
that \TeX{} understands. The preprocessor scans the document for
\textsf{devnag} delimiters. Once it finds a delimiter, the program operates
on the text only within the scope of the delimiter. All other document text,
with the exception of \TeX{} macros and \textsf{devnag}-specific preprocessor
directives, is ignored.

Shown below is a \devnag{} passage followed by the 7-bit transliterated
input that produced it.

\begin{quote}
{\dn Dm\0\322w\?/\? \7{k}z\322w\?/\? smv\?tA \7{y}\7{y}(sv,.\\
mAmkA, pA\317wXvA\396w\4v Ekm\7{k}v\0t s\2jy..\par}
\end{quote}

\begin{quote}
\begin{verbatim}
{\dn dharmak.setre kuruk.setre samavetaa yuyutsava.h | \\
maamakaa.h paa.n.davaa"scaiva kimakurvata sa.mjaya ||}
\end{verbatim}
\end{quote}

\subsection{Macros and Font Definition Files}

\begin{description}
\item[\texttt{dnmacs.tex}] This file contains Plain \TeX{} macros for
\textsf{devnag} and various font-sizing commands. It must be loaded at
the beginning of the document with the command \verb+\input dnmacs+.

\item[\texttt{devanagari.sty}] This file provides \LaTeX{} support for
\textsf{devnag}. It must must be loaded in the preamble of the document
with the command \verb+\usepackage{devanagari}+. Section \ref{catcodes}
discusses advanced package options that may be declared with
\texttt{devanagari.sty}. The associated font definition files \texttt{udn*.fd}
provide NFSS support for \LaTeX{} for the \textsf{dvng} fonts.

\item[\texttt{dev.sty}] This file is kept for compatibility with old documents.
It merely loads \texttt{devanagari.sty}. Do not use it in new documents.

\item[\texttt{dev209.sty}] This file provides legacy support for the
obsolete \LaTeX{} 2.09. It should not be used. \end{description}

\subsection{Text Delimiters}

The preprocessor recognizes the text it is to act upon by use of
delimiters, of which there are two types. The basic delimiter is the
\verb+\dn+ macro. This delimiter is used by enclosing \devnag{} text
between \verb+{\dn ... }+, eg. \verb+{\dn acchaa}+.

The second delimiter is the \verb+$+ character. \devnag{} text
is enclosed between \verb+$ ... $+, eg. \verb+$acchaa$+. The
\verb+@dollars+ preprocessor directive must be specified to activate
this delimiter (section \ref{dol}).

The first delimiter is recommended especially for large blocks of
\devnag{} text. The second delimiter is useful when there is a need
to switch often between \devnag{} and roman text. Any text outside
of delimiters is not parsed by the preprocessor.

There are very few restrictions on what may be placed between the
delimiters. The 7-bit Velthuis encoding shown in Table \ref{chars}, all
punctuation marks, and all \TeX{} macro commands are acceptable input.
The preprocessor will produce a warning for unrecognized input
characters and commands.

\subsection{Example Input Files}

Two sample \devnag{} documents are bundled with this distribution.
Please refer to the contents of these files for examples of producing a
\devnag{} document. The file \texttt{misspaal.dn} contains an excerpt
from the Hindi short story \textit{Miss Pal} by Mohan Rakesh. The file
\texttt{examples.dn} contains some advanced examples of \devnag{}
typesetting. Shown below are two small examples of Plain \TeX{} and
\LaTeX{} documents with \devnag{} text.

\begin{multicols}{2}

\begin{verbatim}
  % Sample TeX input file
  \input dnmacs
  {\dn devaanaa.m priya.h}
  \bye
\end{verbatim}

\columnbreak

\begin{verbatim}
  % Sample LaTeX input file
  \documentclass{article}
  \usepackage{devanagari}
  \begin{document}
  {\dn devaanaa.m priya.h}
  \end{document}
\end{verbatim}

\end{multicols}

The filename of the \TeX{} document that contains \devnag{} should
be given a \texttt{.dn} extension. The preprocessor will produce
a filename with a \texttt{.tex} extension after processing the
input file.

\section{Input Encoding}
\label{trans}

\devnag{} text is prepared using a 7-bit (ASCII-based) transliterated
input encoding in which \devnag{} characters are represented by Roman
characters. The input encoding for \textsf{devnag} was developed by
Frans Velthuis with the objective to keep the format of the input source
text as close as possible to the accepted scholarly practices for
transliteration of \devnag{}. The Velthuis encoding is widely used and
has been adapted by other Indic language \TeX{} packages, and also
serves as the basis of other Indic transliteration schemes.

\subsection{Supplemental Notes}
Attention should be paid to the following points:
\begin{enumerate}
\item There are different ways to produce consonant conjuncts. For
example, the sequence \texttt{ktrya} can be represented as {\dn \3FCw}
and as {\dn ?\3ECwy}. The creation of conjuncts may be controlled
through the use of the preprocessor directives \verb+@sanskrit+,
\verb+@hindi+, and \verb+@modernhindi+, and more strictly through the
\verb+@lig+ directive. Please refer to Table \ref{ligs} for a list of
supported conjuncts.

\item There are two different ways to produce long vowels:
typing the short vowel twice or by capitalizing the short vowel, eg.
\texttt{aa} or \texttt{A} for {\dn aA}.

\item Aspirated consonants may be produced alternately by capitalizing
the voiceless counterpart. For example, the standard encoding for
{\dn B} is \texttt{bha}, but it may also be produced by \texttt{Ba};
{\dn G} is \texttt{gha} or \texttt{Ga}; etc.

\item For words which have two successive short vowels, a sequence of
brackets \verb+{}+ may be used to separate the vowels, eg. {\dn
\3FEw{}ug} \verb+pra{}uga+, as opposed to {\dn \3FEwOg} \verb+prauga+ .
This is required because the combinations \verb+ai+ and \verb+au+
represent the dipthongs {\dn e\?} and {\dn aO}.

\item The use of uppercase letters to indicate long vowels may be
preferred in cases where ambiguity might arise. When encoding a word
like {\dn kI}, the sequence \verb+kaii+ will produce the incorrect form
{\dn k\4i}, while \verb+kaI+ will yield the correct form {\dn kI}. A
sequence of brackets, as in the previous note, will also produce the
correct form, eg. \verb+ka{}ii+.

\item The standard ligatures {\dn\symbol{34}}, {\dn \3E2w}, and {\dn /}
are produced by \verb+k.sa+, \verb+j~na+, and \verb+tra+.

\item Candrabindu may be encoded either as a slash, as given in
Table~\ref{chars}, or by \verb+~m+.

% \item \verb+~a+ produces an `English a' Marathi style. For example,
% Hindi {\dn V\4?sF} and Marathi {\dn V\<?sF}.

\item Numerals are printed as Arabic numerals by default. The command
\verb+\dnnum+ switches to \devnag{} numerals. Every numeral after this
command is printed as a \devnag{} numeral. The command \verb+\cmnum+
switches back to Arabic numerals.

\item In Hindi mode the character \verb+&+ can be put at the end of a
word to produce a \textit{vir\={a}ma} sign under the final consonant.
For example, \verb+pari.sad&+ produces {\dn pErq\qq{d}} . The underscore
character \verb+_+ also produces a \textit{vir\=ama}.

\item In many Hindi words an \verb+a+ needs to be written between
consonants to produce the correct spelling, otherwise a conjunct
consonant will be produced. The correct form of Hindi {\dn krnA} is
\verb+karanaa+, not \verb+karnaa+, which produces {\dn knA\0}.

\item Tab characters in the input file, which previously were treated as
fatal errors, are now silently converted to spaces.
\end{enumerate}

%%% Velthuis Encoding Scheme %%%%%%%%%%%%%%%%%%%%%%%%%%%%%%%%%%%%%%%%%%%%%%%%

\begin{table}[p]
\begin{center}
\renewcommand{\doublerulesep}{.5cm}
\renewcommand{\arraystretch}{1.40}

\raggedcolumns
\begin{multicols}{3}

{\renewcommand{\arraystretch}{1.40}
\begin{tabular}{|llll|}
\hline
\multicolumn{4}{|c|}{\textsc{vowels}} \\
\hline
\textit{a}      & \texttt{a}  & {\dn a}              & \\
\textit{\=a}    & \texttt{aa} & {\dn a\symbol{'101}} & {\dn\symbol{'101}} \\
\textit{i}      & \texttt{i}  & {\dn i}              & {\dn\symbol{'105}} \\
\textit{\={\i}} & \texttt{ii} & {\dn I}              & {\dn\symbol{'106}} \\
\textit{u}      & \texttt{u}  & {\dn u}              & {\dn\symbol{0}} \\
\textit{\=u}    & \texttt{uu} & {\dn U}              & {\dn\symbol{'1}} \\
\textit{\underrng{r}} & \texttt{.r} & {\dn\symbol{'33}} & {\dn\symbol{2}} \\
\textit{\diatop[\=|\underrng{r}]} & \texttt{.R} & {\dn\symbol{'21}} & {\dn\symbol{'16}} \\
\textit{\underrng{l}} & \texttt{.l} & {\dn\symbol{'30}} & {\dn\symbol{'37}} \\
\textit{\diatop[\=|\underrng{l}]} & \texttt{.L} & {\dn\symbol{'31}} & {\dn\symbol{'174}} \\
\textit{e}      & \texttt{e}  & {\dn e}              & {\dn\symbol{3}} \\
\textit{ai}     & \texttt{ai} & {\dn e\?}            & {\dn\symbol{'173}} \\
\textit{o}      &  \texttt{o} & {\dn ao}             & {\dn o} \\
\textit{au}     & \texttt{au} & {\dn aO}             & {\dn O} \\
\hline
\end{tabular}}
\vspace{.325in}

{\renewcommand{\arraystretch}{1.40}
\begin{tabular}{|lll|}
\hline
\multicolumn{3}{|c|}{\textsc{signs}} \\
\hline
\textit{anusv\=ara} & \texttt{.m} & {\dn\symbol{'25}} \\
\textit{candrabindu} & \texttt{/} & {\dn\symbol{'40}} \\
\textit{visarga} & \texttt{.h} & {\dn\symbol{'54}} \\
\textit{avagraha} & \texttt{.a} & {\dn\symbol{'137}} \\
\textit{vir\=ama} & \texttt{\&} & {\dn\symbol{'136}} \\
\textit{candra a} & \texttt{\char`~a} & {\dn\symbol{'4}} \\
\textit{candra o} & \texttt{\char`~o} & {\dn A\<} \\
\textit{AUM} & \texttt{.o} & {\dn\symbol{'72}} \\
\textit{da\d{n}\d{d}\=a} & \texttt{\symbol{'174}} & {\dn\symbol{'56}} \\
double \textit{da\d{n}\d{d}\=a} & \texttt{\symbol{'174}\symbol{'174}} & {\dn\symbol{'73}} \\
`eyelash' \textit{repha} & \texttt{\char`~r} & {\dn\symbol{'35}} \\
abbreviation & \texttt{@} & {\dn\symbol{'177}} \\
ellipsis & \texttt{\symbol{'43}} & {\dn\symbol{'25}} \\
period & \texttt{..} & {\dn\symbol{'24}} \\
\hline
\end{tabular}}

\columnbreak

{\renewcommand{\arraystretch}{1.40}
\begin{tabular}{|lll|}
\hline
\multicolumn{3}{|c|}{\textsc{consonants}} \\
\hline
\textit{ka} & \texttt{ka} & {\dn k} \\
\textit{kha} & \texttt{kha} & {\dn K} \\
\textit{ga} & \texttt{ga} & {\dn g} \\
\textit{gha} & \texttt{gha} & {\dn G} \\
\textit{\.na} & \texttt{"na} & {\dn R} \\
\textit{ca} & \texttt{ca} & {\dn c} \\
\textit{cha} & \texttt{cha} & {\dn C} \\
\textit{ja} & \texttt{ja} & {\dn j} \\
\textit{jha} & \texttt{jha} & {\dn J} \\
\textit{\~na} & \texttt{\char`~na} & {\dn\symbol{'32}} \\
\textit{\d{t}a} & \texttt{.ta} & {\dn V} \\
\textit{\d{t}ha} & \texttt{.tha} & {\dn W} \\
\textit{\d{d}a} & \texttt{.da} & {\dn X} \\
\textit{\d{d}ha} & \texttt{.dha} & {\dn Y} \\
\textit{\d{n}a} & \texttt{.na} & {\dn Z} \\
\textit{ta} & \texttt{ta} & {\dn t} \\
\textit{tha} & \texttt{tha} & {\dn T} \\
\textit{da} & \texttt{da} & {\dn d} \\
\textit{dha} & \texttt{tha} & {\dn D} \\
\textit{na} & \texttt{na} & {\dn n} \\
\textit{pa} & \texttt{pa} & {\dn p} \\
\textit{pha} & \texttt{pha} & {\dn P} \\
\textit{ba} & \texttt{ba} & {\dn b} \\
\textit{bha} & \texttt{bha} & {\dn B} \\
\textit{ma} & \texttt{ma} & {\dn m} \\
\textit{ya} & \texttt{ya} & {\dn y} \\
\textit{ra} & \texttt{ra} & {\dn r} \\
\textit{la} & \texttt{la} & {\dn l} \\
\textit{va} & \texttt{va} & {\dn v} \\
\textit{\'sa} & \texttt{"sa} & {\dn f} \\
\textit{\d{s}a} & \texttt{.sa} & {\dn q} \\
\textit{sa} & \texttt{sa} & {\dn s} \\
\textit{ha} & \texttt{ha} & {\dn h} \\
\hline
\end{tabular}}

\columnbreak

{\renewcommand{\arraystretch}{1.40}
\begin{tabular}{|lll|}
\hline
\multicolumn{3}{|c|}{\textsc{consonants}} \\
\hline
\textit{qa} & \texttt{qa} & {\dn\symbol{'52}} \\
\textit{\underbar{kh}a} & \texttt{.kha} & {\dn\symbol{'14}} \\
\textit{\.ga} & \texttt{.ga} & {\dn\symbol{'13}} \\
\textit{za} & \texttt{za} & {\dn\symbol{'51}} \\
\textit{\d{r}a} & \texttt{Ra} & {\dn w} \\
\textit{\d{r}ha} & \texttt{Rha} & {\dn x} \\
\textit{fa} & \texttt{fa} & {\dn\symbol{'47}} \\
\textit{\d{l}a} & \texttt{La} & {\dn\symbol{'17}} \\
\hline
\end{tabular}}
\vspace{.325in}

{\renewcommand{\arraystretch}{1.20}
\begin{tabular}{|lll|}
\hline
\multicolumn{3}{|c|}{\textsc{digits}} \\
\hline
\textit{0} & \texttt{0} & {\dn\dnnum \rn{0}} \\
\textit{1} & \texttt{1} & {\dn\dnnum \rn{1}} \\
\textit{2} & \texttt{2} & {\dn\dnnum \rn{2}} \\
\textit{3} & \texttt{3} & {\dn\dnnum \rn{3}} \\
\textit{4} & \texttt{4} & {\dn\dnnum \rn{4}} \\
\textit{5} & \texttt{5} & {\dn\dnnum \rn{5}} \\
\textit{6} & \texttt{6} & {\dn\dnnum \rn{6}} \\
\textit{7} & \texttt{7} & {\dn\dnnum \rn{7}} \\
\textit{8} & \texttt{8} & {\dn\dnnum \rn{8}} \\
\textit{9} & \texttt{9} & {\dn\dnnum \rn{9}} \\
\hline
\end{tabular}}

\end{multicols}

\caption{The Velthuis Encoding Scheme}
\label{chars}
\end{center}
\end{table}

%%%%%%%%%%%%%%%%%%%%%%%%%%%%%%%%%%%%%%%%%%%%%%%%%%%%%%%%%%%%%%%%%%%%%%%%%%%%%

\section{The Preprocessor}
The ANSI C program \texttt{devnag.c} is a preprocessor that
reads transliterated \devnag{} input delimited by \verb+\dn+ and
converts it into a form with which \TeX{} is familiar.
To use the preprocessor, \texttt{devnag.c} must be compiled into
an executable.

The preprocessor
handles the details of character placement such as the alignment of
vowel diacritics and consonant ligatures. The rest of the layout,
however, must be managed by the user. The preprocessor is invoked
as

\begin{center}
\texttt{devnag} \textit{in}[\texttt{.dn}] (\textit{out}[\texttt{.tex}])
\end{center}

The default file extension for an input file is \verb+.dn+ and for an output
file \verb+.tex+. The output filename is optional. If an output filename is
not specified, the preprocessor will name it after the input file.

For example, typing \texttt{devnag hindi} will instruct the preprocessor
to read from the file \texttt{hindi.dn} and write to the file
\texttt{hindi.tex}. The program will prompt for the names of the
input and output files if they are not given in the command-line.

% If the input file has a name such as \texttt{x.y.dn}, the
% output file name will be \texttt{x.y.tex}. Addition, files must have
% the suffixes \texttt{.dn} and \texttt{.tex}, though as before these are
% supplied by the program if omitted by the user. This change has been
% made for safety reasons: formerly, typing \texttt{devnag myfile.tex}
% would delete the contents of \texttt{myfile.tex}.

The preprocessor tries to be safe yet flexible. The extensions are not forced but the input and 
output file names must be distinct. The exact algorithm (since version~2.14) is:

\begin{enumerate}
\item If the source filename does not have the default extension, it is
    appended. If such a file does not exist, the preprocessor tries
    again with the original file name.

\item If the destination filename was given, its extension is checked.
    It is explicitly forbidden for the destination file to have the
    \texttt{.dn} extension. If the filename is equal
    to the name of the source file, \texttt{.tex} is appended,
    otherwise the file name is used as is.

\item If the destination filename was not given, it is based upon the
    source filename. If the source filename contains the
    \texttt{.dn} extension, it is stripped of. Afterwards the
    \texttt{.tex} extension is appended.
\end{enumerate}

The algorithm was designed to prevent typing errors but it is far
from foolproof. The filesystem properties are not examined and all
filename tests are case insensitive. The file names are not expanded.
For instance, if the working directory is \verb"/some/path", then \verb"x.y",
\verb"./x.y", \verb"../path/x.y" as well as \verb"/some/path/x.y" refer to the same
file but they will be treated as different by the preprocessor.
Moreover, the algorithm does not try to follow symlinks neither does it
examine inodes in order to discover hard links.

If \texttt{devnag} is invoked with the \texttt{-v} option, it will
display the version number and then exit.

\subsection{Preprocessor Directives}
\label{directives}
The preprocessor creates a \TeX{} file from the input file by acting on
two different parts of the input: directives and transliterated input.
As it creates a \TeX{} file from the input file, the preprocessor can be
told to modify the way in which it operates by means of special commands
called directives. Directives are optional commands to the preprocessor
that instruct it to process the input text in a given manner, such as
permitting hyphenation, suppressing the use of certain ligatures, etc.
Directives do not affect typesetting or layout.

Directives must occupy a line by themselves and always begin with the
character \verb+@+. Directives may occur anywhere in the document,
but not within \devnag{} delimiters (where \verb+@+ is the
continuation symbol {\dn \char`\^^?}). Directives specific to a
particular passage of text should appear immediately before that passage;
directives applying to the entire file should appear just before the first
line of actual text to be typeset.

Since \verb+@+ is a perfectly legal character in \TeX{}, lines beginning
with \verb+@+ that do not match any valid \verb+@+ command are flagged
with a warning, but processing of the file continues. (In the somewhat
unlikely event that there is actually a need to have a line of \TeX{} text
consisting exactly of, for example, \verb+@hindi+, the preprocessor may
be fooled by typing \verb+{}@hindi+ or \verb+{@hindi}+.

New ``negative'' commands have
been added to reverse the effect of most existing commands: thus it is
now possible to enable or disable specific features for specific
passages of text, eg. \verb+@hindi+ may be disabled with
\verb+@sanskrit+.

In previous releases of \textsf{devnag}, the preprocessor would split
long lines in its output. The \verb+@obeylines+ command was provided
to disable this feature. The line-splitting feature has been disabled,
so that the preprocessor now outputs lines as they appear in the input
file. The \verb+@obeylines+ directive is no longer recognized by the
preprocessor as a valid directive and will be ignored; it will therefore
be typeset as a part of the text of the document.

\begin{description}
\item[\texttt{@sanskrit}]
The \verb+@sanskrit+ directive is the default mode for the preprocessor.
This command may also be used to reinstate the default behavior of the
preprocessor after the use of \verb+@hindi+ and \verb+@modernhindi+.
See Table \ref{ligs} for a list of conjuncts and the forms produced in
\texttt{@sanskrit} mode.

\item[\texttt{@hindi}]
The \verb+@hindi+ directive switches the preprocessor to Hindi mode. The
difference between the Sanskrit and Hindi modes is that in Sanskrit mode
the full forms of conjuncts are used, whereas in Hindi mode certain
simplified forms are used instead, eg. {\dn Qc} in place of {\dn \3CEw}.
See Table \ref{ligs} for a list of conjuncts and the forms produced by
\texttt{@hindi}. Additionally, in Sanskrit mode a \textit{vir\=ama} is
automatically added at the end of a word if it ends in a consonant,
while in Hindi mode the preprocessor assumes the presence of the
inherent vowel \textit{a}. The directive \verb+@hindi+, if used, must
precede the \verb+@lig+ and \verb+@nolig+ commands. See Table \ref{ligs}
for a list of conjuncts and the forms produced in \texttt{@hindi} mode.

\item[\texttt{@modernhindi}]
This directive switches the preprocessor to Hindi mode, similar to
\verb+@hindi+, but uses far fewer Sanskrit-style ligatures. Conjuncts are
created from half-consonant forms wherever possible. See Table \ref{ligs}
for a list of conjuncts and the forms produced in \texttt{@modernhindi}
mode.

\item[\texttt{@dollars} / \texttt{@nodollars}]
\label{dol}
In addition to the \verb+{\dn+ ... \verb+}+ delimiters, \devnag{} text
can also be delimited by dollar signs, eg. \verb+$acchaa$+. The directive
\verb+@dollars+ instructs the preprocessor to switch to dollar mode and to
recognize \verb+$+ as a delimiter. In dollar mode, the dollar sign cannot
be used for other purposes, such as printing a dollar sign or switching to
math mode. Dollar signs may be printed by through low-level font commands,
eg. \verb+\char36+ in Plain \TeX{} or \verb+\symbol{36}+ in \LaTeX{}.
Switching to math mode when \texttt{@dollars} is active is accomplished
by using \verb+\(+ and \verb+\)+ as math delimiters.

\item[\texttt{@dolmode0} / \texttt{@dolmode1} / \texttt{@dolmode2} / \texttt{dolmode3}]
\label{advtop}

When \texttt{@dollars} is active, the behavior of the preprocessor
can be modified further through the \verb+@dolmode0+, \verb+@dolmode1+,
\verb+@dolmode2+, and \verb+@dolmode3+ directives.

\begin{quote}
\begin{tabular}{llll}
\verb+@dolmode0+ & \verb+$acchaa$+  & $\rightarrow$  & \verb+$acchaa$+ \\
\verb+@dolmode1+ & \verb+$acchaa$+  & $\rightarrow$  & \verb+\dn aQCA+ \\
\verb+@dolmode2+ & \verb+$acchaa$+  & $\rightarrow$  & \verb+\pdn aQCA+ \\
\verb+@dolmode3+ & \verb+$acchaa$+  & $\rightarrow$  & \verb+aQCA+
\end{tabular}
\end{quote}

Alternately, \verb+@dolmode3+ can be mimicked using the macro
\verb+\dn#+, \verb+{\dn# acchaa}+ $\rightarrow$ \texttt{aQCA}. Also, the
Plain \TeX{} macro \verb+\pdn+ changes the current font into \devnag{}
in the current size. \LaTeX{} automatically adjusts the font sizing for
\devnag{} to the document font size.

\item[\texttt{@lig} / \texttt{@nolig}]
\label{ligcom}
Certain conjuncts may be enabled or disabled by using the directives
\verb+@lig+ and \verb+@nolig+. The former enables conjuncts while
the latter disables them. Supported conjuncts are assigned codes
and are showned in Table \ref{ligs}. For example, the command
\verb+@nolig 2+ produces {\dn ?t} instead of {\dn \3C4w} from
the input \verb+kta+.

More than one conjunct may be specified with a \verb+@lig+ or
\verb+@nolig+ command, eg. \verb+@lig 20 43 90+. There is no limit to
the number of \verb+@lig+ or \verb+@nolig+ directives issued within a
document. However, when a certain conjunct is disabled, all other
conjunct combinations involving the disabled conjunct are also disabled.
For example, if conjunct 3 {\dn\symbol{'307}} (\verb+kna+) is disabled
then conjunct 10 {\dn\symbol{'346}} (\verb+knya+) will also be disabled.

Some basic \devnag{} conjuncts like {\dn\symbol{34}}, {\dn \3E2w} and
{\dn /} cannot be disabled and are not shown in Table \ref{ligs}. Also,
most two element ligatures involving \textit{ra}, eg. {\dn\symbol{135}},
{\dn g\5}, and {\dn \3FEw} cannot be disabled.

\item[\texttt{@hyphen} / \texttt{@nohyphen}]
These directives control hyphenation of \devnag{} text. They
provide the ability to enclose a section of particularly dense text
between \verb+@hyphen+ and \verb+@nohyphen+ without affecting other
parts of text in the document. To type a hyphen
in a \devnag{} document simply type \texttt{-}. For example, typing
\verb+{\dn dive-dive}+ will produce {\dn Edv\?{\rs -\re}Edv\?}. Please
refer to section \ref{hyphenation} for more information.

\item[\texttt{@tabs} / \texttt{@notabs}]
The \verb+@tabs+ directive instructs the preprocessor to recognize
the \verb+&+ character as a \TeX{} tabular character, not as a method
of encoding \textit{vir\=ama}. The command \verb+@notabs+ resets
this feature. If \verb+&+ appears word medially, eg.
\verb+.sa.t&"siraa.h+, it will be processed as a \textit{vir\=ama} even
if \verb+@tabs+ is specified. This avoids possible incompatibility
issues with legacy documents. The \verb+_+ character now doubles as a
way of producing \textit{vir\=ama}, particularly if \verb+@tabs+ is
used, eg. \verb+pari.sad_+

\item[\texttt{@vconjuncts} / \texttt{@novconjuncts}]
A change has been introduced in the way text like \verb+{\dn .sa.t"siraa.h}+
is processed; specifically, instances where an \textit{i} vowel is associated
with a consonant sequence containing a \textit{vir\=ama}. The previous
behavior was to treat the consonant sequence as if it were a normal conjunct
by placing the vowel diacritic before the sequence as a whole, eg.
{\dn qE\qq{V}frA,}. The majority opinion seems to be that this is undesirable,
and that the vowel symbol should follow the consonant to which the
\textit{vir\=ama} is attached, eg. {\dn q\qq{V}EfrA,}. This is now the
default behavior of the preprocessor, but the \verb+@vconjuncts+
directive has been implemented to reinstate the previous output method.
The command \verb+@novconjuncts+ resets this feature.
\end{description}

\subsection{Protecting Text from Conversion}\label{protecting}
The preprocessor will convert all text found in a \devnag{} environment.
Text may be protected from the preprocessor using the \verb+<+ and \verb+>+
delimiters. In the example below, the font command between the angle brackets
will be ignored by the preprocessor, but will be removed from the
output file.

For example, with \verb+{\dn dharmak.setre <\font\zzz=dvng10 at 18pt> kuruk.setre}+
the preprocessor will operate on \texttt{dharmak.setre} and \texttt{kuruk.setre},
but will ignore \verb+\font\zzz=dvng10 at 18pt+ because it occurs within the
\verb+<+ and \verb+>+ delimiters.

\subsection{Breaking Pre-Defined Conjuncts}
The preprocessor will automatically produce predefined ligatures
from certain sequences of consonants. Placing the \texttt{+} character
between two consonants prevents any predefined ligature representing
those consonants from being produced. Instead a conjunct will be
created from half-forms, or, if half-forms do not exist, full forms
stopped with \textit{vir\=ama} will be used.

For example, the sequence \texttt{kha} will produce {\dn K}. Using
\texttt{+} to break the sequence -- \texttt{k+ha} -- will create {\dn ?h}.

The use of \texttt{+} is similar to the use of \verb+{}+.
For example, write \verb+t{}ha+ to produce {\dn \qq{t}{}h}, if
desired instead of \verb+tha+ {\dn T}.

The \texttt{+} character can be used independently of the \verb+@lig+/\verb+@nolig+
directives, and it can disable any conjunct. Note that the \texttt{+} character
only disables single occurrences of a conjunct. To disable all occurrence
of a conjunct use the \verb+@lig+ directive.

\subsection{Supported \LaTeX{} Commands}
The preprocessor recognizes some \LaTeX{} macros with arguments. The
following command types are legal within delimiters.

\begin{itemize}
\item Font commands: Standard \TeX{} size-changing commands,
eg. \verb+\small+, \verb+\large+, \verb+\huge+.

\item Environments, including the three table environments:
\verb+tabular+, \verb+supertabular+, and \verb+longtable+. Note: To use
table environments within delimited text, the \verb+@tabs+ directive
must be specified in order to enable the use of the ampersand as a tab
marker instead of a marker for \textit{vir\=ama}. Refer to Section
\ref{directives} for more details on preprocessor directives.

\item Spaces: \verb+\hspace+, \verb+\hspace*+, \verb+\vspace+,
\verb+\vspace*+, \verb+\addvspace+, \verb+\setlength+, \verb+\addtolength+,
\verb+\enlargethispage+, \verb+\enlargethispage*+,  and
\verb+\\[+\textit{n}\verb+]+. Plain \TeX{} commands may also be used
when placed between brackets: \verb+{\hskip }+, \verb+{\vskip }+,
\verb+{\vadjust }+, and \verb+{\kern }+.

\item Counters: \verb+\setcounter+, \verb+\stepcounter+,
\verb+\addtocounter+, and \verb+\refstepcounter+. Page numbering in
\devnag{} is also available through the \verb+dev+ counter.

\item Boxes and rules: \verb+\parbox+, \verb+\makebox+,
\verb+\framebox+, \verb+raisebox+, and \verb+\rule+.

\item References: \verb+\label+, \verb+\ref+, \verb+\pageref+,
\verb+\index+, \verb+\cite+, and \verb+\bibitem+. If the argument of the
\verb+\index+ command is in \devnag{}, it will appear in \devnag{}
in the index file.

\item File commands: \verb+\input+ and \verb+\include+.

\item Roman text may also be embedded within \devnag{} delimited text as
long as the Roman does not exceed the length of one line. Use
\verb+{\rm ... }+ to produce embedded Roman text.
\end{itemize}

\subsection{Using Custom \LaTeX\ Commands}
\LaTeX\ users take advantage of numerous commands from various packages. All packages will never be
supported directly but solution is easy. As a matter of fact, all control sequences without 
arguments or taking \devnag\ text as argument can be used without problem. Suppose the you want to 
colorize text {\dn \textcolor{red}{: t(s\qq{t}}} using the \verb;\textcolor; command from the 
\textsc{color} package. The bare word \textit{red} must not be converted. Conversion can be 
suppressed by mechanism described in section~\ref{protecting}. One can write e.\,g.:

\begin{verbatim}
{\dn \textcolor<{red}>{.o tatsat}}
\end{verbatim}

It is, however, preferable to separate meaning and form. If a macro, e.\,g.\@ \verb;\myemph; is 
defined, it is possible to decide later how the text will be emphasized. The text will then be 
entered as follows:

\begin{verbatim}
\def\myemph#{\textcolor{red}}
{\dn \myemph{.o tatsat}}
\end{verbatim}

Using such an approach one does not need any escaping mechanism at all.

\section{\devnag{} Fonts}

The \textsf{devnag} package provides three font families in addition to
the Standard family: Bombay, Calcutta, and Nepali. All families are
available in regular, oblique, bold, bold oblique, and pen shapes
and weights. The Bombay, Calcutta, and Nepali families provide
variant glyphs which are predominant regional forms for certain
characters, as shown in Table \ref{diffs}.

\begin{itemize}
\item \verb+\dnbombay+ switches to the Bombay family
\item \verb+\dncalcutta+ switches to the Calcutta family
\item \verb+\dnnepali+ switches to the Nepali family
\item \verb+\dnoriginal+ switches back to the default regular family
\item \verb+\dnpen+ switches to the Pen family
\item \verb+\dnpenbombay+ switches to the Bombay Pen family
\item \verb+\dnpencalcutta+ switches to the Calcutta Pen family
\item \verb+\dnpennepali+ switches to the Nepali Pen family
\end{itemize}

The oblique, bold, and bold oblique shapes and weights are produced
using standard \LaTeX{} macros. Oblique is obtained with either of the
\verb+\textit{}+ or \verb+\itshape+ font-changing commands. Bold is
obtained with either \verb+\textbf{}+ or \verb+\bfseries+. Bold oblique
requires a combination of the bold and oblique commands, such as
\verb+\bfseries\itshape+.

To use bold, oblique, and bold oblique varieties in Plain \TeX{}, use the macros
\verb+\dnbf+ and \verb+\dnit+. The regional families are accessed using the
macro commands described above. See \texttt{dnmacs.tex} for further information.

Font size may be controlled in \LaTeX{} with the standard font sizing
commands. In Plain \TeX{}, font size may be controlled with the following
macros: \verb+\dnsmall+, \verb+\dnnine+, \verb+\dnnormal+,
\verb+\dnhalf+, \verb+\dnbig+, \verb+\dnlarge+, and \verb+\dnhuge+. See
\texttt{dnmacs.tex} for further information.

\subsection{Bombay-Style Fonts}

The family name for the Bombay \devnag{} fonts is \textsf{dnb}. To
access this family, use the command \verb+\dnbombay+ after the
\verb+\dn+ macro. Standard \LaTeX{} font commands like
\verb+\fontfamily{dnb}+ and \verb+\usefont{U}{dnb}{}{}+ may be used to
access the Bombay fonts, however, these commands conflict with the
preprocessor. Access to the Bombay family within \devnag{}
environments should be restricted to the \verb+\dnbombay+ macro.
Use the command \verb+\dnoriginal+ to return to the standard \devnag{}
font.

\subsection{Calcutta-Style Fonts}

The family name for the Calcutta \devnag{} fonts is \textsf{dnc}. In roder to
access this family, use the command \verb+\dncalcutta+ after the
\verb+\dn+ macro. Standard \LaTeX{} font commands like
\verb+\fontfamily{dnc}+ and \verb+\usefont{U}{dnb}{}{}+ may be used to
access the Calcutta fonts, however, these commands conflict with the
preprocessor. Access to the Calcutta family within \devnag{}
environments should be restricted to the \verb+\dncalcutta+ macro.
Use the command \verb+\dnoriginal+ to return to the standard \devnag{}
font.

\subsection{Nepali-Style Fonts}

The family name for the Nepali \devnag{} fonts is \textsf{dnn}. To
access this family, use the command \verb+\dnnepali+ after the
\verb+\dn+ macro. Standard \LaTeX{} font commands like
\verb+\fontfamily{dnn}+ and \verb+\usefont{U}{dnn}{}{}+ may be used to
access the Nepali fonts, however, these commands conflict with the
preprocessor. Access to the Nepali family within \devnag{}
environments should be restricted to the \verb+\dnnepali+ macro.
Use the command \verb+\dnoriginal+ to return to the standard \devnag{}
font.


\subsection{\devnag{} Pen Fonts}

The \devnag{} Pen family is a simple modification of the Standard
face created by Tom Ridgeway, which resembles Devanagari written
with a pen. Standard Pen fonts are available as the family \textsf{dnp} and may be
accessed within \devnag{} environments with the command \verb+\dnpen+. The
Pen family for the Bombay style are available as the family \textsf{dnpb},
and may be accessed with the command \verb+\dnpenbombay+. The Pen family for
the Calcutta style is called \textsf{dnpc} and may be accessed with the command
\verb+\dnpencalcutta+. The Pen family for the Nepali style is called \textsf{dnpn}
and may be accessed with the command \verb+\dnpennepali+. Use the command
\verb+\dnpen+ to return to the standard \devnag{} Pen font.


\subsection{Default \devnag{} Font (\LaTeX{} Only)}

The \devnag{} package provides options \verb+bombay+, \verb+calcutta+,
\verb+nepali+, \verb+pen+, \verb+penbombay+, \verb+pencalcutta+, and
\verb+pennepali+ so as to set the corresponding font as the default one. It may
seem that using \verb+\dncalcutta+ at the beginning of the document is
sufficient. However, as we will show later in this document, the \devnag{}
package may create automatically some captions as well as a running head. When
producing such texts, \LaTeX{} is set to use Roman fonts and the automatic text
switches to \devnag{} just by \verb+\dn+. You would thus see {\dn\dncalcutta
a@yAy} in the normal text but {\dn a@yAy} in automatic captions which is
undesirable. The package options inform which font style should be used as
default.

It is also possible to change the default font by defining macro
\verb+\dnfamilydefault+.

The font switching commands described in the previous subsections can be used
for local changes of the style.

\subsection{PostScript Type 1}

The package now includes Type~1 fonts created by Karel P\'{\i}\v{s}ka using an 
analytic fit. The fonts included here are a subset of his release of Indic 
Type~1 fonts that are available from CTAN:fonts/ps-type1/indic and licensed 
under GPL.

 An accurate analytic fit of outline contour curves taken from the \MP{}
output helps to avoid artifacts produced by an autotracing bitmap approach.
It allows to keep preciseness of calculations and to produce the outline
fonts faithful, optimal (to minimize their space amount) and hinted.
Therefore the results are not only more precise than fonts presented earlier
but also occupy a smaller place even if they include hinting additionally.
The distribution contains the PFB and TFM files both with accurate glyph
widths. The user UniqueID are present to distinguish fonts
during download process in PostScript printers and other RIP devices.

The AFM files have be considered as derived files not usable for TFM creation, 
because the \textsc{afm2tfm} program has a feature to round the glyph widths 
and is not able to reproduce the original metrics. Use these files only with 
tools that explicitly require AFM.

To use the Type 1 fonts with \texttt{dvips} and pdf\TeX, it is in modern
\TeX{} distributions sufficient to run
\texttt{mktexlsr} or \texttt{texhash} and then issue:

\begin{verbatim}
updmap --enable MixedMap=dvng.map
\end{verbatim}

If you do not have \texttt{updmap}, you must
edit the local \textsf{dvips} \texttt{psfonts.map}
file to contain a reference to \texttt{dvng.map}; or copy the contents
of \texttt{dvng.map} into \texttt{config.ps}.

Detailed installation instructions can be found in the README file in the root 
directory of the CVS working copy or in the \texttt{doc/generic/velthuis/} 
directory in the release package.

% \subsection{Creating Custom Font Varieties}
%
% The \MF{} source files have been organized to make it easy to produce
% custom font varieties, such as larger point sizes, fonts that contain
% some alternate letters but not others, and non-standard bold and oblique
% fonts. For further information, see the comments in \texttt{dngen.mf}.
% If you are using \LaTeX{}, you may need to modify \texttt{devanagari.sty}, and
% add or modify \texttt{.fd} files. For Plain \TeX{}, you may want to
% modify \texttt{dnmacs.tex}.

\section{Special Topics}\label{spec}

\subsection{Delimiter Scope}
The \LaTeX{} font-size commands may be used within
\devnag{} delimited text, however, as a general rule, the font-size
command should follow the \verb+\dn+ delimiter, otherwise
the font definition commands of \verb+\dn+ will be overridden. For
example, items 1, 2, and 3 below produce the correct forms, but
4 does not:

\begin{quote}
\begin{tabular}{llcl}
1. & \verb+{\dn \large acchaa}+   & $\rightarrow$ & {\dn aQCA} \\
2. & \verb+{\dn {\large acchaa}}+ & $\rightarrow$ & {\dn aQCA} \\
3. & \verb+{\large {\dn acchaa}}+ & $\rightarrow$ & {\dn aQCA} \\
4. & \verb+{\large \dn acchaa}+   & $\rightarrow$ & {\dn acchaa} \\
\end{tabular}
\end{quote}

\subsection{Line Spacing}
Due to the super- and subscript characters of the \devnag{} script, the
default line spacing (leading) often needs to be increased to prevent
the crowding of lines. The parameter \verb+\baselineskip+
(\verb+\linespread+ for \LaTeX{}) controls the line spacing.

\TeX{} determines and adjusts the value of \verb+\baselineskip+ after it
finishes processing a paragraph. If a paragraph contains a mixture of
\devnag{} and Roman text, and ends with Roman text, then \TeX{} will set
the value of \verb+\baselineskip+ according to the Roman text. This may
result in crowding of \devnag{} text.

There are, however, solutions to this. An explicit value can be assigned
to \verb+\baselineskip+ before the paragraph ends. The macro file
\texttt{dnmacs.tex} shows examples of the value of \verb+\baselineskip+
at different font sizes. Default line spacing is also set in
\texttt{devanagari.sty}. Alternately, `dummy' \textsf{devnag} text containing
\verb+\par+ can be placed at the end of the paragraph, eg.
\verb+{\dn \par}+.

Even when a paragraph has only \textsf{devnag} text, the paragraph-end
command must be included within \textsf{devnag} text, meaning that the
closing delimiter, which ends the \textsf{devnag} text, must follow
the empty line or the \verb+\par+ command that forces the paragraph to
end.

\subsection{Hyphenation}
\label{hyphenation}

The \texttt{devnag} package does more or less all that needs to be done
from the point of view of hyphenating Sanskrit in \devnag{} through the
\texttt{@hyphen} and \texttt{@nohyphen} directives, which are discussed in
section \ref{directives}. If hyphenation is off, then there are no hyphens,
and very stretchy inter-word space. This is acceptable for ragged-right
settings or for text in verse form, but may produce poor results
in right-justified prose text, especially if the given passage contains
long compound words. If hyphenation is on then discretionary hyphens are
set between all syllables.

\subsection{Captions and Date Formats (\LaTeX{} only)}
The language modules of the \textsf{babel} package change captions texts and
date formats. Although \textsf{devanagari.sty} is not a \textsf{babel} module, similar
mechanism is implemented here. Macros \verb=\datehindi= and
\verb=\datemodernhindi= enable Europian style Hindi date generated by the
standard \verb=\today= command. The ``traditional'' and ``modern'' variants
comtain the same names of the months, they differ only in the ligatures used.
You should therefore use \verb=\datemodernhindi= in documents processed with
\verb=@modernhindi=. The names of the months used in the definition of
\verb=\datemodernhindi= are summarized in Table~\ref{months}.

\begin{table}[bth]
\centering
\extrarowheight 2pt
\begin{tabular}{|rl|rl|}\hline
1 & {\dn jnvrF} & 7 & {\dn \7{j}lAI}\\
2 & {\dn \327wrvrF} & 8 & {\dn ag-t}\\
3 & {\dn mAc\0} & 9 & {\dn EstMbr}\\
4 & {\dn a\3FEw\4l} & 10 & {\dn a?\8{t}br}\\
5 & {\dn mI} & 11 & {\dn nvMbr}\\
6 & {\dn \8{j}n} & 12 & {\dn EdsMbr}\\\hline
\end{tabular}

\caption{Names of the months in the definition of
\texttt{\char92 datemodernhindi}}\label{months}
\end{table}

The captions are similarly switched to Hindi by \verb=\captionshindi= or
\verb=\captionsmodernhindi=, respectively. Again the texts differ only in the
ligatures used. The captions for the modern Hindi variant are given in
Table~\ref{captions}.

\begin{table}[bt]
\centering
\extrarowheight 2pt
\begin{tabular}{|>{\tt\char92 }l|l|}\hline
\multicolumn{1}{|l|}{\bfseries Macro} & \bfseries Caption \\\hline
abstractname & {\dn sArA\2f}\\
appendixname & {\dn pErEf\309wV}\\
bibname & {\dn s\2df\0 g\5\306wT}\\
ccname & \\
chaptername & {\dn a@yAy}\\
contentsname & {\dn Evqy{\rs -\re}\8{s}cF}\\
enclname & {\dn }\\
figurename & {\dn Ec/}\\
headpagename & {\dn \9{p}\309wW}\\
headtoname & \\
indexname & {\dn \8{s}cF}\\
listfigurename & {\dn Ec/o{\qva} kF \8{s}cF}\\
listtablename & {\dn tAElkAao\2 kF \8{s}cF}\\
pagename & {\dn \9{p}\309wW}\\
partname & {\dn K\317wX}\\
prefacename & {\dn \3FEw-tAvnA}\\
refname & {\dn hvAl\?}\\
tablename & {\dn tAElkA}\\
seename & {\dn d\?EKe}\\
alsoname & {\dn aOr d\?EKe}\\
alsoseename & {\dn aOr d\?EKe}\\\hline
\end{tabular}

\caption{Modern Hindi captions}\label{captions}
\end{table}

The macros for the \textsc{letter} class are left intentionally empty. The idea
of the \textsf{babel} package is to prepare a universal template for business
letters using a set of macros. The header of the letter would make use of the
\verb=\headtoname= macro which will produce ``To: Mr.\@ Kumar'' in English
letters and ``Komu: pan Kumar'' in Czech letters. If we simply defined
\verb=\headtoname= to {\dn ko}, the universal template would put it before the
name which would be wrong. Hindi requires different word order, namely {\dn
\399wF \7{k}mAr ko}. The universal templates are thus useless in Hindi and the
letter template must be redesigned almost from scratch. It therefore makes no
sense to define the letter macros.

Two package options are provided: \verb=hindi= and \verb=modernhindi=. If used,
they cause the \verb=\dn= command to switch the caption text and date format as
well. The date format and captions may be switched back by macros
\verb=\dateenglish=, \verb=\dateUSenglish=, and \verb=\captionsenglish=.

\subsection{Customizing the date and captions}
The user might prefer different caption texts. If just a few texts are to be
changed, they can simply be redefined in the main document, for instace by:

\begin{verbatim}
\def\indexname{{\dn anukrama.nikaa}}
\end{verbatim}

This redefinition must appear \textbf{after} \verb=\captionshindi= or
\verb=\captionsmodernhindi= was invoked.

It is also possible to change all definitions. The source texts in the Velthuis
transliteration can be found in the \texttt{input} directory
(\verb=$TEXMF/tex/latex/devanagr=) in file
\texttt{captions.dn} with some suggested variants in comments. You can either
put modified definitions to your main document (after \textsf{devanagari.sty}) or to a
package of your own. Remember that the preprocessor will not see your package,
you must preprocess it separately. Your package must either reside in the same
directory as your document or in some directory which is searched by \LaTeX. In
the latter case you will have to rebuild the database by running
\textsf{mktexlsr} or \textsf{texhash} in many \TeX{} distributions.

Do not put your packages to standard distribution directories. You may lose
them when upgrading your \TeX{} distribution.

\renewcommand\thepage{{\dn\arabic{page}}}

\subsection{Using \texorpdfstring{{\dn d\?vnAgrF}}{Devanagari} in Sections and References}
All macros necessary for typesetting \devnag{} text are robust. The
section/chapter titles as well as figure and table captions can contain
\devnag{} words. However, the font is changed to the standard document
font before the title is typeset. It is therefore mandatory to use
\verb+\dn+ even if the section title appears inside the \verb+\dn+
environment. Thus, \verb+\chapter{{\dn mis paal}}+ will be printed
correctly while \verb+{\dn\chapter{mis paal}}+ will always
create garbage text. Section numbers as well as page numbers will be
printed in Roman numerals.

\subsection{\devnag{} Page Numbers}
Changing page numbers to print \devnag{} numerals is possible by
redefining the \verb+\thepage+ macro. This redefinition places the
\verb+page+ counter in the scope of the \verb+\dn+ delimiter. However,
note that \verb+\arabic{page}+ is enclosed within angle brackets. This
is required because the preprocessor does not recognize the counter
as a command. The pagination for this page till the end of Chapter~\ref{spec}
occurs in \devnag{} through
the following redefintion of \verb+\thepage+:

\begin{quote}
\verb+\renewcommand\thepage{{\dn<\arabic{page}>}}+
\end{quote}

Other counters can be printed in \devnag{} by redefinition macros. For
example, to change section numbering to \devnag{}, redefine
\verb+\thesection+ in a manner similar to \verb+\thepage+ above, using
the \verb+section+ counter instead of \verb+page+:

\begin{quote}
\verb+\renewcommand\thesection{{\dn<\arabic{section}>}}+
\end{quote}

\subsection{Category Codes}
\label{catcodes}
\TeX{} assigns a category code (\verb+\catcode+) to each character. For
example, normal characters are assigned to category 11, and because the
backslash belongs to category 0, it is treated as the first character of
macro commands.

The fonts in the \textsf{devnag} package use characters with codes below
32. In previous releases of the package the category of these characters was
to 11. However, these catcode assignments caused conflicts with some
packages and with tables where tab characters were used.
Most of these problems could be solved at the macro level, but
unfortunately not all of them. The most serious problem is that words
like {\dn v\char10 t} do not get correctly transfered from section
titles to the table-of-contents.

A modification of the preprocessor was necessary to resolve this
issue. As a result, a change of character categories is no longer
needed. The output of the revised preprocessor is compatible with
previous releases of \texttt{devanagari.sty}. This fix solves the
table-of-contents problem, but not the conflicts. The new
\texttt{devanagari.sty} is still able to process files generated by the
previous versions of the preprocessor.

To indicate which version of the preprocessor was used to process a
given \devnag{} file, a string is written to the beginning of the output
\TeX{} file. The macro definition \verb+\DevnagVersion+ is written to
the first line of the output file and indicates the preprocessor
version. If the whole document is present in a single file, the
definition will appear before reading the macro package. The package
then changes its behaviour according to the existence or non-existence
of the above mentioned macro. If the macro is defined, no categories are
changed. If the macro is undefined, the \textsf{devanagari} package assumes
that it is processing an output from an older version of the
preprocessor and the categories of the characters are changed.

The \textsf{nocatcodes} option is intended for use with files produced
by the old preprocessor. This option changes the categories only within
the \verb+\dn+ environment, not globally for the document. The
\textsf{catcodes} option instructs the package to change the categories
globally. This does not, however, change the categories as a part of
the \verb+\dn+ command. If you assume that the categories can be changed
somewhere in the middle of the document and you wish to set them
properly by \verb+\dn+, you can use the \textsf{compat} option. The
macro \verb+\UnDevCatcodes+ changes catcodes back to the normal values
within the \verb+\dn+ environment.

\clearpage % forced page break, remove it if the next section does not start
           % at the same page as the previous subsection
\section{Vedic Macros}

\renewcommand\thepage{{\arabic{page}}}

These macros put Vedic intonation marks above and below individual
Devanagari letters and construct other characters generally used only in
Vedic text.

There are two groups of these macros, one for Rig Veda, and one for Sama
Veda.  To use the Rig Veda macros, you must first enter the command
\verb=\dnveda=
at some point after \verb=\input dnmacs= in plain \TeX\ or after \verb=\usepackage{devanagari}= in 
\LaTeX , and to use the Sama Veda accents, you must first type \verb=\dnsamaveda=.

Both of these modes redefine standard macro names already used in Plain \TeX{}
and \LaTeX.  In Rig Veda mode the macros \verb=\_=, \verb+\=+, \verb=\|=, and
\verb=\~= are redefined,
while in Sama Veda mode, \verb=\^= and \verb=\@= are redefined. If your document already
uses these macros in their original sense, then use \verb=\dnveda= or
\verb=\dnsamaveda= only within \verb=\dn= mode.  Otherwise, use \verb=\dnveda=
or \verb=\dnsamaveda= once
at the beginning of the document.

This approach to macro names has been used because, when intonation marks
are needed, they are needed very frequently and are inserted into parts of
words, so the macro names should be very short and symbolic.

\subsection{Rig Veda Macros}

\subsubsection{Anudatta (low) tone macro\texorpdfstring{ \texttt{\char92\_}}{}, variable width}

This macro takes one argument, the text letter.
Example: \verb=\dnveda ... {\dn \_{a}gnim}=

This macro may be combined with \verb=\|= for a pluta mark: \verb=\_{\|{3}}=.

The anudatta mark produced by this macro is nearly as wide as the letter and
thus varies in width from one letter to another.


\subsubsection{Anudatta (low) tone macro\texorpdfstring{ \texttt{\char92=}}{}, fixed width}

This macro takes one argument, the text letter.
Example: \verb+\dnveda ... {\dn \={a}gnim}+

This macro may be combined with \verb=\|= for a pluta mark: \verb+\={\|{3}}+.

The anudatta mark produced by this macro has a fixed width and is centered
under the letter.


\subsubsection{Svarita (rising) tone macro\texorpdfstring{ \texttt{\char92|}}{}}

This macro takes one argument, the text letter.
Example: \verb=\dnveda ... {\dn \|{ii}Le}=

This macro may be combined with \verb=\|= for a pluta mark: \verb=\|{\_{3}}=.


\subsubsection{Pada separator macro\texorpdfstring{ \texttt{\char92\char126}}{}}

This macro inserts a pada separtor between two Devanagari letters.

Example: \verb=\dnveda ... {\dn na\_{ra}\~maa}=

\subsection{Usage Samples}
This subsection provides two small usage samples of Vedic macros.


{\dn\dnnum\Large\dncalcutta
{\dnveda

\_{a}E`n\|{mF}\30Fw\? \_{\7{p}}ro\|{Eh}t\2 \_{y}\3E2w\|{-y}
\_{d\?}v\_{\9{m}}E(v\|{j}\qq{m} . \
ho\|{tA}r\2 r\_{\3D7w}DA\|{t}m\qq{m} \quad \rn{1}

}}

The text above was typeset by:

\begin{verbatim}
{\dn\dnnum\Large\dncalcutta
{\dnveda

\_{a}gni\|{mii}Le \_{pu}ro\|{hi}ta.m \_{ya}j~na\|{sya}
\_{de}va\_{m.r}tvi\|{ja}m | \
ho\|{taa}ra.m ra\_{tna}dhaa\|{ta}mam \quad 1

}}
\end{verbatim}

\def\samaindent{\parindent=1.0in}
\def\dnitem#1{\noindent\llap{#1\space}\leftskip\parindent}
{\dn\dnnum\dncalcutta
{\dnsamaveda\samaindent
\dnitem{\rn{1}} \^{a}{\rn{2}}\^{`n}{\rn{3}} \^{aA}{\rn{1}} \^{yA}{\rn{2}}Eh
\^{vF}{\rn{3}}\^{t}{\rn{1}}\^{y\?}{\rn{2}} \9{g}\^{ZA}{\rn{3}}\^{no}{\rn{2}}
\^{h}{\rn{3}}\^{\326wy}{\rn{1}}\^{dA}{\rn{2}}ty\? \\
\^{En}{\rn{1}} ho\^{tA}{\rn{2}r} sE(s \^{b}{\rn{3}}\^{Eh\0}{\rn{1}}\^{Eq}{\rn{2}} \quad \rn{1}

}}

The text above was typeset by:

\begin{verbatim}
\def\samaindent{\parindent=1.0in}
\def\dnitem#1{\noindent\llap{#1\space}\leftskip\parindent}
{\dn\dnnum\dncalcutta
{\dnsamaveda\samaindent
\dnitem{1} \^{a}{2}\^{gna}{3} \^{aa}{1} \^{yaa}{2}hi
\^{vii}{3}\^{ta}{1}\^{ye}{2} g.r\^{.naa}{3}\^{no}{2}
\^{ha}{3}\^{vya}{1}\^{daa}{2}taye \\
\^{ni}{1} ho\^{taa}{2ra} satsi \^{ba}{3}\^{rhi}{1}\^{.si}{2} \quad 1

}}
\end{verbatim}

%%%%%%%%%%%%%%%%%%%%%%%%%%%%%%%%%%%%%%%%%%%%%%%%%%%%%%%%%%%%%%%%%%%%%%%%%%%%%

\begin{table}[pt]
\begin{center}
\renewcommand{\arraystretch}{1.35}
\begin{tabular}{|l|c|c|c|c|}
\hline
  & \textsc{original} & \textsc{bombay} & \textsc{calcutta} & \textsc{nepali} \\
\hline
\textit{a}       & {\dn a}   & {\dn\dnbombay a} & {\dn\dncalcutta a} & \\
\textit{\underrng{r}}  & {\dn \31Bw} & {\dn\dnbombay \31Bw} & {\dn\dncalcutta \31Bw} & \\
\textit{\diatop[\=|\underrng{r}]} & {\dn \311w} & {\dn\dnbombay \311w} & {\dn\dncalcutta \311w} & \\
\textit{\underrng{l}}  & {\dn \318w} & {\dn\dnbombay \318w} & {\dn\dncalcutta \318w} & \\
\textit{\diatop[\=|\underrng{l}]} & {\dn \319w} & {\dn\dnbombay \319w} & {\dn\dncalcutta \319w} & \\
\textit{cha}     & {\dn C}   & {\dn\dnbombay C} & {\dn\dncalcutta C} & \\
\textit{jha}     & {\dn J}   & {\dn\dnbombay J} & {\dn\dncalcutta J} & {\dn\dnnepali J} \\
\textit{\d{n}a}  & {\dn Z}   & {\dn\dnbombay Z} & {\dn\dncalcutta Z} & \\
\textit{la}      & {\dn l} & {\dn\dnbombay l} & {\dn\dncalcutta l} & \\
\textit{\'sa}    & {\dn f} & {\dn\dnbombay f} & {\dn\dncalcutta f} & \\
\textit{1}       & {\dn\dnnum 1}  & {\dn\dnnum\dnbombay 1} & {\dn\dnnum\dncalcutta 1} & {\dn\dnnum\dnnepali 1}\\
\textit{5}       & {\dn\dnnum 5}  & {\dn\dnnum\dnbombay 5} & {\dn\dnnum\dncalcutta 5} & \\
\textit{8}       & {\dn\dnnum 8}  & {\dn\dnnum\dnbombay 8} & {\dn\dnnum\dncalcutta 8} & \\
\textit{9}       & {\dn\dnnum 9}  & {\dn\dnnum\dnbombay 9} & {\dn\dnnum\dncalcutta 9} & {\dn\dnnum\dnnepali 9}\\
\hline
\end{tabular}
\end{center}
\bigskip

\begin{center}
\renewcommand{\arraystretch}{1.35}
\begin{tabular}{|l|c|c|c|c|}
\hline
  & \textsc{original} & \textsc{bombay} & \textsc{calcutta} & \textsc{nepali} \\
\hline
\textit{k\d{s}a} & {\dn "}   & {\dn\dnbombay "} & {\dn\dncalcutta "} & \\
\textit{k\d{s}}-    & {\dn \symbol{'43}} & {\dn\dnbombay \symbol{'43}} & {\dn\dncalcutta \symbol{'43}} & \\
\textit{\.nk\d{s}a}  & {\dn \3B0w}  & {\dn\dnbombay \3B0w} & {\dn\dncalcutta \3B0w} & \\
\textit{\.nk\d{s}va} & {\dn \3B1w}  & {\dn\dnbombay \3B1w} & {\dn\dncalcutta \3B1w} & \\
\textit{chya}        & {\dn \3D0w}  & {\dn\dnbombay \3D0w} & {\dn\dncalcutta \3D0w} & \\
\textit{j\~na}   & {\dn \3E2w}  & {\dn\dnbombay \3E2w} & {\dn\dncalcutta \3E2w} & \\
\textit{j\~n}-      & {\dn \symbol{'352}}  & {\dn\dnbombay \symbol{'352}} & {\dn\dncalcutta \symbol{'352}} & \\
\textit{jh}-         & {\dn \324w} & {\dn\dnbombay \324w} & {\dn\dncalcutta \324w} & {\dn\dnnepali \324w} \\
\textit{\d{n}}-      & {\dn \symbol{'27}} & {\dn\dnbombay \symbol{'27}} & {\dn\dncalcutta \symbol{'27}} & \\
\textit{\d{n}\d{n}a}  & {\dn \symbol{'233}} & {\dn\dnbombay \symbol{'233}} & {\dn\dncalcutta \symbol{'233}} & \\
\textit{lla}         & {\dn \3A5w}  & {\dn\dnbombay \3A5w} & {\dn\dncalcutta \3A5w} & \\
\textit{\'{s}}-     & {\dn \symbol{'133}} & {\dn\dnbombay \symbol{'133}} & {\dn\dncalcutta \symbol{'133}} & \\
\textit{-ya}        & {\dn \symbol{'053}} & {\dn\dnbombay \symbol{'053}} & {\dn\dncalcutta \symbol{'053}} & \\
\textit{h\d{n}a}     & {\dn \3A2w}  & {\dn\dnbombay \3A2w} & {\dn\dncalcutta \3A2w} & \\
\hline
\end{tabular}
\end{center}

\caption{Standard and Variant \devnag{} Characters}
\label{diffs}
\end{table}

%%%%%%%%%%%%%%%%%%%%%%%%%%%%%%%%%%%%%%%%%%%%%%%%%%%%%%%%%%%%%%%%%%%%%%%%%%%%%

\begin{center}
\begin{table}[p]
\renewcommand{\arraystretch}{1.35}
\begin{tabular}{|lllll||lllll||lllll|}
\hline
\# & & \textsc{s} & \textsc{h} & \textsc{mh} &
\# & & \textsc{s} & \textsc{h} & \textsc{mh} &
\# & & \textsc{s} & \textsc{h} & \textsc{mh} \\ \hline
1  & {\dn k k} & {\dn\symbol{'303}} & {\dn \3C3w} & {\dn ?k} &
36 & {\dn\symbol{'32} c} & {\dn\symbol{'321}} & {\dn \316wc} & {\dn \316wc} &
71 & {\dn d r y} & {\dn\symbol{'357}} & {\dn \3EFw} & {\dn \qb{d}+}\\
2  & {\dn k t } & {\dn\symbol{'304}} & {\dn \3C4w} & {\dn ?t} &
37 & {\dn\symbol{'32} j} & {\dn\symbol{'322}} & {\dn \316wj} & {\dn \316wj} &
72 & {\dn d v y} & {\dn\symbol{'225}} & {\dn \395w} & {\dn \392w+}\\
3  & {\dn k n} & {\dn\symbol{'307}} & {\dn ?n} & {\dn ?n} &
38 & {\dn V k} & {\dn\symbol{'326}} & {\dn \3D6w} & {\dn \qq{V}k} &
73 & {\dn D n} & {\dn\symbol{'360}} & {\dn @n} & {\dn @n}\\
4  & {\dn k m} & {\dn\symbol{'311}} & {\dn ?m} & {\dn ?m} &
39 & {\dn V V} & {\dn\symbol{'323}} & {\dn \3D3w} & {\dn \qq{V}V} &
74 & {\dn n n} & {\dn\symbol{'340}} & {\dn \3E0w} & {\dn \306wn}\\
5  & {\dn k y} & {\dn\symbol{'310}} & {\dn \3C8w} & {\dn ?y} &
40 & {\dn V W} & {\dn\symbol{'341}} & {\dn \3E1w} & {\dn \qq{V}W} &
75 & {\dn p t} & {\dn\symbol{'330}} & {\dn =t} & {\dn =t}\\
6  & {\dn k l} & {\dn\symbol{'312}} & {\dn ?l} & {\dn ?l} &
41 & {\dn V y} & {\dn\symbol{'324}} & {\dn \3D4w} & {\dn V+} &
76 & {\dn p n} & {\dn\symbol{'331}} & {\dn =n} & {\dn =n}\\
7  & {\dn k v} & {\dn\symbol{'313}} & {\dn ?v} & {\dn ?v} &
42 & {\dn W y} & {\dn\symbol{'325}} & {\dn \3D5w} & {\dn W+} &
77 & {\dn p l} & {\dn\symbol{'332}} & {\dn =l} & {\dn =l}\\
8  & {\dn k t y} & {\dn\symbol{'305}} & {\dn \3C5w} & {\dn ?(y} &
43 & {\dn X g} & {\dn\symbol{'263}} & {\dn \qq{X}g} & {\dn \qq{X}g} &
78 & {\dn b n} & {\dn\symbol{'247}} & {\dn Nn} & {\dn Nn}\\
9  & {\dn k t v} & {\dn\symbol{'306}} & {\dn \3C6w} & {\dn ?(v} &
44 & {\dn X G} & {\dn\symbol{'264}} & {\dn \qq{X}G} & {\dn \qq{X}G} &
79 & {\dn b b} & {\dn\symbol{'251}} & {\dn Nb} & {\dn Nb}\\
10 & {\dn k n y} & {\dn\symbol{'346}} & {\dn ?\306wy} & {\dn ?\306wy} &
45 & {\dn X X} & {\dn\symbol{'345}} & {\dn \3E5w} & {\dn \qq{X}X} &
80 & {\dn b v} & {\dn\symbol{'333}} & {\dn Nv} & {\dn Nv}\\
11 & {\dn k r y} & {\dn\symbol{'347}} & {\dn \3E7w} & {\dn \3E7w} &
46 & {\dn X m} & {\dn\symbol{'273}} & {\dn \qq{X}m} & {\dn \qq{X}m} &
81 & {\dn B n} & {\dn\symbol{'336}} & {\dn <n} & {\dn <n}\\
12 & {\dn k v y} & {\dn\symbol{'314}} & {\dn ?\&y} & {\dn ?\&y} &
47 & {\dn X y} & {\dn\symbol{'267}} & {\dn \3B7w} & {\dn X+} &
82 & {\dn m n} & {\dn\symbol{'337}} & {\dn Mn} & {\dn Mn}\\
13 & {\dn k t r y} & {\dn\symbol{'374}} & {\dn \3FCw} & {\dn ?\3ECwy} &
48 & {\dn X g y} & {\dn\symbol{'270}} & {\dn \qq{X}`y} & {\dn \qq{X}`y} &
83 & {\dn m l} & {\dn\symbol{'335}} & {\dn Ml} & {\dn Ml}\\
14 & {\dn G n} & {\dn\symbol{'315}} & {\dn \35Dwn} & {\dn \35Dwn} &
49 & {\dn X G r} & {\dn\symbol{'266}} & {\dn \qq{X}G\5} & {\dn \qq{X}G\5} &
84 & {\dn l l} & {\dn\symbol{'245}} & {\dn Sl} & {\dn Sl}\\
15 & {\dn R k} & {\dn\symbol{'254}} & {\dn \3ACw} & {\dn \qq{R}k} &
50 & {\dn X r y} & {\dn\symbol{'373}} & {\dn \3FBw} & {\dn \6{X}+} &
85 & {\dn v n} & {\dn\symbol{'246}} & {\dn \&n} & {\dn \&n}\\
16 & {\dn R K} & {\dn\symbol{'262}} & {\dn \3B2w} & {\dn \qq{R}K} &
51 & {\dn Y y} & {\dn\symbol{'344}} & {\dn \3E4w} & {\dn Y+} &
86 & {\dn v v} & {\dn\symbol{'250}} & {\dn \&v} & {\dn \&v}\\
17 & {\dn R g} & {\dn\symbol{'275}} & {\dn \3BDw} & {\dn \qq{R}g} &
52 & {\dn t t} & {\dn\symbol{'201}} & {\dn \381w} & {\dn \381w} &
87 & {\dn f c} & {\dn\symbol{'226}} & {\dn \35Bwc} & {\dn \35Bwc}\\
18 & {\dn R G} & {\dn\symbol{'277}} & {\dn \3BFw} & {\dn \qq{R}G} &
53 & {\dn t n} & {\dn\symbol{'327}} & {\dn (n} & {\dn (n} &
88 & {\dn f n} & {\dn\symbol{'227}} & {\dn \35Bwn} & {\dn \35Bwn}\\
19 & {\dn R R} & {\dn\symbol{'274}} & {\dn \3BCw} & {\dn \qq{R}R} &
54 & {\dn d g} & {\dn\symbol{'213}} & {\dn \qq{d}g} & {\dn \qq{d}g} &
% 89 & {\dn f b} & {\dn\symbol{'233}} & {\dn \35Bwb} & {\dn \35Bwb}\\
89 & ---       & ---                & ---         & ---           \\
20 & {\dn R n} & {\dn\symbol{'265}} & {\dn \3B5w} & {\dn \qq{R}n} &
55 & {\dn d G} & {\dn\symbol{'212}} & {\dn \qq{d}G} & {\dn \qq{d}G} &
90 & {\dn f l} & {\dn\symbol{'232}} & {\dn \35Bwl} & {\dn \35Bwl}\\
21 & {\dn R m} & {\dn\symbol{'301}} & {\dn \3C1w} & {\dn \qq{R}m} &
56 & {\dn d d} & {\dn\symbol{'214}} & {\dn \38Cw} & {\dn \38Cw} &
91 & {\dn f v} & {\dn\symbol{'230}} & {\dn \35Bwv} & {\dn \35Bwv}\\
22 & {\dn R y} & {\dn\symbol{'302}} & {\dn \3C2w} & {\dn R+} &
57 & {\dn d D} & {\dn\symbol{'210}} & {\dn \388w} & {\dn \388w} &
92 & {\dn q V} & {\dn\symbol{'243}} & {\dn \309wV} & {\dn \309wV}\\
23 & {\dn R k t} & {\dn\symbol{'255}} & {\dn \qq{\3ACw}t} & {\dn \qq{R}?t} &
58 & {\dn d n} & {\dn\symbol{'221}} & {\dn \qq{d}n} & {\dn \qq{d}n} &
93 & {\dn q W} & {\dn\symbol{'244}} & {\dn \309wW} & {\dn \309wW}\\
24 & {\dn R k y} & {\dn\symbol{'257}} & {\dn \3AFw} & {\dn \qq{R}?y} &
59 & {\dn d b} & {\dn\symbol{'223}} & {\dn \393w} & {\dn \393w} &
94 & {\dn q V y} & {\dn\symbol{'367}} & {\dn \309w\3D4w} & {\dn \309wV+}\\
25 & {\dn R k q} & {\dn\symbol{'260}} & {\dn \qq{\3ACw}q} & {\dn \qq{R}"} &
60 & {\dn d B} & {\dn\symbol{'211}} & {\dn \389w} & {\dn \qq{d}B} &
95 & {\dn q V v} & {\dn\symbol{'253}} & {\dn \309w\qq{V}v} & {\dn \309w\qq{V}v}\\
26 & {\dn R K y} & {\dn\symbol{'272}} & {\dn \3BAw} & {\dn \qq{R}Hy} &
61 & {\dn d m} & {\dn\symbol{'224}} & {\dn \qq{d}m} & {\dn \qq{d}m} &
96 & {\dn q V r y} & {\dn\symbol{'252}} & {\dn \3AAw} & {\dn \309w\6{V}+}\\
27 & {\dn R g y} & {\dn\symbol{'276}} & {\dn \3BEw} & {\dn \qq{R}`y} &
62 & {\dn d y} & {\dn\symbol{'215}} & {\dn \38Dw} & {\dn \38Dw} &
97 & {\dn s n} & {\dn\symbol{'334}} & {\dn -n} & {\dn -n}\\
28 & {\dn R G y} & {\dn\symbol{'271}} & {\dn \3B9w} & {\dn \qq{R}\35Dwy} &
63 & {\dn d v} & {\dn\symbol{'222}} & {\dn \392w} & {\dn \392w} &
98 & {\dn s r} & {\dn\symbol{'372}} & {\dn \3FAw} & {\dn \3FAw}\\
29 & {\dn R G r} & {\dn\symbol{'300}} & {\dn \3C0w} & {\dn \qq{R}G\5} &
64 & {\dn d g r} & {\dn\symbol{'355}} & {\dn \qq{d}g\5} & {\dn \qq{d}g\5} &
99 & {\dn h Z} & {\dn\symbol{'242}} & {\dn \3A2w} & {\dn \3A2w}\\
30 & {\dn R k t y} & {\dn\symbol{'256}} & {\dn \qq{\3ACw}(y} & {\dn \qq{R}?(y} &
65 & {\dn d G r} & {\dn\symbol{'356}} & {\dn \qq{d}G\5} & {\dn \qq{d}G\5} &
100 & {\dn h n} & {\dn\symbol{'241}} & {\dn \3A1w} & {\dn \3A1w}\\
31 & {\dn R k q v} & {\dn\symbol{'261}} & {\dn \qq{\3ACw}\309wv} & {\dn \qq{R}\#v} &
66 & {\dn d d y} & {\dn\symbol{'220}} & {\dn \390w} & {\dn \38Cw+} &
101 & {\dn h m} & {\dn\symbol{'234}} & {\dn \39Cw} & {\dn \39Cw}\\
32 & {\dn c c} & {\dn\symbol{'316}} & {\dn Qc} & {\dn Qc} &
67 & {\dn d d v} & {\dn\symbol{'370}} & {\dn \3F8w} & {\dn \qq{d}\392w} &
102 & {\dn h y} & {\dn\symbol{'235}} & {\dn \39Dw} & {\dn \39Dw}\\
33 & {\dn c \symbol{'32}} & {\dn\symbol{'317}} & {\dn Q\31Aw} & {\dn Q\31Aw} &
68 & {\dn d D y} & {\dn\symbol{'217}} & {\dn \38Fw} & {\dn \388w+} &
103 & {\dn h r} & {\dn\symbol{'240}} & {\dn \3A0w} & {\dn \3A0w}\\
34 & {\dn C y} & {\dn\symbol{'320}} & {\dn \3D0w} & {\dn C+} &
69 & {\dn d D v} & {\dn\symbol{'371}} & {\dn \3F9w} & {\dn \qq{d}@v} &
104 & {\dn h l} & {\dn\symbol{'236}} & {\dn \39Ew} & {\dn \39Ew}\\
35 & {\dn j r} & {\dn\symbol{'205}} & {\dn \385w} & {\dn \385w} &
70 & {\dn d B y} & {\dn\symbol{'216}} & {\dn \38Ew} & {\dn \qq{d}<y} &
105 & {\dn h v} & {\dn\symbol{'237}} & {\dn \39Fw} & {\dn \39Fw}\\
\hline
\end{tabular}
\caption{Supported \devnag{} Ligatures}
\label{ligs}
\end{table}
\end{center}

%%%%%%%%%%%%%%%%%%%%%%%%%%%%%%%%%%%%%%%%%%%%%%%%%%%%%%%%%%%%%%%%%%%%%%%%%%%%%

\begin{table}[pt]

\begin{center}
\begin{tabular}{ll}
\multicolumn{2}{c}{Regular} \\
& \\
Original (\texttt{dvng}) & {\dn \dnnum \large a \31Bw C f l \rn{5} \rn{8} Z \3A2w J \322w \317wX \35Bwl} \\
Bombay   (\texttt{dvnb}) & {\dn \dnnum \large \dnbombay a \31Bw C f l \rn{5} \rn{8} Z \3A2w J \322w \317wX \35Bwl} \\
Calcutta (\texttt{dvnc}) & {\dn \dnnum \large \dncalcutta a \31Bw C f l \rn{5} \rn{8} Z \3A2w J \322w \317wX \35Bwl} \\
Nepali   (\texttt{dvnn}) & {\dn \dnnum \large \dnnepali a \31Bw C f l \rn{5} \rn{8} Z \3A2w J \322w \317wX \35Bwl} \\
& \\ \hline
& \\
\multicolumn{2}{c}{Oblique} \\
& \\
Original (\texttt{dvngi}) & {\dn \dnnum \large \itshape a \31Bw C f l \rn{5} \rn{8} Z \3A2w J \322w \317wX \35Bwl} \\
Bombay   (\texttt{dvnbi}) & {\dn \dnnum \large \dnbombay \itshape a \31Bw C f l \rn{5} \rn{8} Z \3A2w J \322w \317wX \35Bwl} \\
Calcutta (\texttt{dvnci}) & {\dn \dnnum \large \dncalcutta \itshape a \31Bw C f l \rn{5} \rn{8} Z \3A2w J \322w \317wX \35Bwl} \\
Nepali   (\texttt{dvnni}) & {\dn \dnnum \large \dnnepali \itshape a \31Bw C f l \rn{5} \rn{8} Z \3A2w J \322w \317wX \35Bwl} \\
& \\ \hline
& \\
\multicolumn{2}{c}{Bold} \\
& \\
Original (\texttt{dvngb}) & {\dn \dnnum \large \bfseries a \31Bw C f l \rn{5} \rn{8} Z \3A2w J \322w \317wX \35Bwl} \\
Bombay   (\texttt{dvnbb}) & {\dn \dnnum \large \dnbombay \bfseries a \31Bw C f l \rn{5} \rn{8} Z \3A2w J \322w \317wX \35Bwl} \\
Calcutta (\texttt{dvncb}) & {\dn \dnnum \large \dncalcutta \bfseries a \31Bw C f l \rn{5} \rn{8} Z \3A2w J \322w \317wX \35Bwl} \\
Nepali   (\texttt{dvnnb}) & {\dn \dnnum \large \dnnepali \bfseries a \31Bw C f l \rn{5} \rn{8} Z \3A2w J \322w \317wX \35Bwl} \\
& \\ \hline
& \\
\multicolumn{2}{c}{Bold Oblique} \\
& \\
Original (\texttt{dvngbi}) & {\dn \dnnum \large \bfseries\itshape a \31Bw C f l \rn{5} \rn{8} Z \3A2w J \322w \317wX \35Bwl} \\
Bombay   (\texttt{dvnbbi}) & {\dn \dnnum \large \dnbombay \bfseries\itshape a \31Bw C f l \rn{5} \rn{8} Z \3A2w J \322w \317wX \35Bwl} \\
Calcutta (\texttt{dvncbi}) & {\dn \dnnum \large \dncalcutta \bfseries\itshape a \31Bw C f l \rn{5} \rn{8} Z \3A2w J \322w \317wX \35Bwl} \\
Nepali   (\texttt{dvnnbi}) & {\dn \dnnum \large \dnnepali \bfseries\itshape a \31Bw C f l \rn{5} \rn{8} Z \3A2w J \322w \317wX \35Bwl} \\
& \\ \hline
& \\
\multicolumn{2}{c}{Pen} \\
& \\
Original (\texttt{dvpn})  & {\dn \dnnum \large \dnpen a \31Bw C f l \rn{5} \rn{8} Z \3A2w J \317wX \35Bwl} \\
Bombay   (\texttt{dvpb})  & {\dn \dnnum \large \dnpenbombay a \31Bw C f l \rn{5} \rn{8} Z \3A2w J \317wX \35Bwl} \\
Calcutta (\texttt{dvpc})  & {\dn \dnnum \large \dnpencalcutta a \31Bw C f l \rn{5} \rn{8} Z \3A2w J \317wX \35Bwl} \\
Nepali   (\texttt{dvpnn}) & {\dn \dnnum \large \dnpennepali a \31Bw C f l \rn{5} \rn{8} Z \3A2w J \317wX \35Bwl} \\
\end{tabular}
\end{center}
\caption{\devnag{} Font Specimens}
\label{specs}
\end{table}
\newpage

\begin{table}[pt]

\begin{center}
Devanagari Regular
\end{center}

{\dn b\7{h}t s\? CoV\? CoV\? rAjAao\2 kF bol cAl kA Y\2g
BF{\rs ,\re} {\itshape Ejs smy v\? vA{}isrAy} s\? Emln\? aAe T\?{\rs ,\re}
{\bfseries s\2"\?p k\? sAT} ElKn\? k\? yo`y h\4. koI to
\8{d}r hF s\? {\itshape hAT jow\?} aAe{\rs ,\re} {\bfseries aOr do ek e\?s\?
T\? Ek} jb eEXkA\2g k\? bdn {\bfseries \itshape \7{J}kAkr
ifArA krn\?} pr BF u\306who{\qva} n\? {\bfseries slAm n EkyA}
to eEXkA\2g n\? {\itshape pFW pkw kr} u\306wh\?{\qva} DFr\? s\?
\7{J}kA EdyA.\par}
\bigskip

\begin{center}
Devanagari Bombay
\end{center}

{\dn \dnbombay b\7{h}t s\? CoV\? CoV\? rAjAao\2 kF bol cAl kA Y\2g
BF{\rs ,\re} {\itshape Ejs smy v\? vA{}isrAy} s\? Emln\? aAe T\?{\rs ,\re}
{\bfseries s\2"\?p k\? sAT} ElKn\? k\? yo`y h\4. koI to
\8{d}r hF s\? {\itshape hAT jow\?} aAe{\rs ,\re} {\bfseries aOr do ek e\?s\?
T\? Ek} jb eEXkA\2g k\? bdn {\bfseries \itshape \7{J}kAkr
ifArA krn\?} pr BF u\306who{\qva} n\? {\bfseries slAm n EkyA}
to eEXkA\2g n\? {\itshape pFW pkw kr} u\306wh\?{\qva} DFr\? s\?
\7{J}kA EdyA.\par}
\bigskip

\begin{center}
Devanagari Calcutta
\end{center}

{\dn \dncalcutta b\7{h}t s\? CoV\? CoV\? rAjAao\2 kF bol cAl kA Y\2g
BF{\rs ,\re} {\itshape Ejs smy v\? vA{}isrAy} s\? Emln\? aAe T\?{\rs ,\re}
{\bfseries s\2"\?p k\? sAT} ElKn\? k\? yo`y h\4. koI to
\8{d}r hF s\? {\itshape hAT jow\?} aAe{\rs ,\re} {\bfseries aOr do ek e\?s\?
T\? Ek} jb eEXkA\2g k\? bdn {\bfseries \itshape \7{J}kAkr
ifArA krn\?} pr BF u\306who{\qva} n\? {\bfseries slAm n EkyA}
to eEXkA\2g n\? {\itshape pFW pkw kr} u\306wh\?{\qva} DFr\? s\?
\7{J}kA EdyA.\par}
\bigskip

\begin{center}
Devanagari Pen Regular
\end{center}

{\dn \dnpen b\7{h}t s\? CoV\? CoV\? rAjAao\2 kF bol cAl kA Y\2g
BF{\rs ,\re} {\itshape Ejs smy v\? vA{}isrAy} s\? Emln\? aAe T\?{\rs ,\re}
{\bfseries s\2"\?p k\? sAT} ElKn\? k\? yo`y h\4. koI to
\8{d}r hF s\? {\itshape hAT jow\?} aAe{\rs ,\re} {\bfseries aOr do ek e\?s\?
T\? Ek} jb eEXkA\2g k\? bdn {\bfseries \itshape \7{J}kAkr
ifArA krn\?} pr BF u\306who{\qva} n\? {\bfseries slAm n EkyA}
to eEXkA\2g n\? {\itshape pFW pkw kr} u\306wh\?{\qva} DFr\? s\?
\7{J}kA EdyA.\par}
\bigskip

\begin{center}
Devanagari Pen Bombay
\end{center}

{\dn \dnpenbombay b\7{h}t s\? CoV\? CoV\? rAjAao\2 kF bol cAl kA Y\2g
BF{\rs ,\re} {\itshape Ejs smy v\? vA{}isrAy} s\? Emln\? aAe T\?{\rs ,\re}
{\bfseries s\2"\?p k\? sAT} ElKn\? k\? yo`y h\4. koI to
\8{d}r hF s\? {\itshape hAT jow\?} aAe{\rs ,\re} {\bfseries aOr do ek e\?s\?
T\? Ek} jb eEXkA\2g k\? bdn {\bfseries \itshape \7{J}kAkr
ifArA krn\?} pr BF u\306who{\qva} n\? {\bfseries slAm n EkyA}
to eEXkA\2g n\? {\itshape pFW pkw kr} u\306wh\?{\qva} DFr\? s\?
\7{J}kA EdyA.\par}
\bigskip

\begin{center}
Devanagari Pen Calcutta
\end{center}

{\dn \dnpencalcutta b\7{h}t s\? CoV\? CoV\? rAjAao\2 kF bol cAl kA Y\2g
BF{\rs ,\re} {\itshape Ejs smy v\? vA{}isrAy} s\? Emln\? aAe T\?{\rs ,\re}
{\bfseries s\2"\?p k\? sAT} ElKn\? k\? yo`y h\4. koI to
\8{d}r hF s\? {\itshape hAT jow\?} aAe{\rs ,\re} {\bfseries aOr do ek e\?s\?
T\? Ek} jb eEXkA\2g k\? bdn {\bfseries \itshape \7{J}kAkr
ifArA krn\?} pr BF u\306who{\qva} n\? {\bfseries slAm n EkyA}
to eEXkA\2g n\? {\itshape pFW pkw kr} u\306wh\?{\qva} DFr\? s\?
\7{J}kA EdyA.\par}
\bigskip

\begin{center}
Devanagari Pen Nepali
\end{center}

{\dn \dnpennepali b\7{h}t s\? CoV\? CoV\? rAjAao\2 kF bol cAl kA Y\2g
BF{\rs ,\re} {\itshape Ejs smy v\? vA{}isrAy} s\? Emln\? aAe T\?{\rs ,\re}
{\bfseries s\2"\?p k\? sAT} ElKn\? k\? yo`y h\4. koI to
\8{d}r hF s\? {\itshape hAT jow\?} aAe{\rs ,\re} {\bfseries aOr do ek e\?s\?
T\? Ek} jb eEXkA\2g k\? bdn {\bfseries \itshape \7{J}kAkr
ifArA krn\?} pr BF u\306who{\qva} n\? {\bfseries slAm n EkyA}
to eEXkA\2g n\? {\itshape pFW pkw kr} u\306wh\?{\qva} DFr\? s\?
\7{J}kA EdyA.\par}
\bigskip

% \begin{center}
% Source text
% \end{center}

% \begin{verbatim}
% @hindi
% {\dn bahut se cho.te cho.te raajaao.m kii bol caal kaa .dha.mg bhii,
% {\itshape jis samay ve vaa{}isaraay} se milane aae the,
% {\bfseries sa.mk.sep ke saath} likhane ke yogya hai| koii to duur
% hii se {\itshape haath joRe} aae, {\bfseries aur do ek aise the ki}
% jab e.dikaa.mg ke badan {\bfseries \itshape jhukaakar i"saaraa karane}
% par bhii unho.m ne {\bfseries salaam na kiyaa} to e.dikaa.mg ne
% {\itshape pii.th pakaR kar} unhe.m dhiire se jhukaa diyaa|}
% \end{verbatim}

\caption{Examples of \devnag{} Faces}
\label{mixedface}
\end{table}

\end{document}
