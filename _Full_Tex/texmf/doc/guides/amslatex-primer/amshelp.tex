%%% amshelp.tex
%%% This is an AMS-LaTeX file

%%% Copyright (c) 2000 Philip S. Hirschhorn
%
% This document may be distributed and/or modified under the
% conditions of the LaTeX Project Public License, either version 1.2
% of this license or (at your option) any later version.
% The latest version of this license is in
%   http://www.latex-project.org/lppl.txt
% and version 1.2 or later is part of all distributions of LaTeX 
% version 1999/12/01 or later.
%


%%% Philip Hirschhorn
%%% Department of Mathematics
%%% Wellesley College
%%% Wellesley, MA 02481
%%% psh@poincare.wellesley.edu
%%% psh@math.mit.edu


%%% This is an attempt to explain how to get up and running with
%%% AmS-LaTeX, if you've already got some familiarity with either 
%%% TeX or AMS-TeX or LaTeX.




\newcommand{\filedate}{August 10, 2000}
\newcommand{\fileversion}{Version 2.0}

%---------------------------------------------------------------------
\documentclass[12pt]{amsart}
%\documentclass{amsart}
\usepackage{amscd}


\newcommand{\tensor}{\otimes}
\newcommand{\homotopic}{\simeq}
\newcommand{\homeq}{\cong}
\newcommand{\iso}{\approx}
\newcommand{\ho}{\operatorname{Ho}}




\newcommand{\C}{{\cal C}}
\newcommand{\M}{{\cal M}}
\newcommand{\W}{{\cal W}}


\newenvironment{technical}{\begingroup \small}{\endgroup}



%---------------------------------------------------------------------
%---------------------------------------------------------------------

\numberwithin{equation}{section}


%       Theorem environments

\theoremstyle{plain} %% This is the default, anyway
\newtheorem{thm}[equation]{Theorem}
\newtheorem{cor}[equation]{Corollary}
\newtheorem{lem}[equation]{Lemma}
\newtheorem{prop}[equation]{Proposition}


\theoremstyle{definition}
\newtheorem{defn}[equation]{Definition}

\theoremstyle{remark}
\newtheorem{rem}[equation]{Remark}
\newtheorem{ex}[equation]{Example}
\newtheorem{notation}[equation]{Notation}
\newtheorem{terminology}[equation]{Terminology}




%---------------------------------------------------------------------
%---------------------------------------------------------------------
%---------------------------------------------------------------------
%---------------------------------------------------------------------

\begin{document}


\title[Running \AmS-\LaTeX]
{Getting up and running\\
with \AmS-\LaTeX}

\author{Philip S. Hirschhorn}

\address{Department of Mathematics\\
         Wellesley College\\
         Wellesley, Massachusetts 02481}

\email{psh@math.mit.edu}


\date{\filedate, \fileversion}


\begin{abstract}
  This is an attempt to tell you enough about \LaTeX{} and
  \AmS-\LaTeX{} so that you can get started with it \emph{without}
  having to read the book.
\end{abstract}

\maketitle




\tableofcontents
%---------------------------------------------------------------------
%---------------------------------------------------------------------
\section{Introduction}

This is an attempt to get you up and running with \AmS-\LaTeX{} with
the least possible pain.  These instructions won't be a substitute for
the User's Guide, but they may get you started quickly enough so that
you'll only need to refer to the guide occasionally, which should
eliminate most of the pain.

The current version of \AmS-\LaTeX{} (version 1.2) is really just an
optional package for the new standard \LaTeX.  \AmS-\LaTeX{} provides
the document classes \verb"amsart" and \verb"amsbook" (see
Section~\ref{sec:DocClsCom}) to replace the standard document classes
\verb"article" and \verb"book", and an optional package \verb"amsmath"
that can be used with the standard \LaTeX{} document classes.  Thus,
using \AmS-\LaTeX{} is really using a variety of \LaTeX.  If you're
new to \LaTeX{}, and these last few sentences made no sense to you at
all, don't worry about it.  You don't have to know what the standard
\LaTeX{} document classes are in order to use the \AmS-\LaTeX{}
replacements for them.


I'll be assuming that you have at least some experience with either
plain \TeX, \AmS-\TeX{} or \LaTeX, and I'll try to tell you what you
need to know so that you can get started with \AmS-\LaTeX{}
\emph{without} actually reading the \LaTeX{} User's
Guide~\cite{latex}, or even taking much of a look at the \AmS-\LaTeX{}
User's Guide~\cite{amslatexusersguide}.


If you've never used \emph{any} version of \TeX, then I recommend
``The not so short introduction to \LaTeXe{}'', by Tobias Oetiker,
Hubert Partl, Irene Hyna, and Elisabeth Schlegl \cite{NotShort}.  This
is intended for those with no knowledge of \TeX{} or \LaTeX, and
concisely gives a description of what a \LaTeX{} document looks like
and how you type text and simple mathematics in a \LaTeX{} document.

I've also given you a template file \verb"template.tex", which is an
attempt to give you enough to mostly fake your way through an
\AmS-\LaTeX{} file, \emph{almost} without even reading these
instructions.  I've included the text of that file in these
instructions as Section~\ref{sec:template}, so you might want to take a
look at that now, and then just use the table of contents of these
instructions to get more information on whatever in that file
confuses you.

In case you haven't guessed, these instructions were printed using
\AmS-\LaTeX, so you can get some idea what it all looks like.



%---------------------------------------------------------------------
%---------------------------------------------------------------------
\section{Basic \LaTeX{} stuff}
\label{sec:basicstuff}

In this section, we'll describe the three commands that have to be
part of any \LaTeX{} document: \verb"\documentclass",
\verb"\begin{document}", and \verb"\end{document}".  The complete
explanation of these can be found in the \LaTeX{} User's
Guide~\cite{latex} or in \emph{The not so short introduction to
  \LaTeXe}~\cite{NotShort}.  We'll also explain how to begin a new
section or subsection of the paper, and how \LaTeX{} manages to get
the cross-references right (which is also the explanation of why you
need to run a file through \LaTeX{} \emph{twice} to be sure that all
the cross-references are correct).


\subsection{The  \texttt{documentclass} command}
\label{sec:DocClsCom}

Before you type anything that actually appears in the paper, you must
include a \verb"\documentclass" command.  It's easiest to just put the
\verb"\documentclass" command at the very beginning of the file,
possibly with a few lines of comments before it.

It's actually the choice of document class that determines whether
you're using \AmS-\LaTeX{} or just plain old \LaTeX.  There are two
document classes that are a part of \AmS-\LaTeX: \verb"amsart" and
\verb"amsbook".  There is also the \verb"amsmath" package that can be
used with the standard \LaTeX{} \verb"article" document class.  I'll
only be discussing the \verb"amsart" document class here.  For the
others, see the \AmS-\LaTeX{} User's Guide~\cite{amslatexusersguide}.


The simplest version of the \verb"documentclass" command is
\begin{center}
\verb"\documentclass{amsart}"
\end{center}
This will give you the default type size, which is 10~point type.  If
you'd like to use 12~point type, then you should include the optional
argument \verb"[12pt]"; this makes the command
\begin{center}
\verb"\documentclass[12pt]{amsart}"
\end{center}


There are at least two optional packages that might be of interest.
The first is for when you want to include the macros that make it
easier to draw commutative diagrams.  (These aren't included
automatically, since they take up a lot of memory, and not everyone
wants to use them.)  If you want 10~point type and you want to use the
commutative diagram macros, then the commands to use are
\begin{center}
  \begin{tabular}{l}
    \verb"\documentclass{amsart}"\\
    \verb"\usepackage{amscd}"
  \end{tabular}
\end{center}
If you want 12~point type and you want to use the commutative
diagram macros, then the commands are
\begin{center}
  \begin{tabular}{l}
    \verb"\documentclass[12pt]{amsart}"\\
    \verb"\usepackage{amscd}"
  \end{tabular}
\end{center}


The other optional package is for use when you want to use some of the
special symbols contained in the \AmS-Fonts package.  (These are the
fonts \verb"msam" and \verb"msbm".)  If you want the standard names
for these symbols to be defined for your use, then you need to use the
optional package \verb"amssymb".  Thus, to use the default 10~point
type and have the special symbols defined, use the commands
\begin{center}
  \begin{tabular}{l}
    \verb"\documentclass{amsart}"\\
    \verb"\usepackage{amssymb}"
  \end{tabular}
\end{center}
If you want to use 12~point type, the commutative diagram macros, and
special symbols from the \AmS-Fonts collection, then use the commands
\begin{center}
  \begin{tabular}{l}
    \verb"\documentclass[12pt]{amsart}"\\
    \verb"\usepackage{amssymb}"
  \end{tabular}
\end{center}
This document uses both of these packages, and so we used the commands
\begin{center}
  \begin{tabular}{l}
    \verb"\documentclass[12pt]{amsart}"\\
    \verb"\usepackage{amscd}"\\
    \verb"\usepackage{amssymb}"
  \end{tabular}
\end{center}

%---------------------------------------------------------------------

\subsection{\texttt{begin\{document\}} and \texttt{end\{document\}}}

Everything that is to appear in the document must appear in between
the \verb"\begin{document}" and \verb"\end{document}" commands. There
are no optional arguments for these commands, so they always look the
same.  Anything following the \verb"\end{document}" command is
ignored.  You \emph{are} allowed to have macro definitions (i.e.,
newcommands; see Section~\ref{sec:definitions}) before the
\verb"\begin{document}", and that's actually a good place for them,
  but that's about all.

%---------------------------------------------------------------------
\subsection{Sections and subsections}
\label{sec:sections}

To begin a new section, you give the command
\begin{center}
\verb"\section{Section name}"
\end{center}
To begin the present section, I gave the command
\begin{center}
\verb"\section{Basic \LaTeX{} stuff}"
\end{center}
A section number is supplied automatically.  If you want to be able to
make reference to that section, then you need to \emph{label} it.
Since I wanted to be able to demonstrate the cross-reference commands,
I actually began this section with the lines
\begin{center}
  \begin{tabular}{l}
    \verb"\section{Basic \LaTeX{} stuff}"\\
    \verb"\label{sec:basicstuff}"
  \end{tabular}
\end{center}
This allows me to say ``\verb"Section~\ref{sec:basicstuff}"'' and have
it printed as Section~\ref{sec:basicstuff}.

To begin a new subsection, you give the command
\begin{center}
\verb"\subsection{Subsection name}"
\end{center}
To begin the present subsection, I gave the command
\begin{center}
\verb"\subsection{Sections and subsections}"
\end{center}
A subsection number is supplied automatically.  If you want to be able
to make reference to that subsection, then you need to \emph{label}
it.  This subsection was begun with the lines
\begin{center}
  \begin{tabular}{l}
    \verb"\subsection{Sections and subsections}"\\
    \verb"\label{sec:sections}"
  \end{tabular}
\end{center}
so if we say ``\verb"Section~\ref{sec:sections}",'' it is printed as
Section~\ref{sec:sections}.

Labels always take the number of the smallest enclosing structure.
Thus, a \verb"\label" command that's inside a section but \emph{not}
inside a subsection or Theorem or anything else will take the value of
the section counter, while a \verb"\label" command that's inside the
statement of a Theorem will take the value of that Theorem number.
For more information on this, see Section~\ref{sec:xreferences}.

%--------------------------------------------------------------------
\subsubsection{Yes, there are subsubsections too}

I began this subsubsection with the command
\begin{center}
\verb"\subsubsection{Yes, there are subsubsections too}"
\end{center}
I refuse to even experiment to see if there are subsubsubsections.



%--------------------------------------------------------------------
\subsubsection*{Sections without numbers}

I began this subsubsection with the command
\begin{center}
\verb"\subsubsection*{Sections without numbers}"
\end{center}
and got a subsubsection that wasn't numbered.  If you give the command
\begin{center}
\verb"\section*{A Section Title}"
\end{center}
then you'll begin a new section that will not have a number.



%---------------------------------------------------------------------
\subsection{Italics \emph{for emphasis}}

If you want to use italics to emphasize a word or two, the \LaTeX{}
convention is not to switch explicitly to italics, but rather to use
the command \verb"\emph" (which means \emph{emphasize}).  This command
works just like a font change command, except that it switches you
\emph{into} italics if the current font is upright, and switches you
\emph{out of} italics if the current font is italics.

For example, if you type
\begin{center}
\verb"The whole is \emph{more} than the sum of its parts."
\end{center}
you'll get
\begin{center}
The whole is \emph{more} than the sum of its parts.
\end{center}
but if you type
\begin{verbatim}
\begin{thm}
  The whole is \emph{more} than the sum of its parts.
\end{thm}
\end{verbatim}
you'll get
\begin{thm}
  The whole is \emph{more} than the sum of its parts.
\end{thm}


\subsubsection*{Note}

The \verb"\emph" command is a recent addition to \LaTeX, and it has
the feature that it automatically inserts an italic correction where
needed.  If you don't know what an italic correction is, you can
safely ignore this paragraph (but I will at least mention that all
those ``\verb"\/"'' commands frequently seen in \TeX{} (and older
\LaTeX) documents are all inserting italic corrections; the point of
this paragraph is that, with the current version of \LaTeX, you don't
have to do that anymore).



%---------------------------------------------------------------------
\subsection{Once is never enough}

This is an explanation of how \LaTeX{} manages to fill in
cross-references to parts of the file it hasn't processed yet, and
what those \verb".aux" and \verb".toc" files are.


\subsubsection*{Cross-References}

Every time \LaTeX{} processes your file, it writes an \emph{auxiliary}
file.  Since the file containing these instructions is called
\verb"amshelp.tex", the auxiliary file is called \verb"amshelp.aux".
The auxiliary file contains the definitions of all the keys used for
cross-references.  When \LaTeX{} begins to process your file, it first
looks for an \verb".aux" file, and reads it in if it exists.  Of
course, this is the \verb".aux" file that was produced the \emph{last}
time that your file was processed, so the Theorem numbers, Section
numbers, etc., are all the ones from the last time the file was
processed.

The very first time that \LaTeX{} processes your file, there is no
\verb".aux" file, and so \LaTeX{} gives \emph{lots} of warning
messages about undefined labels, or whatever.  Ignore all of this.
The \emph{next} time that you run \LaTeX, there \emph{will} be an
\verb".aux" file, and all the references will be filled in.  (Yes, it
is possible, at least in theory, for some page number to change every
time you run \LaTeX{} on your file, even without any changes in the
source file, but this isn't very likely.)

\subsubsection*{The Table of Contents}

If you give the command \verb"\tableofcontents", then \LaTeX{} will
try to write a table of contents, including the page numbers of the
sections.  Obviously, \LaTeX{} can't know those page numbers or
section titles yet, so as \LaTeX{} processes your file, it writes a
\verb".toc" file containing the information it needs.  (The
\verb".toc" file for these instructions is \verb"amshelp.toc".)  Once
again, \LaTeX{} is always using the information from the \emph{last}
time that it processed your file.

If you \emph{do} include a table of contents in your document, and if
the table of contents takes up at least a page or so of space, then
you might have to run \LaTeX{} \emph{three} times in order to get all
of the cross-references right.  The reason for this is that the first
time you run \LaTeX{} there isn't any \verb".toc" file listing the
section titles, and so the table of contents has nothing in it.  The
second time you run \LaTeX{} you'll get a table of contents that lists
the page numbers for the sections from the last time you ran \LaTeX,
when the table of contents took up no space at all.  Unfortunately,
during this second run, the table of contents will be created, and
will take up enough space to change the page numbers of the sections
from what they were during the first run.  Only during the
\emph{third} run will the correct page numbers be written into the
table of contents.  Since this doesn't change the amount of space that
the table of contents occupies, this version will be correct.

\subsubsection*{How do I know when everything is correct?}

After processing your file, \LaTeX{} checks whether all the
cross-reference numbers that it read from the \verb".aux" file are
correct.  If any of them are incorrect, it prints a warning on the
screen at the very end of the run advising you that labels may have
changed, and that you should run \LaTeX{} again to get the
cross-references right.  Unfortunately, \LaTeX{} doesn't seem to check
that the table of contents entries are correct, so if you change the
name of a section in a way that doesn't make any page references
incorrect, you won't be warned to run \LaTeX{} again.





%---------------------------------------------------------------------
%---------------------------------------------------------------------
\section{Title, Author,  and the \texttt{maketitle} command}

This stuff should go right after the \verb"\begin{document}" command.
  I'll give a quick sketch here, which is probably all you'll ever
  need, but the full explanation is given in \emph{Instructions for
    preparation of papers and monographs: \AmS-\LaTeX} \cite{instr-l}.
  If you are already familiar with \LaTeX, then you should be warned
  that this part is slightly different from what you do when using the
  standard \LaTeX{} \verb"article" documentclass.

%---------------------------------------------------------------------

\subsection{The title}
You announce the title with the
command
\begin{center}
\verb"\title[Optional running title]{Actual title}"
\end{center}
These instructions used the title command
\begin{center}
  \begin{tabular}{l}
    \verb"\title[Running \AmS-\LaTeX]"\\
    \verb"{Getting up and running\\"\\
    \verb"with \AmS-\LaTeX}"
  \end{tabular}
\end{center}
Notice that you indicate line breaks in the title with a double
backslash.  If I had decided to let the \emph{full} title be printed
in the head of the odd numbered pages, I would have used the command
\begin{center}
  \begin{tabular}{l}
    \verb"\title{Getting up and running\\"\\
    \verb"with \AmS-\LaTeX}"
  \end{tabular}
\end{center}



%---------------------------------------------------------------------
\subsection{The author, and the author's address}


The author is specified with an \verb"author" command:
\begin{center}
\verb"\author{Author's name}"
\end{center}
These directions used the command
\verb"\author{Philip S. Hirschhorn}".
The author's address is given in an address command, with double
backslashes to indicate line breaks.  These instructions used the
command
\begin{center}
  \begin{tabular}{l}
    \verb"\address{Department of Mathematics\\"\\
    \verb"Wellesley College\\"\\
    \verb"Wellesley, Massachusetts 02481}"
  \end{tabular}
\end{center}
If the author's current address is different from the address at which
the research was carried out, then you can specify the current address
with the command \verb"\curraddr".  For example, you might say
\begin{center}
  \begin{tabular}{l}
    \verb"\curraddr{Department of Mechanics\\"\\
    \verb"Brake and Wheel Bearing Division\\"\\
    \verb"Serene Service Center\\"\\
    \verb"Salem, Massachusetts 02139}"
  \end{tabular}
\end{center}
You can also include an email address, with the \verb"\email"
command.  These instructions used the command
\begin{center}
  \verb"\email{psh@math.mit.edu}"
\end{center}
To acknowledge support, use the command \verb"\thanks", e.g.,
\begin{center}
  \verb"\thanks{Supported in part by NSF grant 3.14159}"
\end{center}
This will be printed as a footnote on the first page.





\subsubsection*{Multiple authors}

If there are several authors, then each one should have a separate
\verb"\author" command, with each individual's address (and current
address, and email address) following that individual's \verb"\author"
command, in its own \verb"\address" command (and \verb"\curraddr"
command, and \verb"\email" command).  If there \emph{are} several
authors, and their combined names are too long for the running head on
the even numbered pages, you can give an optional argument to each
\verb"\author" command to supply a shortened form to use in the
running head.  (It's apparently a convention that the running head in
a multiple author paper should have only initials for the first and
middle names, but I don't think that I was invited to that
convention.)


%---------------------------------------------------------------------

\subsection{The date}

This is pretty straightforward:
\begin{center}
\verb"\date{Whatever date you please}"
\end{center}
To have the date of processing used, use the command
\verb"\date{\today}".

%---------------------------------------------------------------------
\subsection{\texttt{maketitle}}


After you've given all of the commands mentioned in this section, you
can give the command \verb"\maketitle".  The exact arrangement is
determined by the document class.  In particular, the \verb"amsart"
document class puts the author's address at the \emph{end} of the
paper.  If you \emph{don't} give the command \verb"\maketitle", a
title won't be made.



%---------------------------------------------------------------------
%---------------------------------------------------------------------
\section{Theorems, Propositions, Lemmas, etc.}

The instructions in this section assume that you're using the
\verb"newtheorem" commands that I put in the file
\verb"template.tex".

\subsection{Stating theorems, propositions, etc.}

To state a theorem, you do the following:
\begin{verbatim}
\begin{thm}
The square of the hypotenuse of a right triangle
is equal to the sum of the squares of the two
adjacent sides.
\end{thm}
\end{verbatim}
If you do that, you'll get the following:
\begin{thm}
\label{pythagthm}
The square of the hypotenuse of a right triangle
is equal to the sum of the squares of the two
adjacent sides.
\end{thm}
If you thought that it was only a proposition, you'd use
\begin{verbatim}
\begin{prop}
The square of the hypotenuse of a right triangle
is equal to the sum of the squares of the two
adjacent sides.
\end{prop}
\end{verbatim}
and you'd get
\begin{prop}
The square of the hypotenuse of a right triangle
is equal to the sum of the squares of the two
adjacent sides.
\end{prop}

If you think it's a theorem again, but you'd like to make reference to
it in some other part of the paper, you have to choose a \emph{key}
with which you'll refer to it, and then \emph{label} the theorem.  If
you want to use the key \emph{pythagthm}, then it would look like the
following:
\begin{verbatim}
\begin{thm}
\label{pythagthm}
The square of the hypotenuse of a right triangle
is equal to the sum of the squares of the two
adjacent sides.
\end{thm}
\end{verbatim}
If you later give the command \verb"\ref{pythagthm}", then that
command will expand to the \emph{number} that was assigned to that
theorem (in this case, \ref{pythagthm}).  For more explanation of
cross-references, see Section~\ref{sec:xreferences}.

If you'd like to state a theorem and give a \emph{name} to it, then
you can add an optional argument to the \verb"\begin{thm}" command. 
If you type
\begin{verbatim}
\begin{thm}[Pythagorus]
The square of the hypotenuse of a right triangle
is equal to the sum of the squares of the two
adjacent sides.
\end{thm}
\end{verbatim}
you'll get
\begin{thm}[Pythagorus]
The square of the hypotenuse of a right triangle
is equal to the sum of the squares of the two
adjacent sides.
\end{thm}


\subsubsection*{Summary of environments provided in the template}

All of the following structures are numbered in the same sequence,
in the form SectionNumber.Number.  Equations (i.e., displayed
formulas, whether they are equations or not) will be numbered in the
same sequence.

$$
\begin{tabular}{c@{\hspace{4em}}l@{\hspace{4em}}c}

\multicolumn{3}{c}{Theorem Environments}\\*[8pt]

\hspace{1em}Name&  Printed Form& Body font\\*[6pt]

\texttt{thm}&           \textbf{Theorem}&      Italic\\
\texttt{cor}&           \textbf{Corollary}&    Italic\\
\texttt{lem}&           \textbf{Lemma}&        Italic\\
\texttt{prop}&          \textbf{Proposition}&  Italic\\
\texttt{defn}&          \textbf{Definition}&   Normal\\
\texttt{rem}&           \textit{Remark}&       Normal\\
\texttt{ex}&            \textit{Example}&      Normal\\
\texttt{notation}&      \textit{Notation}&     Normal\\
\texttt{terminology}&   \textit{Terminology}&  Normal\\*[3pt]
\end{tabular}
$$

For full details, see the beginning of the template file (reproduced
here in Section~\ref{sec:template}), after the comment ``The Theorem
Environments.''



%---------------------------------------------------------------------
\subsection{Proofs}

To give a proof, you do the following:
\begin{verbatim}
\begin{proof}
As any fool can plainly see, it's true!
\end{proof}
\end{verbatim}
and you'll get the following:
\begin{proof}
As any fool can plainly see, it's true!
\end{proof}
If the theorem said that a condition was both necessary and sufficient
for something, and you want to prove each part separately,
you can do the following:
\begin{verbatim}
\begin{proof}[Proof (sufficiency)]
Well, it's \emph{obviously} sufficient!
\end{proof}
\end{verbatim}
and you'll get
\begin{proof}[Proof (sufficiency)]
Well, it's \emph{obviously} sufficient!
\end{proof}
that is, the \verb"proof" environment allows you to use an optional
second argument that will appear in place of the word \verb"Proof".

If the proof of Theorem~\ref{pythagthm} does not appear immediately
after its statement, you might use the following:
\begin{verbatim}
\begin{proof}[Proof of Theorem~\ref{pythagthm}]
As any fool can plainly see, it's true!
\end{proof}
\end{verbatim}
and you'd get
\begin{proof}[Proof of Theorem~\ref{pythagthm}]
As any fool can plainly see, it's true!
\end{proof}



%---------------------------------------------------------------------
%---------------------------------------------------------------------
\section{Cross-References}
\label{sec:xreferences}

This section explains how to make reference to numbered sections,
theorems, equations, and bibliography items, with the correct
reference numbers filled in automatically by \LaTeX.

%---------------------------------------------------------------------
\subsection{References to sections, theorems and equations}
\label{sec:thmrefs}

For each structure in the manuscript to which you'll be making
reference, you must assign a \emph{key} that you'll use to refer to
that structure.  For sections, theorems and numbered equations, you
assign the key using the \verb"\label" command.  This command takes
one argument, which is the \emph{key} you're assigning to the object.
The command \verb"\ref{key}" then produces the number that was
assigned to that structure.  If the structure is an equation, then the
command \verb"\eqref{key}" should be used instead of the command
\verb"\ref{key}".


Consider the following example.
\begin{thm}
\label{homotopy}
If the maps $f\colon X \to Y$ and $g\colon X \to Y$ are homotopic,
then $f_* = g_* \colon \mathrm{H}_*X \to \mathrm{H}_*Y$.
\end{thm}

We typed this theorem as follows.
\begin{verbatim}
\begin{thm}
\label{homotopy}
If the maps $f\colon X \to Y$ and $g\colon X \to Y$ are
homotopic, then 
$f_* = g_* \colon \mathrm{H}_*X \to \mathrm{H}_*Y$.
\end{thm}
\end{verbatim}
If we now type ``\verb"see Theorem~\ref{homotopy}",'' then it will be
printed as ``see Theorem~\ref{homotopy}.''





\subsubsection*{So, what exactly is the label labeling?}

We began this section by typing
\begin{center}
  \begin{tabular}{l}
    \verb"\section{Cross-References}"\\
    \verb"\label{sec:xreferences}"
  \end{tabular}
\end{center}
and we began this subsection by typing
\begin{center}
  \begin{tabular}{l}
    \verb"\subsection{References to sections, theorems and equations}"\\
    \verb"\label{sec:thmrefs}"
  \end{tabular}
\end{center}
The phrase ``\verb"See Section~\ref{sec:xreferences}"'' is printed as
``See Section~\ref{sec:xreferences},'' while the phrase 
``\verb"See Section~\ref{sec:thmrefs}"'' is printed as ``See
Section~\ref{sec:thmrefs}.''


The command \verb"\label{key}" assigns to \verb"key" the value of the
\emph{smallest enclosing structure}.  That's why the command
\verb"\ref{sec:xreferences}" is printed as~\ref{sec:xreferences},
while \verb"\ref{sec:thmrefs}" is printed as~\ref{sec:thmrefs}: the
key \verb"sec:xreferences" was defined inside of
Section~\ref{sec:xreferences} but \emph{outside} of
Section~\ref{sec:thmrefs}, while the key \verb"sec:thmrefs" was
defined \emph{inside} of Section~\ref{sec:thmrefs}.


\subsubsection*{References to equations}

To make reference to a numbered equation, you assign the \emph{key}
as before, but you replace \verb"\ref" with \verb"\eqref".  For
example, if you type
\begin{verbatim}
\begin{equation}
\label{additivity}
\mathrm{H}_* \bigvee_{\alpha\in A} X_\alpha   \iso
          \bigoplus_{\alpha\in A}\mathrm{H}_* X_\alpha
\end{equation}
\end{verbatim}
then you'll get
\begin{equation}
\label{additivity}
\mathrm{H}_* \bigvee_{\alpha\in A} X_\alpha   \iso
          \bigoplus_{\alpha\in A}\mathrm{H}_* X_\alpha
\end{equation}
If we now say
\begin{verbatim}
\begin{thm}
Equation~\eqref{additivity} is true for all sorts of
functors $\mathrm{H}$.
\end{thm}
\end{verbatim}
then we'll get
\begin{thm}
Equation~\eqref{additivity} is true for all sorts of
functors $\mathrm{H}$.
\end{thm}
Notice the parentheses around the equation number, and the fact that
even though the theorem is set in slanted type, the equation
number is set in an upright font.  This is the advantage of using
\verb"\eqref" instead of \verb"\ref"; the command \verb"\eqref"
arranges it so that the number and surrounding parentheses are in an
upright font no matter what the surrounding font, and supplies an
italic correction if it's needed.


%---------------------------------------------------------------------
\subsection{References to page numbers}

If you want to make reference to the \emph{page} that contains a
label, rather than to the structure that is labeled, use the command
\verb"\pageref{key}".  For example, if you type
\begin{verbatim}
Look at page~\pageref{homotopy} to find
Theorem~\ref{homotopy}.
\end{verbatim}
you'll get ``Look at page~\pageref{homotopy} to find
Theorem~\ref{homotopy}.''







%---------------------------------------------------------------------
\subsection{Bibliographic references}
\label{sec:bibreferences}


Bibliography entries receive a \emph{key} as part of their basic
structure.  Each item in the bibliography is entered as
\begin{verbatim}
\bibitem{key} The actual bibliography item goes here.
\end{verbatim}
(For more detail on this, see Section~\ref{sec:bibliography}.)

You refer to bibliography items using the \verb"\cite" command.  For
example, the bibliography of these instructions contains the entry
%
\begin{verbatim}
\bibitem{HA}
D. G. Quillen, \emph{Homotopical Algebra,} Lecture Notes in
Mathematics number 43, Springer-Verlag, Berlin, 1967.
\end{verbatim}
%
If we say ``\verb"This is the work of Quillen~\cite{HA}",'' then it
will be printed as ``This is the work of Quillen~\cite{HA}.''  Notice
that square brackets have been inserted around the bibliography item
number.

The \verb"\cite" command takes an optional argument, which allows you
to annotate the reference.  If we say
``\verb"see~\cite[Chapter I]{HA}"'', then it will be printed as
``see~\cite[Chapter I]{HA}''.





%---------------------------------------------------------------------
%---------------------------------------------------------------------
\section{Mathematics}


\subsection{Mathematics in running text}

This is pretty much exactly as it is in plain \TeX, except that you
have an extra option (which you can ignore).  The simplest thing is to
just enclose between dollar signs (\verb"$") any material that should
be in math mode.  Thus, if you type
\begin{center}
\verb"Let $f\colon X \to Y$ be a continuous function."
\end{center}
you'll get
\begin{center}
Let $f\colon X \to Y$ be a continuous function.
\end{center}

The only novelty that \LaTeX{} introduces is that, instead of using
a dollar sign  to toggle math mode on and off, you can use
`\verb"\("' to \emph{begin} math mode, and `\verb"\)"' to \emph{end}
math mode.  Thus, the example above is equivalent to typing
\begin{center}
\verb"Let \(f\colon X \to Y\) be a continuous function."
\end{center}
This provides a tiny bit more error checking, but can
otherwise be safely ignored.


%---------------------------------------------------------------------
\subsection{Displayed mathematics}

For simple displayed mathematics without an equation number, this is
very much like plain \TeX, again with extra choices that can be
ignored.  If you enclose material between double dollar
signs(\verb"$$"), it will be interpreted in math mode and displayed.
Thus, if you've previously given the command
\begin{center}
  \verb"\newcommand{\iso}{\approx}"
\end{center}
(see Section~\ref{sec:definitions}), and you type
\begin{verbatim}
$$
\pi_1(X \vee Y) \iso \pi_1X * \pi_1Y
$$
\end{verbatim}
you'll get
$$
\pi_1(X \vee Y) \iso \pi_1X * \pi_1Y
$$
The new choices are that exactly the same thing will be obtained by
either
\begin{verbatim}
\[
\pi_1(X \vee Y) \iso \pi_1X * \pi_1Y
\]
\end{verbatim}
or by
\begin{verbatim}
\begin{displaymath}
\pi_1(X \vee Y) \iso \pi_1X * \pi_1Y
\end{displaymath}
\end{verbatim}
or by
\begin{verbatim}
\begin{equation*}
\pi_1(X \vee Y) \iso \pi_1X * \pi_1Y
\end{equation*}
\end{verbatim}


If you'd like the displayed formula to be \emph{numbered}, then you
should use the \verb"equation" environment.  (\LaTeX{} calls all
formula numbers \emph{equation numbers}, whether or not the
mathematics has anything to do with equations.)  If you type
\begin{verbatim}
\begin{equation}
\pi_1(X \vee Y) \iso \pi_1X * \pi_1Y
\end{equation}
\end{verbatim}
you'll get
\begin{equation}
\label{pi1eqn}
\pi_1(X \vee Y) \iso \pi_1X * \pi_1Y
\end{equation}
(Notice that the \verb"equation" environment produces an equation
number, while the \verb"equation*" environment doesn't.  This is a
standard \LaTeX ism: Adding an asterisk to the name of a numbered 
\LaTeX{} environment often gives the unnumbered equivalent.)


If you'd like to be able to make reference to the equation number,
you need to \emph{label} the equation, using a \emph{key} that you
can use for referencing it:
\begin{verbatim}
\begin{equation}
\label{pi1eqn}
\pi_1(X \vee Y) \iso \pi_1X * \pi_1Y
\end{equation}
\end{verbatim}
If you later type ``\verb"see formula~\eqref{pi1eqn}"'' you'll get 
``see formula~\eqref{pi1eqn}.''  (For more on cross-references to
formulas, see Section~\ref{sec:thmrefs}.)

\AmS-\LaTeX{} has several environments that make it easier to typeset
complicated multi-line displays.  These are explained in the
\AmS-\LaTeX{} User's Guide~\cite{amslatexusersguide} and illustrated
in \verb"testmath.tex"~\cite[pages 29--40]{testmath}.



%---------------------------------------------------------------------
\subsection{Commutative diagrams}

\AmS-\LaTeX{} provides the \verb"CD" environment for drawing
commutative diagrams.  These only allow for \emph{rectangular}
diagrams, but they're very convenient to use.  If you need diagonal
arrows, or curving arrows, or arrows that make a number of right angle
turns, I recommend the XY-Pic package, which can do all of these
things (and more).

\textbf{Important}: To use the \verb"CD" environment, you need to load
it with the command \verb"\usepackage{amscd}".  For example, the
template file gives the commands
\begin{center}
  \begin{tabular}{l}
    \verb"\documentclass[12pt]{amsart}"\\
    \verb"\usepackage{amscd}"
  \end{tabular}
\end{center}
which select the \verb"amsart" document class with twelve point type,
and loads the macros for the \verb"CD" environment.

\subsubsection*{A simple example}

To produce the diagram
$$
\begin{CD}
A    @>>>  B   @=     B\\
@AAA       @|         @VVV\\
X    @<<<  B   @>>>   Y
\end{CD}
$$
you type
\begin{verbatim}
$$
\begin{CD}
A    @>>>  B   @=     B\\
@AAA       @|         @VVV\\
X    @<<<  B   @>>>   Y
\end{CD}
$$
\end{verbatim}
This illustrates several things.  First of all, the \verb"CD"
environment must be inside of a displayed mathematics environment.
(Here we used the standard \verb"$$" toggle to get displayed
mathematics.  If we had used, e.g., \verb"\begin{equation}" and
  \verb"\end{equation}", we would have had an equation number assigned
to the display.)  Right pointing arrows are obtained with \verb"@>>>",
left pointing arrows with \verb"@<<<", up pointing arrows with
\verb"@AAA", down pointing arrows with \verb"@VVV", horizontal equals
signs with \verb"@=", and vertical equals signs with \verb"@|".  Every
line except the last is ended with a double backslash (\verb"\\").

\subsubsection*{Labeling the arrows}

The arrows can also be labeled.  For horizontal arrows, anything
between the first and second inequality sign goes above the arrow,
and anything between the second and third inequality sign goes below
it.  For downward arrows, anything between the first and second
\verb"V" goes to the left, and anything between the second and third
goes to the right (and similarly for upward arrows).  Thus, if you
type
\begin{verbatim}
\begin{displaymath}
\begin{CD}
\mathrm{H}_iX    @>f_*>>   \mathrm{H}_iY   @<g_*<\iso<    E_fA\\
@V\phi VV                  @V\psi VV                 @AA\Omega A\\
\pi_iQ  @>\alpha\beta\gamma>>  \pi_i(R,S)
                   @<<\text{A long arrow}<   \prod_{k=1}^i H_kZ
\end{CD}
\end{displaymath}
\end{verbatim}
you'll get
\begin{displaymath}
\begin{CD}
\mathrm{H}_iX  @>f_*>>  \mathrm{H}_iY   @<g_*<\iso<    E_fA\\
@V\phi VV               @V\psi VV                @AA\Omega A\\
\pi_iQ   @>\alpha\beta\gamma>>    \pi_i(R,S)
                   @<<\text{A long arrow}<   \prod_{k=1}^i H_kZ
\end{CD}
\end{displaymath}


\subsubsection*{Leaving out part of the rectangle}

If you want to end a line in the diagram early (omitting the right
end of that line), just type the double backslash.  If you want to
leave out the \emph{beginning} of a line, you type ``\verb"@."''
(the ``at'' sign followed by a period) to denote an invisible arrow. 
(It's the arrows that are the column markers in the \verb"CD"
environment.)  Thus, if you type
\begin{verbatim}
\begin{displaymath}
\begin{CD}
X\\
@VfVV\\
Y  @=     Y\\
@.       @VVgV\\
  @.     Z
\end{CD}
\end{displaymath}
\end{verbatim}
you'll get
\begin{displaymath}
\begin{CD}
X\\
@VfVV\\
Y  @=     Y\\
@.       @VVgV\\
  @.     Z
\end{CD}
\end{displaymath}





%---------------------------------------------------------------------
%---------------------------------------------------------------------
\section{Macro definitions, a.k.a.\ \texttt{newcommand}}
\label{sec:definitions}


\LaTeX{} allows you to use the same \verb"\def" command that you use
in plain \TeX, but it's considered bad style.  Instead, \LaTeX{} has
the \verb"\newcommand" and \verb"\renewcommand" commands, which do a
little error checking for you.  In plain \TeX, you might use the
command
\begin{center}
\verb"\def\tensor{\otimes}"
\end{center}
but in \LaTeX, the preferred form is
\begin{center}
\verb"\newcommand{\tensor}{\otimes}"
\end{center}
The advantage of this is that \LaTeX{} will check to see if there
already is a command with the name \verb"\tensor", and give you an
error message if there is.  If you know that there is a previous
definition of \verb"\tensor" but you \emph{want} to override it,
then you use the command
\begin{center}
\verb"\renewcommand{\tensor}{\otimes}"
\end{center}


If you want to use macros with replaceable parameters, the
\verb"newcommand" command allows this.  For the equivalent of the
plain \TeX{} command
\begin{center}
\verb"\def\pushout#1#2#3{#1\cup_{#2}#3}"
\end{center}
you use the \LaTeX{} command
\begin{center}
\verb"\newcommand{\pushout}[3]{#1\cup_{#2}#3}"
\end{center}
i.e., the command name is enclosed in braces, and the number of
parameters is enclosed in square brackets.



%---------------------------------------------------------------------
%---------------------------------------------------------------------
\section{Lists: \texttt{itemize, enumerate, and description}}

There are three list making environments: \texttt{itemize},
\texttt{enumerate}, and \texttt{description}.  The \texttt{itemize}
environment just lists the items with a marker in front of each one. 
If you type
%
\begin{verbatim}
\begin{itemize}
\item
This is the first item in the list, which runs on long enough to
spill over onto a second line.
\item
This is the second item in the list, which is a bit shorter.
\item
This is the last item.
\end{itemize}
\end{verbatim}
%
then you'll get
%
\begin{itemize}
\item
This is the first item in the list, which runs on long enough to
spill over onto a second line.
\item
This is the second item in the list, which is a bit shorter.
\item
This is the last item.
\end{itemize}



The \texttt{enumerate} environment looks the same, except that the items
in the list are numbered.  If you type
\begin{verbatim}
\begin{enumerate}
\item
This is the first item in the list, which runs on long enough to
spill over onto a second line.
\item
This is the second item in the list, which is a bit shorter.
\item
This is the last item.
\end{enumerate}
\end{verbatim}
then you'll get
\begin{enumerate}
\item
This is the first item in the list, which runs on long enough to
spill over onto a second line.
\item
This is the second item in the list, which is a bit shorter.
\item
This is the last item.
\end{enumerate}

The \texttt{description} environment requires an extra argument for each
\verb"\item" command, which will be printed at the beginning of the
item.  If you type
\begin{verbatim}
\begin{description}
\item[sedge]
A green plant, found in both wetlands and uplands.
Sedges are often confused with grasses and rushes.
\item[grass]
A green plant, found in both wetlands and uplands.
Grasses are often confused with sedges and rushes.
\item[rush]
A green plant, found in both wetlands and uplands.
Rushes are often confused with sedges and grasses
\end{description}
\end{verbatim}
you'll get
\begin{description}
\item[sedge]
A green plant, found in both wetlands and uplands.  Sedges are often
confused with grasses and rushes.
\item[grass]
A green plant, found in both wetlands and uplands.  Grasses are often
confused with sedges and rushes.
\item[rush]
A green plant, found in both wetlands and uplands.  Rushes are often
confused with sedges and grasses
\end{description}


These environments can be inserted within each other, and the
\verb"enumerate" environment keeps track of what level it's at, and
numbers its items accordingly.  If you type
\begin{verbatim}
\begin{enumerate}
\item I went to the dry cleaners.
\item I went to the supermarket.  I bought
\begin{enumerate}
\item Bread,
\item cheese, and
\item Tabasco sauce.
\end{enumerate}
\item I went to the bank.
\end{enumerate}
\end{verbatim}
%
you'll get
%
\begin{enumerate}
\item I went to the dry cleaners.
\item I went to the supermarket.  I bought
\begin{enumerate}
\item Bread,
\item cheese, and
\item Tabasco sauce.
\end{enumerate}
\item I went to the bank.
\end{enumerate}








%---------------------------------------------------------------------
%---------------------------------------------------------------------
\section{The bibliography}
\label{sec:bibliography}

\subsection{\texttt{begin\{thebibliography\}} and 
  \texttt{end\{thebibliography\}}}


The bibliography is begun with the command
\begin{center}
\verb"\begin{thebibliography}{number}"
\end{center}
where \emph{number} is a random number that, when printed, is as
wide as the widest number of any item in the bibliography.  (The only
use made of \verb"number" is that \LaTeX{} assumes that the numbers
that it will assign to the actual items in the bibliography will be
no wider (when printed) than \verb"number".) For example, if the
bibliography will contain between 10 and~19 items, you can use
\verb"\begin{thebibliography}{10}".

After listing each item in the bibliography, you end the bibliography
with the \verb"\end{thebibliography}" command.


\subsection{Bibliography items}

Each item is begun with a \verb"\bibitem" command.  The format is
\begin{center}
\verb"\bibitem{key for cross-references}Item entry"
\end{center}
For example, the bibliography in these instructions contains the
entry
%
\begin{verbatim}
\bibitem{yellowmonster}
A. K. Bousfield and D. M. Kan, \emph{Homotopy Limits,
Completions and Localizations,} Lecture Notes in
Mathematics number 304, Springer-Verlag, New York, 1972.
\end{verbatim}


The above entry allows you to say
\begin{verbatim}
Homotopy inverse limits are discussed
in~\cite[Chapter 11]{yellowmonster}.
\end{verbatim}
and have it print as
``Homotopy inverse limits are discussed
in~\cite[Chapter 11]{yellowmonster}.''
For more on this, see Section~\ref{sec:bibreferences}.

















%---------------------------------------------------------------------
%---------------------------------------------------------------------
\section{The template file}
\label{sec:template}

The following is the text of the file \verb"template.tex".

\begin{verbatim}
%%% template.tex
%%% This is a template for making up an AMS-LaTeX file
%%% Version of August 10, 2000
%%%---------------------------------------------------------
%%% The following commands choose 12 point type (instead
%%% of the default 10 point), allow us to use the
%%% commutative diagram macros, and define the standard
%%% names for all of the special symbols in the AMSfonts
%%% package:
\documentclass[12pt]{amsart}
\usepackage{amscd}
\usepackage{amssymb}





%%% This part of the file (after the \documentclass command,
%%% but before the \begin{document}) is called the ``preamble''.
%%% This is a good place to  put our macro definitions.

\newcommand{\tensor}{\otimes}
\newcommand{\homotopic}{\simeq}
\newcommand{\homeq}{\cong}
\newcommand{\iso}{\approx}

\DeclareMathOperator{\ho}{Ho}
\DeclareMathOperator*{\colim}{colim}



\newcommand{\R}{\mathbb{R}}
\newcommand{\C}{\mathbb{C}}
\newcommand{\Z}{\mathbb{Z}}

\newcommand{\M}{\mathcal{M}}
\newcommand{\W}{\mathcal{W}}



\newcommand{\itilde}{\tilde{\imath}}
\newcommand{\jtilde}{\tilde{\jmath}}
\newcommand{\ihat}{\hat{\imath}}
\newcommand{\jhat}{\hat{\jmath}}



%%%-------------------------------------------------------------------
%%%-------------------------------------------------------------------
%%% The Theorem environments:
%%%
%%%
%%% The following commands set it up so that:
%%% 
%%% All Theorems, Corollaries, Lemmas, Propositions, Definitions,
%%% Remarks, Examples, Notations, and Terminologies  will be numbered
%%% in a single sequence, and the numbering will be within each
%%% section.  Displayed equations will be numbered in the same
%%% sequence. 
%%% 
%%% 
%%% Theorems, Propositions, Lemmas, and Corollaries will have the most
%%% formal typesetting.
%%% 
%%% Definitions will have the next level of formality.
%%% 
%%% Remarks, Examples, Notations, and Terminologies will be the least
%%% formal.
%%% 
%%% Theorem:
%%% \begin{thm}
%%% 
%%% \end{thm}
%%% 
%%% Corollary:
%%% \begin{cor}
%%% 
%%% \end{cor}
%%% 
%%% Lemma:
%%% \begin{lem}
%%% 
%%% \end{lem}
%%% 
%%% Proposition:
%%% \begin{prop}
%%% 
%%% \end{prop}
%%% 
%%% Definition:
%%% \begin{defn}
%%% 
%%% \end{defn}
%%% 
%%% Remark:
%%% \begin{rem}
%%% 
%%% \end{rem}
%%% 
%%% Example:
%%% \begin{ex}
%%% 
%%% \end{ex}
%%% 
%%% Notation:
%%% \begin{notation}
%%% 
%%% \end{notation}
%%% 
%%% Terminology:
%%% \begin{terminology}
%%% 
%%% \end{terminology}
%%% 
%%%       Theorem environments

% The following causes equations to be numbered within sections
\numberwithin{equation}{section}

% We'll use the equation counter for all our theorem environments, so
% that everything will be numbered in the same sequence.

%       Theorem environments

\theoremstyle{plain} %% This is the default, anyway
\newtheorem{thm}[equation]{Theorem}
\newtheorem{cor}[equation]{Corollary}
\newtheorem{lem}[equation]{Lemma}
\newtheorem{prop}[equation]{Proposition}


\theoremstyle{definition}
\newtheorem{defn}[equation]{Definition}

\theoremstyle{remark}
\newtheorem{rem}[equation]{Remark}
\newtheorem{ex}[equation]{Example}
\newtheorem{notation}[equation]{Notation}
\newtheorem{terminology}[equation]{Terminology}


%%%-------------------------------------------------------------------
%%%-------------------------------------------------------------------
%%%-------------------------------------------------------------------
%%%-------------------------------------------------------------------
%%%-------------------------------------------------------------------
%%%-------------------------------------------------------------------
%%%-------------------------------------------------------------------
\begin{document}

%%% In the title, use a double backslash "\\" to show a linebreak:
%%% Use one of the following two forms:
%%% \title{Text of the title}
%%% or
%%% \title[Short form for the running head]{Text of the title}
\title




\author{}



%%% In the address, show linebreaks with double backslashes:
\address{}




%%% Email address is optional.
\email{}

%%% To have the current date inserted, use \date{\today}:
\date{}


\maketitle

%%% To include a table of contents, uncomment the next line:
% \tableofcontents
%%%-------------------------------------------------------------------
%%%-------------------------------------------------------------------
%%% Start the body of the paper here!  E.G., maybe use:
%%% \section{Introduction}
%%% \label{sec:intro}

















%%%-------------------------------------------------------------------
%%%-------------------------------------------------------------------
%%% The number "10" that appears in the next command is a TOTALLY
%%% RANDOM NUMBER which is chosen so that if it was printed, it would
%%% be at least as wide as any number of an item in the bibliography:

\begin{thebibliography}{10}




%%% The format of bibliography items is as in the following examples:
%%% 
%%% \bibitem{yellowmonster}
%%% A. K. Bousfield and D. M. Kan, \emph{Homotopy Limits, Completions
%%% and Localizations,} Lecture Notes in Mathematics number 304,
%%% Springer-Verlag, New York, 1972.
%%% 
%%% \bibitem{HA}
%%% D. G. Quillen, \emph{Homotopical Algebra,} Lecture Notes in
%%% Mathematics number 43, Springer-Verlag, Berlin, 1967.








\end{thebibliography}
\end{document}
\end{verbatim}











%---------------------------------------------------------------------
%---------------------------------------------------------------------

\begin{thebibliography}{10}


\bibitem{amslatexusersguide} 
  American Mathematical Society, \emph{\AmS-\LaTeX{} Version 1.2 User's
    Guide,} November, 1996.  This is the file \texttt{amsldoc.dvi},
  available from the AMS ftp site \texttt{e-math.ams.org}.

\bibitem{instr-l} 
  American Mathematical Society, \emph{Instructions for preparation of
    papers and monographs: \AmS-\LaTeX,} November, 1966.  This is the
  file \texttt{instr-l.dvi}, available from the AMS ftp site
  \texttt{e-math.ams.org}.

\bibitem{testmath}
  American Mathematical Society, \emph{Sample paper for the amsmath
    package,} November, 1996.  This is the file \texttt{testmath.tex},
  available from the AMS ftp site \texttt{e-math.ams.org}.

\bibitem{latex}
  Leslie Lamport, \emph{\LaTeX{} User's Guide and Reference Manual,}
  Addison-Wesley, 1986.

\bibitem{NotShort} 
  Tobias Oetiker, Hubert Partl, Irene Hyna, and Elisabeth Schlegl
  \emph{The not so short introduction to \LaTeXe}, available by ftp
  from CTAN (the Comprehensive \TeX{} archive network), at
  \verb"ftp.tex.ac.uk", \verb"ftp.dante.de", and many mirrors, under
  the name \verb"lshort2e.tex" or \verb"lshort2e.dvi".

\bibitem{yellowmonster}
  A. K. Bousfield and D. M. Kan, \emph{Homotopy Limits, Completions
    and Localizations,} Lecture Notes in Mathematics number 304,
  Springer-Verlag, New York, 1972.

\bibitem{HA}
  D. G. Quillen, \emph{Homotopical Algebra,} Lecture Notes in
  Mathematics number 43, Springer-Verlag, Berlin, 1967.


\end{thebibliography}
\end{document}

