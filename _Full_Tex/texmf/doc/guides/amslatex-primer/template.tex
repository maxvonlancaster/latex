%%% template.tex
%%% This is a template for making up an AMS-LaTeX file
%%% Version of August 10, 2000
%%%---------------------------------------------------------
%%% The following commands choose 12 point type (instead
%%% of the default 10 point), allow us to use the
%%% commutative diagram macros, and define the standard
%%% names for all of the special symbols in the AMSfonts
%%% package:
\documentclass[12pt]{amsart}
\usepackage{amscd}
\usepackage{amssymb}





%%% This part of the file (after the \documentclass command,
%%% but before the \begin{document}) is called the ``preamble''.
%%% This is a good place to  put our macro definitions.

\newcommand{\tensor}{\otimes}
\newcommand{\homotopic}{\simeq}
\newcommand{\homeq}{\cong}
\newcommand{\iso}{\approx}

\DeclareMathOperator{\ho}{Ho}
\DeclareMathOperator*{\colim}{colim}



\newcommand{\R}{\mathbb{R}}
\newcommand{\C}{\mathbb{C}}
\newcommand{\Z}{\mathbb{Z}}

\newcommand{\M}{\mathcal{M}}
\newcommand{\W}{\mathcal{W}}



\newcommand{\itilde}{\tilde{\imath}}
\newcommand{\jtilde}{\tilde{\jmath}}
\newcommand{\ihat}{\hat{\imath}}
\newcommand{\jhat}{\hat{\jmath}}



%%%-------------------------------------------------------------------
%%%-------------------------------------------------------------------
%%% The Theorem environments:
%%%
%%%
%%% The following commands set it up so that:
%%% 
%%% All Theorems, Corollaries, Lemmas, Propositions, Definitions,
%%% Remarks, Examples, Notations, and Terminologies  will be numbered
%%% in a single sequence, and the numbering will be within each
%%% section.  Displayed equations will be numbered in the same
%%% sequence. 
%%% 
%%% 
%%% Theorems, Propositions, Lemmas, and Corollaries will have the most
%%% formal typesetting.
%%% 
%%% Definitions will have the next level of formality.
%%% 
%%% Remarks, Examples, Notations, and Terminologies will be the least
%%% formal.
%%% 
%%% Theorem:
%%% \begin{thm}
%%% 
%%% \end{thm}
%%% 
%%% Corollary:
%%% \begin{cor}
%%% 
%%% \end{cor}
%%% 
%%% Lemma:
%%% \begin{lem}
%%% 
%%% \end{lem}
%%% 
%%% Proposition:
%%% \begin{prop}
%%% 
%%% \end{prop}
%%% 
%%% Definition:
%%% \begin{defn}
%%% 
%%% \end{defn}
%%% 
%%% Remark:
%%% \begin{rem}
%%% 
%%% \end{rem}
%%% 
%%% Example:
%%% \begin{ex}
%%% 
%%% \end{ex}
%%% 
%%% Notation:
%%% \begin{notation}
%%% 
%%% \end{notation}
%%% 
%%% Terminology:
%%% \begin{terminology}
%%% 
%%% \end{terminology}
%%% 
%%%       Theorem environments

% The following causes equations to be numbered within sections
\numberwithin{equation}{section}

% We'll use the equation counter for all our theorem environments, so
% that everything will be numbered in the same sequence.

%       Theorem environments

\theoremstyle{plain} %% This is the default, anyway
\newtheorem{thm}[equation]{Theorem}
\newtheorem{cor}[equation]{Corollary}
\newtheorem{lem}[equation]{Lemma}
\newtheorem{prop}[equation]{Proposition}


\theoremstyle{definition}
\newtheorem{defn}[equation]{Definition}

\theoremstyle{remark}
\newtheorem{rem}[equation]{Remark}
\newtheorem{ex}[equation]{Example}
\newtheorem{notation}[equation]{Notation}
\newtheorem{terminology}[equation]{Terminology}


%%%-------------------------------------------------------------------
%%%-------------------------------------------------------------------
%%%-------------------------------------------------------------------
%%%-------------------------------------------------------------------
%%%-------------------------------------------------------------------
%%%-------------------------------------------------------------------
%%%-------------------------------------------------------------------
\begin{document}

%%% In the title, use a double backslash "\\" to show a linebreak:
%%% Use one of the following two forms:
%%% \title{Text of the title}
%%% or
%%% \title[Short form for the running head]{Text of the title}
\title




\author{}



%%% In the address, show linebreaks with double backslashes:
\address{}




%%% Email address is optional.
\email{}

%%% To have the current date inserted, use \date{\today}:
\date{}


\maketitle

%%% To include a table of contents, uncomment the next line:
% \tableofcontents
%%%-------------------------------------------------------------------
%%%-------------------------------------------------------------------
%%% Start the body of the paper here!  E.G., maybe use:
%%% \section{Introduction}
%%% \label{sec:intro}

















%%%-------------------------------------------------------------------
%%%-------------------------------------------------------------------
%%% The number "10" that appears in the next command is a TOTALLY
%%% RANDOM NUMBER which is chosen so that if it was printed, it would
%%% be at least as wide as any number of an item in the bibliography:

\begin{thebibliography}{10}




%%% The format of bibliography items is as in the following examples:
%%% 
%%% \bibitem{yellowmonster}
%%% A. K. Bousfield and D. M. Kan, \emph{Homotopy Limits, Completions
%%% and Localizations,} Lecture Notes in Mathematics number 304,
%%% Springer-Verlag, New York, 1972.
%%% 
%%% \bibitem{HA}
%%% D. G. Quillen, \emph{Homotopical Algebra,} Lecture Notes in
%%% Mathematics number 43, Springer-Verlag, Berlin, 1967.








\end{thebibliography}
\end{document}
