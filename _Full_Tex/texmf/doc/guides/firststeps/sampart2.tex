% Sample file: sampart2.tex 
% The sample article for the amsart document class
% with user-defined commands
% Typeset with LaTeX format 

\documentclass{amsart}
\usepackage{amssymb,latexsym}
\usepackage{lattice}

\theoremstyle{plain}
\newtheorem{theorem}{Theorem}
\newtheorem{corollary}{Corollary}
\newtheorem*{main}{Main~Theorem}
\newtheorem{lemma}{Lemma}
\newtheorem{proposition}{Proposition}

\theoremstyle{definition}
\newtheorem{definition}{Definition}

\theoremstyle{remark}
\newtheorem*{notation}{Notation}

\numberwithin{equation}{section}

\newcommand{\Prodm}[2]{\gP(\,#1\mid#2\,)}
   % product with a middle
\newcommand{\Prodsm}[2]{\gP^{*}(\,#1\mid#2\,)}
   % product * with a middle
\newcommand{\vct}[2]{\vv<\dots,0,\dots,\overset{#1}{#2},% 
\dots,0,\dots>}% special vector
\newcommand{\fp}{\F{p}}% Fraktur p
\newcommand{\Ds}{D^{\langle2\rangle}}

\begin{document}
\title[Complete-simple distributive lattices]
      {A construction of complete-simple\\ 
       distributive lattices}
\author{George~A. Menuhin}
\address{Computer Science Department\\
         University of Winnebago\\
         Winnebago, Minnesota 23714} 
\email{menuhin@ccw.uwinnebago.edu}
\urladdr{http://math.uwinnebago.ca/homepages/menuhin/}
\thanks{Research supported by the NSF under grant number~23466.} 
\keywords{Complete lattice, distributive lattice, complete 
   congruence, congruence lattice} 
\subjclass{Primary: 06B10; Secondary: 06D05}
\date{March 15, 1995}

\begin{abstract}
   In this note we prove that there exist \emph{complete-simple 
   distributive lattices,} that is, complete distributive 
   lattices in which there are only two complete congruences. 
\end{abstract}
\maketitle

\section{Introduction}\label{S:intro} 
In this note we prove the following result:

\begin{main}
   There exists an infinite complete distributive lattice 
   $K$ with only the two trivial complete congruence relations. 
\end{main}

\section{The $\Ds$ construction}\label{S:Ds}  
For the basic notation in lattice theory and universal algebra, 
see Ferenc~R. Richardson~\cite{fR82} and George~A. Menuhin~\cite{gM68}.  
We start with some definitions:

\begin{definition}\label{D:prime}
   Let $V$ be a complete lattice, and let $\fp = [u, v]$ be
   an interval of $V$.  Then $\fp$ is called 
   \emph{complete-prime} if the following three conditions are satisfied:
   \begin{enumerate}
      \item[(1)] $u$ is meet-irreducible but $u$ is \emph{not}
         completely meet-irreducible;
      \item[(2)] $v$ is join-irreducible but $v$ is \emph{not} 
         completely join-irreducible;
      \item[(3)] $[u, v]$ is a complete-simple lattice.
   \end{enumerate}
\end{definition}

Now we prove the following result:

\begin{lemma}\label{L:ds} 
   Let $D$ be a complete distributive lattice satisfying 
   conditions~\textup{(1)} and~\textup{(2)}.  
   Then $\Ds$ is a sublattice of $D^{2}$; hence $\Ds$ is
   a lattice, and $\Ds$ is a complete distributive lattice 
   satisfying conditions~~\textup{(1)} and~~\textup{(2)}.  
\end{lemma}

\begin{proof}
   By conditions~(1) and (2), $\Ds$ is a sublattice of 
   $D^{2}$.  Hence, $\Ds$ is a lattice.

   Since $\Ds$ is a sublattice of a distributive lattice, $\Ds$ is 
   a distributive lattice.  Using the characterization of
   standard ideals in Ernest~T. Moynahan~\cite{eM57}, 
   $\Ds$ has a zero and a unit element, namely, 
   $\vv<0, 0>$ and $\vv<1, 1>$.  To show that $\Ds$ is
   complete, let $\es \ne A \ci \Ds$, and let $a = \JJ A$
   in $D^{2}$.  If $a \in \Ds$, then 
   $a = \JJ A$ in $\Ds$; otherwise, $a$ is of the form 
   $\vv<b, 1>$ for some $b \in D$ with $b < 1$.  Now 
   $\JJ A = \vv<1, 1>$ in $D^{2}$, and   
   the dual argument shows that $\MM A$ also exists in 
   $D^{2}$.  Hence $D$ is complete. Conditions~(1) and (2)
   are obvious for $\Ds$.   
\end{proof}

\begin{corollary}\label{C:prime}
   If $D$ is complete-prime, then so is $\Ds$.
\end{corollary}

The motivation for the following result comes from Soo-Key 
Foo~\cite{sF90}.

\begin{lemma}\label{L:ccr} 
   Let $\gQ$ be a complete congruence relation of $\Ds$ such 
   that
   \begin{equation}\label{E:rigid}
      \vv<1, d> \equiv \vv<1, 1> \pod{\gQ},
   \end{equation}
   for some $d \in D$ with $d < 1$. Then $\gQ = \gi$.
\end{lemma}

\begin{proof}
   Let $\gQ$ be a complete congruence relation of $\Ds$ 
   satisfying \eqref{E:rigid}. Then $\gQ = \gi$.
\end{proof}

\section{The $\gP^{*}$ construction}\label{S:P*} 
The following construction is crucial to our proof of the 
Main~Theorem:

\begin{definition}\label{D:P*} 
   Let $D_{i}$, for $i \in I$, be complete distributive 
   lattices satisfying condition~\tup{(2)}.  Their $\gP^{*}$
   product is defined as follows: 
   \[
      \Prodsm{ D_{i} }{i \in I} = \Prodm{ D_{i}^{-} }{i \in I} +1;
   \]
   that is, $\Prodsm{ D_{i} }{i \in I}$ is 
   $\Prodm{ D_{i}^{-} }{i \in I}$ with a new unit element.  
\end{definition}

\begin{notation}
   If $i \in I$ and $d \in D_{i}^{-}$, then
   \[
      \vct{i}{d}
   \]
   is the element of $\Prodsm{ D_{i} }{i \in I}$ whose 
   $i$-th component is $d$ and all the other
   components are $0$. 
\end{notation}

See also Ernest~T. Moynahan~\cite{eM57a}.  Next we verify:

\begin{theorem}\label{T:P*}  
   Let $D_{i}$, for $i \in I$, be complete distributive 
   lattices satisfying condition~\tup{(2)}.  Let $\gQ$ be a
   complete congruence relation on  
   $\Prodsm{ D_{i} }{i \in I}$.  If there exist  
   $i \in I$ and $d \in D_{i}$ with $d < 1_{i}$ such 
   that for all $d \leq c < 1_{i}$,
   \begin{equation}\label{E:cong1}
      \vct{i}{d} \equiv \vct{i}{c} \pod{\gQ}, 
   \end{equation}
   then $\gQ = \gi$.
\end{theorem}

\begin{proof}  
   Since 
   \begin{equation}\label{E:cong2}
      \vct{i}{d} \equiv \vct{i}{c} \pod{\gQ}, 
   \end{equation}
   and $\gQ$ is a complete congruence relation, it follows 
   from condition~(3) that
   \begin{align}\label{E:cong}
      &\vct{i}{d} \equiv \notag\\ 
      &\qq\q{\JJm{\vct{i}{c}}{d \leq c < 1}=1} \pod{\gQ}. 
   \end{align}
   Let $j \in I$ for $j \neq i$, and let 
   $a \in D_{j}^{-}\).  Meeting both sides of the congruence
   \eqref{E:cong} with $\vct{j}{a}$, we obtain
   \begin{align}\label{E:comp}
       0 &= \vct{i}{d} \mm \vct{j}{a}\\
           &\equiv \vct{j}{a}\pod{\gQ}. \notag
   \end{align}
   Using the completeness of $\gQ$ and \eqref{E:comp}, we get:
   \begin{equation}\label{E:cong3}
       0=\JJm{ \vct{j}{a} }{ a \in D_{j}^{-} } \equiv 1 \pod{\gQ}, 
   \end{equation}
   hence $\gQ = \gi$.
\end{proof}

\begin{theorem}\label{T:P*a}  
   Let $D_{i}$ for $i \in I$ be complete distributive 
   lattices satisfying
   conditions~\tup{(2)} and \tup{(3)}.  Then 
   $\Prodsm{ D_{i} }{i \in I}$ also satisfies 
   conditions~\tup{(2)} and \tup{(3)}.  
\end{theorem}

\begin{proof}
   Let $\gQ$ be a complete congruence on 
   $\Prodsm{ D_{i} }{i \in I}$. Let $i \in I$.  Define 
   \begin{equation}\label{E:dihat}
      \widehat{D}_{i} = \setm{ \vct{i}{d} }{ d \in D_{i}^{-} } 
       \uu \set{1}.
   \end{equation}
   Then $\widehat{D}_{i}$ is a complete sublattice of 
   $\Prodsm{ D_{i} }{i \in I}$, and $\widehat{D}_{i}$  
   is isomorphic to $D_{i}$.  Let $\gQ_{i}$ be the 
   restriction of $\gQ$ to $\widehat{D}_{i}$.  Since
   $D_{i}$ is complete-simple, so is $\widehat{D}_{i}$,
   hence $\gQ_{i}$ is $\go$ or $\gi$.  If $\gQ_{i} = \go$ 
   for all $i \in I$, then $\gQ = \go$.  
   If there is an $i \in I$, such that $\gQ_{i} = \gi$, 
   then $0 \equiv 1 \pod{\gQ}$, and hence $\gQ = \gi$.  
\end{proof}

The Main Theorem follows easily from Theorems~\ref{T:P*} and 
\ref{T:P*a}.

\begin{thebibliography}{9}

   \bibitem{sF90}
      Soo-Key Foo, \emph{Lattice Constructions,} Ph.D. thesis, University 
      of Winnebago, Winnebago, MN, December, 1990.

   \bibitem{gM68}
      George~A. Menuhin, \emph{Universal Algebra,} D.~van Nostrand,
      Princeton-Toronto-London-Mel\-bourne, 1968.

   \bibitem{eM57}
      Ernest~T. Moynahan, \emph{On a problem of M.H. Stone,} Acta Math.
       Acad.Sci. Hungar. \textbf{8} (1957), 455--460.

   \bibitem{eM57a}
      \bysame, \emph{Ideals and congruence relations in lattices.~II,}
     Magyar Tud. Akad. Mat. Fiz. Oszt. K\"{o}zl. \textbf{9} (1957), 
     417--434  (Hungarian).

   \bibitem{fR82}
      Ferenc~R. Richardson, \emph{General Lattice Theory,} Mir, Moscow, 
      expanded and revised ed., 1982 (Russian).

\end{thebibliography}

\end{document}