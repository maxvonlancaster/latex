%% File: gentl-gr.tex
%%
%% M. Doob's "Gentle Introduction to TeX" (Updated 01/04/90)
%% translated into Modern Greek by Dimitrios Filippou (including
%% an extra chapter on typesetting Greek texts by TeX).
%%
%% Version: 1.000 (11 jan 2001)
%%
%% A NOTE TO THE CODE READER
%%
%% Thanks for your interest in the Modern Greek translation of Michael
%% Doob's "A Gentle Introduction to TeX".  Any comments on this manual
%% would be appreciated. These may be typesetting, Greek, English or TeX
%% criticisms.
%% 
%% This file is a complete TeX input file, but, in order to run it
%% smoothly, you will need to have the KD* Greek fonts (that is
%% Dryllerakis' GreeKTeX fonts version 3.1, or, better, version 4.0a)
%% installed on your system.  If you don't have these fonts, you may get
%% them by ftp from any CTAN site or from "laotzu.doc.ic.ac.uk".  In
%% case you have KD* fonts on your system, just run this file through TeX
%% and print out the resulting "DVI" file.  Note that you don't need
%% to load any special Greek TeX format (*.fmt) file.
%%  
%% If you are familiar with TeX, the macros at the top of the file have
%% a few switches which you may want to set.  If you have problems or 
%% can't run TeX at all, you may try to print the PostScript(TM) file that
%% comes with it.  In the worst case, you may write to Michael Doob and
%% he'll send you a hard copy of his original English text (sorry, the
%% translator cannot offer such a service!).  You can also get a bound copy
%% of the English original from the TeX Users Group (e-mail: tug@tug.org).
%% 
%% Please, note that this translation is copyrighted by the translator. The
%% complete copyright statement says:  
%% 
%%    "The present document is distributed by the author and the translator
%%    in the hope that it will be useful to the reader.  However, the author
%%    and the translator are not prepared to provide further information at
%%    this time, and do not accept any liability for the use of this 
%%    document for any purpose.  No part of the present document may be 
%%    reproduced in any form for commercial or any other profit-driven
%%    use.  Limited reproduction and distribution is permitted only for
%%    purely educational purposes.  Also, the TeX Users Group (TUG) or any 
%%    TeX Local User Group (LUG) can distribute these files in a commercial 
%%    fashion, provided that profits go to TUG or to the respective LUGs."
%% 
%% The above statement should not be violated!
%% 
%% Author:         Michael Doob 
%%                 Department of Mathematics 
%%                 The University of Manitoba 
%%                 Winnipeg, Manitoba  R3T 2N2 
%%                 Canada 
%%                 mdoob@uofmcc                    (bitnet) 
%%                 mdoob@ccu.umanitoba.ca          (internet) 
%% 
%% Translator:     Dimitrios Filippou
%%                 Kato Gatzea
%%                 GR-385 00 Volos
%%                 Greece
%%                 dfilipp@danaos.ntua.gr          (internet)
%% 
%%
%% CHARACTER LISTING
%% 
%% Here is a character listing to check to be sure that no 
%% unwanted translations took place within the bowels of the net.
%%
%% Upper case letters: ABCDEFGHIJKLMNOPQRSTUVWXYZ 
%% Lower case letters: abcdefghijklmnopqrstuvwxyz 
%% round parentheses, square brackets, curly braces: ()  []  {} 
%% Exclaim, at, sharp, dollar, percent: ! @ # $ % 
%% Caret, ampersand, star, underscore, hyphen: ^ & * _ - 
%% vertical bar, backslash, tilde, backprime, plus: | \ ~ ` + 
%% plus, equal, prime, quote, colon: + = ' " :
%% less than, greater than, slash, question, comma: < > / ? , 
%% period, semicolon: . ; 
%%
%% HISTORY
%%
%% Translation started: Jan. 1994
%% First draft completed:  Mar. 1997 (ver. 0.9)
%% First public release (Internet): May 8, 1997 (ver. 0.992)
%%
%% Revisions:
%%
%% May 8, 1997 (ver. 0.993, not released): A few typos only.
%% June 1, 1997 (ver. 0.994, released): Lots of typos corrected.
%%                           Corrected the doubleside output macros.
%% July 29, 1997 (ver. 0.995, not released): A few more typos.
%% Feb 04, 1998 (ver. 0.996, released): Lots of typos corrected.
%%                           Added a note on Greek stigma numeral.
%% Feb. 05, 1998 (ver. 0.997, released): A few typos.
%% July 27, 1998 (ver. 0.998, released): A few typos.
%% Jan. 27, 1999 (ver. 0.999, released): Added info. about
%%                            Syropoulos' Greek book on LaTeX.
%% Mar. 20, 2001 (ver. 1.000, released): Final version. The next
%%                            version is a printed book!

%% Now here comes the code with the macros used in the manual.  If you are 
%% already familiar with TeX, you may want to fiddle with them.  In
%% particular, the hooks are left to generate a new control word index and
%% table of contents if you change section 1.2, or section 10.



%%%%%%%%%%%%%%%%%%%%%%% Version/Edition no. %%%%%%%%%%%%%%%%%%%%%%%%%

\def\version{1,000}
\def\grversiondate{20`h Mart'iou 2001}

%%%%%%%%%%%%%%%%%%%%%%%%%%%%%%%%%%%%%%%%%%%%%%%%%%%%%%%%%%%%%%%%%%%%%


%%%%%%%% Here are the fonts other than the sixteen defined in %%%%%%%%
%%%%%%%% plain.tex that are used.  Note that only "cmr" fonts %%%%%%%%
%%%%%%%% are used for latin-alphabet texts (DF)               %%%%%%%%

\font\brm=cmr10 scaled \magstep1
\font\halfrm=cmr10 scaled \magstephalf
\font\eightrm=cmr8
\font\eightit=cmti8
\font\bbrm=cmr10 scaled \magstep2
\font\bbbrm=cmr10 scaled \magstep3
\font\bbbbrm=cmr10 scaled \magstep4
\font\bbbbbrm=cmr10 scaled \magstep5
\font\sf=cmss10
\font\lchapfont=cmbx10 scaled \magstep2      % lchapfont = latin "chapter" font
\font\lsecfont=cmbx10 scaled \magstep1       % lsecfont = latin "section" font
\font\sc=cmcsc10

%%%%%%%%%%%%%%%%%%%%%%%%%%%%%%%%%%%%%%%%%%%%%%%%%%%%%%%%%%%%%%%%%%%%%


%%%%%%%% Here are the Greek fonts added by DF %%%%%%%%%%%%%%%%%%%%%%%

\font\tengr=kdgr10 scaled \magstep0          % 10 pt regular greek
\font\ninegr=kdgr9 scaled \magstep0          % 9 pt ....
\font\eightgr=kdgr8 scaled \magstep0         % 8 pt ....
\font\tengb=kdbf10 scaled \magstep0          % 10 pt reg. greek bold
\font\ninegb=kdbf9 scaled \magstep0          % 9 pt ....
\font\eightgb=kdbf8 scaled \magstep0         % 8 pt ....
\font\chapfont=kdbf10 scaled \magstep2       % 14.4 pt .... ("chapfont")
\font\secfont=kdbf10 scaled \magstep1        % 12 pt .... ("secfont")
\font\tengs=kdsl10 scaled \magstep0          % 10 pt reg. greek slanted
\font\tengi=kdti10 scaled \magstep0          % 10 pt greek "pseudo-italics"
\font\tengt=kdtt10 scaled \magstep0          % 10 pt greek typewritter

%%% and here are some oblivious characters

\def\digamma{\char'020}    %%%%% digamma (or former 6) %%%%
\def\stigma{\char'143}     %%%%% stigma (or latter 6) %%%%%
\def\Koppa{\char'022}      %%%%% capital Qoppa (or 90) %%%%
\def\koppa{\char'023}      %%%%% small qoppa (or 90) %%%%%%
\def\varkoppa{\char'021}   %%%%% a small qoppa variant %%%%
%%%                              which has been mistakenly
%%%                              identified by KD as \stigma
\def\sampi{\char'024}      %%%%% sampi (or 900) %%%%%%%%%%%
\def\numbertick{\char'003} %%%%% upper tick for ordinal Greek numbers
\def\pretick{\char'004}    %%%%% lower tick for Greek thousands

%%%%%%%%%%%%%%%%%%%%%%%%%%%%%%%%%%%%%%%%%%%%%%%%%%%%%%%%%%%%%%%%%%%%%%%%


%%%%%%%%%%%%%%%%%%%%  document dimensions, etc. %%%%%%%%%%%%%%%%%%%%%%%%

%%%% \vbadness=10000 %%%%%% I don't want to hear about underfull vboxes
\raggedbottom        %%%%%% delete this line for aligned page bottoms
    \hsize=5.5 in
    \vsize=7.0 in
    \voffset=.75 in
    \hoffset=.5 in
\parskip=\baselineskip
\widowpenalty=1000 \clubpenalty=1000  %%% I hate widows and orphans! %%%


%%%%%%%%%%%%%%% Shall we print double-side or single-side? %%%%%%%%%%

\newif \ifdoubleside
\doublesidefalse  % use this line if you want to print single-side
%\doublesidetrue  % use this line if you want to print double-side
                  % (you may have to do some fiddling with your printer
                  % driver so as to align both sides).

\newdimen\paperwidth      % true paperwidth (set below)
\newdimen\oddleftmargin   % true left margin of an odd page (set below)
\newdimen\oddrightmargin  % true right margin of an odd page (set below)
\newdimen\evensideshift   % even-numbered pages shift to the left (set below)

\newif\iflettersize
\lettersizetrue        %%% to print on A4 paper, comment out this line 
%%% \lettersizefalse   %%% and uncomment this.
\iflettersize\paperwidth=8,5 true in  % set paperwidth
\else\paperwidth=210 true mm\fi
\oddleftmargin=1 true in              % set oddleftmargin
\advance\oddleftmargin by \hoffset
\oddrightmargin=\paperwidth           % set oddrightmargin
\advance\oddrightmargin by -\hsize
\advance\oddrightmargin by -\oddleftmargin
\evensideshift=\oddrightmargin        % set evensideshift
\advance\evensideshift by -\oddleftmargin

\def\newplainoutput{%
  \ifodd\pageno%
    \shipout\vbox{%
    \makeheadline\pagebody\makefootline}%
    \advancepageno
    \ifnum\outputpenalty>-2000 
    \else\dosupereject
    \fi%
  \else
    \shipout\hbox{\hglue\evensideshift\vbox{%
    \makeheadline\pagebody\makefootline}}%
    \advancepageno
    \ifnum\outputpenalty>-2000 
    \else\dosupereject
    \fi%
  \fi
}

\ifdoubleside\output{\newplainoutput}
\else\output{\plainoutput}
\fi

%%%%%%%%%%%%%%%%%%%%%%%%%%%%%%%%%%%%%%%%%%%%%%%%%%%%%%%%%%%%%%%%%%%%%%%%


%%%%%%% choosing between Canadian (British) and American spellings %%%%%

\newif \ifcanspell
\canspelltrue      %%%% Canadian spelling
% \canspellfalse   %%%% use this line for American spelling
\def\centimetre{\ifcanspell centimetre\else centimeter\fi}
    \let\centimeter=\centimetre
\def\centre{\ifcanspell centre\else center\fi}
    \let\center=\centre
\def\centred{\ifcanspell centred\else centered\fi}
    \let\centered=\centred

%%%%%%%%%%%%%%%%%%%%%%%%%%%%%%%%%%%%%%%%%%%%%%%%%%%%%%%%%%%%%%%%%%%%%%%


%%%%%%%%%%%%%%%%%%%%%%% headline routines  %%%%%%%%%%%%%%%%%%%%%%%%%%%%
%%%%%%%%%%%%%%%%%% changed by DF to Greek ones %%%%%%%%%%%%%%%%%%%%%%%%

\def\gentleheadline{%
    \vbox {\hrule%
        \line {\strut \vrule \quad \tengr E>isagwg`h st`o {\tenrm \TeX }
              (<ellhnik`h mtf.)\null
              \hfil
              \ifnum \secnum > 0 Kef.\ \the\secnum: \fi \sectiontitle \quad
              \vrule}%
        \hrule}%
}

%%%%%%%%%%%% we don't want any headlines on title pages %%%%%%%%%%%%%%%%%%
\newif \iftitlepage \titlepagetrue
\headline=
  {\iftitlepage \hfil \global\titlepagefalse \else \gentleheadline \fi}
%%%%%%%%%%%%%%%%%%%%%%%%%%%%%%%%%%%%%%%%%%%%%%%%%%%%%%%%%%%%%%%%%%%%%%%%%%


%%%%%%%%%%%%%%%%%%%%% define contents and index files %%%%%%%%%%%%%%%%%%%%%

%%% Normally the contents and index are hard coded into the input file. %%%
%% To generate new ones, use \writingcontentstrue and \writingindextrue. %%

\newif \ifwritingcontents
\newif \ifwritingindex
\newwrite\contents \newwrite\index

\writingcontentsfalse 
\writingindexfalse
%%%\writingcontentstrue  %%% use this line to make the contents
%%%\writingindextrue     %%% use this line to make the index

\ifwritingcontents \openout\contents=contents.tex  \fi
\ifwritingindex \openout\index=index.tex
          \def\toindex#1{\immediate\write\index{#1 \the\pageno}}
          \else \def\toindex#1{} \fi
%%%%%%%%%%%%%%%%%%%%%%%%%%%%%%%%%%%%%%%%%%%%%%%%%%%%%%%%%%%%%%%%%%%%%%%%%%

%%%%%%%%%%%%%%%%%%%% determine whether answers are printed %%%%%%%%%%%%%%%
\newif \ifwritinganswers
%%%\writinganswersfalse     %%% use this line to suppress answer section
\writinganswerstrue         %%% use this line to include answer section
%%%%%%%%%%%%%%%%%%%%%%%%%%%%%%%%%%%%%%%%%%%%%%%%%%%%%%%%%%%%%%%%%%%%%%%%%%


%%%%%%%%%%%%%%%%      footnote macro with counter      %%%%%%%%%%%%%%%%%%%

\newcount\footnotenum \footnotenum=0
\def\fnote#1{\advance \footnotenum by 1%
\footnote{$^{\the\footnotenum}$}{#1}}

%%%%%%%%%%%%%%%%%%%%%%%%%%%%%%%%%%%%%%%%%%%%%%%%%%%%%%%%%%%%%%%%%%%%%%%%%%


%%%%%%%%%%%%%%%% exercise, section, and subsection macros %%%%%%%%%%%%%%%%

\newcount\exno                   %%%%%% counter for exercises %%%%%%%%
\newcount\secnum \secnum=0       %%%% counter for section numbers %%%%
\newcount\subsecnum

\def\section#1{
            \vfill\eject %%%%% new section starts on a new page
            %%%\ifodd\pageno \else\ \vfill\eject \fi %start on an odd page
            \advance\secnum by 1 \subsecnum=0 \exno=0
            \ifnum \secnum = 1 \pageno=1 \fi
            \ifnum \secnum > 0
                 \leftline{\chapfont Kef'alaio \the\secnum}
                 \vskip 3pt \fi
            \leftline{\chapfont #1}
            \def\sectiontitle{#1}
            \vskip\baselineskip
            \hrule
            \vskip 1cm
            \ifwritingcontents \write\contents{\string\line\string{#1
                   \string\dotfill{}
                   \ifnum \pageno < 0 \romannumeral-\pageno
                   \else \the\pageno \fi
                   \string}}\fi
            \titlepagetrue}

\def\subsection#1{\advance\subsecnum by 1
               \vskip 30pt
               \leftline{\secfont \the\secnum .\the\subsecnum\ #1}
               \nobreak
               \ifwritingcontents
                     \write\contents{\string\line\string{\string\qquad{}#1
                     \string\dotfill{} \the\pageno\string}}\fi
                 }

\def\exercise{\global\advance \exno by 1
         \vskip\baselineskip
         \noindent $\triangleright$ {\tengb 
                     >'Askhsh \the\secnum.\the\exno}\quad
         }
%%%%%%%%%%%%%%%%%%%%%%%%%%%%%%%%%%%%%%%%%%%%%%%%%%%%%%%%%%%%%%%%


%% Definitions of control sequences for characters in the typewriter font %%
%%%% for short phrases this is easier than using a literal construction. %%%

\def\\{\char92{}}          %%%%% backslash %%%%%
\def\_{\char'137{}}        %%%%% underscore %%%%%
\def\lb{\char'173{}}       %%%%% left brace %%%%%
\def\rb{\char'175{}}       %%%%% right brace %%%%%
\def\sp{\char32{}}         %%%%% special space symbol %%%%%
\def\underscore{\leavevmode \kern .06em \vbox {\hrule width.3em}}

\def\beginliteral{
\vskip\baselineskip
\begingroup
\tt
\obeylines
%{\obeyspaces\global\let =\ }
\catcode`\@=0
\parskip=0pt\parindent=0pt
\catcode`\$=12\catcode`\&=12\catcode`\^=12\catcode`\#=12
\catcode`\_=12\catcode`\~=12
\def\par{\leavevmode\endgraf}
\catcode`\{=12\catcode`\}=12\catcode`\%=12\catcode`\\=12
}

\def\endliteral{\endgroup}
%%%%%%%%%%%%%%%%%%%%%%%%%%%%%%%%%%%%%%%%%%%%%%%%%%%%%%%%%%%%%%%%%%%%%%%%%%%

%%%%%%%%%%%%%% inhibit hyphenation of typewriter text %%%%%%%%%%%%%%%%%%%%%

\hyphenchar\tentt=-1

%%%%%%%%%%%%%%%%%%%%%%%%%%%%%%%%%%%%%%%%%%%%%%%%%%%%%%%%%%%%%%%%%%%%%%%%%%


%%%%%%%%%%%% grouping to make input listing in typewriter type %%%%%%%%%%%

\def\beginuser{\vskip\parskip
               \everypar={\nobreak}
               \begingroup
               \tt \obeylines \parskip=0pt \parindent=0pt}

\def\enduser{\endgroup}


%%%%%%%%%%% macro to construct tables (easily) %%%%%%%%%%%%%%%%%%%%%%%%%%
%%% parameters: title goes between brackets, rest of the %%%%%%%%%%%%%%%%
%%%             paragraph is the table                   %%%%%%%%%%%%%%%%

\def\maketable[#1]#2\par{
\setbox1=\vbox{#2}
\vskip\baselineskip
\centerline{
     \vbox{
        \hbox to \wd1{\secfont \hss #1 \hss}
        \vskip 12pt
        \box1
      }
}\par
}

%%%%%%%%%%%  end of macro to construct tables  %%%%%%%%%%%%%%%%%%%%%%%%%%


%%%%%%%%%%% macro to put a box around the text %%%%%%%%%%%%%%%%%%%%%%%%%%

\def\makebox#1#2#3% vsize, hsize, inserted text
{\hbox{\vrule
       \vbox to  #1{\hrule \vss
                   \hbox to #2{\hss#3\hss}\vss
                   \hrule}\vrule}}

\def\displaytext#1{$$\hbox{#1}$$}

%%%%%%%%%%%%%%%%%%%%%%%%%%%%%%%%%%%%%%%%%%%%%%%%%%%%%%%%%%%%%%%%%%%%%%%%%


%%%%%%%%%%%%%%%% LaTeX etc. logos %%%%%%%%%%%%%%%%%%%%%%%%%%%%%%%%%%%%%%%

\def\LaTeX{{\rm L\kern-.36em\raise.3ex\hbox{\sc a}\kern-.15em
   T\kern-.1667em\lower.7ex\hbox{E}\kern-.125emX}}

\def\AMSTeX{{$\cal A$}\kern-.1667em\lower.5ex\hbox
 {$\cal M$}\kern-.125em{$\cal S$}-\TeX}

\def\greektex{{\sc GreeK}{\rm \TeX}}

%%%%%%%%%%%%%%%%%%%%%%%%%%%%%%%%%%%%%%%%%%%%%%%%%%%%%%%%%%%%%%%%%%%%%%


%%%%%%%%%%%% macro to put TeX references in right margin %%%%%%%%%%%%%
%%%%%%%%% (doesn't work very well with double-side printing) %%%%%%%%%

\newdimen\theight
\def\TeXref#1{%
 \vadjust{%
    \setbox0=\vbox{\setbox1=\hbox{\sevenrm\qquad\TeX book:\qquad}%
                   \setbox2=\hbox to \wd1{%
                            \sevenrm\null\hfil #1\hfil\null}%
                   \box1\vskip -.5ex\box2}%
    \theight=\ht0
    \advance\theight by \dp0    \advance\theight by \lineskip
    \kern -\theight \vbox to \theight{\rightline{\rlap{\box0}}%
                         \vss}}%
}%

%%%%%%%%%%%%%%%%%%%%%%%%%%%%%%%%%%%%%%%%%%%%%%%%%%%%%%%%%%%%%%%%%%%%%%


%%%%%%%%%%%% macro to write the date out the date and time %%%%%%%%%%%
%%%%%%%%%%%%          TeXbook   p. 406 for date            %%%%%%%%%%%
%%%%%%%%%%%%          Changed to Greek date (DF)          %%%%%%%%%%%%

\def\grtoday{\number\day%
\ifnum\day=20 `h\else\ifnum\day=30 `h\else h\fi\fi %%% 20th, 30th take
                                                   %%% an accent
\space \ifcase\month\or >Ianouar'iou\or Febrouar'iou\or Mart'iou\or
>April'iou\or Ma"'iou\or >Ioun'iou\or >Ioul'iou\or A>ugo'ustou\or
Septembr'iou\or >Oktwbr'iou\or Noembr'iou\or Dekembr'iou\fi
\space \number\year}

\newcount\hour \newcount\minute
\hour=\time  \divide \hour by 60
\minute=\time
\loop  \ifnum \minute > 59 \advance \minute by -60 \repeat
\def\writetime{\ifnum \hour<13 \number\hour:%  % supresses leading 0's
                      \ifnum \minute<10 0\fi%  % so add it it
                      \number\minute
                      \ifnum \hour < 12 \ p.m. \else \ m.m. \fi  % Greek time (DF)
      \else \advance \hour by -12 \number\hour:%  % supresses leading 0's
                      \ifnum \minute<10 0\fi%     % add it in
                      \number\minute \ m.m. \fi} % Greek time (DF)

%%%%%%%%%%%%%%%%%%%%%%%%%%%%%%%%%%%%%%%%%%%%%%%%%%%%%%%%%%%%%%%%%%%%%%


%%%%%      Macro which puts a date stamp on the last page    %%%%%%%%%
%%%%%      according to the Greek typography traditions (DF) %%%%%%%%%

\def\datestamp{%
   \vfill\eject                  % Put datestamp on a new page.
   \null\titlepagetrue\nopagenumbers
        \ifodd\pageno            % If page is odd leave it blank.
        \else 
            \vfill\eject
        \fi
   \titlepagetrue\nopagenumbers  % No page numbers and no headline.
   \hbadness = 10000             % Don't complain for horizontal underfull.
   \null\vfill                   % Put datestamp in the page centre.
   \centerline{\vbox{%           % Create a square vbox of 2.0 in width.
      \noindent\parfillskip 0pt \hsize 2.0 in 
         {\eightgr T O B I B L I O
            ((M I A E U K O L H E I S A G W G H S T O {\eightrm \TeX})) 
            (E L L H N I K H M E T A F R A S H, E K D O S H \version)
             E T O I M A S J H K E A P O T O N M E T A F R A S T H M E E
             P I M E L E I A T O U I D I O U T H N \grversiondate\ K A I
             S T O I Q E I O J E T H J H K E M E T O {\eightrm \TeX} T H
             N \grtoday, <'wra \writetime%
}% end eightgr
}% end vbox
}% end centerline
$$ \clubsuit $$ % we add a bit of decoration
\vfill\null
} % end Greek datestamp

%%%%%%%%%%%%%%%%%%%%%%%%%%%%%%%%%%%%%%%%%%%%%%%%%%%%%%%%%%%%%%%%%%%%%


%%%%%%%%%%%% Here we do some fiddling to get around with the %%%%%%%%
%%%%%%%%%%%% circumflex accent in Greek text! %%%%%%%%%%%%%%%%%%%%%%%

\catcode`\~=12            % we de-activate the ~
\def\NB{\penalty10000\ }  % and we use \NB as an unbreakable tie

%%%%%%%%%%%%%%%%%%%%%%%%%%%%%%%%%%%%%%%%%%%%%%%%%%%%%%%%%%%%%%%%%%%%%


%%%%%%%%%%%%%%%%%%%%%  NOW WE START WITH THE TEXT %%%%%%%%%%%%%%%%%%%

%%% Cover pages %%%%%%%

{\nopagenumbers

%%% The front cover %%%%%%%%%

\def\coverpage{%
\titlepagetrue
\topinsert \vskip 6 cm \endinsert
\centerline{\chapfont M'ia e>'ukolh e>isagwg`h st`o {\lchapfont \TeX}}
\vskip 15 pt
\centerline{\secfont >egqeir'idio a>utodidaskal'iac}
\vskip 2cm
\leftline{\rm Michael Doob}
\leftline{\rm Department of Mathematics}
\leftline{\rm The University of Manitoba}
\leftline{\rm Winnipeg, Manitoba, Canada R3T 2N2}
\vskip\baselineskip
\leftline{{\tt MDOOB@UOFMCC.BITNET}}
\leftline{{\tt mdoob@ccu.umanitoba.ca}}
\vskip 1cm
\leftline{\tengs Met'afrash ka`i prosarmog`h st`hn neoellhnik`h 
gl'wssa:}
\vskip .5\baselineskip
\leftline{\tengr Dhm'htrioc >A. Fil'ippou}
\leftline{\tengr K'atw Gatz'ea}
\leftline{\rm GR-385 00 \tengr B'oloc}
\vfill\eject
} % end definition of coverpage

\coverpage  % Now we put the coverpage

%%% Back front cover (or blank page) %%%%%%%

\titlepagetrue
\null\vfill\eject

%%% Small title page %%%%%%%

\titlepagetrue
\null\vfil
\centerline{\secfont MIA EUKOLH EISAGWGH STO {\lsecfont \TeX}}
\vfil\vfil
\null\eject

%%% Edition and copyright page %%%%%%

\titlepagetrue

\newdimen\squaredimen
\def\square#1{\setbox0=\hbox{\rm #1}%
   \dp0=.4ex \squaredimen=\ht0 
   \ifdim \wd0 > \ht0 \squaredimen=\wd0\fi
   \advance \squaredimen by \dp0
   \ht0=\squaredimen \advance \squaredimen by \dp0
   \leavevmode
   \lower .4ex \vbox{%
      \hsize=\squaredimen 
      \hrule
      \hbox to \squaredimen{\vrule\hfil\box0\hfil\vrule}%
      \hrule height .4pt depth 0pt}}

\noindent
\tengr T`o prwt'otupo >agglik`o >egqeir'idio kuklof'orhse
st`o {\rm Internet} t`o 1990 ka`i >apot'elese  t`hn b'ash to~u
bibl'iou:\ {\rm Michael Doob}, {\sl \TeX: Starting from \square1}
(1993), po`u ku\-klo\-fore~i >ap`o t`on >ekdotik`o o>~iko {\rm
Springer--Verlag (ISBN 3-540-56441-1} >`h 0-387-56441-1). <H <ellhnik`h
met'afrash to~u bi\-bl'iou, m`e t'itlo {\tengs T`o pr~wto b~hma st`o} {\sl
\TeX}, kuklof'orhse t`o 2000 >ap`o t`ic >Ekd'oseic Para\-th\-rh\-t`hc t~hc
Jessalon'ikhc ({\rm ISBN 960-374-081-0}). 

\vfill
\item{{\rm\copyright}} Gi`a t`o prwt'otupo ke'imeno: {\rm
Michael Doob}, 1990.
{\parskip=0pt\item{} Gi`a t`hn <ellhnik`h met'afrash: Dhm'htrioc
>A. Fil'ippou, 1997, 1998, 1999, 2001.
}

{\rm\item{\copyright} For the original text in English:
Michael Doob, 1990.
\parskip=0pt\item{} For the translation in Modern Greek: 
Dimitrios Filippou, 1997, 1998, 1999, 2001.
}

\noindent {\tengs T`o par`on >'entupo diat'ijetai >ap`o t`on suggraf'ea
ka`i t`on metafrast'h tou m`e t`hn >elp'ida <'oti j`a fane~i qr'hsimo
st`on >anagn'wsth.  <Wst'oso, t'oso <o suggraf'eac <'oso ka`i <o
metafrast`hc d`en br'iskontai s`e j'esh n`a prosf'eroun periss'oterec
sqetik`ec plhrofor'iec, o>'ute >apod'eqontai e>uj'unec gi`a <'opoiec
sun'epeiec mpore~i n`a >'eqei <h qr'hsh to~u >ent'upou. <H >anaparagwg`h
to~u pa\-r'ontoc >ent'upou, >`h m'erouc a>uto~u, s`e <opoi\-ad'h\-pote
mor\-f`h gi`a >emporik`h >`h >'allh kerdoskopik`h qr'h\-sh
>apagore'uetai. >Epitr'epetai <h periorism'enh >anaparagwg`h ka`i
dianom'h tou m'onon gi`a kajar`a >ekpaideutiko`uc skopo'uc. >Ep'ishc, <h
<om'ada {\sl\TeX{} Users Group (TUG)} ka`i <opoiad'hpote topik`h m`h
kerdoskopik`h <om'ada {\sl\TeX{} (Local User Group, LUG)} >'eqoun t`o
dika'iwma n`a >anapar'agoun <eto~uto ke'imeno ka`i t`a sqetik`a >arqe~ia
gi`a >emporik`o skop'o, >ef'' <'oson m'eroc to~u k'erdouc diat'ijetai
gi`a to`uc skopo`uc to~u {\sl TUG} >`h t~wn >antisto'iqwn {\sl LUG}.  } 

\noindent {\sl The present document is distributed by the author and the
translator in the hope that it will be useful to the reader. However,
the author and the translator are not prepared to provide further
information at this time, and do not accept any liability for the use of
this document for any purpose. No part of the present document may be
reproduced in any form for commercial or any other profit-driven use. 
Limited reproduction and distribution is permitted for purely
educational purposes. Also, the \TeX{} Users Group (TUG) or any
non-profit \TeX{} Local User Group (LUG) can distribute this document
and the related files in a commercial fashion, provided that profits go
to TUG or to the respective LUGs.} \eject

%%% Titles again %%%%%%%%

\coverpage

%%% Back big titles page (or blank page) %%%%%%%

\titlepagetrue
\null\vfill\eject

%%% End of coverpages %%%%%%%
} % end of nopagenumbers

\secnum=-3 \pageno=-1

\section{E>isagwg`h}

\tengr

>`Ac po~ume pr~wta t`a >'asqhma n'ea: t`o {\rm \TeX} e>~inai <'ena
meg'alo ka`i polus'unjeto pr'ogram\-ma po`u proqwre~i ((pol`u p'eran
to~u kanoniko~u)) st`hn prosp'ajei'a tou n`a par'agei >'omorfa
stoiqeio\-je\-thm'ena >'entupa.  A>ut`h <h po\-lu\-plo\-k'o\-th\-ta to~u
{\rm \TeX} mpore~i kat`a kai\-ro`uc n`a prokal'esei >an'elpista
>apotel'esmata.  Ka`i t'wra t`a kal`a n'ea:  <apl`a ke'imena e>~inai
pol`u e>'ukolo n`a stoi\-qei\-o\-jethjo~un m`e t`o {\rm \TeX}\null.
>'Etsi, mpore~i n`a xekin'hsei kane`ic t`hn qr'hsh to~u {\rm \TeX} st`hn
stoi\-qei\-ojes'ia sqetik`a <apl~wn >ent'upwn ka`i kat'opin, m`e t`hn
>ap'okthsh pe'irac, n`a proqwr'hsei st`hn stoiqeiojes'ia pi`o
polupl'okwn >ent'upwn.

Skop`oc a>uto~u to~u >egqeirid'iou e>~inai n`a e>isag'agei t`on         
>arq'ario >`h ka`i t`on >entel~wc >an'ideo to~u {\rm \TeX} st`hn qr'hsh 
a>uto~u to~u progr'ammatoc pern'wntac >ap`o t`ic pi`o <apl`ec
katast'aseic st`ic pl'eon pol'uplokec.  Proqwr'wntac diadoqik`a >ap`o
t`o <'ena kef'alaio st`o >'allo, >ese~ic, o<i >ana\-gn~wstec to~u
>egqeirid'iou j`a de~ite t`ic <ikan'otht'ec sac st`o {\rm \TeX} suneq~wc
n`a belti'wnontai ka`i j`a mpore~ite n`a <etoim'azete <'olo kai pi`o
poik'ila ka`i pol'uploka >'entupa.

<Or'iste merik`ec >ak'oma sumboul'ec: S`e k'aje kef'alaio <up'arqoun
>ask'hseic;  m'hn >amele~ite n`a t`ic k'anete!  <O m'onoc tr'opoc gi`a 
n`a m'ajete t`o {\rm \TeX} e>~inai n`a t`o qrhsimopoie~ite.  Ka`i >ak'oma
kal'utera, peiramatisje~ite m'onoi sac m`e t`o {\rm \TeX}; prospaj~hste
n`a epil'usete k'apoiec parallag`ec t~wn >ask'hsewn.  M'hn fob'aste,
d`en <up'arqei kam'ia per'iptwsh n`a prokal'esete zhmi`a st`o >'idio
t`o {\rm \TeX} m`e t`a peir'amat'a sac.   Mpore~ite n`a bre~ite t`hn
pl'hrh l'ush t~wn perissot'erwn >ask'hsewn r'iqnontac m'ia              
mati`a st`o >arqe~io {\tt gentle.tex}, dhl.\ st`on k'wdika po`u
sunt'aqjhke gi`a t`hn paragwg`h to~u >aggliko~u prwt'otupou <eto'utou
to~u >egqeirid'iou.  >Ep'ishc, j`a parathr'hsete p`wc st`o dex`i
perij'wrio to~u >egqeirid'iou <up'arqoun >anafor`ec s`e sel'idec t~hc
((B'iblou to~u {\rm \TeX})), {\sl The \TeX book}\fnote{{\rm Addison
Wesley, Reading, Massachusetts, 1984, ISBN 0-201-13448-9.}}. >E`an
qrei'azetai n`a >embaj'unete s`e ak'omh periss'oterec leptom'ereiec, d`en
>'eqete par`a n`a >anatr'exete st`ic sugkekrim'enec a>ut`ec sel'idec
to~u {\sl \TeX book}\null.

Parepipt'ontoc, s`e <eto~uto t`o >egqeir'idio, po~u ka`i po~u           
>eskemm'ena j`a l'eme ka`i >ap`o kan'ena yemat'aki;  t`o k'anoume m`e   
kal`h pr'ojesh ka`i m`e m'ono skop`o n`a >apokr'uyoume pi`o pol'uplokec 
katast'aseic (>`ac po~ume <'oti t`o k'anoume ((poihtik~h|               
>ade'ia|)){}).  <'Oso <h empeir'ia sac st`o {\rm \TeX} j`a megal'wnei,  
t'oso ka`i j`a g'ineste pi`o <ikano`i st`o n`a br'iskete a>ut`a t`a     
>aj~wa yemat'akia mac.

T`o {\rm \TeX} e>~inai pr'ogramma po`u qarakthr'izetai <wc {\rm public  
domain}, dhl.\ diat'ijetai dwre'an. Dhmiourg'oc tou e>~inai <o
kajhght`hc plhroforik~hc st`o Panepist'hmio {\rm Stanford} t~wn H.P.A,  
{\rm Donald Knuth} (Nt'onalnt Kano'uj).  >E`an t`o pr'ogramma a>ut`o
>'ebgaine st`hn >agor`a <wc >emporik`o pro"i'on, s'igoura <h >ax'ia tou
j`a >'eftane s`e qili'adec dol\-l'aria --- gi`a n`a m`hn mil'hsoume ka`i
s`e draqm'ec!   <O m`h kerdo\-skopik`oc >organism`oc {\rm \TeX\ Users
Group (TUG)} e>~inai a>ut`oc po`u dia\-j'etei >ant'igrafa ka`i n'eec
belti'wseic to~u progr'ammatoc {\rm \TeX}\null. >Ep'ishc, <o
>organism`oc  {\rm TUG} >ekd'idei t`a periodik`a {\sl TUGboat\/} ka`i
{\sl \TeX{} and TUG News\/} <'opou dhmosie'uontai plhrofor'iec
sqetik`ec m`e n'eec >exel'ixeic t'oso s`e j'emata logismiko~u
(progr'ammata, k.lp.)\ <'oso ka`i s`e j'emata <uliko~u (<upologist'ec,
k.lp.)\ po`u >'eqoun >'amesh sq'esh m`e t`o {\rm \TeX}\null.  T`o n`a
g'inei k'apoioc m'eloc to~u {\rm TUG} d`en kost'izei par`a >el'aqista;
>e`an >endiaf'ereste, >arke~i n`a gr'ayete <'ena gr'amma st`hn
die'ujunsh:
{\rm
\vskip\baselineskip
\centerline{\TeX{} Users Group}
\centerline{P.O. Box 869}
\centerline{Santa Barbara, CA 93102}
\centerline{U.S.A.}
\centerline{e-mail: {\tt tug@tug.org}}
\centerline{http://{\tt www.tug.org}}
} % end roman

A>ut`o t`o >egqeir'idio d`en j`a mporo~use n`a de~i t`o f~wc t~hc
<hm'erac qwr`ic t`hn bo'hjeia k'apoiwn >'allwn >at'omwn.  >Exairetik`a
meg'alhc >ax'iac <up~hrxe <h prosektik`h >an'a\-gnwsh ka`i o<i
<upode'ixeic t~wn parak'atw >at'omwn:
 \begingroup \catcode`\~=13 \rm
 \frenchspacing
Waleed A.~Al-Salam (University of Alberta),
Debbie L.~Alspaugh (University of California),
Nelson H.~F.~Beebe (University of Utah),
Barbara Beeton (American Mathematical Society),
Bart Childs (Texas A.~\&~M.~University),
Mary Coventry (University of Washington),
Dimitrios Diamantaras (Temple University),
Roberto Dominimanni (Naval Underwater Systems Center),
Victor Eijkhout (University of Nijmegen),
Moshe Feder (St.~Lawrence University),
Josep~M.~Font (Uviversidad Barcelona),
Jonas de Miranda Gomes (Instituto de Matematica Pura e Aplicada, Brazil),
Rob Gross (Boston College),
Klaus Hahn (University of Marburg),
Anita Hoover (University of Delaware),
J\"urgen Koslowski (Macalester College),
Kees van der Laan (Rijksuniversiteit Groningen),
John Lee (Northrop Corporation),
Silvio Levy (Princeton University),
Robert Messer (Albion College),
Emily H.~Moore (Grinnell College),
Young Park (University of Maryland),
Craig Platt (University of Manitoba),
David Roberts (Colorado),
Kauko Saarinen (University of Jyv\"askyl\"a),
Jim Wright (Iowa State University)
{\tengr ka`i}
Dominik Wujastyk (Wellcome Institute for the History of Medicine).
 \endgroup

>Epipl'eon, poll`a >'alla >'atoma mo~u >'esteilan merik`a >`h ka`i
pl'hrh dik'a touc ((topik`a >egqeir'i\-dia)) to~u {\rm \TeX}\null.        
Pi`o sugkekrim'ena, o<i parak'atw >'eqoun gr'ayei shmei'wseic >ep'anw   
st`o {\rm \TeX} o<i <opo~iec <up~hrxan shmantik`h bo'hjeia gi`a t`hn      
proetoimas'ia <eto'utou to~u >egqeirid'iou:
 \begingroup \catcode`\~=13 \rm
 \frenchspacing
Elizabeth Barnhart (TV Guide),
Stephan v.~Bechtolsheim (Purdue University),
Nelson H.~F.~Beebe (University of Utah)
     {\tengr ka`i} Leslie Lamport (Digital Equipment Corporation),
Marie McPart\-land-Conn {\tengr ka`i} Laurie Mann (Stratus Computer),
Robert Messer (Albion College),
Noel Peterson (Library of Congress),
Craig Platt (University of Manitoba),
Alan Spragens (Stanford Linear Accelerator Center, 
     {\tengr t'wra m`e t`hn} Apple Computers),
Christina Thiele (Carleton University)
{\tengr ka`i} Daniel M.~Zirin (California Institute of Technology).
 \endgroup
\vfill\eject

\section{E>isagwg`h st`hn <ellhnik`h met'afrash}

<'Opwc gr'afei ka`i <o suggraf'eac st`hn dik'h tou e>isagwg'h, t`o {\rm
\TeX} e>~inai <'ena pr'ogramma stoiqeiojes'iac keim'enou (ka`i >'oqi
>epexergas'iac) polus'unjeto ka`i >'oqi >idiait'erwc filik`o st`on
qr'hsth.  A>ut`oc e>~inai <'enac l'ogoc gi`a t`on <opo~io <h
dhmotik'othta to~u {\rm \TeX} d`en >'eqei katorj'wsei n`a fj'asei a>ut`h
t~wn programm'atwn >epexergas'iac keim'enou <'opwc t`o {\rm Word}, t`o
{\rm WordPerfect}, k.lp.  

<'Enac de'uteroc l'ogoc gi`a t`on <opo~io t`o {\rm \TeX} d`en e>~inai
t'oso gnwst`o <'oso t`a progr'ammata >epexergas'iac keim'enou e>~inai
kajar`a j'ema {\rm marketing}.  T`o {\rm \TeX} e>~inai dhmio'urghma
<en`oc >an\-jr'w\-pou, to~u {\rm Donald Knuth}, <o <opo~ioc d`en >'ebale
skop`o t~hc zw~hc tou t`o n`a gem'isei t`o portof'oli tou ka`i gi''\NB
a>ut`o >apof'asise n`a t`o diaj'esei dwre`an st`o koin'o (m'esw| to~u
dikt'uou {\rm Internet}, k.lp.).  St`o >'idio pne~uma prosfor~ac,
pollo`i >'alloi sun'ebalan st`hn pro'wjhsh ka`i >ex'elixh to~u {\rm
\TeX}\null.  >'Etsi s'hmera mil~ame gi`a ((pak'eta)) <'opwc t`o {\rm
\LaTeX}, ((<ellhnik`o {\rm \TeX})), ((rwsik`o {\rm \TeX})), k.>'a., po`u
m~ac diat'ijontai dwre`an m'esw| to~u {\rm Internet} >`h >ant`i mi~ac
sumbolik~hc tim~hc >ap`o t`on >organism`o {\rm \TeX\ Users Group}. 

T`o par`on >egqeir'idio >apotele~i >ep'ishc m'eroc a>ut~hc t~hc
prosfor~ac pr`oc t`o koin`o >ap`o t`on {\rm Michael Doob}, <o <opo~ioc
e>~iqe >ep'ishc t`hn kalws'unh n`a >epitr'eyei t`hn met'afrash to~u
>egqeirid'iou st`hn neoellhnik`h gl'wssa.  T`o xek'inhma t~hc
met'afrashc st`hn neoellhnik`h gl'wssa >'egine pr`oc t`o t'eloc to~u
1993.  <O kakomo'irhc <o metafrast`hc >'elpize t'ote p`wc m'esa s`e
<'ena >'etoc j`a t`hn e>~iqe <'etoimh --- <am''\NB d'e!  Mesol'abhsan <o
Strat'oc, t`o y'aximo gi`a doulei'a, >all`a ka`i >'allec trikum'iec
proswpik'ec, m`e >apot'elesma <h met'afrash n`a <oloklhrwje~i kat`a t`hn
>'anoixh to~u 1997.

O<i >anagn~wstec to~u >egqeirid'iou j`a diapist'wsoun --- ka`i >'iswc
n`a >apogohteujo~un >ep'ishc --- <'oti t`o periss'otero m'eroc t~hc
met'afrashc >asqole~itai m`e t`hn stoiqeiojes'ia >agglik~wn >ent'upwn,
>`h genik'wtera >ent'upwn po`u sthr'izontai st`o latinik`o >alf'abhto
(p.q., gallik'a, >ispanik'a, k.lp.).  T`o {\rm \TeX} <up~hrxe
dhmio'urghma <en`oc >aggl'ofwnou ka`i sunep~wc, <'enac >arq'arioc st`o
{\rm \TeX} j`a bre~i pi`o e>'ukolo n`a k'anei t`ic pr~wtec tou dokim`ec
m`e t`o pr'ogramma stoiqeiojet'wntac >ag\-gli\-k`a ke'imena.  <Wst'oso,
st`o t'eloc t~hc met'afrashc >'eqei prosteje~i >ap`o t`on metafrast`h
<'ena kef'alaio gi`a t`hn stoiqeiojes'ia <ellhnik~wn keim'enwn m`e t`o
{\rm \TeX}\null.  <'Osoi nom'izoun <'oti kat'eqoun kal`a t`ic basik`ec
>arq`ec to~u {\rm \TeX} ka`i >endiaf'erontai m'onon gi`a t`hn
stoiqeiojes'ia <ellhniko~u keim'enou, d`en >'eqoun par`a n`a diab'asoun
t`o kef'alaio\NB 10.

T'eloc, <o metafrast`hc j`a >'hjele n`a proeidopoi'hsei t`on >anagn'wsth
<'oti o>'ute kat'eqei kan'ena ptuq'io plhroforik~hc o>'ute e>~inai
>epaggelmat'iac tupogr'afoc.  <H sq'esh tou m`e to`uc <upologist`ec
perior'izetai m'allon st`hn stoiqeiojes'ia dik~wn tou >ent'upwn m`e t`o
{\rm \TeX}\null.  E>~inai pol`u pijan`o k'apoio <'oroi n`a m`hn >'eqoun
metafrasje~i >apol'utwc s'umfwna m`e t`o go~usto to~u >anagn'wsth.  Gi`a
par'adeigma, >ant`i t~hc l'exhc ((>entol'h)), st`hn paro'usa met'afrash
>'eqoun qrhsimo\-poihje~i o<i <'oroi ((l'exh >el'egqou)) ka`i
((s'umbolo >el'eg\-qou)).  T`o giat'i qrhsimopoi'hjhkan a>uto`i o<i
<'oroi --- >all`a ka`i >'alloi par'omoioi --- <o >anagn'wsthc
pijan'otata j`a t`o katal'abei kaj`wc o<i gn'wseic tou g'urw >ap`o t`o
{\rm \TeX} j`a plhja'inoun.  >Ak'oma <o >anagn'wsthc >'iswc >anarwthje~i
giat'i >'eqei g'inei <h met'afrash st`o pali`o polutonik`o s'usthma.  <H
>ap'anthsh e>~inai a>ut`h po`u j`a >'edine <opoiosd'hpote pali`oc
tupogr'afoc po`u >exakolouje~i n`a stoiqeiojete~i s`e monotupik`ec
mhqan'ec: {\tengs T`o polutonik`o st`hn tupograf'ia >'eqei m'ia
>omorfi`a >axep'erasth!}  <'Osoi loip`on param'enete >erast`ec to~u
palio~u polutoniko~u sust'hmatoc, >agal\-li'a\-sa\-te!  <'Osoi t`o
>apeqj'aneste, d`en >'eqete par`a st`a dik'a sac >'entupa n`a b'azete
<'enan ka`i m'onon <'enan t'ono.

St`hn paro~usa met'afrash bo'hjhsan >'amesa >`h >'emmesa tre~ic f'iloi
po`u <o metafrast`hc j`a >'hje\-le n`a katanom'a\-sei.  Pr'okeitai gi`a
t`on Kwst`h I.\NB Druller'akh, po`u >'eqei fti'axei t`o pak'eto
\greektex, t`on Gi'annh Qaral'ampouc, t`on dhmiourg`o poll~wn
<ellhnik~wn ka`i >'allwn grammatoseir~wn to~u {\rm \TeX}, ka`i t`on
>Ap'ostolo Sur'opoulo, <idrut`h to~u Sull'ogou <Ell'hnwn F'ilwn to~u
{\rm \TeX}\null.  <O S'ullogoc a>ut'oc, po`u dhmiourg'hjhke m'olic t`o
1997, prosf'erei pol'utimh bo'hjeia s`e <'osouc j'eloun n`a >asqolhjo~un
m`e t`hn stoiqeiojes'ia <ellhnik~wn >ent'upwn m`e t`o {\rm \TeX}\null. 
<H die'ujuns'h tou e>~inai:
\vskip\baselineskip
\centerline{S'ullogoc <Ell'hnwn F'ilwn to~u {\rm \TeX}}
\centerline{(<Up'' <'oyh >A. Sur'opoulou)}
\centerline{28hc >Oktwbr'iou 366}
\centerline{671 00 X'anjh}
\centerline{\rm e-mail: {\tt eft@platon.ee.duth.gr}}
\centerline{\rm http://{\tt obelix.ee.duth.gr/eft}}
\vskip\baselineskip
\rightline{--- D.F.}
\rightline{M'aioc 1997}   

\section{Perieq'omena}

\line{E>isagwg`h \dotfill{} {\rm i}}
\line{E>isagwg`h st`hn <ellhnik`h met'afrash \dotfill{} {\rm iii}}
\line{Perieq'omena \dotfill{} {\rm v}}
\line{1.\ T`o xek'inhma \dotfill{} 1}
\line{\qquad{}1.1 T'i e>~inai t`o {\rm\TeX} ka`i t'i d`en e>~inai \dotfill{} 1}
\line{\qquad{}1.2 >Ap`o t`o >arqe~io {\rm\TeX} st`o >'entupo, <h meg'alh <etoimas'ia \dotfill{} 2}
\line{\qquad{}1.3 Ka'i\dots\ f'ugame! \dotfill{} 4}
\line{\qquad{}1.4 T`o {\rm\TeX} >el'egqei t`a p'anta \dotfill{} 7}
\line{\qquad{}1.5 T'i d`en k'anei t`o {\rm\TeX} \dotfill{} 9}
\line{2.\ <'Oloi o<i qarakt~hrec, meg'aloi ka`i mikro`i \dotfill{} 10}
\line{\qquad{}2.1 Meriko`i qarakt~hrec e>~inai pi`o ((sp'esial)) \dotfill{} 10}
\line{\qquad{}2.2 Stoiqeiojes'ia tonik~wn shme'iwn \dotfill{} 11}
\line{\qquad{}2.3 Tele~iec, pa~ulec, e>isagwgik'a\dots\ \dotfill{} 14}
\line{\qquad{}2.4 T'upoi stoiqe'iwn \dotfill{} 16}
\line{3.\ <H di'ataxh t~wn pragm'atwn \dotfill{} 21}
\line{\qquad{}3.1 Mon'adec, mon'adec, mon'adec \dotfill{} 21}
\line{\qquad{}3.2 <H di'ataxh t~hc sel'idac \dotfill{} 22}
\line{\qquad{}3.3 <H di'ataxh t~hc paragr'afou \dotfill{} 24}
\line{\qquad{}3.4 <H di'ataxh t~hc >ar'adac \dotfill{} 29}
\line{\qquad{}3.5 <Uposhmei'wseic \dotfill{} 31}
\line{\qquad{}3.6 <H kefal`h ka`i t`o p'odi t~hc sel'idac \dotfill{} 33}
\line{\qquad{}3.7 X'eqeila ka`i >'adeia pla'isia \dotfill{} 32}
\line{4.\ $\Bigl\{$S'unola, $\bigl\{$<upos'unola
        $\{$ka`i <upo"upos'unola$\}\bigr\}\Bigr\}$ \dotfill{} 36}
\line{5.\ Majhmatik`a qwr`ic >'agqoc! \dotfill{} 39}
\line{\qquad{}5.1 Poll`a n'ea s'umbola \dotfill{} 39}
\line{\qquad{}5.2 Kl'asmata \dotfill{} 45}
\line{\qquad{}5.3 De~iktec ka`i >ekj'etec \dotfill{} 46}
\line{\qquad{}5.4 R'izec, tetragwnik`ec ka`i >'allec \dotfill{} 47}
\line{\qquad{}5.5 Gramm'ec, p'anw ka`i k'atw \dotfill{} 47}
\line{\qquad{}5.6 <Oroj'etec, mikro`i ka`i meg'aloi \dotfill{} 48}
\line{\qquad{}5.7 K'apoiec e>idik`ec sunart'hseic \dotfill{} 49}
\line{\qquad{}5.8 >Ako'usate, >ako'usate! \dotfill{} 50}
\line{\qquad{}5.9 Majhmatik`ec parat'axeic \dotfill{} 51}
\line{\qquad{}5.10 Diakrit`ec kentrwm'enec >exis'wseic \dotfill{} 54}
\line{6.\ Stoiqhje~ite! \dotfill{} 56}
\line{\qquad{}6.1 Qrhsimopoi~hste t`o TAB \dotfill{} 56}
\line{\qquad{}6.2 <Oriz'ontia sto'iqish m`e pi`o pol'uplokec mej'odouc \dotfill{} 60}
\line{7.\ K'an'' to m'onoc sou \dotfill{} 64}
\line{\qquad{}7.1 T`o makr`u ka`i t`o kont`o \dotfill{} 64}
\line{\qquad{}7.2 Par'ametroi st`ic makroentol`ec \dotfill{} 67}
\line{\qquad{}7.3 M`e <'ena >'allo >'onoma \dotfill{} 70}
\line{8.\ T`a l'ajh e>~inai >anjr'wpina \dotfill{} 71}
\line{\qquad{}8.1 T`o xeqasm'eno >ant'io \dotfill{} 71}
\line{\qquad{}8.2 <H lanjasm'enh >`h >'agnwsth l'exh >el'egqou \dotfill{} 71}
\line{\qquad{}8.3 <H lanjasm'enh grammatoseir`a \dotfill{} 73}
\line{\qquad{}8.4 Majhmatik`a qwr`ic ta'iri \dotfill{} 74}
\line{\qquad{}8.5 >Agk'ulec qwr`ic ta'iri \dotfill{} 75}
\line{9.\ Sk'abontac l'igo baj'utera \dotfill{} 78}
\line{\qquad{}9.1 Meg'ala ka`i mikr`a >arqe~ia \dotfill{} 78}
\line{\qquad{}9.2 Megal'utera pak'eta {\rm macro} \dotfill{} 79}
\line{\qquad{}9.3 <Oriz'ontiec ka`i katak'orufec gramm`ec \dotfill{} 81}
\line{\qquad{}9.4 Pla'isia >ent`oc plais'iwn \dotfill{} 83}
\line{10.\ P'ec mou to <ellhnik'a! \dotfill{} 89}
\line{\qquad{}10.1 <H pi`o <apl`h l'ush \dotfill{} 90}
\line{\qquad{}10.2 Gi`a k'ati kal'utero \dotfill{} 93}
\line{\qquad{}10.3 K'apoioi sp'anioi <ellhniko`i qarakt~hrec \dotfill{} 96}
\line{\qquad{}10.4 <H leptom'ereia po`u k'anei t`hn diafor`a \dotfill{} 97}
\line{\qquad{}10.5 <Ellhnik`a majhmatik`a \dotfill{} 100}
\line{\qquad{}10.6 Mikr`oc >ep'ilogoc gi`a >ep'idoxouc stoiqeioj'etec \dotfill{} 101}
\line{11.\ Kat'alogoc >akolouji~wn >el'egqou \dotfill{} 103}
\ifwritinganswers
      \line{12.\ D~ws'' mou t`o q'eri sou \dotfill{} 106}
\fi

\section{T`o xek'inhma}

\subsection{T'i e>~inai t`o {\lsecfont \TeX} ka`i t'i d`en e>~inai}

>Arqik'a, >`ac do~ume poi'a e>~inai t`a >apara'ithta b'hmata gi`a  
t`hn paragwg`h <en`oc >ent'upou m`e t`o {\rm \TeX}\null.  T`o pr~wto    
b~hma e>~inai n`a <etoim'asoume <'ena >arqe~io t`o <opo~io j`a diab'asei
t`o {\rm \TeX}\null.  A>ut`o sun'hjwc >apokale~itai {\tengs >arqe~io
{\sl \TeX}} >`h ka`i {\tengs >arqe~io to~u k'wdika\/} ({\rm source
code}), ka`i mporo~ume n`a t`o <etoim'asoume m`e <'ena <opoiod'hpote
<apl`o pr'ogramma s'untaxhc keim'enou {\rm ASCII} ({\rm text editor},
<'opwc p.q., t`o {\rm EMACS}, k.>'a.)  M'alista, >e`an d`en
qrhsi\-mopoio~ume k'apoio pr'ogramma s'untaxhc keim'enou, >all`a k'apoio
pr'ogramma >epexargas'iac keim'enou (p.q., t`o {\rm WordPerfect},
k.>'a.), t'ote >`ac pros'exoume t`o >arqe~io {\rm \TeX} n`a t`o s'wsoume
<wc >arqe~io {\rm ASCII} ka`i m'onon, <'wste n`a m`hn peri'eqei
k'apoiouc per'iergouc qarakt~hrec, qam'ogela, k.lp., to`uc <opo'iouc
d`en katalaba'inei t`o {\rm \TeX}\null.  Kat'opin, pr'epei n`a tr'exoume
t`o pr'ogramma {\rm \TeX}, t`o <opo~io  diab'azei t`on k'wdika ka`i
par'agei t`o ((>arqe~io {\rm DVI})) (<h >onomas'ia {\rm DVI} pro'erqetai
>ap`o t`on <'oro {\rm DeVice Independent}, po`u st`hn >Ag\-gli\-k`h
shma'i\-nei ((anex'arthto mhqan'hmatoc)){}).  T`o >arqe~io {\rm DVI}
e>~inai >ad'unato n`a t`o diab'asei >'an\-jrwpoc; t`o >arqe~io {\rm DVI}
diab'azetai m'ono >ap`o <'ena >'allo pr'ogramma, t`o >apokalo'umeno
{\tengs <odhg`oc mhqan'hmatoc\/} ({\rm device driver}), po`u
par'agei ka`i t`o telik`o >'entupo t`o <opo~io mporo~ume n`a 
diab'asoume\TeXref{23}. Giat'i n`a <up'arqei a>ut`o t`o >arqe~io {\rm
DVI}\null?  Giat`i t`o >'idio {\rm DVI} mporo~ume n`a t`o
qrhsimopoi'hsoume gi`a n`a do~ume t`o dhmio'urghm'a mac st`hn >oj'onh
to~u termatiko~u >`h to~u proswpiko~u mac <upologist~h, >`h gi`a n`a t`o
>ektup'wsoume s`e m'ia mhqan`h fwtos'unjeshc. >E`an t`o >apo\-t'e\-lesma
st`hn >oj'onh <ikanopoie~i t`o go~usto mac, mporo~ume n`a e>'imaste
b'e\-baioi p`wc t`o >apo\-t'elesma t~hc ekt'upwshc st`hn mhqan`h
fwtos'unjeshc, s`e <'enan >ektupwt`h {\rm laser} >`h s`e
<opoio\-d'h\-pote >'allo >ektupwtik`o mhq'anhma j`a e>~inai <ol'oidio.  
<'Otan >'eqoume >'hdh fti'axei t`o >arqe~io {\rm DVI} d`en qrei'azetai
n`a xanatr'exoume t`o pr'ogramma {\rm \TeX}.

Sqhmatik'a, <h <'olh diadikas'ia, >ap`o t`hn s'untaxh to~u k'wdika      
>'ewc t`hn >ekt'upwsh, >'eqei <wc >ex~hc:
$$
{{\hbox{s'untaxh\quad} \atop \hbox{k'wdika\quad}} \atop \longrightarrow}
\lower .6cm \makebox{1.5cm}{1.5cm}%
{\vbox{\hbox to 1.5cm{\hfil >arqe~io\hfil}
       \hbox to 1.5cm{\hfil k'wdika\hfil}
       \hbox to 1.5cm{\hfil {\rm \TeX}\hfil}}
}%    
{\hbox{\quad pr'ogramma {\rm \TeX}\quad} \atop \longrightarrow}
\lower .6cm \makebox{1.5cm}{1.5cm}%
{\vbox{\hbox to 1.5cm{\hfil >arqe~io\hfil}
       \hbox to 1.5cm{\hfil {\rm DVI}\hfil}}
}%
{{\hbox{<odhg`oc} \atop \hbox{\quad mhqan'hmatoc\quad}} \atop \longrightarrow}
\lower .6cm \makebox{1.5cm}{1.5cm}%
{\vbox{\hbox to 1.5cm{\hfil >'entupo\hfil}}
       %
}%
$$

A>ut`o shma'inei <'oti d`en mporo~ume n`a do~ume t`o >'entupo st`hn     
telik'h tou morf`h kaj`wc sunt'assoume t`on k'wdika {\rm \TeX} st`o     
termatik`o >`h st`on proswpik'o mac <upologist'h.  >All`a <h <upomon'h  
mac st`o t'eloc >antamo'ibetai:  poll`a >ap`o t`a s'umbola po`u d`en
<up'arqoun sto`uc koino`uc >epexergast`ec keim'enou, <up'arqoun st`o    
{\rm \TeX}\null.  >Ekt`oc >ap`o a>ut'o, <h stoiqeiojes'ia to~u >ent'upou
mac g'inetai m`e >exairetik`h >akr'ibeia ka`i t`a >arqe~ia t~wn kwd'ikwn
mac mporo~ume n`a t`a ste'iloume >ap`o t`hn m'ia >'akrh t~hc g~hc st`hn
>'allh e>'ite m'esw| <en`oc mikro~u magnhtiko~u d'iskou  (m'iac koin~hc
disk'etac) e>'ite (pol`u pi`o gr'hgora) m'esw| >hlektroniko~u
taqudrome'iou.

S`e <eto~uto t`o >egqeir'idio, j`a >epikentr'wsoume t`hn prosoq'h mac    
st`hn dhmiourg'ia to~u >arqe'iou (to~u k'wdika) {\rm \TeX} ka`i st`o    
tr'eximo to~u >'idiou to~u progr'ammatoc {\rm \TeX} gi`a t`hn paragwg`h 
sugke\-krim'enwn >apotele\-sm'atwn s`e <'ena f'ul\-lo qartio~u.     
<Up'arqoun d'uo tr'opoi m`e to`uc <opo'iouc mpore~i n`a tr'exei kane`ic 
t`o {\rm \TeX}: <wc >adi'aleipto ({\rm batch}) >`h <wc                 
>al\-lhlo\-epidr'on ({\rm interactive}) pr'ogramma.  >E`an j'eloume t`o 
 {\rm \TeX} n`a tr'exei m'ono tou <wc >adi'aleipto, t'ote d'inoume st`on
<upologist`h t`o >arqe~io m`e t`on k'wdika {\rm \TeX}; kat'opin t`o
pr'ogramma {\rm \TeX} >epexerg'azetai t`o sugkekrim'eno >arqe~io ka`i
sta\-ma\-t'a\-ei m'onon <'o\-tan >'eqei te\-lei'w\-sei t`hn
>epexergas'ia ka`i >'eqei <etoim'asei t`o kat'al\-lh\-lo >ar\-qe~io {\rm
DVI}\null.  <'Otan t`o pr'ogramma tr'eqei <wc >al\-lhlo\-epidr`on m`e
t`on qr'hsth, <o teleuta~ioc mpore~i <opoia\-d'hpote stigm`h n`a
diak'oyei t`o {\rm \TeX} ka`i n`a k'anei metabol`ec st`on k'wdika po`u
t`o {\rm \TeX} >epexer\-g'azetai, dhl.\ <o qr'hsthc mpore~i st`hn
per'iptwsh a>ut`h n`a >allhloepidr~a m`e t`o  pr'ogramma.  T`o n`a
qrhsimopoio~ume t`o {\rm \TeX} <wc >allhloepidr'on, m~ac >epitr'epei
t`hn stigmia'ia di'orjwsh to~u keim'enou; st`hn per'iptwsh po`u t`o {\rm
\TeX} doule'uei <wc >adi'aleipto, t'ote k'anei <'opoiec diorj'wseic
jewre~i >apara'ithtec >ap`o m'ono tou ka`i <'oso pi`o kal'utera mpore~i.
<'Olec o<i >ekd'oseic to~u {\rm \TeX} po`u kukloforo~un gi`a          
proswpiko`uc <upologist'ec, kaj`wc ka`i >ark'etec po`u kukloforo~un gi`a
meg'ala <upo\-lo\-gistik`a d'iktua, dou\-le'u\-oun <wc >allhlo\-epidr'onta
progr'am\-mata.  <Wst'oso, s`e <orism'ena d'iktua po`u qrhsimopoio~un   
leitourgik`a sust'hmata <'opwc t`o {\rm MVS}, <o m'onoc tr'opoc
po`u mporo~ume n`a qrhsimopoi'hsoume t`o {\rm \TeX} e>~inai <wc
>adi'aleipto pr'ogramma.

\subsection{>Ap`o t`o >arqe~io {\lsecfont \TeX} st`o >'entupo,
<h meg'alh <etoimas'ia}

\vskip\baselineskip                           

{\parskip = 0pt \noindent
[Shme'iwsh to~u suggraf'ea ka`i to~u metafrast~h:
{\tengs A>ut`h e>~inai <h m'onh par'agrafoc to~u <'olou >egqeirid'iou   
<'opou g'inetai >anafor`a s`e m'ia sugkekrim'enh >'ekdosh (<'ena
pak'eto) {\sl \TeX} <h <opo'ia e>~inai kat'al\-lhlh m'ono gi`a
upologist`ec m`e <'ena <orism'eno leitourgik`o s'usthma.  >E`an <o
<upologist`hc po`u qrhsimopoie~ite doule'uei m`e diaforetik`o
leitourgik`o s'usthma >`h <h dik'h sac >'ekdosh to~u {\sl \TeX} e>~inai
diaforetik'h, t'ote <h par'agrafoc <eto'uth mpore~i n`a >antikatastaje~i
>ap`o t`on dik'o sac ((topik`o <odhg`o qr'hshc)) ({\sl local guide}). 
Jumhje~ite p`wc <'enac swst`oc topik`oc odhg`oc pr'epei n`a peri'eqei
t`ic parak'atw plhrofor'iec:

\item{$\bullet$} Poi'a e>~inai t`a >arqik`a proparaskeuastik`a b'hmata
  po`u pr'epei n`a >akolouj'hsei <o qr'hsthc <'wste n`a tr'exei kat'opin
  m`e >epituq'ia t`o {\sl \TeX} ka`i to`uc kat'al\-lhlouc <odhgo`uc
  mhqa\-nhm'atwn gi`a t`hn pr'obleyh >`h t`hn >ekt'upwsh to~u
  >ent'upou tou.

\item{$\bullet$} P~wc mpore~i n`a tr'exei k'apoioc t`o {\sl \TeX}.

\item{$\bullet$} P~wc diab'azetai t`o >arqe~io {\tt .log}, <'opou
  katagr'afontai leptom'ereiec <'opwc sf'almata st`on k'wdika, k.>'a.

\item{$\bullet$} P~wc mpore~i n`a de~i <o qr'hsthc t`o >'entup'o tou
  st`hn >oj'onh tou, ka`i p~wc mporei n`a t`o tup'wsei st`on >ektupwt'h.

T`o parak'atw ke'imeno >anaf'eretai st`hn qr'hsh <en`oc pol`u           
dhmofilo~uc pak'etou {\sl \TeX}, to~u {\sl em\TeX} po`u proor'izetai
gi`a proswpiko`uc <upologist`ec po`u leitourgo~un m`e t`o {\sl MS-DOS}
>`h t`o {\sl OS/2}.%
}% end slanted
]
}% end parskip = 0

S`e <eto'uth t`hn par'agrafo, j`a do~ume p~wc mpore~i n`a tr'exei       
kane`ic t`o {\rm \TeX} s`e <'enan proswpik`o <upologist`h m`e           
leitourgik`o s'usthma {\rm MS-DOS} >`h {\rm OS/2}.  <Upot'ijetai        
<'oti >'eqoume k'anei pr~wta swst`a t`hn >egkat'astash to~u
pak'etou {\rm em\TeX} st`on <upologist'h mac (m`e t`ic >apara'ithtec
>allag`ec st`a >arqe~ia {\tt autoexec.bat} ka`i {\tt config.sys}), p`wc 
>'eqoume st`hn di'ajes'h mac <'ena kat'al\-lhlo pr'ogramma s'untaxhc
keim'enou {\rm ASCII} (p.q., t`o {\tt edit} tou {\rm MS-DOS}, k'apoia
parallag`h to~u {\rm EMACS} gi`a t`o {\rm MS-DOS}, k.>'a.), ka`i <'oti  
gnwr'izoume p~wc n`a qeirisjo~ume a>ut`o t`o teleuta~io pr'ogramma.   

>Ef'' <'oson <'ola t`a parap'anw >alhje'uoun, <or'iste poi'a e>~inai t`a
b'hmata po`u pr'epei n`a >ako\-lou\-jo~ume {\tengb k'aje for`a} po`u
j'eloume n`a <etoim'asoume <'ena >'entupo m`e t`o {\rm \TeX}
(plh\-ktro\-lo\-go~ume st`on <upologist`h m'onon <'o,ti st`o parak'atw
ke'imeno >emfan'izetai m`e stoiqe~ia gra\-fo\-mhqa\-n~hc):

\item{(1)} Dhmiourgo~ume m`e t`o pr'ogramma s'untaxhc keim'enou ({\rm
  text editor}) t`o >arqe~io m`e t`on k'wdika {\rm \TeX}.

\item{(2)} >Apojhke'uoume t`o >arqe~io m`e <'ena <opoiod'hpote <'onoma
  ka`i pro'ektama {\tt .tex} (p.q., {\tt src.tex}) ka`i >egkatale'ipoume
  t`o pr'ogramma s'untaxhc.

\item{(3)} D'inoume t`o >arqe~io a>ut`o st`o pr'ogramma {\rm \TeX} gi`a
>epexergas'ia m`e t`hn >entol'h:
    \itemitem{} $>$ {\tt tex src}

M`e t`hn teleuta'ia >entol'h, xekin~a n`a tr'eqei t`o {\rm \TeX} ka`i   
di'afora mhn'umata >emfan'izontai suneq~wc st`hn >oj'onh mac. St`o
t'eloc, >e`an <'ola p~ane kal'a, >e`an dhl.\ t`o >arqe~io to~u k'wdika  
d`en peri'eqei sf'almata po`u d`en mpore~i n`a t`a diorj'wsei m'ono tou
t`o {\rm \TeX}, j`a l'aboume <'ena m'hnuma s`an t`o parak'atw:

\beginuser
Output written on src.dvi (3 pages, 1230 bytes)
Transcript written on scr.log.
\enduser

<'Otan l'aboume a>ut`o t`o m'hnuma st`hn >oj'onh mac, t'ote mporo~ume
n`a diab'asoume m`e t`o pr'ogramma s'untaxhc keim'enou {\rm ASCII} t`o
>arqe~io {\tt src.log}.  St`o {\tt src.log} >'eqoun katagrafe~i pijan`a
sf'almata t`a <opo~ia br~hke t`o {\rm \TeX} st`o >arqe~io to~u k'wdika,
<'opwc >ep'ishc o<i diorj'wseic po`u >'eginan, o<i <'opoiec dik'ec mac
paramb'aseic, k.lp.  M`e l'iga l'ogia, t`o >arqe~io {\tt src.log}
e>~inai t`o {\tengs >arqe~io >anafor~ac\/} >`h t`o ((<hmerol'ogio
katastr'wmatoc)) to~u {\rm \TeX}. 

\item{(4)}  Telik'a, >e`an t`o >arqe~io to~u k'wdik'a mac d`en peri'eqei
kan'ena sobar`o sf'alma, t`o {\rm \TeX} j`a <etoim'asei ka`i <'ena
>ak'oma >arqe~io m`e t`o >'onoma: {\tt src.dvi}.  A>ut`o t`o >arqe~io
mporo~ume n`a t`o do~ume st`hn >oj'onh mac.  >Arke~i n`a d'wsoume t`hn
>entol'h:
\itemitem{} $>$ {\tt dviscr @lj.cnf src}

Mporo~ume >ak'oma ka`i n`a t`o tup'wsoume m`e t`hn >entol'h:
\itemitem{} $>$ {\tt dvihplj @lj.cnf src}

Prosoq'h: o<i d'uo parap'anw >entol`ec e>~inai kat'allhlec m'onon     
st`hn per'iptwsh po`u >'eqoume >ektupwt`h {\rm Hewlett Packard          
Laser\-Jet} >`h >'allon sumbat`o m`e a>ut`o t`o mont'elo.  >E`an d`en   
>'eqoume a>uto~u to~u e>'idouc >ektupwt~h, t'ote kal`o j`a >~htan n`a   
sumbouleujo~ume t`o >arqe~io {\tt dvidrv.doc} po`u br'isketai st`on    
kat'alogo {\tt \\emtex\\doc\\english}.  >Eke~i, j`a bro~ume plhrofor'iec
gi`a t`o p~wc n`a tup'wsoume t`o >'entup'o mac s`e >'allou e>'idouc
>ektupwt'h.

\def\bdots{$\textfont0=\tenbf \ldots$} %%% boldface version of \ldots
%
\subsection{Ka'i\bdots\ f'ugame!}

Skop`oc loip`on to~u paiqnidio~u mac e>~inai <h dhmiourg'ia (s'untaxh)  
katall'hlwn >arqe'iwn (kwd'ikwn) {\rm \TeX} t`a <opo~ia j`a m~ac d'wsoun
telik`a <'ena >'entupo st`hn morf`h po`u >eme~ic >epiju\-mo~ume. <'Omwc
m`e t'i moi'azei <o k'wdikac {\rm \TeX}\null?  <'Ena >arqe~io k'wdika
{\rm \TeX} peri\-'eqei m'ono lati\-niko`uc qara\-kt~h\-rec po`u
mporo~ume n`a daktulo\-graf'h\-soume s`e <'ena <opoio\-d'hpote
plhktrol'ogio: mikr`a gr'ammata ka`i kefala~ia, >arijmo'uc, shme~ia
st'ixhc ka`i tonik`a shme~ia, m`e l'iga l'ogia <'olouc to`uc
sunhjism'enouc qarakt~hrec {\rm ASCII}\null.  <'Ena <apl`o >agglik`o
ke'imeno daktulografe~itai <'opwc ka`i s`e m'ia gra\-fo\-mh\-qa\-n'h.   
E>idik`ec >entol`ec d'inontai st`o pr'ogramma {\rm \TeX} m`e meriko`uc  
{\tengs e>idiko`uc qara\-kt~hrec\/} <'opwc t`o ((karr`e)) {\tt \#} ka`i
t`o ((ka`i)) {\tt \&} (<'olouc to`uc e>idiko`uc qara\-kt~hrec j`a to`uc
>exet'asoume >analutik'wtera st`a <ep'omena kef'alaia).  <Or'iste <'ena
par'adeigma <en`oc >arqe'iou {\rm \TeX}:
\beginuser
Here is my first \\TeX\\ sentence.
\\bye
\enduser       
\toindex{TeX{} } \toindex{ }
                    
>Arqik`a bl'epoume <'oti <'oloi o<i qarakt~hrec to~u parade'igmatoc  
moi'azoun s`an n`a >'eqoun grafe~i m`e t`hn gra\-fo\-mh\-qan'h.  (S`e
<'ola t`a parade'igmata po`u >akoloujo~un s`e <eto~uto t`o
>eg\-qei\-r'i\-dio, <'o,ti >emfan'izetai m`e stoiqe~ia
gra\-fo\-mh\-qa\-n~hc <upot'ijetai p`wc t`o plh\-ktro\-log\-o~ume st`o
termatik`o >`h st`on proswpik'o mac  <upologist'h.)  Kat`a de'utero
l'ogo, >`ac pros'exoume p`wc <o qarakt'hrac t~hc >antipl'agiac gramm~hc
{\tt \\} (st`hn >Agglik'h, >all`a ka`i st`hn >argk`o t~hc plhroforik~hc,
<o qarakt'hrac a>ut`oc >onom'azetai {\rm backslash}) >emfan'izetai
tre~ic for`ec st`o par'adeigma.  S'untoma pr'okeitai n`a do~ume <'oti <h
>antipl'agia e>~inai <'enac >ap`o to`uc e>idiko`uc qarakt~hrec po`u
pro\-ana\-f'erame.  Loip'on, >`ac fti'axoume <'ena >arqe~io {\rm ASCII}
po`u n`a peri'eqei t`o parap'anw par'adeigma. >`Ac t`o tr'exoume
kat'opin m`e t`o {\rm \TeX} gi`a n`a fti'axoume t`o >arqe~io {\rm
DVI}\null.  T'eloc, m`e t`on kat'allhlo <odhg'o, >`ac t`o tup'wsoume.
>E`an <'ola p~ane kal'a, j`a l'aboume st`a q'eria mac m'ia tupwm'enh
sel'ida m`e t`hn >ex~hc fr'ash:

{\rm Here is my first \TeX\ sentence.}

St`o k'atw m'eroc to~u f'ullou, j`a pr'epei >ep'ishc n`a <up'arqei ka`i 
<o >arijm`oc t~hc sel'idac.  >E`an t`a kataf'erame, m~ac >ax'izoun
sug\-qarht'hria!  >Ap`o t`hn stigm`h po`u mpor'esame n`a <etoim'asoume
<'ena t'etoio <apl`o >'entupo m`e t`o {\rm \TeX}, e>~inai j'ema l'igou
qr'onou m'eqri n`a ft'asoume st`o shme~io n`a stoiqeiojeto~ume t`a
pl'eon pol'uploka >'entupa. T'wra <'omwc >`ac sugkr'inoume t'i d'wsame
>eme~ic st`o {\rm \TeX} ka`i t'i m~ac >ep'estreye a>ut'o.  T`ic <apl`ec
>agglik`ec l'exeic t`ic plh\-ktro\-log'h\-same st`on k'wdik'a mac <wc
>'eqoun, dhl.\ qwr`ic t'ipota t`o >idia'itero, ka`i t`o {\rm \TeX} m~ac
t`ic stoiqeioj'ethse m`e <aplo`uc >'orjiouc qarakt~hrec (<'opwc j`a
do~ume parak'atw pr'okeitai gi`a qarakt~hrec t'upou {\rm roman}). 
>All`a <'omwc, t`hn l'exh (({\rm \TeX})), t`hn <opo'ia d`en mporo~ume
n`a t`hn plhktrolog'hsoume st`o termatik`o mi~ac ka`i o<i qarakt~hrec
thc, {\rm T}, {\rm E} ka`i {\rm X}, d`en br'iskontai st`hn >'idia
<oriz'ontia e>uje'ia gramm'h, t`hn gr'ayame <wc m'ia l'exh po`u xekin~a
m`e t`on e>idik`o qarakt'hra t~hc >antipl'agiac.  >Ex a>it'iac a>ut~hc
t~hc antipl'agiac, t`o {\rm \TeX} kat'alabe <'oti <h l'exh {\tt \\TeX}
e>~inai k'ati t`o e>idik`o ka`i >'epraxe kat'allhla.  T`a periss'otera
s'umbola t`a <opo~ia d`en e>~inai koin`a gr'ammata, >arijmo`i >`h
shme~ia st'ixhc t`a gr'afoume st`on k'wdika (>arqe~io {\rm \TeX}) <wc
l'exeic po`u t`o pr~wto touc gr'amma e>~inai <h >antipl'agia gramm'h. 
>E`an pros'exoume l'igo >ak'oma periss'otero, j`a do~ume p`wc ka`i <h
l'exh (({\rm first})) >'eqei >all'axei:\TeXref{4} t`a d'uo pr~wta
gr'ammat'a thc >'eqoun <enwje~i ma\-z`i ka`i d`en <up'arqei <h diakrit`h
tele'ia >ep'anw >ap`o t`o gr'amma (({\rm i})). A>ut`h e>~inai m'ia
koin`h praktik`h t~wn tupogr'afwn; <orism'enoi sunduasmo`i gramm'atwn
ka`i >'allwn tupografik~wn stoiqe'iwn >anti\-kaji\-st~w\-ntai >ap`o
<'ena stoiqe~io t`o <opo~io >apokale~itai st`hn >Agglik`h {\rm ligature}
ka`i st`hn <Ellhnik`h {\tengs s'unjeto\/} >`h {\tengs pol\-la\-pl`o
stoiqe~io}. <O l'ogoc <'uparxhc t~wn s'unjetwn stoiqe'iwn e>~inai
kajar`a a>isjhtik'oc; >arke~i n`a sugkr'inoume t`a d'uo pr~wta gr'ammata
t~hc l'exhc (({\rm first})) m`e t`a >ant'istoiqa t~hc l'exhc (({\rm
f{}irst})), gi`a n`a do~ume t`hn diafor'a.  T'eloc, s''\NB a>ut`o t`o
mikr`o par'adeigma, <up'arqei <h l'exh {\tt \\bye}\toindex{bye} <h
<opo'ia d`en m~ac d'inei t'ipota st`o f'ullo to~u qartio~u po`u      
>'eqoume st`a q'eria mac.  A>ut`h <h >entol'h --- giat`i per`i >entol~hc
pr'okeitai --- <apl`a l'eei st`o {\rm \TeX} <'oti st`o shme~io a>ut`o
telei'wnei t`o ke'imeno po`u e>~inai pr`oc stoi\-qeio\-j'e\-thsh.  J`a
m'ajoume pol`u periss'oterec t'etoiec >entol`ec kaj`wc j`a proqwro~ume
<'olo ka`i baj'utera st`a mustik`a to~u {\rm \TeX}.

>`Ac r'ixoume <'omwc m'ia mati`a ka`i st`o >arqe~io {\tt .log} po`u m~ac
dhmio'urghse t`o {\rm\TeX}\null.  T`o >ar\-qe~io a>ut`o mpore~i n`a      
diaf'erei >ap`o >'ekdosh s`e >'ekdosh to~u {\rm \TeX} gi`a
diaforetiko`uc <upologist`ec ka`i diaforetik`a leitourgik`a sust'hmata.  
<'Omwc genik'a, j`a pr'epei n`a de'iqnei k'apwc >'etsi:
\beginuser
1.\ This is TeX, Vers.\ 3.14159 (preloaded format=plain 92.6.8) 97.1.18 12:10
2.\ **src
3.\ (src.tex [1] )
4.\ Output written on src.dvi (1 page, 256 bytes).
\enduser

<'Opwc >'hdh >anaf'erame, a>ut`o e>~inai t`o >arqe~io po`u j`a          
peri'eqei <'ola t`a mhn'umata gi`a sf'al\-mata st`on k'wdik'a mac. St`hn
>ar'ada 3, t`o {\tt (src.tex\ } de'iqnei <'oti t`o {\rm \TeX} >'arqise
n`a diab'azei a>ut`o t`o >arqe~io.  <H parous'ia to~u {\tt [1]} de'iqnei
<'oti <h selida\NB 1 >'eqei >'hdh stoiqeiojethje~i. >E`an <up~hrqan
l'ajh st`hn sel'ida\NB 1, j`a e>'iqan katagrafe~i s''\NB a>ut`o >akrib~wc
t`o shme~io.

\exercise Prosj'este m'ia >ak'oma pr'otash st`o >arqik`o par'adeigma,
>'etsi <'wste t`o >arqe~io {\rm \TeX} n`a de'iqnei <wc >ex~hc:
\beginuser
Here is my first \\TeX\\ sentence.
I was the one who typeset it!
\\bye
\enduser

\noindent
Tr'exte t`o {\rm \TeX} ka`i de~ite t`o >apot'elesma.  E>~inai <h
de'uterh pr'otash stoiqeiojethm'enh s`e m'ia n'ea >ar'ada?

\exercise T'wra prosj'este t`hn parak'atw >entol`h st`o >arqe~io:
\beginuser
\\nopagenumbers
\enduser

\noindent 
Mant'eyte t'i j`a sumbe~i <'otan j`a tr'exete t`o n'eo >arqe~io to~u
k'wdika m`e t`o {\rm \TeX}\null.  T'wra, doki\-m'aste st`hn pr'axh n`a
de~ite t'i >akrib~wc sumba'inei.
\toindex{nopagenumbers}

\exercise Prosj'este tre~ic >`h t'esseric >ak'oma prot'aseic st`on     
k'wdika (dhl.\ st`o >arqe~io).  Qrhsimopoi~hste latinik`a gr'ammata,
>arijmo'uc, tele~iec, k'ommata, >erwthmatik`a ka`i jau\-ma\-sti\-k'a,
>all`a >'oqi >'allou e>'idouc s'umbola (>idia'itera >apof'ugete n`a
b'al\-lete <ellhnik`a gr'am\-ma\-ta --- gi''\NB a>ut`a j`a mil'hsoume
sto kef'alaio\NB 10!).

\exercise >Af~hste m'ia ken`h >ar'ada ka`i prosj'este kat'opin merik`ec 
>ak'oma prot'aseic.  >'Etsi j`a de~ite p~wc mpore~ite n`a l'abete st`o
>'entup'o sac n'eec paragr'afouc.                
\medskip

>'Eqoume de~i m'eqric >ed~w t`hn basik`h >arq`h gi`a t`hn <etoimas'ia   
to~u k'wdika (>`h >arqe'iou) {\rm\TeX}:  <h morf`h tou keim'enou st`on
k'wdika {\rm \TeX} d`en e>~inai <h >'idia m`e a>ut`h to~u teliko~u
>ent'upou. D`en mporo~ume, gi`a par'adeigma, n`a prosj'esoume
megal'utero ken`o di'asthma metax`u d'uo l'exewn s`e m'ia >ar'ada to~u
>ent'upou m`e t`o n`a b'aloume merik`a epipl'eon ken`a diast'hmata st`on
k'wdika. <'Ena >`h ka`i peris\-s'otera ken`a diast'hmata st`on k'wdika
j`a m~ac d'wsoun t`o >'idio ken`o di'asthma st`hn >ar'ada to~u >ent'upou.
>Epipl'eon, <'opwc >'iswc j`a t`o perim'ename, m'ia l'exh st`o t'eloc
m'iac gramm~hc to~u k'wdika d`en j`a <enwje~i m`e t`hn <ep'omenh l'exh
st`o telik`o >'entupo.  M'alista, meri\-k`ec for`ec <'otan doule'uoume
<'ena ke'imeno st`o <opo~io e>~inai pol`u pijan`o n`a g'inoun poll`ec
ka`i sh\-manti\-k`ec >allag'ec, >'iswc e>~inai protim'wtero n`a gr'afoume
k'aje pr'otash <wc m'ia xeqwrist`h >ar'ada.  <Wst'oso, ken`a
diast'hmata st`hn >arq`h m'iac gramm~hc k'wdika, p'antote parabl'epontai
>ap`o t`o {\rm\TeX}.

\exercise Prosj'este t`hn parak'atw pr'otash <wc m'ia n'ea par'agrafo  
st`on k'wdik'a sac ka`i kat'opin tr'exte t`o {\rm \TeX} gi`a n`a t`hn
stoiqeiojet'hsei:
\beginuser
Congratulations! You received a grade of 100\% on your latest
examination.
\enduser

\noindent
T`o s'umbolo ((>ep`i to~ic <ekat`o)) {\tt\%} qrhsimopoie~itai gi`a
sq'olia >ent`oc to~u >arqe'iou {\rm\TeX}\null.  <Oti\-d'h\-pote
>akolouje~i met`a >ap`o a>ut`o t`o s'umbolo st`hn >'idia gramm`h to~u
k'wdika parabl'epetai >ap`o t`o {\rm \TeX}\null. >`Ac pros'exoume
>ak'oma <'oti ka`i t`o ken`o di'asthma po`u kanonik`a qwr'izei t`hn
teleuta'ia l'exh t~hc pr'wthc gramm~hc {\tt 100} >ap`o t`hn pr'wth
l'exh t~hc de'uterhc gramm~hc {\tt examination} >'eqei qaje~i.  T'wra,
>`ac b'aloume m'ia >antipl'agia >empr`oc >ap`o t`o s'umbolo {\tt\%} gi`a
n`a diorjwje~i <h pr'otash.
\toindex{\%}

\exercise Prosj'este t`hn <ep'omenh pr'otash <wc m'ia n'ea par'agrafo:
\beginuser
You owe me \$10.00 and it's about time you sent it to me!
\enduser

\noindent
<H fr'ash t~hc teleuta'iac >'askhshc j`a m~ac d'wsei <'ena m'hnuma
l'ajouc st`o >arqe~io  {\tt .log}. >E`an tr'eqoume t`o {\rm \TeX} <wc
>allhloepidr'on, t'ote t`o {\rm \TeX} j`a m~ac d'wsei <'ena m'hnuma
st`hn >oj'onh ka`i j`a diak'oyei t`hn leitour\-g'ia tou; st`hn
per'iptwsh a>ut`h pr'epei n`a pi'esoume t`o pl~h\-ktro {\rm Enter} >`h 
{\rm Return} gi`a n`a suneq'isei t`o {\rm \TeX} t`hn stoiqeiojes'ia to~u
k'wdika.  <'Omwc, t`o >apot'elesma st`o qart`i d`en j`a e>~inai a>ut`o
po`u perim'enoume.  >E`an >ano'ixoume ka`i diab'asoume t`o >arqe~io {\tt
.log}, j`a p'aroume m'ia ge'ush p~wc katagr'afei t`a l'ajh mac t`o {\rm
\TeX}\null. >All`a gi`a t`hn <'wra >`ac m`hn d'wsoume kam'ia shmas'ia
st`o peri'e\-qomeno t~wn mhnum'atwn a>ut~wn, mi~ac ka`i j`a po~ume pol`u
periss'otera pr'agmata gi`a l'ajh parak'atw (ka`i gi`a t`o sugkekrim'eno
po`u m'olic e>'idame). Kal'utera, >`ac diorj'wsoume t`on k'wdik'a mac
b'azontac m'ia >anti\-pl'a\-gia >empr`oc >ap`o t`o s'umbolo {\tt \$}
ka`i >`ac xanatr'exoume t`o {\rm \TeX}\null. (<Up'arqoun l'igoi
qara\-kt~hrec, <'opwc t`o ((>ep`i to~ic <ekat`o)) >`h t`o s'umbolo to~u
dolar'iou, to`uc <opo'iouc qrhsi\-mopoie~i t`o {\rm \TeX} gi`a
<orism'enouc e>idiko`uc skopo'uc.  S'untoma, j`a do~ume <'enan
p'inaka m`e <'olouc a>uto`uc to`uc qarakt~hrec.)
\toindex{\$}

\subsection{T`o {\lsecfont \TeX} >el'egqei t`a p'anta}

E>'idame parap'anw <'oti <h >antipl'agia >'eqei <'enan e>idik`o r'olo.  
<Opoia\-d'hpote >akolouj'ia qa\-ra\-kt'hrwn xekin~a m`e t`hn
>antipl'agia j`a diermhneuje~i kat`a tr'opo e>idik`o >ap`o t`o {\rm
\TeX} <'otan t`o pr'ogramma a>ut`o diab'azei t`on k'wdik'a mac.  M'ia
t'etoia seir`a qarakt'hrwn >apokale~itai {\tengs >ako\-louj'ia
>el'egqou\/} ({\rm control sequence}) >`h {\tengs >entol'h\/} ({\rm
command}).  M'alista, <up'arqoun d'uo t'upoi >ako\-louji~wn >el'egqou:
m'ia {\tengs l'exh >el'egqou\/} ({\rm control word}) >apotele~itai >ap`o
m'ia >anti\-pl'a\-gia >ako\-lou\-jo'u\-menh >am'eswc >ap`o m'ia seir`a
latinik~wn gramm'atwn  (p.q., {\tt \\TeX}); <'ena {\tengs s'umbolo
>el'egqou\/} ({\rm control symbol}) >apotele~itai >ap`o m'ia
>anti\-pl'agia >ako\-lou\-jo'u\-menh >ap`o <'enan  ka`i monadik`o
qarakt'hra po`u <'omwc d`en e>~inai gr'amma to~u latiniko~u >alfab'htou
(p.q., {\tt \\\$})\null.  Mi~ac ka`i t`o ken`o di'a\-sthma d`en e>~inai
gr'amma to~u latiniko~u >alfab'htou, m'ia >antipl'agia
>ako\-lou\-jo'u\-menh >ap`o <'ena ken`o di'asthma >apotele~i gi`a t`o
{\rm \TeX} <'ena kanonik`o s'umbolo >el'egqou.
\TeXref{7--8}
St`hn sun'eqeia <eto'utou to~u >egqeirid'iou, <'otan j`a j'e\-loume     
n`a ton'isoume <'oti pr'ag\-mati <up'ar\-qei s`e k'apoio shme~io ken`o  
di'asthma, j`a qrhsimopoio~ume t`o e>idik`o s'um\-bolo\NB {\tt\sp}\null.
<O >'idioc sumbolism`oc <up'arqei ka`i st`o {\sl \TeX book}.

<'Otan  t`o {\rm \TeX} diab'azei t`on k'wdik'a mac ka`i sunant~a m'ia
>antipl'agia t~hc <opo'iac <'epetai <'ena latinik`o gr'amma, t'ote t`o  
pr'ogramma >antilamb'anetai <'oti pr'okeitai gi`a m'ia l'exh >el'egqou. 
T`o {\rm \TeX} suneq'izei n`a diab'azei t`a gr'ammata a>ut~hc t~hc      
l'exhc <'ena-pr`oc-<'ena, >'ewc <'otou sunant'hsei k'apoion qarakt'hra
po`u n`a m`hn >apotele~i gr'amma to~u latiniko~u >alfab'htou.  Gi`a     
par'adeigma, >e`an <o k'wdik'ac mac peri'eqei t`hn pr'otash

\displaytext{\tt I like \\TeX!}

\noindent
<h l'exh >el'egqou {\tt \\TeX} termat'izetai >ap`o t`o jaumastik'o, t`o 
<opo~io d`en >apo\-tele~i gr'amma to~u latiniko~u >alfab'htou. <Wst'oso,
a>ut`o m~ac dhmiourge~i <'ena pr'oblhma >e`an j'eloume n`a <up'arqei     
<'ena ken`o di'asthma met`a >ap`o m'ia l'exh >el'egqou.  >'Estw, p.q.,
<'oti >'eqoume t`hn pr'otash

\displaytext{\tt I like \\TeX and use it all the time.}

\noindent St`on k'wdika a>ut'o, <h l'exh >el'egqou {\tt\\TeX}
termat'izetai >ap`o <'ena ken`o di'asthma (t`o <opo~io, pro\-fa\-n~wc,
d`en e>~inai gr'amma to~u latiniko~u >alfab'htou).  Sunep~wc st`o
>'entup'o mac, o<i l'exeic (({\rm\TeX})) ka`i (({\rm and})) j`a
koll'hsoun metax'u touc.  T`o n`a prosj'esoume periss'otera ken`a
diast'hmata metax`u t~wn d'uo l'exewn st`on k'wdika d`en >wfele~i s`e
t'ipota, >epeid'h, <'opwc proanaf'erame, t`o {\rm \TeX} d`en k'anei
di'akrish metax`u <en`oc >`h poll~wn ken~wn diasthm'atwn to~u k'wdika. 
<'Omwc, m`e t`o n`a prosj'esoume m'ia >antipl'agia ka`i <'ena ken`o
di'asthma (dhl.\ t`o s'umbolo >el'egqou {\tt\\\sp}) >am'eswc met`a t`hn
l'exh >el'egqou, j`a d'wsoume st`o {\rm \TeX} n`a katal'abei <'oti <h
l'exh >el'egqou tele'iwse ka`i j`a t`o >anagk'asoume n`a b'alei >eke~i
<'ena ken`o di'asthma.  E>~inai pr'agmati pol`u e>'ukolo n`a xeqn~a
kane`ic n`a b'azei k'ati <'opwc {\tt\\\sp} met`a >ap`o l'exeic >el'egqou
--- st`a s'igoura <'oloi k'anoun a>ut`o t`o l'ajoc toul'aqiston m'ia
for`a kaj`wc maja'inoun t`o {\rm \TeX}\null.

\exercise <Etoim'aste <'enan k'wdika (>arqe~io) {\rm \TeX} po`u n`a s~ac
d'wsei t`hn <ep'omenh par'agrafo:

{\rm
I like \TeX! Once you get the hang of it, \TeX{} is really easy to use.
You just have to master the \TeX nical aspects.
}

O<i periss'oterec l'exeic >el'egqou >apotelo~un >agglik`a >akr'wnuma 
po`u d'inoun k'apoia >'endeixh gi`a t`o t'i skop`o
>exu\-phreto~un.  Gi`a par'a\-deigma, mpo\-ro~ume n`a qrhsimopoi'hsoume
t`hn l'exh {\tt \\par} gi`a n`a dhl'wsoume <'oti xekino~ume m'ia n'ea
par'agrafo, >ant`i n`a k'anoume a>ut`o >af'h\-nontac m'ia ken`h
gramm`h st`on k'wdika.  Profan~wc, <h l'exh >el'egqou {\tt \\par} >'eqei
>onomasje~i >ap`o t`hn l'exh {\rm paragraph} (<h <opo'ia d`en e>~inai
m'al\-lon >apokleistik`a >ag\-glik'h).
\toindex{par}

\subsection{T'i d`en k'anei t`o {\lsecfont \TeX}}

T`o {\rm \TeX} e>~inai >'aristo st`hn stoiqeiojes'ia >ent'upwn, >all`a  
<up'arqoun <orism'ena pr'agmata st`a <opo~ia d`en e>~inai ka`i t'oso
kal'o.  <'Ena >ap`o a>ut`a t`a pr'agmata e>~inai t`a sq'hmata. Mporo~ume
n`a >af'hnoume meg'alouc keno`uc q'wrouc gi`a t`hn met'epeita >'enjesh
sqhm'atwn, >all`a t`o >'idio t`o {\rm \TeX} d`en e>~inai pr'ogramma gi`a
t`hn paragwg`h s'unjetwn grafik~wn parast'asewn.  <Wst'oso t`o {\rm
\TeX} >epitr'epei t`hn a>ut'o\-math >'enjesh sqhm'atwn po`u peri'eqontai
s`e >arqe~ia e>ik'onwn (p.q., >arqe~ia {\rm bitmap}) m`e t`hn l'exh
>el'egqou {\tt \\special}\TeXref{228--229}, >all`a gi'' a>ut`o pr'epei
n`a <up'arqei ka`i <o ka\-t'al\-lhloc <odhg`oc po`u n`a mpore~i n`a
diab'azei t'etoiou e>'idouc >arqe~ia.

T`o {\rm \TeX} <etoim'azei p'anta >'entupa t~wn <opo'iwn o<i >ar'adec   
e>~inai <oriz'ontiec ka`i pot`e pl'agiec.  M`e t`o {\rm \TeX}, e>~inai  
genik`a >ad'unato n`a pro\-stejo~un s`e <'ena >'entupo katak'orufec
>ar'adec, dhl.\ s`e t'etoia morf`h <'wste o<i gramm`ec b'ashc t~wn
>ar'adwn n`a e>~inai par'allhlec m`e t`hn me\-ga\-l'u\-te\-rh pleu\-r`a
to~u qartio~u.  M'ono <orism'enoi <odhgo`i mporo~un n`a kataf'eroun
t'etoia >apotel'esmata  (p.q., <o <odhg`oc {\tt dvips} gi`a >ektu\-pwt`h
{\rm Post\-Script}); t`o {\rm \TeX} >ap`o m'ono tou d`en mpore~i. 
>Ep'ishc, t`o {\rm \TeX} d`en t`a kataf'ernei pol`u kal`a ka`i m`e t`ic
megej'unseic t~wn stoiqe'iwn, t`hn qr'hsh stoiqe'iwn diaforetiko~u
e>'idouc, dhl.\ diaforetik~wn {\tengs grammatoseir~wn}, ka`i t`ic
parallag`ec t~wn grammatoseir~wn po`u >onom'azontai {\tengs t'upoi\/}
(p.q., >'orjia, pl'agia, >'entona, k.>'a.). 

E>'idame m'eqric >ed~w <'oti gi`a k'aje >'entupo po`u >'eqoume n`a 
<etoim'asoume, <up'arqei <'enac k'ukloc >ergasi~wn po`u pr'epei n`a     
>ektel'esoume: {\tengs s'untaxh k'wdika $\to$ tr'eximo to~u {\sl\TeX}
$\to$ <od'hghsh s`e pr'o\-ble\-yh >`h >ekt'upwsh}. A>ut`o >isq'uei gi`a
<'olec t`ic >ekd'oseic to~u {\rm\TeX}, ak'omh ka`i gi''\NB a>ut`ec po`u
<up'arqoun gi`a meg'ala <upologistik`a d'iktua.  Dhl., d`en e>~inai
dunat`o n`a gr'afoume t`on k'wdik'a mac ka`i t`hn >'idia stigm`h n`a
bl'epoume st`hn >oj'onh t`o dhmio'ur\-ghm'a mac; pr'epei p'anta n`a
>epa\-na\-lam\-b'anoume t`on >'idio k'uklo >ergasi~wn.  <Wst'oso
<orism'enec parallag`ec to~u {\rm \TeX} >epitr'epoun t`o sqed`on
taut'oqrono gr'ayimo to~u k'wdika ka`i parous'iash to~u
stoiqeiojethm'enou >ent'upou (>'iswc m`e l'iga deute\-r'olepta
kaju\-st'erhsh).  Kaj`wc o<i <upologist`ec ka`i o<i
mikro\-epexer\-gast`ec g'inontai <'olo ka`i pi`o gr'hgoroi, mpore~i
s'untoma n`a do~ume belti'wseic st`o j'ema a>ut'o.

\section{<'Oloi o<i qarakt~hrec, meg'aloi ka`i mikro`i}

\subsection{Meriko`i qarakt~hrec e>~inai pi`o ((sp'esial))}

E>'idame st`o prohgo'umeno kef'alaio <'oti <o k'wdikac {\rm \TeX}
gr'afetai st`o megal'utero m'eroc tou <wc <apl`o ke'imeno
daktulo\-grafh\-m'eno st`hn grafo\-mhqa\-n'h.  >All`a e>'idame >ep'ishc
<'oti <h >antipl'agia mpore~i n`a qrhsime'usei gi`a d'uo toul'aqiston
skopo'uc. Mpore~i n`a qrhsimopoihje~i gi`a n`a m~ac d'wsei s'umbola >`h
sunduasmo`uc sumb'olwn po`u d`en <up'ar\-qoun st`o plhktrol'ogi'o mac:
gr'afontac st`on k'wdik'a mac {\tt \\TeX}, lamb'anoume t`o s'umbolo
(log'otupo, gi`a t`hn >akr'ibeia) {\rm \TeX}\null.  <H >anti\-pl'agia
mpore~i >ak'oma n`a qrhsimopoihje~i gi`a n`a d'wsoume st`o pr'ogramma
{\rm \TeX} <orism'enec <odhg'iec: t`o {\tt \\bye} diat'azei t`o {\rm
\TeX} n`a stamat'hsei n`a diab'azei t`o >arqe~io m`e t`on k'wdika. 
Genik'a, m'ia l'exh st`on k'wdika po`u xekin~a m`e t`hn >antipl'agia
<ermhne'uetai >ap`o t`o {\rm \TeX} <wc k'ati po`u >apaite~i
>idia'i\-terh prosoq'h.  <Up'arqoun merik`ec <ekaton\-t'adec t'etoiwn
>akolouji~wn (l'exewn >`h sumb'olwn) >el'egqou po`u t`o {\rm \TeX}
gnwr'izei >ek t~wn prot'erwn. >Epi\-pl'eon, <'opwc ja do~ume >arg'otera,
mporo~ume n`a <or'isoume ka`i >eme~ic t`ic dik'ec mac par'omoiec l'exeic
>el'egqou.  Sunep~wc <h >anti\-pl'agia >eqei meg'alh shmas'ia. 
Parak'atw, j`a xod'e\-youme k'amposo qr'ono prospaj'wntac n`a m'ajoume
t`hn qr'hsh merik~wn l'exewn  >el'egqou; e>utuq~wc, st`hn kajhmerin`h
pr'axh, m~ac >arke~i m'onon <'enac mikr`oc >arij\-m`oc t'etoiwn l'exewn.

<Up'arqoun d'eka qarakt~hrec o<i <opo~ioi, <'opwc ka`i <h >antipl'agia, 
qrhsi\-mopoio~u\-ntai >ap`o t`o {\rm \TeX} gi`a e>idiko`uc skopo`uc,    
ka`i j`a pr'epei t'wra n`a do~ume poio'i e>~inai a>uto'i.\TeXref{37--38}
P~wc <'omwc j`a mporo'u\-same n`a gr'ayoume  m'ia pr'otash po`u n`a     
peri'eqei <'enan >ap`o a>uto`uc to`uc qarakt~hrec? >'Eqontac a>ut`o     
st`on no~u mac, j`a rwt'hsoume:

\item{---} Poio'i e>~inai o<i di'aforoi e>idiko`i qarakt~hrec?

\item{---} P~wc mporo~ume n`a l'aboume <'enan e>idik`o qarakt'hra st`o
>'entup'o mac >e`an k'ati t'etoio e>~inai >apara'ithto?

<Or'iste <'enac p'inakac m`e <'olouc to`uc e>idiko`uc qarakt~hrec ka`i  
t`o p~wc gr'afetai <o kaj'enac (dhl.\ o<i >apara'ithtec >entol`ec po`u
pr'epei n`a grafo~un st`o >arqe~io {\rm \TeX}) prokeim'enou n`a to`uc
l'aboume st`o >'entup'o mac:                

\maketable [Qarakt~hrec po`u >apaito~un prosoq`h]
\halign{
   \strut \hfil\rm #\hfil & \quad # \hfil & \quad\hfil\tt #\hfil\cr
   \tengb Qarakt'hrac      & \tengb Skop'oc & \tengb K'wdikac gi`a t`hn \cr
                           &                & \tengb >emf'anish \cr
   \noalign{\smallskip} \noalign{\hrule} \noalign{\smallskip}
   $\backslash$ & >Akolouj'iec >el'egqou (>entol'ec)  & \$\\backslash\$  \cr
   $\{$ & >'Anoigma sun'olou                          & \$\\\lb\$    \cr
   $\}$ & Kle'isimo sun'olou                          & \$\\\rb\$    \cr
   \%   & Sq'olia                                     & \\\%         \cr
   \&{} & E>ujugr'ammish keim'enou ka`i pin'akwn      & \\\&         \cr
   \~{} & >Adi'akopto di'asthma >ar'adac (s'undesmoc) & \\\~{}\lb\rb \cr
   \${} & >Arq`h >`h t'eloc majhmatiko~u keim'enou    & \\\$         \cr
   \^{} & Majhmatiko`i >ekj'etec                      & \\\^{}\lb\rb \cr
   {\rm\underscore} & Majhmatiko`i de~iktec           & \\\_{}\lb\rb \cr
   \#{} & Sumb'olo >antikat'astashc s`e n'eo <orism'o & \\\#         \cr
       }

\toindex{\lb}
\toindex{\rb}
\toindex{\%}
\toindex{\&}
\toindex{\~{}}
\toindex{\$}
\toindex{\^{}}
\toindex{\_{}}
\toindex{\#}


\subsection{Stoiqeiojes'ia tonik~wn shme'iwn}

Ka`i t'wra j`a do~ume merik`a >ap`o t`a kal`a to~u {\rm    \TeX}\null! 
M'eqri stig\-m~hc >'eqoume qrhsimo\-poi'hsei t`o {\rm \TeX} m'ono gi`a
n`a m~ac stoiqeiojet'hsei k'apoiec >agglik`ec fr'aseic.  <'Omwc t'wra
j`a >arq'isoume n`a k'anoume pr'agmata po`u e>~inai >ap`o pol`u d'uskolo
>'ewc >ad'unato n`a g'inoun m`e m'ia grafo\-mhqan'h.  E>idik'wtera, j`a
do~ume p~wc e>~inai dunat`o n`a gr'afoume lati\-niko`uc qara\-kt~hrec
po`u >'eqoun ka`i t'onouc!  P~wc j`a gr'afame <'enan t'ono s`e <'ena
plh\-ktro\-l'ogio po`u d`en >'eqei t'onouc?  <'Opwc ka`i st`hn
per'iptwsh to~u log'otupou {\rm \TeX}, >'etsi ka`i >ed~w e>~inai
>apara'i\-thto n`a gr'ayoume st`on k'wdik'a mac l'exeic po`u xekino~un
m`e t`hn >anti\-pl'agia.  Gi`a t`hn gallik`h  l'exh {\rm premi\`{e}re}
(<h <opo'ia >'eqei per'asei ka`i st`hn gl'wssa mac <wc ((premi'era)){}),
qrei'azetai n`a gr'ayoume st`on k'wdik'a mac    {\tt premi\\`ere}
(>'iswc n`a qreiasje~i n`a y'axoume lig'aki gi`a n`a bro~ume t`o
s'umbolo t~hc dase'iac {\tt `} st`o plhktrol'ogi'o mac, >all`a k'apou
>eke~i br'isketai\fnote{>E`an t`o s'umbolo a>ut`o d`en <up'arqei st`o
plhktro\-l'ogi'o mac, mpo\-ro~ume n`a gr'a\-you\-me st`on k'wdika {\tt
\\lq\lb\rb}.  Kat`a par'o\-moio  tr'opo, gr'afontac st`on k'wdika {\tt
\\rq\lb\rb} j`a l'aboume t`o s'um\-bolo\NB{\tt '}.  Mporo~ume n`a
jum'omaste a>ut`ec t`ic >entol`ec <wc suntomo\-graf'iec t~wn
>agglik~wn <'orwn {\rm left quote} ka`i {\rm right quote}. >Ak'oma,
gr'afontac {\tt \\lq\\lq\lb\rb} ka`i {\tt \\rq\\rq\lb\rb} lamb'anoume
t`a sunh\-jism'ena e>isagwgik`a t~wn >agglik~wn: {\rm ``} ka`i\NB{\rm
''}. (Prosoq'h: t`a e>isagwgik`a t~wn <ellhnik~wn d`en e>~inai <'omoia
m`e a>ut`a t~wn >agglik~wn >`h germanik~wn k.>'a.\ >ent'upwn!  T`a
<ellhnik`a e>isagwgik`a ((\NB ka`i\NB )) j`a t`a >exet'asoume st`o
kef'alaio\NB 10.) <'Omwc m`e >ento\-l`ec <'opwc {\tt \\lq\lb\rb}, d`en
j`a kataf'e\-roume n`a l'aboume to`uc t'onouc po`u j'eloume, gi''\NB
a>ut`o kal'utera >`ac bro~ume <'ena swst`o plh\-ktro\-l'ogio.}).  Genik'a,
gi`a n`a l'aboume <'ena toni\-sm'e\-no gr'amma, pr'epei n`a gr'ayoume m'ia
mikr`h >ako\-lou\-j'ia >el'egqou, <'ena s'umbolo >el'egqou, >empr`oc
>ap`o t`o gr'amma po`u j'eloume n`a ton'isoume.

<Or'iste merik`a parade'igmata:
$$\vbox{

\halign{
\strut \hfil\tt # & \quad\rm # \hfil\cr
\tengb K'wdikac {\bf \TeX} & \tengb >Apot'elesma \cr
\noalign{\hrule} \noalign{\smallskip}
\\`a la mode          & \`a la mode \cr
r\\'esum\\'e          & r\'esum\'e  \cr
soup\\c{\sp}con       & soup\c con  \cr
No\\"el               & No\"el      \cr
na\\"\\i{\sp}ve       & na\"\i ve   \cr
}
}$$

St`a parap'anw parade'igmata bl'epoume <'oti o<i periss'oteroi t'onoi   
lamb'anontai qrhsimopoi'wntac s'umbola >el'egqou par'omoiac morf~hc. 
Merik`a >ap`o t`a tonik`a shme~ia par'agontai m`e l'exeic >el'egqou o<i 
<opo~iec peri'eqoun <'ena m'ono gr'amma.  St`ic peript'wseic a>ut`ec
>apaite~itai l'igh prosoq'h; <'ena ken`o di'asthma pr'epei n`a
>akolouje~i met`a t`hn l'exh >el'egqou.  >E`an gr'a\-you\-me st`on
k'wdik'a mac {\tt soup\\ccon}, t`o {\rm \TeX} j`a prospaj'hsei n`a bre~i
t'i shma'inei <h l'exh >el'egqou {\tt\\ccon}\fnote{J`a do~ume pi`o k'atw
p`wc <up'arqei m'ia >ak'oma m'ejodoc gi`a t`hn >apofug`h t'etoiwn
sfalm'atwn.  E>~inai <h m'ejodoc t~hc dhmiourg'iac sun'olwn, t`ic
>arq`ec t~hc <opo'iac j`a t`ic suzh\-t'h\-sou\-me st`o kef'a\-laio\NB
4.}.\TeXref{52--53}

>Ax'izei >ep'ishc n`a pros'exoume <'oti, s`e <'ena >ap`o t`a parap'anw
parade'igmata, <up'arqei <h l'exh >el'eg\-qou\NB{\tt \\i}.  A>ut`h <h
mono\-gr'am\-math >entol`h m~ac d'inei t`o lati\-nik`o gr'amma (({\rm
i})) qwr`ic t`hn tele'ia; <h >afa'iresh t~hc tele'iac >ap`o t`o (({\rm
i})) >epitr'epei t`hn swst`h topo\-j'ethsh to~u t'onou >ep'anw >ap`o t`o
kat'wtero m'eroc {\rm \i} to~u gr'am\-matoc.  <Up'arqei m'ia >ak'oma
>an'alogh l'exh >el'egqou {\tt\\j} po`u m~ac d'i\-nei t`o gr'amma (({\rm
j})) qwr`ic tele'ia gi`a peript'wseic tonismo~u.

\maketable [Tonik`a shme~ia >akoloujo'umena >am'eswc >ap`o gr'amma]
\halign{
   \strut \hfil #\hfil & \quad \hfil \tt #\hfil & \hfil \quad\rm #\hfil\cr
   \tengb >'Onoma & \tengb K'wdikac {\bf \TeX} & \tengb >Apot'elesma \cr
   \noalign{\hrule} \noalign{\smallskip}
   >oxe'ia ({\rm acute})                         & \\'o    & \'o \cr
   bare'ia ({\rm grave})                         & \\`o    & \`o \cr
   gallik`h perispwm'enh ({\rm circumflex})      & \\\^{}o & \^o \cr
   dialutik'a ({\rm umlaut/dieresis/tr\'emat})   & \\"{}o  & \"o \cr
   <ellhnik`h perispwm'enh ({\rm tilde})         & \\\~{}o & \~o \cr
   makr'o ({\rm macron})                         & \\={}o  & \=o \cr
   tele'ia ({\rm dot})                           & \\\char'056{}o    & \.o \cr
       }

\toindex{`}
\toindex{'}
\toindex{\^{}}
\toindex{"}
\toindex{i}
\toindex{j}
\toindex{=}
\toindex{.}

\maketable [Tonik`a shme~ia po`u >apaito~un >endi'ameso ken`o di'asthma]
\halign{
   \strut \hfil\rm #\hfil & \quad \hfil\tt #\hfil & \hfil\rm\quad #\hfil\cr
   \tengb >'Onoma & \tengb K'wdikac {\bf \TeX} & \tengb >Apot'elesma \cr
   \noalign{\hrule} \noalign{\smallskip}
   cedilla                    & \\c o   & \c{o} \cr
   underdot                   & \\d o   & \d{o} \cr
   underbar                   & \\b o   & \b{o} \cr
   h\'a\v{c}ek                & \\v o   & \v{o} \cr
   breve                      & \\u o   & \u{o} \cr
   tie                        & \\t oo  & \t oo \cr
   {\tengr o>uggrik`o} umlaut & \\H o   & \H{o} \cr
      }

\toindex{c}
\toindex{d}
\toindex{b}
\toindex{v}
\toindex{u}
\toindex{t}
\toindex{H}

T`o {\rm \TeX} >ep'ishc >epitr'epei t`hn stoiqeiojes'ia merik~wn        
qarakt'hrwn po`u d`en sumpe\-ri\-lam\-b'anontai st`o >alf'abhto t~hc
>agglik~hc gl'wssac.

\maketable [Gr'ammata m`h >agglik~wn glwss~wn]
\halign{
   \strut \hfil\rm #\hfil & \quad\hfil\tt #\hfil & \hfil\quad\rm #\hfil\cr
   \tengb Par'adeigma & \tengb K'wdikac {\bf \TeX} & \tengb >Apot'elesma \cr
   \noalign{\hrule} \noalign{\smallskip}
   \AE gean, \ae sthetics     & \\AE, \\ae    & \AE, \ae \cr
   \OE uvres, hors d'\oe uvre & \\OE \\oe     & \OE, \oe \cr
   \AA ngstrom                & \\AA, \\aa    & \AA, \aa \cr
   \O re, K\o benhavn         & \\O, \\o      & \O, \o   \cr
   \L odz, \l\'odka           & \\L, \\l      & \L, \l   \cr
   Nu\ss                      & \\ss          & \ss      \cr
   ?`Si?                      & {?}{`}        & ?`       \cr
   !`Si!                      & {!}{`}        & !`       \cr
   Se\~nor                    & \\\~{}        & \~{}     \cr
   My {\it\$} of flesh        & \lb\\it\\\$\rb & {\it\$} \cr
      }

\toindex{AE}
\toindex{ae}
\toindex{OE}
\toindex{oe}
\toindex{AA}
\toindex{aa}
\toindex{O}
\toindex{o}
\toindex{L}
\toindex{l}
\toindex{ss}

\exercise Stoiqeiojet~hste: {\rm Does \AE schylus understand \OE dipus? }

\exercise Stoiqeiojet~hste: 
{\rm The smallest internal unit of \TeX{} is about 53.63 \AA. } 

\exercise Stoiqeiojet~hste: 
{\rm They took some honey and plenty of money wrapped up in a {\it \$}5 note.}

\exercise Stoiqeiojet~hste: 
{\rm \'El\`eves, refusez vos le\c cons! Jetez vos cha\^\i nes! }

\exercise Stoiqeiojet~hste: 
{\rm Za\v sto tako polako pijete \v caj? }

\exercise Stoiqeiojet~hste:
{\rm Mein Tee ist hei\ss. }

\exercise Stoiqeiojet~hste:
{\rm Peut-\^etre qu'il pr\'ef\`ere le caf\'e glac\'e. }

\exercise Stoiqeiojet~hste: 
{\rm ?`Por qu\'e no bebes vino blanco? !`Porque est\'a avinagrado! } 

\exercise Stoiqeiojet~hste: 
{\rm M\'\i\'\j n idee\"en worden niet be\"\i nvloed. }

\exercise Stoiqeiojet~hste: 
{\rm Can you take a ferry from \"Oland to \AA land? }

\exercise Stoiqeiojet~hste:
{\rm T\"urk\c ce konu\c san ye\u genler nasillar? }


\subsection{Tele~iec, pa~ulec, e>isagwgik'a\bdots}

<H daktulogr'afhsh >~htan p'anta <'enac sumbibasm'oc. <O mikr`oc   
>arijm`oc t~wn pl'h\-ktrwn t~hc grafo\-mhqan~hc >`h to~u termatiko~u 
(s`e s'ugkrish m`e t`on >arijm`o t~wn sumb'olwn po`u qrhsimopoio~un
di'aforoi suggrafe~ic), e>~iqe >anagk'asei t`on daktulogr'afo s`e fjhn`a
k'olpa ka`i l'useic t~hc stigm~hc, po`u poll`ec for`ec >~htan ka`i
>anti\-aisjh\-tik'ec.  <'Otan <'omwc stoiqeiojeto~ume m`e t`o {\rm \TeX}
d`en <up'arqoun peri\-orismo'i. S`e <eto'uth t`hn par'agrafo j`a do~ume
merik`ec diafor`ec metax`u t~hc daktu\-logr'a\-fhshc st`hn grafomhqan`h
ka`i t~hc stoiqeiojes'iac m`e t`o {\rm \TeX}.

P'osoi >ap`o ''m~ac gnwr'izoun <'oti <up'arqoun tess'arwn e>id~wn
pa~ulec? (<H grafo\-mhqa\-n`h >'eqei sun'hjwc m'ia ka`i m'ono.) 
<Up'arqei <h mikr`h pa~ula, t`o {\tengs <enwtik'o\/} ({\rm hyphen}) po`u
qrhsimopoie~itai gi`a t`on sullabism`o ka`i t`hn kop`h l'exewn st`o
t'eloc t~wn >ar'adwn, >`h gi`a t`hn <'enwsh d'uo >`h peris\-sot'erwn
l'exewn po`u prof'erontai <wc m'ia, p.q., ((kura-M'arw)),
((<Ai-Dhm'htrhc)), {\rm mother-in-law}, k.lp.\TeXref{3--5} <H {\tengs
<apl`h pa~u\-la\/} ({\rm en-dash}) qrhsimopoie~itai gi`a n`a dhl'wsei
<'ena e>~uroc >arijm~wn: tim~wn, sel'idwn, qr'onou, k.lp. <H {\tengs
meg'alh pa~ula\/} ({\rm em-dash}), t`hn <opo'ia t`a bibl'ia t~hc
<ellhnik~hc grammatik~hc t`hn >apokalo~un ((dipl`h pa~ula)), e>~inai
shme~io st'ixhc po`u mpa'inei st`hn j'esh t~hc par'enjeshc ka`i
spani'wtera st`hn j'esh t~hc >'anw tele'iac. T'eloc, m'ia >ak'oma pa~ula
e>~inai ka`i t`o ma\-jh\-matik`o s'umbolo {\tengs me~ion}.  St`on
<ep'omeno p'inaka fa'inetai p~wc m~ac d'inei t`o {\rm \TeX} <'olec
a>ut`ec t`ic pa~ulec ka`i t`o p~wc t`ic qrhsimopoio~ume s`e >agglik`a
ke'imena --- prosoq'h: o<i pa~ulec ka`i <'ola t`a shme~ia st'ixhc d`en
e>~inai t`a >idia >ap`o gl'wssa s`e gl'wssa; periss'oterec leptom'ereiec
gi''\NB a>ut`o t`o j'ema j`a po~ume st`o kef'alaio\NB 10.

\maketable [O<i di'aforec pa~ulec t~hc >agglik~hc tupograf'iac]
\halign{
   \strut \hfil #\hfil & \quad\hfil\tt #\hfil & \hfil\quad\rm #\hfil
                        & \hfil\rm #\hfil\cr
   \tengb >'Onoma & \tengb K'wdikac {\bf \TeX}
                            & \tengb >Apot'elesma & \tengb Par'adeigma \cr
   \noalign{\hrule} \noalign{\smallskip}
   <enwtik'o ({\rm hyphen})      & -         & -
                            & The space is 3-dimensional. \cr
   <apl`h pa~ula ({\rm en-dash}) & {-}{-}    & --
                            & Read pages 3--4. \cr
   dipl`h pa~ula ({\rm em-dash})  & {-}{-}{-} & ---
                            & I saw them---three men alive! \cr
   me~ion ({\rm minus sign})     & \$-\$     & $-$
                            & Two minus five equals $-$3. \cr
      }

\exercise Stoiqeiojet~hste: {\rm I entered the room and---horrors---I
saw both my father-in-law and my mother-in-law. }

\exercise Stoiqeiojet~hste: {\rm The winter of 1484--1485 was one of
discontent. }
\bigskip

M'ia >ak'oma meg'alh diafor`a metax`u t~hc koin~hc daktulogr'afhshc ka`i
t~hc stoiqeiojes'iac m`e t`o {\rm \TeX} e>~inai <h qr'hsh t~wn
e>isa\-gw\-gik~wn.  Sun'hjwc t`a e>isagwgik`a <'olwn t~wn
grafo\-mh\-qa\-n~wn --- e>'ite pr'okeitai gi`a <ellhnik`ec e>'ite
pr'okeitai gi`a >agglik'ec --- e>~inai <'ena pl~hktro m`e t`o
s'umbolo\NB{\tt "}\null.  <'Otan j'eloume n`a b'aloume e>isagwgik`a m`e
t`o {\rm \TeX} s`e <'ena >agglik`o ke'imeno, qrhsimopoio~ume t`a
pl~h\-ktra\NB{\tt '}  ka`i\NB{\tt `}\null.
\TeXref{3}
T`o pr~wto dipl`o e>isagwgik`o lamb'anetai gr'afontac st`on k'wdik'a mac
{\tt ``} ka`i t`o de'utero gr'afontac {\tt ''}\null. Kat`a par'omoio   
tr'opo t`a <apl`a e>isagwgik'a (p.q., e>isagwgik`a >ent`oc
e>isa\-gw\-gi\-k~wn) lamb'anontai gr'a\-fontac\NB{\tt `} ka`i\NB{\tt
'}\null.  (T`a <ellhnik`a e>isa\-gwgi\-k`a ((\NB ka`i\NB)) e>~inai m'ia
>'allh <istor'ia po`u j`a t`hn suzht'hsoume st`o kef'alaio\NB 10.)

\exercise Stoiqeiojet~hste: {\rm His ``thoughtfulness'' was impressive. }

\exercise Stoiqeiojet~hste: {\rm Frank wondered, ``Is this a girl that
can't say `No!'?'' }
\bigskip

K'apoiec for`ec qrhsimopoio~ume t`a >aposiwphtik'a, tre~ic tele~iec     
dhlad'h, gi`a n`a dhl'wsoume siwp`h >`h ke'imeno po`u   >'eqei
paralhfje~i. >E`an gr'ayoume st`on k'wdik'a mac tre~ic tele~iec, t`o
>apot'elesma j`a e>~inai tre~ic tele~iec kollhm'enec <h m'ia >ep'anw
st`hn >'allh.  T`a >aposiwphtik`a st`hn swst'h touc morf`h lamb'anontai
gr'afontac st`on k'wdika t`hn l'exh >el'egqou {\tt \\dots}.%
\TeXref{173}
\toindex{dots}

\exercise Stoiqeiojet~hste: {\rm He thought, ``\dots and this goes on
forever, perhaps to the last recorded syllable.'' }

<'Ena pr'oblhma m`e t`hn tele'ia e>~inai <'oti <'otan qrhsimopoie~itai  
gi`a n`a dhl'wsei t`o t'eloc m'iac pr'otashc, par'agetai <'ena sqetik`a 
meg'alo (per'ipou dipl'o) ken`o di'asthma >an'amesa st`o t'eloc t~hc
pr'otashc (t~hc tele'iac) ka`i st`hn >arq'h (t`o pr~wto kefala~io  
gr'amma) t~hc <ep'omenhc pr'otashc.  A>ut`o d`en e>~inai o>usia\-stik`a 
pr'o\-blhma; >antij'etwc t`o dipl`o di'asthma metax`u prot'a\-se\-wn     
e>~inai m'ia sun'hjhc praktik`h t~wn tupogr'afwn.  <'Omwc t`o dipl`o    
di'asthma met`a t`hn tele'ia e>~inai >anepij'umhto <'otan pr'o\-keitai  
gi`a suntomo\-graf'ia <'opwc ((k.)), ((kk.)), {\rm Mr.}, k.lp.
<Up'arqoun d'uo tr'opoi gi`a n`a >apof'ugoume a>ut`o t`o pr'oblhma:
gr'afoume >am'eswc met`a t`hn tele'ia st`on k'wdika e>'ite {\tt\\\sp}
e>'ite m'ia perispwm'enh {\tt\~{}} gi`a n`a >all'axoume t`o m'egejoc to~u
keno~u diast'hmatoc.%
\TeXref{91--92}
<H de'uterh >epilog`h j`a m~ac d'wsei <'ena {\tengs >adi\-'ako\-pto
di'asthma\/} >`h >alli~wc {\tengs s'un\-desmo\/}; dhl.\ >e`an b'aloume
m'ia peri\-spw\-m'enh metax`u d'uo l'exewn, t'ote a>ut`ec o<i l'exeic
j`a >emfa\-nisto~un st`o >'entupo st`hn >'idia >ar'ada.  Gr'afontac    
st`on k'wdika {\tt Prof.\~{}Knuth}, j`a l'aboume ka`i t`ic d'uo         
a>ut`ec l'exeic {\rm Prof.}\ ka`i {\rm Knuth} st`hn >'idia
>ar'ada. A>ut`o e>~inai genik`a qr'h\-simo gi`a t`hn stoiqei\-ojes'ia
>onom'atwn <'opwc {\rm Vancouver, B.\NB C.} ka`i {\rm Mr.\NB Jones}
>'etsi <'wste o<i l'exeic {\rm Mr.}\ ka`i {\rm Jones} n`a m`hn
bga'inoun s`e d'uo qwrist`ec >ar'adec.  >`Ac pro\-s'e\-xoume >ep'ishc
<'oti d`en qrei'azetai >antipl'agia >empr`oc >ap`o t`hn peri\-spw\-m'enh
to~u sund'esmou.  (Periss'otera gi`a t`hn perispwm'enh t~wn <ellhnik~wn,
>all`a ka`i gi`a >'alla <ellhnik`a tonik`a shme~ia, j`a po~ume st`o
kef'alaio\NB 10.)

M'ia >'allh >idiorujm'ia to~u {\rm \TeX} e>~inai n`a m`hn b'azei dipl`o
ken`o di'asthma met`a >ap`o tele'ia >e`an >empr`oc >ap`o a>ut`hn
<up'arqei kefala~io gr'amma.  >Aut`o e>~inai swst`o <'otan
qrhsimopoio~ume kefala~ia gi`a suntomograf'iec. P.q., d`en j'eloume t`o
di'asthma metax`u t~wn {\rm U}, {\rm S} ka`i {\rm A} n`a e>~inai
dipl'asio to~u kanoniko~u; >arke~i n`a sugkr'inoume t`o {\rm U\null.
S\null. A.} m`e t`o {\rm U. S. A.}, gi`a n`a katal'aboume t`hn diafor'a.
 T'i g'inetai <'omwc st`hn per'iptwsh po`u m'ia pr'otash tele'iwnei m`e
kefala~io ka'i, sunep~wc, pr'epei n`a mpe~i dipl`o di'asthma met`a t`hn
tele'ia?  <H l'ush e>~inai n`a gr'ayoume {\tt\\null} pr`in t`hn tele'ia
<'opwc: {\tt I was born in U. S. A\\null.\ I was raised in Canada.} <O
k'wdikac a>ut`oc d'inei: {\rm I was born in U. S. A\null. \ I was raised in
Canada.}
\toindex{null}

\exercise Stoiqeiojet~hste: {\rm Have you seen Ms.\NB Jones? }

\exercise Stoiqeiojet~hste: {\rm Prof.\NB Smith and Dr.\NB Gold flew from
Halifax N.\NB S. to Montr\'eal, Qc.\ via Moncton, N.\NB B. }


\subsection{T'upoi stoiqe'iwn}

<H pl'eon faner`h diafor`a metax`u daktulografhm'enwn >ent'upwn ka`i
>ent'upwn stoiqeio\-je\-thm'enwn m`e t`o {\rm \TeX} e>~inai --- qwr`ic
kam'ia >amfibol'ia --- o<i diaforetiko`i {\tengs t'upoi stoiqe'iwn\/}
gramm'atwn ka`i sumb'olwn >`h {\tengs grammatoseir`ec\/} po`u
qrhsimopoio~untai. <'Otan t`o {\rm \TeX} xekin~a, >'eqei sthn di'ajes'h
tou deka'exi diaforetik`ec grammatoseir'ec.  Merik`ec >ap`o a>ut`ec t`ic
gram\-ma\-to\-seir`ec qrhsimopoio~untai m'ono s`e >episthmonik`a >`h
teqnik`a >'entupa.  <'Enac pl'hrhc kat'alogoc <'olwn t~wn deka'exi
grammatoseir~wn to~u {\rm \TeX} d'inetai st`o {\sl \TeX{}book}.%
\TeXref{427--432}
O<i periss'oteroi t'upoi qrhsimopoio~untai a>ut'omata; <'enac
majhmatik`oc de'ikthc, p.q., bga'inei >ap`o t`o {\rm \TeX} s`e
mikr'otero m'egejoc qwr`ic n`a qreiasje~i n`a d'wsei k'apoia >idia'iterh
>entol`h <o qr'hsthc.

Gi`a n`a >all'axoume >ap`o t`on sunhjism'eno t'upo latinik~wn           
gramm'atwn, t`a {\rm roman} (rwma"ik'a), s`e pl'agia--kalligrafik'a, t`a 
{\rm italic} (>italik'a), qrhsimopoio~ume t`hn l'exh >el'egqou {\tt
\\it}. Gi`a n`a xana\-epi\-str'e\-youme st`a {\rm roman}, gr'afoume
st`on k'wdika {\tt \\rm}.  Gi`a par'adeigma, e>~inai dunat`o n`a
>'eqoume t`hn >ak'oloujh >agglik`h pr'otash st`on k'wdik'a mac: {\tt I
started with roman type, \\it switched to italic type, \\rm and returned
to roman type}.  T`o >apot'elesma j`a e>~inai: {\rm I started with roman
type, \it switched to italic type,\fnote{Notice that the comma and this
footnote are in italic type, and this looks a little funny. We'll see
that there is another method for changing fonts when we talk about
grouping in Section 4. ({\tengs >`Ac parathr'hsoume <'oti t`o k'omma ka`i
t`o >agglik`o ke'imeno <eto'uthc t~hc <upo\-sh\-me'i\-wshc >'eqoun
stoiqeiojethje~i s`e pl'agia--kalligrafik'a.  St`o kef'alaio\NB 4, j`a
do~ume p~wc e>~inai dunat`o n`a >al\-l'a\-zoume t'upouc stoiqe'iwn m`e
t`hn dhmiourg'ia sun'olwn}.)} \rm and returned to roman
type}.

>Ekt`oc t~wn {\rm italic} <up'arqoun ka`i >'alla e>'idh grammatoseir~wn.
St`on <ep'omeno p'inaka d'i\-no\-ntai o<i pl'eon suqn`a
qrhsimopoio'umenoi t'upoi.  A>uto`i o<i t'upoi diat'ijentai a>ut'omata
>ap`o t`o {\rm \TeX} st`on qr'h\-sth.  L'igo pi`o k'atw j`a do~ume
>ak'oma p~wc mporo~ume n`a qrh\-simo\-poi'h\-soume ka`i >'allec
grammatoseir`ec po`u d`en m~ac t`ic prosf'erei a>ut'omata t`o {\rm
\TeX}. 

\global\advance\footnotenum by 2
\maketable [De'igmata t'upwn (grammatoseir~wn)]
\halign{
   \strut \hfil\rm # \hfil & \quad \hfil \tt # \hfil
                                 & \hfil \quad \rm # \hfil\cr
   \tengb >'Onoma t'upou & \tengb K'wdikac {\bf \TeX}
                                 & \tengb De~igma t'upou \cr
   \noalign{\hrule} \noalign{\smallskip}
   Roman ({\tengr rwma"ik'a})              & \\rm
                                 & \rm Roman type. \cr
   Boldface ({\tengr >'entona})            & \\bf
                                 & \bf Boldface type. \cr
   Italic ({\tengr >italik'a}) & \\it
                                 & \it Italic type. \cr
   Slanted ({\tengr pl'agia})              & \\sl
                                 & \sl Slanted type. \cr
   Typewriter ({\tengr grafomhqan~hc})     & \\tt
                                 & \tt Typewriter type. \cr
   Math symbol ({\tengr s'umbola majhmatik~wn})${}^\the\footnotenum$
                                 & \\cal
                                 & $\cal SCRIPT\hbox{\ }LETTERS$\rm. \cr
      }

\footnote{}{\llap{$^\the\footnotenum$} A>ut`o t`o par'adeigma >apat~a,
mi~ac ka`i pr'epei n`a gnwr'izoume merik`a pr'agmata peris\-s'o\-te\-ra
gi`a t`hn stoiqeiojes'ia majhmatik~wn <'wste n`a mporo~ume n`a
stoiqeiojeto~ume s`e t'etoiou e>'idouc stoiqe~ia (kef'alaio\NB 5). }
\nobreak
\toindex{it}
\toindex{rm}
\toindex{bf}
\toindex{sl}
\toindex{tt}

T`a {\rm slanted} (pl'agia) ka`i t`a {\rm italic} (>italik'a) fa'inontai
<'omoia >ek pr'wthc >'oyewc.  E>~inai e>'ukolo <'omwc n`a do~ume t`hn
diafor`a >e`an pros'exoume t`o gr'amma (({\rm a})) st`on <'ena ka`i
st`on >'allo t'upo. <'Otan >all'azoume >ap`o <'enan pl'agio t'upo ({\rm
slanted} >`h {\rm italic}) s`e >'enan >'orjio ({\rm upright} >`h {\rm
roman}), t`o teleuta~io gr'amma t~wn plag'iwn g'ernei ka`i fa'inetai n`a
plhsi'azei t`o pr~wto gr'amma t~wn >orj'iwn.  Gi`a n`a >apof'ugoume
t'etoiou e>'idouc a>isjhtik`a probl'hmata, prosj'etoume l'igo parap'anw
di'asthma m`e t`hn leg'omenh {\tengs di'orjwsh >italik~wn\/} ({\rm
italic correction}). A>ut`o g'inetai m`e t`o s'umbolo >el'egqou {\tt
\\/}.%
\toindex{/}
St`hn parak'atw >agglik`h pr'otash d`en b'alame di'orjwsh >italik~wn
st`hn pr'wth per'iptwsh <'opou e>'iqame {\rm italic}, >en~w b'alame     
st`hn de'uterh --- >`ac pros'exoume t`hn diafor'a: {\rm {\it If} the
italic correction is not used the letters are too close together, but
{\it if\/} the correction is used, the spacing is better.}  D`en       
qrei'azetai di'orjwsh >italik~wn st`hn per'iptwsh <'opou >am'e\-swc
met`a t`a pl'agia >akolouje~i k'omma >`h tele'ia.  <Wst'oso, t`o      
>'entupo j`a de'iqnei pi`o <wra~io, >e`an qrhsimopoio~ume t`hn di'orjwsh
>italik~wn pr`in >ap`o e>isagwgik`a >`h parenj'eseic.

E>~inai dunat`o n`a qrhsimopoi'hsoume ka`i >'allec grammatoseir`ec
>ekt`oc >ap`o a>ut`ec po`u m~ac prosf'erei >arqik`a t`o {\rm \TeX}
(>ef'' <'oson t`a >arqe~ia m`e t`ic >epipl'eon grammatoseir`ec
<up'arqoun st`on <upologist'h mac). Di'afora meg'ejh t'upwn mporo~un n`a
qrhsimopoihjo~un m`e t`hn bo'hjeia t~hc l'exhc >el'egqou {\tt
\\magstep}.  Gi`a n`a <or'isoume m'ia n'ea grammatoseir`a >`h <'enan
n'eo t'upo stoiqe'iwn, j`a pr'epei n`a gnwr'izoume p~wc >onom'azetai t`o
sqetik`o >arqe~io st`on <upologist'h mac. Gi`a par'adeigma, <o t'upoc
{\rm roman} >onom'azetai {\tt cmr10} st`ic pi`o poll`ec >ekd'oseic to~u
{\rm \TeX}\null.  >E`an gr'ayoume st`on k'wdik'a mac {\tt \\font\\bigrm
= cmr10 scaled \\magstep\NB{}1}, t'o\-te mporo~ume n`a
qrh\-simo\-poi'h\-soume t`hn l'exh >el'egqou {\tt\\bigrm} <'opwc
qrhsimopoio~ume ka`i t`ic l'exeic >el'egqou {\tt\\it} >`h {\tt\\rm}.%
\TeXref{13--17}
\toindex{magstep}
\toindex{font}
\toindex{scaled}

>All'azontac t'upo m'esw| t~hc >entol~hc {\tt\\bigrm}, j`a l'aboume
stoiqe~ia {\rm roman} po`u j`a e>~inai kat`a 20\NB \% megal'utera t~wn
kanonik~wn.  M`e t`hn >entol`h {\tt \\font\\bigbigrm = cmr10 scaled
\\magstep\NB{}2} <or'izoume <'enan n'eo t'upo {\rm roman} <o <opo~ioc   
e>~inai kat`a 44\NB \% megal'uteroc to~u ka\-no\-ni\-ko~u.  Sunolik`a
t`o {\rm \TeX} >epitr'epei <'exi megej'unseic t'upwn, >ap`o {\tt
\\magstep\NB 0} >'ewc {\tt \\magstep\NB 5}. S`e poll`ec >ekd'oseic to~u
{\rm \TeX}, <up'arqei ka`i m'ia <'ebdomh meg'ejunsh {\tt \\magstephalf}
--- pr'okeitai gi`a meg'enjush kat`a 9,5\NB \% per'ipou. <Or'iste
merik`a de'igmata <en`oc t'upou s`e di'afora meg'ejh:

\maketable [Megej'unseic t'upwn]
\halign{
   \strut \tt # \hfil & \quad # \hfil \cr
   \tengb Meg'ejunsh & \tengb De~igma \cr
   \noalign{\hrule} \noalign{\smallskip}
   \\magstep 0     & \rm  Sample text at magstep 0. \cr
   \\magstephalf   & \halfrm  Sample text at magstephalf. \cr
   \\magstep 1     & \brm  Sample text at magstep 1. \cr
   \\magstep 2     & \bbrm  Sample text at magstep 2. \cr
   \\magstep 3     & \bbbrm  Sample text at magstep 3. \cr
   \\magstep 4     & \bbbbrm  Sample text at magstep 4. \cr
   \\magstep 5     & \bbbbbrm  Sample text at magstep 5. \cr
      }
\bigskip

E>~inai >ep'ishc dunat`o n`a qrhsimopoi'hsoume tele'iwc n'eec           
grammatoseir'ec (p.q., <ellhnik'ec). Beba'iwc, p'anta a>ut`o >exart~atai
>ap`o t`o t'i <up'arqei diaj'esimo st`on <upologist`h po`u
qrhsi\-mo\-poi\-o~ume ka`i >'iswc >ap`o t`ic dunat'othtec t~hc ts'ephc
mac (poll`ec grammatoseir`ec kuklo\-fo\-ro~un <wc >emporik`a
pro"i'onta).  S`e pollo`uc <upologist`ec <up'arqei <'ena >arqe~io m`e
t`o <'onoma {\tt cmss10}; pr'okeitai gi`a m'ia grammatoseir`a stoiqe'iwn
{\rm sans serif} (qwr`ic >apol'hxeic >'h, >alli~wc, qwr`ic pato~urec).
B'azontac t`hn >entol`h {\tt \\font\\sf = cmss10} s`e k'apoio shme~io
to~u k'wdik'a mac, mporo~ume kat'opin n`a qrh\-simo\-poi'h\-soume t`on
t'upo {\rm sans serif} m`e t`hn l'exh >el'egqou {\tt\\sf}, <'opwc j`a
gr'afoume, p.q., {\tt\\bf} gi`a >'entona stoiqe~ia.  >'Eqo\-ntac loip`on
<or'isei t`on t'upo {\rm sans serif}, <o k'wdikac: {\tt\\sf a sample of
our new Sans Serif font}, d'inei: {\sf a sample of our new Sans Serif
font.}

\exercise T'i pr'oblhma j`a dhmiourgo~use <h qr'hsh m'iac l'exhc
>el'egqou {\tt \\ss} >ant`i t~hc {\tt \\sf} gi`a n`a stoiqeiojet'hsoume
s`e {\rm sans serif}? <Up'odeixh: >E`an <h >ap'anthsh d`en s~ac
fa'inetai ka`i t'oso <apl'h, skefje~ite t`o germanik`o >alf'abhto pr~wta
ka`i t'ote j`a t`hn bre~ite.                 

\exercise Stoiqeiojet~hste m'ia par'agrafo s`e megejunm'eno t'upo {\rm
sans serif}.

O<i >epipl'eon grammatoseir`ec ka`i t'upoi po`u mpore~i n`a
qrhsimopoi'hsei kane`ic m`e t`o {\rm \TeX} dia\-f'eroun >ap`o
<upologist`h s`e <upologist'h.  O<i t'upoi po`u >anaf'erontai st`on
parak'atw p'inaka e>~inai a>uto`i po`u <up'arqoun sto`uc periss'oterouc
<upologist'ec. 

\maketable [O<i pi`o koino`i t'upoi to~u {\lsecfont\TeX}]
\halign{
\strut \tt # \hfil & \quad \tt # \hfil & \quad \tt # \hfil
                   & \quad \tt # \hfil & \quad \tt # \hfil
                   & \quad \tt # \hfil  \cr
\noalign{\hrule} \noalign{\smallskip}
cmbsy10   &cmbxsl10  &cmbxti10  &cmbx10    &cmbx12    &cmbx5      \cr
cmbx6     &cmbx7     &cmbx8     &cmbx9     &cmb10     &cmcsc10    \cr
cmdunh10  &cmex10    &cmff10    &cmfib8    &cmfi10    &cmitt10    \cr
cmmib10   &cmmi10    &cmmi12    &cmmi5     &cmmi6     &cmmi7      \cr
cmmi8     &cmmi9     &cmr10     &cmr12     &cmr17     &cmr5       \cr
cmr6      &cmr7      &cmr8      &cmr9      &cmsltt10  &cmsl10     \cr
cmsl12    &cmsl8     &cmsl9     &cmssbx10  &cmssdc10  &cmssi10    \cr
cmssi12   &cmssi17   &cmssi8    &cmssi9    &cmssqi8   &cmssq8     \cr
cmss10    &cmss12    &cmss17    &cmss8     &cmss9     &cmsy10     \cr
cmsy5     &cmsy6     &cmsy7     &cmsy8     &cmsy9     &cmtcsc10   \cr
cmtex10   &cmtex8    &cmtex9    &cmti10    &cmti12    &cmti7      \cr
cmti8     &cmti9     &cmtt10    &cmtt12    &cmtt8     &cmtt9      \cr
cmu10     &cmvtt10                                                \cr
}
\bigskip

O<i t'upoi a>uto`i >'eqoun >onomasje~i (({\rm Computer Modern})) >ap`o
t`on sqediast'h touc, po`u d`en e>~inai >'alloc >ap`o t`on >'idio t`on
dhmiourg`o to~u {\rm \TeX}, t`on {\rm Donald Knuth}.  >'Etsi t`a d'uo
pr~wta gr'ammata to~u >on'omatoc a>ut~wn t~wn t'upwn {\tt cm} shma'inoun
{\rm Computer Modern}.  <O >arijm`oc st`o t'eloc t~wn >onom'atwn touc
dhl'wnei t`o m'egej'oc touc: t'upoc 10 stigm~wn ({\rm points}) e>~inai
kanoniko~u meg'ejouc, t'upoi 7 stigm~wn qrh\-si\-me'uoun sun'hjwc <wc
majhmatiko`i de~iktec ka`i t'upoi 5 stigm~wn <wc >ekj'etec; t'upoi 12
stigm~wn e>~inai kat`a 20\NB \% megal'uteroi a>ut~wn t~wn 10 stigm~wn,
k.o.k. >E`an st`a gr'ammata d'uo pr~wta gr'am\-mata to~u >on'omatoc,
{\tt cm}, akolouje~i t`o gr'amma {\tt b}, <o t'upoc e>~inai {\bf bold}
(>'entona stoiqe~ia).  Par'omoia, t`o {\tt r} shma'inei {\rm roman}, t`o
{\tt I} {\it italic}, t`o {\tt csc} {\sc Small Caps}
(mikro\-kefa\-la~ia, t`a gnwst`a sto`uc <'ellhnec tupogr'afouc <wc
((kapital'akia)){}), t`o {\tt sl} {\sl slanted}, t`o {\tt sy} {$\cal
SYMBOL$} (s'umbola) ka`i t`o {\tt tt} {\tt typewriter} (t'upoc
grafomhqan~hc).

\exercise Bre~ite <'olouc to`uc t'upouc po`u diaj'etei t`o s'usthm'a sac
ka`i tup~wste <'ola t`a gr'ammata ka`i to`uc >arijmo`uc s`e meriko`uc
>ap`o a>uto`uc to`uc t'upouc.

\exercise T`a stoiqe~ia to~u t'upou {\tt cmr12} e>~inai kat`a 20\NB{}\%
megal'utera s`e m'egejoc >ap`o a>ut`a to~u t'upou {\tt cmr10}. 
>Ep'ishc, <h >entol`h {\tt \\magstep\NB 1} megej'unei t`a stoiqe~ia
kat`a 20\NB \%. Tup~wste <'ena ke'imeno qrhsimopoi'wntac t`on t'upo {\tt
cmr12} ka`i kat'opin tup~wste t`o >'idio ke'imeno m`e t`on t'upo {\tt
cmr10} megejunm'eno kat`a 20\NB{}\%.  T`a >apotel'esmata j`a e>~inai
diaforetik'a!


\section{<H di'ataxh t~wn pragm'atwn}

St`o kef'alaio a>ut'o, j`a >exet'asoume p~wc n`a stoiqeiojeto~ume <'ena
ke'imeno s`e di'a\-forec diat'axeic (morf'ec) ka`i meg'ejh.
<Opwc\-d'hpote, mporo~ume n`a qrhsimopoi'hsoume t`o {\rm \TeX} m`e t`ic
dik'ec tou (t`ic {\tengs >ex <orismo~u\/} >`h {\rm default}) diat'axeic
ka`i meg'ejh, <'opwc >'eqoume >'eqoume k'anei m'eqri a>ut`o t`o shme~io.
<'Omwc >ap`o >ed~w ka`i st`o >ex~hc, j`a e>'imaste k'apwc pi`o
dhmiourgiko'i.  Gi`a n`a katal'aboume <'omwc t'i shma'inoun di'afora
meg'ejh, kal`a j`a >~htan n`a xekino'usame di\-eu\-kri\-n'izontac
<orism'enec >'ennoiec mon'adwn.

\subsection{Mon'adec, mon'adec, mon'adec}

T`o {\rm \TeX} mpore~i n`a metr'hsei m'hkh s`e poll`ec diaforetik`ec
mon'adec.  O<i pi`o koin`ec e>~inai <h >'intsa, t`o <ekatost'o, <h
tupografik`h stigm'h ({\rm point}) ka`i t`o tupo\-gra\-fik`o
te\-tr'a\-gwno >`h p'ika ({\rm pica}). O<i sunto\-mo\-graf'iec a>ut~wn
t~wn mon'adwn e>~inai >ant'istoiqa: {\tt in}, {\tt cm}, {\tt pt} ka`i
{\tt pc}.  <H stigm`h <or'izetai >ap`o t`hn >ex'iswsh: $1\,\rm in =
72{,}27\,\rm pt$, ka`i t`o te\-tr'a\-gwno >ap`o t`hn >ex'iswsh: $1\,\rm
pc = 12\,\rm pt$\null.  Sunep~wc, <h stigm`h e>~inai m'ia pol`u mikr`h
mon'ada m'etrhshc m'hkouc --- per'ipou t`o m'egejoc m'iac tele'iac
a>uto~u to~u >egqei\-rid'iou.   <'Ena te\-tr'a\-gwno e>~inai per'ipou <h
>ap'ostash >ap`o t`hn b'ash m'iac >ar'adac <eto'utou to~u keim'enou
>'ewc t`hn b'ash t~hc <ep'omenhc >ar'adac.  <Or'iste m'ia sug\-kritik`h
e>ik'ona merik~wn diast'asewn m'hkouc: \def\pip{\vrule height 4 true
pt}%
\TeXref{57}

\maketable [\null]
\halign{
\strut \hfil # & \quad # \hfil \cr
\noalign{\smallskip}
        1 >'intsa:    &  $\hbox to 1 true in{\pip\hrulefill\pip}$ \cr
        1 <ekatost'o: &  $\hbox to 1 true cm{\pip\hrulefill\pip}$ \cr
        20 stigm'ec:  &  $\hbox to 20 true pt{\pip\hrulefill\pip}$ \cr
        1 tetr'agwno: &  $\hbox to 1 true pc{\pip\hrulefill\pip}$ \cr
}
\bigskip

O<i stigm`ec qrhsimopoio~untai gi`a >allag`ec pol`u mikr~wn
diast'asewn. M'alista t`o {\rm \TeX} e>~inai pol`u >akrib`ec st`hn
m'etrhsh diast'asewn; <h mikr'oterh mon'ada m'etrhshc m'hkouc po`u
qrhsimopoie~i t`o {\rm \TeX} e>~inai mikr'oterh >ap`o <'ena
<ekatommuriost`o t~hc >'intsac. >'Etsi, <h le\-pto\-m'e\-reia to~u
>ent'u\-pou >exar\-t~a\-tai o>u\-sia\-sti\-k`a >ap`o t`hn e>ukr'ineia
({\rm resolution}) to~u >ektupwt~h.

<Up'arqoun d'uo >ak'oma mon'adec o<i <opo~iec d`en e>~inai tele'iwc
stajer'ec, >all`a metab'alontai >an'aloga m`e t`o m'egejoc t~wn t'upwn
po`u qrhsimopoio~ume.\TeXref{60}  <H mon'ada {\tt ex} e>~inai per'ipou
<'oso t`o <'uyoc <en`oc mikro~u latiniko~u {\rm x}, ka`i <h mon'ada {\tt
em}  e>~inai per'ipou >'ish m`e t`o pl'atoc <en`oc kefala'iou {\rm M}
(gi`a t`hn >akr'ibeia, <h mon'ada {\tt em} e>~inai dipl'asia to~u
pl'atouc <en`oc <opoioud'hpote >ap`o t`a yhf'ia $0,\ldots,9$).

T`hn di'ataxh (diast'aseic, k.lp.)\ to~u >ent'upou t`hn >el'egqoume
>ep'ishc m`e l'exeic >el'egqou.  T`o {\rm \TeX} m~ac d'inei poll`ec
t'etoiec l'exeic >el'egqou po`u >epitr'epoun t`on pol`u >akrib`h
>'elegqo to~u >apotel'esmatoc. <'Omwc, st`hn pleioyof'ia t~wn
peript'wsewn merik`ec m'ono >ap`o >aut`ec >arko~un.

\subsection{<H di'ataxh t~hc sel'idac}

T`o ke'imeno m'iac sel'idac >apotele~itai >ap`o tr'ia basik`a m'erh. T`o
megal'utero m'eroc t~hc sel'idac t`o katalamb'anei t`o {\tengs s~wma\/}
({\rm body}): t`o {\tengs k'urio ke'imeno\/} m`e t`ic  {\tengs
<uposhmei'wseic}. >Ep'anw >ap`o t`o s~wma, <up'arqei <h {\tengs
kefal'h\/} ({\rm headline}). <H kefal`h sun'hjwc peri'eqei t`on t'itlo
to~u >ent'upou, t`on t'itlo to~u kefala'iou >`h ka`i t`on >arijm`o t~hc
sel'idac; >ep'ishc <h kefal`h mpore~i n`a diaf'erei >ap`o mon`h s`e
zug`h sel'ida.  K'atw >ap`o t`ic  <uposhmei'wseic <up'arqei t`o {\tengs
p'odi} ({\rm footline}) t~hc sel'idac, m'ia >ar'ada po`u sun'hjwc
peri'eqei t`on >arijm`o t~hc sel'idac ka`i m'onon.

St`a parade'igmata po`u e>'idame >'ewc >ed~w, <h kefal`h >~htan ken'h.  
T`o p'odi t~hc sel'idac perie~iqe e>'ite t`on >arijm`o t~hc sel'idac
st`o m'eso tou, e>'ite >~htan tele'iwc ken`o <'otan qrhsimopoi'hsame
t`hn >entol`h {\tt \\nopagenumbers}.  Periss'otera gi`a t`hn kefal`h
ka`i t`o p'odi t~wn sel'idwn j`a >anaf'eroume pi`o k'atw.  Pr`oc
stigm'hn, >`ac >epikentr'wsoume t`hn prosoq'h mac st`o s~wma.

Gi`a n`a kle'isoume m'ia sel'ida ka`i n`a xekin'hsoume m'ia n'ea,
mporo~ume n`a gr'ayoume st`on k'wdik'a mac: {\tt \\vfill \\eject}.  <H
l'exh >el'egqou {\tt \\eject} >anagk'azei t`o {\rm \TeX} n`a
<oloklh\-r'w\-sei t`hn paro'usa sel'ida po`u >epexerg'azetai, >en~w <h
l'exh >el'egqou {\tt \\vfill} to~u l'eei n`a gem'i\-sei t`o k'atw m'eroc
to~u s'wmatoc m`e ken'o. (>E`an j'elete, m`hn gr'afete st`on
k'wdik'a sac {\tt \\vfill} gi`a n`a de~ite p`wc j`a tentwje~i t`o
ke'imeno gi`a n`a gem'isei <h sel'ida.)
\toindex{vfill}
\toindex{eject}

{\hsize=4in
T`o pl'atoc to~u keim'enou <or'izetai m`e t`hn l'exh >el'egqou {\tt     
\\hsize}.  Mpore~i n`a >allaqje~i s`e <opoiod'hpote shme~io to~u
keim'enou mac, p.q., s`e t'esseric >'intsec, m`e t`hn >entol'h: {\tt
\\hsize = 4\NB in} ka`i m`e mej'odouc po`u j`a perigr'ayoume pi`o k'atw.
<H tim`h to~u {\tt \\hsize} st`o t'e\-loc m'iac pa\-ra\-gr'a\-fou
kajor'izei ka`i t`o pl'atoc t~hc para\-gr'a\-fou.  <'Opwc mporo~ume n`a
parathr'hsoume s`e <eto'uth t`hn par'agrafo, t`o pl'atoc to~u
kei\-m'e\-nou mpore~i n`a >allaqje~i gi`a m'ia ka`i monadik`h par'agrafo
(>ed~w >'egine 4\NB >'intsec).  >Ep'ishc, mi~ac ka`i <h l'exh >el'egqou
{\tt \\hsize} e>~inai m'ia metablht'h (t`o pl'atoc to~u keim'enou), m`e
>ekfr'aseic <'opwc {\tt \\hsize =  0,75\\hsize} mporo~ume n`a t`hn
>all'axoume s`e sq'esh m`e t`hn pali`a tim'h thc (<'osoi >'eqoun k'anei
p'ente stoiqei'wdh maj'hmata programmatismo~u j`a katal'aboun kal'utera
t`hn shmas'ia a>uto~u to~u k'wdika).
\par }
\toindex{hsize}

<H >ant'istoiqh l'exh >el'egqou gi`a t`o <'uyoc to~u keim'enou e>~inai
<h {\tt \\vsize}. <H tim`h to~u {\tt \\vsize} >all'azetai <'opwc ka`i <h
tim`h to~u {\tt \\hsize}.  >'Etsi gr'afontac {\tt \\vsize = 8 in}
<or'izoume t`o <'uyoc to~u s'wmatoc s`e >okt`w >'intsec. >`Ac
pros'exoume <'oti <h tim`h {\tt \\vsize} e>~inai t`o <'uyoc to~u
s'wmatoc m'onon qwr`ic n`a sumperi\-lam\-b'ano\-ntai <h kefal`h ka`i t`o
p'odi.
\toindex{vsize}

<H <'olh sel'ida mpore~i n`a metakinhje~i >ep'anw st`o qart`i <'opou
t`hn tup'wnoume.  <H >ep'anw >arister`h gwn'ia t~hc sel'idac, dhl.\ <h
>ep'anw >arister`h gwn'ia t~hc kefal~hc thc, e>~inai s`e >ap'o\-sta\-sh
m'iac >'intsac >ap`o t`hn >ep'anw >'akrh to~u qartio~u kai s`e
>ap'o\-sta\-sh m'iac >'in\-tsac >ap`o t`hn >arister`h >'akrh to~u
f'ullou to~u qartio~u. O<i l'exeic >el'egqou {\tt\\hoffset} ka`i {\tt
\\voffset} qrh\-simo\-poio~u\-ntai gi`a n`a >all'axoun >aut`ec o<i
>apost'aseic.  P.q., m`e {\tt\\hoffset = ,75 in} ka`i {\tt \\voffset =
-,5 in} <h sel'ida j`a metakinhje~i kat`a 0,75 >'intsec pr`oc t`a dexi`a
to~u qartio~u ka`i kat`a 0,5 >'intsec pr`oc t`a >ep'anw. T`ic
periss'oterec for`ec d`en j`a qreiasje~i n`a <or'isoume t`ic tim`ec t~wn
{\tt \\hoffset}, {\tt \\voffset} ka`i {\tt \\vsize} par`a m'ono m'ia
for`a st`hn >arq`h to~u k'wdik'a mac.   
\TeXref{251}
\toindex{hoffset}
\toindex{voffset}

\global\advance\footnotenum by 1
\maketable [L'exeic >el'egqou gi`a t`ic diast'aseic t~hc sel'idac]
\halign{
   \strut \hfil #\hfil & \quad\hfil\tt #\hfil & \hfil\quad #\hfil\cr
   \tengb >'Onoma & \tengb K'wdikac {\bf \TeX} & \tengb >Arqik`h tim`h
                            to~u {\bf \TeX\ (in)} \cr
   \noalign{\hrule} \noalign{\smallskip}
   <oriz'ontio pl'atoc  & \\hsize   & 6,5 \cr
   katak'orufo <'uyoc   & \\vsize   & 8,9 \cr
   <oriz'ontia metat'opish${}^\the\footnotenum$  & \\hoffset & 0 \cr
   katak'orufh metat'opish${}^\the\footnotenum$   & \\voffset & 0 \cr
      }

\footnote{}{\llap{$^\the\footnotenum$}>Ex <orismo~u, t`o {\rm \TeX}
xekin~a t`hn stoiqeiojes'ia t~hc sel'idac s`e >ap'ostash $1\,\rm in$
>ap`o t`hn koruf`h to~u qartio~u, ka`i s`e >ap'ostash $1\,\rm in$ >ap`o
t`hn >arister`h >'akrh to~u qartio~u.} \nobreak

\exercise <Etoim'aste m'ia par'agrafo keim'enou po`u n`a >'eqei merik`ec
>ar'adec. >Anti\-gr'ay\-te a>ut`h t`hn par'agrafo merik`ec for`ec ka`i
b'alte pr`in >ap`o t`hn pr'wth {\tt \\hsize = 5\NB in} ka`i {\tt \\hsize
= 10\NB cm} >empr`oc >ap`o t`hn de'uterh.  Dokim'aste merik`ec >ak'oma
tim`ec st`o {\tt \\hsize}.
 
\exercise B'alte {\tt \\hoffset = ,5 in\ } ka`i {\tt \\voffset = 1      
in} >empr`oc >ap`o t`hn pr'wth par'agrafo t~hc prohgo'umenhc >'askhshc.

\exercise  B'alte {\tt \\vsize = 2 in} >empr`oc >ap`o t`hn pr'wth
par'agrafo t~hc prohgo'umenhc >'askhshc.

St`hn prohgo'umenh par'agrafo e>'idame <'oti mporo~ume n`a 
qrhsimopoi'hsoume diaforetik`a meg'ejh t'upwn qrhsimopoi'wntac t`hn
l'exh >el'egqou {\tt \\magstep}.  E>~inai >ep'ishc dunat`h <h
meg'e\-jun\-sh <'olou to~u >ent'upou mac monomi~ac.  >E`an j'esoume {\tt
\\magnification = \\magstep\NB 1} st`hn >arq`h to~u k'wdika, t`o
>apot'elesma j`a e>~inai t`o >'entup'o mac n`a  megejunje~i <'olo kat`a
20\NB \%\null. <H meg'ejunsh mpore~i n`a g'inei ka`i m`e >'allec tim`ec
{\tt \\magstep}. <Wst'oso {\tengb <h l'exh >el'egqou {\tt
\\magnification} mpore~i n`a qrhsimopoihje~i m'onon st`hn >arq`h to~u
k'wdika, pr`in t`o {\bf \TeX} sunant'hsei >'estw ka`i <'enan qarakt'hra
gi`a stoiqeioj'ethsh.}  A>ut`h <h meg'ejunsh mpore~i n`a dhmiourg'hsei
k'apoia probl'hmata. >E`an <'olo t`o ke'imeno e>~inai
n`a megejunje~i kat`a 20\NB \% ka`i >'eqoume <or'isei {\tt \\hsize = 5
in} st`on k'wdik'a mac, t`o telik`o >'entupo j`a >'eqei pl'atoc 5
>'intsec, >`h j`a megejunje~i kat`a 20\NB{}\% s`e 6 >'intsec?  >E`an
d`en l'aboume t`a kat'al\-lhla m'etra, m`e t`hn >entol`h {\tt
\\magnification} <'olec o<i diast'aseic j`a megejunjo~un kat`a
20\NB{}\%, dhl.\ t`o {\tt \\hsize} j`a g'inei 6 >'intsec.  S`e merik`ec
peript'wseic a>ut`o mpore~i n`a e>~inai >epijumht'o, <'omwc sun'hjwc <h
<omoi'omorfh >allag`h <'olwn t~wn diast'asewn e>~inai >anepij'umhth. 
Gi`a par'adeigma, mpore~i n`a j'eloume n`a >af'hsoume 3 >'intsec
>akrib~wc keno~u q'wrou gi`a n`a >enj'esoume kat'opin <'ena sq~hma. 
St`hn per'iptwsh a>ut'h, k'aje di'astash po`u d`en j'eloume n`a
>all'axei pr'epei n`a t`hn <or'isoume <wc {\tt \\true}, p.q., gr'afontac
{\tt \\hsize = 5\NB{}true\NB{}in} t`o m~hkoc t~wn >ar'adwn (dhl.\ t`o
pl'atoc to~u keim'enou) j`a parame'inei 5 >'intsec >anex'arthta >ap`o
t`hn tim`h to~u {\tt \\magnification}.\TeXref{59--60}
\toindex{magnification}

\exercise B'alte {\tt \\magnification = \\magstep 1} st`hn pr'wth
gram\-m`h t~wn >arqe'iwn sac ka`i de~ite t`o >apot'elesma.

\subsection{<H di'ataxh t~hc paragr'afou}

<'Otan t`o {\rm \TeX} diab'azei t`on k'wdik'a mac, diab'azei m'ia
par'agrafo k'aje for`a ka`i met`a t`hn stoiqeiojete~i.  A>ut`o shma'inei
<'oti mporo~ume n`a >'eqoume pl'hrh >'elegqo t~hc di'ataxhc (t~hc
morf~hc) t~hc paragr'afou; <'omwc <up'arqoun merik`ec leptom'ereiec o<i
<opo~iec >apaito~un >idia'iterh prosoq'h. E>'idame >'hdh p~wc m`e t`o
{\tt \\hsize} mporo~ume n`a >el'egqoume t`o pl'atoc t~hc paragr'afou. 
>All'a, >`ac <upoj'esoume <'oti st`on k'wdika e>'iqame gr'ayei:
\beginuser
\\hsize = 5 in
Four score and seven years
$\vdots$
from this earth.
\\hsize = 6,5 in
\enduser

Poi'o e>~inai t`o pl'atoc t~hc paragr'afou?  <H tim`h to~u {\tt \\hsize}
<or'isjhke m'ia for`a st`hn >arq`h t~hc paragr'afou, ka`i kat'opin
>epanor'isjhke st`o t'eloc thc.  >Ef'' <'oson <h par'agrafoc d`en e>~iqe
<oloklhrwje~i m'eqri t`on de'utero <orism`o t~hc tim~hc to~u {\tt
\\hsize} (m`e t`hn parembol`h m'iac ken~hc gramm~hc >`h to~u {\tt \\par}
st`on k'wdika), j`a stoiqeiojethje~i s'umfwna m`e t`hn teleuta'ia tim`h
pl'atouc, dhl.\ m`e pl'atoc 6,5 >'intsec.  <'Opwc, >e`an <up~hrqe m'ia
ken`h gramm`h st`on k'wdika, t`o pl'atoc thc j`a >~htan 5 >'intsec. 
Bl'epoume loip'on, <'oti k'aje par'agrafoc stoiqeiojete~itai m`e t`ic
tim`ec t~wn  param'etrwn po`u e>~inai o<i pi`o pr'osfatec po`u >'eqei
diab'asei t`o   {\rm \TeX}.

<Or'iste <'enac p'inakac m`e merik`ec param'etrouc po`u kajor'izoun t`hn
di'ataxh (morf'h) m'iac para\-gr'a\-fou:

\maketable [Par'ametroi di'ataxhc paragr'afou]
\halign{
   \hfil #\hfil & \quad\hfil\tt #\hfil & \hfil\quad #\hfil\cr
   \tengb Leitourg'ia & \tengb K'wdikac {\bf \TeX} & \tengb >Arqik`h
		tim`h to~u {\bf \TeX} \cr
   \noalign{\hrule} \noalign{\smallskip}
   pl'atoc & \\hsize & 6,5 >'intsec \cr
   >od'ontwsh pr'wthc >ar'adac
                         & \\parindent & 20 stigm'ec\cr
   >ap'ostash metax`u >ar'adwn (di'astiqo) & \\baselineskip & 12 stigm'ec\cr
   >ap'ostash metax`u paragr'afwn & \\parskip & 0 stigm'ec\cr
      }
\par
\toindex{parindent}
\toindex{parskip}
\toindex{baselineskip}

<H l'exh >el'egqou {\tt \\noindent} mpore~i n`a qrhsimopoihje~i st`hn
>arq`h m'iac paragr'afou gi`a t`hn >apofug`h t~hc {\tengs >od'ontwshc}
(t~hc >eswterik~hc metat'opishc, >`h st`hn >Agglik`h {\rm indentation})
t~hc pr'wthc >ar'adac.  A>ut`h <h >entol`h j`a >ephre'asei m'ono t`hn
morf`h t~hc paragr'afou st`hn <opo'ia qrhsimopoie~itai.  >Antij'etwc,
j'etontac {\tt \\parindent = 0\NB{}pt} <'olec o<i par'agrafoi to~u
>ent'upou d`en j`a >'eqoun >od'ontwsh. \toindex{noindent}

{\narrower
<'Enac pi`o e>'ukoloc tr'opoc gi`a n`a >el'egxoume t`o pl'atoc m'iac
paragr'afou e>~inai n`a qrhsimopoi'hsoume t`ic l'exeic >el'egqou {\tt
\\rightskip} ka`i {\tt \\leftskip}.  >'Etsi, j'etontac {\tt \\leftskip =
20 pt}, t`o >arister`o perij'wrio t~hc paragr'afou megal'wnei kat`a
e>'ikosi stigm'ec.  >E`an j'eloume t`o >arister`o perij'wrio n`a 
mikr'unei ka`i <h par'agrafoc n`a >epe\-ktaje~i >ekt`oc to~u >aristero~u
<'or'iou t~hc sel'idac, d`en >'eqoume par`a n`a <or'isoume m'ia
>arnhtik`h tim`h gi`a t`hn di'astash {\tt \\leftskip}.  Paromo'iwc,
>all'azontac t`hn tim`h to~u {\tt \\rightskip}, kajor'izoume t`o dexi`o
<'orio t~hc paragr'afou.  <H l'exh >el'egqou {\tt \\narrower} d'inei t`o
>'idio >apot'elesma <'opwc >e`an e>'iqame <or'isei t`ic tim`ec t~wn {\tt
\\leftskip} ka`i {\tt \\rightskip} >'isec m`e t`hn tim`h to~u {\tt
\\parindent}.  A>ut`o e>~inai suqn`a qr'hsimo gi`a t`hn par'ajesh  
makr~wn dane'iwn qwr'iwn --- <eto'uth <h par'agrafoc e>~inai <'ena
par'adeigma.  <'Opwc sumba'inei ka`i m`e t`o {\tt \\hsize}, <'otan
<h par'agrafoc <oloklhr'wnetai t`o {\rm\TeX} lamb'anei <up'' >'oyh tou
t`ic pi`o pr'osfatec tim`ec t~wn {\tt\\leftskip} ka`i {\tt\\rightskip}
gi`a n`a t`hn stoiqeiojet'hsei. \TeXref{100} \par
}
\toindex{leftskip}
\toindex{rightskip}
\toindex{narrower}

\exercise Stoiqeiojet~hste d'uo paragr'afouc m`e t`ic >ex~hc
diast'aseic: t`o >arister`o perij'wrio ka`i t~wn d'uo paragr'afwn
e>~inai megal'utero kat`a 1,5 >'intsec >ap`o t`o perij'wrio to~u <'olou
>ent'upou; t`o dexi`o perij'wrio t~hc pr'wthc paragr'afou e>~inai kat`a 
0,75 t~hc >'intsac megal'utero >ap`o a>ut`o to~u >ent'upou; ka`i t`o
dexi`o perij'wrio t~hc de'uterhc paragr'afou e>~inai megal'utero s`e
s'ugkrish m`e a>ut`o to~u >ent'upou kat`a 1,75 >'intsec.
\bigskip

\def\hangparagraph{%
O<i >ar'adec (>`h >alli~wc {\tengs st'iqoi\/}) m'iac paragr'afou
mporo~un n`a stoiqeiojethjo~un s`e diaforetik`a m'hkh qrhsimopoi'wntac  
t`ic l'e\-xeic >el'eg\-qou {\tt \\hangindent} ka`i {\tt \\hangafter}. 
T`o m'egejoc ka`i <h j'esh t~hc >o\-d'o\-ntw\-shc t~wn >ar'adwn
>exart~wntai >ap`o t`hn tim`h to~u {\tt \\hangindent}. >E`an t`o {\tt
\\hangindent} e>~inai jetik'o, <h >od'o\-ntw\-sh g'inetai >a\-p`o t`a
>arister'a; >e`an e>~inai >ar\-nh\-ti\-k'o,  <h >od'ontwsh g'inetai
>a\-p`o t`a dexi'a.  T`o poi'ec >ar'adec metatop'izontai kat`a t`hn
>od'ontwsh >exart~atai >ap`o t`hn tim`h to~u {\tt \\hangafter}.  >E`an
t`o {\tt \\hangafter} e>~inai jetik'o, t'ote <h tim`h tou kajor'izei t`o
p'osec >ar'adec >ap`o t`hn >arq`h t~hc paragr'afou j`a >'eqoun pl~hrec
pl'atoc (<'opwc a>ut`o kajor'izetai >ap`o t`hn tim`h to~u {\tt
\\hsize}).  >'Etsi, >e`an >'eqoume {\tt \\hangindent = 1,75\NB in} ka`i
{\tt \\hangafter =\NB{}6}, t'ote o<i <'exi pr~wtec >ar'adec j`a >'eqoun
pl~hrec pl'atoc, >en`w gi`a t`ic <up'oloipec t`o >arister`o
pe\-ri\-j'wrio j`a e>~inai kat`a 1,75 >'intsec megal'utero. >Ap`o t`hn
>'al\-lh me\-ri'a, >e`an >'eqoume {\tt \\hangindent =
\hbox{-1,75}\NB{}in} ka`i {\tt \\hangafter = -6}, t'ote o<i <'exi
pr~wtec >ar'adec j`a >'eqoun >arister`o perij'wrio megal'utero kat`a
1,75 >'intsec >ap`o t`ic <up'oloipec po`u j`a >'eqoun pl~hrec pl'atoc. 
T`o {\rm \TeX} >epano\-r'izei {\tt \\hangindent = 0\NB{}pt} ka`i {\tt
\\hangafter = 1} met`a t`o t'eloc k'a\-je para\-gr'a\-fou.  A>ut`ec o<i
l'exeic >el'egqou e>~inai >idia'itera qr'hsimec gi`a paragr'afouc m`e
>od'ontwsh, kaj`wc ka`i gi`a paragr'afouc o<i <opo~iec mpa'inoun g'urw
>ap`o sq'hmata >`h e>ik'onec.
\TeXref{355}%
J'etontac t`hn l'exh >el'egqou {\tt \\hang} st`hn >ar\-q`h t~hc
paragr'afou, <h pr'w\-th >ar'ada j`a bge~i st`o pl~hrec pl'atoc thc,
>en~w o<i <up'oloipec j`a >'eqoun >arister`o perij'wrio megal'utero
kat`a t`hn tim`h to~u {\tt \\parindent}.\TeXref{102} <W\-st'o\-so, j`a
pr'epei n`a qrhsi\-mopoi'h\-sou\-me ka`i t`hn >e\-nto\-l`h {\tt \\noindent}
>e`an j'eloume <h pr'wth >ar'ada n`a m`hn >'eqei >od'ontwsh.
\par}

\hangafter=6 \hangindent=1.75in
\hangparagraph
\toindex{hangindent}
\toindex{hangafter}
\toindex{hang}

<Or'iste xan`a <h parap'anw par'agrafoc m`e {\tt \\hangafter = -6} ka`i
{\tt \\hangindent = -1,75 in}.

\hangafter=-6 \hangindent=-1.75in \hangparagraph

M`e t`hn l'exh >el'egqou {\tt \\parshape} mporo~ume n`a
<etoim'asoume paragr'afouc s`e megal'uterh poikil'ia diat'axewn.
\TeXref{101}
\toindex{parshape}

M'ia >ak'oma l'exh >el'egqou pol`u qr'hsimh gi`a t`hn stoiqeiojes'ia    
paragr'afwn e>~inai ka`i <h {\tt \\item}. Mpore~i n`a qrhsimopoihje~i   
gi`a t`hn <etoimas'ia katal'ogwn <'opou kajet`i katagr'afetai          
qwrist'a. Gr'afontac st`on k'wdik'a mac {\tt \\item\lb$\ldots$\rb},     
dhmiourgo~ume m'ia par'agrafo m`e >arister`h >od'ontwsh (megal'utero
>arister`o perij'wrio) <'osh ka`i <h tim`h to~u {\tt \\parindent}; 
>epipl'eon <h pr'wth >ar'ada t~hc paragr'afou shmei'wnetai m`e <'o,ti   
>'eqoume gr'ayei >ent`oc t~wn >agkul~wn.  Sun'hjwc <h l'exh >el'egqou   
{\tt \\item} qrhsimopoie~itai m`e {\tt \\parskip = 0 pt}, >epeid`h
a>ut`h <h teleuta'ia l'exh >el'egqou kajor'izei t`hn katak'orufh        
>ap'ostash metax`u diaforetik~wn >antikeim'enwn >en`oc kata\-l'o\-gou. <H   
l'exh >el'egqou {\tt \\itemitem} m~ac d'inei t`o >'idio >apot'elesma
<'opwc ka`i <h {\tt \\item}, m`e t`hn diafor`a <oti <h >od'ontwsh
e>~inai dipl'asia, dhl.\ d'uo for`ec megal'uterh >ap`o t`hn tim`h to~u 
{\tt\\parindent}.\TeXref{102} <Or'iste <'ena par'adeigma:
\toindex{item}
\toindex{itemitem}

\beginliteral
\parskip = 0pt \parindent = 30 pt
\noindent
Answer all the following questions:
\item{(1)} What is question 1?
\item{(2)} What is question 2?
\item{(3)} What is question 3?
\itemitem{(3a)} What is question 3a?
\itemitem{(3b)} What is question 3b?
@endliteral

\noindent
T`o >apot'elesma to~u parade'igmatoc e>~inai:
\vskip\baselineskip

\vbox{{\parskip = 0pt \parindent = 30 pt \rm
\noindent
Answer all the following questions:
\item{(1)} What is question 1?
\item{(2)} What is question 2?
\item{(3)} What is question 3?
\itemitem{(3a)} What is question 3a?
\itemitem{(3b)} What is question 3b?
}}

\exercise <Etoim'aste m'ia par'agrafo >arket~wn >ar'adwn ka`i           
qrhsimopoi~hste t`hn l'exh >el'egqou {\tt \\item} gi`a n`a de~ite t'i
shma'inei >od'ontwsh. Kat'opin stoiqeiojet~hste t`hn >'idia
par'a\-gra\-fo m`e diaforetik`ec tim`ec {\tt \\parindent} ka`i {\tt
\\hsize}.
\bigskip

Ka`i t'wra >`ac do~ume p~wc <or'izoume t`o ken`o di'asthma metax`u t~wn 
paragr'afwn.  <H l'exh >el'egqou {\tt \\parskip} qrhsi\-mo\-poie~itai 
gi`a t`on kajorism`o to~u keno~u diast'hmatoc metax`u t~wn paragr'afwn.
>'Etsi loip'on, >e`an gr'ayoume {\tt \\parskip = 12\NB pt} st`hn >arq`h
to~u >arqe'iou {\rm\TeX}, <h >ap'ostash metax`u t~hc b'ashc m'iac
paragr'afou ka`i t~hc koruf~hc t~hc <ep'omenhc j`a e>~inai 12 stigm`ec
s`e <'olo t`o >'entupo (>ekt`oc >e`an >epanor'isoume s`e k'apoio
<ep'omeno shme~io to~u k'wdika t`hn tim`h to~u {\tt \\parskip}). 
>Ep'ishc m`e t`hn l'exh >el'egqou {\tt \\vskip} mporo~ume n`a j'esoume
pr'osjeto ken`o di'asthma metax`u d'uo paragr'afwn.  P.q., m`e {\tt
\\vskip 1\NB{}in} prosj'etoume ken`o di'asthma m'iac >'intsac metax'u
d'uo paragr'afwn, >en~w me {\tt \\vskip 20\NB{}pt} prosj'etoume ken`o
di'asthma e>'ikosi stigm~wn.

<Up'arqoun kanadu`o paraxeni`ec m`e t`o {\tt \\vskip}.  >E`an j'esoume
{\tt \\vskip 3 in} ka`i t`o >epi\-pl'e\-on ken`o di'asthma xekin~a  d'uo
>'intsec >ap`o t`o t'eloc t~hc sel'idac, t'ote {\tengs d`en\/} j`a
prosteje~i m'ia >'intsa keno~u diast'hmatoc st`hn >arq`h t~hc <ep'omenhc
sel'idac.  M`e >'alla l'ogia, {\tengb <h l'exh >el'eg\-qou} {\tt
\\vskip} {\tengb d`en prosj'etei katak'akorufo ken`o di'asthma metax`u
d'uo suneq'omenwn se\-l'i\-dwn}.  M'alista, <h >entol`h {\tt
\\vskip\NB{}1\NB{}in} d`en j`a >'eqei kan'ena >apot'elesma >e`an
>emfanisje~i st`hn koruf`h m'iac sel'idac!  St`ic periss'oterec
peript'wseic, a>ut`o e>~inai pol`u swst'o. Gi`a par'adeigma, pr`in >ap`o
k'aje t'itlo paragr'afou, sun'hjwc j'etoume l'igo parap'anw katak'orufo
ken`o di'asthma, <'omwc a>ut`o t`o di'asthma d`en pr'epei n`a <up'arqei
>e`an <o t'itloc t~hc paragr'afou xekin~a >ap`o t`hn koruf`h t~hc
sel'idac.

<'Ena par'omoio fain'omeno sumba'inei ka`i st`hn >arq`h to~u keim'enou.
>E`an j'eloume, p.q., <o t'itloc to~u >ent'upou mac n`a br'isketai st`o
m'eso t~hc sel'idac, d`en mporo~ume n`a prosj'esoume katak'orufo         
ken`o di'asthma qrhsimopoi'wntac {\tt \\vskip}.

<'Omwc p~wc j`a kataf'ername k'ati t'etoio?  J`a mporo'usame n`a
xekin'hsoume t`hn sel'ida m`e <'ena {\tt\\\sp} >all`a a>ut`o j`a m~ac
dhmiourg'hsei m'ia ken`h par'agrafo.  Ka`i >'etsi, par''\NB{}<'oti
a>ut`h <h par'agrafoc d`en peri'eqei t'ipota, j`a mpe~i >epipl'eon ken`o
di'asthma >ex ait'iac t~wn {\tt \\baselineskip} ka`i {\tt \\parskip}.
M'ia e>ukol'wterh l'ush e>~inai n`a qrhsimopoi'hsoume t`hn l'exh
>el'egqou {\tt \\vglue} >ant`i t~hc {\tt \\vskip}.  >'Etsi j'etontac    
{\tt \\vglue 1\NB{}in} j`a l'aboume katak'orufo ken`o di'asthma m'iac
>'intsac st`hn koruf`h t~hc sel'idac.%
\TeXref{352}
\toindex{vglue}

Ka`i mi~ac ka`i t`o >'efere <h koub'enta --- <up'arqei ka`i m'ia >ak'oma
pi`o genik`h m'ejodoc gi`a n`a prosj'esoume k'ati (ken`o di'asthma,
ke'imeno, k.lp.)\ st`hn koruf`h t~hc sel'idac m`e t`ic l'exeic >el'egqou
{\tt \\topinsert} ka`i {\tt \\endinsert}.  >E`an gr'a\-you\-me {\tt
\\topinsert $\ldots$ \\endinsert} s`e m'ia sel'ida to~u k'wdika, t`o
<ulik`o metax`u t~wn {\tt \\topinsert} ka`i {\tt \\endinsert} j`a
>emfanisje~i st`o >ep'anw m'eroc t~hc sel'idac --- >e`an beba'iwc t`o
{\rm\TeX} bre~i <'oti <up'arqei q~wroc gi`a k'ati t'etoio. T`o
<ep'omeno par'adeigma e>~inai pol`u qr'hsimo gi`a t`hn >'enjesh
sqhm'atwn:%
\TeXref{115}
\toindex{topinsert}
\toindex{endinsert}

\vbox{
   \beginuser
   \\topinsert
   \\vskip 1 in
   \\centerline\lb Figure 1\rb
   \\endinsert
   \enduser
}

T'eloc, <up'arqoun ka`i merik`ec e>idik`ec l'exeic >el'egqou gi`a mikr`a
katak'orufa ken`a dia\-st'h\-mata.  A>ut`ec e>~inai: {\tt \\smallskip},
{\tt \\medskip} ka`i {\tt \\bigskip}. <Or'iste t`o m'egejoc kajen`oc
>ap`o a>ut`a t`a ken'a:
\smallskip
\def\hrl{\hrule width .5 in}
\centerline{{\tt \\smallskip}: \vbox{\hrl \smallskip \hrl} \quad
            {\tt \\medskip}: \vbox{\hrl \medskip \hrl} \quad
            {\tt \\bigskip}: \vbox{\hrl \bigskip \hrl}
            }
\toindex{smallskip}
\toindex{medskip}
\toindex{bigskip}

\subsection{<H di'ataxh t~hc >ar'adac}

St`a periss'otera ke'imena, t`o {\rm\TeX} t`a kataf'ernei >arket`a
kal`a m`e t`hn kop`h t~wn >ar'adwn s`e paragr'afouc.  <'Omwc,
merik`ec for`ec e>~inai >anagka~io n`a to~u d'wsoume k'apoiec para\-p'anw
<odhg'iec.  >'Etsi e>~inai dunat`o n`a prokal'esoume t`hn >ap'otomh
kop`h m'iac >ar'adac ka`i t`hn sun'eqish to~u keim'enou st`hn <ep'omenh
>ar'ada, j'etontac st`on k'wdika: {\tt \\hfill \\break}.  E>~inai
>ep'ishc dunat`o n`a dhmiourg'hsoume m'ia >ar'ada po`u n`a >ekte'inetai
>ap`o t`o <'ena >'akro t~hc sel'idac >'ewc t`o >'allo m`e t`hn l'exh
>el'egqou {\tt \\line\lb $\ldots$\rb}; >'etsi t`o ke'imeno po`u >'eqei
grafe~i >ent`oc t~wn >agkul~wn j`a tentwje~i <'wste n`a katal'abei
m'ia >ar'ada (par''\NB{}<'oti t`o >apot'elesma mpore~i n`a e>~inai
a>isjhtik`a >apar'adekto).  O<i l'exeic >el'egqou  {\tt \\leftline\lb
$\ldots$\rb}, {\tt \\rightline\lb $\ldots$\rb} ka`i {\tt \\centerline\lb
$\ldots$\rb} qrhsimopoio~untai gi`a n`a dhmiourghje~i m`e t`o ke'imeno
>ent`oc >agkul~wn m'ia  >ar'ada po`u j`a e>~inai >ant'istoiqa kollhm'enh
st`o >arister`o >'akro t~hc sel'idac (>arister`h sto'iqish), kollhm'enh
st`o dexi`o >'akro t~hc sel'idac (dexi`a sto'iqish), >`h
topo\-jeth\-m'enh st`o k'entro t~hc sel'idac (k'entrwsh). 
Kat''\NB{}a>ut`on t`on tr`opo, <o <ep'omenoc k'wdikac {\rm \TeX}
\toindex{hfill}
\toindex{break}
\toindex{centerline}
\toindex{leftline}
\toindex{rightline}
\toindex{line}

\beginliteral
\leftline{I'm over on the left.}
\centerline{I'm in the @centre.}
\rightline{I'm on the right.}
\line{I just seem to be spread out all over the place.}
@endliteral

\noindent
dhmiourge~i t`ic >ex~hc t'esseric >ar'adec:
\vskip\baselineskip

%%%%%% supress underfulls reported
\hbadness = 10000 {\rm
\leftline{I'm over on the left.}
\centerline{I'm in the \center.}
\rightline{I'm on the right.}
\line{I just seem to be spread out all over the place.}
}

>'Allou e>'idouc ken`a diast'hmata >ent`oc >ar'adwn mporo~ume n`a
l'aboume m`e t`hn l'exh >el'egqou {\tt \\hfil}.  A>ut`h <h l'exh
>el'egqou prokale~i t`hn sugk'etrwsh <'olou to~u keno~u diast'hmatoc    
m'iac >ar'adac st`o shme~io <'opou br'isketai. >E`an >all'axoume t`on
k'wdika to~u teleuta'iou parade'igmatoc s'e: {\tt \\line\lb I just seem
to be spread out \\hfil all over the place.\rb}, t`o >apot'e\-le\-sma
j`a e>~inai: \hfil\break
\medskip
{\rm
\line{I just seem to be spread out \hfil all over the place.}
}
\medskip

>E`an gr'ayoume periss'oterec >ap`o m'ia for`a t`hn l'exh >el'egqou {\tt
\\hfil}, t`o ken`o di'asthma t~hc >ar'adac moi\-r'a\-ze\-tai s`e >'isa
m`erh metax`u t~wn {\tt \\hfil}.  >'Etsi, <o parak'atw k'wdikac {\rm
\TeX} {\tt \\line\lb left text \\hfil \centre{} text\\hfil right
text.\rb} d'inei
\medskip
{\rm
\line{left text \hfil \centre{} text \hfil right text.}
}
\toindex{hfil}

\exercise Stoiqeiojet~hste t`hn <ep'omenh >ar'ada:\hfil\break
{\rm
\line{left end \hfil left tackle \hfil left guard \hfil \centre{} \hfil
right guard \hfil right tackle \hfil right end}
}

\exercise Stoiqeiojet~hste t`hn <ep'omenh >ar'ada >'etsi <'wste t`o
ken`o di'asthma metax`u t~wn l'exewn (({\rm left})) ka`i (({\rm
right-\centre{}})) n`a e>~inai dipl'asio >ap`o a>ut`o metax`u t~wn      
l'exewn (({\rm right-\centre{}})) ka`i (({\rm right})):\hfil\break
{\rm
\line{left \hfil \hfil right-\centre{} \hfil right}
}

>Ak'oma, e>~inai dunat`o n`a prosj'esoume <oriz'ontio ken`o di'asthma
>ent`oc m'iac >ar'adac qrhsimopoi'wntac t`hn l'exh >el'egqou {\tt
\\hskip} kat`a tr'opo >an'alogo m`e t`hn per'iptwsh to~u {\tt \\vskip}. 
\toindex{hskip}

\exercise T'i sumba'inei m`e t`on parak'atw k'wdika?\hfil\break {\tt
\\line\lb\\hskip 1 in ONE \\hfil TWO \\hfil THREE\rb}
\bigskip

\hbadness = 1000
%%%%%%% restore original \hbadness

>E`an j'eloume k'apoiec sel'idec >`h ka`i <'olo t`o >'entup'o   
mac n`a m`hn e>~inai stoiqism'eno (>alli~wc, {\tengs n`a m`hn >'eqei
perasi'a\/}) >ap`o t`hn dexi`a pleur'a, d`en >'eqoume par`a n`a
qrhsimopoi'hsoume t`hn l'exh >el'egqou {\tt \\raggedright}.
\toindex{raggedright}

T'eloc, m'ia >ak'oma pol`u qr'hsimh l'exh >el'egqou gi`a t`hn
stoiqeiojes'ia poihm'atwn >`h k'wdika programm'atwn e>~inai <h
{\tt\\obeylines}. <H {\tt\\obeylines} >isoduname~i m`e t`hn <'uparxh
m'iac l'exhc >el'egqou {\tt\\par} st`o t'eloc k'aje gramm~hc; >'etsi t`o
{\rm\TeX} >anagk'azetai n`a k'obei k'aje >ar'ada >akrib~wc st`o shme~io
<'opou k'obetai <h >ant'istoiqh gramm`h to~u k'wdik'a mac.  J`a pr'epei
<'omwc n`a e>'imasje prosektiko`i ka`i n`a perikle'ioume t`o po'ihm'a
mac m`e >agk'ulec, giat`i diaforetik`a <'oloc <o k'wdik'ac mac j`a m~ac
stoiqeiojethje~i <wc <'ena po'ihma. >Ep'ishc, j`a pr'epei n`a <or'izoume
swst`a t`o m'egejoc {\tt\\parskip} <'wste n`a m`hn e>~inai <uperbolik`a
meg'alo t`o di'astiqo. <Or'iste <'ena sqetik`o par'adeigma:\TeXref{92}
\toindex{obeylines}

\beginliteral@obeyspaces
{\obeylines\parskip=0pt\narrower
Nous n'avons pas les yeux \`a l'\'epreuve des belles,
\qquad\qquad Ni les mains \`a celle de l'or :
\qquad\qquad Peu de gens gardent un tr\'esor
\qquad\qquad Avec des soins assez fid\`eles.
\medskip
\rightline{Jean de La Fontaine, {\it Fables}, Livre VIII, vii} 
}
@endliteral

\noindent
<O parap'anw k'wdikac m~ac d'inei:
\vskip\baselineskip

{\rm\obeylines\parskip=0pt\narrower
Nous n'avons pas les yeux \`a l'\'epreuve des belles,
\qquad\qquad Ni les mains \`a celle de l'or :
\qquad\qquad Peu de gens gardent un tr\'esor
\qquad\qquad Avec des soins assez fid\`eles.
\medskip
\rightline{Jean de La Fontaine, {\it Fables}, Livre VIII, vii} 
}

\subsection{<Uposhmei'wseic}

<O genik`oc tr'opoc gi`a n`a dhmiourg'hsoume m'ia <uposhme'iwsh m`e t`o
{\rm \TeX} e>~inai n`a gr'ayoume st`on k'wdika: {\tt
\\footnote\lb$\ldots$\rb\lb$\ldots$\rb}\null.

T`o s'umbolo t~hc <uposhme'iwshc mpa'inei m'esa st`ic d'uo pr~wtec
>agk'ulec: merik`a >ap`o t`a pl'eon kat'allhla s'umbola po`u diaj'etei
t`o {\rm \TeX} gi`a <uposhmei'wseic e>~inai t'a: {\tt \\dag (\dag)},
{\tt \\ddag (\ddag)}, {\tt \\S (\S)} ka`i {\tt \\P (\P)}\null.  T`o
ke'imeno t~hc <uposhme'iwshc mpa'inei >an'amesa st`ic d'uo de'uterec    
>agk'ulec.  <H stoiqeiojes'ia >arijmhm'enwn <uposhmei'wsewn e>~inai
l'igaki pi`o pol'uplokh. Gi`a par'adeigma, <h <uposhme'iwsh%            
\footnote{${}^{21}$}{\rm This is the footnote at the bottom of the
page.}
st`o k'atw m'eroc <eto'uthc t~hc sel'idac dhmiourg'hjhke gr'afontac     
>am'eswc met`a t`hn l'exh ((<uposhme'iwsh)) t`on k'wdika: {\tt
\\footnote\lb\$\lb \rb\^{}\lb{21}\rb\$\rb\lb\\rm This is the footnote at
the bottom of the page.\rb} A>ut`oc <o k'wdikac >'iswc m~ac fa'inetai
k'apwc pol'uplokoc, >all`a j`a t`on katal'aboume kal'utera m'olic
>anaferjo~ume st`hn stoiqeiojes'ia majhmatik~wn keim'enwn. Pr`oc
stigm'hn, >`ac t`on deqjo~ume <wc >'eqei, mi~ac ka`i k'anei t`hn doulei'a
tou. >Ak'oma, >e`an j'eloume n`a e>'imaste s'igouroi <'oti <h
<uposhme'iwsh j`a stoiqeiojethje~i st`on t'upo po`u >eme~ic j'eloume,
mporo~ume n`a qrhsimopoi'hsoume m'ia >entol`h >allag~hc t'upou, p.q.,
{\tt \\rm} gi`a latinik`a stoiqe~ia t'upou {\rm roman}, st`hn >arq`h
to~u keim'enou t~hc <uposhme'iwshc. T'eloc, kal`o j`a >~htan n`a m`hn
>af'hnoume ken`o di'asthma metax`u t~hc l'exhc >el'egqou {\tt
\\footnote} ka`i t~hc prohgo'umenhc l'exhc to~u k'uriou keim'enou ---
diaforetik'a, t`o shme~io t~hc <uposhme'iwshc j`a e>~inai
((xekr'emasto)).%
\TeXref{117}
\toindex{footnote}
\toindex{dag}
\toindex{ddag}
\toindex{S}
\toindex{P}

\exercise <Etoim'aste m'ia sel'ida m`e m'ia meg'alh <uposhme'iwsh
>arket~wn >ar'adwn.

\exercise <Etoim'aste m'ia sel'ida m`e d'uo diaforetik`ec
<uposhmei'wseic.


\subsection{<H kefal`h ka`i t`o p'odi t~hc sel'idac}

O<i >ar'adec t~wn t'itlwn ka`i t~wn >arijm~wn t~wn sel'idwn po`u
mpa'inoun st`o >ep'anw >`h t`o k'atw m'eroc  to~u s'wmatoc (dhl.\ st`hn
kefal`h >`h t`o p'odi), dhmiourgo~untai <or'izontac st`on k'wdika: {\tt
\\headline=\lb$\dots$\rb} ka`i {\tt \\footline=\lb$\dots$\rb}\
>ant'istoiqa.%
\TeXref{252--253}
           
<H >arq`h t~hc stoiqeiojes'iac kefal~hc >`h podio~u st`hn sel'ida mac
e>~inai <h >'idia m''\NB a>ut`hn t~hc stoiqeiojes'iac >ar'adwn m`e t`hn
l'exh >el'egqou {\tt \\line\lb $\dots$\rb}\null. M'ia pol`u qr'hsimh
l'exh >el'egqou gi`a t`hn per'iptwsh <eto'uth e>~inai ka`i <h {\tt
\\pageno}, <h <opo'ia e>~inai <'enac metrht`hc po`u katagr'afei t`on
>arijm`o t~hc sel'idac.  >'Etsi <o k'wdikac {\tt \\headline=\lb\\hfil
\\rm Page \\the\\pageno\rb} j`a m~ac d'wsei m'ia kefal`h <h <opo'ia
j`a peri'eqei t`hn l'exh (({\rm Page})) ka`i t`on >arijm`o t~hc sel'idac
st`o dex'i thc >'akro (t'wra koit'axte st`hn >ep'anw dexi`a gwn'ia t~hc
sel'idac).  
%
\headline={\hfil \rm Page \the\pageno} %example for this page
%
E>~inai pi`o >asfal`ec n`a <or'isoume >akri\-b~wc s`e poi'on t'upo
j'eloume n`a stoiqeiojethje~i <h kefal`h >`h t`o p'odi t~hc sel'idac,
giat`i >alli~wc mpore~i n`a brejo~ume pr`o >ekpl'hxewn! (St`hn
per'iptwsh to~u parade'igmatoc, gr'ayame {\tt \\rm} gi`a n`a bge~i <h
kefal`h s`e latinik`a stoiqe~ia t'upou {\rm roman}.) <H l'exh >el'egqou
{\tt \\the} pa'irnei t`hn tim`h <en`oc metrht~h, <'opwc to~u {\tt
\\pageno}, ka`i t`hn tup'wnei <wc kanonik`o ke'imeno. Mporo~ume >ep'ishc
n`a qrhsimopoi'hsoume t`hn l'exh >el'egqou {\tt \\folio}, >ant`i to~u
k'wdika {\tt \\the \\pageno}. <H diafor`a t~wn d'uo tr'opwn gi`a t`hn
>ekt'upwsh to~u >arijmo~u t~hc sel'idac e>~inai <'oti t`o {\tt \\folio}
j`a d'wsei latiniko`uc >arijmo`uc (p.q., {\rm iii}, {\rm xiv}, k.lp.)\
<'otan <h tim`h to~u {\tt \\pageno} e>~inai >arnhtik'h.
\toindex{headline}
\toindex{footline}
\toindex{pageno}
\toindex{the}
\toindex{folio}

>E`an j'eloume o<i sel'idec sac n`a mhn >'eqoun t`hn sunhjism'enh
>ar'ijmhsh, t'ote mporo~ume n`a <or'isoume t`hn tim`h to~u {\tt
\\pageno}. T`o {\rm \TeX} j`a tup'wsei to`uc >arijmo`uc t~wn sel'idwn   
<wc latiniko`uc >e`an d'wsoume >arnhtik`ec tim`ec st`on metrht`h t~wn
sel'idwn.  P.q., j'etontac {\tt \\pageno=-1} st`hn >arq`h to~u k'wdika,
o<i >arijmo`i <'olwn t~wn sel'idwn to~u >ent'upou j`a bgo~un <wc
latiniko'i.\TeXref{252} Diaforetik`ec kefal`ec mporo~un n`a
<orisjo~un gi`a mon`ec >`h zug`ec sel'idec, s'umfwna m`e t`on >ak'oloujo
tr'opo:  

{\tt
\\headline=\lb\\ifodd \\pageno \lb\dots\rb \\else \lb\dots\rb\\fi\rb}

\noindent
<'opou t`o ke'imeno po`u br'isketai >ent`oc t~wn d'uo pr'wtwn >agkul~wn 
qrhsimopoie~itai <wc kefal`h dexi~wn sel'idwn, ka`i t`o ke'imeno >ent`oc
t~wn de'uterwn >agkul~wn <wc kefal`h >arister~wn sel'idwn.

\exercise >All'axte t`o p'odi <en`oc >ent'upou sac >'etsi <'wste n`a
peri'eqei st`o k'entro t`on >arijm`o t~hc sel'idac >an'amesa s`e d'uo
<apl`ec pa~ulec ({\rm en-dash}).

%%%%%%%%%%%%%% restore original headline
%%%%%%%%%%%%%%
\headline={\iftitlepage \hfil \global\titlepagefalse
                                      \else \gentleheadline \fi}

\subsection{X'eqeila ka`i >'adeia pla'isia}

\nobreak

M'ia >ap`o t`ic pl'eon >ekneuristik`ec >empeir'iec po`u >'eqoun <'oloi
o<i n'eoi qr~hstec to~u {\rm \TeX} e>~inai <h >emf'anish x'eqeilwn ({\rm
overfull}) ka`i >'adeiwn ({\rm underfull}) {\tengs kouti~wn\/} >`h
{\tengs plais'iwn}.  Gi`a t`o {\rm \TeX}, <'ena pla'isio e>~inai <'ena
noht`o parallhl'ogramo qwr`ic <orat`h per'imetro po`u peri'eqei <'ena pl~hjoc
stoiqe'iwn s`e sugkekrim'enec j'eseic. T`o {\rm \TeX} katagr'afei <'ola
t`a problhmatik`a pla'isia <'ena-pr`oc-<'ena st`o >arqe~io {\tt .log},
kaj`wc >ep'ishc m~ac t`a de'iqnei ka`i st`hn >oj'onh >ef'' <'oson t`o
pr'ogramma tr'eqei <wc >al\-lhlo\-epi\-dr'on. T`o xeqe'ilisma to~u
keim'enou p'era >ap`o t`o dexi`o <'orio t~hc sel'idac shmei'wnetai ka`i
st`o >'entup'o mac m`e mikr`o ma~uro parallhl'ogrammo (s`an
a>ut'o:\NB{}\vrule width \overfullrule $\,$) st`o dex`i perij'wrio.  T`o
shm'adi a>ut`o mpa'inei >'estw ka`i >`an d`en <up'arqei kan'ena l'ajoc
st`on k'wdika.  T'i e>~inai loip`on a>ut`o t`o shm'adi ka`i p~wc
mporo~ume n`a t`o >apof'ugoume?

<'Enac kal`oc tr'opoc gi`a n`a katal'aboume p~wc t`o {\rm \TeX}
fti'aqnei m'ia sel'ida e>~inai n`a jewr'hsoume <'oti t`o kajet`i po`u
tup'wnetai st`hn sel'ida e>~inai <'ena >orjog'wnio parallhl'ogrammo
pla'isio ({\rm box}).  <Up'arqoun d'uo e>id~wn noht`a pla'isia gi`a t`o
{\rm \TeX}, t`a >apokalo'umena {\sl hbox\/} ka`i {\sl vbox}.  T`ic
periss'oterec for'ec, t`a pr~wta >antistoiqo~un st`hn stoiqeiojes'ia
keim'enou s`e <oriz'ontiec >ar'adec, >en~w t`a de'utera s`e paragr'afouc
po`u topojeto~untai katak'orufa <h m'ia >ep'anw st`hn >'allh.  Sunep~wc,
t`o <uperbolik`a meg'alo pl'atoc <en`oc {\rm hbox} po`u >antistoiqe~i
s`e m'ia >ar'ada to~u >ent'upou e>~inai >aut`o po`u prokale~i t`hn
>emf'anish to~u\NB{}\vrule width \overfullrule$\,$.

>`Ac jumhjo~ume <'oti t`o {\rm \TeX} diab'azei pr~wta m'ia <ol'oklhrh
par'agrafo ka`i kat'opin t`hn k'obei s`e >ar'adec.  A>ut`h <h
>epexergas'ia ka`i stoiqeiojes'ia <olokl'hrwn paragr'afwn >'eqei k'apoio
sugkekrim'eno skop'o; >e`an t`o {\rm \TeX} di'abaze ka`i
stoiqeiojeto~use m'ia >ar'ada t`hn for'a, t'ote m'ia mikr`h belt'iwsh
t~hc paro'usac >ar'adac mpore~i n`a kat'elhge s`e kak`h stoiqeioj'ethsh
m'iac >'allhc >ar'adac l'igo pi`o k'atw st`hn >'idia par'agrafo. <'Otan
loip`on t`o {\rm \TeX} topojete~i t`ic l'exeic m'iac >ar'adac t`hn m'ia
d'ipla st`hn >'allh, b'azei metax'u touc k'apoio ken`o di'asthma >'etsi
<'wste n`a g'inei swst`h sto'iqish to~u keim'enou st`o dex`i perij'wrio.
Profan~wc, d`en e>~inai >epijumht`o t`o di'asthma metax`u t~wn l'exewn  
n`a e>~inai pol`u meg'alo; t`o p'oso >'asqhmh e>~inai <h >emf'anish t~hc   
>ar'adac (st`hn gl'wssa to~u {\rm \TeX}: {\rm badness}), >exart~atai
>ap`o t`o ken`o di'asthma metax`u t~wn l'exewn.  <'Ena >'adeio ({\rm
underfull}) {\rm hbox}, shma'inei <'oti <up'arqei pol`u ken`o di'asthma
metax`u t~wn l'exewn.  T`o {\rm \TeX} baj\-mo\-lo\-ge~i t`o k'aje
pla'isio m`e <'enan bajm`o {\rm badness}: >e`an <o bajm`oc a>ut`oc gi`a
m'ia >ar'ada e>~inai 0, <h >ar'ada a>ut`h e>~inai t'eleia; >e`an e>~inai
10000, <h >ar'ada e>~inai >apa'isia.  <Up'arqei m'ia par'ametroc po`u
<onom'azetai {\tt \\hbadness} ka`i po`u <h >arqik'h thc tim`h e>~inai
1000\null. <Opoiad'hpote >ar'ada m`e bajm`o {\rm badness} megal'utero
>ap`o t`hn tim`h to~u {\tt \\hbadness} j`a >anaferje~i <wc {\rm
underfull hbox}. >E`an <or'isoume m'ia megal'uterh tim`h gi`a t`hn {\tt
\\hbadness}, t'ote <o >arijm`oc t~wn {\rm underfull hbox} po`u j`a m~ac
>anaf'erei t`o {\rm \TeX} j`a >elattwje~i.  M'alista, >e`an j'esoume {\tt
\\hbadness = 10000}, t`o {\rm \TeX} d`en j`a m~ac >anaf'erei kan'ena
{\rm undefull hbox}.  Kat`a tr'opo par'omoio, >e`an o<i l'exeic m'iac
>ar'adac pr'epei n`a strimwqjo~un ka`i <h >ar'ada >ekte'inetai >ent`oc
to~u dexio~u perijwr'iou, dhmiourge~itai <'ena x'eqeilo ({\rm overfull})
{\rm hbox}.
\toindex{hbadness}

T`o {\rm \TeX} >epitr'epei merik`ec for`ec o<i >ar'adec n`a
>ekte'inontai kat`a t`i >ent`oc to~u dexio~u perijwr'iou, dhl.\ n`a     
>'eqoun pl'atoc megal'utero >ap`o t`hn tim`h {\tt \\hsize}, prokeim'enou
<h par'agrafoc n`a e>~inai pi`o <omoi'omorfh (ka`i >'omorfh).  <H l'exh
>el'egqou {\tt \\tolerance} kajor'izei t`o p'ote sumba'inei a>ut'o.     
>E`an <o bajm`oc {\rm badness} m'iac >ar'adac e>~inai megal'uteroc >ap`o
t`hn tim`h {\tt \\tolerance}, t`o {\rm \TeX} j`a k'anei t`hn >ar'ada
megal'uterh prosj'etontac m'ia n'ea l'exh st`o t'eloc t~hc >ar'adac,
>'estw ka`i >`an a>ut`o k'anei t`o pl'atoc t~hc >ar'adac megal'utero
>ap`o t`o {\tt \\hsize}.  M'ia >ar'ada po`u t`o pl'atoc thc m'olic
xepern~a t`o {\tt \\hsize} d`en j`a >anaferje~i >ap`o t`o {\rm
\TeX}\null. <H l'exh >el'egqou {\tt\\hfuzz} e>~inai a>ut`h po`u
kajor'izei p'oso >epitr'epetai t`o pl'atoc m'iac >ar'adac n`a xepern~a
t`o {\tt\\hsize}.  <H >arqik`h tim'h thc e>~inai: {\tt \\hfuzz         
=\NB{}0,1\NB{}pt}.  M'ia >ar'ada po`u e>is'erqetai periss'otero >ap`o
{\tt \\hfuzz} st`o dexi`o perij'wrio dhmiourge~i pr'oblhma; t`o
{\rm\TeX} m~ac t`o de'iqnei b'azontac t`o 
shm'adi\NB{}\vrule width\overfullrule$\,$.  E>~inai dunat`o n`a
>apof'ugoume t`hn parous'ia t~wn x'eqeilwn {\rm hbox}, megal'wnontac
t`hn tim`h to~u {\tt \\tolerance}.  M`e {\tt \\tolerance = 10000}, d`en
j`a do~ume pot`e t`o {\rm \TeX} n`a  paraponi'etai gi`a {\rm overfull
box}.  <H >arqik`h tim`h to~u {\tt \\tolerance} e>~inai 200.%
\TeXref{29}
\toindex{hfuzz}
\toindex{tolerance}

T`o pl'atoc to~u shmadio~u \vrule width \overfullrule\ kajor'izetai m`e
t`hn l'exh >el'egqou {\tt \\overfullrule}.  <Or'i\-zon\-tac st`o
k'wdik'a mac {\tt \\overfullrule = 0 pt}, >apofe'ugoume t`hn >emf'anish
a>ut~wn t~wn >enoqlhtik~wn shmadi~wn.  T`a x'eqeila pla'isia j`a
parame'inoun; m'ono po`u j`a e>~inai pi`o d'uskolo n`a t`a
>ento\-p'i\-soume.
\toindex{overfullrule}

E>'idame loip`on gi`a poi'o l'ogo >emfan'izontai x'eqeila >`h >'adeia
pla'isia.  E>'idame >ep'ishc <'oti mporo~ume n`a k'anoume t`o {\rm \TeX}
n`a stamat'hsei t`a par'apona >all'azontac t`ic tim`ec t~wn {\tt
\\badness}, {\tt \\hfuzz} ka`i {\tt \\tolerance}. >Epipl'eon, m'ia
mikr`h tim`h to~u {\tt \\hsize}, profan~wc, k'anei pi`o d'uskolh t`hn
t`hn kop`h t~wn >ar'adwn ka`i prokale~i periss'otera probl'hmata m`e
x'eqeila ka`i >'adeia pla'isia.  S`e <'olec t`ic peript'wseic m`e
problhmatik`a pla'isia t`o {\rm \TeX} m~ac d'inei k'apoiec
proeidopoi'hseic po`u mporo~ume ka`i n`a t`ic >agno'hsoume, <'omwc  t`o
kr'ima j`a e>~inai <'olo dik'o mac!

Bohj'wntac l'igo st`on sullabism`o t~wn l'exewn, mporo~ume n`a
>apof'ugoume <orism'ena pro\-bl'h\-ma\-ta m`e x'eqeila ka`i >'adeia
pla'isia. T`o {\rm \TeX} gnwr'izei pol`u kal`a p~wc n`a sullab'izei
>agglik`ec l'exeic (ka`i >'iswc k'apoiec m`h >agglik'ec). <'Omwc e>~inai
dunat`o >eme~ic n`a <or'isoume n'ea shme~ia sullabismo~u <'opou j`a
>epitr'epetai <h kop`h m'iac l'exhc ka`i kat`a sun'epeia ka`i <h kop`h
m'iac <ol'oklhrhc >ar'adac. Gi`a par'adeigma, <o a>ut'omatoc
mhqanism`oc sullabismo~u to~u {\rm \TeX} pot`e d`en j`a k'oyei t`hn
l'exh {\rm database}.  >E`an <'omwc gr'ayoume t`hn l'exh a>ut`h st`on
k'wdik'a mac <wc {\tt data\\-base}, t'o\-te t`o {\rm \TeX} katalaba'inei
<'oti mpore~i n`a k'oyei a>ut`h t`hn l'exh met`a t`o de'utero (({\rm
a})).  Genik'wtera, >e`an gr'ayoume st`hn >arq`h to~u k'wdika {\tt
\\hyphenation\lb data-base\rb}, t'ote s`e <'olec t`ic <ep'omenec
>emfan'iseic t~hc l'exhc {\rm database} t`o {\rm \TeX} j`a gnwr'izei
p~wc n`a k'anei t`on swst`o sullabism'o thc.\TeXref{28} T`o >arqe~io
{\tt .log} de'iqnei <'ola t`a pijan`a shme~ia sullabismo~u t~hc >ar'adac
>eke'inhc <'opou >em\-fa\-n'i\-sjhke <'ena pr'oblhma {\rm overfull} >`h
{\rm underfull}. Merik`ec for'ec, ka`i k'apoiec mikroallag`ec st`o
ke'imeno bohjo~un st`hn >apofug`h t'etoiwn problhm'atwn.
\toindex{hyphenation}

<H suz'hths'h mac m'eqric >ed~w, peristr'afhke g'urw >ap`o t`o j'ema    
t~hc stoiqeiojes'iac >ar'adwn s`e paragr'afouc. Par'omoia m`e t`hn
stoiqeiojes'ia >ar'adwn, g'inetai ka`i <h katak'orufh topo\-j'e\-thsh
t~hc m'iac paragr'afou >ep'anw st`hn >'allh gi`a t`hn dhmiourg'ia t~hc
sel'idac.  Kat`a sun'epeia, t`o {\rm \TeX} mpore~i n`a paraponeje~i gi`a
x'eqeila >`h >'adeia katak'orufa pla'isia ({\rm vbox}), <'opwc j`a
parapono~utan gi`a x'eqeila >`h >'adeia <oriz'ontia pla'isia ({\rm
hbox}).  P.q., <'enac meg'aloc p'inakac <o <opo~ioc d`en mpore~i n`a
kope~i st`hn m'esh j`a >anaferje~i st`o >arqe~io {\tt .log} <wc (({\rm
overfull vbox})).  <H l'exh >el'egqou {\tt \\vbadness} e>~inai t`o
>an'alogo t~hc l'exhc >el'egqou {\tt \\hbadness} gi`a t`hn katak'orufh
topoj'ethsh paragr'afwn, k.lp.
\toindex{vbadness}

\exercise Stoiqeiojet~hste merik`ec paragr'afouc m`e di'aforec (mikr`ec
>`h meg'alec) tim`ec {\tt \\hsize}, gi`a n`a de~ite t'i e>~idouc
x'eqeila {\rm hbox} j`a s~ac parousiasto~un. >Epa\-na\-l'a\-be\-te t`hn
>'askhsh m`e di'aforec tim`ec t~wn {\tt \\hbadness}, {\tt \\hfuzz} ka`i
{\tt \\tolerance}.

\section{$\Bigl\{$S'unola, $\bigl\{$<upos'unola
        $\{$ka`i <upo"upos'unola$\}\bigr\}\Bigr\}$}

<H sugk'entrwsh keim'enou s`e {\tengs s'unola\/} >`h
{\tengs <om'adec\/} >`h {\tengs topik`a ped'ia dr'ashc\/} po`u
diakr'inontai >ap`o k'apoio >idia'itero koin`o qarakthristik`o (p.q.,
t'upoc stoiqe'iwn) e>~inai k'ati po`u <aplouste'uei shmanti\-k'wtata
t`hn <'olh >ergas'ia t~hc stoiqeiojes'iac.  <'Omwc t'i e>~inai >akrib~wc
<'ena s'unolo gi`a t`o {\rm \TeX}?  >`Ac xekin'hsoume l'egontac
<'oti <'ena n'eo s'unolo st`on k'wdik'a mac >arq'izei m`e t`on
qarakt'hra {\tt\lb} (t`o >arister`o >'agkistro) ka`i telei'wnei m`e t`on
qarakt'hra {\tt\rb} (t`o dexi`o >'agkistro).  <'Opoiec >all'agec
ka`i n`a k'anoume >ent`oc <en`oc sun'olou, d`en >'eqoun ka\-m'ia >ep'idrash
m'olic t`o s'unolo kle'isei. P.q., >e`an st`o >arqe~io {\rm \TeX}
gr'ayoume {\tt \lb \\bf three boldface words\rb}, t`o pr~wto >'agkistro
<or'izei t`hn >arq`h to~u sun'olou, <h l'exh >el'egqou {\tt\\bf}
diat'azei t`o {\rm \TeX} n`a >arq'isei n`a stoiqeiojete~i s`e >'entonouc
t'upouc ({\rm boldface}), ka`i t`o de'utero >'agkistro kle'inei t`o
s'unolo.  M`e t`o kle'isimo to~u sun'olou, t`o {\rm \TeX} stamat~a t`hn
stoiqeiojes'ia s`e >'entonouc t'upouc ka`i suneq'izei st`on >'idio t'upo
po`u qrhsimopoio~use pr`in t`hn >arq`h to~u sun'olou.  A>ut`oc e>~inai
<o e>ukol'wteroc tr'opoc gi`a n`a k'anoume parembol`ec st`o ke'imen'o
mac m`e diaforetik`o t'upo (pl'agia, >'entona, k.lp.)\null.  E>~inai
>ep'ishc dunat`on n`a >'eqoume <'ena s'unolo <wc m'eroc --- dhl.\ <wc
<upos'unolo --- <en`oc megal'uterou sun'olou (st`hn gl'wssa t~wn
pro\-gram\-mati\-st~wn mil~ame gi`a {\rm nested local scope groups}).

\def\sectiontitle{$\Bigl\{$S'unola, $\bigl\{$<upos'unola 
$\dots \bigr\}\Bigr\}$}

<Wc <'ena >ak'oma par'adeigma, >`ac do~ume p~wc g'inontai k'apoiec
proswrin`ec >allag`ec s`e diast'aseic to~u >ent'upou.  Gr'a\-fo\-ntac
loip`on t`a >ak'olouja st`on k'wdik'a mac
\beginuser
\lb
\\hsize = 4 in
\\parindent = 0 pt
\\leftskip = 1 in
\\TeX\\ will produce a paragraph that is four
$\vdots$
(this is an easy mistake to make).
\\par
\rb
\enduser

\noindent 
j`a l'aboume m'ia par'agrafo pl'atouc 4 >ints~wn m`e >od'ontwsh m'iac
>'intsac s`e <'olo t`o <'uyoc thc (ka\-ja\-r`o pl'atoc >ar'adac 3
>'intsec).  <H parak'atw par'agrafoc, po`u e>~inai o>usiastik`a <eto~uto
t`o ke'imeno metafrasm'eno st`hn >agglik`h gl'wssa, >apotele~i <'ena
par'adeigma. M`e t'eloc to~u sun'olou, t`o {\rm \TeX} j`a suneq'isei n`a
stoiqeiojete~i t`ic paragr'afouc po`u >akoloujo~un st`ic dia\-st'a\-seic
po`u >'hxere pr`in sunant'hsei t`o s'unolo. <'Omwc prosoq'h: gi`a n`a
l'aboume t`o swst`o >apot'elesma, pr'epei n`a gr'a\-you\-me {\tt \\par}
>`h n`a >af'hsoume m'ia ken`h gramm`h st`on k'wdik'a mac pr`in t`o
>'agkistro {\tt\rb} po`u kle'inei t`o s'unolo.  Dia\-fore\-tik`a t`o
{\rm \TeX} j`a stoiqeiojet'hsei ka`i t`hn par'agrafo >ent`oc to~u
sun'olou st`ic prohgo'umenec diast'aseic po`u >'hxere pr`in diab'asei
t`o s'unolo (e>~inai e>'ukolo n`a m~ac xef'ugei <'ena t'etoio l'ajoc).

{\hsize = 4 in
\parindent = 0 pt
\leftskip = 1 in \rm
\TeX\ will produce a paragraph that is four inches wide with the text
offset into the paragraph by one inch regardless of the settings in
effect before the start of the group.  This paragraph is set with those
values. After the end of the group, the old settings are in effect
again. Note that it is necessary to include {\tt \\par} or to use a
blank line before the closing brace to end the paragraph, since
otherwise the group will end and \TeX{} will go back to the old
parameters before the paragraph is actually typeset (this is an easy
mistake to make). \par}

<'Otan m'ia l'exh el'egqou (<'opwc {\tt \\centerline}) prohge~itai
<en`oc keim'enou po`u perikle'ietai m`e >'agkistra, t'ote t`o ke'imeno
a>ut`o >apotele~i <'ena s'unolo.  >'Etsi, gr'afontac {\tt
\\centerline\lb\\bf A bold title\rb} dhmiourgo~ume m'ia kentrarism'enh
>ar'ada s`e >'entonouc t'u\-pouc, >en~w t`o ke'imeno po`u >akolouje~i
met`a >ap`o a>ut`h t`hn >ar'ada j`a bge~i st`on t'upo po`u
qrhsi\-mo\-poio'usame pr`in t`hn >entol`h {\tt \\centerline}.

T`o ken`o s'unolo {\tt\lb\rb} e>~inai >ep'ishc pol`u qr'hsimo. 
Mporo~ume n`a t`o qrhsimopoi'hsoume gi`a n`a tup'wsoume tonik`a shme~ia
qwr`ic t`hn parous'ia k'apoiou gr'ammatoc.  Gi`a par'adeigma, <h
>entol`h {\tt \\\~{}\lb\rb} m~ac d'inei m'ia perispwm'enh qwr`ic k'apoio
gr'amma k'atw >ap'' a>ut'hn.  >Ak'oma, t`o ken`o s'unolo mpore~i n`a
stamat'hsei t`o {\rm \TeX} >ap`o t`o n`a ((tr'wei)) suneq'omena ken`a
diast'hmata.  Gr'a\-fo\-ntac loip'on, {\tt I use \\TeX\lb\rb{} all the  
time}, lamb'anoume <'ena ken`o di'asthma met`a t`o log'otupo
{\rm\TeX}\null. A>ut`oc e>~inai <'enac >enallaktik`oc tr'opoc gi`a n`a
j'etoume ken`a diast'hmata (<o >'alloc tr'opoc e>~inai n`a
qrhsimopoi'hsoume t`o s'umbolo >el'egqou {\tt \\\sp} <'opwc k'aname st`o
kef'alaio\NB{}1.)%
\TeXref{19--21}

<H <omadopo'ihsh qarakt'hrwn s`e <'ena >`h ka`i periss'otera s'unola
mpore~i n`a g'inei >ak'oma ka`i st`hn m'esh m'iac l'exhc, p.q., <'otan
<h l'exh a>ut`h peri'eqei tonism'enouc qarakt~hrec.  Sunep~wc,
gr'afontac st`on k'wdika e>'ite {\tt soup\\c\sp con} e>'ite {\tt
soup\\c\lb c\rb on}, st`o >'entupo lamb'anoume t`hn l'exh {\rm
soup\c{c}on}.

\exercise >All'axte t`ic diast'aseic m'iac paragr'afou qrhsimopoi'wntac
t`hn >id'ea to~u sun'olou.

\exercise O<i majhmatiko`i merik`ec for`ec gr'afoun (({\rm i{f}f})) <wc 
suntomograf'ia t~hc fr'ashc ((>e`an ka`i m'onon >e'an)) ({\rm if and
only if}).  St`hn per'iptwsh a>ut`h e>~inai protim'wtero t`o pr~wto ka`i
t`o de'utero (({\rm f})) n`a m`hn <enwjo~un <wc <'ena s'unjeto
stoiqe~io.  P~wc j`a pet'uqete k'ati t'etoio? (<Up'arqoun poll`ec
l'useic!)

\bigskip

<'Otan fti'aqnoume <'ena s'unolo, e>~inai pol`u e>'ukolo n`a xeq'asoume
<'ena >ap`o t`a d'uo >'agkistra, sun'hjwc t`o dex'i.  T`o >apot'elesma
mpore~i n`a e>~inai katastrofik'o; >e`an do~ume <'olo t`o >'entupo n`a
bga'inei s`e pl'agiouc t'upouc, >ant`i p.q.\ {\rm roman}, t'ote k'apou
>'iswc n`a >'eqoume >af'hsei <'ena >'agkistro qwr`ic ta'iri.  >E`an
<up'arqei <'ena {\tt\lb} qwr`ic t`o >ant'istoiqo {\tt\rb}, t`o {\rm
\TeX} j`a paraponeje~i: 

{\tt (\\end occurred inside a group at level 1)}. 

>Ant'ijeta, <'ena {\tt\rb} qwr`ic ta'iri j`a k'anei t`o {\rm \TeX}
n`a dia\-mar\-turh\-je~i <wc >ex~hc: 

{\tt! Too many \rb's.}

<Or'iste p~wc mporo~ume n`a >apof'ugoume t`o mp'erdema m`e t`a
>'agkistra s`e pol'uploka s'unola: St`on k'wdik'a mac gr'afoume t`o
>arister`o >'agkistro s`e m'ia xeqwrist`h gramm`h m'ono tou ka`i
>ep'ishc gr'afoume t`o dex`i >'agkistro m'ono tou s`e m'ia gramm'h. 
<'Otan dhmiourgo~ume n'ea <upo\-s'u\-nola >ent`oc to~u >arqiko~u
sun'olou, gr'afoume >ep'ishc t`a >'agkistra k'aje <uposun'olou >ep'ishc
s`e xeqwrist`ec gramm'ec, >all`a >'oqi st`hn pr'wth j'esh t~hc gramm~hc;
mporo~ume n`a dhmiourg'hsoume m'ia >od'ontwsh (p.q., m`e t`o pl~hktro
{\rm TAB})\null.  >Epipl'eon, mporo~ume n`a metakin'hsoume t`o ke'imeno
po`u perikle'ioun a>ut`a t`a >'agkistra l'igo pr`oc t`o >eswterik`o t~hc
gramm~hc <'opwc st`o parak'atw par'adeigma:
\beginuser \obeyspaces 
\lb
\    This text belongs to the first group. 
\    $\vdots$
\    \lb
\         This text belongs to the first subgroup.
\         $\vdots$
\    \rb
\rb
\enduser

\noindent  
>'Etsi t`a >'agkistra to~u k'wdik'a mac g'inontai pi`o e>udi'akrita.
M'alista, >e`an t`o pr'ogramma s'untaxhc po`u qrhsimopoio~ume e>~inai
l'igo pi`o >'exupno, >'iswc n`a mporo~ume n`a gr'ayoume pr~wta t`o
ze~ugoc t~wn >agk'istrwn ka`i met'a, metax`u t~wn >agk'istrwn, t`o
ke'imeno m`e a>ut'omath >od'ontwsh.

\exercise St`o kef'alaio 2, >all'axame t'upo m`e t`hn >ak'oloujh
m'ejodo: {\tt I started with roman type, \\it switched to italic type,
\\rm and returned to roman type}.  N`a k'a\-ne\-te t`o >'idio,
qrhsimopoi'wntac t`hn >id'ea to~u sun'olou.


\section{Majhmatik`a qwr`ic >'agqoc!}

T`o {\rm \TeX} e>~inai t`o >idanik`o >ergale~io gi`a t`hn stoiqeiojes'ia
>ent'upwn po`u peri'eqoun majhmatiko`uc t'upouc >`h >ekfr'aseic.  O<i
majhmatik`ec >ekfr'aseic mpore~i n`a e>~inai poll~wn e>id~wn ka`i
>arket`a pol'uplokec, >all`a t`o {\rm \TeX} t`ic >epexerg'azetai
>aristoteqnik`a k'anontac dunat`h t`hn paragwg`h majhmatik~wn >ent'upwn
>exairetik~hc poi'othtac. >E`an pr'okeitai n`a <etoim'asoume k'a\-poiec
dhmosie'u\-seic po`u peri'eqoun majhmatik`a s'umbola, st`o kef'alaio
a>ut`o j`a do~ume <'olec t`ic basik`ec >entol`ec po`u j`a qreiasjo~ume
s`e <opoiad'hpote per'iptwsh. T`o {\rm \TeX} beba'iwc mporo~ume n`a t`o
qrhsimopoi'hsoume ka`i gi`a >'entupa m`e l'igouc >`h kaj'olou
majhmatiko`uc sumbolismo'uc;  t'ote o<i d'uo par'agrafoi po`u
>akoloujo~un e>~inai m'allon >arket`ec gi`a t`ic >an'agkec mac.

\subsection{Poll`a n'ea s'umbola}

O<i majhmatik`ec >ekfr'aseic e>is'agontai >ent`oc to~u kanoniko~u
keim'enou kat`a d'uo tr'opouc: mpore~i na mpo~un {\tengs >ent`oc
st'iqou}, <wc m'eroc kanonik~wn >ar'adwn keim'enou l'ogou, >`h <wc {\tengs
diakrit'ec}, dhl.\ >ent`oc <en`oc nohto~u kentrwm'enou plais'iou s`e
<'enan ken`o q~wro metax`u >ar'adwn kanoniko~u keim'enou.  T`o
>apot'elesma st`hn topoj'ethsh ka`i t`a diast'hmata metax`u t~wn
sumb'olwn j`a e>~inai s`e k'aje per'iptwsh diaforetik'o.  <H >ent`oc
st'iqou >ex'iswsh  $\sum_{k=1} ^{\infty} {1\over k^2} = {\pi^2\over6}$
d`en de'iqnei t`o >'idio <'otan mpa'inei <wc diakrit'h:

$$\sum_{k=1}^{\infty} {1\over k^2} = {\pi^2\over6}.$$

>Ef'' <'oson t`a diast'hmata ka`i o<i t'upoi stoiqe'iwn po`u
qrhsimopoio~untai s`e majhmatik`ec >ekfr'aseic diaf'eroun >arket`a >ap`o
a>ut`a po`u qrhsimopoio~untai gi`a ke'imeno l'ogou, pr'epei n`a d'wsoume
st`o {\rm \TeX} n`a katal'abei p'ote >'eqei n`a stoiqeiojet'hsei m'ia
majhmatik`h >'ekfrash >ant`i keim'enou l'ogou.  A>ut`o g'inetai
qrhsimopoi'wntac st`on k'wdika t`o s'umbolo to~u dolar'iou {\tt\$}.
E>idik'wtera, m'ia majhmatik`h >'ekfrash po`u stoiqeiojete~itai >ent`oc
st'iqou, t`hn gr'afoume st`on k'wdika perikle'iontac thn m`e mon`a
dol'aria: {\tt\$$\ldots$\$}\null.  >E`an t`hn perikle'isoume metax`u
dipl~wn dolar'iwn: {\tt\$\$$\ldots$\$\$}, j`a stoiqeiojethje~i <wc
diakrit`h kentrwm'enh.  >'Etsi <o k'wdikac {\tt \$x = y+1\$} m~ac d'inei
$x=y+1$ >ent`oc st'iqou, >en~w <o k'wdikac {\tt \$\$x = y+1.\$\$}
d'inei: $$x=y+1.$$

T`a diast'hmata gi`a majhmatik`ec >ekfr'aseic t'oso >ent`oc st'iqou
<'oso ka`i kentrwm'enec >el'egqontai >ap'oluta >ap`o t`o {\rm
\TeX}\null.  T`o n`a prosj'esoume ken`a diast'hmata st`on k'wdika d`en
j`a >'eqei kan'ena >apo\-t'e\-lesma.  Ka`i >e`an j'eloume n`a b'aloume
k'apoio ken`o di'asthma >`h k'apoio ke'imeno l'ogou st`hn m'esh m'iac
majhmatik~hc >'ekfrashc?  Mporo~ume n`a b'aloume k'apoio ke'imeno l'ogou
>ent`oc m'iac majhmatik~hc >'ekfrashc j'etont'ac to s`e <'ena {\rm
hbox}: {\tt \\hbox\lb$\ldots$\rb}\null.  A>ut`o e>~inai pol`u qr'hsimo
gi`a kentrwm'enec majhmatik`ec >ekfr'aseic.  Kat`a t`on tr'opo a>ut'o,
<h >'ekfrash (($x=y+1 \hbox{ \rm whenever } y=x-1$)) stoiqeiojete~itai
m`e t`on k'wdika {\tt \$x=y+1 \\hbox\lb\ whenever \rb y=x-1\$}. >`Ac
pros'exoume t`a diast'hmata <ekat'erwjen t~hc l'exhc >ent`oc t~wn
>agk'istrwn.  Sun'hjwc d`en qrei'azetai n`a b'azoume ken`a diast'hmata
>ent`oc majhmatik~wn sumbolism~wn, >all`a st`hn pe\-r'i\-ptw\-sh po`u
a>ut`o e>~inai >apara'ithto o<i parak'atw >akolouj'iec >el'egqou j`a
m~ac k'anoun t`hn doulei'a.% 
\TeXref{167}

\maketable [Prosj'hkh diasthm'atwn s`e majhmatiko`uc t'upouc]
\halign{
\strut \hfil # \hfil & \quad \hfil\tt# \hfil \quad
   & \hbox to 2cm{\hrulefill\vrule height 8pt#\vrule height 8pt\hrulefill} \cr
   >Onomas'ia & \tengr >Akolouj'ia el'egqou & $\gets$M~hkoc$\to$\cr
   \noalign{\hrule} \noalign{\smallskip}
   dipl`o tetr'agwno            & \\qquad &\qquad \cr
   <apl`o tetr'agwno            & \\quad  &\quad \cr
   di'asthma                    & \\\sp\  &\ \cr
   meg'alo di'asthma            & \\;     &$\;$\cr
   mesa~io di'asthma            & \\>     &$\>$\cr
   mikr`o di'asthma             & \\,     &$\,$\cr
   >arnhtik`o mikr`o di'asthma  & \ \\!\  &$\!$\cr
       }

\toindex{quad}
\toindex{qquad}
\toindex{\sp}
\toindex{;}
\toindex{>}
\toindex{,}
\toindex{!}

>E`an parathr'hsoume prosektik`a t`o mikr`o >arnhtik`o di'asthma, j`a
do~ume <'oti s`e >ant'ijesh m`e t`a >'alla diast'hmata, t`a d'uo <'oria
to~u diast'hmatoc >allhlo\-epikal'u\-ptontai.  A>ut`o sumba'inei giat`i
t`o >arnhtik`o di'asthma e>~inai >ant'ijethc kate'ujunshc, dhl.\ >en~w
<'olec o<i >'allec >akolouj'iec >el'egqou a>ux'anoun t`o ken`o di'asthma
metax`u d'uo sumb'olwn, t`o mikr`o >arnhtik`o di'asthma t`o >elatt'wnei
>'estw ka`i >`an prokale~itai >allhlo\-epi\-k'aluyh t~wn sumb'olwn.

\exercise Stoiqeiojet~hste: $C(n,r) = n!/(r!\,(n-r)!)$\null.
Pros'exte t`a diast'h\-ma\-ta st`on paronomast'h.    
\bigskip

St`on k'wdika, metax`u t~wn sumb'olwn {\tt \$} po`u perikle'ioun k'apoio
majhmatik`h >'ekfrash d`en pr'epei n`a <up'arqoun ken`ec gramm'ec.  T`o
{\rm\TeX} <upoj'etei <'oti <'olh <h majhmatik`h  >'ekfrash >apotele~i
m'ia par'agrafo ka`i <'oti m'ia ken`h gramm`h shma'inei n'ea par'agrafo.
 Sunep~wc, j`a m~ac d'wsei m'h\-nu\-ma sf'almatoc.  A>ut`h <h
>idiotrop'ia to~u {\rm \TeX} e>~inai >arket`a qr'hsimh, giat`i <'ena
>ap`o t`a pi`o suqn`a sf'almata st`on k'wdika e>~inai n`a parale'ipetai
t`o de'utero s'umbolo {\tt \$} (>`h {\tt \$\$}) po`u kle'i\-nei t`hn
majhmatik`h >'ekfrash (s'igoura j`a k'anoume toul'aqiston m'ia for`a
a>ut`o t`o l'ajoc kaj`wc maja'i\-noume t`o {\rm\TeX})\null.  >E`an t`o
{\rm\TeX} >ep'etrepe periss'oterec >ap`o m'ia paragr'afouc metax`u t~wn
sumb'olwn {\tt \$}, t'ote xeqn'wntac <'ena de'utero {\tt\$}, j`a
prokalo'usame t`hn stoiqeiojes'ia <'olou to~u <upolo'ipou keim'enou
st`hn morf`h m'iac  majhmatik~hc >'ekfrashc.

O<i periss'oterec majhmatik`ec >ekfr'aseic e>is'agontai kat`a t`on
>'idio tr'opo e>'ite >ent`oc st'iqou e>'ite <wc diakrit`ec
kentrwm'enec. T`ic >exair'eseic po`u >aforo~un m'onon t`ic diakrit`ec
>ekfr'aseic, <'opwc t`hn katak'orufh sto'iqish pollapl~wn t'upwn ka`i
t`hn >ar'ijmhsh >exis'wsewn st`o dexi`o >`h t`o >arister`o perij'wrio,
j`a t`ic suzht'hsoume st`o t'eloc to~u kefala'iou.

Kat`a t`hn stoiqeiojes'ia majhmatik~wn t'upwn >emfan'izontai poll`a n'ea
s'umbola.  T`a peris\-s'otera >ap`o t`a s'umbola to~u plhktrolog'iou
mporo~un n`a qrhsimopoihjo~un <'opwc >'eqoun.  T`a s'umbola {\tt + - / *
= ' | < > (} ka`i {\tt)} e>is'agontai <'ola <'opwc >'eqoun.  <Or'iste
t'i m~ac d'inoun:  $+ \>  - \> / \> * \> = \> ' \> | < \> > \> ( \> )$.

\exercise Stoiqeiojet~hste t`hn >ex'iswsh $a+b=c-d=xy=w/z$ >ent`oc
st'iqou ka`i <wc diakrit'h.

\exercise Stoiqeiojet~hste t`hn >ex'iswsh $(fg)' = f'g + fg'$ >ent`oc
st'iqou ka`i <wc diakrit'h.                 
\bigskip

Poll`a >'alla s'umbola d'inontai <wc prokajorism'enec >akolouj'iec
>el'egqou.  P.q., t`o {\rm \TeX} m~ac d'inei <'olouc to`uc <ellhniko`uc
qarakt~hrec gi`a majhmatik`a s'umbola <wc l'exeic >el'egqou. 
Para\-k'a\-tw d'inetai <'enac p'inakac m`e t`a <ellhnik`a majhmatik`a
s'umbola; <'omwc prosoq'h: t`a s'umbola a>ut'a, <'opwc j`a do~ume st`o
kef'alaio\NB{}10, e>~inai kat'allhla m'ono gi`a majhmatik`a ka`i >'oqi
gi`a t`hn stoiqeiojes'ia <aplo~u <ellhniko~u keim'enou l'ogou.% 
\TeXref{434}

\maketable [<Ellhnik`a s'umbola majhmatik~wn t'upwn]
\halign{
\strut \hfil$#$ & \quad \tt# \hfil \qquad &\hfil$#$ & \quad \tt# \hfil \qquad
      &\hfil$#$ & \quad \tt# \hfil \qquad &\hfil$#$ & \quad \tt# \hfil \cr
\noalign{\hrule} \noalign{\smallskip}
\alpha & \\alpha &\beta & \\beta &\gamma & \\gamma &\delta & \\delta \cr
\epsilon & \\epsilon & \varepsilon & \\varepsilon & \zeta & \\zeta &
\eta & \\eta \cr
\theta & \\theta & \vartheta & \\vartheta & \iota & \\iota & \kappa &
\\kappa \cr
\lambda & \\lambda & \mu & \\mu & \nu & \\nu & \xi & \\xi \cr
o & o & \pi & \\pi & \varpi & \\varpi & \rho & \\rho \cr
\varrho & \\varrho & \sigma & \\sigma & \varsigma & \\varsigma & \tau & \\tau \cr
\upsilon & \\upsilon & \phi & \\phi & \varphi & \\varphi & \chi & \\chi \cr
\psi & \\psi & \omega & \\omega & \Gamma & \\Gamma & \Delta & \\Delta \cr
\Theta & \\Theta & \Lambda & \\Lambda & \Xi & \\Xi & \Pi & \\Pi \cr 
\Sigma & \\Sigma & \Upsilon & \\Upsilon & \Phi & \\Phi & \Psi & \\Psi \cr
\Omega & \\Omega \cr
\noalign{\smallskip} \noalign{\hrule}
       }

\toindex{alpha}
\toindex{beta}
\toindex{gamma}
\toindex{delta}
\toindex{epsilon}
\toindex{varepsilon}
\toindex{zeta}
\toindex{eta}
\toindex{theta}
\toindex{vartheta}
\toindex{iota}
\toindex{kappa}
\toindex{lambda}
\toindex{mu}
\toindex{nu}
\toindex{xi}
\toindex{pi}
\toindex{varpi}
\toindex{rho}
\toindex{varrho}
\toindex{sigma}
\toindex{varsigma}
\toindex{tau}
\toindex{upsilon}
\toindex{phi}
\toindex{varphi}
\toindex{chi}
\toindex{psi}
\toindex{omega}
\toindex{Gamma}
\toindex{Delta}
\toindex{Theta}
\toindex{Lambda}
\toindex{Xi}
\toindex{Pi}
\toindex{Sigma}
\toindex{Upsilon}
\toindex{Phi}
\toindex{Psi}
\toindex{Omega}

\exercise Stoiqeiojet~hste t`hn >ex'iswsh $\alpha\beta=\gamma+\delta$
>ent`oc st'iqou ka`i <wc diakrit'h.

\exercise Stoiqeiojet~hste $\Gamma(n) = (n-1)!$ >ent`oc st'iqou ka`i
<wc diakrit`h >ex'iswsh.

Merik`ec for`ec >ep'anw >ap`o t`a s'umbola >`h ka`i k'atw >ap`o a>ut`a
topojeto~untai k'apoia diakritik`a shme~ia >`h t'onoi.  O<i l'exeic
>el'egqou gi`a t`hn >'enjesh a>ut~wn t~wn shme'iwn e>~inai diaforetik`ec
>ap`o t`ic >ant'istoiqec gi`a t`on tonism`o qarakt'hrwn kanoniko~u
keim'enou.  O<i >akolouj'iec el'egqou gi`a tonism`o kanoniko~u keim'enou
d`en qrhsimopoio~untai st`hn stoiqeiojes'ia majhmatik~wn t'upwn ka`i
>antistr'ofwc.%
\TeXref{135--136}

\maketable [Tonik`a shme~ia majhmatik~wn t'upwn]
\halign{
   \strut \hfil$#$ & \quad \tt# \hfil \qquad &\hfil$#$ &
      \quad \tt# \hfil \qquad &\hfil$#$ & \quad \tt# \hfil \qquad  \cr
   \noalign{\hrule} \noalign{\smallskip}
   \hat o   & \\hat o   & \check o & \\check o  & \tilde o & \\tilde o \cr
   \acute o & \\acute o & \grave o & \\grave o  & \dot o & \\dot o \cr
   \ddot o  & \\ddot o  &\breve o  & \\breve o  & \bar o & \\bar o \cr
   \vec o   & \\vec o   & \widehat {abc} & \\widehat \lb abc\rb
                    & \widetilde {abc} & \\widetilde \lb abc\rb\cr
       }

\toindex{hat}
\toindex{check}
\toindex{tilde}
\toindex{acute}
\toindex{grave}
\toindex{dot}
\toindex{ddot}
\toindex{breve}
\toindex{bar}
\toindex{vec}
\toindex{widehat}
\toindex{widetilde}

St`hn <orolog'ia t~wn majhmatik~wn, o<i duadiko`i telest`ec <en'wnoun
d'uo >antike'imena gi`a n`a d'wsoun <'ena tr'ito >antike'imeno.  <H
pr'osjesh ka`i <o pollaplasiasm'oc, gi`a par'adeigma, <en'wnoun d'uo
>arijmo`uc ka`i d'inoun <'enan tr'ito >arijm'o, >'ara pr'okeitai gi`a
duadiko`uc telest'ec. <'Otan t`o {\rm \TeX} stoiqeiojete~i <'enan
duadik`o telest'h, prosj'etei l'igo parap'anw ken`o di'asthma >arister`a
ka`i dexi'a tou.  <Or'iste <'enac p'inakac m`e meriko`uc >ap`o to`uc
diaj'esimouc duadiko`uc telest'ec:%
\TeXref{436}

\maketable [Duadiko`i telest`ec]   
\halign{
\strut \hfil$#$ & \quad \tt# \hfil \qquad &\hfil$#$ & \quad \tt# \hfil \qquad
      &\hfil$#$ & \quad \tt# \hfil \qquad &\hfil$#$ & \quad \tt# \hfil \cr
\noalign{\hrule} \noalign{\smallskip}
\cdot & \\cdot &\times & \\times &\ast & \\ast &\star & \\star \cr
\circ & \\circ & \bullet & \\bullet & \div & \\div & \diamond & \\diamond \cr
\cap & \\cap & \cup & \\cup & \vee & \\vee & \wedge & \\wedge \cr
\oplus & \\oplus &\ominus & \\ominus & \otimes &\\otimes &\odot &\\odot \cr
       }

\toindex{cdot}
\toindex{times}
\toindex{ast}
\toindex{star}
\toindex{circ}
\toindex{bullet}
\toindex{div}
\toindex{diamond}
\toindex{cap}
\toindex{cup}
\toindex{vee}
\toindex{wedge}
\toindex{oplus}
\toindex{ominus}
\toindex{otimes}
\toindex{odot}

Suqn`a maz`i m`e to`uc duadiko`uc telest'ec, qrhsimopoie~itai ka`i <h
>'elleiyh, dhl.\ >aposiwphtik`a po`u <upodhl'wnoun <'oti k'ati par'omoio
parale'ipetai st`on majhmatik'o mac t'upo. <H l'exh >el'egqou {\tt
\\cdots} prokale~i m'ia katak'orufh metat'opish t~hc >'elleiyhc <'wste
o<i tre~ic tele~iec thc n`a brejo~un st`on >'idio <oriz'ontio >'axona
summetr'iac m`e to`uc <up'opoloipouc telest'ec. >'Etsi loip'on, m`e t`on
kwdik`o {\tt \$a + \\cdots + z\$}, lamb'anoume $a + \cdots + z$.  <H
>akolouj'ia >el'egqou {\tt \\ldots} d`en >anuy'wnei t`hn >'elleiyh
>all`a t`hn j'etei >ep'anw st`hn gramm`h b'ashc to~u majhmatiko~u
t'upou; >'etsi <o kwdik`oc {\tt \$1\\ldots n\$} d'inei $1\ldots n$.
\toindex{cdots}
\toindex{ldots}

\exercise  Stoiqeiojet~hste: $x\wedge (y\vee z) = (x\wedge y) \vee
(x\wedge z)$.
               
\exercise Stoiqeiojet~hste: $2+4+6+\cdots +2n = n(n+1)$.
\bigskip

M'ia sq'esh de'iqnei m'ia >idi'othta d'uo majhmatik~wn >antikeim'enwn. 
Gnwr'izoume >'hdh p~wc n`a de'ixoume <'oti d'uo >antike'imena e>~inai
>'isa, >`h p~wc n`a de'ixoume <'oti <'enac >arijm`oc e>~inai
megal'uteroc >`h mikr'oteroc <en`oc >'allou >arijmo~u (>ef'' <'oson t`a
>apara'ithta s'umbola <up'arqoun st`a plhktol'ogia <'olwn sqed`on t~wn
<upologist~wn).  Gi`a n`a stoiqeiojet'hsoume m'ia >arnhtik`h sq'esh,
gr'afoume st`on k'wdika t`hn l'exh >el'egqou {\tt \\not} >empr`oc >ap`o
t`hn sq'esh.  <Or'iste merik`ec t'etoiec majhmatik`ec sq'eseic:%
\TeXref{436}
\toindex{not}

\maketable [Majhmatik`ec sq'eseic]
\halign{
\strut \hfil$#$ & \quad \tt# \hfil \qquad &\hfil$#$ & \quad \tt# \hfil \qquad
      &\hfil$#$ & \quad \tt# \hfil \qquad &\hfil$#$ & \quad \tt# \hfil \cr
\noalign{\hrule} \noalign{\smallskip}
\leq  & \\leq  &\not\leq & \\not \\leq
       & \geq & \\geq & \not\geq & \\not \\geq \cr
\equiv & \\equiv & \not\equiv & \\not \\equiv
       & \sim & \\sim & \not\sim & \\not \\sim \cr
\simeq & \\simeq & \not\simeq & \\not \\simeq
       & \approx & \\approx & \not\approx & \\not \\approx \cr
\subset & \\subset & \subseteq & \\subseteq
        & \supset & \\supset & \supseteq & \\supseteq \cr
\in & \\in & \ni & \\ni & \parallel & \\parallel & \perp & \\perp \cr
       }

\toindex{leq}
\toindex{geq}
\toindex{equiv}
\toindex{sim}
\toindex{simeq}
\toindex{approx}
\toindex{subset}
\toindex{subseteq}
\toindex{supset}
\toindex{supseteq}
\toindex{in}
\toindex{ni}
\toindex{parallel}
\toindex{perp}

\exercise Stoiqeiojet~hste: {\rm $\vec x\cdot \vec y  = 0$ if and only
if $\vec x \perp \vec y$. }

\exercise Stoiqeiojet~hste: {\rm $\vec x\cdot \vec y \not= 0$ if and
only if $\vec x \not\perp \vec y$. }

<Or'iste ka`i merik`a >ak'oma diaj'esima majhmatik`a s'umbola:%
\TeXref{435--438}

\maketable [Di'afora majhmatik`a s'umbola]
\halign{
\strut \hfil$#$ & \quad \tt# \hfil \qquad &\hfil$#$ & \quad \tt# \hfil \qquad
      &\hfil$#$ & \quad \tt# \hfil \qquad &\hfil$#$ & \quad \tt# \hfil \cr
\noalign{\hrule} \noalign{\smallskip}
\aleph & \\aleph & \ell & \\ell & \Re & \\Re & \Im & \\Im \cr
\partial & \\partial & \infty & \\infty & \| & \\| & \angle & \\angle \cr
\nabla & \\nabla & \backslash &\\backslash & \forall & \\forall
              & \exists & \\exists \cr
\neg & \\neg & \flat & \\flat & \sharp & \\sharp & \natural & \\natural \cr
        }

\toindex{aleph}
\toindex{ell}
\toindex{Re}
\toindex{Im}
\toindex{partial}
\toindex{infty}
\toindex{|}
\toindex{angle}
\toindex{nabla}
\toindex{backslash}
\toindex{forall}
\toindex{exists}
\toindex{neg}
\toindex{flat}
\toindex{sharp}
\toindex{natural}

\exercise Stoiqeiojet~hste: $(\forall x\in \Re)(\exists y\in\Re)$ $y>x$.

S`e <'ola t`a prohgo'umena parade'igmata, mporo~ume n`a parathr'hsoume
<'oti t`a stoiqe~ia po`u qrhsimopoio~untai a>utom'atwc >ap`o t`o
{\rm\TeX} gi`a t`hn stoiqeiojes'ia majhmatik~wn sumb'olwn e>~inai
pl'agia--kalligrafik'a ({\rm italic}).  A>ut`o g'inetai gi`a n`a
xeqwr'izei t`o <apl`o ke'imeno >ap`o t`a ma\-jh\-ma\-ti\-k`a s'umbola
ka`i to`uc sunduasmo'uc touc.  P.q., m`e t`on k'wdika: {\tt \$I\$ is the
product \$i s\$, where \$i\$ is the current density and \$s\$ the
cross-cut area}, lam\-b'a\-nou\-me: {\rm $I$ is the product $is$, where
$i$ is the current density and $s$ the cross-cut area}.  >Arke~i n`a
do~ume t`hn >'idia fr'ash m`e t`a s'umbola $I$, $i$ ka`i $s$
stoiqeiojethm'ena >'oqi m`e pl'agia--kalligrafik'a, >all`a m`e >'orjia
stoiqe~ia {\rm roman}, gi`a n`a katal'aboume t`hn diafor'a: {\rm I is
the product $\rm i s$, where i is the current density and s the
cross-cut area}.

<Wst'oso, merik`ec for`ec qrhsimopoio~untai ka`i >'allwn e>id~wn
stoiqe~ia gi`a t`hn stoiqeiojes'ia majhmatik~wn sumb'olwn.  Gi`a
par'adeigma, t`a qhmik`a s'umbola ka`i o<i di'aforec stajer`ec
sumbol'izontai m`e >'orjiouc qarakt~hrec.  >Ep'ishc, o<i mon'adec po`u
sunode'uoun t'upouc t~hc fusik~hc ka`i t~hc qhme'iac, p'anta
stoiqeiojeto~untai m`e >'orjiouc qarakt~hrec.  >'Allote p'ali, o<i
majhmatiko`i p'inakec sumbol'izontai gi`a suntom'ia m`e >'entona
kefala~ia. T`o {\rm\TeX} m~ac >epitr'epei n`a >all'azoume grammatoseir`a
>ak'oma ka`i st`hn stoiqeiojes'ia majhmatik~wn t'upwn.  >'Etsi m`e t`on
k'wdika: {\tt\$\\rm R = 8.2054\\, J\\, mol\^{}\lb-1\rb\\, K\^{}\lb-1\rb
\$}, lamb'anoume t`hn pagk'osmia sta\-je\-r`a t~wn >aer'iwn: $\rm R =
8.2054\, J\, mol^{-1}\,K^{-1}$\null.  Paromo'iwc, <o k'wdikac:
{\tt\$\lb\\bf x\rb =  \lb\\bf A\rb \\times \lb\\bf B\rb\^{}\lb-1\rb\$},
m~ac d'inei:  ${\bf x} = {\bf A} \times {\bf B}^{-1}$.\TeXref{164--165}

Gi`a t`hn stoiqeiojes'ia majhmatik~wn t`o {\rm\TeX} m~ac d'inei m'ia
>ak'oma grammatoseir`a m`e kalligrafik`a kefala~ia <'opwc ${\cal A},
\ldots, {\cal Z}$\null. T`hn grammatoseir`a t~wn kalligrafik~wn
kefala'iwn mporo~ume n`a t`hn qrhsimopoi'hsoume {\tengs m'onon\/} gi`a
t`hn stoiqeiojes'ia majhmatik~wn (dhl.\ >ent`oc {\tt\$$\ldots$\$}
>`h >ent`oc {\tt\$\$$\ldots$\$\$}) ka`i t`hn kalo~ume m`e t`hn l'exh
>el'egqou {\tt\\cal}.  >'Etsi m`e {\tt\$\lb\\cal N\rb = 0\rb\$},
lamb'anoume: ${\cal N} = 0$.
\toindex{cal}

\exercise Stoiqeiojet~hste: $\rm Si + C \to SiC$\null. (<Up'odeixh:
qrhsimopoi~hste t`hn l'exh >el'egqou {\tt\\to} gi`a n`a l'abete t`o
b'eloc pr`oc t`a dexi'a.)
\toindex{to}

\exercise Stoiqeiojet~hste: {\rm The number sets are: $\bf N
\in Q \in R \in C$.}

\exercise Stoiqeiojet~hste: {\rm The Laplace transform of a constant $c$
is ${\cal L} (c) = c / s$.}


\subsection{Kl'asmata}

<Up'arqoun d'uo tr'opoi gi`a n`a stoiqeiojet'hsoume <'ena kl'asma:
e>'ite <wc $1/2$ e>'ite <wc ${1\over2}$\null.  St`hn pr'wth per'iptwsh
d`en qrei'azontai e>idik`ec >akolouj'iec >el'egqou; >arke~i n`a
gr'ayoume {\tt \$1/2\$}\null.  St`hn de'uterh per'iptwsh <'omwc
qrhsimopoie~itai <h l'exh >el'egqou {\tt \\over}: {\tt\lb $<${\tengs
>arijmht`hc\/}$>$ \\over $<${\tengs paronomast'hc\/}$>$\rb}\null.
Gr'afontac loip`on {\tt \$\$\lb a+b \\over c+d\rb.\$\$} lamb'anoume:%
\TeXref{139--140}
\toindex{over}
$${a+b\over c+d}.$$

\exercise Stoiqeiojet~hste t`o >ak'oloujo: ${a+b\over c}\quad {a\over
b+c} \quad {1\over a+b+c} \not= {1\over a}+{1\over b}+{1\over c}$.

\exercise Stoiqeiojet~hste: {\rm What are the points where ${\partial
\over \partial x} f(x,y) = {\partial \over \partial y} f(x,y) = 0$?}
      
\subsection{De~iktec ka`i >ekj'etec}

O<i de~iktec ka`i o<i >ekj'etec e>~inai e>'ukolo n`a stoiqeiojethjo~un
m`e t`o {\rm \TeX}\null.  O<i qarakt~hrec {\tt \_{}} (<upogr'ammish)
ka`i {\tt \^{}} (gallik`h perispwm'enh) qrhsimopoio~untai gi`a n`a
dhl'wsoun <'oti <o <ep'omenoc qarakt'hrac e>~inai de'ikthc >`h >ekj'ethc
>ant'istoiqa. >'Etsi <o k'wdikac {\tt \$x\^{}2\$} d'inei $x^2$, ka`i <o
k'wdikac {\tt \$x\_{}2\$}, $x_2$\null.   Gi`a n`a l'aboume
periss'oterouc >ap`o <'enan qarakt'hra m`e t`hn morf`h de'ikth >`h
>ekj'eth, >arke~i n`a to`uc kle'isoume m`e >'agkistra s`e <'ena s'unolo.
P.q., m`e t`on k'wdika {\tt \$x\^{}\lb 21\rb\$} lamb'anoume $x^{21}$
ka`i m`e {\tt \$x\_{}\lb 21\rb\$}, $x_{21}$\null. >`Ac pros'exoume <'oti
o<i de~iktec ka`i o<i >ekj'etec stoiqeiojeto~untai a>ut'omata s`e
t'upouc mikr'oterou meg'ejouc.  <H kat'astash g'inetai el'aqista pi`o
pol'uplokh <'otan pr'okeitai gi`a de'ikth to~u de'ikth >`h >ekj'eth to~u
>ekj'eth, k.lp.  D`en mporo~ume n`a gr'ayoume {\tt \$x\_{}2\_{}3\$}
>epeid`h k'ati t'etoio mpore~i n`a >'eqei dipl`h shmas'ia, dhl.\ {\tt
\$x\_{}\lb 2\_{}3\rb\$} >`h  {\tt \$\lb x\_{}2\rb\_{}3\$}, m`e d'uo     
diaforetik`a >apotel'esmata: $x_{2_3}$ ka`i ${x_2}_3$, >ek t~wn <opo'iwn
t`o pr~wto e>~inai <o pi`o koin`oc majhmatik`oc sumbolism'oc. Gi`a t`on
l'ogo a>ut'o, e>~inai sk'opimo n`a qrhsimopoio~ume >'agkistra gi`a n`a
perigr'ayoume pollapl'a (<'osa j'eloume) >ep'ipeda deikt~wn ka`i
>ekjet~wn.%
\TeXref{128--130}

Gi`a n`a j'esoume de~iktec ka`i >ekj'etec st`o >'idio s'umbolo,
qrhsimopoio~ume t`hn {\tt \_{}} ka`i t`hn {\tt \^{}} m`e <opoiad'hpote
seir'a.  >'Etsi e>'ite m`e {\tt \$x\_{}2\^{}1\$} e>'ite m`e
{\tt\$x\^{}1\_{}2\$}, lamb'anoume $x_2^1$.

\exercise Stoiqeiojet~hste t`a <ep'omena: $e^x \quad e^{-x} \quad  
e^{i\pi}+1=0 \quad x_0 \quad x_0^2 \quad {x_0}^2 \quad 2^{x^x}$.

\exercise Stoiqeiojet~hste:  $\nabla^2 f(x,y) = {\partial^2 f
\over\partial x^2} + {\partial^2 f \over\partial y^2}$.
\bigskip

Par'omoia stoiqeiojeto~untai o<i seir'ec (>ajro'ismata) ka`i t`a
<oloklhr'wmata.  <O k'wdikac {\tt \$\\sum\_{}\lb k=1\rb\^{}n k\^{}2\$}
j`a d'wsei $\sum_{k=1}^n k^2$, ka`i <o k'wdikac {\tt \$\\int\_{}0\^{}x
f(t) dt\$}, $\int_0^x f(t) dt$.%
\TeXref{144--145}
\toindex{sum}
\toindex{int}

M'ia >ak'oma par'omoia >efarmog`h e>~inai ka`i <h stoiqeiojes'ia
majhmatik~wn >ekfr'asewn m`e <'oria.  Mporo~ume n`a gr'ayoume t`on
k'wdika {\tt \$\\lim\_\lb n\\to \\infty\rb (\lb n+1 \\over n\rb)\^{}n =
e\$}, gi`a n`a l'aboume $\lim_{n\to \infty} ({n+1\over n})^n = e$.
\toindex{lim}

\exercise Stoiqeiojet~hste t`hn >ak'oloujh >ex'iswsh: $\lim_{x\to 0}
(1+x)^{1\over x}=e$.

\exercise Stoiqeiojet~hste: {\rm The cardinality of $(-\infty, \infty)$
is $\aleph_1$.   }

\exercise Stoiqeiojet~hste: $\lim_{x\to {0^+}} x^x = 1$.
\bigskip

<Or'iste ka`i m'ia mikr`h sumboul`h gi`a pi`o >'omorfa <oloklhr'wmata: 
>`ac pros'exoume t`hn diafor`a metax`u to~u  $\int_0^x f(t) dt$ ka`i
to~u $\int_0^x f(t)\, dt$\null.  St`hn de'uterh per'iptwsh <up'arqei
<'ena mikr`o ken`o di'asthma met`a t`o $f(t)$, ka`i >'etsi fa'inetai
kal'utero. <H prosj'hkh to~u diast'hmatoc >'egine gr'afontac {\tt\\,}
met`a t`o {\tt f(t)} st`on k'wdika.                    
\toindex{,}
                    
\exercise Stoiqeiojet~hste t`o >ak'oloujo <olokl'hrwma: $\int_0^1
3x^2\,dx = 1$.

\subsection{R'izec, tetragwnik`ec ka`i >'allec}
                   
<H stoiqeiojes'ia tetragwnik~wn riz~wn g'inetai m`e t`hn l'exh >el'egqou
{\tt \\sqrt\lb$\ldots$\rb}\null. >'Etsi m`e t`on k'wdika {\tt
\$\\sqrt\lb x\^{}2+y\^{}2\rb\$} j`a l'aboume $\sqrt{x^2+y^2}$\null. 
>`Ac pros'exoume <'oti t`o {\rm \TeX} front'izei t`o p~wc j`a mpo~un t`a
s'umbola, poi'o j`a e>~inai t`o <'uyoc ka`i poi'o t`o m~hkoc to~u
riziko~u.  Gi`a kubik`ec >`h >'allou bajmo~u r'izec qrhsimopoio~ume t`ic
l'exeic >el'egqou {\tt \\root} ka`i {\tt \\of}.  Gi`a n`a l'aboume
$\root n \of {1+x^n}$, pr'epei n`a gr'ayoume t`on k'wdika {\tt \$\\root
n \\of \lb1+x\^{}n\rb\$}.%
\TeXref{130--131}
\toindex{root}
\toindex{sqrt}                 

M'ia >enallaktik`h l'ush gi`a e>idik`ec peript'wseic e>~inai ka`i <h
l'exh >el'egqou {\tt \\surd}; gr'afo\-ntac {\tt \$\\surd 2\$} j`a
l'aboume $\surd 2$.
\toindex{surd}

\exercise Stoiqeiojet~hste t`a >ak'olouja: $\sqrt2 \quad \sqrt {x+y\over
x-y} \quad \root 3 \of {10}$ \quad $e^{\sqrt x}$.

\exercise Stoiqeiojet~hste: $\|x\| = \sqrt{x\cdot x}$.

\exercise Stoiqeiojet~hste: $\phi(t) = {1 \over \sqrt{2\pi}} \int_0^t
e^{-x^2/2}\,dx$.

\subsection{Gramm'ec, p'anw ka`i k'atw}
                 
Gi`a n`a j'esoume <oriz'ontiec gramm`ec >ep'anw >`h k'atw >ap`o 
majhmatik`a s'umbola, gr'afoume t`on k'wdika {\tt
\\overline\lb$\ldots$\rb} >`h {\tt \\underline\lb$\ldots$\rb}
>ant'istoiqa. Kat`a t`on tr'opo a>ut'o, m`e t`on k'wdika {\tt
\$\\overline\lb x+y\rb=\\overline x + \\overline y\$} lamb'anoume
$\overline{x+y}=\overline x + \overline y$\null. >All`a >`ac
parathr'hsoume <'oti o<i gramm`ec >ep'anw >ap`o t`a s'umbola e>~inai s`e
diaforetik`a <'uyh, gi''\NB{}a>ut`o qrei'azetai l'igh prosoq'h. 
Gr'afontac {\tt \\overline\lb\\strut x\rb} <h <oriz'ontia gramm`h
>ep'anw >ap`o t`o $x$ j`a metakinhje~i >ak'oma l'igo pi`o
p'anw.\TeXref{130--131}\toindex{overline}\toindex{underline} <O
>ant'istoiqoc k'w\-di\-kac gi`a <upogr'ammish m`h majh\-mati\-ko~u
keim'enou e>~inai: {\tt \\underbar\lb$\dots$\rb}.
\toindex{underbar}

\exercise Stoiqeiojet~hste t`a >ak'olouja: $\underline x \quad \overline
y \quad \underline{\overline{x+y}}$.

\subsection{<Oroj'etec, mikro`i ka`i meg'aloi}

O<i pi`o koino`i <oroj'etec po`u qrhsimopoio~untai st`a majhmatik`a
e>~inai o<i parenj'eseic, o<i >agk'ulec ka`i t`a >'agkistra.  <'Opwc
>'eqei >'hdh >anaferje~i, gr'afontac st`on k'wdika {\tt [ ] \\\lb\
\\\rb\ ( )} lamb'anoume to`uc <oroj'etec: $[\>]\>\{\>\}\>(\>)\>$.  
Merik`ec for'ec, m`e <oroj'etec megal'uterou meg'ejouc belti'wnetai <h
>anagnwsim'othta t~wn majhmatik~wn, <'opwc p.q.
$$\bigl(a\times(b+c)\bigr) \bigl((a\times b)+c\bigr).$$

Gi`a megal'uterouc >aristero`uc <oroj'etec pr'epei n`a qrhsimopoi'hsoume
t`ic l'exeic >el'egqou {\tt \\bigl}, {\tt \\Bigl}, {\tt \\biggl} ka`i
{\tt  \\Biggl} >empr`oc >ap`o a>uto`uc. Par'omoia, m`e t`ic l'exeic
>el'egqou {\tt \\bigr}, {\tt \\Bigr}, {\tt \\biggr} ka`i {\tt \\Biggr}
lamb'anoume meg'alouc dexio`uc <oroj'etec. >'Etsi m`e t`on k'wdika {\tt
\$\\Bigl[ \\ldots \\Bigr]\$} pa'irnoume $\Bigl[ \ldots
\Bigr]$\null.\TeXref{145--147} <Or'iste ka`i <'enac p'inakac m`e t`a
di'afora meg'ejh <orojet~wn.
\toindex{bigl}
\toindex{Bigl}
\toindex{biggl}
\toindex{Biggl}
\toindex{bigr}
\toindex{Bigr}
\toindex{biggr}
\toindex{Biggr}

%% \everycr can add 4 points between lines in the following table %%
\everycr={\noalign{\vskip 4 pt}}
\maketable [<Oroj'etec diaf'orwn megej~wn]
\halign{
\strut \hfil$#$ & \quad \tt# \hfil \quad\qquad
      &\hfil$#$ & \quad \tt# \hfil \quad\qquad
      &\hfil$#$ & \quad \tt# \hfil \quad\qquad
      &\hfil$#$ & \quad \tt# \hfil \cr
\noalign{\hrule} \noalign{\smallskip}
\{ & \\\lb & \} & \\\rb & ( & ( & ) & )\cr
\bigl\{ & \\bigl\\\lb & \bigr\} & \\bigr\\\rb & \bigl( & \\bigl( & \bigr) &
\\bigr)\cr
\Bigl\{ & \\Bigl\\\lb & \Bigr\} & \\Bigr\\\rb & \Bigl( & \\Bigl( & \Bigr) &
\\Bigr)\cr
\biggl\{ & \\biggl\\\lb & \biggr\} & \\biggr\\\rb & \biggl(
         & \\biggl( & \biggr) & \\biggr) \cr
\Biggl\{ & \\Biggl\\\lb & \Biggr\} & \\Biggr\\\rb & \Biggl(
         & \\Biggl( & \Biggr) & \\Biggr)\cr
       }

\everycr={} %%% leave a blank line always above this command! (DF)

Mporo~ume >ak'oma n`a >af'hsoume t`o {\rm \TeX} n`a >apofas'isei m'ono
tou t`o m'egejoc t~wn <orojet~wn gr'afontac{\tt \\left} ka`i {\tt
\\right} >empr`oc >ap`o a>uto'uc.\TeXref{148} >'Etsi <o k'wdikac {\tt
\\left[$\ldots$\\right]} >'eqei <wc >apot'elesma o<i perieq'omenoi
t'upoi n`a perikle'iontai >ap`o >agk'ulec kat'allhlou meg'ejouc. 
{\tengb Prosoq'h:} gi`a k'aje {\tt \\left} pr'epei n`a <up'arqei ka`i
t`o >ant'istoiqo {\tt \\right} (>'estw ki >`an <o >ari\-ste\-r`oc
<oroj'ethc d`en e>~inai <'omoioc m`e t`on dexi'o).  Gi`a par'adeigma, <o
k'wdikac  {\tt \$\$\\left|\lb a+b \\over c+d\rb\\right|.\$\$} d'inei 
$$\left|{a+b \over c+d}\right|.$$

\maketable [Majhmatiko`i <oroj'etec]
\halign{
\strut \hfill$#$ & \quad \tt # \qquad\qquad &
       \hfill$#$ & \quad \tt # \qquad\qquad &
       \hfill$#$ & \quad \tt #  \cr
\noalign{\hrule} \noalign{\smallskip}
(       & (        & )          & )           & [        & [         \cr
]       & ]        &\{          & \\\lb       & \}       & \\\rb     \cr
\lfloor & \\lfloor &\rfloor     &\\rfloor     & \lceil   & \\lceil   \cr
\rceil  & \\rceil  &\langle     & \\langle    & \rangle  & \\rangle  \cr
/       & /        & \backslash & \\backslash &|         & |         \cr
\|      & \\|      &\uparrow    & \\uparrow   & \Uparrow & \\Uparrow \cr
\downarrow & \\downarrow & \Downarrow & \\Downarrow
                        & \updownarrow & \\updownarrow \cr
\Updownarrow & \\Updownarrow \cr
       }   

\toindex{lfloor}
\toindex{rfloor}
\toindex{lceil}
\toindex{rceil}
\toindex{langle}
\toindex{rangle}
\toindex{|}
\toindex{uparrow}
\toindex{Uparrow}
\toindex{downarrow}
\toindex{Downarrow}
\toindex{updownarrow}
\toindex{Updownarrow}

\exercise Stoiqeiojet~hste $\bigl \lceil \lfloor x \rfloor \bigr \rceil
\leq \bigl \lfloor \lceil x \rceil \bigr \rfloor$.

\subsection{K'apoiec e>idik`ec sunart'hseic}

<Up'arqoun k'apoiec e>idik`ec sunart'hseic po`u parousi'azontai suqn`a
st`a majhmatik'a.  S`e m'ia >ex'iswsh <'opwc (($\sin^2x + \cos^2x = 
1$)) o<i  trigwnometrik`ec sunart'hseic {\rm sin} (<hm'itono) ka`i {\rm
cos} (sunhm'itono) >'eqoun stoiqeiojethje~i <wc ke'imeno l'ogou, dhl.\
>'oqi s`e pl'agia stoiqe~ia.  A>ut`oc e>~inai <o sun'hjhc tr'opoc gi`a
n`a dhlwje~i >ent`oc <en`oc majhmatiko~u t'upou <'oti <up'arqei m'ia
e>idik`h sun'arthsh (p.q., {\rm cos}) ka`i >'oqi t`o gin'omeno tri~wn
metablht~wn (p.q., $ c o s$)\null. O<i l'exeic >el'egqou {\tt \\sin}
ka`i {\tt \\cos} j'etoun a>ut'omata to`uc swsto`uc qarakt~hrec st`on
majhmatik`h >'ekfrash.\TeXref{162}
<Or'iste <'enac p'inakac a>ut~wn ka`i merik~wn >'al\-lwn e>idik~wn
sunart'hsewn:

\maketable [E>idik`ec majhmatik`ec sunart'hseic]
\halign{
   \strut \tt {\\}#\hfil && \quad \tt {\\}#\hfil \cr
   \noalign{\hrule} \noalign{\smallskip}
   sin    & cos    & tan  & cot  & sec  & csc & arcsin & arccos \cr
   arctan & sinh   & cosh & tanh & coth & lim & sup    & inf    \cr
   limsup & liminf & log  & ln   & lg   & exp & det    & deg    \cr
   dim    & hom    & ker  & max  & min  & arg & gcd    & Pr     \cr
       }

\toindex{sin}
\toindex{cos}
\toindex{tan}
\toindex{cot}
\toindex{sec}
\toindex{csc}
\toindex{arcsin}
\toindex{arccos}
\toindex{arctan}
\toindex{sinh}
\toindex{cosh}
\toindex{tanh}
\toindex{coth}
\toindex{lim}
\toindex{sup}
\toindex{inf}
\toindex{limsup}
\toindex{liminf}
\toindex{log}
\toindex{ln}
\toindex{lg}
\toindex{exp}
\toindex{det}
\toindex{deg}
\toindex{dim}
\toindex{hom}
\toindex{ker}
\toindex{max}
\toindex{min}
\toindex{arg}
\toindex{gcd}
\toindex{Pr}

\exercise Stoiqeiojet~hste: $\sin(2\theta) = 2\sin\theta\cos\theta \quad
\cos(2\theta) = 2\cos^2\theta - 1  $.

\exercise Stoiqeiojet~hste: $$\int \csc^2x\, dx = -\cot x+ C \qquad
\lim_{\alpha\to 0} {\sin\alpha \over \alpha} = 1 \qquad \lim_{\alpha\to
\infty} {\sin\alpha \over \alpha} = 0.$$

\exercise Stoiqeiojet~hste: $$\tan(2\theta) = {2\tan\theta \over 
1-\tan^2\theta}.$$          

\subsection{>Ako'usate, >ako'usate!}

<Up'arqei m'ia e>idik`h {\tengs makroentol'h\/} (>`h {\sl macro\/}) <h
<opo~ia e>~inai qr'hsimh gi`a k'aje majhmatik`h dhmos'ieush.  Pr'okeitai
gi`a t`hn makroentol`h {\tt\\proclaim}.  Qrhsimopoie~itai gi`a
jew\-r'h\-mata, sumper'asmata, prot'aseic, k.lp.  <H par'agrafoc met`a 
t`o {\tt \\proclaim} qwr'izetai s`e d'uo m'erh: t`o pr~wto m'eroc
fj'anei ka`i sumperilamb'anei t`hn pr'wth tele'ia st`hn <opo'ia
>akolouje~i ken`o di'asthma; t`o de'utero m'eroc e>~inai t`o <up'oloipo
t~hc paragr'afou.\TeXref{202--203}\ <H >id'ea p'isw >ap`o a>ut`o t`o
t'eqnasma e>~inai <'oti t`o pr~wto m'eroc pr'epei n`a e>~inai k'ati
<'opwc (({\rm Theorem 1.}))\ >`h (({\rm Corollary\NB{}B.}))\null.  T`o
de'utero m'eroc e>~inai t`o perieq'omeno to~u jewr'hmatoc >`h to~u
sumper'asmatoc.  Gi`a par'adeigma, m`e t`on k'wdika: \toindex{proclaim}

\beginliteral
\proclaim Theorem 1 (H.~G.~Wells). In the country of the blind,
the one-eyed man is king.
@endliteral

\noindent lamb'anoume

\proclaim Theorem 1 (H.\NB G.\NB Wells). In the country of the blind,
the one-eyed man is king.

Fusik'a, t`o perieq'omeno to~u jewr'hmatoc mpore~i n`a peri'eqei ka`i
majhmatik`a s'umbola.

\exercise Stoiqeiojet~hste:
\nobreak
\proclaim Theorem (Euclid). There exist an infinite number of
primes.

\exercise Stoiqeiojet~hste:
\nobreak
\proclaim Proposition 1.
$\root n \of {\prod_{i=1}^n X_i} \leq {1 \over n} \sum_{i=1}^n X_i$
with equality if and only if $X_1=\cdots=X_n$.

\subsection{Majhmatik`ec parat'axeic}
          
<H stoiqeiojes'ia majhmatik~wn parat'axewn (pin'akwn, <orizous~wn,
k.lp.)\ g'inetai qrhsimopoi'wntac sunduasmo`uc to~u qa\-ra\-kt'h\-ra
sto'iqishc {\tt \&} ka`i t~hc l'exhc >el'egqou {\tt \\cr}\null.  Gi`a
n`a >arq'isoume loip`on t`hn stoiqeiojes'ia <en`oc majhmatiko~u p'inaka,
gr'a\-foume st`on k'w\-di\-k'a mac: {\tt\$\$\\pmatrix\lb$\dots$\rb\$\$}.
Metax`u t~wn >agk'istrwn mpa'inoun o<i gramm`ec to~u p'inaka, <h
kajem'ia >ek t~wn <opo'iwn telei'wnei m`e {\tt\\cr}.  T`o
perieq'o\-me\-no k'aje st'hlhc qwr'izetai >ap`o t`o perieq'o\-me\-no
t~wn geito\-ni\-k~wn thc sthl~wn (st`hn >'idia gramm'h) m`e t`on
qarakt'hra sto'iqishc {\tt \&}\null. Gi`a n`a g'inoume pi`o katanohto'i,
<or'iste <'ena par'adeigma:%
\TeXref{176--178}

\beginliteral
$$\pmatrix{
a & b & c & d \cr
b & a & c+d & c-d \cr
0 & 0 & a+b & a-b \cr
0 & 0 & ab  & cd \cr
}.$$
@endliteral

\noindent
<O k'wdikac a>ut`oc d'inei
$$\pmatrix{
a & b & c & d \cr
b & a & c+d & c-d \cr
0 & 0 & a+b & a-b \cr
0 & 0 & ab  & cd \cr
}.$$
\toindex{pmatrix}%
>`Ac pros'exoume <'oti k'aje st'hlh to~u parap'anw p'inaka e>~inai
kentrwm'enh m`e l'igo di'asthma dexi`a ka`i >arister'a. E>~inai dunat`h
<h dexi`a >`h <h >arister`h sto'iqish t~wn sthl~wn qrhsimopoi'wntac
kat'allhla t`o {\tt \\hfill}\null, <'opwc, p.q., st`o <ep'omeno
par'adeigma (prosoq`h st`hn diafor`a m`e t`o prohgo'umeno):

\beginliteral
$$\pmatrix{
a & b & c \hfill  & \hfill d  \cr
b & a & c+d      & c-d      \cr
0 & 0 & a+b      & a-b      \cr
0 & 0 & ab \hfill & \hfill cd \cr
}.$$
@endliteral

\noindent <O parap'anw k'wdikac d'inei

$$\pmatrix{
a & b & c \hfill & \hfill d   \cr
b & a & c+d     & c-d       \cr
0 & 0 & a+b     & a-b       \cr
0 & 0 & ab \hfill & \hfill cd \cr
}.$$

\vbox{
\exercise Stoiqeiojet~hste
$$ I_4 = \pmatrix{ 1 &0 &0 &0 \cr
                   0 &1 &0 &0 \cr
                   0 &0 &1 &0 \cr
                   0 &0 &0 &1 \cr}$$
}

E>~inai dunat`o n`a stoiqeiojet'hsoume majhmatik`ec parat'axeic m`e
diaforetiko`uc <oroj'etec.  Qrhsimo\-poi\-'wntac {\tt \\matrix} >ant`i
to~u {\tt \\pmatrix}, >afairo~untai o<i parenj'eseic to~u p'inaka, ka`i
>'etsi mporo~ume n`a j'esoume <'opoion <oroj'eth j'eloume m`e  t`o {\tt
\\left} ka`i t`o {\tt \\right}.  <Or'iste p~wc mporo~ume n`a l'aboume
t`hn <or'izousa to~u p'inaka to~u pr'wtou parade'igmatoc.
\toindex{matrix}
\toindex{left}
\toindex{right}

\beginliteral
$$ \left |
\matrix{
a & b & c & d \cr
b & a & c+d & c-d \cr
0 & 0 & a+b & a-b \cr
0 & 0 & ab  & cd \cr
}
\right | $$
@endliteral

\noindent A>ut`oc <o k'wdikac d'inei

$$ \left |
\matrix{
a & b & c & d \cr
b & a & c+d & c-d \cr
0 & 0 & a+b & a-b \cr
0 & 0 & ab  & cd \cr
}
\right | $$

Mporo~ume >ak'oma  n`a gr'ayoume {\tt \\left.}\ >`h/ka`i {\tt \\right.}\
gi`a n`a dhl'wsoume <'oti <o >ari\-ste\-r`oc >`h/ka`i <o dexi`oc
<oroj'ethc parale'ipetai (prosoq`h st`hn tele'ia po`u pr'epei n`a
qrh\-si\-mo\-poi'hsoume).

\exercise Qrhsimopoi~hste t`ic >entol`ec stoiqeiojes'iac pin'akwn to~u
{\rm \TeX} gi`a t`hn parak'atw >ex'iswsh
$$ |x| = \left\{ \matrix{ x & x \ge 0 \cr
                         -x & x \le 0 \cr} \right.$$

\noindent <H >ex'iswsh t~hc parap'anw >'askhshc, >all`a ka`i >'alloi
par'omoioi t'upoi mporo~un n`a stoiqeiojethjo~un >ep'ishc m`e t`hn
makroentol`h {\tt \\cases}.\TeXref{175} P.q., <o k'wdikac

\beginliteral@obeyspaces
$$ \delta (x) = \cases{ \infty, & if $ x = 0 $; \cr
@                             0, & otherwise. \cr }$$
@endliteral

\noindent
d'inei:
$$ \delta (x) = \cases{ \infty, & {\rm if $ x = 0 $;} \cr
                             0, & {\rm otherwise.} \cr }$$

Merik`ec for`ec sto`uc p'inakec b'azoume merik`ec suneq'omenec tele~iec
<wc >'endeixh para\-lei\-po\-m'enwn sumb'olwn.  O<i >akolouj'iec
>el'egqou {\tt\\cdots}, {\tt\\vdots} ka`i {\tt\\ddots}
qrhsi\-mo\-poio~u\-ntai gi`a >'enjesh <orizont'iwn, kaj'etwn ka`i
diagwn'iwn telei~wn >ant'istoiqa.  >'Etsi mporo~ume n`a gr'ayoume

\beginliteral
$$ \left [
\matrix{
aa     & \cdots & az     \cr
\vdots & \ddots & \vdots \cr
za     & \cdots & zz     \cr }
\right ] $$
@endliteral

\noindent gi`a n`a l'aboume
$$ \left [
\matrix{
aa     & \cdots & az     \cr
\vdots & \ddots & \vdots \cr
za     & \cdots & zz     \cr }
\right ] $$

Mporo~ume n`a stoiqeiojet'hsoume p'inakec ka`i >ent`oc st'iqou, >all`a
t`o >apot'elesma j`a e>~inai m'allon >'asqhmo.

\subsection{Diakrit`ec kentrwm'enec >exis'wseic}
\nobreak

M'eqri t'wra, <'ola <'osa >anaf'erjhkan sqetik`a m`e t`hn stoiqeiojes'ia
majhmatik~wn >efar\-m'o\-zo\-ntai t'oso gi`a >ekfr'aseic >ent`oc st'iqou <'oso
ka`i gi`a >ekfr'aseic diakrit`ec kentrwm'enec.  >Ed~w j`a m'ajoume
merik`a pr'agmata po`u >efarm'ozontai m'onon s`e diakrit`ec >exis'wseic.

T`o pr~wto pr'agma e>~inai <h sto'iqish diakrit~wn majhmatik~wn
>ekfr'asewn (p.q., >exis'wsewn) po`u katalamb'anoun poll`ec >ar'adec. 
A>ut`o g'inetai m`e t`on qarakt'hra sto'iqishc {\tt \&} ka`i t`ic
>akolouj'iec >el'egqou {\tt \\cr} ka`i {\tt \\eqalign}.  Xeki\-n'w\-ntac
m`e {\tt \$\$\\eqalign\lb$\dots$\rb\$\$}, gr'afoume t`ic >exis'wseic
po`u pr'okeitai n`a stoiqhjo~un <olo\-klh\-r'wno\-ntac t`hn k'aje m'ia
m`e {\tt \\cr}.  S`e k'aje m'ia >ex'iswsh pr'epei n`a mpa'inei ka`i
<'ena {\tt \&} gi`a n`a dhl'wnetai <h sto'iqish.  Sun'hjwc <h sto'iqish
g'inetai s`e s'umbola >is'othtac, par'' <'oti a>ut`o d`en e>~inai p'anta
<o kan'onac.  Gi`a par'adeigma, gr'afontac%
\TeXref{190--192}
\toindex{eqalign}

\beginliteral
$$\eqalign{
a+b &= c+d \cr
x &= w + y + z \cr
m + n + o + p &= q \cr
}$$
@endliteral

\noindent lamb'anoume
$$\eqalign{
a+b &= c+d \cr
x &= w + y + z \cr
m + n + o + p &= q \cr
}$$

T`ic diakrit`ec >exis'wseic mporo~ume n`a t`ic >arijm'hsoume st`o
dexi`o >`h st`o >arister`o perij'wri'o touc.  >E`an gr'ayoume
{\tt\\eqno} s`e m'ia >ex'iswsh >ent`oc plais'iou st`on k'wdik'a mac,
t'ote <'o,ti br'isketai met`a >ap`o a>ut`h t`hn l'exh >el'egqou
metatop'izetai pr`oc t`o dexi`o perij'wrio. >'Etsi m`e {\tt \$\$ x+y=z.\
\\eqno (1)\$\$}, t`o >apot'elesma e>~inai
$$ x+y=z. \eqno (1)$$
Gi`a n`a >arijm'hsoume m'ia >ex'iswsh st`o >arister`o perij'wrio
gr'afoume {\tt \\leqno} >ant`i to~u {\tt \\eqno}.
\toindex{eqno}
\toindex{leqno}

E>~inai >ep'ishc dunat`o n`a >arijm'hsoume stoiqism'enec >exis'wseic
qrhsimopoi'wntac t`hn l'e\-xh >el'egqou {\tt \\eqalignno}.  <O qarakt'hrac
>el'egqou {\tt \&} qrhsimopoie~itai gi`a n`a xeqwr'isei <o >arij\-m`oc
t~hc >ex'iswshc >ap`o t`hn >'idia t`hn >ex'iswsh. P.q.\ m`e

\beginliteral
$$\eqalignno{
a+b &= c+d & (1) \cr
x &= w + y + z \cr
m + n + o + p &= q & * \cr
}$$
@endliteral

\noindent lamb'anoume
$$\eqalignno{
a+b &= c+d & (1) \cr
x &= w + y + z \cr
m + n + o + p &= q & * \cr
}$$

>Ant'istoiqa, qrhsimopoio~ume {\tt \\leqalignno} gi`a n`a j'esoume
>arijmo`uc >exis'wsewn st`o >arister`o peri\-j'wrio stoiqism'enwn
>exis'wsewn.%
\TeXref{192--193}
\toindex{eqalignno}
\toindex{leqalignno}

T'eloc, >`ac <upoj'esoume <'oti j'eloume n`a e>is'agoume m'eroc mikro~u
keim'enou st`o >endi'ameso m'iac diakrit~hc >ex'iswshc.  A>ut`o
mporo~ume n`a t`o pet'uqoume j'etontac t`o ke'imeno s`e <'ena {\rm
hbox}. >Ak'oma mporo~ume n`a qrhsimopoi'hsoume t`o {\rm hbox} <'wste n`a
j'esoume ka`i ken`a dia\-st'h\-mata metax`u l'exewn >`h/ka`i majhmatik~wn
sumb'olwn (>`ac jumhjo~ume <'oti <'ola t`a diast'hmata >agno\-o~untai
st`hn stoiqei\-ojes'ia majhma\-tik~wn t'upwn). Gr'a\-fo\-ntac loip`on
t`on k'wdika {\tt \$\$X=Y \\hbox\lb{} if and only if \rb x=y.\$\$} j`a
l'aboume
$$X=Y \hbox{ \rm if and only if } x=y.$$
>Ax'izei n`a pros'exoume t`a ken`a diast'hmata st`o {\rm hbox}.

\exercise Prospaj~hste n`a k'anete merik`a >ap`o t`a d'uskola
probl'hmata t~wn sel'idwn 180--181 to~u {\sl \TeX book}.


\section{Stoiqhje~ite!}        

>Arket`ec for`ec j`a qreiasje~i n`a fti'axoume <'enan m`h majhmatik`o
p'inaka st`o >'entup'o mac.  E>utuq~wc t`o {\rm \TeX} mpore~i n`a
stoiqeiojet'hsei m`h majhmatiko`uc p'inakec pol`u e>'ukola ka`i m'alista
kat`a d'uo tr'opouc.  <O pr~wtoc tr'opoc e>~inai m`e t`o {\tengs
perib'allon} {\rm tabbing} (pinakopo'ihsh).  Gi`a <'osouc gnwr'izoun
>ap`o grafomhqan'h, t`o {\rm tabbing}  e>~inai k'ati <'omoio m`e t`hn
r'ujmish t~wn meg'alwn diasthm'atwn t~hc grafomhqan~hc, t~wn {\rm
TAB}\null.  K'aje gramm`h >epexerg'azetai xeqwrist'a (kat`a tr'opo pol`u
kal'utero >ap`o a>ut`on t~hc grafomhqan~hc), >an'aloga m`e t`ic j'eseic
sto'iqishc t~wn sthl~wn <'opwc a>ut`ec >'eqoun kajorisje~i m`e t`o {\rm
tab}.  <O de'uteroc tr'opoc  e>~inai m`e t`o {\tengs perib'allon} {\rm
alignment} (e>ujugr'ammish) m`e t`o <opo~io <'oloc <o p'inakac
stoiqeiojete~itai <wc m'ia <en'othta kat`a m'ia prokajorism'enh morf'h.

\subsection{Qrhsimopoi~hste  t`o {\lsecfont TAB}}         
        
Gi`a n`a <etoim'asoume <'enan p'inaka m`e t`o perib'allon {\rm tabbing},
j`a pr'epei pr~wta n`a <or'isoume t`a shme~ia sto'iqishc, dhl.\ t`hn
j'esh k'aje st'hlhc, m`e t`hn l'exh >el'egqou {\tt \\settabs}.         
>Ef'' <'oson k'anoume a>ut'o, gr'afoume st`on k'wdik'a mac k'aje gramm`h
to~u p'inak'a mac xekin'wntac m`e t`o s'umbolo >el'egqou {\tt \\+} ka`i
telei'wnontac m`e {\tt \\cr}.  <'Opwc >'eqoume de~i >epaneilhm\-m'ena,
t`a ke\-n`a diast'hmata st`on k'wdika d`en >epidro~un st`hn telik`h
morf`h to~u >ent'upou.
\toindex{settabs}

<H <aplo'usterh qr'hsh to~u {\tt \\settabs} e>~inai gi`a n`a fti'axoume
<'enan p'inaka m`e st~hlec >'idiou pl'atouc.\TeXref{231} M`e {\tt
\\settabs 5 \\columns} j`a l'aboume <'enan p'inaka p'ente sthl~wn
>'idiou pl'atouc.  <H metap'hdhsh >ap`o m'ia st'hlh st`hn >'allh
g'inetai m`e t`on qarakt'hra sto'iqishc {\tt\&}\null.  >'Etsi, gi`a
par'adeigma, <o k'wdikac \toindex{columns}

%%% EU countries, 1997
\beginliteral
\settabs 5 \columns
\+ Austria & Finland & Greece & Luxemburg & Spain \cr
\+ Belgium & France & Ireland & The Netherlands & Sweden \cr
\+ Denmark & Germany & Italy & Portugal & United Kingdom \cr
@endliteral

\noindent
m~ac d'inei
\vskip\baselineskip

{\rm % begin roman
\settabs 5 \columns
\+ Austria & Finland & Greece & Luxemburg & Spain \cr
\+ Belgium & France & Ireland & The Netherlands & Sweden \cr
\+ Denmark & Germany & Italy & Portugal & United Kingdom \cr
} % end roman

D`en e>~inai >apara'ithto n`a <up'arqei k'ati metax`u k'aje sumb'olou
sto'i\-qi\-shc; k'apoiec st~hlec mporo~ume n`a t`ic >af'hsoume ken'ec. 
Gi`a n`a fti'axoume t`on >'idio p'inaka m`e <'exi st~hlec, d`en
qrei'azetai par`a n`a gr'ayoume {\tt \\settabs 6 \\columns}; kat`a
a>ut`on t`on  tr'opo t`o prohgo'umeno par'adeigma g'inetai:        
\vskip\baselineskip

{\rm % begin roman
\settabs 6 \columns
\+ Austria & Finland & Greece & Luxemburg & Spain \cr
\+ Belgium & France & Ireland & The Netherlands & Sweden \cr
\+ Denmark & Germany & Italy & Portugal & United Kingdom \cr
} % end roman

St`o teleuta~io par'adeigma o<i st~hlec >'eqoun mikr'otero pl'atoc.
>Ep'ishc <up'arqoun ka`i d'uo ped'ia to~u p'inaka\fnote{<O <'oroc
{\tengs ped'io to~u p'inaka\/} shma'inei <otid'hpote st`on k'wdika
perilamb'anetai metax`u d'uo diadoqik~wn {\tt \& $\ldots$ \&}, metax`u
{\tt \\+ $\ldots$ \&} >`h metax`u {\tt \& $\ldots$ \\cr}.} po`u
>al\-lh\-loepi\-kal'u\-ptontai st`hn teleuta'ia gramm'h tou. A>ut`o
sum\-ba'inei giat`i t`o {\rm \TeX}, s`e >ant'ijesh m`e m'ia koin`h
grafomhqan'h, metakine~i k'aje stoiqe~io to~u p'inaka st`hn <ep'omenh
gramm`h sto'iqishc, >'estw ki >`an a>ut`o shma'inei k'inhsh pr`oc t`a
p'isw ka`i >allhloepik'aluyh k'apoiwn stoiqe'iwn.

<Up'arqei m'ia >endiaf'erousa sq'esh metax`u t~hc >'ennoiac to~u
sun'olou ka`i to~u perib'al\-lon\-toc {\rm tabbing}. P.q., o<i tim`ec
{\tt\\settabs} >'eqoun >isq`u m'onon >ent`oc to~u sun'olou <'opou
<or'izontai.  Kat`a sun'epeia, e>~inai dunat`o n`a >all'axoume
proswrin`a t`ic j'eseic sto'iqishc (dhl., t`o {\tt\\settabs})
dhmiourg'wntac <'ena s'unolo m`e t`hn qr'hsh >agk'istrwn. >Epipl'eon,
k'aje ped'io to~u p'inaka >apo\-te\-le~i <'ena a>utotel`ec s'unolo.
Mporo~ume loip`on n`a stoiqeiojet'hsoume <'ena ped'io to~u p'inaka
m`e >'entonouc t'upouc qrhsimopoi'wntac t`hn >entol`h {\tt \\bf} qwr`ic
>'agkistra.  Ka`i k'aje st'hlh >ekt`oc t~hc teleuta'iac mporo~ume n`a
t`hn stoiq'isoume st`o k'entro, >arister`a >`h dexi'a, >`h >ak'oma n`a
t`hn gem'isoume m`e m'ia gramm`h >`h m`e tele~iec (<wc        
>aposiwphtik'a).  >Ex <orismo~u, t`o {\rm\TeX} j'etei st`o t'eloc k'aje
ped'iou to~u p'inaka <'ena {\tt \\hfil}, >'etsi <'wste <ola t`a
ped'ia n`a stoiq'izontai >arister'a, <'opwc sumba'inei ka`i m`e t`hn
>entol`h {\tt \\line}.  Prosj'etontac <'ena {\tt \\hfil} pr`in >ap`o
<'ena stoiqe~io to~u p'inaka, t`o >apot'elesma j`a e>~inai t`o stoiqe~io
a>ut`o n`a metakinhje~i st`o k'entro t~hc st'hlhc.  Prosj'etontac <'omwc
<'ena {\tt \\hfill} >ant`i to~u {\tt \\hfil}, t`o >apot'elesma j`a
e>~inai t`o stoiqe~io n`a metakinhje~i st`hn dexi`a >'akrh t~hc st'hlhc.
(T`o {\tt \\hfill} >'eqei t`hn >'idia leitourg'ia m`e t`o {\tt \\hfil},
dhl.\ ka`i t`a d'uo d'inoun >epipl'eon ken`o di'asthma; m'ono po`u
<'otan >emfan'izontai ka`i t`a d'uo maz'i, t`o {\tt \\hfill} >'eqei
proterai'othta.)
\toindex{hfill}

%%% EU countries which meet (more-or-less) the Maastricht criteria
%%% for monetary union (excluded: Belgium, Italy, Greece).
%%% Luxemburg has been set in bold, 'cause it's the only one that
%%% really meets these criteria (data 1997).
\beginliteral @obeyspaces
\settabs 5 \columns
\+ \hfil Austria \hfil & \hfill Finland \quad & \dotfill
@                                  & \bf Luxemburg   & Spain \cr
\+ \hfil --- \hfil     & \hfill France  \quad & Ireland
@                                  & The Netherlands & Sweden \cr
\+ \hfil Denmark \hfil & \hfill Germany \quad & \hrulefill & Portugal 
@                                  & Portugal        & United Kingdom \cr
@endliteral

\noindent
T`o parap'anw par'adeigma j`a d'wsei <'enan p'inaka m`e t`hn pr'wth     
st'hlh kentrwm'enh, t`hn de'uterh stoiqism'enh dexi'a (m`e        
k'apoio ken`o di'asthma >ap`o t`o {\tt\\quad}), ka`i <'ena ped'io
({\rm Luxemburg}) s`e >'entonouc t'upouc.  O<i l'exeic >el'egqou
{\tt\\dotfill} kai {\tt\\hrulefill} d'inoun >ep'ishc k'apoia
>enal\-la\-ktik`a ped'ia st`on p'inak'a mac.
\toindex{dotfill}
\toindex{hrulefill}
\vskip\baselineskip        
        
{\rm % begin roman
\settabs 5 \columns
\+ \hfil Austria \hfil & \hfill Finland \quad & \dotfill 
                                   & \bf Luxemburg   & Spain \cr
\+ \hfil --- \hfil     & \hfill France  \quad & Ireland 
                                   & The Netherlands & Sweden \cr
\+ \hfil Denmark \hfil & \hfill Germany \quad & \hrulefill
                                   & Portugal        & United Kingdom \cr
}% end roman

\exercise J'este t`a ped'ia to~u parap'anw p'inaka st`o k'entro k'aje
st'hlhc.

T`ic st~hlec to~u p'inaka mporo~ume n`a t`ic k'anoume n`a >'eqoun ka`i
diaforetik`o pl'atoc <h m'ia >ap`o t`hn >'allh.  A>ut`o g'inetai 
qrhsimopoi'wntac m'ia {\tengs gramm`h--de~igma} st`on k'wdik'a mac
<'opwc: {\tt \\settabs \\+ $\ldots$ \& $\ldots$ \& $\ldots$ \\cr}.  T`o
di'asthma metax`u t~wn qarakt'hrwn sto'iqishc {\tt \&} kajor'izei ka`i
to pl'atoc t~wn sthl~wn.  Gi`a par'adeigma, m`e t`on k'wdika {\tt
\\settabs \\+ \\hskip 1 in \& \\hskip 2 in \& \\hskip 1,5 in \& \\cr}
j`a l'aboume t`hn >arq`h t~hc pr'wthc st'hlhc s`e >ap'ostash m'iac
>'intsac >ap`o t`o >arister`o perij'wrio, t`hn >arq`h t~hc <ep'omenhc
st'hlhc d'uo >'intsec dexi'wtera t~hc pr'wthc ka`i t`hn tr'ith 1,5
>'intsec >ak'oma pi`o dexi'a. E>~inai >ep'ishc dunat`o n`a
qrhsimopoi'hsoume k'apoio {\tengs ke'imeno--de~igma} gi`a n`a
kajor'isoume t`o pl'atoc k'aje st'hlhc. >'Etsi, p.q., m'ia
pijan`h gramm`h--de~igma j`a >~htan <h >ak'oloujh: {\tt \\settabs \\+
\\quad Country \\quad \& \\quad Population \\quad \& \\quad Area \\quad
\& \\cr}.  Kat'' a>ut'on t`on tr'opo <h k'aje st'hlh j`a >'eqei pl'atoc
>'iso m`e t`o pl'atoc to~u t'itlou thc (dhl.\ {\rm Country}, k.lp.)\
ka`i <'ena ken`o di'a\-sth\-ma pl'atouc <en`oc tetrag'wnou <ekat'erwjen to~u
t'itlou.  <Or'iste <'ena pi`o pl~hrec par'a\-deigma:

\beginuser \obeyspaces
\\settabs \\+ \\quad Year \\quad \& \\quad Price \\quad
\                                            \& \\quad Dividend \& \\cr
\\+ \\hfill Year \\quad \& \\quad Price  \\quad \& \\quad Dividend \\cr
\\+ \\hfill 1971 \\quad \& \\quad 41--54 \\quad \& \\qquad \\\$2.60   \\cr
\\+ \\hfill 2    \\quad \& \\quad 41--54 \\quad \& \\qquad \\\$2.70 \\cr
\\+ \\hfill 3    \\quad \& \\quad 46--55 \\quad \& \\qquad \\\$2.87 \\cr
\\+ \\hfill 4    \\quad \& \\quad 40--53 \\quad \& \\qquad \\\$3.24 \\cr
\\+ \\hfill 5    \\quad \& \\quad 45--52 \\quad \& \\qquad \\\$3.40 \\cr
\enduser

\noindent
<O parap'anw k'wdikac d'inei:\TeXref{247}
\vskip\baselineskip

{\rm
\settabs \+ \quad Year \quad & \quad Price \quad & \quad
Dividend \quad & \cr
\+ \hfill Year \quad & \quad Price  \quad & \quad Dividend \cr 
\+ \hfill 1971 \quad & \quad 41--54 \quad & \qquad \$2.60  \cr
\+ \hfill 2    \quad & \quad 41--54 \quad & \qquad \$2.70 \cr
\+ \hfill 3    \quad & \quad 46--55 \quad & \qquad \$2.87 \cr
\+ \hfill 4    \quad & \quad 40--53 \quad & \qquad \$3.24 \cr
\+ \hfill 5    \quad & \quad 45--52 \quad & \qquad \$3.40 \cr
} % end roman

\exercise Metakin~hste t`on parap'anw p'inaka pi`o kont`a st`o k'entro
t~hc sel'idac.

\exercise <'Enac tr'opoc gi`a n`a f'eroume st`o k'entro t~hc sel'idac
ke'imeno t`o <opo~io katalamb'anei >arket`ec >ar'adec, e>~inai n`a
qrhsimopoi'hsoume: {\tt \$\$\\vbox\lb$\ldots$\rb\$\$}.  M`e t`on tr'opo
a>ut`o metakin~hste pr`oc t`o k'entro t`on parap'anw p'inaka.  Pr'epei
<h >entol`h {\tt \\settabs} n`a peri\-lam\-b'a\-netai st`o {\tt\\vbox}?

\exercise Belti~wste t`on teleuta~io p'inaka j'etontac m'ia ken`h
gramm`h met`a to`uc t'itlouc.  <H l'exh >el'egqou {\tt\\hrule} j'etei
m'ia <oriz'ontia e>uje'ia metax`u d'uo gramm~wn to~u p'inaka. 
>Epanal'abete t`hn >'askhsh j'etontac t`hn l'exh >el'egqou {\tt\\strut}
met`a t`o {\tt \\+} st`hn gramm`h po`u peri'eqei to`uc t'itlouc t~wn
sthl~wn.  (T`o {\tt \\strut} o>usiastik`a fti'aqnei t`o di'asthma
metax`u t~wn gramm~wn to~u p'inaka k'apwc megal'utero. >'Etsi >all'azei
t`o di'astiqo t~wn gramm~wn to~u p'inaka.)\TeXref{82} Pros'exte loip`on
t`o parap'anw di'asthma po`u prost'ijetai metax`u t~wn gramm~wn.
\toindex{strut}

\exercise Stoiqeiojet~hste t`on >ak'oloujo p'inaka m`e sto'iqish st`o
dekadik`o shme~io, dhl.\ >'etsi <'wste t`a dekadik`a yhf'ia, d'ekata    
ka`i <ekatost'a, n`a br'iskontai stoiqism'ena st`hn >'idia j'esh.
(<Up'odeixh: jewr~hste t`o >ak'eraio m'eroc t~wn >arijm~wn stoiqism'eno
dexi`a t~hc tele'iac ka`i t`o dekadik`o m'eroc stoiqism'eno >arister`a
t~hc tele'iac.)
{\rm
\vskip\baselineskip
\settabs \+ \hskip 2 in & \hskip .75in & \hskip 1cm & \cr
\+ & Plums          & \hfill \$1&.22 \cr
\+ & Coffee         & \hfill   1&.78 \cr
\+ & Granola        & \hfill   1&.98 \cr
\+ & Mushrooms      &           &.63 \cr
\+ & {Kiwi fruit}   &           &.39 \cr
\+ & {Orange juice} & \hfill   1&.09 \cr
\+ & Tuna           & \hfill   1&.29 \cr
\+ & Zucchini       &           &.64 \cr
\+ & Grapes         & \hfill   1&.69 \cr
\+ & {Smoked beef}  &           &.75 \cr
\+ & Broccoli       & \hfill\underbar{\ \ 1}&\underbar{.09} \cr
\+ & Total          & \hfill \$12&.55 \cr
     } % end roman
        
\exercise Skefje~ite p~wc j`a qrhsimopoi'hsete t`o {\tt\\settabs} gi`a
n`a fti'axete <'enan pr'oqeiro p'inaka perieqom'enwn <'opwc:
\medskip
\leftline{\tt Getting Started \\dotfill \& \\hfill 1}
\leftline{\tt All Characters Great and Small \\dotfill \& \\hfill 9.}

\subsection{<Oriz'ontia sto'iqish m`e pi`o pol'uplokec mej'odouc}

T`o perib'allon {\tt \\settabs} e>~inai e>'ukolo st`hn qr'hsh tou, ka`i
<'otan kajor'isoume m'ia for`a t`o sq~hma to~u p'inaka, mporo~ume n`a
t`o qrhsimopoi'hsoume gi`a n`a par'agoume <'omoiouc p'inakec s`e
di'afora m'erh to~u keim'enou po`u >akolouje~i. <Wst'oso, t`o
perib'allon a>ut`o mpore~i n`a m`hn e>~inai t`o pl'eon e>'uqrhsto.  Gi`a
par'adeigma, t`o pl'atoc k'aje st'hlhc pr'epei n`a dhlwje~i pr`in t`o
gr'ayimo to~u perieqom'enou t~wn sthl~wn.  >Ep'ishc, st`hn per'iptwsh
po`u j'eloume m'ia st'hlh n`a stoiqeiojethje~i <'olh m`e >'entona
stoiqe~ia pr'epei n`a t`o dhl'wnoume a>ut`o s`e k'aje gramm'h. A>ut`a
t`a probl'hmata mporo~ume n`a t`a xeper'asoume m`e t`o perib'allon
{\tt\\halign}.\TeXref{235--238}
\toindex{halign}        
<H genik`h morf`h to~u perib'allontoc {\tt \\halign} >'eqei <wc >ex~hc:
\beginuser
\\halign\lb{} $<${\tengs gramm`h--de~igma\/}$>$ \\cr
$<${\tengs pr'wth gramm`h to~u p'inaka\/}$>$ \\cr
$<${\tengs de'uterh gramm`h to~u p'inaka\/}$>$ \\cr
$\vdots$
$<${\tengs teleuta'ia gramm`h to~u p'inaka\/}$>$ \\cr
\rb
\enduser

<H gramm`h--de~igma, <h <opo'ia d`en j`a tupwje~i st`o t'eloc, kaj`wc
ka`i oi <up'oloipec >emfan'isimec gramm`ec to~u p'inaka ({\rm display lines})
qwr'izontai s`e katak'orufec st~hlec m`e t`o s'umbolo
sto'i\-qi\-shc\NB{}{\tt\&}\null.  S`e k'aje st'hlh t~hc
gramm~hc--de'igmatoc qrhsimopoio~untai l'exeic >el'egqou <'opwc
sumba'inei ka`i m`e t`hn >entol`h  {\tt \\line\lb\rb}\null. Gi`a
par'adeigma, <h l'exh >el'egqou {\tt \\hfil} mpore~i n`a qrhsimopoihje~i
gi`a sto'iqish m'iac st'hlhc dexi'a, >arister`a >`h st`o k'entro.
>Ep'ishc, mporo~ume n`a >all'axoume grammatoseir`a m`e t`ic
>ant'istoiqec >entol`ec {\tt \\bf}, {\tt \\it}, k.lp.  S`e k'aje st'hlh
t~hc gramm~hc--de'igmatoc, mporo~ume >ak'oma n`a <or'isoume ka`i k'apoio
{\tengs stajer`o ke'imeno} po`u j`a peri'eqetai s`e <'ola ta ped'ia
t~hc st'hlhc.  <'Omwc prosoq'h!  S`e k'aje st'hlh t~hc
gramm~hc--de'igmatoc j`a pr'epei <opwsd'hpote n`a peri'eqetai t`o
e>idik`o {\tengs s'umbolo >antikat'astashc} {\tt \#} m'ia ka`i m'ono
m'ia for'a. T`o {\rm \TeX}, <'otan stoiqeiojete~i m'ia gramm`h <en`oc
p'inaka, j'etei t`o k'aje ped'io t~hc gramm~hc m`e t`hn seir`a
<'opwc br'iskei t`ic >ant'istoiqec j'eseic t~wn sumb'olwn {\tt \#}\null.
T`o <ep'omeno par'adeigma m~ac bohj~a n`a katal'aboume kal'utera t`hn
qr'hsh to~u {\tt \\halign}:

\beginuser
\\halign\lb\\hskip 2 in \$\#\$\& \\hfil \\quad \# \\hfil \& \\qquad \$\#\$ 
\hfill 3 in \& \\hfil \\quad \# \\hfil \\cr
\\alpha   \& alpha   \& \\beta  \& beta  \\cr
\\gamma   \& gamma   \& \\delta \& delta \\cr
\\epsilon \& epsilon \& \\zeta  \& zeta  \\cr
\rb
\enduser

\noindent
<H gramm`h--de~igma de'iqnei <'oti <h pr'wth st'hlh j`a peri'eqei
majhmatik`a s'umbola po`u j`a br'i\-sko\-ntai s`e >ap'ostash d'uo
>ints~wn >ap`o t`o >arister`o perij'wrio.  <H de'uterh st'hlh j`a
e>~inai kentrwm'enh m`e ken`o di'asthma <en`oc tetrag'wnou >arister'a. 
<H tr'ith st'hlh ka`i t'etarth j`a e>~inai par'omoiec m`e t`ic d'uo
pr~wtec.  <Or'iste ka`i t`o >apot'elesma:
\medskip       
{\rm
\halign{\hskip 2in $#$& \hfil\quad # \hfil & \qquad $#$ %
                       & \hfil\quad # \hfil\cr
\alpha   & alpha   & \beta  & beta \cr
\gamma   & gamma   & \delta & delta \cr
\epsilon & epsilon & \zeta  & zeta \cr
}
} % end roman

St`hn per'iptwsh to~u parap'anw p'inaka, <h pr'wth gramm`h sqhmat'izetai
m`e >antikat'a\-sta\-sh to~u pr'wtou {\tt \#} t~hc gramm~hc--de'igmatoc
>ap`o t`o {\tt \\alpha}, to~u de'uterou {\tt\#} >ap`o t`o {\tt alpha},
to~u tr'itou {\tt \#} >ap`o t`o {\tt\\beta} ka`i to~u t'etartou {\tt \#}
>ap`o t`o {\tt beta}.  <H >al'hjeia e>~inai <'oti t`o {\rm \TeX} d`en
proqwre~i >am'eswc st`hn stoiqeiojes'ia a>ut~hc t~hc gramm`h to~u
p'inaka, >all`a t`hn ful'assei st`hn mn'hmh tou.  T`o >'idio
>epanalamb'anetai ka`i m`e t`ic <up'oloipec gramm`ec to~u p'inaka. 
<'Otan t`o {\rm \TeX} diab'asei kai t`hn teleuta'ia gramm`h to~u
p'inaka, t'ote proqwre~i st`o <oristik`o fti'aximo to~u p'inaka
d'inontac s`e k'aje st'hlh >arket`o pl'atoc <'wste n`a <up'arqei q~wroc
gi`a <'ola thc t`a ped'ia.  (>`Ac >'eqoume <up'' >'oyh mac <'oti <h
stoiqeiojes'ia pin'akwn kat'' aut`on t`on tr`opo e>~inai m'ia {\tengs
diadikas'ia swreutik`h} gi`a t`hn mn'hmh to~u <upologist~h ka`i a>ut`o
mpore~i n`a prokal'esei probl'hmata mn'hmhc st`o {\rm \TeX}\null, p.q.,
mpore~i n`a stamat'hsei n`a tr'eqei d'inont'ac mac t`o m'hnuma:   
(({\rm out of memory})).  Gi`a t`on l'ogo a>ut'o, e>~inai protim'wtero
n`a >apofe'ugoume to`uc p'inakec po`u xeperno~un t`hn m'ia sel'ida.) 
M`e l'iga l'ogia, <h gramm`h--de~igma kajor'izei t`hn morf`h ka`i t`o
sq~hma t~wn gramm~wn to~u p'inaka, >en`w o<i <up'oloipec gramm`ec
d'inoun t`a stoiqe~ia po`u j`a peri'eqei telik`a <o p'inakac.

Merik`ec for`ec j`a qreiasje~i n`a kajor'isoume t`a <'oria m'iac
gramm~hc >`h m'iac st'hlhc to~u p'inaka m`e <oriz'ontiec >`h ka`i
katak'orufec e>uje~iec.  Gi`a n`a j'esoume <oriz'ontiec e>uje~iec,
qrhsimopoio~ume t`hn >entol`h {\tt\\hrule}, <'opwc ka`i st`hn       
per'iptwsh to~u perib'allontoc  {\tt \\settabs}. <'Omwc >epeid`h d`en
j'eloume <h <oriz'ontia e>uje'ia n`a e>~inai stoiqism'enh s'umfwna m`e
t`hn gramm`h--de~igma, gi'' a>ut`o ka`i qrhsimopoio~ume t`hn l'exh      
>el'egqou {\tt \\noalign}. Sunep~wc, gi`a n`a j'esoume m'ia <oriz'ontia
e>uje'ia st`on p'inaka, gr'afoume: {\tt \\noalign\lb\\hrule\rb}. <'Oso
gi`a t`ic katak'orufec e>uje~iec, a>ut`ec j'etontai gr'afontac {\tt
\\vrule} e>'ite st`hn gramm`h--de~igma e>'ite s`e k'apoia >ap`o t`ic
<up'oloipec gramm`ec to~u p'inaka.  <Wst'oso, t`a pr'agmata d`en e>~inai
t'oso <apl'a.  >`Ac p'aroume t`o teleuta~io par'adeigma ka`i >`ac
>all'axoume t`hn gramm`h--de~igma gi`a n`a j'esoume katak'orufec
e>uje~iec, >all`a >`ac prosj'esoume ka`i merik`ec <oriz'ontiec:
\toindex{noalign}
 
\beginuser 
\\halign\lb\\hskip 2in\\vrule\\quad \$\#\$\\quad \& \\vrule \\hfil\\quad % 
\# \\hfil 
\hskip 2 in \& \\quad \\vrule \\quad \$\#\$\\quad 
\hskip 2 in \& \\vrule\\hfil \\quad \# \\quad \\hfil \\vrule \\cr 
\\noalign\lb\\hrule\rb 
\\alpha   \& alpha   \& \\beta  \& beta \\cr 
\\noalign\lb\\hrule\rb 
\\gamma   \& gamma   \& \\delta \& delta \\cr 
\\noalign\lb\\hrule\rb 
\\epsilon \& epsilon \& \\zeta  \& zeta \\cr 
\\noalign\lb\\hrule\rb 
\rb 
\enduser 

\noindent 
<O parap'anw k'wdikac {\rm \TeX} d`en d'inei <'o,ti >akrib~wc j`a
j'elame, >all`a
\vskip\baselineskip
{\rm
\halign{\hskip 2in\vrule\quad $#$\quad & \vrule \hfil\quad # \hfil 
& \quad \vrule \quad $#$\quad & \vrule \hfil\quad # \quad \hfil 
\vrule \cr 
\noalign{\hrule} 
\alpha   & alpha   & \beta  & beta \cr 
\noalign{\hrule} 
\gamma   & gamma   & \delta & delta \cr 
\noalign{\hrule} 
\epsilon & epsilon & \zeta  & zeta \cr 
\noalign{\hrule} 
} % end halign
} % end roman

St`on p'inaka a>ut'o, <up'arqoun poll`a probl'hmata;  t`o pi`o faner`o
e>~inai o<i <uperbolik`a me\-g'a\-lec <oriz'ontiec e>uje~iec, >all`a
ka`i t`o ke'imeno po`u fa'inetai strimwgm'eno m'esa st`a pla'isia to~u
p'inaka. >Epipl'eon, t`a ped'ia k'aje st'hlhc fa'inetai n`a >'eqoun
l'igo periss'otero ken`o di'asthma st`hn dexi`a pleur'a touc >ant`i n`a
e>~inai t'eleia kentrwm'ena.  <'Opwc ka`i st`hn per'i\-ptw\-sh to~u
{\tt \\settabs}, >'etsi ka`i >ed~w o<i gramm`ec to~u p'inaka mporo~un
n`a g'inoun pi`o >arai`ec qrhsi\-mo\-poi\-'wntac t`hn l'exh >el'egqou {\tt
\\strut} st`hn gramm`h--de~igma.\TeXref{82} <'Omwc, <'ena >ak'oma
pr'oblhma mpore~i n`a >emfanisje~i kaj`wc t`o {\rm\TeX} dhmiourge~i t`hn
sel'ida; >'iswc t`o {\rm\TeX} n`a >arai'wsei >elafr`a t`ic gramm`ec to~u
p'inaka <'wste n`a beltiwje~i <h e>ik'ona t~hc <'olhc sel'idac.  A>ut`o
mpore~i n`a >'eqei <wc >apot'elesma t`hn >emf'anish mikr~wn ken~wn st`ic
katak'orufec e>uje~iec to~u p'inaka.  Gi`a n`a >apof'ugoume k'ati
t'etoio, qrhsimopoio~ume t`hn l'exh >el'egqou {\tt \\offinterlineskip}
m'esa st`o perib'allon {\tt \\halign}. Mporo~ume >ak'oma n`a
>apof'ugoume t`o pr'oblhma t~wn e>ujei~wn po`u >ex'eqoun st`hn
>ari\-ste\-r`h pleur`a to~u p'inaka, >afair'wntac t`hn >entol`h {\tt
\\hskip 2 in} >ap`o t`hn gramm`h--de~igma.  Gi`a n`a metakin'hsoume 
t`on p'inaka st`hn >arqik'h tou j'esh, qrhsimopoio~ume t`hn >entol`h 
{\tt\\moveright}.  T'eloc, mporo~ume n`a katal'aboume p~wc n`a
kentr'wsoume kal`a t`a ped'ia to~u p'inaka, para\-th\-r'wntac <'oti
t`o >epipl'eon ken`o di'asthma >ap`o t`a >arister`a >emfan'izetai st`hn 
gramm`h--de~igma met`a t`o s'umbolo {\tt\#}, <'opou dhl.\ g'inetai <h
>antikat'astash to~u keim'enou t~wn <upolo'ipwn >ar'adwn. Sunolik`a <o
parap'anw p'inakac mpore~i n`a beltiwje~i <wc >ex~hc:
\toindex{offinterlineskip}
\toindex{moveright}
 
\beginuser 
\\moveright 2 in 
\\vbox\lb\\offinterlineskip 
 
\\halign\lb\\strut \\vrule \\quad \$\#\$\\quad \&\\vrule \\hfil \\quad % 
\#\\quad \\hfil 
\&\\vrule \\quad \$\#\$\\quad \&\\vrule \\hfil \\quad \#\\quad \\hfil % 
\\vrule \\cr 
\\noalign\lb\\hrule\rb 
\\alpha   \& alpha   \& \\beta  \& beta \\cr 
\\noalign\lb\\hrule\rb 
\\gamma   \& gamma   \& \\delta \& delta \\cr 
\\noalign\lb\\hrule\rb 
\\epsilon \& epsilon \& \\zeta  \& zeta \\cr 
\\noalign\lb\\hrule\rb 
\rb\rb 
\enduser 

\noindent
<O k'wdikac a>ut`oc d'inei:
\vskip\baselineskip
{\rm
\moveright 2 in 
\vbox{\offinterlineskip 
\halign{\strut \vrule \quad $#$\quad &\vrule \hfil \quad #\quad \hfil 
&\vrule \quad $#$\quad &\vrule \hfil \quad #\quad \hfil 
\vrule \cr 
\noalign{\hrule} 
\alpha   & alpha   & \beta  & beta \cr 
\noalign{\hrule} 
\gamma   & gamma   & \delta & delta \cr 
\noalign{\hrule} 
\epsilon & epsilon & \zeta  & zeta \cr 
\noalign{\hrule} 
} % end halign
} % end vbox
} % end roman

Genik'wtera, >e`an j'eloume n`a fti'axoume <'enan p'inaka po`u n`a
e>~inai kentrwm'enoc st`hn sel'ida, mporo~ume n`a j'esoume t`o {\tt
\\vbox} m'esa s`e m'ia >entol`h {\tt\\centerline\lb\rb}\null.  <'Omwc
<or'iste ka`i m'ia pi`o >'exupnh l'ush: >E`an b'aloume t`o {\tt\\vbox}
metax`u dipl~wn dolar'iwn {\tt \$\$ $\ldots$ \$\$}, <h stoiqeiojes'ia
tou j`a e>~inai <'omoia m`e a>ut`h diakrit~wn majhmatik~wn >ekfr'asewn,
dhl.\ kentrwm'enh.  Profan~wc, t`o >apot'elesma d`en j`a e>~inai m'ia
majhmatik`h >ex'iswsh, >all`a >akrib~wc >epeid`h t`o {\rm\TeX} j`a
nom'isei <'oti >'eqei n`a k'anei m`e m'ia majhmatik`h >ex'iswsh, j`a
b'alei l'igo parap'anw ken`o di'asthma >ap`o t`o >ep'anw ka`i k'atw
m'eroc to~u p'inaka d'inont'ac mac <'ena a>isjhtik`a >arti'wtero
>apot'elesma. Sunoy'izontac t`a parap'anw, mporo~ume n`a po~ume p`wc
mporo~ume n`a stoiqeiojet'hsoume <'enan >'omorfo kentrwmm'eno p'inaka
>akolouj'wntac t`a >ex~hc t'essera b'hmata: (1)\NB{}j'e\-tou\-me <'ena
{\tt \\vbox} metax`u dipl~wn dolar'iwn; (2)\NB{}gr'afoume {\tt
\\offinterlineskip} ka`i {\tt\\halign} m'esa st`o {\tt \\vbox};
(3)\NB{}>ent`oc to~u perib'allontoc {\tt \\halign} <etoim'azoume m'ia
gramm`h--de~igma m`e <'ena {\tt \\strut} st`hn >arq`h ka`i {\tt\\vrule}
metax`u k'aje de'igmatoc st'hlhc; ka`i (4)\NB{}>an'amesa s`e k'aje
gramm`h to~u p'inaka gr'afoume {\tt \\noalign\lb\\hrule\rb}. <Or'iste
p~wc <o parap'anw kan'onac fa'inetai <wc k'wdikac to~u {\rm\TeX}:

\beginuser
\$\$\\vbox\lb
\\offinterlineskip
\\halign\lb
\\strut \\vrule \# \& \\vrule \# \& \dots \& \\vrule \# \\vrule \\cr
\\noalign\lb\\hrule\rb
$<${\tengs stoiqe~io 1hc st'hlhc\/}$>$ \& \dots \& %
$<${\tengs stoiqe~io teleuta'iac st'hlhc\/}$>$ \\cr
\\noalign\lb\\hrule\rb
\dots
\\noalign\lb\\hrule\rb
$<${\tengs stoiqe~io 1hc st'hlhc\/}$>$ \& \dots \& %
$<${\tengs stoiqe~io teleuta'iac st'hlhc\/}$>$ \\cr
\\noalign\lb\\hrule\rb
\rb
\rb\$\$
\enduser


\section{K'an'' to m'onoc sou}
      
S'' a>ut`o t`o kef'alaio j`a do~ume p~wc e>~inai dunat`o n`a <or'isoume
t`ic dik'ec mac l'exeic >el'egqou.  <H dhmiourg'ia a>ut~wn t~wn n'ewn
{\tengs <orism~wn\/}, po`u st`hn <orolog'ia t~hc Plhroforik~hc
>apokalo~untai ka`i {\tengs makroentol`ec\/} >`h {\sl macro}, e>~inai
m'ia >ap`o t`ic pi`o >isqur`ec teqnik`ec po`u m~ac prosf'erei t`o
{\rm\TeX}\null. <Wc pr'wth >efarmog`h t~hc dhmiourg'iac n'ewn <orism~wn,
j`a do~ume p~wc mpore~i kane`ic n`a kerd'isei pol`u qr'ono
daktulogr'afhshc >antikajist'wntac meg'ala m'erh >epanalamban'omenou
keim'enou m`e <'enan mikr`o <orism'o.

\subsection{T`o makr`u ka`i t`o kont`o}

<H l'exh >el'egqou {\tt\\def} qrhsimopoie~itai gi`a t`on <orism`o n'ewn
>akolouji~wn (l'exewn >`h sumb'olwn) >el'egqou.  <O <aplo'u\-ste\-roc
tr'opoc dhmiourg'iac m'iac n'eac >akolouj'iac >el'egqou e>~inai: {\tt
\\def\\newname\lb$\ldots$\rb}. Met`a t`on <orism'o, <'opote m'esa st`on
k'wdik'a mac >emfan'izetai {\tt \\newname}, a>ut`h <h l'exh >el'egqou
j`a >anti\-kaj'i\-statai >ap`o t`o {\rm \TeX} m`e t`o {\tengs
<'orism'a\/} thc, dhl.\ m`e <'o,ti peri'eqoun o<i >agk'ulec to~u
<orismo~u thc.   E>~inai profan`ec <'oti <h n'ea >entol`h {\tt
\\newname} pr'epei n`a <orisje~i s'umfwna m`e to`uc kan'onec to~u {\rm
\TeX}, dhl.\ j`a pr'epei n`a e>~inai e>'ite m'ia n'ea l'exh >el'egqou
>apotelo'umenh >ap`o qarakt~hrec to~u latiniko~u >alfab'htou ka`i m'ono,
 e>'ite  <'ena n'eo s'umbolo >el'egqou >apotelo'umeno >ap`o <'enan ka`i
m'onon qarakt'hra >ekt`oc a>ut~wn to~u latiniko~u >alfab'htou.  >`Ac
<upoj'esoume, p.q., <'oti >'eqoume n`a stoiqeiojet'hsoume <'ena ke'imeno
po`u peri'eqei poll`ec for`ec t`hn fr'ash (({\rm European Union})).  M`e
t`on <orism`o {\tt \\def\\eu\lb European Union\rb} >'eqoume m'ia n'ea
l'exh >el'egqou, t`hn {\tt \\eu}, t`hn <opo'ia mporo~ume n`a t`hn
qrhsimopoi'hsoume <opoud'hpote m'esa st`on k'wdik'a mac met`a t`on
<orism'o thc.  <H fr'ash {\tt I am a citizen of \\eu.}\ e>~inai swst`h
>ef'' <'oson >'eqei prohghje~i <o <orism`oc t~hc l'exhc >el'egqou
{\tt\\eu}.  St`hn per'i\-ptw\-sh a>ut'h, t`o {\rm\TeX} j`a
>antikatast'hsei t`hn {\tt\\eu} m`e t`o <'orism'a thc (t`hn >'idia
>epexergas'ia k'anei t`o {\rm\TeX} ka`i m`e t`ic dik'ec tou >eswterik`ec
>akolouj'iec >el'egqou, gi'' a>ut`o qrei'azetai l'igh prosoq`h st`hn
>epilog`h t~wn >onom'atwn t~wn n'ewn mac <orism~wn).  <Wst'oso, k'aje
n'ea >akolouj'ia >el'egqou >'eqei topik`h >isq`u st`o sun'olo >ent`oc
to~u <opo'iou <or'izetai. Gi`a par'adeigma, <o parak'atw k'wdikac
{\rm\TeX}
\toindex{def}

\beginuser
\\def\\eu\lb European Union\rb
I worked as a clerk for the \\eu.
\lb
\\def\\eu\lb European University\rb
Then I took a sabbatical leave to study at the \\eu.
\rb
Now I am working again for the \\eu.
\enduser

\noindent d'inei

{\rm
\def\eu{European Union}
I worked as a clerk for the \eu.
{
\def\eu{European University}
Then I took a sabbatical leave to study at the \eu.
}
Now I am working again for the \eu.
} % end roman

>`Ac jumhjo~ume <'oti <'ola t`a ken`a diast'hmata po`u >akoloujo~un m'ia
l'exh >el'egqou >agnoo~untai >ap`o t`o {\rm\TeX} kat`a t`hn
>epexergas'ia to~u k'wdika; a>ut`o >isq'uei ka`i gi`a t`ic n'eec l'exeic
>el'egqou po`u >eme~ic <or'izoume.  St`o prohgo'umeno par'adeigma, k'aje
ken`o di'asthma met`a t`hn l'exh >el'egqou {\tt\\eu} j`a >agnohje~i.
<'Omwc, t`o ken`o di'asthma met`a t`hn pr'wth per'iodo (.)\ ka`i met`a
t`hn pr'wth >ari\-ste\-r`h >agk'ulh ({\tt\lb}) d`en >agnoo~untai >ap`o
t`o {\rm \TeX}\null; >e`an parathr'hsoume prosektik`a t`o t'eloc t~hc
pr'wthc pr'otashc po`u stoiqeiojet'hjhke s'umfwna m`e t`o parap'anw
par'adeigma, j`a do~ume <'oti peri'eqei k'apoio parap'anw ken`o
di'asthma.  A>ut`o mporo~ume n`a t`o >apof'ugoume j'etontac <'ena
s'umbolo sqol'iou {\tt \%} met`a t`hn >ari\-ste\-r`h >agk'ulh, <'wste
t`o <up'oloipo t~hc gramm~hc to~u k'wdika n`a >agnohje~i >ap`o t`o {\rm
\TeX}\null.  T`o >'idio mporo~ume n`a k'anoume ka`i st`hn gramm`h to~u
k'wdika m`e t`hn teleuta'ia dexi`a >agk'ulh ({\tt\rb})\null.  Suqn'a,
a>ut`h <h {\tengs >apenergopo'ihsh\/} t~wn <upolo'ipwn t~wn gramm~wn
<en`oc <orismo~u ({\rm commenting out}) e>~inai >apara'ithth gi`a t`on
>apotelesmatik`o >'elegqo t~wn ken~wn diasthm'atwn st`o telik'o mac
>'entupo.

<'Otan m'ia n'ea makroentol`h >'eqei <orisje~i, <h >'idia mpore~i n`a
qrhsimopoihje~i ka`i gi`a t`on <orism`o >'allwn makroentol~wn.  A>ut`oc
e>~inai, p.q., <'enac tr'opoc n`a <etoim'asei k'apoioc >epistol`ec
<apl~hc morf~hc.  >`Ac <or'isoume pr~wta m'ia <apl`h >epistol'h:

\beginuser
\\def\\letter\lb
\\par \\noindent
Dear \\name,
\
This is a little note to let you know that your name is \\name.
\
\\hskip 2 in Sincerely yours,
\\vskip 2\\baselineskip
\\hskip 2 in The NameNoter
\\smallskip \\hrule
\rb
\enduser

\def\letter{%
{\rm
\par \noindent
Dear \name,

This is a little note to let you know that your name is \name.

\hskip 2 in Sincerely yours,
\vskip 2\baselineskip\nobreak
\hskip 2 in The NameNoter
\smallskip \hrule
} % end roman
} % end letter

St`hn >epistol`h qrhsimopoie~itai <h l'exh >el'egqou {\tt \\name}, <h
<opo'ia <'omwc d`en >'eqei <orisje~i >ak'oma. <'Otan qrhsimopoi'hsoume
t`hn l'exh >el'egqou {\tt \\letter}, <h {\tt \\name} j`a
>antika\-ta\-sta\-je~i m`e t`on tr'eqonta <orism'o thc.  Sunep~wc, <o  
k'wdikac

\beginuser
\\def\\name\lb Michael Bishop\rb
\\letter
\\def\\name\lb Michelle L\\'ev\\\^{}eque\rb
\\letter
\enduser

\noindent j`a m~ac d'wsei d'uo >ant'igrafa t~hc >epistol~hc, t`o kaj'ena
m`e t`on swst`o paral'hpth ka`i m`e m'ia <oriz'ontia e>uje'ia gramm`h
st`o t'eloc, dhl.\

\def\name{Michael Bishop}
\letter
\goodbreak
\def\name{Michelle L\'ev\^eque}
\letter
\vskip\baselineskip

J`a mporo'usame n`a e>'iqame j'esei <otid'hpote (>`h sqed`on
<otid'hpote) metax`u t~wn >agkul~wn to~u <orismo~u {\tt
\\def\\name\lb$\ldots$\rb}; j`a mporo'usame n`a j'etame merik`ec
paragr'afouc ka`i n`a qrhsimopoio~usame ka`i >'allec >entol'ec (par''
<'oti st`hn per'i\-ptw\-sh m'iac <apl~hc >epistol~hc <'ola a>ut`a j`a
>~htan m'allon <uperbolik'a).  Fusik'a, j`a mporo'usame n`a j'etame
st`on <orism`o t~hc {\tt \\letter} ka`i {\tt \\vfill \\eject}, <'wste
k'aje m'ia >epistol`h n`a tup'wnetai s`e xeqwrist`h sel'ida.

\exercise <Etoim'aste m'ia f'orma <apl~hc >epistol~hc qrhsimopoi'wntac
t`ic >ak'oloujec l'exeic >el'eqou: {\tt \\name} (>'onoma), {\tt
\\address} (die'ujunsh), {\tt \\postcode} (taqudromik`oc k'wdikac), {\tt
\\city} (p'olh) ka`i {\tt \\country} (q'wra).

\exercise Suqn`a s`e ke'imena qrei'azetai n`a fti'axoume m`h
>arijmhm'enec l'istec >antikei\-m'e\-nwn, jem'atwn, k.lp.\
qrhsimopoi'wntac {\tt \\item\lb\$\\bullet\$\rb}.  <Or'iste m'ia
makro\-entol`h m`e t`hn >onomas'ia {\tt \\bitem} po`u k'anei a>ut`h t`hn
doulei`a gi`a merik`ec paragr'afouc.  Kat'opin, >all'axte t`o shme~io
({\rm bullet}) m`e m'ia pa'ula. J`a parathr'hsete <'oti m'ia m'onon
mikr`h >allag`h st`on <orism`o t~hc makro\-entol~hc prokale~i <'olec
t`ic >apara'ithtec >allag`ec s`e <'olec t`ic paragr'afouc.    
    
\exercise <Upoj'este <'oti >'eqete n`a <etoim'asete poll`ec     
paragr'afouc s`e <'ena ke'imeno qrhsimopoi'wntac {\tt \\hangindent = 30
pt}, {\tt \\hangafter 4} ka`i {\tt \\filbreak} (m`hn >anhsuqe~ite gi`a
t`o t'i prokalo~un a>ut`ec o<i par'ametroi;  t`o m'ono po`u metr'aei  
t'wra e>~inai <'oti, >ef'' <'oson <orisjo~un, param'enoun s`e >isq`u
m'onon gi`a m'ia par'agrafo).  <Or'iste m'ia l'exh >el'egqou  {\tt
\\setpar} <h <opo'ia n`a mpore~i n`a qrhsimopoihje~i >empr`oc >ap`o
k'aje par'agrafo po`u pr'epei morfopoihje~i s'umfwna m`e t`ic parap'anw
param'etrouc.     
    
\subsection{Par'ametroi st`ic makroentol`ec}    
    
O<i n'eec makroentol`ec mporo~un n`a g'inoun pol`u pi`o genik`ec <'otan
peri'eqoun param'e\-trouc.  <H >id'ea t~wn param'etrwn e>~inai par'omoia
a>ut~hc t~hc gramm~hc--de'igmatoc to~u peri\-b'al\-lontoc {\tt
\\halign}\null. Pr~wta, >`ac do~ume t`hn per'i\-ptw\-sh m'iac n'eac
l'exhc >el'egqou m`e m'ia par'ametro. St`hn per'i\-ptw\-sh a>ut'h, <h
n'ea l'exh >el'egqou <or'izetai <wc {\tt \\def\\newword\#1\lb
$\ldots$\rb}\null.  T`o s'umbolo {\tt \#1} mpore~i n`a <up'arqei
periss'oterec >ap`o m'ia for`a metax`u t~wn >agkul~wn to~u <orismo~u
t~hc {\tt \\newword}. <'O,ti >'eqei grafe~i metax`u t~wn >agkul~wn to~u
<orismo~u dr~a <'opwc ka`i <h gramm`h--de~igma to~u peri\-b'al\-lontoc
{\tt \\halign}\null.  >'Etsi, <'opou m'esa st`on k'wdika >emfan'izetai
<h {\tt \\newword\lb$\ldots$\rb}, a>ut`h j`a >antikaj'istatai >ap`o t`o
<'orism'a thc ka`i st`hn j'esh to~u {\tt \#1} j`a mpa'inei t`o <ulik`o
>ent`oc t~wn >agkul~wn. {\tengb <H parous'ia ken~wn diasthm'atwn st`on
<orism`o t~hc makroentol~hc >'eqei meg'alh shmas'ia; d`en pr'epei n`a
<up'ar\-qoun ken`a diast'hmata pr`in t`hn pr'wth >arister`h >agk'ulh
({\tt\lb}) to~u <orismo~u.}  

<Wc par'adeigma, j`a mporo~usame n`a tropopoi'hsoume t`hn di'ataxh t~hc
>epistol~hc t~hc prohgo'umenhc paragr'afou kat`a t`on >ak'oloujo tr'opo:

\beginuser
\\def\\letter\#1\lb
\\par \\noindent
Dear \#1,
\
This is a little note to let you know that your name is \#1.
\
\\hskip 2 in Sincerely yours,
\\vskip 2\\baselineskip
\\hskip 2 in The NameNoter
\\smallskip \\hrule
\rb
\enduser

\def\letter#1{%
{\rm
\par \noindent
Dear #1,

This is a little note to let you know that your name is #1.

\hskip 2 in Sincerely yours,
\vskip 2\baselineskip
\hskip 2 in The NameNoter
\smallskip \hrule
}
}

\noindent T'wra mporo~ume n`a gr'ayoume

\beginuser
\\letter\lb Michael Bishop\rb
\\letter\lb Michelle L\\'ev\\\^{}eque\rb
\enduser

\noindent gi`a n`a l'aboume \medskip

\letter{Michael Bishop}
\goodbreak
\letter{Michelle L\'ev\^eque}

T'wra >`ac <or'isoume m'ia n'ea  makroentol`h <wc {\tt
\\def\\displaytext\#1\lb \$\$\\vbox\lb\\hsize = 12 cm \#1\rb\$\$\rb}
gi`a n`a parousi'azoume k'apoio m'eroc to~u keim'enou mac kentrarism'eno
xeqwrist`a >ap`o t`o <up'oloipo ke'imeno (p.q., gi`a d'aneio ke'imeno). 
T'ote, <h >entol`h {\tt \\displaytext\lb$\ldots$\rb} j`a    d'wsei t`o
ke'imeno >ent`oc t~wn >agkul~wn s`e m'ia kentrarism'enh    par'agrafo
pl'atouc $12\,\rm cm$ m`e l'igo >epipl'eon ken`o di'asthma st`o >ep'anw
ka`i k'atw m'eroc thc, >'etsi <'wste n`a xeqwr'izei >ap`o t`o <up'oloipo
ke'imeno.  >`Ac do~ume <'ena t'etoio par'adeigma m`e t`o ke'imeno
<eto'uthc t~hc paragr'afou st`hn >agglik`h gl'wssa:
\def\displaytext#1{$$\vbox{\hsize=12cm #1}$$}
\displaytext{\rm
Now let's define
{\tt\\def\\displaytext\#1\lb \$\$\\vbox\lb\\hsize = 12 cm \#1\rb\$\$\rb}
as a new macro to display text.
Then {\tt \\displaytext\lb$\ldots$\rb} will cause the material
between the braces to be put in a paragraph with width 12 \centimetre
s and then \centred{} with some space added above and
below as is appropriate for a display.  This paragraph was set
using this {\tt \\displaytext} macro.}

K'aje par'ametroc m'iac makroentol~hc d`en mpore~i n`a xepern~a s`e m~hkoc   
t`hn m'ia par'agrafo.  >E`an m'ia de'uterh par'agrafoc e>isaqje~i <wc   
m'eroc m'iac param'etrou, t'ote t`o {\rm \TeX} j`a stamat'hsei   
d'inont'ac mac <'ena m'hnuma l'ajouc.  A>ut`h e>~inai m'ia dikl'ida   
>asfale'iac to~u {\rm \TeX}\null; diaforetik'a, m'ia tuqa'ia par'aleiyh   
m'iac dexi~ac >agk'ulhc j`a >an'agkaze t`o {\rm \TeX} n`a jewr'hsei <'olo   
t`o <up'oloipo >arqe~io <wc m'ia par'ametro k'apoiac makroentol~hc.   
  
\exercise <Or'iste m'ia n'ea makroentol`h m`e t`o >'onoma {\tt
\\yourgrade} (<o bajm'oc sou) >'etsi <'wste <o k'wdikac {\tt
\\yourgrade\lb89\rb} n`a d'inei stoiqeiojethm'enh t`hn >ak'oloujh
fr'ash: {\rm The grade you received is 89\%}\null. Fusik'a, j`a
pr'epei n`a mpore~i n`a qrhsimopoihje~i gi`a <opoiond'hpote bajm'o
(p.q., 45\%\null, 73\%\null, k.lp.).
\medbreak

<H qr'hsh perissot'erwn param'etrwn d`en e>~inai >idia'itera d'uskolh.  
 Gi`a n`a <or'isoume m'ia n'ea l'exh >el'egqou d'uo param'etrwn, >arke~i
n`a    gr'ayoume {\tt \\def\\newword\#1\#2\lb$\ldots$\rb}. T`o <'orisma
mpore~i n`a peri'eqei >an'amesa st`ic >agk'ulec t`ic param'etrouc {\tt  
 \#1} ka`i {\tt \#2} ka`i m'alista periss'oterec >ap`o m'ia for`a t`hn
k'aje m'ia.  <'Otan kat'opin t`o {\rm \TeX} bre~i st`on k'wdik'a mac
t`hn >entol`h {\tt \\newword\lb$\ldots$\rb\lb$\ldots$\rb}, <'o,ti
diab'asei metax`u t~wn d'uo pr'wtwn >agkul~wn t`o >antikajist~a st`hn
j'esh {\tt \#1} to~u <or'ismatoc; ka`i <'o,ti diab'asei metax`u t~wn
de'uterou ze'ugouc >agkul~wn t`o >anti\-ka\-ji\-st~a st`hn j'esh {\tt
\#2}\null. <Or'iste <'ena sqetik`o par'adeigma:

\beginuser
\\def\\talks\#1\#2\lb \#1 talks to \#2.\rb
\\talks\lb John\rb\lb Jane\rb
\\talks\lb Jane\rb\lb John\rb
\\talks\lb John\rb\lb me\rb
\\talks\lb She\rb\lb Jane\rb
\enduser

\def\talks#1#2{{\rm #1 talks to #2.}}
\talks{John}{Jane}
\talks{Jane}{John}
\talks{John}{me}
\talks{She}{Jane}
  
\exercise Kat`a tr'opo par'omoio m`e t`hn prohgo'umenh >'askhsh,
<or'iste m'ia n'ea >entol`h m`e t`o >'onoma {\tt\\yourgrade}, >'etsi
<'wste <o k'wdikac {\tt \\yourgrade\lb89\rb\lb85\rb} n`a d'inei
stoiqeiojethm'enh t`hn >ak'oloujh fr'ash: {\rm You received a grade of
89\% on your first exam and a grade of 85\% on your second exam}. 

\exercise <Or'iste m'ia n'ea l'exh >el'egqou m`e t`hn >onomas'ia {\tt
\\frac}, >'etsi <'wste <o k'wdikac {\tt \\frac\lb a\rb\lb b\rb} n`a
d'inei stoiqeiojethm'eno t`o kl'asma ${a\over b}$.
\bigskip 

E>~inai shmantik`o n`a m`hn <up'arqoun ken`a diast'hmata st`on <orism`o
pr`in t`hn pr'wth >arister`h >agk'ulh tou. >E`an <up'arqoun ken`a
diast'hmata, t`o {\rm\TeX} j`a katal'abei t`on <orism`o diaforetik`a
>ap'' <'o,ti perigr'ayame pi`o p'anw.  Gi`a periss'oterec >ap`o d'uo
param'etrouc, <o tr'opoc <orismo~u n'ewn >akolouji~wn >el'egqou e>~inai
par'omoioc.  Gi`a n`a <or'isoume m'ia n'ea l'exh >el'egqou m`e tre~ic
param'etrouc, xekin~ame gr'afontac {\tt
\\def\\newword\#1\#2\#3\lb$\ldots$\rb}.  Kat'opin, gr'afoume >ent`oc
t~wn >agkul~wn to~u <orismo~u t`a {\tt \#1}, {\tt \#2} ka`i {\tt \#3}
<'opwc >eme~ic >epijumo~ume.  Met`a t`on <orism'o, <'opou t`o {\rm \TeX}
sunant~a {\tt \\newword\lb$\ldots$\rb\lb$\ldots$\rb\lb$\ldots$\rb},
>antikajist~a t`o <ulik`o po`u <up'arqei metax`u k'aje ze'ugouc
>agkul~wn st`ic >ant'istoiqec j'eseic {\tt \#1}, {\tt \#2} ka`i {\tt
\#3} to~u <orismo~u.  Sunolik'a, <o >arijm`oc t~wn param'etrwn <en`oc
<orismo~u mpore~i n`a fj'anei t`ic >enni'a ({\tt\#9}).

\subsection{M`e <'ena >'allo >'onoma}  
  
Merik`ec for`ec e>~inai bolik`o n`a d'inoume s`e m'ia l'exh >el'egqou
<'ena diaforetik`o >'onoma.  P.q., >e`an >'eqoume sunhj'isei st`hn
bretanik`h >orjograf'ia t~wn >agglik~wn >ant`i t~hc >amerikanik~hc,
>'iswc n`a protimo~ume n`a gr'afoume {\tt \\centreline} >ant`i gi`a {\tt
\\centerline}.  A>ut`h <h >allag`h >onomas'iac mpore~i n`a g'inei
e>'u\-kola m`e t`hn l'exh >el'egqou {\tt \\let}.  Gr'afontac loip`on
{\tt \\let \\centreline = \\centerline}, <or'izoume m'ia n'ea l'exh
>el'egqou <h <opo'ia mpore~i n`a qrhsimopoihje~i st`hn j'esh t~hc
pali~ac (a>ut`o d`en shma'inei <'oti <h pali`a d`en mpore~i n`a
qrhsimopoihje~i).  A>ut`h <h metonomas'ia l'exewn >el'egqou mpore~i n`a g'inei
ka`i m`e majhmatik`a s'umbola, <'opwc p.q.\ <h metonomas'ia {\tt \\let
\\tensor = \\otimes} m~ac >epitr'epei n`a gr'afoume:%
\TeXref{206--207}
\toindex{let}  
\toindex{centreline}
\toindex{tensor}

\let\tensor=\otimes
\beginuser
\$\$ (A \\tensor B) (C \\tensor D) = AC \\tensor BD. \$\$
\enduser

\noindent gi`a n`a l'aboume

$$ (A \tensor B) (C \tensor D) = AC \tensor BD. $$

\exercise <Or'iste t`ic makroentol`ec {\tt \\ll}, {\tt \\cl} ka`i {\tt
\\rl}, o<i <opo~iec n`a >isodunamo~un m`e t`ic {\tt \\leftline}, {\tt
\\centerline} ka`i {\tt \\rightline}.

M`e l'iga l'ogia, <h l'exh >el'egqou {\tt \\let} >epitr'epei st`on
qr'hsth to~u {\rm \TeX} n`a >onom'azei t`ic basik`ec >akolouj'iec
>el'egqou <'opwc to~u >ar'esei ka`i to~u e>~inai bolik'o.  >'Etsi, <o
qr'hsthc >'eqei t`ic dik'ec tou makroentol`ec po`u mpore~i n`a
qrhsimopoi'hsei st`hn j'esh a>ut~wn po`u prosf'erei t`o {\rm \TeX}.  


\section{T`a l'ajh e>~inai >anjr'wpina}

T`o {\rm \TeX} e>~inai <'ena p'ara pol`u >'exupno pr'ogramma, >all`a
>'oqi ka`i je"ik'o!  >'Etsi >e`an sunant'hsei k'apoio
m'eroc to~u k'wdika gramm'eno kat`a tr'opo lanjasm'eno, j`a >apant'hsei
m`e <'ena {\tengs m'hnuma sf'al\-matoc} st`hn >oj'onh to~u <upologist~h
(>ef'' <'oson t`o {\rm \TeX} tr'eqei <wc >allhloepidr`on ka`i m~ac
>epitr'epei n`a sunomilo~ume maz'i tou). T`o >'idio m'hnuma j`a
kataqwrhje~i >ep'ishc ka`i st`o >ant'istoiqo >arqe~io {\tt .log}. 
>Epeid`h t`o {\rm \TeX} e>~inai >arket`a pol'uploko pr'ogramma, t`o
m'hnuma sf'almatoc po`u j`a d'wsei mpore~i n`a m`hn e>~inai e>'ukola
katan'ohto gi`a t`on >arq'ario.  M'alista t`o {\rm \TeX} mpore~i n`a
prospaj'hsei n`a diorj'wsei t`o sf'alma m'ono tou, d'inontac <'omwc
pl'hrh >anafor`a gi`a t`o t'i >akrib~wc diorj'wseic >'ekane.  <Wst'oso
ka`i t`a mhn'umata t~wn diorj'wsewn mpore~i n`a m`hn e>~i\-nai >idia'itera
<apl'a. >Eke~ino po`u >'eqei shmas'ia gi`a n`a katano'hsoume <'ena
m'hnuma sf'almatoc to~u {\rm\TeX}, e>~inai n`a katal'aboume poi'o
m'eroc to~u mhn'umatoc e>~inai shmantik`o ka`i poi'o m'eroc to~u
mhn'umatoc mporo~ume n`a >agno'hsoume qwr`ic kam'ia sun'epeia.  >`Ac
do~ume merik`a tupik`a sf'al\-ma\-ta st`on k'wdika ka`i t`a >ant'istoiqa
mhn'umata po`u m~ac d'inei t`o {\rm\TeX}.

\subsection{T`o xeqasm'eno >ant'io}

%%% poor man's carriage return symbol:
\def\CarRet{$\gets$\hglue -1.25pt\raise 2.5pt\hbox{\vrule\vbox to 4.5pt{}}}

T`o pr~wto sf'alma po`u j`a >exet'asoume e>~inai >eke~ino po`u <'oloi
k'apoia stigm`h t`o k'anoun, dhlad`h t`hn par'aleiyh n`a  b'aloume t`o
>apara'ithto {\tt\\bye} st`o t'eloc to~u k'wdik'a mac. >E`an
tr'e\-qou\-me t`o {\rm \TeX} <wc >allhloepidr'on, st`hn >oj'on'h mac j`a
>emfanisje~i <'enac >aster'iskoc\hfil\break
\leftline{\tt *}
ka`i met`a t'ipota periss'otero.  T`o {\rm \TeX} perim'enei n`a to~u
d'wsoume ka`i >'allon k'wdika n`a >epe\-xer\-gasje~i >ap`o t`o
plh\-ktro\-l'ogio.  <Otid'hpote plhktrolog'hsoume st`hn sun'eqeia j`a
prosteje~i st`on >arqik`o k'wdika po`u pro'erqetai >ap`o t`a sqetik`a
>arqe~ia mac.  Sun'hjwc a>ut`o po`u plh\-ktro\-lo\-go~u\-me e>~inai
{\tt\\bye<CR>}\fnote{{\tt <CR>} e>~inai t`o pl~hktro po`u m~ac d'inei
m'ia n'ea gramm`h st`on k'wdika.  >Apokale~itai {\rm Carriage Return},
{\rm Enter} >`h {\rm Return}, >`h sumbol'izetai m`e <'ena meg'alo
>arister`o >an'a\-stro\-fo b'eloc (s`an \CarRet$\,$).} >'etsi <'wste 
t`o {\rm\TeX} n`a <oloklhr'wsei t`hn >epexergas'ia to~u k'wdika.
  

\subsection{<H lanjasm'enh >`h >'agnwsth l'exh >el'egqou}

<'Ena >ak'oma suqn`o sf'alma e>~inai <h qr'hsh mi~ac lanjasm'ena   
gramm'enhc >`h >'agnwsthc gi`a t`o {\rm \TeX} l'exhc >el'egqou.    
>E`an t`o {\rm\TeX} tr'eqei <wc >adi'aleipto, <'otan
sunant'hsei m'ia lanjasm'enh >`h >'agnwsth l'exh >el'egqou, j`a m~ac
d'wsei <'ena m'hnuma sf'almatoc, j`a >agno'hsei t`hn sugkekrim'enh l'exh
>el'egqou ka`i j`a suneq'isei st`hn >epexergas'ia to~u <up'oloipou
k'wdika.  <'Otan <'omwc t`o {\rm \TeX} tr'eqei <wc >allhloepidr'on, m~ac
d'inei t`hn dunat'othta n`a diorj'wsoume t'etoiou e>'idouc sf'almata
kaj`wc tr'eqei (<'omwc prosoq'h: a>ut`ec o<i diorj'wseic po`u k'anoume
<'otan t`o {\rm \TeX} tr'eqei d`en katagr'afontai ka`i st`o >arqe~io m`e
t`on >arqik'o mac k'wdika; t`ic diorj'wseic st`o >arqe~io pr'epei n`a
t`ic k'anoume kat'opin).  >`Ac <upoj'esoume <'oti >'eqoume <'ena
>arqe~io k'wdika {\rm \TeX} po`u peri'eqei t`ic >ak'oloujec d'uo
gramm'ec: 
\beginuser 
\\line\lb The left side \\hfli the right side\rb 
\\bye
\enduser

Profan~wc >ant`i gi`a t`hn l'exh >'el'egqou {\tt \\hfli} j`a >'eprepe n`a
e>'iqame gr'ayei {\tt \\hfil}.  <Or'iste t`o m'hnuma sf'almatoc po`u j`a
do~ume st`hn >oj'onh:
\beginuser
\obeyspaces
! Undefined control sequence.
l.1 \\line\lb The left side \\hfli
\                               the right side\rb
?
\enduser

<H pr'wth gramm'h, po`u xekin~a m`e <'ena jaumastik`o ({\tt !}),
d'inei m'ia mikr`h >ex'hghsh to~u probl'hmatoc.  Kat'opin,
parousi'azetai <o >arijm`oc t~hc gramm~hc to~u >arqe'iou m`e t`on
k'wdika <'opou br'isketai t`o sf'alma kaj`wc ka`i t`o m'eroc >eke~ino
t~hc gramm~hc po`u t`o {\rm \TeX} mp'orese n`a >epexergasje~i.  St`hn
<ep'omenh gramm`h to~u mhn'umatoc sf'almatoc d'inetai <h sun'eqeia t~hc
gramm~hc to~u k'wdika met`a t`o shme~io <'opou t`o {\rm \TeX} sun'anthse
t`o sf'alma.  St`o t'eloc to~u mhn'umatoc, t`o {\rm \TeX} m~ac
parousi'azei <'ena >erwthmatik`o gi`a n`a m~ac pe~i <'oti perim'enei
k'apoia >ap'okrish >ap`o m'erouc mac.

O<i >epitrept`ec >apokr'iseic po`u mporo~ume n`a d'wsoume
s''\NB{}a>ut`hn t`hn per'iptwsh (>all`a ka`i s`e k'aje per'iptwsh po`u
t`o {\rm \TeX} stamat~a >exait'iac k'apoiou sf'almatoc) e>~inai
o<i >ak'oloujec:

\maketable [Dunat`ec >apokr'iseic s`e mhn'umata sf'almatoc to~u 
{\lsecfont \TeX}]
\halign{
\strut # \hfil & \hfil \tt # \hfil & # \hfil\cr
   \tengb Skop`oc & \tengb >Ap'okrish
					& \tengb \hfil >Apot'elesma\cr
   \noalign{\hrule} \noalign{\smallskip}
   bo'hjeia ({\rm help})
                & h<CR>& >Exhge~itai leptomer~wc <o l'ogoc t~hc diakop~hc. \cr
   parembol'h ({\rm insert}) 
                & i<CR>& <H <ep'omenh gramm`h paremb'alletai st`on k'wdika.\cr 
   >'exodoc ({\rm exit})
                & x<CR>& >'Exodoc ka`i kle'isimo to~u >arqe'iou {\rm DVI}. \cr 
   metak'ulish ({\rm scroll}) 
                & s<CR>& T`o {\rm\TeX} st`o >ex~hc j`a d'inei mhn'umata sf'almatoc, \cr  
                &      & qwr`ic n`a diak'optei s`e >as'hmanta sf'almata. \cr 
   tr'eximo ({\rm run})
                & r<CR>& T`o {\rm \TeX} st`o >ex~hc j`a d'inei mhn'umata sf'almatoc, \cr  
                &      & qwr`ic <'omwc n`a diak'optei pot'e. \cr
   siwp'h ({\rm quiet})      
                & q<CR>& T`o {\rm \TeX} suneq'izei qwr`ic n`a d'inei kan'ena \cr 
                &      & m'hnuma sf'almatoc. \cr
   sun'eqeia ({\rm carry on})  
                & <CR>  & T`o {\rm \TeX} suneq'izei <'oso kal'utera mpore~i. \cr
      }

St`o teleuta'io par'adeigma, m'ia logik`h >ap'okrish e>~inai {\tt
h<CR>}, <'wste n`a l'aboume m'ia pi`o leptomer`h >ex'hghsh to~u
sf'almatoc. Kat'opin plhktrologo~ume {\tt i<CR>} gi`a n`a >enj'esoume
k'apoio ke'imeno, <op'ote t`o {\rm \TeX} >antapokr'inetai m`e t`hn
pr'otash {\tt insert>}$\,$. T'eloc, gr'afoume t`hn swst`h l'exh >el'egqou
{\tt \\hfil}. <Or'iste p~wc j`a bl'epame a>ut`o st`hn >oj'onh mac:

\beginuser
? h <CR>
The control sequence at the end of the top line
of your error message was never \\def'ed. If you have
misspelled it (e.g., `\\hobx'), type `I' and the correct
spelling (e.g., `I\\hbox'). Otherwise just continue,
and I'll forget about whatever was undefined.

? i <CR>
insert>\\hfil
[1]
\enduser

T`o {\tt [1]} po`u m~ac gr'afei t`o {\rm \TeX} shma'inei <'oti <h pr'wth
(ka`i monadik'h) sel'ida >'eqei <olo\-klhrwje~i ka`i t`o >apot'elesma
>'eqei katagrafe~i st`o >arqe~io {\rm DVI}\null.  Fusik'a, kat'opin
pr'epei n`a diorj'wsoume t`o >arqik`o >arqe~io m`e t`on k'wdik'a mac,
<'wste n`a m`hn m~ac parousiasje~i xan`a t`o >'idio sf'alma.

\subsection{<H lanjasm'enh grammatoseir`a}

T`o n`a gr'ayoume l'ajoc st`on k'wdik'a mac t`o >'onoma mi~ac
grammatoseir~ac prokale~i pro\-bl'h\-ma\-ta <'omoia m`e >eke~ina po`u
e>'idame parap'anw m`e t`ic lanjasm'enec l'exeic >el'egqou.  <'Omwc, t`o
m'h\-nu\-ma sf'al\-matoc e>~inai diaforetik`o ka`i k'apwc pol'uploko gi`a
<'enan >arq'ario.  >`Ac <upoj'esoume, p.q., <'oti st`on k'wdik'a mac
peri'eqetai <h >ak'oloujh gramm'h:
\vskip\baselineskip

\leftline{\tt \\font\\sf = cmss01}

\noindent
T`o l'ajoc mac e>~inai st`o <'oti, >ant`i gi`a {\tt cmss01}, j`a
>'eprepe n`a e>'iqame gr'ayei {\tt cmss10}.  <Or'iste t`o m'hnuma
sf'almatoc ka`i <h bo'hjeia po`u m~ac d'inei t`o {\rm \TeX}:

\beginuser \obeyspaces
! Font \\sf=cmss01 not loadable: Metric (TFM) file not found.
<to be read again>
\                   \\par
\\bye ->\\par
\            \\vfill \\supereject \\end
l.2 \\bye

? h <CR>
I wasn't able to read the size data for this font,
so I will ignore the font specification.
[Wizards can fix TFM files using TFtoPL/PLtoTF.]
You might try inserting a different font spec;
e.g., type `I\\font<same font id>=<substitute font name>'.
\enduser

T`o >arqe~io {\rm TFM} (<h >onomas'ia tou pro'erqetai >ap`o t`a >arqik`a
t~wn l'exewn {\rm \TeX{} font metric}), e>~inai <'ena bohjhtik`o
>arqe~io po`u qrhsimopoie~i t`o {\rm \TeX}\null.  Kat`a sun'epeia, t`o
per'iergo m'hnuma sf'almatoc po`u m~ac d'inei t`o {\rm \TeX} d`en
shma'inei t'ipota >'allo >ap`o t`o <'oti <h grammatoseir`a {\tt cmss01}
d`en <up'arqei st`o s'usthma to~u <upologist~h mac.

\subsection{Majhmatik`a qwr`ic ta'iri}

<'Ena >ak'oma pol`u koin`o sf'alma e>~inai t`o n`a qrhsimopoio~ume {\tt
\$} >`h {\tt \$\$} gi`a n`a stoiqeiojet'hsoume m'ia majhmatik`h
>'ekfrash ka`i kat'opin n`a xeqno~ume n`a kle'isoume t`hn >'ekfrash
a>ut`h m`e t`o >ant'istoiqo de'utero {\tt \$} >`h {\tt \$\$}.  T`o
>apot'elesma a>uto~u to~u sf'almatoc e>~inai <'o,ti >akolouje~i t`a
majhmatik`a n`a stoiqeiojete~itai >ep'ishc <wc m'ia majhmatik`h >'ekfrash;
ka`i t`o qeir'otero e>~inai <'oti <'otan t`o {\rm \TeX} sunant'hsei t`hn
>arq`h mi~ac n'eac majhmatik~hc >'ekfrashc, t'ote >ant`i n`a suneq'isei
n`a m~ac d'inei majhmatik`a s'umbola, j`a >arq'isei n`a m~ac
stoiqeiojete~i kanonik`o ke'imeno.  Beba'iwc, d`en qrei'azetai n`a
>anaferjo~ume st`o pl~hjoc t~wn mhnum'atwn sf'almatoc po~u genn~a <'ena
t'etoio sf'alma.  <'Opwc ka`i n`a >'eqei <h kat'astash to~u k'wdik'a mac,
t`o {\rm \TeX} j`a prospaj'hsei n`a >epa\-nor\-j'w\-sei t`o l'ajoc
paremb'allontac <'ena n'eo {\tt \$} >`h {\tt \$\$}.  >Epipl'eon, t`o
pr'oblhma j`a stamat'hsei m`e t`o t'e\-loc t~hc paragr'afou; t`o {\rm
\TeX} xekin~a a>utom'atwc t`hn stoiqeiojes'ia kanoniko~u keim'enou ---
ka`i >'oqi majhmatik~wn >ekfr'asewn --- <'otan sunat'hsei n'ea par'agrafo.

>`Ac koit'axoume t`on >ak'oloujo swst`o k'wdika ka`i t`o >apot'elesm'a
tou:  
\beginuser
Since \$f(x) > 0\$, \$a<b\$,  and \$f(x)\$ is continuous, we know that
\$\\int\_{}a\^{}b f(x)\\,dx >0\$.
\enduser

{\rm
Since $f(x) > 0$, $a<b$,  and $f(x)$ is continuous, we know that
$\int_a^b f(x)\,dx >0$.
} % end \rm 

>E`an parale'iyoume t`o de'utero s'umbolo to~u dolar'iou st`o {\tt
\$f(x)\$}, t'ote t`o {\rm \TeX} j`a m~ac d'wsei t`a >ak'olouja mhn'umata
sf'almatoc ka`i bo'hjeiac:

\beginuser \obeyspaces
! Missing \$ inserted.
<inserted text>
\               \$
<to be read again>
\                  \\intop
\\int ->\\intop
\             \\nolimits
l.2 \$\\int
\          \_{}a\^{}b f(x)\\,dx >0\$.
? h <CR>
I've inserted a begin-math/end-math symbol since I think
you left one out. Proceed, with fingers crossed.

?
\enduser

<H gramm`h po`u xekin~a m`e t`o jaumastik`o ({\tt !}\null) m~ac >exhge~i
t'i >'eqei sumbe~i.  <H gramm`h po`u xekin~a m`e t`o {\tt l.2} m~ac
de'iqnei t`hn gramm`h to~u k'wdik'a mac <'opou sk'ontaye t`o
{\rm\TeX}\null.  <'Opwc ka`i st`ic >'allec peript'wseic sf'almatoc,
>'etsi ka`i >ed~w t`o m'eroc t~hc gramm~hc to~u k'wdika po`u t`o {\rm
\TeX} di'abase qwr`ic kan'ena pr'oblhma parousi'azetai st`hn >oj'onh s`e
m'ia gramm`h (>'ewc t`o {\tt \\int}), ka`i st`hn <ep'omenh gramm`h
>akolouje~i t`o <up'oloipo m'eroc t~hc gramm~hc met`a t`o problhmatik`o
shme~io.  <'O,ti <'epetai kat'opin fa'inetai m'allon d'uskolo n`a
>exhghje~i.  A>ut`a t`a >endi'amesa mhn'umata de'iqnoun t'i sun'epeiec
e>~iqe t`o l'ajoc mac st`a >end'wtera to~u {\rm \TeX} (t`o {\rm \TeX}
e>~inai k'apwc pol'uploko!). <O n'eoc qr'hsthc to~u {\rm \TeX} mpore~i
n`a >agno'hsei t`a >endi'amesa mhn'umata sf'almatoc. 

>E`an >epitr'eyoume st`o {\rm \TeX} n`a diorj'wsei m'ono tou t`o
sf'alma mac, <or'iste t`o >apot'elesma:

{ \rm
Since $f(x) > 0$, $a<b$,  and $f(x) is continuous, we know that
\int_a^b f(x)\,dx >0$.
} % end \rm

T`o ke'imen'o mac tent'wnetai ka`i stoiqeiojete~itai m`e pl'agiouc
qarakt~hrec majhmatik~wn sumb'olwn qwr`ic >endi'amesa ken`a diast'hmata.
A>ut`o sumba'inei <'otan kanonik`o ke'imeno stoiqeiojete~itai <wc m'ia
majhmatik`h >'ekfrash. >E`an parathr'hsoume k'ati t'etoio st`o
stoiqeiojethm'eno mac >'entupo, s'igoura k'apou xeq'asame k'apoio mon`o
{\tt \$} >`h k'apoio dipl`o {\tt \$\$}.

\subsection{>Agk'ulec qwr`ic ta'iri}

E>~inai pol`u e>'ukolo  n`a xeq'asei kane`ic >`h n`a prosj'esei kat`a
l'ajoc parap'anw >agk'ulec <'otan fti'aqnei s'unola.  T`o >apot'elesma
mpore~i n`a e>~inai tele'iwc >an'wduno >'ewc katastrofik'o.  >`Ac
<upoj'esoume, gi`a par'adeigma, <'oti j'eloume n`a >'eqoume <'enan
t'itlo m`e >'entonouc t'upouc ka`i <'oti gr'afoume {\tt \lb\\bf A bold
title} xeqn'wntac t`hn >apara'ithth dexi`a >agk'ulh.  T`o >apot'elesma
j`a e>~inai <'olo t`o ke'imeno po`u >akolouje~i n`a stoiqeiojethje~i m`e
>'entonouc t'upouc, >en~w st`o t'eloc t`o {\rm \TeX} j`a paraponeje~i
<wc >ex~hc:

{\tt (\\end occurred inside a group at level 1)}

>E`an e>'iqame k'anei t`o >'idio sf'alma d'uo for'ec, dhl.\ >e`an
e>iqame d'uo >arister`ec >agk'ulec qwr`ic t`ic >ant'istoiqec dexi'ec,
t'ote t`o par'apono to~u {\rm \TeX} j`a >~htan t`o >ex~hc:

{\tt (\\end occurred inside a group at level 2)}

T`o {\rm \TeX} d`en mpore~i n`a katal'abei <'oti parale'ipontai m'ia >`h
periss'oterec dexi`ec >agk'ulec pa\-r`a m'onon <'otan fj'asei st`o
t'eloc to~u k'wdika qwr`ic n`a t`ic >'eqei sunant'hsei.  Sunep~wc, t`o
m'hnuma sf'al\-matoc d`en m~ac l'eei po~u k'aname t`o l'ajoc.  >E`an <h
j'esh <'opou j`a >'eprepe n`a e>'iqame j'esei t`hn dexi`a >agk'ulh d`en
e>~inai ka`i t'oso faner'h, t'ote m'ia l'ush st`on >entopism`o to~u
problhmatiko~u shme'iou e>~inai n`a gr'ayoume {\tt \\bye} st`hn m'esh
to~u k'wdika.  >E`an tr'eqontac xan`a t`o {\rm\TeX} sunant'hsoume t`o
>'idio sf'alma, t'ote t`o sf'alma mac br'isketai st`o pr~wto mis`o to~u
k'wdika (>ef'' <'oson t`hn de'uterh for`a t`o {\rm\TeX} >epexerg'asjhke
m'onon t`o pr~wto mis`o to~u k'wdika >'ewc t`o {\tt\\bye}). 
Metakin'wntac t`o {\tt\\bye} s`e diaforetik`ec j'eseic to~u k'wdika,
mporo~ume n`a >ento\-p'i\-sou\-me telik`a t`o >akrib`ec shme~io to~u
sf'almatoc.  >Ep'ishc, m'ia mati`a st`o stoiqeiojethm'eno ke'imeno
p'anta bohj~a st`on >entopism`o t'etoiwn sfalm'atwn ka`i poll~wn >'allwn
paroram'atwn.

O<i paraleip'omenec >arister`ec >agk'ulec e>~inai pi`o e>'ukolec st`on
>entopism'o touc.  <Or'iste <'enac k'wdikac d'uo gramm~wn ka`i t`a
sqetik`a mhn'umata sf'almatoc ka`i bo'hjeiac po`u d'inei t`o {\rm \TeX}: 

\beginuser
\\bf Here is the start\rb, but there is the finish.
\\bye
\enduser

\beginuser \obeyspaces
! Too many \rb's.
l.1 \\bf Here is the start\rb
\                          , but there is the finish.

? h <CR>
You've closed more groups than you opened.
Such booboos are generally harmless, so keep going.
\enduser

\noindent
Beba'iwc, e>~inai pol`u pijan`o <h gramm`h to~u k'wdika <'opou j`a
>'eprepe n`a e>'iqame j'esei t`hn >arister`h >agk'ulh n`a m`hn
sump'iptei m`e t`hn gramm`h <'opou t`o {\rm \TeX} br~hke t`o sf'alma.

<H par'aleiyh mi~ac >agk'ulhc st`on <orism`o mi~ac n'eac l'exhc
>el'egqou mpore~i n`a dhmiourg'hsei sobar'wtata probl'hmata.  >Ef''
<'oson <'enac t'etoioc <orism`oc mpore~i n`a perilamb'anei
peris\-s'o\-te\-rec >ap`o m'ia paragr'afouc, e>~inai pijan`o t`o {\rm
\TeX} n`a m`hn katal'abei t`o sf'alma ka`i n`a suneq'isei n`a
susswre'uei <ulik`o >ap`o t`on k'wdik'a mac s`e <'enan <orism`o d'iqwc
t'eloc.  A>ut`o mpore~i n`a pro\-ka\-l'esei >ak'oma ka`i <uperf'ortwsh
t~hc mn'hmhc to~u <upologist~h.  A>ut`o st`hn >argk`o to~u {\rm \TeX}
l'egetai {\tengs >aperi'oristoc <orism'oc} ({\rm runaway definion}).%
\TeXref{206}
<Or'iste <'ena par'adeigma <en`oc k'wdika pou peri'eqei <'enan t'etoio
problhmatik`o <orism'o:

\vbox{
\beginuser
\\def\\newword\lb the def
\\newword
\\bye
\enduser
}

\noindent
<Or'iste ka`i t`a sqetik`a mhn'umata sf'almatoc ka`i bo'hjeiac:

\beginuser \obeyspaces
Runaway definition?
->the def
! Forbidden control sequence found while scanning definition of \\newword.
<inserted text>
\                \rb
<to be read again>
\                   \\bye
l.3 \\bye

? h <CR>
I suspect you have forgotten a `\rb', causing me
to read past where you wanted me to stop.
I'll try to recover; but if the error is serious,
you'd better type `E' or `X' now and fix your file.

? <CR>
No pages of output.
\enduser

\noindent
Profan~wc t`o sf'alma mac st`hn per'iptwsh a>ut`h e>~inai sobar'o.  >E`an
sumbe~i st`hn >arq`h to~u k'wdika (<'opwc st`o parap'anw par'adeigma),
t'ote t`o {\rm \TeX} d`en j`a m~ac d'wsei o>'ute m'ia sel'ida
stoiqeiojethm'enou keim'enou!

>E`an parale'iyoume m'ia dexi`a >agk'ulh kat`a t`hn qr'hsh m'iac
makroentol~hc, t'ote <h problhmatik`h makroentol`h j`a stamat'hsei m`e t`o
t'eloc t~hc paragr'afou.  Gi`a par'adeigma, >e`an >'eqou\-me <or'isei {\tt
\\def\\newword\#1\lb$\ldots$\rb} ka`i kat'opin gr'ayoume st`on k'wdika 
{\tt \\newword\lb$\dots$ } pa\-ra\-le'i\-pontac t`hn dexi`a >agk'ulh, t'ote
st`hn qeir'oterh per'iptwsh j`a katastr'eyoume m'ia par'a\-grafo.%
\TeXref{205}

M`e l'iga l'ogia, <'otan >emfanisje~i <'ena sf'alma, kal`o e>~inai n`a
shmei'wsoume t`on >arijm`o t~hc gramm~hc <'opou t`o {\rm \TeX}
>ent'opise (>e`an >ent'opise) t`o sf'alma.  >Ep'ishc kal`o e>~inai n`a
shmei'wsoume ka`i t`hn gramm`h po`u xekin~a m`e t`o jaumastik`o ka`i
m~ac d'inei m'ia  s'untomh di'agnwsh to~u sf'almatoc.  >E`an t`o t'i
>'eqei sumbe~i d`en m~ac e>~inai xek'ajaro, mporo~ume >ep'ishc n`a
zht'hsoume >ap`o t`o {\rm \TeX} periss'oterec leptom'ereiec
plhktrolog'wntac {\tt h<CR>}.  St`ic peript'wseic mikr~wn sfalm'atwn,
t`o {\rm \TeX} mpore~i n`a bre~i k'apoia l'ush, <'otan >eme~ic
>apokrijo~ume m'onon m`e {\tt <CR>}.

                  
\section{Sk'abontac l'igo baj'utera}

St`o kef'alaio a>ut`o j`a >exet'asoume merik`a j'emata po`u m~ac
>epitr'epoun n`a qrhsimopoio~ume t`o {\rm \TeX} m`e megal'uterh
e>ukol'ia ka`i >apotelesmatik'othta.  Kaj`wc t`a >'entupa po`u j`a
stoiqeiojeto~ume j`a g'inontai <'olo ka`i megal'utera ka`i
poluplok'wtera, k'apoiec >'allec teqnik`ec j`a m~ac fano~un >exairetik`a
qr'hsimec. 

\subsection{Meg'ala ka`i mikr`a >arqe~ia}

T`o {\rm \TeX} >'eqei t`hn dunat'othta n`a diab'azei >all`a ka`i n`a
gr'afei >arqe~ia kaj`wc tr'eqei.  A>ut`h <h dunat'othta to~u {\rm \TeX}
m~ac >epitr'epei n`a qrhsimopoio~ume mikr`a >arqe~ia t`a <opo~ia e>~inai
pi`o e>'ukola st`hn qr'hsh touc. >'Etsi dhmiourgo~ume <'ena {\tengs
k'urio >arqe~io\/} >`h {\tengs >arqe~io--korm`o\/} m'esw| to~u <opo'iou
kalo~ume poll`a mikr'wtera >arqe~ia m`e t`hn seir`a po`u >eme~ic
<or'izoume.  A>ut`o t`o ke'imeno, p.q., >apotele~itai >ap`o
\ifwritinganswers
   \def\next{d'wdeka}%
\else
   \def\next{<'endeka}%
\fi
\next\
kef'alaia ka`i d'uo e>isagwg'ec.  >Epipl'eon, <up'arqoun ka`i k'apoioi
<orismo`i ({\rm macro}) po`u qrhsimopoio~untai s`e <'ola t`a kef'alaia. 
To`uc <orismo`uc mporo~ume n`a to`uc b'aloume s`e <'ena >arqe~io m`e
t`hn >onomas'ia, p.q., {\tt macros.tex}; t`ic e>isagwg`ec mporo~ume n`a
t`ic >'eqoume s`e d'uo >'alla kef'alaia, p.q., {\tt intro1.tex} ka`i
{\tt intro2.tex}; ka`i t`o k'aje kef'alaio mporo~ume n`a  t`o >'eqoume
s`e <'ena xeqwrist`o >arqe~io.  Gi`a n`a po~ume st`o {\rm \TeX} n`a
diab'asei <'ena >arqe~io, qrhsimopoio~ume t`hn l'exh >el'egqou {\tt
\\input}.  Genik'a, gr'afontac st`on k'wdik'a mac {\tt \\input
filename}, d'inoume st`o {\rm \TeX} t`hn >entol`h n`a diab'asei ka`i n`a
>epexergasje~i >am'eswc t`o >arqe~io m`e t`o >'onoma {\tt filename.tex}.
T`o telik`o >apot'elesma j`a e>~inai <'ena <enia~io >'entupo, <'opwc j`a
e>'iqame st`hn per'iptwsh po`u <o k'wdikac po`u peri'eqetai st`o
>arqe~io {\tt filename.tex} >apotelo~use m'eroc to~u >arqe'iou po`u
peri'eqei t`hn >entol`h {\tt \\input filename}.  T`o >arqe~io m`e t`hn
>entol`h {\tt \\input...} mpore~i n`a kale~i ka`i >'alla >arqe~ia
>ekt`oc to~u {\tt filename.tex}; ka`i >ak'oma t`o {\tt filename.tex}
mpore~i ka`i a>ut`o m`e t`hn seir'a tou n`a kale~i >'alla >arqe~ia. 
St`hn pi`o sunhjism'enh per'iptwsh <'omwc, fti'aqnoume <'ena ka`i m'ono
>arqe~io--korm`o t`o <opo~io kale~i t`a di'afora >'alla >arqe~ia po`u
peri'eqoun tm'hmata to~u keim'enou, <'opwc dhl.\ sumba'inei st`o
parak'atw par'adeigma: \toindex{input}

\beginuser
\\input macros
\\input intro1
\\input intro2
\\input sec1
\\input sec2
\\input sec3
\\input sec4
\\input sec5
\\input sec6
\\input sec7
\\input sec8
\\input sec9
\\input sec10
\\input sec11
\ifwritinganswers
   \def\next{\\input sec12}%
\else
   \def\next{\relax}%
\fi
\next
\enduser

<'Otan <h s'untaxh to~u keim'enou mac (t`o gr'ayimo to~u k'wdika
dhlad'h) d`en e>~inai >ak'oma pl'hrhc, mporo~ume n`a >epexergasjo~ume
m`e t`o {\rm \TeX} m'ono <'osa mikr`a >arqe~ia >'eqoume <oloklhr'wsei
j'etontac t`o s'umbolo to~u sqol'iou >empr`oc >ap`o k'aje gramm`h to~u
k'wdika po`u kale~i <'ena m`h <oloklhrwm'eno >arqe~io ({\rm commenting
out}).

<H l'exh >el'egqou {\tt \\input} m~ac >epitr'epei >ak'oma n`a
qrhsimopoio~ume >arqe~ia po`u peri'eqoun prosqediasm'enouc <orismo'uc
({\rm macro}).  Gi`a par'adeigma, mporo~ume n`a >'eqoume <'ena >arqe~io
m`e t`hn >onomas'ia {\tt memo.tex} po`u n`a peri'eqei m'onon <orismo`uc
gi`a t`hn stoiqeiojes'ia <upomnhm'atwn ({\rm memorandum}).  A>uto`i o<i
<orismo`i mpore~i n`a e>~inai o<i kajorismo`i t~wn diast'asewn {\tt
\\hsize}, {\tt \\vsize} ka`i >'allwn paromo'iwn param'etrwn, >`h mpore~i
n`a j'etoun a>ut'omata t`hn <'wra ka`i t`hn <hmeromhn'ia st`hn
>epikefal'ida to~u <upomn'hmatoc.  >Ap`o t`hn stigm`h po`u >'eqoume
<etoim'asei t`o >arqe~io {\tt memo.tex}, d`en qrei'azetai n`a
>epanalamb'anoume to`uc <orismo`uc k'aje for`a po`u <etoim'azoume <'ena
<up'omnhma; >arke~i ka`i m'ono n`a b'azoume t`hn >entol`h {\tt \\input
memo} st`hn pr'wth gramm`h k'aje n'eou <upomn'hmatoc.

<'Omwc prosoq'h: t`o k'aje >arqe~io po`u kalo~ume m`e {\tt\\input...}
d`en j`a pr'epei n`a peri'eqei t`hn l'exh >el'egqou {\tt\\bye}, giat`i
t`o {\rm \TeX} j`a stamat'hsei t`hn >epexergas'ia tou s''\NB{}a>ut`o
>akrib~wc t`o shme~io!

\exercise Dhmiourg~hste <'ena >arqe~io {\rm \TeX} po`u n`a kale~i <'ena
de'utero >arqe~io.  Dokim'aste n`a kal'este d'uo for`ec t`o de'utero
>arqe~io, gr'afontac t`hn >entol`h {\tt \\input...} d'uo for`ec >ent`oc
to~u pr'wtou.


\subsection{Megal'utera pak'eta {\lsecfont macro}}

Profan~wc, <h <etoimas'ia <orism~wn ({\rm macro}) po`u mporo~un n`a
qrhsimopoihjo~un gi`a poll`ec morf`ec >ent'upwn e>~inai >exairetik`a
qr'hsimh.  Gi`a par'adeigma, poll`a panepist'hmia >apaito~un o<i
diatrib`ec t~wn spoudast~wn n`a >'eqoun m'ia sugkekrim'enh (ka`i suqn`a
pol'uplokh) morf'h. <H s'untaxh m'iac {\tengs sullog~hc <orism~wn}, >`h
>alli~wc <'enoc {\tengs pak'etou} {\sl macro}, po`u n`a <ikanopoie~i
<'olouc to`uc kano\-ni\-smo`uc stoiqeiojes'iac <en`oc >ent'upou, e>~inai
doulei`a >ep'iponh ka`i t`o >arqe~io (>`h t`a >arqe~ia) po`u j`a
prok'uyoun mpore~i n`a >'eqei <uperbolik`a meg'alec diast'aseic. 
Beba'iwc, <o kaj'enac mpore~i n`a qrhsimopoi'hsei <'ena t'etoio pak'eto
m`e t`hn >entol`h {\tt \\input...}, <'opwc perigr'ayame pi`o p'anw. 
<Wst'oso, t`o {\rm \TeX} >'eqei k'ati kal'utero gi`a meg'ala pak'eta
{\rm macro}. 

<'Ena pak'eto {\rm macro} mpore~i n`a sumpuknwje~i s`e m'ia e>idik`h
morf`h po`u diab'azetai pol`u gr'hgora >ap`o t`o {\rm \TeX\null}.  T`o
sumpuknwm'eno >arqe~io >onom'azetai {\tengs >arqe~io morf~hc} (st`hn
gl'wssa to~u {\rm \TeX}, {\rm format file}). (T`o p~wc >akrib~wc e>~inai
a>ut`o t`o >arqe~io, >'eqei m'allon m'onon teqnik`o >endiaf'eron ka`i
gi'' a>ut`o d`en j`a m~ac >apasqol'hsei periss'otero.) T`o
>arqe~io morf~hc m~ac >epitr'epei n`a tr'eqoume t`o {\rm \TeX} >'eqontac
prokajor'isei poll`ec n'eec >akolouj'iec >el'egqou.  D`en j`a pr'epei
<wst'oso n`a xeqno~ume <'oti <orism'enec >akolouj'iec >el'egqou
>apotelo~un m'eroc t~hc kardi~ac to~u {\rm \TeX} ka`i gi`a t`on l'ogo
a>ut`o >apokalo~untai {\tengs prwt'ogonec} ({\rm primitive}).

<'O,ti perigr'ayame s`e <eto~uto t`o >egqeir'idio suqn`a >apokale~itai
{\rm plain \TeX} (<apl`o {\rm \TeX}), ka`i perilamb'anei prwt'ogonec
>entol`ec kaj`wc ka`i <'ena pak'eto {\rm macro} s`e <'ena >arqe~io
morf~hc po`u <onom'azetai {\tt plain.fmt}.  Sun'hjwc, t`o >arqe~io {\tt
plain.fmt} kale~itai a>ut'omata >ap`o t`o {\rm \TeX} k'aje for`a po`u
t`o tr'eqoume.  

Gi`a to`uc per'iergouc, <up'arqei <h l'exh >el'egqou {\tt \\show} po`u
d'inei t`o p~wc >akrib~wc <or'izetai m'ia >akolouj'ia >el'egqou.  P.q.,
<h >entol`h {\tt \\show\\centerline} j`a m~ac d'wsei st`hn >oj'onh ka`i
j`a katagr'ayei st`o sqetik`o >arqe~io {\tt .log} t`a >ak'olouja:

\beginliteral
> \centerline=macro:
#1->\line {\hss #1\hss }.
@endliteral

\noindent 
Mporo~ume >ak'oma n`a qrhsimopoi'hsoume t`hn >entol`h {\tt \\show...}
gi`a n`a >exet'asoume ka`i diko'uc mac <orismo'uc.  >Epipl'eon, <'otan
qrhsimopoio~ume poll`a pak'eta {\rm macro}, mporo~ume m`e t`hn >entol`h
{\tt \\show...} n`a >el'eg\-xoume >e`an k'apoio sugkekrim'eno {\rm
macro} >'eqei <orisje~i >`h >'oqi.

Sun'hjwc maz`i m`e k'aje >egkat'astash to~u {\rm\TeX} d'inetai ka`i t`o
pak'eto {\rm macro \LaTeX\null}.  A>ut`o t`o pak'eto m~ac >epitr'epei
n`a dhmiourgo~ume m`e e>ukol'ia (sqed`on a>ut'omata) e<uret'hria,
p'inakec perieqom'enwn ka`i bibliografiko`uc katal'ogouc.  M~ac par'eqei
>ep'ishc t`hn dunat'othta n`a paremb'aloume st`o >'entup'o mac <apl`ec
grafik`ec parast'aseic <'opwc k'uklouc, >elle'iyeic, e>uje~iec ka`i
b'elh.  T`o {\rm\LaTeX} >ak'oma qrhsimopoie~i e>idik`a prokajorism'ena
{\tengs >arqe~ia t'axhc\/} ({\rm class files}) ka`i {\tengs >arqe~ia
<'ufouc\/} ({\rm style files}) gi`a n`a d'wsei sugkekrim'enh <'ufoc
(morf'h) st`ic sel'idec to~u >ent'upou >an'alogac m`e t`hn qr'hsh tou
(p.q., bibl'io, >episthmonik`o >'arjro, >'ekjesh, k.>'a.)  >Arqe~ia
t'axhc ka`i <'ufouc <up'arqoun p'ara poll'a.  >Epipl'eon, poll`a
>episthmonik`a periodik`a d'eqontai >'arjra pr`oc dhmos'ieush <'opwc t`a
>'eqei <etoim'asei <o suggraf'eac touc s`e m'ia disk'eta, >ef'' <'oson
e>~inai gramm'ena s'umfwna m`e t`o {\rm\LaTeX} ka`i s`e sunduasm`o m`e
k'apoio sugkekrim'eno >arqe~io <'ufouc.  Mi~ac ka`i >'eqoume >'hdh
m'ajei >arket`a gi`a t`o {\rm\TeX}, t`o n`a per'asoume st`hn qr'hsh to~u
{\rm\LaTeX} d`en e>~inai kaj'olou d'uskolo.  Gi`a to`uc
>endiafer'omenouc <up'arqei <o <odhg`oc to~u {\rm \LaTeX} gramm'enoc
>ap`o t`on {\rm Leslie Lamport}, t`on sqediast`h a>uto~u to~u pak'etou:
{\sl \LaTeX: A document preparation system} ({\rm 2nd
edition})\fnote{\rm Addison-Wesley, Reading, Massachusetts, 1994, ISBN
0-201-52983-1.}.  >Ep'ishc, <o >Ap'ostoloc Sur'opouloc to~u
Dhmokr'iteiou Panepisthm'iou Jr'akhc >'eqei gr'ayei t`o pr~wto
<ellhnik`o >egqeir'idio gi`a t`o {\rm\LaTeX}\fnote{(({\sl\LaTeX})),
>Ekd'oseic Parathrht'hc, Jessalon'ikh 1998, {\rm ISBN} 960-260-990-7.}.

<H >Amerikanik`h Majhmatik`h <Etaire'ia ({\rm American Mathematical
Society}) qrhsimopoie~i t`o pak'eto {\rm macro \AMSTeX}  gi`a t`hn
stoiqeiojes'ia t~wn periodik~wn po`u >ekd'idei.  T`o >egqeir'idio
a>uto~u to~u pak'etou, gramm'eno >ap`o t`on {\rm Michael Spivak} m`e
t'itlo: {\sl The Joy of \TeX}\fnote{\rm American Mathematical
Society, 1986, ISBN 0-8218-2999-8.}, diat'ijetai >ap`o t`hn
>Amerikanik`h Majhmatik`h <Etaire'ia. 

>Ekt`oc >ap`o t`a parap'anw pak'eta {\rm macro}, <up'arqoun ka`i poll`a
>'alla.  Sun'hjwc diat'ijentai >ap`o to`uc sqediast'ec touc dwre`an >`h
s`e pol`u qamhl`h tim'h, ka`i s`e poll`ec peript'wseic
>apodei\-kn'u\-ontai >exairetik`a qr'hsima (p.q., gi`a n`a
stoiqeiojet'hsoume n'otec s`e pent'agrammo, n`a stoiqeiojet'hsoume s`e
gl~wssec p'era >ap`o t`hn >Agglik'h, k.lp.).  <O >organism`oc {\rm \TeX{}
Users Group} >anakoin'wnei suqn`a st`a periodik`a po`u >ekd'idei t`hn
parous'ia n'ewn pak'etwn {\rm macro}.

\subsection{<Oriz'ontiec ka`i katak'orufec gramm`ec}

T`o n`a j'esoume <oriz'ontiec ka`i katak'orufec e>uje~iec gramm`ec st`o
>'entupo po`u stoiqeiojeto~ume e>~inai >exairetik`a e>'ukolo m`e t`o
{\rm \TeX\null}. <'Otan, k'ajwc gr'afoume st`on k'wdik'a mac <apl`o
ke'imeno, paremb'aloume t`hn l'exh >el'egqou {\tt \\hrule}, t`o
>apot'elesma j`a e>~inai t`o {\rm\TeX} n`a diak'oyei t`hn par'agrafo s`e
>eke~ino >akrib~wc t`o shme~io, n`a j'esei m'ia <oriz'ontia e>uje'ia
gramm`h m'hkouc >'isou m`e t`hn tr'eqousa tim`h to~u {\tt \\hsize}, ka`i
kat'opin n`a suneq'isei st`hn stoiqeiojese'ia m'iac n'eac >ar'adac. 
E>~inai dunat`o >eme~ic n`a kajor'isoume t`o m~hkoc t~hc <oriz'ontiac
e>uje'iac, p.q., st`a $5\,\rm cm$, gr'afontac {\tt \\hrule width 5 cm}.
>Ep'ishc m`e t`ic >entol`ec {\tt \\vskip} >`h {\tt \\bigskip}, mporo~ume
n`a j'esoume k'apoio ken`o di'asthma >ep'anw >`h k'atw >ap`o t`hn
<oriz'ontia e>uje'ia.  <Or'iste <'ena par'adeigma: 

\beginuser
\\parindent = 0 pt \\parskip = 12 pt
Here is the text before the hrule.
\\bigskip
\\hrule width 3 in
And here is some text after the hrule.
\enduser

\noindent po`u m~ac d'inei

{\rm
\parindent = 0 pt
Here is the text before the hrule.
\bigskip
\hrule width 3 in 
And here is some text after the hrule.
}

St`hn pragmatik'othta a>ut`h <h <oriz'ontia e>uje'ia >'oqi m'onon >'eqei
m~hkoc tre~ic >'intsec, >all`a >ex <orismo~u >'eqei <'uyoc 0,4
stigm`ec (t'oso >ekte'inetai >ep'anw >ap`o t`hn basik`h gramm`h <'opou
stoiqeiojete~itai) ka`i b'ajoc 0 stigm'ec (t'oso >ekte'inetai k'atw
>ap`o t`hn basik'h gramm`h <'opou stoiqeiojete~itai).  Mporo~ume n`a
metab'aloume <opoiad'hpote >ap`o t`ic param'etrouc m~hkoc, <'uyoc >`h
b'ajoc.  >'Etsi mporo~ume n`a metab'aloume t`o prohgo'umeno par'adeigma
<wc >ex~hc:

{\tt \\hrule width 3 in height 2 pt depth 3 pt }

\noindent gi`a n`a l'aboume

{\rm
\parindent = 0 pt
Here is the text before the hrule.
\bigskip
\hrule width 3 in height 2 pt depth 3 pt
And here is some text after the hrule.
}

\noindent
T`ic tre~ic param'etrouc {\tt width}, {\tt height} ka`i {\tt depth}
mporo~ume n`a t`ic <or'isoume m`e <opoia\-d'h\-pote seir'a.%
\toindex{hrule}
\TeXref{221--222}

>Anal'ogwc, mporo~ume n`a >enj'esoume m'ia katak'orufh e>uje'ia gramm`h
kajor'izontac --- >e`an kr'inoume >apara'ithto --- t`a >ant'istoiqa {\tt
width}, {\tt height} ka`i {\tt depth}.%
\TeXref{221--222}
<'Omwc, s`e >ant'ijesh m`e t`ic <oriz'ontiec e>uje~iec, m'ia katak'orufh
e>uje'ia d`en sunep'agetai t`hn >'enarxh n'eac >ar'adac.  >Ex <orismo~u
t`o pl'atoc thc j`a e>~inai 0,4 stigm`ec ka`i t`o <'uyoc thc <'oso t`o
<'uyoc t~hc >ar'adac <'opou paremb'alletai.  Sunep~wc <o k'wdikac
\toindex{vrule}

\beginuser
Here is some text before the vrule
\\vrule\\
and this follows the vrule.
\enduser

\noindent j`a m~ac d'wsei

{\rm 
Here is some text before the vrule
\vrule\
and this follows the vrule.
} % end \rm

\exercise Sqedi'aste m`e t`o {\rm \TeX} tre~ic <oriz'ontiec e>uje~iec
gramm`ec po`u n`a >ap'eqoun 15 stigm`ec <h m'ia >ap`o t`hn >'allh, n`a
>'eqoun m~hkoc 3 >'intsec ka`i n`a br'iskontai m'ia >'intsa pi`o m'esa
(pi`o dexi'a) >ap`o t`o >arister`o perij'wrio.
                     
O<i <oriz'ontiec ka`i katak'orufec e>uje~iec mpore~i n`a >'eqoun pol`u
periss'oterec qr'hseic >ap`o <'osec martur~a <h >onomas'ia touc.
Gi`a par'adeigma, <o k'wdikac

\beginuser
\\noindent
Name:\ \\vrule height 0 pt depth 0.4 pt width 3 in
\enduser

\noindent j`a m~ac d'wsei

\noindent
{\rm
Name: \vrule height 0 pt depth 0.4 pt width 3 in
} % end \rm

\exercise Sqedi'aste m`e t`o {\rm \TeX} t`o >ak'oloujo tetr'agwno (k'aje
br'ogqoc to~u tetra\-g'w\-nou >'eqei >embad`o $1\,\rm cm^2$):
\medskip
\settabs \+ \hskip 1cm&\hskip 1 cm&\hskip 1 cm& \cr
\moveright 2 in
\vbox{
\hrule width 3 cm
\+  \vrule height 1 cm & \vrule height 1 cm & \vrule height 1 cm
  & \vrule height 1 cm \cr
\hrule width 3 cm
\+  \vrule height 1 cm & \vrule height 1 cm & \vrule height 1 cm
  & \vrule height 1 cm \cr
\hrule width 3 cm
\+  \vrule height 1 cm & \vrule height 1 cm & \vrule height 1 cm
  & \vrule height 1 cm \cr
\hrule width 3 cm
}

\subsection{Pla'isia >ent`oc plais'iwn}

E>'idame >'hdh st`hn suz'hths'h mac gi`a t`hn di'ataxh t~wn >ar'adwn
(kef'alaio\NB{}3) <'oti t`a pla'isia {\rm vbox} ka`i {\rm hbox} mpore~i
n`a m~ac parousi'asoun probl'hmata {\rm underfull} kai {\rm overfull},
dhl.\ e>'ite n`a e>~inai miso'adeia m`e <uperbolik`a meg'ala ken`a
diast'hmata e>'ite n`a xeqeil'izoun p'era >ap`o t`a <'ori'a touc.
S`e <eto'uth t`hn par'agrafo j`a >exet'asoume a>ut`a t`a <oriz'ontia
>`h katak'orufa pla'isia pi`o le\-pto\-mer~wc.  Genik'a, >`ac po~ume
<'oti m`e t`a pla'isia {\rm vbox} ka`i {\rm hbox} e>~inai dunat`o n`a
stoiqeiojethje~i m'ia sel'ida m`e ke'imeno s`e di'aforouc sunduasmo`uc
sq'hmatoc ka`i j'eshc.

<'Ena <oriz'ontio pla'isio dhmiourge~itai m`e t`hn >entol'h: {\tt
\\hbox\lb $\ldots$\rb}\null.  T`o <ulik'o (ke'imeno) po`u peri'eqetai
metax`u t~wn >agkul~wn to~u {\rm hbox} jewre~itai <wc m'ia mon'ada ka`i
d`en mpore~i n`a dia\-spa\-sje~i.  A>ut`o shma'inei p`wc <'otan j'eloume
k'ati n`a mpe~i s`e m'ia xeqwrist`h >ar'ada, mporo~ume n`a t`o j'esoume
>ent`oc <en`oc {\rm hbox} ka`i j`a parame'inei <enia~io.  E>~inai
dunat`o n`a kajor'isoume t`o m'egejoc <en`oc <orizont'iou
plais'iou.\TeXref{64--66}  >'Etsi m`e t`on k'wdika {\tt \\hbox to 5
cm\lb contents of the box\rb}, j`a l'aboume stoiqeiojethm'enh t`hn
fr'ash (({\rm contents of the box})) >ent`oc <en`oc  plais'iou <oliko~u
m'hkouc $5\,\rm cm$.  <'Omwc, m`e t`on tr'opo a>ut'o, e>~inai pol`u
pijan`o t`o {\rm \TeX} n`a paraponeje~i gi`a {\rm underfull} >`h {\rm
overfull}.  M'ia <apl`h l'ush gi`a n`a xeper'asoume <'ena pr'oblhma {\rm
underfull} e>~inai n`a qrhsimopoi'hsoume {\tt\\hfil} po`u j`a
>aporrof'hsei t`o pleon'azon ken`o di'asthma.  <'Otan d`en kajor'izoume
t`ic diast'aseic to~u {\rm hbox}, t'ote t`o {\rm\TeX} fti'aqnei <'ena
<oriz'ontio pla'isio t'etoiou m'hkouc po`u n`a qwr'aei m'olic t`o
ke'imeno po`u <up'arqei >ent`oc to~u plais'iou. \toindex{hbox}   

Par'omoia, mporo~ume n`a fti'axoume katak'orufa pla'isia ({\rm vbox})
qrhsimopoi'wntac t`hn >entol'h: {\tt \\vbox\lb $\ldots$\rb}.  T`o
>endiaf'eron <'omwc t~wn katakor'ufwn plais'iwn br'isketai st`o <'oti
<'otan <'ena {\rm vbox} peri'eqei periss'otera >ap`o <'ena {\rm hbox},
t'ote t`a {\rm hbox} topojeto~untai <'ena >ep'anw st`o >'allo. 
>Anal'ogwc, <'otan <'ena {\rm hbox} peri'eqei periss'otera >ap`o <'ena
{\rm vbox}, t'ote t`a {\rm vbox} topojeto~untai t`o <'ena pl'ai st`o
>'allo.  >`Ac <upoj'esoume <'oti >'eqoume j'esei tr'ia {\rm hbox} >ent`oc
<en`oc {\rm vbox}
\toindex{vbox}

\beginuser \obeyspaces
\\vbox\lb
\      \\hbox\lb{}Contents of box 1\rb
\      \\hbox\lb{}Contents of box 2\rb
\      \\hbox\lb{}Contents of box 3\rb
\      \rb
\enduser

\noindent T`o >apot'elesma e>~inai
\vskip \baselineskip
\vbox{ \rm
      \hbox{Contents of box 1}
      \hbox{Contents of box 2}
      \hbox{Contents of box 3}
      }

Kat'opin >`ac <upoj'esoume <'oti >'eqoume <'ena >'allo {\rm vbox}:

\beginuser \obeyspaces
\\vbox\lb
\      \\hbox\lb{}Contents of box 4\rb
\      \\hbox\lb{}Contents of box 5\rb
\      \rb
\enduser

A>ut`a t`a d'uo {\rm vbox} mporo~un n`a topojethjo~un s`e <'ena {\rm
hbox}.  T`o >apot'elesma j`a e>~inai t`a d'uo {\rm vbox} n`a mpo~un t`o
<'ena d'ipla st`o >'allo.  Dhlad`h <o k'wdikac

\beginuser \obeyspaces
\\hbox\lb
\     \\vbox\lb
\           \\hbox\lb{}Contents of box 1\rb
\           \\hbox\lb{}Contents of box 2\rb
\           \\hbox\lb{}Contents of box 3\rb
\           \rb
\     \\vbox\lb
\           \\hbox\lb{}Contents of box 4\rb
\           \\hbox\lb{}Contents of box 5\rb
\           \rb
\     \rb
\enduser

\noindent
d'inei
\vskip \baselineskip
\hbox{
     \vbox{\rm 
           \hbox{Contents of box 1}
           \hbox{Contents of box 2}
           \hbox{Contents of box 3}
           }
     \vbox{\rm 
           \hbox{Contents of box 4}
           \hbox{Contents of box 5}
          }
     }

>Ax'izei n`a parathr'hsoume <'oti t`a d'uo {\rm vbox} topojeto~untai
>'etsi <'wste t`o k'atw m'eroc touc n`a br'isketai st`hn >'idia
e>uje'ia.  >Epipl'eon, st`hn >arq`h k'aje >ar'adac, <'opwc >ep'ishc ka`i
metax`u t~wn {\rm vbox}, <up'arqei <'ena mikr`o ken`o di'asthma.  <H
a>it'ia po`u prokale~i t`hn >emf'anish a>ut~wn t~wn ken~wn d`en e>~inai
profan'hc. T`o ken`o di'asthma metax`u t~wn d'uo {\rm
vbox} pro'erqetai >ap`o t`o ken`o di'asthma >`h t`on qarakt'hra {\tt
<CR>} (po`u <up'arqoun >all`a d`en fa'inontai) met`a t`hn dexi`a
>agk'ulh ({\tt \rb}) po`u kle'inei t`o pr~wto {\tt \\vbox} st`on
k'wdika.  Paromo'iwc, t`o mikr`o ken`o di'asthma st`hn >arq`h k'aje
>ar'adac to~u pr'wtou {\rm vbox}, pro'erqetai >ap`o k'apoio ken`o
di'asthma >`h {\tt <CR>} >am'eswc met`a t`hn pr'wth >arister`h >agk'ulh
to~u {\tt \\hbox} po`u peri'eqei t`a d'uo {\tt \\vbox}.  <H parous'ia
a>ut~wn t~wn ken~wn mpore~i >'allote n`a m~ac e>~inai qr'hsimh --- p.q.,
>e`an j'eloume n`a m`hn koll'aei t`o <'ena {\rm vbox} >ep'anw st`o
>'allo --- ka`i >'allote >epiz'hmia --- p.q., n`a m~ac prokale~i
<uperbolik`h >ara'iwsh t~wn {\rm vbox}.  >E`an j'eloume n`a m`hn
<up'arqoun t'etoiou e>'idouc ken`a diast'hmata, j`a pr'epei >am'eswc
met`a >ap`o k'aje >agk'ulh po`u >emfan'izetai m'onh thc st`o dexi`o
>'akro mi~ac gramm~hc to~u k'wdika, n`a gr'afoume t`on qarakt'hra to~u
sqol'iou {\tt \%} ({\rm commenting out}).  Mporo~ume loip`on n`a
>all'axoume t`on parap'anw k'wdika j'etontac {\tt \%} met`a t`hn pr'wth
>arister`h >agk'ulh to~u {\tt \\hbox} ka`i met`a t`hn dexi`a >agk'ulh
po`u kle'inei t`o pr~wto {\tt \\vbox}:

\beginuser \obeyspaces
\\hbox\lb\%
\     \\vbox\lb
\           \\hbox\lb{}Contents of box 1\rb
\           \\hbox\lb{}Contents of box 2\rb
\           \\hbox\lb{}Contents of box 3\rb
\           \rb\%
\     \\vbox\lb
\           \\hbox\lb{}Contents of box 4\rb
\           \\hbox\lb{}Contents of box 5\rb
\           \rb
\     \rb
\enduser

\noindent
T`o >apot'elesma j`a e>~inai n`a m`hn <up'arqoun ken`a diast'hmata st`hn
>arq`h k'aje >ar'adac to~u pr'wtou {\rm vbox}, ka`i t`a d'uo {\rm vbox}
n`a koll'hsoun t`o <'ena pl'ai st`o >'allo:
\vskip \baselineskip
\hbox{\rm%
     \vbox{
           \hbox{Contents of box 1}
           \hbox{Contents of box 2}
           \hbox{Contents of box 3}
           }%
     \vbox{
           \hbox{Contents of box 4}
           \hbox{Contents of box 5}
          }
     }

Mporo~ume n`a j'esoume parap'anw ken`o di'asthma, p.q., $1\,\rm cm$,
metax`u t~wn {\rm vbox} gr'afontac {\tt \\hskip 1 cm} >an'amesa st`a
{\tt \\vbox\lb...\rb} to~u k'wdika.  >Ak'oma, mporo~ume n`a
e>ujugramm'isoume t`a {\rm vbox} <wc pr`oc t`hn koruf'h touc
qrhsimopoi'wntac t`hn l'exh >el'egqou {\tt \\vtop} >ant`i t~hc {\tt
\\vbox}.  <Or'iste t`o >apot'elesma a>ut~wn t~wn >allag~wn:

\toindex{vtop}
\vskip \baselineskip
\hbox{
\rm  \vtop{
           \hbox{Contents of box 1}
           \hbox{Contents of box 2}
           \hbox{Contents of box 3}
           }
     \hskip 1 cm
     \vtop{
           \hbox{Contents of box 4}
           \hbox{Contents of box 5}
          }
     }

M`e kat'allhlo sunduasm`o {\rm vbox}, {\rm hbox}, katak'orufwn ka`i
<orizont'iwn e>ujei~wn, mporo~ume n`a perikle'isoume <'ena m'eroc to~u
keim'enou mac m`e <orat`o pla'isio.  P~wc mporo~ume n`a pet'uqoume
a>ut`o >akrib~wc?  <'Enac tr'opoc e>~inai n`a j'esoume t`o ke'imeno
>ent`oc <en`oc {\rm hbox} t`o <opo~io >'eqei st`hn >arq`h ka`i st`o
t'eloc tou (dhl.\ st`a >arister`a ka`i st`a dexi'a tou) m'ia mikr`h
katak'orufh e>uje'ia ({\tt \\vrule}).  Kat'opin, j'etoume t`o {\rm hbox}
>ent`oc <en`oc {\rm vbox} po`u kal'uptetai >ap`o >ep'anw ka`i k'atw m`e
d'uo <oriz'ontiec e>uje~iec ({\tt \\hrule}).  <Or'iste <o sqetik`oc
k'wdikac:

\beginuser \obeyspaces
\\vbox\lb
\      \\hrule
\      \\hbox\lb\\vrule{} The text to be boxed \\vrule\rb
\      \\hrule
\     \rb
\enduser

\noindent ka`i t`o >apot'elesma:
\vskip \baselineskip
\vbox{
      \hrule
      \hbox{\vrule\rm The text to be boxed \vrule}
      \hrule
     }

>'Etsi pr'agmati lamb'anoume ke'imeno perigegramm'eno >ap`o <'ena
parallhl'ogrammo, >all`a t`o a>isjhtik`o >apot'elesma d`en e>~inai ka`i
>idia'itera <ikanopoihtik'o; t`o ke'imen'o mac fa'inetai <uperbolik`a
strimwgm'eno!  (<'Omwc, d`en fta'iei t`o {\rm \TeX}; m~ac >'edwse
>akrib~wc <'o,ti to~u zht'hsame!)  Mporo~ume n`a belti'wsoume t`o
parap'anw par'adeigma j'etontac <'ena {\tt \\strut} st`hn >arq`h to~u
{\rm hbox}, >'etsi <'wste t`o {\rm hbox} n`a g'inei l'igo yhl'otero ka`i
makr'utero.  Dhlad'h:
\vskip \baselineskip
\vbox{
      \hrule
      \hbox{\strut \vrule{} \rm The text to be boxed \vrule}
      \hrule
     }

\def\boxtext#1{%
\vbox{%
      \hrule
      \hbox{\strut \vrule{} \rm #1 \vrule}%
      \hrule
     }%
}

\exercise Giat'i pr'epei n`a b'aloume >epipl'eon ken`o di'asthma p'anw
ka`i k'atw >ap`o t`o ke'imeno ka`i >'oqi dexi`a ka`i >arister'a tou?

\exercise Qrhsimopoi~hste t`hn parap'anw m'ejodo gi`a n`a j'esete m'ia
kentrarism'enh >ar'ada >ent`oc <en`oc parallhlogr'ammou m`e per'imetro
po`u j`a >ekte'inetai >ap`o t`o >arister`o >'ewc t`o dexi`o perij'wrio.

\exercise J'etontac >enn'ea mikr`a parallhl'ogramma t`o <'ena >ep'anw
st`o >'allo, >`h t`o <'ena d'ipla st`o >'allo, kataskeu'aste t`o
<ep'omeno magik`o tetr'agwno:
\vskip\baselineskip

\moveright 2 in \vbox{\offinterlineskip
\hbox{\boxtext 6\boxtext 1\boxtext 8}
\hbox{\boxtext 7\boxtext 5\boxtext 3}
\hbox{\boxtext 2\boxtext 9\boxtext 4}
}

\exercise Parathr~hste <'oti o<i >eswterik`ec e>uje~iec to~u parap'anw
magiko~u tetrag'wnou >'eqoun dipl'asio p'aqoc >ap`o t`ic >exwterik'ec.
>'Iswc >ep'ishc n`a <up'arqei ka`i <'ena mikr`o ken`o di'a\-sthma >eke~i
po`u dia\-stau\-r'wnontai o<i >eswterik`ec e>uje~iec.  Diorj~wste a>ut`a
t`a probl'hmata to~u magiko~u te\-tra\-g'wnou.

\def\boxtext#1 {%
\vbox{%
      \hrule
      \hbox{\strut \vrule #1\vrule}%
      \hrule
     }
}

\exercise <Etoim'aste <'ena {\rm macro} m`e t`o >'onoma { \tt
\\boxtext\#1\lb$\ldots$\rb}, t`o <opo~io j`a j'etei a>utom'a\-twc t`o
ke'imeno po`u perikle'ietai >ap`o t`ic >agk'ulec >ent`oc <en`oc
paral\-lhlogr'am\-mou.  Do\-ki\-m'a\-ste t`hn >efarmog`h a>uto~u to~u
{\rm macro} m`e m'ia pr'otas'h sac st`hn <opo~ia k'aje de'uterh l'exh
j`a e>~inai >ent`oc <en`oc mikro~u <orato~u plais'iou.  Beba'iwc
\boxtext d`en 
e>~imai 
\boxtext ka`i 
t'oso 
\boxtext s'igouroc 
<'oti 
\boxtext t`o
a>isjhtik`o 
\boxtext >apot'elesma 
j`a 
\boxtext s~ac 
>enjousi'asei.
Parathr~hste p~wc e>ujugramm'izontai <orizont'iwc a>ut`a t`a pla'isia
>ep'anw st`hn gramm`h b'ashc t~hc >ar'adac.

\def\boxtext#1 {%
\lower 3.5pt \hbox{%
    \vbox{%
         \hrule
         \hbox{\strut \vrule #1\vrule}%
         \hrule   
        }
    }
}

<H metak'inhsh plais'iwn {\rm hbox} >`h {\rm vbox} pr`oc t`a p'anw,
k'atw, >arister`a >`h dexi`a >ep`i t~hc sel'idac to~u >ent'upou, mpore~i
n`a g'inei pol`u e>'ukola.  Gi`a par'adeigma, >e`an j'eloume n`a
meta\-kin'h\-soume <'ena {\rm vbox} m'ia >'intsa pr`oc t`a dexi'a,
>arke~i n`a gr'ayoume st`on k'wdika {\tt \\moveright 1 in
\\vbox\lb\dots\rb}.  Gi`a n`a t`o metakin'hsoume pr`oc t`a >arister'a,
gr'afoume {\tt \\moveleft}.  Paro\-mo'iwc, mporo~ume n`a
meta\-kin'h\-soume <'ena {\rm hbox} pr`oc t`a p'anw >`h pr`oc t`a k'atw
qrhsimopoi'wntac t`ic >ant'istoiqec l'exeic >el'egqou {\tt \\raise} >`h
{\tt \\lower}.
\toindex{moveright}
\toindex{moveleft}
\toindex{raise}
\toindex{lower}

\exercise Diorj~wste t`o {\rm macro} {\tt \\boxtext} t~hc prohgo'umenhc
>'askhshc >'etsi <'wste <'olec o<i l'exeic m'iac >ar'adac (>ent`oc ka`i
>ekt`oc plais'iou) n`a br'iskontai >ep`i t~hc >'idiac <oriz'ontiac
e>uje'iac. (<Up'odeixh: >ex <orismo~u t`o b'ajoc <en`oc {\tt \\strut}
e>~inai 3,5 stigm'ec.)  M`e t`o diorjwm'eno {\rm macro} j`a pr'epei
na mpore~ite n`a gr'afete m'ia pr'otash s`an ka`i t`hn >ak'oloujh: 
{\rm I'm \boxtext not quite \boxtext sure why \boxtext 
someone would \boxtext do this \boxtext since the \boxtext result 
is \boxtext pretty strange} (>`h 
\boxtext <ellhnik'a:
beba'iwc 
\boxtext d`en 
e>~imai 
\boxtext ka`i 
t'oso 
\boxtext s'igouroc 
<'oti 
\boxtext t`o
a>isjhtik`o 
\boxtext >apot'elesma 
j`a 
\boxtext s~ac 
>enjousi'asei).

>E`an e>~inai >apara'ithto mporo~ume n`a gem'isoume <'ena {\rm hbox} m`e
m`e m'ia <oriz'ontia e>uje'ia >`h suneq'omenec tele~iec.  A>ut`o
mporo~ume n`a t`o pet'uqoume qrhsimopoi'wntac t`ic l'exeic >el'egqou
{\tt \\hrulefill} >`h {\tt \\dotfill} >ent`oc to~u {\tt
\\hbox\lb\dots\rb}, <'opwc st`o <ep'omeno par'adeigma:

\beginuser
\\hbox to 5 in\lb Getting Started\\hrulefill 1\rb
\\hbox to 5 in\lb All Characters Great and Small\\hrulefill 9\rb
\\hbox to 5 in\lb The Shape of Things to come\\hrulefill 17\rb
\\hbox to 5 in\lb No Math Anxiety Here!\\hrulefill 30\rb
\enduser

\noindent <O k'wdikac a>ut`oc m~ac d'inei:
\vskip \baselineskip

\hbox to 5 in{\rm Getting Started\hrulefill 1}
\hbox to 5 in{\rm All Characters Great and Small\hrulefill 9}
\hbox to 5 in{\rm The Shape of Things to come\hrulefill 17}
\hbox to 5 in{\rm No Math Anxiety Here!\hrulefill 30}

>E`an st`hn j'esh to~u {\tt \\hrulefill} gr'ayoume {\tt \\dotfill}, t`o
>apot'elesma j`a e>~inai:
\vskip \baselineskip

\hbox to 5 in{\rm Getting Started\dotfill 1}
\hbox to 5 in{\rm All Characters Great and Small\dotfill 9}
\hbox to 5 in{\rm The Shape of Things to come\dotfill 17}
\hbox to 5 in{\rm No Math Anxiety Here!\dotfill 30}

\exercise <Etoim'aste t`hn kefal`h m'iac sel'idac >ent`oc <en`oc
paral\-lh\-logr'am\-mou plai\-s'i\-ou po`u n`a moi'azei m''\NB{}a>ut`hn
t~wn sel'idwn <eto'utou to~u >egqeirid'iou.


\section{P'ec mou to <ellhnik'a!}

\noindent  
[Shme'iwsh to~u metafrast~h, po`u e>~inai ka`i <o suggraf'eac a>uto~u
to~u kefala'iou: {\tengs <Eto~uto t`o kef'alaio >'eqei grafe~i gi`a
<'ena sugkekrim'eno pak'eto <ellhniko~u {\sl \TeX}\null.  <'Osoi
qrhsimopoio~un >'alla pak'eta <ellhniko~u {\sl \TeX}, mporo~un n`a
>all'axoun t`o ke'imeno ka`i t`a parade'igmata s'umfwna m`e t`ic dik'ec
touc >an'agkec.  <'Omwc d`en j`a pr'epei n`a >all'axoun t`hn dom`h to~u
kefala'iou.}]

S`e <'ola t`a prohgo'umena kef'alaia >exet'asame t`ic basik`ec >arq`ec
to~u {\rm \TeX} >anafer'omenoi <'omwc p'anta s`e parade'igmata
stoiqeiojes'iac keim'enwn po`u sthr'izontai st`o latinik`o >alf'a\-bhto. 
E>~inai pol`u pijan`o k'apoioi >anagn~wstec n`a bar'ejhkan <'olec
a>ut`ec t`ic >exhg'hseic per`i to~u {\rm \TeX} m`e >agglik`a
parade'igmata.  <'Omwc, <'opwc xananaf'erame, t`o {\rm \TeX}
prwtosqedi'asjhke gi`a t`hn stoiqeiojes'ia >agglik~wn >ent'upwn.  >All`a
p~wc t`a kataf'ernei m`e t`a <ellhnik'a?

T`o {\rm\TeX} mpore~i n`a stoiqeiojet'hsei s`e <opoiad'hpote gl'wssa
--- >ak'oma ka`i kin'ezika --- >arke~i n`a >'eqei t`ic kat'allhlec
grammatoseir'ec (ka`i >'iswc ka`i merik`ec <odhg'iec sullabismo~u).  O<i
pr~wtec grammatoseir`ec <ellhnik~wn stoiqe'iwn gi`a stoiqeiojes'ia
kanoniko~u keim'enou (ka`i >'oqi majhmatik~wn sumb'olwn) m`e t`o {\rm
\TeX}, sqedi'asjhkan >ap`o t`on {\rm Sylvio Levy} st`o Panepist'hmio
{\rm Princeton} t~wn H.P.A\null.  <O {\rm Levy} gi`a t`on sqediasm`o
t~wn grammatoseir~wn tou bas'isjhke sto`uc palio`uc qarakt~hrec {\rm
Didot}, po`u e>~inai gnwsto`i sto`uc <'ellhnec tupogr'afouc <wc
((<apl'a)).  T`ic grammatoseir`ec to~u {\rm Levy} belt'iwse l'igo
>arg'otera <o Gi'annhc Qaral'ampouc st`o Panepist'hmio t~hc {\rm Lille}
t~hc Gall'iac, >en`w kat'opin <o Kwst`hc Druller'akhc st`o {\rm Imperial
College} to~u Lond'inou t`ic sugk'entrwse maz`i m`e k'apoia {\rm macro}
s`e <'ena pak'eto m`e t`o >'onoma \greektex.  

>Ekt`oc t~wn grammatoseir~wn t~wn {\rm Levy}--Qaral'ampouc, teleuta~ia
parousi'asjhkan ka`i meriko`i >'alloi <ellhniko`i t'upoi.  >Ep'ishc,
>ekt`oc to~u \greektex\ <up'arqoun ka`i >'alla <ellhnik`a pak'eta {\rm
\TeX} po`u diaf'eroun s`e k'apoia shme~ia t`o <'ena >ap`o t`o
>'allo\fnote{<'Osoi >anagn~wstec >'eqoun pr'osbash st`o {\rm Internet}
ka`i x'eroun p~wc n`a qeirisjo~un <'ena >arqe~io \LaTeX, mporo~un n`a
>anazht'hsoun periss'oterec plhrofor'iec gi`a <el\-lh\-nik`ec
grammatoseir`ec ka`i pak'eta {\rm \TeX} st`o >arqe~io {\tt
greekinf2.ltx} t`o <opo~io peri'eqei t`o >'arjro: {\rm I. Dimakos,
``It's all Greek\TeX\ to me: An updated summary of all available \TeX\
and \LaTeX\ tools'', (1996)}. T`o >arqe~io a>ut`o mpore~i n`a t`o l'abei
kane`ic m`e {\tt ftp} >ap`o t`on kat'alogo {\tt tex-archive/help/greek}
t~wn k'ombwn {\tt ftp.tex.ac.uk} ka`i {\tt ftp.dante.de}.}.  <Wst'oso
st`ic <ep'omenec sel'idec j`a >exet'asoume t`hn stoiqeiojes'ia
<ellhnik~wn keim'enwn m'onon m`e t`o \greektex, t`o <opo~io mporo~ume
n`a t`o qrhsimopoi'hsoume s`e <opoiod'hpote <upologist`h m`e
<opoiod'hpote leitourgik`o s'usthma (>ak'oma ka`i st`hn per'iptwsh po`u
d`en <up'arqoun o<i <ellhniko`i qarakt~hrec t~hc >oj'onhc).  

\subsection{<H pi`o <apl`h l'ush}

<H pi`o <apl`h l'ush (>all`a >'oqi ka`i <h pi`o komy'h) gi`a t`hn
stoiqeiojes'ia <ellhnik~wn keim'enwn e>~inai n`a fort'wsoume ka`i n`a
qrhsimopoi'hsoume t`ic <ellhnik`ec grammatoseir`ec po`u pijan`o n`a
<up'arqoun st`on <upologist'h mac, <'opwc j`a k'aname ka`i gi`a
<opoiesd'hpote latinik`ec gramma\-to\-seir'ec (kef'alaio\NB{}2). >E`an
d`en <up'arqoun <ellhnik`ec grammatoseir'ec, kal`o j`a >~htan n`a t`ic
>ana\-zh\-t'h\-soume e>'ite m'esw| f'ilwn ka`i gnwst~wn e>'ite m'esw|
to~u {\rm Internet}.

>E`an st`on <upologist'h mac <up'arqoun o<i grammatoseir`ec to~u
\greektex, t'ote j`a pr'epei n`a <up'arqoun k'apoia >arqe~ia m`e t`o
>'onoma {\tt kdgr10.mf}, {\tt kdgr10.tfm}, k.lp.  St`hn per'iptwsh
a>ut`h gi`a n`a fort'wsoume to`uc <aplo`uc <ellhniko`uc t'upouc s`e
m'egejoc 10 stigm~wn <wc m'ia gram\-mato\-seir`a m`e t`o >'onoma {\tt
\\tengr}, gr'afoume st`on k'wdik'a mac:

\beginliteral
\font\tengr=kdgr10 scaled \magstep0
@endliteral

\noindent 
Sunolik'a, o<i diaj'esimec grammatoseir`ec (t'upoi) to~u \greektex\
e>~inai >enn'ea <'opwc fa'inetai st`on <ep'omeno p'inaka:

\maketable [<Ellhnik`ec grammatoseir`ec to~u 
                  \lsecfont G\bf REE\lsecfont K\TeX]
\halign{%
\strut \tt # \hfil & \quad\hfil # \hfil \cr
\tengb >Onoma & \tengb T'upoc \cr
\noalign{\hrule} \noalign{\smallskip}
\ kdrg10 & <apl'a ({\rm Didot}) 10 stigm~wn \cr
\ kdrg9 & {\ninegr <apl'a ({\rm Didot}) 9 stigm~wn} \cr
\ kdrg8 & {\eightgr <apl'a ({\rm Didot}) 8 stigm~wn} \cr
\ kdbf10 & {\tengb >'entona <apl`a 10 stigm~wn} \cr
\ kdbf9 & {\ninegb >'entona <apl`a 9 stigm~wn} \cr
\ kdbf8 & {\eightgb >'entona <apl`a 8 stigm~wn} \cr
\ kdsl10 & {\tengs pl'agia <apl`a 10 stigm~wn} \cr
\ kdti10 & {\tengi pl'agia--kalligrafik'a (yeudo"italik'a) 10 stigm~wn} \cr
\ kdtt10 & {\tengt grafomhqan~hc 10 stigm~wn} \cr
      }

\noindent
<'Olouc to`uc parap'anw t'upouc mporo~ume n`a to`uc fort'wsoume s`e
<opoiod'hpote m'egejoc <'opwc k'aname ka`i gi`a t`on t'upo {\tt
\\tengr}.  P.q., <h <ep'omenh gramm`h k'wdika d'inei st`o {\rm \TeX}
t`hn >entol`h n`a fort'wsei pl'agiouc <ellhniko`uc t'upouc s`e m'egejoc
12 stigm~wn <wc m'ia grammatoseir`a m`e t`hn >onomas'ia {\tt \\bgrsl}:

\beginliteral
\font\bgrsl=kdsl10 scaled \magstep1
@endliteral

>Ef'' <'oson >'eqoume fort'wsei to`uc <ellhniko`uc t'upouc po`u
>epijumo~ume n`a qrhsimopoi'hsoume gi`a t`hn stoiqeiojes'ia to~u
keim'enou mac, t'ote mpo\-ro~u\-me n`a gr'ayoume

\beginuser
\\tengr Kalhm'era, k'osme!
\enduser

\noindent
gi`a n`a l'aboume

Kalhm'era, k'osme!

St`o parap'anw par'adeigma, bl'epoume <'oti st`on k'wdik'a mac gr'afoume
t`o <ellhnik`o ke'imeno m`e latiniko`uc qarakt~hrec, ka`i telik`a st`o
>'entup'o mac lamb'anoume <ellhniko'uc.  T`o par'adoxo a>ut`o
>ofe'iletai st`on tr'opo m`e t`on <opo~io >'eqoun kwdikopoihje~i o<i
<ellhnik`ec grammatoseir`ec to~u {\rm \TeX}\null.  O<i sqediast`ec t~wn
<ellhnik~wn grammatoseir~wn ka`i t~wn <ellhnik~wn pak'etwn {\rm \TeX},
je'wrhsan kaj~hkon touc n`a sebasjo~un t`hn basik`h >arq`h to~u {\rm
\TeX} s'umfwna m`e t`hn <opo'ia k'aje >arqe~io {\tt .tex} (dhl.\
>arqe~io po`u peri'eqei k'wdika {\rm \TeX}) j`a pr'epei n`a mpore~i n`a
metaferje~i >ap`o <upologist`h s`e <upologist`h (p.q., m'esw| {\rm
e-mail}) qwr`ic kan'ena pr'oblhma.  Prokeim'enou loip`on n`a mpore~i
k'apoioc >ap`o t`hn <Ell'ada n`a ste'ilei <'ena >arqe~io {\tt .tex}
po`u gr'afei, p.q., s`e <'enan {\rm Macintosh}, s`e <'enan <upologist`h
st`ic H.P.A., po`u tr'eqei m`e {\rm UNIX} ka`i po`u d`en >'eqei
to`uc <ellhniko`uc qarakt~hrec gi`a t`hn >oj'onh, j`a pr'epei t`o
>arqe~io a>ut`o n`a peri'eqei latiniko`uc ka`i m'onon qarak~hrec. (Gi`a
<'osouc kat'eqoun k'ati periss'otero >ap`o <upologist'ec, a>ut`o
shma'inei <'oti t`o >arqe~io {\tt .tex} d`en j`a pr'epei n`a peri'eqei
qarakt~hrec m`e kwdik`o {\rm ASCII} megal'utero to~u 127.)  Sunep~wc,
<'otan qrhsimopoio~ume to`uc t'upouc to~u \greektex\ gi`a t`hn
stoiqeiojes'ia <ellhniko~u keim'enou m`e t`o {\rm\TeX}, j`a pr'epei n`a
gr'afoume st`on k'wdika t`o <ellhnik`o ke'imeno m`e latiniko`uc
qarakt~hrec s'umfwna m`e t`hn >ak'oloujh >antistoiq'ia:
\medskip
\centerline{\vbox{%
\halign{%
# & # & # & # & # & # & # & # & # & # & % 10
# & # & # & # & # & # & # & # & # & # & % 20 
# & # & # & # & #\cr% 25 
\tt a & \tt b & \tt g & \tt d & \tt e & 
\tt z & \tt h & \tt j & \tt i & \tt k & % 10
\tt l & \tt m & \tt n & \tt x & \tt o & 
\tt p & \tt r & \tt s & \tt t & \tt u & % 20
\tt f & \tt q & \tt y & \tt w & \tt c\cr% 25 
a & b & g & d & e & z & h & j & i & k & % 10
l & m & n & x & o & p & r & s & t & u & % 20 
f & q & y & w & c\cr% 25 
}%
}%
}

O<i t'onoi ka`i t`a pne'umata stoi\-qei\-o\-jeto~untai gr'afontac st`on
k'wdika t`a kat'allhla dia\-kritik`a shme~ia >empr`oc >ap`o t`a
fwn'henta.  P.q., m`e {\tt >'a}, lamb'anoume:\NB{}>'a, dhl.\ t`o >'alfa
m`e yi\-l`h ka`i >oxe'ia.  Genik'wtera, gi`a n`a l'aboume <'ena fwn~hen
m`e t'ono ka`i pne~uma >`h dialutik'a, st`on k'wdika gr'afoume pr~wta
t`o pne~uma >`h t`a dialutik'a, met`a t`on t'ono ka`i met`a t`o
fwn~hen.

<H perispwm'enh st`hn per'iptwsh to~u \greektex\ parousi'azei m'ia
>idiaiter'othta, >epeid`h lamb'anetai m`e t`on qarakt'hra {\tt
\~{}}\null.  St`o kef'alaio\NB{}2 e>'idame <'oti <o qarakt'hrac a>ut`oc
gi`a t`o {\rm \TeX} e>~inai e>idik`oc ka`i shma'inei s'undesmo, dhl.\
>adi'akopto ken`o di'asthma.  Gi`a n`a >apof'ugoume t`a mplex'imata,
>e`an qrhsimopoio~ume t`ic <ellhnik`ec grammatoseir`ec to~u \greektex,
j`a pr'epei <'opou <up'arqei <ellhnik`o polutonik`o ke'imeno n`a
>apo\-ener\-gopoio~ume pr~wta t`on s'undesmo {\tt \~{}} qrhsimopoi'wntac
t`hn l'exh >el'egqou {\tt\\catcode}.\TeXref{43--49} <H l'exh >el'egqou
{\tt\\catcode} m~ac >epitr'epei n`a >al\-l'a\-zou\-me t`hn shmas'ia to~u
k'aje qarakt'hra to~u k'wdik'a mac.  St`o <ep'omeno par'adeigma,
gr'afontac {\tt\\catcode`\\\~{}=12}, d'inoume st`o {\rm\TeX} n`a
katal'abei <'oti <o qarakt'hrac {\tt\~{}} d`en e>~inai pl'eon e>idik'oc;
l'igo pi`o k'atw, gr'afontac {\tt\\catcode`\\\~{}=13
\\def\~{}\lb\\penalty10000\\\NB{}\rb}, >epanor'izoume t`on
qara\-kt'hra {\tt\~{}} <wc e>idik`o po`u <ermhne'uetai >ap`o t`o {\rm\TeX}
<wc >adi'akopto ken`o di'asthma: \toindex{catcode}

\beginliteral @obeyspaces
\tengr                     % We want to typeset greek text
@                           % using GreeKTeX fonts.
\catcode`\~=12             % So we de-activate the tie ~
\def\NB{\penalty10000\ }   % and we define \NB as non-breakable space.
Kal~wc ton!  
S''\NB{}t`o ''pa!
\rm                        % Now we want to switch back to roman.
\catcode`\~=13             % We re-activate the tie ~
\def~{\penalty10000\ }     % and we define it again as non-breakable space.
Hello Mr.~Jones!
@endliteral

\noindent
<O parap'anw k'wdikac d'inei:

Kal~wc ton! S''\NB{}t`o ''pa! {\rm Hello Mr.\NB{}Jones!}

Beba'iwc, t`o monadik`o s'umfwno po~u pa'irnei tonik`o shme~io e>~inai
t`o r`o t~wn >Arqa'iwn <El\-lh\-ni\-k~wn po`u mpore~i n`a e>~inai das`u
>`h yil'o, p.q., (({\tengs T`a p'anta <re~i\/})).  <'Oso gi`a t`hn
<upogegramm'enh po`u mpa'inei merik`ec for`ec k'atw >ap`o t`o {\tengs
a}, t`o {\tengs h\/} ka`i t`o {\tengs w}, >arke~i n`a gr'ayoume m'ia
katak'orufh gramm`h {\tt |} {\tengs met`a} t`o fwn~hen.  <O parak'atw
p'inakac d'inei t`a <'ola t`a tonik`a shme~ia po`u mporo~ume n`a
l'aboume m`e to`uc t'upouc to~u \greektex:

\maketable[Tonik`a shme~ia <ellhniko~u keim'enou]
\halign{%
\strut \ # \hfil & \quad \tt # \hfil & \quad # \hfil \cr
\tengb Shme~io & \tengb K'wdikac {\bf\TeX} & \tengb >Apot'elesma \cr
\noalign{\hrule} \noalign{\smallskip}
>oxe'ia         & M'h!             & M'h! \cr
bare'ia         & T`a yhl`a boun`a & T`a yhl`a boun`a \cr
perispwm'enh    & P~wc t`a p~ac?   & P~wc t`a p~ac? \cr
yil`h           & >'Afhs'e me, >'afhs'e me! & >'Afhs'e me, >'afhs'e me! \cr
dase'ia         &  D`en >'eqw >'allh <upomon'h! 
                                   & D`en >'eqw >'allh <upomon'h! \cr
<upogegramm'enh & T~w| kair~w| >eke'inw| 
                                   & T~w| kair~w| >eke'inw| \cr
dialutik`a      & T`o pro"i`on to~u Ma"'iou 
                                   & T`o pro"i`on to~u Ma"'iou \cr
     } % end halign

T`a shme~ia st'ixhc t~wn <ellhnik~wn keim'enwn e>~inai par'omoia m`e
a>ut`a t~wn >agglik~wn, m`e <orism'enec >exai\-r'e\-seic. <H
>antistoiq'ia k'wdika <ellhnik~wn shme'iwn st'ixhc ka`i >apotel'esmatoc
e>~inai a>ut`h po`u d'inetai st`on <ep'omeno p'inaka (prosoq`h st`hn
diafor`a >'anw tele'iac ka`i >erwthmatiko~u):

\maketable[Shme~ia st'ixhc <ellhniko~u keim'enou]
\halign{%
\strut
# \hfil & \quad \tt # & \quad # 
&\quad # \hfil &\quad \tt # & \quad # \cr
\noalign{\hrule}\noalign{\smallskip}
tele'ia                  & . & . &
k'omma                   & , & , \cr
>'anw tele'ia            & ; & ; &
>'anw ka`i k'atw tele'ia & : & : \cr
jaumastik`o              & ! & ! &
>erwthmatik`o            & ? & ? \cr
>arister`h >ap'ostrofoc  & `{}` & `` &
dexi`a >ap'ostrofoc      & '{}' & '' \cr
>arister`a e>isagwgik`a  & (( & (( &
dexi`a e>isagwgik`a      & )) & )) \cr 
      } % end halign

\noindent
St`hn per'iptwsh po`u qreiaste~i n`a b'aloume dipl`a >agglik`a
e>isagwgik`a s`e <ellhnik`o ke'imeno, j`a pr'epei n`a >all'axoume
proswrin`a s`e latiniko`uc t'upouc.  P.q., m`e t`on k'wdika 

\beginuser
((T'i \lb\\rm  ``\rb bl~hma\lb\\rm ''\rb\ po`u e>~isai!)), to~u f'wnaxe.
\enduser

\noindent
lamb'anoume: 

((T'i {\rm ``}bl~hma{\rm ''} po`u e>~i\-sai!)), to~u f'wnaxe.

\subsection{Gi`a k'ati kal'utero}

<O Qaral'ampouc, >ekt`oc t~wn grammatoseir~wn, >'eqei <etoim'asei ka`i
m'ia seir`a >ap`o <orismo`uc ({\rm macro}) ka`i kan'onec sullabismo~u
gi`a stoiqeiojes'ia <ellhnik~wn keim'enwn.  T`a {\rm macro} to~u
Qaral'ampouc >'eqoun >elafr`a beltiwje~i ka`i sugkentrwje~i st`o
\greektex\ s`e <'ena >arqe~io m`e t`o <'onoma {\tt greektex.tex}.  T`o
>arqe~io {\tt greektex.tex} mporo~ume n`a t`o qrhsimopoi'hsoume gi`a n`a
>apof'ugoume t`a fjhn`a k'olpa gi`a t`hn perispwm'enh po`u m'olic
perigr'ayame. >Arke~i ka`i m'ono n`a po~ume st`o {\rm\TeX} n`a diab'asei
a>ut`o t`o >arqe~io pr`in >arq'isei n`a stoiqeiojete~i <ellhnik'a.
Gr'afoume loip'on {\tt \\input greektex} st`hn pr'wth gramm`h to~u
k'wdik'a mac.  Kat'opin <'opote j'eloume n`a stoiqeiojet'hsoume
<ellhnik`o ke'imeno, qrhsimopoio~ume t`o peri\-b'al\-lon {\rm greek},
xekin'wntac m`e {\tt\\begingreek} ka`i telei'wnontac m`e
{\tt\\endgreek}.  P.q., <o parak'atw k'wdikac \toindex{begingreek}
\toindex{endgreek}

\beginuser
\\input greektex
We start typesetting an English text, melang\\'e avec un peu de
Fran\\c\lb c\rb ais, etc.\\ etc...
but at some point we switch to Greek: 
\\begingreek
Kalhm'era, k'osme!
\\endgreek
\enduser

\noindent
m~ac d'inei:

{\rm We start typesetting an English text, melang\'e avec un peu de
Fran\c{c}ais, etc.\ etc\dots\ but at some point we switch to Greek: 
} % end roman
Kalhm'era, k'osme!

>Ent`oc to~u perib'allontoc {\rm greek}, mporo~ume n`a qrhsimopoi'hsoume
t`ic l'exeic >el'egqou {\tt \\gr}, {\tt \\sl}, {\tt \\bf}, {\tt \\tt}
>`h {\tt \\it}, gi`a n`a l'aboume <aplo`uc, pl'agiouc, >'entonouc, t~hc
grafomhqan~hc >`h pl'agiouc--kalligrafiko`uc <ellhniko`uc t'upouc
>ant'istoiqa.  M'esa st`o perib'allon {\rm greek} mpo\-ro~u\-me >ak'oma n`a
stoiqeiojet'hsoume ka`i k'ati st`o latinik`o >alf'abhto; o<i l'exeic
>el'egqou {\tt \\rm}, {\tt \\lsl}, {\tt \\lbf}, {\tt \\ltt} >`h {\tt
\\lit} m~ac d'inoun latiniko`uc qarakt~hrec {\rm roman}, pl'agiouc {\rm
roman}, grafomhqan~hc ka`i {\rm italic} >ant'istoiqa.  <'Otan e>'imaste
>ekt`oc to~u perib'allontoc {\rm greek}, o<i l'exeic >el'egqou gi`a t`hn
>allag`h t'upou, <'opwc {\tt \\sl}, k.lp., d'inoun m'onon to`uc
>ant'istoiqouc pl'agiouc k.lp.\ latiniko`uc t'upouc.  <Or'iste <'ena
par'adeigma
\toindex{gr}

\beginuser
This text is mixed \lb\\sl English\rb{} and Greek.
\\begingreek
A>ut`o t`o ke'imeno e>~inai >an'amikto \lb\\sl <ellhnik`o\rb{} ka`i
\lb\\lsl English\rb.
A>ut`o t`o ke'imeno e>~inai >an'amikto \lb\\sl <ellhnik`o\rb{} ka`i
\lb\\lsl English\rb.
\\endgreek
\enduser

\noindent
<O k'wdikac a>ut`oc d'inei

{\rm This text is mixed {\sl English} and Greek.} A>ut`o t`o ke'imeno
e>~inai >an'amikto {\tengs <ellhnik`o} ka`i {\sl English}.
{\rm This text is mixed {\sl English} and Greek.} A>ut`o t`o ke'imeno
e>~inai >an'amikto {\tengs <ellhnik`o} ka`i {\sl English}.

<H suqn`h >epan'alhyh t~wn l'exewn >el'egqou {\tt \\begingreek} ka`i
{\tt \\endgreek} mpore~i n`a e>~inai >idia'itera kourastik'h.  Gi`a
e>ukol'ia, t`o \greektex\ m~ac d'inei t`hn dunat'othta n`a t`ic
>anti\-kata\-st'h\-soume m`e t`o s'umbolo to~u dolar'iou ({\tt \$}) >`h
m`e t`o s'umbolo t~hc katak'orufhc gram\-m~hc ({\tt |})\null.   <H
>anti\-kat'a\-stash a>ut`h g'inetai gr'afontac st`on k'wdika {\tt
\\greekdelims\lb dollar\rb} >`h  {\tt \\greekdelims\lb bar\rb}
>ant'i\-stoiqa.  <'Omwc ka`i o<i d'uo peript'wseic >apaito~un prosoq'h. 

St`hn per'iptwsh po`u qrhsimopoio~ume t`o {\tt \$} <wc >'endeixh >arq~hc
ka`i t'elouc <ellhniko~u keim'enou, <'otan j'eloume n`a gr'ayoume
majhmatiko`uc t'upouc j`a pr'epei n`a qrhsimopoi'hsoume t`ic l'exeic
>el'egqou {\tt \\math}, {\tt \\display} ka`i {\tt \\enddisplay}. Gi`a
par'adeigma, m`e {\tt \\math a = b \\math}, lam\-b'a\-noume >ent`oc
st'iqou: $a = b$, >en`w m`e {\tt \\display a = b .\ \\enddisplay},
lam\-b'a\-noume >ent`oc plais'iou:
$$ a = b .$$ 
>E`an qrhsimopoio~ume t`hn katak'orufh gramm'h ({\tt |}) <wc >'endeixh
>arq~hc ka`i t'elouc <ellhniko~u keim'enou, j`a >'eqoume duskol'ia n`a
gr'ayoume k'apoio >arqa~io ke'imeno >`h ke'imeno t~hc kajare'uousac m`e
<upogegramm'enec.  Gi'' aut'o, st`hn per'iptwsh polutoniko~u keim'enou,
e>~inai protim'wtero n`a <or'izoume {\tt \\greekdelims\lb dollar\rb}. 
<Or'iste <'ena k'apwc >anorj'ografo par'adeigma: 
\toindex{greekdelims}
\toindex{math}
\toindex{display}
\toindex{enddisplay}

\beginliteral
\input greektex
\greekdelims{dollar}
@null
This is a latin text: \math a \neq b \math.
Mr.~Jones Mr.~Jones @dots Mr.~Jones Mr.~Jones.
$  <Ell~hnik`oc >ano<rj~wgrafw|c: \math a = b \math\ >`h 
\display a = b. \enddisplay $ 
Now we continue in english!
@endliteral

\noindent 
<O k'wdikac a>ut`oc d'inei

{\rm
This is a latin text $ a \neq b $.
Mr.\NB Jones Mr.\NB Jones Mr.\NB Jones Mr.\NB Jones Mr.\NB Jones
Mr.\NB{}Jones Mr.\NB{}Jones Mr.\NB Jones Mr.\NB{}Jones \dotfill\ 
Mr.\NB{}Jones.
{\tengr <Ell~hnik`oc >ano<rj~wgrafw|c $ a = b $ >`h }
$$ a = b. $$
{\rm  Now we continue in english!}
} % end and roman

T'eloc, t`o \greektex\ >ekt`oc >ap`o t`hn l'ush m`e t`o {\tt
greektex.tex}, m~ac prosf'erei ka`i k'ati >ak'oma kal'utero: <'ena
sumpuknwm'eno >arqe~io morf~hc m`e t`o >'onoma {\tt greek.fmt}. 
Mporo~ume loip`on n`a po~ume st`o {\rm \TeX} n`a >epexergasje~i <'ena
>arqe~io mac, p.q., t`o {\tt mygrtext.tex}, <wc >ex~hc:

\itemitem{} $>$ {\tt tex \&greek mygrtext}

\noindent
St`hn per'iptwsh a>ut`h d`en qrei'azetai n`a <up'arqei st`o >arqe~io
{\tt mygrtext.tex} <h >entol'h: {\tt \\input greektex}.  T`a {\rm macro}
to~u {\tt greektex.tex} d'inontai ka`i >ap`o t`o >arqe~io morf~hc {\tt
greek.fmt}.  >Epi\-pl'eon, t`o {\tt greek.fmt} peri'eqei ka`i <odhg'iec
sullabismo~u <ellhniko~u keim'enou, k'ati po`u d`en m~ac prosf'erei
<h l'ush m`e t`o {\tt greektex.tex}.  Gi`a t`on l'ogo a>ut`o e>~inai
protim'wterh <h l'ush m`e t`o >arqe~io morf~hc {\tt greek.fmt},
>idiait'erwc <'otan pr'okeitai n`a stoiqeiojet'hsoume meg'ala <ellhnik`a
ke'imena.


\exercise Stoiqeiojet~hste t`o >ak'oloujo <ellhnik`o ke'imeno, >afo~u
pr~wta fort'wsete kat'allhla t`ic <ellhnik`ec grammatoseir'ec. 

{\narrower\noindent
>Epip'onwc d`e h<ur'isketo, di'oti o<i par'ontec to~ic >'ergoic
<ek'astoic o>u ta>ut`a per`i a>ut~wn >'elegon, >all'' <wc <ekat'erwn tic
e>uno'iac >`h mn'hmhc >'eqei. (Joukud'idhc, bibl'io\NB{}I, {\rm
xxii},\NB{}3) 

} % end narrower

\exercise Stoiqeiojet~hste t`o >ak'oloujo <ellhnik`o ke'imeno:

{\narrower\noindent K'amfjht'i moi pr`oc to`uc stenagmo`uc t~hc
kard'iac, <o kl'inac to`uc O>urano`uc t~h| >af'atw| Sou ken'wsei.
({\tengs T`o Trop'arion t~hc Kassian~hc\/})

} % end narrower

\exercise Stoiqeiojet~hste t`o >ak'oloujo <ellhnik`o ke'imeno:

{\narrower\noindent Fobhj'hkame n'' >agapo~ume --- m`h gel'asoume to`uc
>'allouc, m`h m~ac gel'asoun\dots\ (Dh\-m'h\-trhc Qatz~hc, {\tengs T`o
dipl`o bibl'io}, 2h >'ekdosh, >Ekd.\ ((Ke'imena)), >Aj'hna 1977,
sel.\NB112)

}

\subsection{K'apoioi sp'anioi <ellhniko`i qarakt~hrec}

O<i grammatoseir`ec to~u \greektex\ peri'eqoun ka`i tre~ic qarakt~hrec
po`u qrh\-si\-mo\-poi\-o~untai sp'ania. Pr'okeitai gi`a t`o d'igamma
(\digamma),  t`o k'oppa (\Koppa, \koppa, \varkoppa) ka`i t`o samp'i
(\sampi).  O<i d'uo teleuta'ioi qrhsime'uoun gi`a t`hn stoiqeiojes'ia
>arijm~wn s'umfwna m`e t`o s'usthma t~wn >alexandrin~wn majhmatik~wn. 
S'umfwna m`e a>ut`o t`o s'usthma, t`o st'igma >antistoiqe~i st`o <'exi,
t`o k'oppa st`o >enen'hnta ka`i t`o samp`i st`o >enniak'osia.  T`o
d'igamma qrhsimopoie~itai sp'ania st`hn stoiqeiojes'ia >arqa'iac
<ellhnik~hc lurik~hc po'ihshc.

Gi`a n`a l'aboume a>uto`uc to`uc qarakt~hrec, j`a pr'epei n`a to`uc
<or'isoume qrhsimopoi'wntac t`hn l'exh >el'egqou {\tt\\char}.%
\TeXref{43--49}
\toindex{char}

\beginliteral@obeyspaces
\def\digamma{\char'020}    % digamma (or former 6)
\def\Koppa{\char'022}      % capital qoppa (or 90)
\def\koppa{\char'023}      % small qoppa (or 90)
\def\varkoppa{\char'021}   % small qoppa variance (or 90)
\def\sampi{\char'024}      % sampi (or 900)
\def\numbertick{\char'003} % upper tick for ordinal Greek numbers
\def\pretick{\char'004}    % lower tick for Greek thousands
@endliteral

\noindent
Dustuq~wc, t`o st'igma le'ipei >ap`o t`ic grammatoseir`ec to~u
\greektex\null.  >Epeid`h <'omwc moi'azei m`e t`o telik`o s'igma,
m'ia k'apwc pr'oqeirh l'ush e>~inai n`a <or'isoume a>ut`on t`on
qarakt'hra <wc >ex~hc:

\beginliteral@obeyspaces
\def\stigma{\char'143}     % stigma (or latter 6)
@endliteral

Sto`uc parap'anw <orismo`uc sumperil'abame ka`i d'uo t'onouc, <'enan
>an'wtero ({\tt\\numbertick}) ka`i <'enan kat'wtero ({\tt\\pretick})
po`u mpa'inoun p'isw >ap`o taktik`a >arijmhtik`a ka`i >empr`oc >ap`o
qili'adec >ant'istoiqa.  Kat'opin mporo~ume n`a gr'ayoume:
{\tt\lb\\tengr \\pretick a\\sampi\\stigma \\numbertick\rb = 1906}, gi`a
n`a l'aboume: {\tengr\pretick a\sampi\stigma\numbertick} = 1906.  >`H
>ak'oma mporo~ume n`a gr'ayoume: {\tt\lb\\tengr ((t`on
\\digamma\lb\rb`on pa~ida kale~i)) (Sapf'w)\rb}, gi`a n`a l'aboume:
((t`on \digamma{}`on pa~ida kale~i)) (Sapf'w).
\toindex{digamma}
\toindex{stigma}
\toindex{koppa}
\toindex{varkoppa}
\toindex{Koppa}
\toindex{sampi}

<'Otan qrhsimopoio~ume t`o >arqe~io {\tt greektex.tex} >`h t`o >arqe~io
morf~hc {\tt greek.fmt}, o<i parap'anw <orismo`i d`en qrei'azetai n`a
<up'arqoun st`on k'wdik'a mac.  O<i >'idiec l'exeic >el'egqou
<or'izontai >'hdh st`o {\tt greektex.tex} ka`i st`o {\tt greek.fmt}, m`e
m'ia <'omwc shmantik`h diafor'a:  <Wc st'igma <or'izetai lanjasm'ena <h
parallag`h to~u mikro~u k'oppa \varkoppa, >en~w <h >entol`h
{\tt\\varkoppa} d`en <up'arqei.  St`hn per'iptwsh a>ut`h e>~inai
protim'wtero n`a >epanor'isoume t`o {\tt\\stigma} ka`i t`o
{\tt\\varkoppa} <'opwc k'aname parap'anw.

\exercise Stoiqeiojet~hste: >Egr'afh >en >'etei
\pretick{}a\sampi\varkoppa\stigma\numbertick

\exercise Stoiqeiojet~hste to`uc >ak'oloujouc d'uo st'iqouc to~u >Alka'iou
(7oc a>'i.\ p.Q.)\ po`u dias'wjhkan >'ewc t`ic <hm'erec mac:

{\leftskip=2in\obeylines
\qquad n'oon d`e \digamma a'utw
\parskip=0pt p'ampan >a'errei

}

\subsection{<H leptom'ereia po`u k'anei t`hn diafor`a}

T`o {\rm\TeX} m~ac >epitr'epei n`a stoiqeiojeto~ume ke'imena m`e m'ia
meg'alh poikil'ia qarakt~hrwn ka`i diat'axewn.  >'Arage a>ut`o shma'inei
<'oti g'iname >epaggelmat'iec tupogr'afoi ka`i <'oti mporo~ume n`a
pet~ame dexi`a ka`i >arister`a st`o qart`i pl'agia, kalligrafik`a ka`i
>'entona?  <H >ap'anthsh e>~inai: ((>'Oqi!))  O<i dunat'othtec po`u m~ac
prosf'erei t`o {\rm\TeX} d`en e>~inai gi`a n`a fti'aqnoume full'adec m`e
t'itlouc {\tengs D'agkws'e me}, {\tengs <H >Arpaqt'h}, k.lp.  T`o
{\rm\TeX} e>~inai m'allon gi`a sobar'wtera >'entupa. M'alista, j`a
diapist'wsoume <'oti kaj`wc maja'inoume t`o {\rm\TeX}, maja'inoume ka`i
>eme~ic n`a seb'omasje t`o >'entupo. >Epeid`h d`e p'anta <h leptom'ereia
k'anei t`hn diafor'a, parak'atw j`a >exet'asoume <orism'ena j'emata po`u
j`a m~ac k'anoun prosektik'wterouc, >'ara kal'uterouc {\rm\TeX}n'itec.

Pr~wta-pr~wta, p'anta pr`in xekin'hsoume t`hn stoiqeiojes'ia <en`oc
>ent'upou j`a pr'epei n`a >afie\-r'w\-sou\-me l'igo qr'ono n`a
skefjo~ume t`hn morf'h tou.  Sun'hjwc, magem'enoi >ap`o t`hn d'unamh
to~u {\rm\TeX}, xekin~ame t`o gr'ayimo to~u k'wdika qwr`ic n`a
>exet'asoume t'i >epijumo~ume n`a l'aboume st`o qart'i.  <'Omwc d`en j`a
pr'epei n`a biaz'omasje.  >Epipl'eon, >e`an skop'oc mac e>~inai n`a
<etoim'asoume <'ena bibl'io, kal`o e>~inai pr`in xekin'hsoume t`hn
s'untaxh to~u k'wdika n`a sumbouleujo~ume <'enan >epaggelmat'ia
sqediast`h bibl'iwn.  A>ut`oc j`a m~ac d'wsei t`ic kat'allhlec <odhg'iec
gi`a t`hn <etoimas'ia t~wn t'itlwn, t`hn >epilog`h to~u e>'idouc, to~u
t'upou ka`i to~u meg'ejouc t~wn grammatoseir~wn, k.lp.

<H >epilog`h t~wn grammatoseir~wn, t~wn t'upwn (>'orjia, pl'agia >`h
>'entona) ka`i to~u meg'ejouc touc ($10\,\rm pt$, $12\,\rm pt$, k.lp.)\
>antikatoptr'izei t`o <'ufoc to~u keim'enou mac. Kat`a kan'ona, t`o
ke'imeno gr'afetai m`e >'orjiouc <aplo`uc qarakt~hrec, >en~w t`a
pl'agia, k.lp.\ qrhsime'uoun gi`a e>idiko`uc skopo'uc.  O<i t'itloi,
gi`a par'adeigma, sun'hjwc stoiqeiojeto~untai s`e meg'alouc >'entonouc
t'upouc >`h m`e kefala~ia.  Meg'aloi qarakt~hrec qwr`ic >apol'hxeic
t'upou {\sf sans serif} qrhsimopoio~untai ka`i gi`a t`hn stoiqeiojes'ia
paidik~wn bibl'iwn po`u >epib'alletai n`a e>~inai >idia'itera <apl`a
st`hn >emf'anis'h touc. St`on parak'atw p'inaka d'inontai merik`ec
genik`ec <odhg'iec gi`a t`hn qr'hsh t~wn diaf'orwn t'upwn stoiqe'iwn:

\maketable[Qr'hseic t'upwn stoiqe'iwn]
\halign{%
\strut
\hfil # \hfil & \ # \hfil & \ # \hfil \cr
{\tengb T'upoc} & {\tengb Qr'hsh} & {\tengb Par'adeigma} \cr
\noalign{\hrule}\noalign{\smallskip}
>'orjia (<apl'a)  & <apl`o ke'imeno, mon'adec,
         & <H >anjrakik`h r'iza: {\rm CO$^{2-}_3$}. \cr 
         & >`h qhmiko`i t'upoi & \cr
\noalign{\smallskip}
>'entona & t'itloi, >'emfash 
         & {\tengb Kef.\ 3: >Asiatik`ec q~wrec} \cr
         & >`h <orismo`i & \cr
\noalign{\smallskip}
pl'agia  & >'emfash, <orismo`i 
         & >Efhm.\ {\tengs <H Jessal'ia}, \cr
         & >`h t'itloi >ent'upwn & B'oloc, 31/12/1898, sel.\ 2\cr
\noalign{\smallskip}
pl'agia--kalligrafik`a & >'emfash, <orismo`i
         & <Wc {\tengi sf'alma\/} <or'izoume: \cr
         & >`h majhmatiko`i t'upoi 
         & $\varepsilon = \epsilon \neq \rho = \varrho$ \cr
\noalign{\smallskip}
grafomhqan~hc & prosomo'iwsh grafomhqan~hc
         & {\tt if (a != b) then \lb\dots\rb } \cr
         & ka`i l'istec programm'atwn & \cr
     } % end halign

<H <upogr'ammish keim'enou l'ogou (dhl.\ m`h majhmatik~wn sumb'olwn) d`en
sunhj'izetai st`hn tupograf'ia.  Pr'okeitai gi`a <'ena kat'aloipo t~hc
>epoq~hc t~hc grafomhqan~hc, <'otan o<i daktulogr'afoi d`en di'ejetan
>'allon tr'opo gi`a n`a k'anoun <'ena m'eroc to~u keim'enou n`a
xeqwr'izei >ap`o t`o <up'oloipo.  >Ef'' <'oson t`o {\rm\TeX} m~ac
prosf'erei >arketo`uc t'upouc stoiqe'iwn gi`a k'aje skop'o, mporo~ume
n`a >apof'ugoume t`hn <upogr'ammish.  J`a pr'epei <wst'oso n`a
pros'eqoume pot`e st`o >'idio >'entupo n`a m`hn qrhsimopoio~ume d'uo
diaforetiko`uc t'upouc stoiqe'iwn gi`a t`on >'idio skop'o.  >E`an, gi`a
par'adeigma, qrhsimopoio~ume pl'agiouc t'upouc gi`a >'emfash, t'ote d`en
j`a pr'epei n`a qrhsimopoio~ume gi`a t`on >'idio skop`o
pl'agia--kalligrafik`a stoiqe~ia >`h >'entona.

Suqn'a, kaj`wc <etoim'azoume t`on k'wdika {\rm\TeX} <en`oc >ent'upou m`e
<ellhnik`o ke'imeno, k'anoume l'ajh >ephreasm'enoi >ap`o t`hn par'adosh
ka`i to`uc kan'onec t~hc x'enhc tupograf'iac, ka`i kur'iwc t~hc
>aggloamerikanik~hc.  Gi`a par'adeigma, pollo`i qrhsimopoio~un t`a
>agglik`a e>isagwgik`a {\rm``}\NB{}ka`i\NB{}{\rm''}, >ant`i t~wn <ellhnik~wn
(( ka`i\NB)).  >'Alloi p'ali >af'hnoun <'ena ken`o di'asthma met`a t`o
>arister`o e>isagwgik`o ka`i pr`in t`o dexi'o, >epeid`h >'iswc >'etsi
to`uc >'edeixe k'apote <h kur'ia t~wn Gallik~wn.  <'Omwc  st`hn
per'iptwsh <ellhniko~u keim'enou, t`a <uperuywm'ena e>isagwgik'a, e>'ite
st`hn <apl`h (``\NB{}''), e>'ite st`hn dipl`h morf'h touc
({\rm``\NB{}''}), qrhsimopoio~untai m'onon <wc e>isagwgik`a >ent`oc
e>isa\-gwgi\-k~wn.  >Ak'oma, t`a <ellhnik`a e>isagwgik'a, par'' <'oti
moi'azoun t~wn gallik~wn, d`en >ako\-lou\-jo~un to`uc >'idiouc kan'onec
stoiqeioj'ethshc.  Pot`e d`en pr'epei n`a >af'hnoume ken`o di'asthma
met`a t`o >arister`o e>isagwgik'o; paromo'iwc, d`en >af'hnoume ken`o
di'asthma pr`in t`o dexi`o e>isagwgik'o, t`hn tele'ia, t`o k'omma, t`hn
>'anw tele'ia, t`hn >'anw ka`i k'atw tele'ia, t`o >erwthmatik`o ka`i t`o
jaumastik'o.  Mporo~ume <wst'oso n`a j'etoume ken`a diast'hmata
<ekat'erwjen k'aje dipl~hc (parenjetik~hc) pa'ulac\NB{}(---), s`e
>ant'ijesh m`e t`hn tupograf'ia >agglik~wn keim'enwn <'opou d`en
>epitr'epontai ken`a diast'hmata <ekat'erwjen k'aje dipl~hc pa'ulac.

Gi`a t`hn j'esh t~wn e>isagwgik~wn s`e sq'esh m`e >'alla shme~ia
st'ixhc, j`a pr'epei p'anta n`a >exet'azoume >e`an t`a >'alla shme~ia
st'ixhc >apotelo~un m'eroc to~u keim'enou >ent`oc e>isagwgik~wn. 
Sunep~wc, >e`an t`o ke'imeno >ent`oc t~wn e>isagwgik~wn >apotele~i m'ia
pl'hrh pr'otash, t'ote <h tele'ia prohge~itai to~u dexio~u
e>isagwgiko~u.  <Or'iste <'ena sqetik`o par'adeigma:

{\narrower\parindent=0pt
--- E>~inai >al'hjeia, kathgoro'umene, <'oti >apek'alesec t`on mhnut`h
((bl'aka))?

\parskip=0pt
--- >'Oqi, k.\ Pr'oedre!  >Eg`w to~u e>~ipa: ((>'Etsi po`u
sumperif'eresai j`a s`e pern~ane gi`a bl'aka.))

}

\noindent
T`o >'idio >isq'uei ka`i gi`a t`ic parenj'eseic; st`hn per'iptwsh m'iac
pl'hrouc pr'otashc >ent`oc pa\-ren\-j'e\-sewn, <h dexi`a par'enjesh
mpa'inei met`a t`hn tele'ia --- diaforetik'a, prohge~itai.

T`a e>isagwgik`a j`a pr'epei n`a t`a qrhsimopoio~ume m`e feid'w.
>Ep'ishc m`e t`o m'etro j`a pr'epei n`a qrhsimopoio~ume t`a kefala~ia,
t`a poll`a jaumastik'a, k.lp.  Gi`a par'adeigma, d`en qrei'azetai n`a
b'azoume tr'ia >`h ka`i periss'otera jaumastik`a gi`a n`a ton'isoume
t`on jaumasm'o mac >`h t`hn >'ekplhx'h mac gi`a k'ati; <'ena ka`i m'onon
<'ena jaumastik`o >arke~i! >Ak'oma ka`i t`a e>isagwgik`a gi`a l'exeic
po`u qrhsimopoio~untai metaforik'a, suqn`a peritte'uoun.  <'Oso gi`a t`a
kefala~ia, j`a pr'epei n`a pros'exoume <'oti qrhsimopoio~untai m'onon
st`hn >arq`h prot'asewn ka`i kur'iwn >onom'atwn.  D`en qrei'azetai n`a
mimo'umasje to`uc >Aggloamerik'anouc ka`i n`a gr'afoume <'olec t`ic
l'exeic t~wn t'itlwn m`e kefala~ia.  >'Etsi t`o swst`o e>~inai n`a
gr'afoume:

\vskip\baselineskip
\centerline{{\tengb Kef.\NB{}2: T`o je'wrhma to~u Jal~h ka`i sqetik`a
por'ismata}}

\noindent
>ant`i to~u lanjasm'enou: 

\vskip\baselineskip
\centerline{{\tengb Kef.\NB{}2: T`o Je'wrhma to~u Jal~h ka`i Sqetik`a
Por'ismata}}

\noindent
Beba'iwc, st`hn po'ihsh <h qr'hsh t~wn kefala'iwn e>~inai diaforetik'h;
pol\-lo`i s'ugqronoi poiht`ec xekino~un k'aje st'iqo touc m`e <'ena
kefala~io gr'amma.  >Epeid`h o<i poi\-h\-t`ec p'anta >~htan >ekt`oc
sumbatik~wn kan'onwn, j`a pr'epei n`a sebasjo~ume t`hn >'apoy'h touc.

>Ak'oma ka`i t`ic <uposhmei'wseic j`a pr'epei n`a t`ic >apofe'ugoume
st`o m'etro to~u dunato~u.  O<i poll`ec <uposhmei'wseic >apospo~un t`hn
prosoq`h to~u >anagn'wsth >ap`o t`o k'urio ke'imeno ka`i, >epipl'eon,
d'inoun t`hn >ent'upwsh <'oti <o suggraf'eac d`en >'eqei sugkrothm'enh
sk'eyh!

<'Ena shme~io >ak'oma po`u >apaite~i prosoq`h st`hn <etoimas'ia to~u
k'wdika {\rm\TeX} e>~inai <h >ap'o\-stro\-foc.  >Arqik'a, t`o
{\rm\TeX} m~ac d'inei t`hn dunat'othta n`a xeqwr'izoume t`hn >ap'ostrofo
>ap`o t`hn yil'h. <H >ap'ostrofoc  ('') lamb'anetai m`e t`on k'wdika
{\tt'{}'} ka`i e>~inai k'apwc megal'uterh >ap`o t`hn yil`h (>) po`u
lamb'anetai m`e t`on k'wdika\NB{\tt>}.  >Ep'ishc, <'opou metax`u d'uo
l'exewn parousi'azetai >'ekjliyh (p.q., ((j''\NB>ako'usete)){}) >`h
>afa'iresh (p.q., ((mo~u ''fere)){}), t'ote met`a >`h pr`in t`hn
>ap'ostrofo >ant'istoiqa, paremb'aletai p'anta <'ena ken`o di'asthma,
dhl.\ o<i d'uo l'exeic pot`e d`en kollo~un <h m'ia >ep'anw st`hn >'allh.
 T`o >ant'ijeto >isq'uei st`hn stoiqeiojes'ia >agglik~wn ka`i gallik~wn
keim'enwn, p.q. (({\rm$\,$C'est par l\`a qu'il s'est
envol\'e$\,$!}$\,$))

J`a kle'isoume <eto'uth t`hn suz'hthsh m`e d'uo l'ogia gi`a t`o
polutonik`o ka`i t`o monotonik'o.  T`o n`a gr'afoume, n`a
daktulografo~ume >`h n`a stoiqeiojeto~ume t`a dik`a mac ke'imena m`e t`o
<'ena >`h t`o >'allo s'usthma e>~inai dik'h mac >epilog'h.  <'Omwc
<'otan paraj'etoume st`o >'entup'o mac d'aneia qwr'ia >'allwn
suggraf'ewn j`a pr'epei n`a seb'omasje t`hn dik'h touc >epilog'h, dhl.\
t`hn >emf'anish to~u prwtot'upou >ap`o <'opou t`a >antigr'afoume. 
>'Etsi ka`i t`a >arqa~ia <ellhnik`a ke'imena j`a pr'epei p'anta n`a t`a
stoiqeiojeto~ume s'umfwna m`e t`o polutonik`o s'usthma ka`i t`hn dik`h
touc >orjograf'ia. Gi`a par'adeigma, st`hn kajare'uousa <h l'exh {\tengs
gl~wssa\/} pa'irnei perispwm'enh, >en~w st`o polutonik`o s'usthma t~hc
dhmotik~hc to~u Triantafull'idh g'inetai: {\tengs gl'wssa\/}!

\subsection{<Ellhnik`a majhmatik`a}

St`o kef'alaio 7, e>'idame p~wc mporo~ume n`a stoiqeiojeto~ume
majhmatik`ec >ekfr'aseic m`e t`o {\rm \TeX}\null.  E>'idame gi`a
par'adeigma <'oti <o k'wdikac {\tt\$\\cos \^{}2 x = 1 - \\sin \^{}2 x\$}
j`a m~ac d'wsei: $\cos^2 x = 1 - \sin^2 x$\null.  <O  s'ugqronoc
majhmatik`oc sumbolism`oc te'inei n`a g'inei m'ia pagk'osmia gl'wssa.
Sunep~wc ka`i t`a n'ea <ellhnik`a bibl'ia t~wn majhmatik~wn
qrhsimopoio~un latinik`a s'umbola <'opwc $\cos$, $\log$, k.lp.  T'i
g'inetai <'omwc st`hn per'iptwsh po`u k'apoioc >epim'enei n`a j'elei n`a
sumbol'isei t`o <hm'itono <wc ((hm))?  A>ut`o ka`i merik`a >ak'oma
probl'hmata po`u sunant~a kane`ic kaj`wc stoiqeiojete~i <ellhnik`a
majhmatik`a ke'imena j`a t`a >exet'asoume s`e <eto'uth t`hn par'agrafo. 

<H pi`o e>'ukolh l'ush st`o pr'oblhma to~u majhmatiko~u sumb'olou ((hm))
e>~inai n`a qrhsimopoi'hsoume <'ena {\tt\\hbox} >ent`oc t~wn
majhmatik~wn.  >'Etsi <o k'wdikac {\tt\$\\hbox\lb\\tengr hm\rb \^{}2 x =
1 - \\hbox\lb\\tengr sun\rb \^{}2 x\$} j`a m~ac d'wsei: $\hbox{\tengr
hm}^2 x = 1 - \hbox{\tengr sun}^2 x$\null. <Wst'oso a>ut`h <h l'u\-sh
d`en e>~inai <h pi`o >'omorfh, giat`i t`a diast'hmata po`u b'azei t`o
{\rm \TeX} g'urw >ap`o t`ic l'exeic--s'umbola ((hm)) ka`i ((sun)) d`en
e>~inai swst'a (par'' <'oti a>ut`o mpore~i n`a m`hn e>~inai ka`i t'oso
>emfan`ec). Gi`a n`a l'aboume swst`a diast'hmata, j`a pr'epei n`a
<or'isoume t`o ((hm)) ka`i t`o ((sun)) <wc l'exeic >el'egqou po`u
>antiproswpe'uoun e>idik`ec majhmatik`ec sunart'hseic ka`i t`ic <opo~iec
t`o {\rm \TeX} t`ic <ermhne'uei kat'allhla.  A>ut`o g'inetai m`e t`hn
bo'hjeia t~hc l'exhc >el'egqou {\tt\\mathop}.\TeXref{361} P.q., m`e
{\tt\\def\\grsin\lb\\mathop\lb\\hbox\lb\\tengr hm\rb\rb\\nolimits\rb}
>'eqoume <or'isei m'ia n'ea sq'esh, t`hn {\tt\\grsin} po`u m~ac d'inei
t`o ((hm)).  <H l'exh >el'egqou {\tt\\nolimits} shma'inei p`wc t`o
s'umbolo ((hm)) d`en >'eqei >'anw >`h/ka`i k'atw <'orio, s`e >ant'ijesh
m`e >'alla (p.q., $\lim$, $\max$, k.>'a.)\ po`u mporo~un n`a
>'eqoun.\TeXref{144} Paromo'iwc mporo~ume n`a <or'isoume ka`i m'ia
>ant'istoiqh l'exh >el'egqou gi`a t`o ((sun)). <Or'iste <'ena pl~hrec
par'adeigma:
\toindex{mathop}
\toindex{nolimits}

\beginuser
\\def\\grsin\lb\\mathop\lb\\hbox\lb\\tengr hm\rb\rb\\nolimits\rb
\\def\\grcos\lb\\mathop\lb\\hbox\lb\\tengr sun\rb\rb\\nolimits\rb
\$\$ \\grcos \^{}2 \\pi = \\grsin \^{}2 (\\pi / 2) = 1 \$\$
\enduser

\noindent
<O k'wdikac a>ut`oc d'inei:
\def\grsin{\mathop{\hbox{\tengr hm}}\nolimits}
\def\grcos{\mathop{\hbox{\tengr sun}}\nolimits}
$$ \grcos ^2 \pi = \grsin ^2 (\pi / 2) = 1 $$

<'Ena >ak'oma pr'oblhma po`u mpore~i n`a >antimetwp'isoume kaj`wc
stoiqeiojeto~ume <'ena <el\-lh\-ni\-k`o majhmatik`o ke'imeno e>~inai <h
dekadik`h <upodiastol'h.  Gi`a to`uc >Aggloamerik'anouc, <h dekadik`h
<upodiastol`h sumbol'izetai m`e t`hn tele'ia; >e`an gr'ayoume {\tt\$e =
2.718\\ldots\$}, t`o {\rm\TeX} --- s`an gn'hsio >Amerikan'aki --- j`a
m~ac t`o stoiqeiojet'hsei qwr`ic kan'ena pr'oblhma: $ e = 2.718\ldots$ 
>E`an gr'ayoume <'omwc {\tt\$e = 2,718\\ldots\$}, t'ote j`a l'aboume
<'ena mikr`o ken`o di'asthma met`a t`o k'omma: $e = 2,718\ldots$\space\
Gi`a n`a >apof'ugoume a>ut`o t`o pr'oblhma, j`a pr'epei st`on k'wdika
n`a b'aloume t`o k'omma metax`u d'uo >agkul~wn, dhl.\ gr'afoume {\tt\$e
= 2\lb,\rb718\\ldots\$} gi`a n`a l'aboume $e = 2{,}718\ldots$

St`o kef'alaio 7 e>'idame >ak'oma <'oti <'ola t`a s'umbola metablht~wn
st`a majhmatik`a stoiqeiojeto~untai m`e pl'agiouc--kalligrafiko`uc
qarakt~hrec, >'etsi <'wste, p.q., n`a m`hn g'inetai s'ugqush metax`u
ginom'enwn ka`i kanoniko~u m`h majhmatiko~u keim'enou.  <'Omwc t`a
kefala~ia <ellhnik`a gr'ammata d`en bga'inoun pl'agia; p.q., <o k'wdikac
{\tt\$\\Psi\$} d'inei: $\Psi$\null.  T`a pl'agia <ellhnik`a kefala~ia
peri'eqontai s`e m'ia grammatoseir`a to~u {\rm\TeX} (ka`i >'oqi to~u
\greektex) po`u kale~itai m`e t`hn l'exh >el'egqou {\tt\\mit}. 
Gr'afontac loip'on, {\tt\$\\widehat\lb AB\lb\\mit \\Gamma\rb\rb = \\pi /
2\$}, lamb'anoume: $\widehat{AB{\mit\Gamma}} = \pi / 2$.
\toindex{mit}

Pr`in kle'isoume <eto'uth t`hn par'agrafo, >ax'izei n`a >anaferjo~ume
s`e <'ena >ap`o t`a pi`o suqn`a l'ajh po`u k'anoun o<i <'ellhnec
majhmatiko'i: >apokalo~un t`o s'umbolo t~hc merik~hc parag'wgou
((j~hta))!  St`hn pragmatik'othta, t`o s'umbolo a>ut`o e>~inai m'ia
kalligrafik`h morf`h to~u la\-ti\-ni\-ko~u {\rm d} ka`i >'oqi t`o
kalligrafik`o j~hta.  >Arke~i n`a dokim'asoume t`on >ak'oloujo k'wdika
{\tt\$\\partial \\neq \\vartheta\$} gi`a n`a >antilhfjo~ume t`hn
diafor'a: $\partial \neq \vartheta$\null.  >Afo~u loip`on t`o {\rm\TeX}
m~ac t`o >epitr'epei, >`ac gr'afoume swst`a t`hn merik`h par'agwgo.

\exercise Stoiqeiojet~hste:
%%% \efptm = tan (in Greek: >efaptom'enh)
%%% \sfptm = cot (in Greek: sunefaptom'enh)
\def\efptm{\mathop{\hbox{\tengr ef}}\nolimits}
\def\sfptm{\mathop{\hbox{\tengr sf}}\nolimits}
$\efptm^2 (\pi / 6) = \sfptm^{-2} (\pi / 6) = 0{,}33333\ldots $

\exercise Stoiqeiojet~hste t`o >ak'oloujo <ellhnik`o ke'imeno:

{\narrower\noindent
>'Estw k'wnou >isoskelo~uc b'asic <o $AB{\mit \Gamma}$ k'ukloc, koruf`h
d`e t`o $\mit \Delta$, ka`i di'hqjw tic e>ic a>ut`on e>uje'ia <h $A\mit
\Gamma$, ka`i >ap`o t~hc koruf~hc >ep`i t`a $A$, $\mit \Gamma$
>epeze'uqjwsan a<i $A\mit \Delta$, $\mit \Delta\Gamma$; l'egw <'oti t`o
$A\mit \Delta\Gamma$ tr'igwnon >'ellas'on >estin t~hc >epifane'iac t~hc
kwnik~hc t~hc metax`u t~wn $A\mit \Delta\Gamma$. (>Arqim'hdhc, {\tengs
Per`i sfa'irac ka`i kul'indrou}, bibl'io A$'$,\NB{}j)

} % end narrower

\subsection{Mikr`oc >ep'ilogoc gi`a >ep'idoxouc stoiqeioj'etec}

<H stoiqeiojes'ia d`en e>~inai e>'ukolh >ergas'ia.  >Ak'oma ka`i m`e t`o
{\rm \TeX} poll`ec for`ec j`a qrei\-a\-sje~i n`a paideuto~ume
prokeim'enou n`a l'aboume <'ena a>isjhtik`a <wra~io >'entupo.  Poll`ec
for`ec j`a qreiasje~i n`a pa'ixoume m`e <oriz'ontia ka`i katok'orufa
diast'hmata; >'allec for`ec j`a pr'epei n`a bo\-h\-j'h\-soume t`o
pr'ogramma st`on sullabism'o (e>idik`a <'otan t`o {\rm \TeX} d`en
gnwr'izei p~wc n`a sullab'isei <ellhnik`o ke'imeno); >'allec for`ec j`a
pr'epei n`a y'axoume t`ic a>it'iec po`u d`en m~ac d'inei a>ut`o po`u
to~u zht~ame (m'hpwc xeq'asame m'ia >agk'ulh >`h <'ena {\tt
\\par}?)\null.  Qrei'azetai <upomon`h ka`i >epimon'h.  T`a parak'atw
l'ogia to~u poiht~h Nt'inou Qristian'opoulou >`ac d'inoun kour'agio
s''\NB{}a>uto`uc po`u >agapo~un t`o >'entupo:
 
{\narrower
<'Otan ni'wseic p`wc >~hrje pi`a <h <'wra gi`a t`o t'upwma --- ki >afo~u
>'eqeic pn'ixei pali'otera poll`ec par'omoiec >epijum'iec ka`i d`e
shk'wnei >'allh >anabol'h --- kajar'ograye t`a poi'hmat'a sou s`e <'ena
tetr'adio, dane'isou merik`a qili'arika ka`i parak'alese t`on f'ilo sou
n`a s`e bohj'hsei st`hn >'ekdosh.  Spouda~io pr'ama n`a >'eqeic d'ipla
sou <'enan >'anjrwpo s`e mi`a t'etoia stigm'h.  >Eg'w, <'otan
prwtoxek'inhsa m'onoc ki >'apeiroc, p~hga s`e <'ena tupografe~io, m`e
x'afrisan gi`a kal`a ka`i st`o t'eloc mo~u t'upwsan mi`a >ahd'ia. 
>Arg'otera kat'alaba <'oti t`o bibl'io j'elei <ol'oklhrh
>arqitektonik'h.  D`en e>~inai m'ono po`u pr'epei n`a dial'exeic
tupografe~io, qart'i, sq~hma, gr'ammata, di'ataxh; e>~inai prop'antwn
o<i diorj'wseic t~wn dokim'iwn, <o tromer`oc >ag'wnac m`e t`a
tupografik`a l'ajh.  Pr'epei n`a e>ugnwmone~ic t`o f'ilo sou po`u s`a
qam'alhc >an'elabe <'olec t`ic diorj'wseic ka`i n`a m`hn paraponi'esai
p`wc to~u x'efugan du`o laj'akia.

} % end narrower

\medskip\leftline{\null\hskip 2.25in 
\setbox1=\hbox{>Ekd.\ Diagwn'iou, Jessalon'ikh 1991, sel.\ 90--101.}
\vbox{%
\hbox to \wd1 {\hfil Nt'inoc Qristian'opouloc \hfil}
\hbox to \wd1 {\hfil ((Sumboul`ec s'' <'ena n'eo koum'asi)), 
{\tengs <H k'atw b'olta},\hfil}
\box1
}}

\section{Kat'alogoc >akolouji~wn >el'egqou}

Parak'atw d'inontai <'olec o<i >entol'ec (>akolouj'iec >el'egqou:
s'umbola ka`i l'exeic >el'egqou) po`u parousi'asjhkan st`o >egqeir'idio
a>ut'o.   Gi`a periss'oterec leptom'ereiec, sumbouleuje~ite ka`i t`on
>ant'istoiqo jematik`o kat'alogo to~u {\sl \TeX book}.

\vskip 2\baselineskip
\centerline{{\tengb S'umbola >el'egqou}}
\vskip\baselineskip

\settabs \+ \hskip 1.5 in & \hskip 1.3in & \hskip 1.65in & \cr
{\tt
\+  \\\sp{} 4, 41 & \\!\ 41  & \\" 12 & \\' 12    \cr
\+  \\, 41, 47    & \\.\ 12  & \\/ 18 & \\; 41    \cr
\+  \\= 12    & \\> 41   & \\\# 11 &  \\\$ 7, 11, 13  \cr
\+  \\\% 7, 11    & \\\& 11  & \\\char '173{} 11 & \\\char '175{} 11 \cr
\+  \\\underbar{ } 11 & \\` 12 & \\{\accent "7E} 11 & \\{\accent 94} 11, 12 \cr
\+  \\| 44, 49 \cr
}

\vskip 2\baselineskip
\centerline{{\tengb L'exeic >el'egqou}}
\vskip\baselineskip

{\tt
\+ \\AA 13 & \\aa 13 & \\acute 43 & \\AE 13 \cr
\+ \\ae 13 & \\aleph 44 & \\alpha 42 & \\angle 44 \cr
\+ \\approx 44 & \\arccos 49 & \\arcsin 49 & \\arctan 49 \cr
\+ \\arg 49 & \\ast 43 & \\b 13 & \\backslash 44 \cr
\+ \\bar 43 & \\baselineskip 25 & \\begingreek 93 &\\beta 42 \cr 
\+ \\bf 17  & \\Biggl 48 & \\biggl 48 & \\Biggr 48 \cr 
\+ \\biggr 48 & \\Bigl 48 & \\bigl 48 & \\Bigr 48 \cr 
\+ \\bigr 48 & \\bigskip 29 & \\break 29 & \\breve 43 \cr 
\+ \\bullet 43 & \\bye 4 & \\c 12 & \\cal 17, 45 \cr
\+ \\cap 43 & \\catcode 91 & \\cdot 43  & \\cdots 43 \cr
\+ \\centerline 29 & \\centreline 70 & \\char 96 & \\check 43 \cr
\+ \\chi 42 & \\circ 43 & \\columns 56 & \\cos 49 \cr
\+ \\cosh 49 & \\cot 49 & \\coth 49 & \\csc 49 \cr
\+ \\cup 43 & \\d 13 & \\dag 31 & \\ddag 31 \cr 
\+ \\ddot 43 & \\def 64 & \\deg 49 & \\Delta 42 \cr
\+ \\delta 42 & \\det 49 & \\diamond 43 & \\digamma 96 \cr
\+ \\dim 49 & \\display 94 & \\div 43 & \\dot 43 \cr
\+ \\dotfill 58 & \\dots 15 & \\Downarrow 49 & \\downarrow 49 \cr
\+ \\eject 22 & \\ell 44 & \\enddisplay 94 & \\endgreek 93 \cr
\+ \\endinsert 28 & \\epsilon 42 & \\eqalign 54 & \\eqalignno 54 \cr
\+ \\eqno 54 & \\equiv 44 & \\eta 42 & \\exists 44 \cr
\+ \\exp 49 & \\flat 44 & \\folio 32 & \\font 18 \cr
\+ \\footline 32 & \\footnote 31 & \\forall 44 & \\Gamma 42 \cr
\+ \\gamma 42 & \\gcd 49 & \\geq 44 & \\gr 93 \cr
\+ \\grave 43 & \\greekdelims 94 & \\H 13 & \\halign 60 \cr
\+ \\hang 26 & \\hangafter 26 & \\hangindent 26 & \\hat 43 \cr
\+ \\hbadness 33 & \\hbox 83 & \\headline 32 & \\hfil 30 \cr
\+ \\hfill 29, 57 & \\hfuzz 34 & \\hoffset 23 & \\hom 49 \cr
\+ \\hrule 81 & \\hrulefill 58 & \\hsize 22 & \\hskip 30 \cr
\+ \\hyphenation 34 & \\i 12 & \\Im 44 & \\in 44 \cr
\+ \\inf 49 & \\infty 44 & \\input 78 & \\int 46 \cr
\+ \\iota 42 & \\it 17 & \\item 27 & \\itemitem 27 \cr
\+ \\j 12 & \\kappa 42 & \\ker 49 & \\Koppa 96 \cr
\+ \\koppa 96 & \\L 13 & \\l 13 & \\Lambda 42 \cr
\+ \\lambda 42 & \\langle 49 & \\lceil 49 & \\ldots 43 \cr
\+ \\left 52 & \\leftline 29 & \\leftskip 25 & \\leq 44 \cr
\+ \\leqalignno 55 & \\leqno 54 & \\let 70 & \\lfloor 49 \cr
\+ \\lg 49 & \\lim 46, 49 & \\liminf 49 & \\limsup 49 \cr
\+ \\line 29 & \\ln 49 & \\log 49 & \\lower 87 \cr
\+ \\magnification 23 & \\magstep 18 & \\math 94 & \\mathop 100 \cr
\+ \\matrix 52 & \\max 49 & \\medskip 29 & \\min 49 \cr
\+ \\mit 101 & \\moveleft 87 & \\moveright 62, 87 & \\mu 42 \cr
\+ \\nabla 44 & \\narrower 25 & \\natural 44 & \\neg 44 \cr
\+ \\ni 44 & \\noalign 61 & \\noindent 25 & \\nolimits 100 \cr
\+ \\nopagenumbers 6 & \\not 43 & \\nu 42 & \\null 16 \cr
\+ \\O 13 &  \\o 13 & \\obeylines 30 & \\odot 43 \cr
\+ \\OE 13 & \\oe 13 & \\offinterlineskip 62 & \\Omega 42 \cr
\+ \\omega 42 & \\ominus 43 & \\oplus 43 & \\otimes 43 \cr 
\+ \\over 45 & \\overfullrule 34 & \\overline 47 & \\P 31 \cr
\+ \\pageno 32 & \\par 9 & \\parallel 44 & \\parindent 25 \cr
\+ \\parshape 27 & \\parskip 25 & \\partial 44 & \\perp 44 \cr
\+ \\Phi 42 & \\phi 42 & \\Pi 42 & \\pi 42 \cr 
\+ \\pmatrix 51 & \\Pr 49 & \\proclaim 50 & \\Psi 42 \cr 
\+ \\psi 42 & \\qquad 41 & \\quad 41 & \\raggedright 30 \cr
\+ \\raise 87 & \\rangle 49 & \\rceil 49 & \\Re 44 \cr
\+ \\rfloor 49 & \\rho 42 & \\right 52 & \\rightline 29 \cr 
\+ \\rightskip 25 & \\rm 17 & \\root 47 & \\S 31 \cr
\+ \\sampi 96 & scaled 18 & \\sec 49 & \\settabs 56 \cr
\+ \\sharp 44 & \\Sigma 42 & \\sigma 42 & \\sim 44 \cr
\+ \\simeq 44 & \\sin 49 & \\sinh 49 &  \\sl 17 \cr
\+ \\smallskip 29 & \\sqrt 47 & \\ss 13 & \\star 43 \cr
\+ \\stigma 96 & \\strut 59 & \\subset 44 & \\subseteq 44 \cr 
\+ \\sum 46 & \\sup 49 & \\supset 44 & \\supseteq 44 \cr
\+ \\surd 47 & \\t 13 & \\tan 49 & \\tanh 49 \cr
\+ \\tau 42 & \\tensor 70 & \\TeX 4 & \\the 32 \cr
\+ \\Theta 42 & \\theta 42 & \\tilde 43 & \\times 43 \cr
\+ \\to 45 & \\tolerance 34 & \\topinsert 28 & \\tt 17 \cr
\+ \\u 13 & \\underbar 48 & \\underline 47 & \\Uparrow 49 \cr
\+ \\uparrow 49 & \\Updownarrow 49 & \\updownarrow 49 & \\Upsilon 42 \cr
\+ \\upsilon 42 & \\v 13 & \\varepsilon 42 & \\varkoppa 96 \cr
\+ \\varphi 42 & \\varrho 42 & \\varsigma 42 & \\vartheta 42 \cr
\+ \\vbadness 35 & \\vbox 84 & \\vec 43 & \\vee 43 \cr
\+ \\vfill 22 & \\vglue 28 & \\voffset 23 & \\vrule 82 \cr
\+ \\vsize 23 & \\vtop 86 & \\wedge 43 & \\widehat 43 \cr
\+ \\widetilde 43 & \\Xi 42 &  \\xi 42 & \\zeta 42 \cr
}

\ifwritinganswers
   \let\next=\relax
\else
   \let\next=\endinput
   \datestamp
\fi

\next

\def\beginliteral{
\vskip\baselineskip
\begingroup
\obeylines
\tt
\catcode`\@=0\catcode`\~=12
\catcode`\$=12\catcode`\&=12\catcode`\^=12\catcode`\#=12
\catcode`\_=12\catcode`\=12
\def\par{\leavevmode\endgraf}
\catcode`\{=12\catcode`\}=12\catcode`\%=12\catcode`\\=12
}
\def\endliteral{\nobreak \vskip 6pt \endgroup}


\section{D~ws'' mou t`o q'eri sou}

Parak'atw d'inontai o<i l'useic <orism'enwn >ask'hsewn. Poll`ec >ap`o
a>ut`ec t`ic >ask'hseic l'unontai m`e diaf'orouc tr'opouc.  >E`an
protim~ate t`on dik'o sac tr'opo, t'ote m`hn dist'azete n`a t`on
qrhsimopoi'hsete --- >arke~i t`o >apot'elesma n`a s~ac <ikanopoie~i!
\vskip 2\baselineskip

%%% Here we switch back to roman (DF)
\rm 
\catcode`\~=13
\parskip=0pt 
\parindent=0pt
\begingroup\raggedright
\hbadness=10000

\hrule
\beginliteral
I like \TeX!
Once you get the hang of it, \TeX{} is really easy to use.
You just have to master the \TeX nical aspects.
@endliteral
I like \TeX! Once you get the hang of it, \TeX{} is really easy
to use. You just have to master the \TeX nical aspects.
\vskip \baselineskip \hrule

\goodbreak \vskip 2pt \hrule
\beginliteral
Does \AE schylus understand \OE dipus?
@endliteral
Does \AE schylus understand \OE dipus?
\vskip \baselineskip \hrule

\goodbreak \vskip 2pt \hrule
\beginliteral
The smallest internal unit of \TeX{} is about 53.63 \AA.
@endliteral
The smallest internal unit of \TeX{} is about 53.63 \AA.
\vskip \baselineskip \hrule

\goodbreak \vskip 2pt \hrule
\beginliteral
They took some honey and plenty of money wrapped up in a {\it \$}5 note.
@endliteral
They took some honey and plenty of money wrapped up in a {\it \$}5 note.
\vskip \baselineskip \hrule

\goodbreak \vskip 2pt \hrule
\beginliteral
\'El\`eves, refusez vos le\c cons! Jetez vos cha\^\i nes!
@endliteral
\'El\`eves, refusez vos le\c cons! Jetez vos cha\^\i nes!
\vskip \baselineskip \hrule

\goodbreak \vskip 2pt \hrule
\beginliteral
Za\v sto tako polako pijete \v caj?
@endliteral
Za\v sto tako polako pijete \v caj?
\vskip \baselineskip \hrule
 
\goodbreak \vskip 2pt \hrule
\beginliteral
Mein Tee ist hei\ss.
@endliteral
Mein Tee ist hei\ss.
\vskip \baselineskip \hrule
 
\goodbreak \vskip 2pt \hrule
\beginliteral
Peut-\^etre qu'il pr\'ef\`ere le caf\'e glac\'e.
@endliteral
Peut-\^etre qu'il pr\'ef\`ere le caf\'e glac\'e.
\vskip \baselineskip \hrule

\goodbreak \vskip 2pt \hrule
\beginliteral
?@begingroup`@endgroup Por qu\'e no bebes vino blanco? !@begingroup`@endgroup Porque est\'a avinagrado!
@endliteral
?`Por qu\'e no bebes vino blanco? !`Porque est\'a avinagrado!
\vskip \baselineskip \hrule

\goodbreak \vskip 2pt \hrule
\beginliteral
M\'\i\'\j n idee\"en wordt niet be\"\i nvloed.
@endliteral
M\'\i\'\j n idee\"en wordt niet be\"\i nvloed.
\vskip \baselineskip \hrule

\goodbreak \vskip 2pt \hrule
\beginliteral
 Can you take a ferry from \"Oland to \AA land? 
@endliteral 
 Can you take a ferry from \"Oland to \AA land? 
\vskip \baselineskip \hrule 
  
  
\goodbreak \vskip 2pt \hrule 
\beginliteral 
 T\"urk\c ce konu\c san ye\u genler nasillar? 
@endliteral 
 T\"urk\c ce konu\c san ye\u genler nasillar? 
\vskip \baselineskip \hrule 
  
  
  
  
\goodbreak \vskip 2pt \hrule 
\beginliteral 
 I entered the room and---horrors---I saw both my father-in-law and my 
mother-in-law. 
@endliteral 
 I entered the room and---horrors---I saw both my father-in-law and my 
mother-in-law. 
\vskip \baselineskip \hrule 
  
  
\goodbreak \vskip 2pt \hrule 
\beginliteral 
The winter of 1484--1485 was one of discontent. 
@endliteral 
The winter of 1484--1485 was one of discontent. 
\vskip \baselineskip \hrule 
  
  
\goodbreak \vskip 2pt \hrule 
\beginliteral 
His ``thoughtfulness'' was impressive. 
@endliteral 
His ``thoughtfulness'' was impressive. 
\vskip \baselineskip \hrule 
  
  
\goodbreak \vskip 2pt \hrule 
\beginliteral 
 Frank wondered, ``Is this a girl that can't say `No!'?'' 
@endliteral 
 Frank wondered, ``Is this a girl that can't say `No!'?'' 
\vskip \baselineskip \hrule 
  
  
\goodbreak \vskip 2pt \hrule 
\beginliteral 
 He thought, ``\dots and this goes on forever, perhaps to the last recorded 
syllable.'' 
@endliteral 
 He thought, ``\dots and this goes on forever, perhaps to the last recorded 
syllable.'' 
\vskip \baselineskip \hrule 
  
  
\goodbreak \vskip 2pt \hrule 
\beginliteral 
 Have you seen Ms.~Jones? 
@endliteral 
 Have you seen Ms.~Jones? 
\vskip \baselineskip \hrule 
  
  
\goodbreak \vskip 2pt \hrule 
\beginliteral 
Prof.~Smith and Dr.~Gold flew from 
Halifax N.~S. to Montr\'eal, P.~Q. via Moncton, N.~B. 
@endliteral 
Prof.~Smith and Dr.~Gold flew from 
Halifax N.~S. to Montr\'eal, P.~Q. via Moncton, N.~B. 
\vskip \baselineskip \hrule 
  
  
\goodbreak \vskip 2pt \hrule 
\beginliteral 
\line{left end \hfil left tackle \hfil left guard \hfil @centre \hfil 
right guard \hfil right tackle \hfil right end} 
@endliteral 
\line{left end \hfil left tackle \hfil left guard \hfil \centre{} \hfil 
right guard \hfil right tackle \hfil right end} 
\vskip \baselineskip \hrule 
  
  
\goodbreak \vskip 2pt \hrule 
\beginliteral 
\line{left \hfil \hfil right-@centre \hfil right} 
@endliteral 
\line{left \hfil \hfil right-\centre{} \hfil right} 
\vskip \baselineskip \hrule 
  
  
\goodbreak \vskip 2pt \hrule 
\beginliteral 
\line{\hskip 1 in ONE \hfil TWO \hfil THREE} 
@endliteral 
\line{\hskip 1 in ONE \hfil TWO \hfil THREE} 
\vskip \baselineskip \hrule 
  
  
\goodbreak \vskip 2pt \hrule 
\beginliteral 
i{f}f if{}f if{f} 
@endliteral 
i{f}f if{}f if{f} 
\vskip \baselineskip \hrule 
  
  
\goodbreak \vskip 2pt \hrule 
\beginliteral 
I started with roman type {\it switched to italic type}, and 
returned to roman type. 
@endliteral 
I started with roman type {\it switched to italic type}, and 
returned to roman type. 
\vskip \baselineskip \hrule 
  
  
\goodbreak \vskip 2pt \hrule 
\beginliteral 
$C(n,r) = n!/(r!\,(n-r)!)$ 
@endliteral 
$C(n,r) = n!/(r!\,(n-r)!)$ 
\vskip \baselineskip \hrule 
  
  
\goodbreak \vskip 2pt \hrule 
\beginliteral 
$a+b=c-d=xy=w/z$ 
$$a+b=c-d=xy=w/z$$ 
@endliteral 
$a+b=c-d=xy=w/z$ 
$$a+b=c-d=xy=w/z$$ 
\vskip \baselineskip \hrule 
  
  
\goodbreak \vskip 2pt \hrule 
\beginliteral 
$(fg)' = f'g + fg'$ 
$$(fg)' = f'g + fg'$$ 
@endliteral 
$(fg)' = f'g + fg'$ 
$$(fg)' = f'g + fg'$$ 
\vskip \baselineskip \hrule 
  
  
\goodbreak \vskip 2pt \hrule 
\beginliteral 
$\alpha\beta=\gamma+\delta$ 
$$\alpha\beta=\gamma+\delta$$ 
@endliteral 
$\alpha\beta=\gamma+\delta$ 
$$\alpha\beta=\gamma+\delta$$ 
\vskip \baselineskip \hrule 
  
  
\goodbreak \vskip 2pt \hrule 
\beginliteral 
$\Gamma(n) = (n-1)!$ 
$$\Gamma(n) = (n-1)!$$ 
@endliteral 
$\Gamma(n) = (n-1)!$ 
$$\Gamma(n) = (n-1)!$$ 
\vskip \baselineskip \hrule 
  
\goodbreak \vskip 2pt \hrule 
\beginliteral 
$x\wedge (y\vee z) = (x\wedge y) \vee (x\wedge z)$ 
@endliteral 
$x\wedge (y\vee z) = (x\wedge y) \vee (x\wedge z)$ 
\vskip \baselineskip \hrule 
  
\goodbreak \vskip 2pt \hrule 
\beginliteral 
$2+4+6+\cdots +2n = n(n+1)$ 
@endliteral 
$2+4+6+\cdots +2n = n(n+1)$ 
\vskip \baselineskip \hrule 
  
  
\goodbreak \vskip 2pt \hrule 
\beginliteral 
$\vec x\cdot \vec y  = 0$ if and only if $\vec x \perp \vec y$. 
@endliteral 
$\vec x\cdot \vec y  = 0$ if and only if $\vec x \perp \vec y$. 
\vskip \baselineskip \hrule 
  
  
\goodbreak \vskip 2pt \hrule 
\beginliteral 
$\vec x\cdot \vec y \not= 0$ if and only if $\vec x \not\perp \vec y$. 
@endliteral 
$\vec x\cdot \vec y \not= 0$ if and only if $\vec x \not\perp \vec y$. 
\vskip \baselineskip \hrule 
  
  
\goodbreak \vskip 2pt \hrule 
\beginliteral 
$(\forall x\in \Re)(\exists y\in\Re)$ $y>x$. 
@endliteral 
$(\forall x\in \Re)(\exists y\in\Re)$ $y>x$. 
\vskip \baselineskip \hrule 
  
  
\goodbreak \vskip 2pt \hrule 
\beginliteral 
${a+b\over c}\quad {a\over b+c}\quad {1\over a+b+c} \not= {1\over a}+ 
{1\over b}+{1\over c}$. 
@endliteral 
${a+b\over c}\quad {a\over b+c}\quad {1\over a+b+c} \not= {1\over a}+ 
{1\over b}+{1\over c}$. 
\vskip \baselineskip \hrule 
  
  
\goodbreak \vskip 2pt \hrule 
\beginliteral 
What are the points where ${\partial \over \partial x} f(x,y) = {\partial \over 
\partial y} f(x,y) = 0$? 
@endliteral 
What are the points where ${\partial \over \partial x} f(x,y) = {\partial \over 
\partial y} f(x,y) = 0$? 
\vskip \baselineskip \hrule 
  
  
\goodbreak \vskip 2pt \hrule 
\beginliteral 
$e^x \quad e^{-x} \quad e^{i\pi}+1=0 \quad x_0 \quad x_0^2 
\quad {x_0}^2 \quad 2^{x^x}$. 
@endliteral 
$e^x \quad e^{-x} \quad e^{i\pi}+1=0 \quad x_0 \quad x_0^2 
\quad {x_0}^2 \quad 2^{x^x}$. 
\vskip \baselineskip \hrule 
  
  
\goodbreak \vskip 2pt \hrule 
\beginliteral 
$\nabla^2 f(x,y) = {\partial^2 f \over\partial x^2}+ {\partial^2 f \over 
\partial y^2}$. 
@endliteral 
$\nabla^2 f(x,y) = {\partial^2 f \over\partial x^2}+ {\partial^2 f \over 
\partial y^2}$. 
\vskip \baselineskip \hrule 
  
  
\goodbreak \vskip 2pt \hrule 
\beginliteral 
$\lim_{x\to 0} (1+x)^{1\over x}=e$. 
@endliteral 
$\lim_{x\to 0} (1+x)^{1\over x}=e$. 
\vskip \baselineskip \hrule 
  
  
\goodbreak \vskip 2pt \hrule 
\beginliteral 
The cardinality of $(-\infty, \infty)$ is $\aleph_1$. 
@endliteral 
The cardinality of $(-\infty, \infty)$ is $\aleph_1$. 
\vskip \baselineskip \hrule 
  
  
\goodbreak \vskip 2pt \hrule 
\beginliteral 
$\lim_{x\to {0^+}} x^x = 1$. 
@endliteral 
$\lim_{x\to {0^+}} x^x = 1$. 
\vskip \baselineskip \hrule 
  
  
\goodbreak \vskip 2pt \hrule 
\beginliteral 
$\int_0^1 3x^2\,dx = 1$. 
@endliteral 
$\int_0^1 3x^2\,dx = 1$. 
\vskip \baselineskip \hrule
  
  
\goodbreak \vskip 2pt \hrule 
\beginliteral 
$\sqrt2 \quad \sqrt {x+y\over x-y} \quad \root 3 \of {10}$ \quad $e^{\sqrt x}$. 
@endliteral 
$\sqrt2 \quad \sqrt {x+y\over x-y} \quad \root 3 \of {10}$ \quad $e^{\sqrt x}$. 
\vskip \baselineskip \hrule 
  
  
\goodbreak \vskip 2pt \hrule 
\beginliteral 
$\|x\| = \sqrt{x\cdot x}$. 
@endliteral 
$\|x\| = \sqrt{x\cdot x}$. 
\vskip \baselineskip \hrule 
  
  
\goodbreak \vskip 2pt \hrule 
\beginliteral 
$\phi(t) = {1 \over \sqrt{2\pi}} \int_0^t e^{-x^2/2}\,dx$. 
@endliteral 
$\phi(t) = {1 \over \sqrt{2\pi}} \int_0^t e^{-x^2/2}\,dx$. 
\vskip \baselineskip \hrule 
  
  
\goodbreak \vskip 2pt \hrule 
\beginliteral 
$\underline x \quad \overline y \quad \underline{\overline{x+y}}$. 
@endliteral 
$\underline x \quad \overline y \quad \underline{\overline{x+y}}$. 
\vskip \baselineskip \hrule 
  
  
\goodbreak \vskip 2pt \hrule 
\beginliteral 
$\bigl \lceil \lfloor x \rfloor \bigr \rceil \leq \bigl \lfloor \lceil x \rceil 
\bigr \rfloor$. 
@endliteral 
$\bigl \lceil \lfloor x \rfloor \bigr \rceil \leq \bigl \lfloor \lceil x \rceil 
\bigr \rfloor$. 
\vskip \baselineskip \hrule 
  
  
\goodbreak \vskip 2pt \hrule 
\beginliteral 
$\sin(2\theta) = 2\sin\theta\cos\theta 
\quad \cos(2\theta) = 2\cos^2\theta - 1  $. 
@endliteral 
$\sin(2\theta) = 2\sin\theta\cos\theta 
\quad \cos(2\theta) = 2\cos^2\theta - 1  $. 
\vskip \baselineskip \hrule 
  
  
\goodbreak \vskip 2pt \hrule 
\beginliteral 
$$\int \csc^2x\, dx = -\cot x+ C 
\qquad \lim_{\alpha\to 0} {\sin\alpha \over \alpha} = 1 
\qquad \lim_{\alpha\to \infty} {\sin\alpha \over \alpha} = 0.$$ 
@endliteral 
$$\int \csc^2x\, dx = -\cot x+ C 
\qquad \lim_{\alpha\to 0} {\sin\alpha \over \alpha} = 1 
\qquad \lim_{\alpha\to \infty} {\sin\alpha \over \alpha} = 0.$$ 
\vskip \baselineskip \hrule 
  
  
\goodbreak \vskip 2pt \hrule 
\beginliteral 
$$\tan(2\theta) = {2\tan\theta \over 1-\tan^2\theta}.$$ 
@endliteral 
$$\tan(2\theta) = {2\tan\theta \over 1-\tan^2\theta}.$$ 
\vskip \baselineskip \hrule 
  
  
\goodbreak \vskip 2pt \hrule 
\beginliteral 
\proclaim Theorem (Euclid). There exist an infinite number of primes. 
@endliteral 
\proclaim Theorem (Euclid). There exist an infinite number of primes. 
  
\vskip \baselineskip \hrule 
  
  
\goodbreak \vskip 2pt \hrule 
\beginliteral 
\proclaim Proposition 1. 
$\root n \of {\prod_{i=1}^n X_i} \leq 
{1 \over n} \sum_{i=1}^n X_i$ with equality if and only if $X_1=\cdots=X_n$. 
@endliteral 
\proclaim Proposition 1. 
$\root n \of {\prod_{i=1}^n X_i} \leq 
{1 \over n} \sum_{i=1}^n X_i$ with equality if and only if $X_1=\cdots=X_n$. 

\vskip \baselineskip \hrule 

\goodbreak \vskip 2pt \hrule 
\beginliteral 
$$ I_4 = \pmatrix{ 
1 &0 &0 &0 \cr 
0 &1 &0 &0 \cr 
0 &0 &1 &0 \cr 
0 &0 &0 &1 \cr}$$ 
@endliteral 
$$ I_4 = \pmatrix{ 
1 &0 &0 &0 \cr 
0 &1 &0 &0 \cr 
0 &0 &1 &0 \cr 
0 &0 &0 &1 \cr}$$ 
\vskip \baselineskip \hrule 

\goodbreak \vskip 2pt \hrule 
\beginliteral 
$$ |x| = \left\{ \matrix{ 
x & x \ge 0 \cr 
-x & x \le 0 \cr} \right.$$ 
@endliteral 
$$ |x| = \left\{ \matrix{ 
x & x \ge 0 \cr 
-x & x \le 0 \cr} \right.$$ 
\vskip \baselineskip \hrule 
  
\goodbreak \vskip 2pt \hrule 
\beginliteral 
\settabs \+ \hskip 2 in & \hskip .75in & \hskip 1cm& \cr 
\+ &Plums &\hfill\$1&.22 \cr 
\+ &Coffee &\hfill1&.78 \cr 
\+ &Granola &\hfill1&.98 \cr 
\+ &Mushrooms & &.63 \cr 
\+ &{Kiwi fruit} & &.39 \cr 
\+ &{Orange juice} &\hfill1&.09 \cr 
\+ &Tuna &\hfill1&.29 \cr 
\+ &Zucchini & &.64 \cr 
\+ &Grapes &\hfill1&.69 \cr 
\+ &{Smoked beef} & &.75 \cr 
\+ &Broccoli &\hfill\underbar{\ \ 1}&\underbar{.09} \cr
\+ &Total &\hfill \$12&.55 \cr
@endliteral
\settabs \+ \hskip 2 in & \hskip .75in & \hskip 1cm& \cr 
\+ &Plums &\hfill\$1&.22 \cr 
\+ &Coffee &\hfill1&.78 \cr 
\+ &Granola &\hfill1&.98 \cr 
\+ &Mushrooms & &.63 \cr 
\+ &{Kiwi fruit} & &.39 \cr 
\+ &{Orange juice} &\hfill1&.09 \cr 
\+ &Tuna &\hfill1&.29 \cr 
\+ &Zucchini & &.64 \cr 
\+ &Grapes &\hfill1&.69 \cr 
\+ &{Smoked beef} & &.75 \cr 
\+ &Broccoli &\hfill\underbar{\ \ 1}&\underbar{.09} \cr 
\+ &Total &\hfill \$12&.55 \cr 
\vskip \baselineskip \hrule 
  
  
\goodbreak \vskip 2pt \hrule 
\beginliteral 
\settabs \+ \hskip 4.5 in & \cr 
\+Getting Started \dotfill &1 \cr 
\+All Characters Great and Small \dotfill &9 \cr 
@endliteral 
\settabs \+ \hskip 4.5 in & \cr 
\+Getting Started \dotfill &1 \cr 
\+All Characters Great and Small \dotfill &9 \cr 
\vskip \baselineskip \hrule 
  
\goodbreak \vskip 2pt \hrule 
\beginliteral 
\settabs \+ \hskip 1cm&\hskip 1 cm&\hskip 1 cm& \cr 
\moveright 2 in 
\vbox{ 
\hrule width 3 cm 
\+  \vrule height 1 cm & \vrule height 1 cm & \vrule height 1 cm 
  & \vrule height 1 cm \cr 
\hrule width 3 cm 
\+  \vrule height 1 cm & \vrule height 1 cm & \vrule height 1 cm 
  & \vrule height 1 cm \cr 
\hrule width 3 cm 
\+  \vrule height 1 cm & \vrule height 1 cm & \vrule height 1 cm 
  & \vrule height 1 cm \cr 
\hrule width 3 cm 
} 
@endliteral 
\settabs \+ \hskip 1cm&\hskip 1 cm&\hskip 1 cm& \cr 
\moveright 2 in 
\vbox{ 
\hrule width 3 cm 
\+  \vrule height 1 cm & \vrule height 1 cm & \vrule height 1 cm 
  & \vrule height 1 cm \cr 
\hrule width 3 cm 
\+  \vrule height 1 cm & \vrule height 1 cm & \vrule height 1 cm 
  & \vrule height 1 cm \cr 
\hrule width 3 cm 
\+  \vrule height 1 cm & \vrule height 1 cm & \vrule height 1 cm 
  & \vrule height 1 cm \cr 
\hrule width 3 cm 
} 
\vskip \baselineskip \hrule 
  
\goodbreak \vskip 2pt \hrule 
\beginliteral 
\def\boxtext#1{% 
\vbox{% 
      \hrule 
      \hbox{\strut \vrule{} #1 \vrule}% 
      \hrule 
     }% 
} 
\moveright 2 in \vbox{\offinterlineskip 
\hbox{\boxtext{6}\boxtext{1}\boxtext {8}} 
\hbox{\boxtext{7}\boxtext{5}\boxtext{3}} 
\hbox{\boxtext{2}\boxtext{9}\boxtext{4}} 
} 
@endliteral 
\def\boxtext#1{% 
\vbox{% 
      \hrule 
      \hbox{\strut \vrule{} #1 \vrule}% 
      \hrule 
     }%
}
\moveright 2 in \vbox{\offinterlineskip
\hbox{\boxtext 6\boxtext 1\boxtext 8}
\hbox{\boxtext 7\boxtext 5\boxtext 3}
\hbox{\boxtext 2\boxtext 9\boxtext 4}
}
\vskip \baselineskip \hrule

\goodbreak \vskip 2pt \hrule 
\beginliteral
{\leftskip=2in\obeylines\tengr
\qquad n'oon d`e \digamma a'utw
p'ampan >a'errei
}
@endliteral
{\leftskip=2in\obeylines\tengr
\qquad n'oon d`e \digamma a'utw
p'ampan >a'errei
}
\endgroup % end raggedright

\datestamp
\bye
