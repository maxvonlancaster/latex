% This is the Very Short Guide to LaTeX, a 4pp folder designed as an
% aide-mémoire for people who have just done a course in LaTeX and
% need something by them on their desk the next day to refresh their
% memories. It is NOT intended as a substitute for full documentation.
%
% Copyright 2006-2009 by Silmaril Consultants <peter@silmaril.ie>
%
% This is free documentation, issued under the LaTeX Rroject Public
% License. For details see
% http://www.ctan.org/tex-archive/help/Catalogue/licenses.lppl.html
%
% 2009-10-01
%
% Please do NOT complain that there is no math here. It won't fit.
% Someone else can do a math leaflet. IANAM.
%
%%%%%%%%%%%%%%%%%%%%%%%%%%%%%%%%%%%%%%%%%%%%%%%%%%%%%%%%%%%%%%%%%%%%%%
%
% Acknowledgments
%
% Thanks to all those on c.t.t whose suggestions made this version more
% useful. 
%
% Alan Munn for the distinction between base font and font package and
% for suggesting ``expected functionality'' packages on sectioning,
% long tables, headers, and double spacing, and links to FAQ and CTAN.
%
% Lars Madsen, Will Robertson, Robin Fairbairns and others for asking
% for the center environment in tables and figures to be replaced with
% the \centering command (I still disagree, but WTF :-)
%
% Tsy for spotting a typo in the table code sample.
%
% Rob Felty and Domink Wass for pointing me at Winston Chang's
% excellent LaTeX Cheat Sheet summary.
%
%%%%%%%%%%%%%%%%%%%%%%%%%%%%%%%%%%%%%%%%%%%%%%%%%%%%%%%%%%%%%%%%%%%%%%
%
% This is laid out for A5 pages with margins optimised for my HP K7100
% so that I can impose two pages per A4 sheet and print double-sided.
% It has not been tested at letterpaper size (and may not fit).
%
\documentclass[a5paper]{article}
%
% The 12mm margin is needed because my printer cannot print the
% bottom 12mm of a sheet, so if I print 2-up A5 pages on an A4
% sheet, I need to allow for the fact that one edge of two of the
% pages must have a 12mm margin -- so it's preferable to make it
% the same all round.
%
% For HP K7100
\usepackage[margin=12mm,nohead]{geometry}
%
%%%%%%%%%%%%%%%%%%%%%%%%%%%%%%%%%%%%%%%%%%%%%%%%%%%%%%%%%%%%%%%%%%%%%%%%
%
% Packages
%
\usepackage{ucs}
\usepackage[utf8x]{inputenc}
\usepackage[T1]{fontenc}
\usepackage{utopia,graphicx,url,fancybox,mdwlist,fix-cm,multicol}
\usepackage{fancyvrb,bbding,parskip,ragged2e,calc,ifthen,textcomp,array}
\usepackage[svgnames]{xcolor}
\usepackage[scaled=.95]{helvet}
%\usepackage[scaled=.87]{luximono}
%
%%%%%%%%%%%%%%%%%%%%%%%%%%%%%%%%%%%%%%%%%%%%%%%%%%%%%%%%%%%%%%%%%%%%%%%%%
%
% Assorted extra commands
%
% Provide a lite font option in case you want to use a font that has it
% (uncomment only ONE of the next two lines!)
\newcommand{\lite}{\relax}
%\newcommand{\lite}{\fontseries{l}\selectfont}
%
% Use a pointy pencil from the bbding package for itemized lists
\renewcommand{\labelitemi}{\raisebox{-.5ex}{\textcolor{Purple}{\PencilRight}}}
%
% Provide commands for distinguishing examples and their output
\newcommand{\example}{\color{DarkGreen}\textsf{\bfseries Example:} \color{Grey}}
\def\endexample{\color{black}}
\newcommand{\gives}[1]{\thinspace\color{DarkGreen}\raisebox{-.7ex}
  {\ArrowBoldRightStrobe}~\color{DarkBlue}{\fontfamily{cmr}\selectfont
    #1}\color{black}}
\newcommand{\nobox}{~\raisebox{-.25ex}{\footnotesize\Square\llap\XSolidBrush}}
\newcommand{\yesbox}{~\raisebox{-.25ex}{\footnotesize\Square\llap\CheckmarkBold}}
%
% In fancyverb's Verbatim, allow rubric embedded as marked text in red
\newcommand{\comment}[1]{\ifthenelse{\equal{#1}{ }}{}{#1}\normalfont
  \itshape\footnotesize\color{red}}
% This reverts rubric back to normal, to avoid embedded spaces
\newcommand{\revert}{\ttfamily\upshape\normalsize\color{black}}
%
% Provide keyboard or screen keys or buttons and their icons
\newcommand{\button}[2][\relax]{\if#1\relax
  \ovalbox{\sffamily\bfseries\footnotesize#2}%
  \else\raisebox{#1}{\ovalbox{\sffamily\bfseries\footnotesize#2}}\fi}
\newcommand{\icon}[1]{\raisebox{-2pt}{\includegraphics[height=1em]{#1}}}
%
% Define a month-year date European way round
\newcommand{\monthdate}{\ifcase\month\or January\or
  February\or March\or April\or May\or June\or July\or August\or
  September\or October\or November\or December\fi\space\number\year}
%
% There's something odd about the hyphenation...these were broken:
\hyphenation{mathe-mati-cal auto-mati-cal-ly docu-ment docu-ments
  statis-ti-cal}
%
%%%%%%%%%%%%%%%%%%%%%%%%%%%%%%%%%%%%%%%%%%%%%%%%%%%%%%%%%%%%%%%%%%%%%%%%
%
% Formatting changes
%
\makeatletter
% title block
\renewcommand{\maketitle}{%
  \hbox to\textwidth{%
    \vbox{\hsize=.75\textwidth\huge\bfseries\raggedright
      \@title}
    \hss
    \vbox{\hsize=.25\textwidth\scriptsize\raggedright
      \@author
      \par
      \textit{\@date}}}
  \par\bigskip\hrule\medskip
}
% notes
\renewenvironment{note}{\small\textbf{Note.}\quad}{\par}
% pack headings together more tightly
\renewcommand\section{\@startsection{subsection}{1}{\z@}%
  {-0pt\@plus -.2ex \@minus -.2ex}%
  {1pt \@plus .2ex}%
  {\normalfont\Large\bfseries}}
\renewcommand\subsection{\@startsection{subsection}{2}{\z@}%
  {-0pt\@plus -.2ex \@minus -.2ex}%
  {1pt \@plus .2ex}%
  {\normalfont\large\bfseries}}
\renewcommand\subsubsection{\@startsection{subsubsection}{3}{\z@}%
  {-.5ex\@plus -1ex \@minus -.2ex}%
  {.25ex \@plus .2ex}%
  {\normalfont\normalsize\bfseries}}
\makeatother
\renewcommand{\footnoterule}{\kern-3pt
  \hrule height.2pt width.25\columnwidth\kern 2.8pt}
\setcounter{secnumdepth}{0}% we don't want numbered sections
\setlength{\leftmargini}{18pt}% don't waste margin space in lists
\pagestyle{empty}% no page numbers, we're faking it
\spaceskip=4pt plus3pt minus2pt% let text go very sloppy to fit
%
%%%%%%%%%%%%%%%%%%%%%%%%%%%%%%%%%%%%%%%%%%%%%%%%%%%%%%%%%%%%%%%%%%%%%%
%
% WARNING: a lot of stuff is faked up in this document, because we're
% combining incompatible things for the sake of showing examples which
% would not normally occur in the same document (eg both scientific
% and humanities-style bibliographic references). The demands of space
% also mean that there is far more manual adjustment needed than would
% be seen in a normal continuous-text document. Live with it.
%
%%%%%%%%%%%%%%%%%%%%%%%%%%%%%%%%%%%%%%%%%%%%%%%%%%%%%%%%%%%%%%%%%%%%%%
%
\begin{document}

\title{The very short guide to typesetting with~\LaTeX}
\author{Silmaril Consultants\\
  \textbf{Textual Therapy Division}\\
  \protect\url{latex.silmaril.ie}}
\date{\monthdate}
\maketitle

\subsection{What's this all about? What's \LaTeX?}

\LaTeX\ is a document preparation system for the \TeX\ typesetting
program. It enables you to produce publication-quality output with
great accuracy and consistency. \LaTeX\ works on any computer and
produces industry-standard PS or PDF documents. It is available both
in free (open-source) and commercial implementations. \LaTeX\ can be
used for any kind of document, but it is especially suited to those
with a complex structure, repetitive formatting, or notations like
mathematics\footnote{For reasons of space this guide does not cover
  details of mathematics typesetting.}; or where technical stability,
dimensional accuracy, or a persistent and non-proprietary file format
are needed.


\subsection{Syntax (how to type \LaTeX\ commands --- these are the rules)}

\begin{itemize*}
  \item \textbf{All \LaTeX\ commands begin with a backslash}.\\
    \example \verb+\tableofcontents+\endexample
  \item \textbf{If a command needs text to work
    with, it goes in curly braces}.\\
    \example \verb+\title{Irisches Tagebuch}\author{Heinrich Böll}+\endexample
  \item \textbf{If options are used, they go in square brackets first}.\\
    \example \verb+\documentclass[a4paper,11pt]{book}+\endexample
  \item \textbf{Space after commands \emph{without} braces gets suppressed}.\\
    \example \verb+Copyright \copyright +\texttt{\number\year} 
    \gives{Copyright \copyright \number\year}~\nobox\\
    To prevent this, put empty curly braces after the command.\\
    \example \verb+Copyright \copyright{} +\texttt{\number\year} 
    \gives{Copyright \copyright{} \number\year}~\yesbox
  \item \textbf{Curly braces are also used to restrict the scope of
    effects inside them}.\\
    \example \verb+Some {\tiny little} word+ \gives{Some {\tiny little} word}
\end{itemize*}

\subsection{Creating and typesetting your document}

\begin{enumerate*}\setlength{\fboxsep}{1pt}
   \item Create your document using any suitable plain-text editor
     with \LaTeX\ controls, eg \textit{\TeX shop} (Mac), \textit{\TeX
       nicCenter} (Win), \textit{Kile} (Linux), \textit{Emacs} (all);
   \item Save the file with a name ending in \verb+.tex+
     (\emph{never} use spaces in filenames!);
   \item Use the toolbar buttons or menu items in your editor to typeset
     and display the document;\label{typeset}
   \item Make any changes needed in your original document and repeat
     step \ref{typeset}. 
\end{enumerate*}

\vfill
\begin{note}
This guide shows only a tiny fraction of \LaTeX's power. For
information, visit the \TeX\ Users Group site (\url{www.tug.org}). For
help, see the FAQ (\url{www.tex.ac.uk/faq}) and the Usenet newsgroup
\url{comp.text.tex}. For packages, use the Comprehensive \TeX\ Archive
Network (\url{www.ctan.org}). For documentation, use the sources in
the \textit{References} [2].
\end{note}

\begin{multicols}{2}\small\parskip4pt

\subsubsection{Basic document structure}

Here's the skeleton of a \LaTeX\ document. These three lines are
\emph{compulsory}: your document will not work without them:

\begin{Verbatim}[frame=single,fontsize=\small,commandchars=!<>]
\documentclass{article}
!comment your preamble goes here (extra setups, if any)
\begin{document}
!comment your document text goes here
\end{document}
\end{Verbatim}

\begin{itemize*}
  \item The document class name must be one of \verb+book+,
    \verb+article+, or \verb+report+, or an extra one you have
    downloaded and installed (eg \verb+thesis+, \verb+memoir+, etc).
  \item There are paper size options \verb+a4paper+
    (210\thinspace mm$\times$297\thinspace mm) and \verb+letterpaper+
    (8½$''\times$11$''$) and others (eg \verb+a5paper+).
  \item There are base type size options \verb+10pt+ (the default),
    \verb+11pt+, and \verb+12pt+.
\end{itemize*}

\subsubsection{Front matter}

The \textbf{preamble} is where you specify any extra \textbf{packages}
(\LaTeX\ plugins) such as typefaces or special formatting
requirements, and where you put any changes to standard features.

\begin{Verbatim}[frame=single,fontsize=\small,commandchars=!<>]
\documentclass[a4paper,11pt]{book}
\usepackage{charter,graphicx}
\setlength{\parindent}{1em}
\begin{document}
\title!comment{your document title!revert}
\author!comment{your name!revert}
\date!comment{date of publication!revert}
\maketitle
\begin{abstract}
!comment the paragraphs of the abstract go here
\end{abstract}
\tableofcontents
!comment rest of the document goes here
\end{document}
\end{Verbatim}

In a typical report or article, the title, author, date, abstract
(summary), and table of contents (optional) all go at the start,
followed by your text. 

Leave a blank line between paragraphs as you type. This means `start a
new paragraph', \emph{not} `leave a blank line'. You can control
spacing and indentation by setting \verb+\parskip+ and
\verb+\parindent+ (see examples), or with the \textsf{parskip}
package.

\subsubsection{Sections and cross-references}

Sections get numbered automatically in bold type, and get included in
the Table of Contents (if any). Numbering can be turned off
selectively. Section heading layout can be modified with the
\textsf{sectsty}, \mbox{\textsf{titlesec}}, and other packages. Use
the \textsf{babel} package for other languages.

\begin{Verbatim}[frame=single,fontsize=\small,commandchars=!<>]
!comment (Preamble, titling, and abstract as above)
\tableofcontents
\section!comment{heading of a section!revert}
!comment text for the section goes here
...as shown in section \ref{blah}.
\subsubsection!comment{heading of a subsection!revert}
!comment text for the subsection goes here
\section!comment{heading of a new section!revert}
\label{blah} !comment make up name for the label
!comment text for the section goes here
\end{document}
\end{Verbatim}

For cross-references, use \verb+\label{...}+ to label the target 
and \verb+\ref{...}+ and/or \verb+\pageref{...}+ to refer to
it. Make up the label values: \LaTeX\ will use them to work
out the right numbers to print.\\
\example{...\ttfamily section \verb+\ref{blah}+ on p.\\
\verb+\pageref{blah}+.}\gives{...section 3 on p.9}.

\subsubsection{Typefaces}

\LaTeX's default typeface is Computer Modern. There is a selection of
other typeface packages (use them in your Preamble):

{\footnotesize\tabcolsep4pt
\def\ff#1{\fontfamily{#1}\selectfont}
\begin{tabular}{l>{\ttfamily}l|l>{\ttfamily}l@{}}
\ff{ptm}Times&mathptmx&\ff{pcr}Courier&courier\\
\ff{ppl}Palatino&mathpazo&\ff{pag}\scriptsize Avant Garde&avant\\
\ff{pbk}Bookman&bookman&\ff{phv}Helvetica&helvet\\
\ff{bch}Charter&charter&\ff{pzc}Zapf Chancery&chancery\\
\ff{put}Utopia&utopia&\fontencoding{OT1}\ff{pnr}Pandora&pandora\\
\multicolumn3l{\ff{pnc}New Century Schoolbook}&newcent\\
\end{tabular}
}

\columnbreak\parskip5pt

Dozens of others are available, including mathematical and decorative
fonts.  To switch to a sans-serif type family (eg Helvetica, Avant
Garde), use \verb+\sffamily+ in your text. To change font for a word
or phrase, use these commands (they can be nested---see below):

{\small
\begin{tabular}{l@{\enspace}>{\ttfamily\char'134 text}l<{\char'173
      Hello\char'175}@{}>{\ \gives\bgroup}l<{\ Hello\egroup}}
Italics&it&\itshape\\
Boldface&bf&\bfseries\\
Smallcaps&sc&\scshape\\
Sans-serif&sf&\sffamily~~\\
Monospace&tt&\ttfamily\\
\end{tabular}}

\example \verb+\textit{\textbf{\textsf+\linebreak
\verb+{bold ital sans}}}+
\gives{\textit{\textbf{\textsf{bold ital sans}}}}

Font sizing is automatic for titles, headings, and footnotes. There
are some named step-size commands (in points, relative to the base size):

{\small
\begin{tabular}{>{\ttfamily\char'134}lrrr}
normalsize&10&11&12\\\hline\vrule height1.1em width0pt
tiny&5&6&7\\
scriptsize&6&7&8\\
footnotesize&7&8&9\\
small&9&10&11\\
large&11&12&14\\
Large&12&14&17\rlap*\\
LARGE&14&17\rlap*&20\rlap*\\
huge&17\rlap*&20\rlap*&24\rlap*\\
Huge&20\rlap*&24\rlap*&28\rlap*\\
\end{tabular}
\quad
\rotatebox[origin=c]{90}{\scriptsize* sizes rounded here to save space}
}

but you can specify an exact size with the \textsf{fix-cm} package and
the command
\verb+\fontsize{+\texttt{\textit{pp}}\verb+}{+\texttt{\textit{bb}}\verb+}\selectfont+
for any point-size (\textit{pp}) on any baseline (\textit{bb}). Group
(enclose) the command \emph{with} its applicable text in curly braces
to prevent it affecting the rest of the document. For wider line-spacing
(eg in theses) use the \textsf{setspace} package. You can also use
\linebreak colour with the 
\\\textsf{xcolor} package and\hfil \smash{\raisebox{0ex}{\Huge 
\textcolor[HTML]{2F50AD}{G}%
\textcolor[HTML]{B32F17}{o}% 
\textcolor[HTML]{F3C20B}{o}%
\textcolor[HTML]{2F50AD}{g}% 
\textcolor[HTML]{48C847}{l}%
\textcolor[HTML]{B32F17}{e}}}%
% web colour for l was 567122
% web colour for o2 was CAB340
\\the \verb+\color{+\texttt{\textit{name}}\verb+}+ command.

\subsubsection{Lists}

There are three basic kinds: \textbf{itemized} lists (bulleted);
\textbf{enumerated} lists (numbered or lettered); and
\textbf{descriptive} lists (topic-and-explanation format). Others can
be defined, or downloaded from CTAN [5].

\begingroup\tiny
\renewcommand{\labelitemi}{\textbullet}
\leftmargini=2em
\setlength{\tabcolsep}{4pt}
\begin{tabular}{%@{}%
    p{.22\columnwidth}|%
    p{.28\columnwidth}|%
    p{.33\columnwidth}}
\begin{Verbatim}
\begin{itemize}
\item 1lb Sugar
\item ½pt Cream
\item Chocolate
\end{itemize}
\end{Verbatim}
&
\begin{Verbatim}[commandchars=!()]
\begin{enumerate}
\item Mix together
\item Boil to 112!textdegree()C
\item Stir and cool
\end{enumerate}
\end{Verbatim}
&
\begin{Verbatim}
\begin{description}
\item[Fudge] is fun...
\item[Broccoli] sucks...
\item[Exercise] is good
\end{description}
\end{Verbatim}
\\\hline
\begin{itemize}
\item 1lb Sugar
\item ½pt Cream
\item Chocolate
\end{itemize}
&
\begin{enumerate}
\item Mix together
\item Boil to 112\textdegree C
\item Stir and cool
\end{enumerate}
&\tiny
\begin{description}
\item[Fudge] is fun but not if made too often.
\item[Broccoli] sucks, period.
\item[Exercise] is good for you if taken daily and not to extremes.
\end{description}
\end{tabular}
\endgroup
You can nest lists inside each other. See packages like
\textsf{paralist} and \textsf{mdwlist} to control list formatting.

\subsubsection{Tables and figures}

Formal tables and figures \emph{float} (change position to fill
available space) so they may not be printed where you typed them. 

\begin{Verbatim}[frame=single,fontsize=\footnotesize,commandchars=!<>]
\begin{table}
\caption{Mean growth rate and intakes 
of supplement, milk, and water for 4 
diets.}
\label{dietgrowth}\centering
\begin{tabular}{|l|r|r|r|r|}\hline
&Growth&Supplement&Milk&Water\\
Supplement&rate&intake&intake&intake\\
&(g/day)&(g/day)&(ml/kg$^{0.75}$)&
(ml/kg$^{0.75}$)\\\hline 
Lucerne &145&450&10.5&144\\\hline 
Sesbania&132&476& 9.2&128\\\hline 
Leucaena&128&364& 8.9&121\\\hline 
None    & 89&  0& 9.8&108\\\hline 
\end{tabular}
\end{table}
\end{Verbatim}

% fake it
\begin{center}\fontfamily{cmss}\fontsize56\selectfont
\setlength{\tabcolsep}{2pt}
\setlength{\arrayrulewidth}{.2pt}
\begin{tabular}{@{}|l|r|r|r|r|@{}}
\multicolumn5l{\textbf{Table 2}: \textit{Mean growth rate and intakes 
of supplement,}}\\
\multicolumn5l{\textit{milk, and water for four diets} (after
    Sherington, J, undated)}\\[6pt]\hline
\vrule height1.1em width0pt&Growth&Supplement&Milk&Water\\[-1pt]
Supplement&rate&intake&intake&intake\\[-1pt]
&(g/day)&(g/day)&(ml/kg\textsuperscript{\fontsize34\selectfont 0.75})&(ml/kg\textsuperscript{\fontsize34\selectfont 0.75})\\[3pt]\hline 
Lucerne &145&450&10.5&144\\\hline 
Sesbania&132&476& 9.2&128\\\hline 
Leucaena&128&364& 8.9&121\\\hline 
None    & 89&  0& 9.8&108\\\hline 
\end{tabular}
\end{center}
Packages like \textsf{longtable} and \textsf{array} can help
with more complex table formats.

\colorbox{LightGrey}{\color{black}\begin{minipage}{.965\columnwidth}%
\sffamily\scriptsize\bfseries\raggedright
For help, see the links on the front and back pages. There is a
summary of common commands at
\url{www.stdout.org/~winston/latex/latexsheet.pdf} and a
comprehensive list at
\url{computing.ee.ethz.ch/.soft/latex/green/ltx-2.html}. 
\end{minipage}}

\end{multicols}

\textbf{Tables and Figures}, \textit{continued}

\begin{Verbatim}[frame=single,fontsize=\footnotesize,commandchars=!<>]
\begin{figure}
\caption{Swiss and Dutch Mennonite migrations of the 1700s and 1800s}\label{lmig}
\centering !comment (graphics must be EPS files for standard !LaTeX; but JPG, PNG, or PDF for pdf!LaTeX)
\includegraphics[width=.8\columnwidth]{menno-a}
\\\scriptsize Courtesy of Paul C. Adams, Department of Geography and the 
Environment, University of Texas at Austin. \cite{adams}\end{figure}
\end{Verbatim}

\begin{center}\fontfamily{cmss}\selectfont
\textbf{Figure 1}: \textit{Swiss and Dutch Mennonite migrations of the
  1700s and 1800s}\\[3pt]
\includegraphics[width=.8\columnwidth]{menno-a}\\\scriptsize
Courtesy of Paul C. Adams, Department of Geography
and the Environment, University of Texas at Austin.~[1]
\end{center}

\subsubsection{Footnotes, citations, references, and indexes (back matter)}
\setcounter{footnote}{1} 
You do footnotes with a simple command,\footnote{Like this.} see
below.  Citations using BIB\TeX{} (Patashnik, 1988) are also easy (see
[2], \S7.4.2) and there are packages for more complex formats for
journals and publishers. You can add indexes with the \verb+\index+
command and the \textsf{makeindex} program.

\begin{Verbatim}[frame=single,fontsize=\footnotesize,commandchars=!<>]
You do footnotes with a simple command,\footnote{Like this.} see below. Citations 
using BIB\TeX{} \citeauthoryear{oren} are also easy (see \cite[§7.4.2]{flynn} and 
there are packages for more complex formats for journals and publishers. You can 
add  indexes with the \verb`\index` command and the \textsf{makeindex} program.
!comment add the following at the end of your document and create!ttfamily myrefs.bib!rmfamily (see BIB!TeX manual [3])
\bibliography{myrefs}\bibliographystyle{apalike}
\end{Verbatim}

\vspace*{-3pt}
\subsubsection{References}

\begin{enumerate*}\footnotesize\RaggedRight
\item Adams, Paul C. \textit{Linguistic Chaos in Montreal},
  \url{www.utexas.edu/depts/grg/adams/chaos.ppt}, 2/59, Oct 2006.
\item {\bfseries Flynn, P. \textit{Formatting Information}, 
  2005, at \url{latex.silmaril.ie/formattinginformation/}}
\item Patashnik, O. \textit{BIB\TeX ing}, \TeX\ Users Group, 1988
  (distributed with all copies of \LaTeX).
\item Sherington, J. example table in `Informative Presentation of
  Tables, Graphs and Statistics', 4.2, Statistical Services Centre,
  University of Reading,
  \url{www.reading.ac.uk/ssc/publications/guides/toptgs.html}
\item \bfseries\TeX\ Users Group, for \TeX\ Live 
  (\url{www.tug.org/texlive/}) and CTAN
  (Comprehensive \TeX\ Archive Network) for downloads (\url{www.ctan.org}).
\end{enumerate*}

\begin{note}
Commercial implementations of \TeX\ with business
support are available from Personal \TeX, Inc (PC\TeX); Blue Sky
Research (Textures~[Mac]); MacKichan Software, Inc (Scientific Word);
Micropress, Inc (V\TeX), True\TeX\ Software (True\TeX), and others.
\end{note}

\end{document}
