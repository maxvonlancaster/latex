%%
%% This file can be distributed and/or modified under the
%% conditions of the LaTeX Project Public License, either version 1.3
%% of this license or (at your option) any later version.
%% The latest version of this license is in
%%   http://www.latex-project.org/lppl.txt
%% and version 1.3 or later is part of all distributions of LaTeX
%% version 2003/12/01 or later.
%% 
%% This work has the LPPL maintenance status "maintained".
%% 
%% The Current Maintainer of this work is Lars Madsen (daleif@imf.au.dk).
%%
\begin{filecontents}{chapterexample.tex}
\chapter{A chapter title}
Some text at the beginning of a chapter. And we add a lot of text to
make sure that it spans more than one line.
\par\fancybreak{$***$}\par
\chapter*{A non-numbered chapter title}
Some text at the beginning of a chapter. And we add a lot of text to
make sure that it spans more than one line.
\thispagestyle{empty}
\end{filecontents}
\begin{filecontents*}{process.pl}
#!/usr/bin/perl
# licensed under the GPL, by Lars Madsen, 2006/03/24
use Getopt::Long;
my $f = '';
my $k = '';
my $p = '';
my $tmppdf  = 'tmp.pdf';
my $postfix = '-style';
GetOptions('f:s' => \$f,'k:s' => \$k,'p:s' => \$p);
my @styles = ();
my %pages  = ();
if ( $k ) { compile_file("$k$postfix"); exit ;}
open my $file ,'<', $f or die "Cannot open '$f': $!";
for my $l (<$file>) {
    chomp $l;
    next if $l =~ /^\s*$/;
    if ( $l =~ / page/ ) {
	($Page) = ( $l =~ / page (.*)/ ) ;
	$l =~ s/ page.*//;
	$pages{$l} = $Page;
    }
    push @styles,$l;
}
close $file;
for my $style ( @styles ) {
    compile_file($style);
}
print "done\n\n";
sub compile_file {
    my $style = shift;
    my @tmp = ();
    system("pdflatex", "$style.tex")          == 0 or warn "$!";
    system("pdfcrop", "$style.pdf","$tmppdf") == 0 or warn "$!";
    system("mv", "$tmppdf","$style.pdf")      == 0 or warn "$!";
    if ( $pages{$style} || $p ) {
	@tmp = split /\,/,  $pages{$style} ? $pages{$style} : $p ;
	for my $p ( @tmp ) {
	    system("pdftops", "-eps","-f","$p","-l","$p", "$style.pdf", "$style-$p.eps" )   == 0 or warn "$!";
	    warn "Created $style-$p.eps\n";
	}
    }
    else {
	system("pdftops", "-eps", "$style.pdf")   == 0 or warn "$!";
    }
    print "Done converting $style.pdf\n";
    return;
}
\end{filecontents*}
%$
\documentclass[a4paper,11pt,article]{memoir}
\def\MyFileVersion{Version 1.4c, 2006/07/28}
\setlrmarginsandblock{2.5cm}{*}{1} 
\setulmarginsandblock{2.5cm}{3.5cm}{*}
\setmarginnotes{2.5mm}{2cm}{1em}
\checkandfixthelayout
\usepackage[latin1]{inputenc}
\usepackage[english]{babel}
\usepackage[T1]{fontenc}
\usepackage{calc,graphicx,url,fancyvrb,multicol,keyval}
\usepackage[draft]{fixme}
\usepackage{fourier}
\usepackage[scaled]{luximono}
\newcommand\starbreak{\fancybreak{\decosix\quad\decosix\quad\decosix}}
\usepackage[scaled]{berasans}


\raggedbottom
\fvset{frame=lines,framesep=3mm,fontsize=\small}
\newoutputstream{StyleList}
 \newoutputstream{OutputStyle}%
 \openoutputfile{\jobname.styles}{StyleList}
\def\OutputStylePostfix{-style}
\def\CurrentChapterStyle{}
\makeatletter
% a little redefinition of keyval
\def\KV@split#1=#2=#3\relax{%
  \KV@@sp@def\@tempa{#1}%
  \ifx\@tempa\@empty\else
    \expandafter\let\expandafter\@tempc
      \csname\KV@prefix\@tempa\endcsname
    \ifx\@tempc\relax
%      \KV@errx
%       {\@tempa\space undefined noget}%
    \xdef\CurrentChapterStyle{\@tempa}%
    \else
      \ifx\@empty#3\@empty
        \KV@default
      \else
        \KV@@sp@def\@tempb{#2}%
        \expandafter\@tempc\expandafter{\@tempb}\relax
      \fi
    \fi
  \fi}
\define@key{MCS}{pages}{%\typeout{xxx: #1}
  \global\@namedef{MCS@pages@\CurrentChapterStyle}{#1}  
}
\newif\ifSCS@full
\newcounter{MCS}
\newenvironment{@showchapterstyle}[1]{%
  \setkeys{MCS}{#1}%
  \ifSCS@full%
%  \gdef\CurrentChapterStyle{#1}%
  \edef\hest{\CurrentChapterStyle\OutputStylePostfix\space page \@nameuse{MCS@pages@\CurrentChapterStyle}}
  \addtostream{StyleList}{\hest}%
  \else%
%  \gdef\CurrentChapterStyle{#1}%
  \addtostream{StyleList}{\CurrentChapterStyle\OutputStylePostfix}%
  \fi%
  \openoutputfile{\CurrentChapterStyle\OutputStylePostfix.tex}{OutputStyle}%
  \ifSCS@full%
  \addtostream{OutputStyle}{%
    \protect\let\protect\STARTCODE\relax^^J%
    \protect\let\protect\STOPCODE\relax^^J%
    \protect\STARTCODE%
  }%
  \else%
  \addtostream{OutputStyle}{%
    \protect\documentclass{memoir}^^J%
    \protect\let\protect\STARTCODE\relax^^J%
    \protect\let\protect\STOPCODE\relax^^J%
    \protect\let\protect\clearforchapter\par^^J%
    \protect\STARTCODE%
  }%
  \fi%
  \writeverbatim{OutputStyle}}{%
  \endwriteverbatim\relax%
  \ifSCS@full%
  \addtostream{OutputStyle}{%
    \protect\STOPCODE%
  }
  \else% 
  \addtostream{OutputStyle}{%
    \protect\chapterstyle{\CurrentChapterStyle}^^J%
    \protect\STOPCODE^^J%
    \protect\setlength\afterchapskip{\onelineskip}^^J%
    \protect\setlength\beforechapskip{\onelineskip}^^J%
    \protect\begin{document}^^J%
      \protect\input{chapterexample.tex}^^J%
      \protect\end{document}%
  }%
  \fi%
  \closeoutputstream{OutputStyle}%
  \edef\FancyVerbStartString{\string\STARTCODE}%
  \edef\FancyVerbStopString{\string\STOPCODE}%
  \vskip\z@\@plus\bottomsectionskip
  \penalty\z@
  \vskip\z@\@plus -\bottomsectionskip
  \phantomsection
  \addcontentsline{toc}{section}{\CurrentChapterStyle}
  \VerbatimInput[
  label=\textnormal{\small Source for the \textsf{\CurrentChapterStyle}  style}
  ]{\CurrentChapterStyle-style.tex}%%
  \par\noindent%
  \IfFileExists{\CurrentChapterStyle\OutputStylePostfix.pdf}{%
    \fboxsep=4pt%
    \begin{adjustwidth}{-\fboxsep-\fboxrule}{-\fboxsep-\fboxrule}%
      \begin{framed}%
        \@ifundefined{MCS@pages@\CurrentChapterStyle}{%
          \includegraphics[width=\textwidth]{\CurrentChapterStyle\OutputStylePostfix}%
        }{%
          \edef\nisse{\@nameuse{MCS@pages@\CurrentChapterStyle}}
          \@for\ITEM:=\nisse\do{
            \ifpdf%
              \includegraphics%
                [width=\textwidth,page=\ITEM]{\CurrentChapterStyle\OutputStylePostfix}%
            \else%
              \includegraphics%
                [width=\textwidth]{\CurrentChapterStyle\OutputStylePostfix-\ITEM}%
            \fi%
            \bigskip%
            \fancybreak{$***$}%
            \bigskip
          }%
        }%
      \end{framed}%
    \end{adjustwidth}
  }{\fbox{File \CurrentChapterStyle-style.* does not exist}}
  \vskip1.5\abovedisplayskip\noindent%
}
% the two actual environments, the stared one will let you add entire
% documents, while the unstared one will only display sniplets
\newenvironment{showchapterstyle}[1]{%
\SCS@fullfalse\@showchapterstyle{#1}}{\end@showchapterstyle}
\newenvironment{showchapterstyle*}[1]{%
\SCS@fulltrue\@showchapterstyle{#1}}{\end@showchapterstyle\SCS@fullfalse}
\newcommand\@Arg[1]{\textnormal{$\langle$\textit{#1}$\rangle$}}
\newcommand\@Args[1]{\texttt{\{\textnormal{$\langle$\textit{#1}$\rangle$}\}}}
\newcommand\Arg{\@ifstar{\@Args}{\@Arg}}
\newcommand\cs[1]{\texttt{\textbackslash #1}}
\makeatother
\newenvironment{syntax}{%
  \vskip.5\onelineskip%
  \begin{adjustwidth}{0pt}{0pt}
    \parindent=0pt%
    \obeylines%
    \let\\=\relax%
  }{%
  \end{adjustwidth}%
  \vskip.5\onelineskip%
}
\newenvironment{syntax*}{%
  \vskip.5\onelineskip%
  \begin{adjustwidth}{0pt}{0pt}
    \parindent=0pt%
  }{%
  \end{adjustwidth}%
  \vskip.5\onelineskip%
}

\newtheorem{remark}{Remark}

\AtEndDocument{\closeoutputstream{StyleList}}
\pagestyle{plain}

\ifpdf
\usepackage[colorlinks]{hyperref}
\usepackage{memhfixc}
\fi



\begin{document}

\title{Various chapter styles for the memoir class\thanks{\MyFileVersion}}
\author{Lars Madsen\thanks{Email: \protect\url{daleif@imf.au.dk}}}
\maketitle

The main idea behind this document is to demonstrate various either
contributed or inspired chapter styles for the memoir class.

If you have style you would like to contribute a style/implementation,
please send it with a minimal example to \url{daleif+memoir@imf.au.dk}
and I will include it into this document.

\bigskip
\starbreak

\bigskip


\noindent The visual examples you will find later in this document
have all been made using external documents and included as images
(eps or pdf). As such, all images are scaled to have the same width as
the  text in this document, therefore some images are scaled down.

Also, please do not trust the spacing between the chapter title and
the start of the following text. This \verb+\afterchapskip+-spacing is
silently reduced (to \verb+\onelineskip+) in order to save space, the
same goes for \verb+\beforechapskip+.

\starbreak

In any good chapter style design one should have given a thought at
both the normal numbered style as well as the unnumbered
style. Therefore the example text features both a numbered chapter and
an unnumbered. (I have relaxed \verb+\clearforchapter+ in order to
have both on the same side.)

The sample text used is
\VerbatimInput[
label={chapterexample.tex},
fontsize=\small
]{chapterexample.tex}

\starbreak

If you want to use one of the styles presented in this document, 
then there is no need to start retyping it all your self. Simply
download the source for this document (\texttt{\jobname.tex}) from
\url{http://www.imf.au.dk/system/latex/artikler/MemoirChapStyles/}
(where you will find the latest version) or from CTAN, via
\url{http://www.ctan.org/tex-archive/info/MemoirChapStyles/}. Run it
once through \LaTeX, then you will 
get a file called \Arg{Name of style}\texttt{-style.tex}, which is the
source code for example displaying that particular style. Then just
copy the code from there.


\section*{Acknowledgement}

Acknowledgement goes (of course) to Peter Wilson for creating the
memoir class in the first place. But also to the people who
contributed with styles or comments: Danie Els, David Chadd, Pluton
(name used on \textsc{ctt}), Erik Quaeghebeur, Donald Arseneau plus
the those who posted memoir chapter styles on news groups, I hope it
is okay that I include them here.




\section*{TODO}
\label{sec:todo}

Have a look at the chapter styles offered by \texttt{fncychap} and
\texttt{titlesec}. 


\setlength\columnsep{8mm}
\begin{multicols}{2}
  \tableofcontents*
\end{multicols}

\newpage

\chapter{A little background}
\label{cha:little-background}

As you might already know the memoir class includes a feature to
switch the look and feel of a chapter title on a chapter to chapter
basis. This is achieved by using \verb+\chapterstyle+\Arg*{style}. The
most extreme use of this is seen in \emph{The Memoir class For
  Configurable Typesetting -- User Guide} by Peter Wilson, also know
as the \emph{Memoir manual}, \cite{memman}.


In general, \LaTeX\ classes use \verb+\@makechapterhead+ to print a
chapter title specified my \verb|chapter|, and
\verb+\@makeschapterhead+ for \verb+\chapter*+. In memoir Peter Wilson
made these two macros a bit more flexible than usual. The idea is
that for numbered chapters (i.e. \verb+\chapter+ and
$\texttt{secnumdepth}\geq 0$) one should think of the chapter title as
build by:
\begin{Verbatim}
\chapterheadstart
\printchaptername \chapternamenum \printchapternum
\afterchapternum
\printchaptertitle{The title}  
\afterchaptertitle
\end{Verbatim}
For unnumbered (i.e. \verb+\chapter*+ and \verb+\chapter+ width
$\texttt{secnumdepth}<0$): 
\begin{Verbatim}
\chapterheadstart
\printchapternonum
\printchaptertitle{The title}
\afterchaptertitle  
\end{Verbatim}
Note that \verb+\printchaptertitle+ is the only macro that takes an
argument. At the start of every memoir chapter style these macros are
initialised to
\begin{Verbatim}
\renewcommand\chapterheadstart{\vspace*{\beforechapskip}}
\renewcommand\printchaptername{\chapnamefont \@chapapp}
\renewcommand\chapternamenum{\space}
\renewcommand\printchapternum{\chapnumfont \thechapter}
\renewcommand\afterchapternum{\par\nobreak\vskip \midchapskip}
\renewcommand\printchapternonum{}
\renewcommand\printchaptertitle[1]{\chaptitlefont #1}
\renewcommand\afterchaptertitle{\par\nobreak\vskip \afterchapskip}
\end{Verbatim}
So one just have to change the ones one need. There are a few other
macros that are nice to know the meaning of. Remember that these are
\emph{not} reset at the start of a new chapter style.
\begingroup
\renewcommand\descriptionlabel[1]{\hspace\labelsep\cs{#1}}
\begin{description}\firmlist
\item[beforechapskip] length, self explanatory,usually set using
  \verb+\chapterheadstart+, default 50pt
\item[midchapskip] length, distance between the chapter name / number and the
title, usually set using \verb+\afterchapternum+, default 20pt
\item[afterchapskip] length, distance between the chapter title and
  the following text, usually set using \verb+\afterchaptertitle+,
  default 40pt
\item[chapnamefont] the font setting used for \emph{Chapter} or
  similar, default \verb+\normalfont\huge\bfseries+
\item[chapnumfont] same for the chapter number, default
  \verb+\normalfont\huge\bfseries+
\item[chaptitlefont] same for the chapter title, default
  \verb+\normalfont\Huge\bfseries+ 
\end{description}
\endgroup
\noindent One might ask what \verb+\printchapternonum+ is good for
when it is always initialised to nothing. Well if a design need to one
could use it to insert a phantom width as wide as the chapter name
plus number would have been. If on the other hand one is creating a
style where the chapter name and number is actually typeset using
\verb+\printchaptertitle+ (like a framed one) then one could first
define a new if construction, say, \verb+\ifNoChapNum+ and then let
\verb+\printchapternonum+ set this to true and so on.

In memoir a new chapter style is defined as
\begin{syntax}
\cs{makechapterstyle}\Arg*{name}\texttt{\{}  
\Arg{code}
\texttt{\}}
\end{syntax}
Where \Arg{code} is redefinitions of the macros mentioned
above. (Remember that if you redefine \verb+\printchaptertitle+ then
you have to use \texttt{\#\#1} to represent the title.)
Activating a given style is done by simply issuing
\begin{syntax}
  \cs{chapterstyle}\Arg*{name}
\end{syntax}
By the way, if you happen to like a given style but wanted to, say,
add color to the chapter title, you could just refine
\verb+\chaptitlefont+ after you have issued \verb+\chapterstyle+. (Even
simpler to just use \verb+\addtodef\chaptitlefont{}{\color{nicered}}+.)

As a simple example, here is the code for the \texttt{section} chapter
style
\begin{Verbatim}[label={Source code for the \textsf{section} chapter style}]
\makechapterstyle{section}{%
  \renewcommand{\printchaptername}{}
  \renewcommand{\chapternamenum}{}
  \renewcommand{\chapnumfont}{\normalfont\Huge\bfseries}
  \renewcommand{\printchapternum}{\chapnumfont \thechapter\space}
  \renewcommand{\afterchapternum}{}
}
\end{Verbatim}



\clearpage

\chapter{Default styles included in memoir}
\label{cha:defa-styl-incl}

First we have the six default chapterstyles in the memoir class. The
source code for these can be found in \texttt{memoir.cls}.

\begin{showchapterstyle}{default}
\end{showchapterstyle}

\begin{showchapterstyle}{section}
\end{showchapterstyle}

\begin{showchapterstyle}{hangnum}
\end{showchapterstyle}

\begin{showchapterstyle}{companion}
\end{showchapterstyle}

\begin{showchapterstyle}{article}
\end{showchapterstyle}

\begin{showchapterstyle}{demo}
\end{showchapterstyle}

\bigskip

\starbreak

\bigskip

\noindent 
The memoir manual also mentions a style called \textsf{veelo}, by
Bastiaan Veelo

\begin{showchapterstyle}{veelo}
\makeatletter
\usepackage{graphicx}
\newlength{\numberheight}
\newlength{\barlength}
\makechapterstyle{veelo}{%
   \setlength{\beforechapskip}{40pt}
   \setlength{\midchapskip}{25pt}
   \setlength{\afterchapskip}{40pt}
   \renewcommand{\chapnamefont}{\normalfont\LARGE\flushright}
   \renewcommand{\chapnumfont}{\normalfont\HUGE}
   \renewcommand{\chaptitlefont}{\normalfont\HUGE\bfseries\flushright}
   \renewcommand{\printchaptername}{%
     \chapnamefont\MakeUppercase{\@chapapp}}
   \renewcommand{\chapternamenum}{}
   \setlength{\numberheight}{18mm}
   \setlength{\barlength}{\paperwidth}
   \addtolength{\barlength}{-\textwidth}
   \addtolength{\barlength}{-\spinemargin}
   \renewcommand{\printchapternum}{%
     \makebox[0pt][l]{%
       \hspace{.8em}%
       \resizebox{!}{\numberheight}{\chapnumfont \thechapter}%
       \hspace{.8em}%
       \rule{\barlength}{\numberheight}}}
   \makeoddfoot{plain}{}{}{\thepage}
}
\makeatother
\end{showchapterstyle}


\bigskip

\noindent 
Another example from the manual is a style based upon the design used
in Robert Bringhursts, \emph{The Elements of Typographic Style}.
\begin{showchapterstyle}{bringhurst}
  \makechapterstyle{bringhurst}{%
  \renewcommand{\chapterheadstart}{}
  \renewcommand{\printchaptername}{}
  \renewcommand{\chapternamenum}{}
  \renewcommand{\printchapternum}{}
  \renewcommand{\afterchapternum}{}
  \renewcommand{\printchaptertitle}[1]{%
    \raggedright\Large\scshape\MakeLowercase{##1}}
  \renewcommand{\afterchaptertitle}{%
    \vskip\onelineskip \hrule\vskip\onelineskip}
}
\end{showchapterstyle}
Which is a very simple, but nice design. The most radical thing about
the Bringhurst design is actually the quite odd headers, which are
placed in the outer margins a bit down the page. See \cite{memman} for
more information.



\section*{Styles memtioned in the memoir addendum}
\label{sec:styl-memt-memo}

Style to replicate the appearance of \verb+\section+ in the article
class. The style is currently a part of \texttt{mempath.sty}.
\begin{showchapterstyle}{reparticle}  
\end{showchapterstyle}
The difference between this style and the \textsf{section} style is
the way a long title text is handled. 

Style originally by Thomas Dye, inspired by a style used in a book by
Aidan Southall.
\begin{showchapterstyle}{southall}
%% Thomas Dye's southall chapter style
\newlength{\headindent}
\newlength{\rightblock}
\makechapterstyle{southall}{%
  \setlength{\headindent}{36pt}
  \setlength{\rightblock}{\textwidth}
  \addtolength{\rightblock}{-\headindent}
  \setlength{\beforechapskip}{2\baselineskip}
  \setlength{\afterchapskip}{5\baselineskip}
  \setlength{\midchapskip}{0pt}
  \renewcommand{\chaptitlefont}{\huge\rmfamily\raggedright}
  \renewcommand{\chapnumfont}{\chaptitlefont}
  \renewcommand{\printchaptername}{}
  \renewcommand{\chapternamenum}{}
  \renewcommand{\afterchapternum}{}
  \renewcommand{\printchapternum}{%
    \begin{minipage}[t][\baselineskip][b]{\headindent}
      {\vspace{0pt}\chapnumfont%%%\figureversion{lining}
        \thechapter}
    \end{minipage}}
  \renewcommand{\printchaptertitle}[1]{%
    \hfill\begin{minipage}[t]{\rightblock}
      {\vspace{0pt}\chaptitlefont ##1\par}\end{minipage}}
  \renewcommand{\afterchaptertitle}{%
    \par\vspace{\baselineskip}%
    \hrulefill \par\nobreak\noindent \vskip\afterchapskip}
}
\end{showchapterstyle}
Style based on the chapter heads found in Warren Chappell and Robert
Bringhurst: \emph{A Short History of the Printed Word}. Hartley \&
Marks, 1999.
\begin{showchapterstyle}{chappell}
\makechapterstyle{chappell}{
  \setlength\beforechapskip{0pt}
  \renewcommand*\chapnamefont{\large\centering}
  \renewcommand*\chapnumfont{\large}
  \renewcommand*\printchapternonum{%
    \vphantom{\printchaptername}%
    \vphantom{\chapnumfont 1}%
    \afterchapternum
    \vskip -\onelineskip}
  \renewcommand*\chaptitlefont{\Large\itshape}
  \renewcommand*\printchaptertitle[1]{%
    \hrule\vskip\onelineskip\centering\chaptitlefont ##1}
}  
\end{showchapterstyle}
The latest version if \cite{memmanadd} is written with a modified
version of \texttt{demo}, according to Peter Wilson (in
\cite{memmanadd}) this one has a better design as the title appears in
the same place, no matter if we are using \cs{chapter} of
\cs{chapter*}. 
\begin{showchapterstyle}{demo2}
\makeatletter
\makechapterstyle{demo2}{
  \renewcommand*\printchaptername{\centering}
  \renewcommand*\printchapternum{\chapnumfont
    \ifanappendix \thechapter \else \numtoName{\c@chapter}\fi}
  \renewcommand*{\chaptitlefont}{\normalfont\Huge\sffamily}
  \renewcommand*{\printchaptertitle}[1]{%
    \hrule\vskip\onelineskip \raggedleft \chaptitlefont ##1}
  \renewcommand*{\afterchaptertitle}{%
    \vskip\onelineskip \hrule\vskip \afterchapskip}
  \setlength{\beforechapskip}{3\baselineskip}
  \renewcommand*{\printchapternonum}{%
    \vphantom{\chapnumfont One}
    \afterchapternum%
    \vskip\topskip}
  \setlength{\beforechapskip}{2\onelineskip}
}
\makeatother
\end{showchapterstyle}




\chapter{Styles found via Google Groups}

\enlargethispage{\onelineskip}


Style posted on \textsc{ctt} 2006/01/31 by Troels
Pedersen. I changed the color, and also altered the original
\verb+\marginpar+ to a \verb+\sidebar+, since the \verb+\marginpar+ is
a movable float, which might look odd in some cases.
\begin{showchapterstyle}{TroelsPedersen}
\usepackage{graphicx,color}
\definecolor{nicered}{rgb}{.647,.129,.149}
\makeatletter
\newlength{\numberheight}
\makechapterstyle{TroelsPedersen}{%
   \setlength{\beforechapskip}{-20pt}
   \setlength{\midchapskip}{0pt}
   \setlength{\afterchapskip}{10pt}
   \renewcommand{\chapnamefont}{\normalfont\LARGE\itshape}
   \renewcommand{\chapnumfont}{\normalfont\HUGE\itshape\color{nicered}}
   \renewcommand{\chaptitlefont}{\normalfont\huge\itshape\color{nicered}}
   \renewcommand{\afterchapternum}{}
   \renewcommand{\printchaptername}{}
   \setlength{\numberheight}{20mm}
   \renewcommand{\chapternamenum}{}%
   \renewcommand{\printchapternum}{%
     \sidebar{\makebox[0pt][l]{%
         \resizebox{!}{\numberheight}{\chapnumfont\thechapter}}}}%
   \renewcommand\printchaptertitle[1]{\chaptitlefont##1}
}
\makeatother 
\end{showchapterstyle}
Posted in a question on \textsc{ctt} 2006/02/09 by Anders Lyhne. (I
added \verb+\raggedleft+ to the \verb+\chaptitlefont+ and added the
\verb+\printchapternonum+  line.) 
\begin{showchapterstyle}{AndersLyhne}
\usepackage{graphicx}
\makechapterstyle{AndersLyhne}{%
  \newlength{\chapterlineskipx}
  \setlength{\chapterlineskipx}{0.2cm}
  \setlength{\beforechapskip}{1.5cm}
  \setlength{\afterchapskip}{1cm}
  \setlength{\midchapskip}{2cm}
  \renewcommand\chapnamefont{\normalfont\normalsize\scshape\raggedleft}
  \renewcommand\chaptitlefont{\normalfont\normalsize\bfseries\sffamily\raggedleft}
  \renewcommand\chapternamenum{}
  \renewcommand\printchapternum{\makebox[0pt][l]{\hspace{0.2em}%
      \resizebox{!}{2ex}{\chapnamefont\bfseries\sffamily\thechapter}}}
  \renewcommand\afterchapternum{\par\hspace{1.5cm}\hrule\vspace{0.2cm}}
  \renewcommand\printchapternonum{\par}
  \renewcommand\afterchaptertitle{\vskip\chapterlineskipx
    \hrule\vskip\afterchapskip}
} 
\end{showchapterstyle}
%
This styles is a modified verion of a style posted by Peter Wilson in
an answer on \textsc{ctt} on 2005/01/18. I made sure that the chapter
number disappears in the unnumbered version. I out-commented the
\verb+\cleardoublepage+ because of space issues for this document.
\begin{showchapterstyle}{PeterWilson1}
\newif\ifNoCHapNum
\makechapterstyle{PeterWilson1}{%
  \setlength{\beforechapskip}{0pt}
  \renewcommand{\printchaptername}{}
  \renewcommand{\printchapternum}{}
  \setlength{\midchapskip}{0pt}
  \renewcommand{\printchaptertitle}[1]{%
    \hrule \vskip 0.5\onelineskip
    \Huge \hspace{0pt}\hfill\ifNoCHapNum\relax\else\thechapter.\ \fi
    ##1 \hfill\hspace{0pt} 
    \NoCHapNumfalse%
    \vskip 0.5\onelineskip
    \hrule}
  \renewcommand\printchapternonum{\NoCHapNumtrue}
%  \renewcommand{\afterchaptertitle}{\cleardoublepage}
} 
\end{showchapterstyle}
%
Style by Scott Thatcher, posted on \textsc{ctt} 2006/01/18.
\begin{showchapterstyle}{ScottThatcher}
\makechapterstyle{ScottThatcher}{%
  \renewcommand{\chapterheadstart}{}
  \renewcommand{\chaptitlefont}{\large}
  \renewcommand{\chapnumfont}{\scshape\MakeLowercase}
  \renewcommand{\printchaptername}{\centerline{\chapnumfont{Chapter
        \thechapter}}}
  \renewcommand{\chapternamenum}{}
  \renewcommand{\printchapternum}{}
  \renewcommand{\afterchapternum}{%
    \par\centerline{\parbox{0.5in}{\hrulefill}}\par}
  \renewcommand{\printchaptertitle}[1]{%
    \centerline{\chaptitlefont\MakeUppercase{##1}}}
}
\end{showchapterstyle}
%
By Alexander Grebenkov 2004/11/25, found via Google Groups on fido.ru.tex.
\begin{showchapterstyle}{AlexanderGrebenkov}
   \makechapterstyle{AlexanderGrebenkov}{%
  \renewcommand{\chapterheadstart}{\vspace*{\beforechapskip}\hrule\medskip}
  \renewcommand{\chapnamefont}{\normalfont\large\scshape}
  \renewcommand{\chapnumfont}{\normalfont\large\scshape}
  \renewcommand{\chaptitlefont}{\normalfont\large\scshape}
  \renewcommand{\printchaptername}{\S}
  \renewcommand{\chapternamenum}{ }
  \renewcommand{\printchapternum}{\chapnumfont \thechapter}
  \renewcommand{\afterchapternum}{. }
  \renewcommand{\afterchaptertitle}{\par\nobreak\medskip\hrule\vskip
\afterchapskip}
} 
\end{showchapterstyle}

\clearpage

\chapter{Contributed styles}
\label{cha:contributed-styles-1}

First here are are few of my own.

\begin{showchapterstyle}{daleif1}
\usepackage{color,calc,graphicx,soul,fourier}
\definecolor{nicered}{rgb}{.647,.129,.149}
\makeatletter
\newlength\dlf@normtxtw
\setlength\dlf@normtxtw{\textwidth}
\def\myhelvetfont{\def\sfdefault{mdput}}
\newsavebox{\feline@chapter}
\newcommand\feline@chapter@marker[1][4cm]{%
  \sbox\feline@chapter{%
    \resizebox{!}{#1}{\fboxsep=1pt%
      \colorbox{nicered}{\color{white}\bfseries\sffamily\thechapter}%
    }}%
  \rotatebox{90}{%
    \resizebox{%
      \heightof{\usebox{\feline@chapter}}+\depthof{\usebox{\feline@chapter}}}%
    {!}{\scshape\so\@chapapp}}\quad%
  \raisebox{\depthof{\usebox{\feline@chapter}}}{\usebox{\feline@chapter}}% 
}
\newcommand\feline@chm[1][4cm]{%
  \sbox\feline@chapter{\feline@chapter@marker[#1]}%
  \makebox[0pt][l]{% aka \rlap
    \makebox[1cm][r]{\usebox\feline@chapter}%
  }}
\makechapterstyle{daleif1}{
  \renewcommand\chapnamefont{\normalfont\Large\scshape\raggedleft\so}
  \renewcommand\chaptitlefont{\normalfont\huge\bfseries\scshape\color{nicered}}
  \renewcommand\chapternamenum{}
  \renewcommand\printchaptername{}
  \renewcommand\printchapternum{\null\hfill\feline@chm[2.5cm]\par}
  \renewcommand\afterchapternum{\par\vskip\midchapskip}
  \renewcommand\printchaptertitle[1]{\chaptitlefont\raggedleft ##1\par}
}
\makeatother
\end{showchapterstyle}
This next style was made one late night for a talk about memoir in the
Danish \TeX\ Users Group. It has later been used in a few master
theses. 
\begin{showchapterstyle}{daleif2}
\usepackage{graphicx}
\makechapterstyle{daleif2}{
  \renewcommand\chapnamefont{\normalfont\Large\scshape\raggedleft}
  \renewcommand\chaptitlefont{\normalfont\Huge\bfseries\sffamily\raggedleft}
  \renewcommand\chapternamenum{}
  \renewcommand\printchapternum{%
    \makebox[0pt][l]{\hspace{0.4em}%
      \resizebox{!}{4ex}{\chapnamefont\bfseries\sffamily\thechapter}}}
  \renewcommand\afterchapternum{\par\hspace{1.5cm}\hrule\vskip\midchapskip}
}
\end{showchapterstyle}
%
Style build upon \texttt{VZ15b}, see later.
\begin{showchapterstyle}{{daleif3}}
\usepackage{fourier}
\makeatletter
\newif\iffelinenonum
\newcommand\MyNumToName[1]{%
  \ifcase#1\relax % case 0
  \or First\or Second\or Third%
  \else Not implemented\fi}
\makechapterstyle{daleif3}{
  \renewcommand\chapternamenum{}
  \renewcommand\printchaptername{}
  \renewcommand\chapnamefont{\small\itshape\centering} 
  \setlength\midchapskip{7pt}
  \renewcommand\printchapternum{%
    \par\chapnamefont\decofourleft\enspace%
    \ifanappendix
    \appendixname\space\thechapter%
    \else%
    \MyNumToName{\thechapter}\space\chaptername%
    \fi%
    \/\enspace\decofourright}
  \renewcommand\printchapternonum{\par\felinenonumtrue}
  \renewcommand\chaptitlefont{\huge\itshape\centering}
  \renewcommand\afterchapternum{%
    \par\nobreak\vskip-5pt%
  }
  \renewcommand\afterchaptertitle{%
    \par\vskip-2\midchapskip%
    \rule\textwidth\normalrulethickness
    \felinenonumfalse
    \nobreak\vskip\afterchapskip%
  }
}
\makeatother
\end{showchapterstyle}




Danie Els contributed the following style along with the BlueBox style
on page \pageref{BlueBox}.
\begin{showchapterstyle}{GreyNum}
\usepackage{fix-cm}
\usepackage{fourier}%................... Roman+math - Utopia
\usepackage[scaled=.92]{helvet}%........ Sans serif - Helvetica
\usepackage[T1]{fontenc}
\usepackage{color}
\definecolor{ChapGrey}{rgb}{0.6,0.6,0.6}
\newcommand{\LargeFont}{% Needs a 'stretchable' font
      \usefont{\encodingdefault}{\rmdefault}{b}{n}%
      \fontsize{60}{80}\selectfont\color{ChapGrey}}
\makeatletter
\makechapterstyle{GreyNum}{%
  \renewcommand{\chapnamefont}{\large\sffamily\bfseries\itshape}
  \renewcommand{\chapnumfont}{\LargeFont}
  \renewcommand{\chaptitlefont}{\Huge\sffamily\bfseries\itshape}
  \setlength{\beforechapskip}{0pt}
  \setlength{\midchapskip}{40pt}
  \setlength{\afterchapskip}{60pt}
  \renewcommand\chapterheadstart{\vspace*{\beforechapskip}}
  \renewcommand\printchaptername{%
    \begin{tabular}{@{}c@{}}
      \chapnamefont \@chapapp\\}
    \renewcommand\chapternamenum{\noalign{\vskip 2ex}}
    \renewcommand\printchapternum{\chapnumfont\thechapter\par}
    \renewcommand\afterchapternum{%
    \end{tabular}
    \par\nobreak\vskip\midchapskip}
  \renewcommand\printchapternonum{}
  \renewcommand\printchaptertitle[1]{%
    {\chaptitlefont{##1}\par}}
  \renewcommand\afterchaptertitle{\par\nobreak\vskip \afterchapskip}
}
\makeatother
\end{showchapterstyle}
Danie notes:
\begin{adjustwidth}{1em}{0pt}
  \itshape
  This looks a lot better with real italics sans-serif
  fonts such as Lucida Sans\\
  \verb|\usepackage[expert,vargreek]{lucidabr}%.. Lucida Bright + Expert (commercial)|
  \\
  or Myrad\\
  \verb|\usepackage{charter}%........... Roman      - Charter|\\
  \verb|\renewcommand{\sfdefault}{fmy}%. Sans serif - Myrad (Springer bundle)|
\end{adjustwidth}

\starbreak


This next style is inspired by a mail I recieved from Erik
Quaeghebeur. It took me a little while to actually get this working as
I wanted it to, partly because apparently there is a small issue
regarding \cs{thispagestyle} and \cs{pagestyle} as to which
\cs{chaptermark} gets used (I got around this by using the
\texttt{afterpage} package). This style is designed to be used with
\texttt{openleft} (i.e. chapters starting on even pages). And since
the design uses pagestyles, we need to show several seperate pages.


\begin{showchapterstyle*}{EQ,pages={2,4,6}}
\documentclass[openleft]{memoir}
\usepackage{calc}
\usepackage{afterpage}
\copypagestyle{EQ-pagestyle}{companion}
\setlength{\headwidth}{\textwidth}
\addtolength{\headwidth}{.382\foremargin}
\makerunningwidth{EQ-pagestyle}{\headwidth}
\makeheadposition{EQ-pagestyle}{flushright}{flushleft}{}{}
\makeevenhead{EQ-pagestyle}{\normalfont\bfseries\thepage}{}{\normalfont\bfseries\leftmark}
\makeoddhead{EQ-pagestyle}{\normalfont\bfseries\rightmark}{}{\normalfont\bfseries\thepage}
\newif\ifNoChapNum
\makeatletter
% chapterpage layout
\copypagestyle{EQ-chapterstyle}{EQ-pagestyle}
\makeheadposition{EQ-chapterstyle}{flushright}{flushleft}{}{}
\makeevenhead{EQ-chapterstyle}{%
  \normalfont\bfseries\thepage}{}{%
  \ifnum \c@secnumdepth>\m@ne%
    \ifNoChapNum%
      \raisebox{-4.5pt}[0pt][0pt]{\chapnamefont \rightmark}%
    \else%
      \raisebox{-4.5pt}[0pt][0pt]{\chapnamefont\@chapapp\ \thechapter}%
    \fi%
  \else%
    \raisebox{-4.5pt}[0pt][0pt]{\chapnamefont\rightmark}%
  \fi%
  }
\makeoddhead{EQ-chapterstyle}{\rightmark}{}{\normalfont\bfseries\thepage}
% build in the shorter headline
\@namedef{EQ-chapterstyleheadrule}{%
  \ifnum \c@secnumdepth>\m@ne%
    \ifNoChapNum%
      \settowidth\@tempdimc{\quad\chapnamefont\rightmark}%
    \else%
      \settowidth\@tempdimc{\quad\chapnamefont\@chapapp\ \thechapter}%
    \fi%
  \else%
  \settowidth\@tempdimc{\quad\chapnamefont\rightmark}%
  \fi%
  \setlength\@tempdimc{\headwidth-\@tempdimc}%
  \hrule\@width \@tempdimc\@height \normalrulethickness \vskip-\normalrulethickness%
}
\aliaspagestyle{chapter}{EQ-chapterstyle}
\pagestyle{EQ-pagestyle}
\makechapterstyle{EQ}{
  \renewcommand{\chapnamefont}{\raggedleft\bfseries\huge} 
  \renewcommand{\chapternamenum}{}
  \renewcommand\printchaptername{}
  \renewcommand\printchapternum{}
  \renewcommand\printchaptertitle[1]{%
    \ifnum \c@secnumdepth>\m@ne%
    \ifNoChapNum\else\chaptitlefont ##1\fi%
    \fi%
    \ifNoChapNum%
    \markboth{##1}{##1}%
    \fi%
    \afterpage{\global\NoChapNumfalse}%
  }
  \renewcommand\afterchapternum{}
  \renewcommand\afterchaptertitle{%
    \ifnum \c@secnumdepth>\m@ne%
    \ifNoChapNum\else\par\nobreak\vskip\afterchapskip\fi%
    \fi}
  \setlength\beforechapskip{15pt}
  \renewcommand\printchapternonum{\global\NoChapNumtrue}
  \renewcommand{\chaptitlefont}{\raggedleft\normalfont\Huge\bfseries}
}
\makeatother
\chapterstyle{EQ}
\begin{document}
\frontmatter
\chapter{Preface}

Some text at the beginning of a chapter. And we add a lot of text to
make sure that it spans more than one line.

\mainmatter

\chapter{A chapter title}
Some text at the beginning of a chapter. And we add a lot of text to
make sure that it spans more than one line.

\chapter*{A non-numbered chapter title}
Some text at the beginning of a chapter. And we add a lot of text to
make sure that it spans more than one line.

\end{document}
\end{showchapterstyle*}
Remember that the line you see is actually the header.


\newpage

\chapter{Vincent Zoonekynd}
\label{sec:vincent-zoonekynd}

Some time ago Vincent Zoonekynd published a long list of general
chapter styles for \LaTeX, see
\url{http://zoonek.free.fr/LaTeX/LaTeX_samples_chapter/0.html}. 
In this section we implement several of these styles. Special thanks
to Danie Els for the BlueBox style (aka VZ39).

The styles are named after Vincent Zoonekynd (VZ) and the number on
the mentioned page.
\begin{showchapterstyle}{VZ14}
\makeatletter
\newcommand\thickhrulefill{\leavevmode \leaders \hrule height 1ex \hfill \kern \z@}
\setlength\midchapskip{10pt}
\makechapterstyle{VZ14}{
  \renewcommand\chapternamenum{}
  \renewcommand\printchaptername{}
  \renewcommand\chapnamefont{\Large\scshape}
  \renewcommand\printchapternum{%
    \chapnamefont\null\thickhrulefill\quad
    \@chapapp\space\thechapter\quad\thickhrulefill}
  \renewcommand\printchapternonum{%
    \par\thickhrulefill\par\vskip\midchapskip
    \hrule\vskip\midchapskip
  }
  \renewcommand\chaptitlefont{\Huge\scshape\centering}
  \renewcommand\afterchapternum{%
    \par\nobreak\vskip\midchapskip\hrule\vskip\midchapskip}
  \renewcommand\afterchaptertitle{%
    \par\vskip\midchapskip\hrule\nobreak\vskip\afterchapskip}
}
\makeatother
\end{showchapterstyle}
Variation over VZ15.
\begin{showchapterstyle}{VZ15b}
\usepackage{pifont,graphicx}
\newcommand\mylleaf{\ding{'247}}
\newcommand\myrleaf{\reflectbox{\mylleaf}}
\newcommand\MyNumToName[1]{%
  \ifcase#1\relax % case 0
  \or First\or Second\or Third%
  \else Not implemented\fi}
\makeatletter
\setlength\midchapskip{10pt}
\makechapterstyle{VZ15b}{
  \renewcommand\chapternamenum{}
  \renewcommand\printchaptername{}
  \renewcommand\chapnamefont{\Large\scshape}
  \renewcommand\printchapternum{%
    \chapnamefont\null\hfill\mylleaf\quad
    \MyNumToName{\thechapter}\space\@chapapp\quad\myrleaf\hfill\null}
  \renewcommand\printchapternonum{%
    \par\hrule\vskip\midchapskip}
  \renewcommand\chaptitlefont{\Huge\scshape\centering}
  \renewcommand\afterchapternum{%
    \par\nobreak\vskip\midchapskip\hrule\vskip\midchapskip}
  \renewcommand\afterchaptertitle{%
    \par\vskip\midchapskip\hrule\nobreak\vskip\afterchapskip}
}
\makeatother
\end{showchapterstyle}
Though I believe this style would look better without the lines. 

Variation over VZ21. Note the use of two different tabulars depending
upon the length of the title. Also note that we use the build-in
booktabs rules, and note that the thickness of these rules can be
individually adjusted. 
\begin{showchapterstyle}{VZ21}
\usepackage{calc,fourier}
\usepackage[T1]{fontenc}
\makeatletter
\setlength\midchapskip{7pt}
\makechapterstyle{VZ21}{
  \renewcommand\chapnamefont{\Large\scshape}
  \renewcommand\chapnumfont{\Large\scshape\centering}
  \renewcommand\chaptitlefont{\huge\bfseries\centering}
  \renewcommand\printchaptertitle[1]{%
    \setlength\tabcolsep{7pt}% used as indentation on both sides
    \settowidth\@tempdimc{\chaptitlefont ##1}%
    \setlength\@tempdimc{\textwidth-\@tempdimc-2\tabcolsep}%
    \chaptitlefont
    \ifdim\@tempdimc > 0pt\relax% one line
    \begin{tabular}{c}
      \toprule  ##1\\ \bottomrule
    \end{tabular}
    \else% two+ lines
    \begin{tabular}{%
        >{\chaptitlefont\arraybackslash}p{\textwidth-2\tabcolsep}}
      \toprule ##1\\ \bottomrule
    \end{tabular}
    \fi
  }
}
\makeatother
\end{showchapterstyle}
Next up is VZ23.
\begin{showchapterstyle}{VZ23}
\setlength\midchapskip{10pt}
\makechapterstyle{VZ23}{
  \renewcommand\chapternamenum{}
  \renewcommand\printchaptername{}
  \renewcommand\chapnumfont{\Huge\bfseries\centering}
  \renewcommand\chaptitlefont{\Huge\scshape\centering}
  \renewcommand\afterchapternum{%
    \par\nobreak\vskip\midchapskip\hrule\vskip\midchapskip}
  \renewcommand\printchapternonum{%
    \vphantom{\chapnumfont \thechapter}
    \par\nobreak\vskip\midchapskip\hrule\vskip\midchapskip}
}  
\end{showchapterstyle}
A variation over VZ34 (in the original the first cell in the tabular
adjusts to the width of the chapter number, here it does not).
\begin{showchapterstyle}{VZ34}
\usepackage{calc}
\newif\ifNoChapNumber
\makeatletter
\makechapterstyle{VZ34}{
  \renewcommand\chapternamenum{}
  \renewcommand\printchaptername{}
  \renewcommand\printchapternum{}
  \renewcommand\chapnumfont{\Huge\bfseries}
  \renewcommand\chaptitlefont{\Huge\bfseries\raggedright}
  \renewcommand\printchaptertitle[1]{%
    \begin{tabular}{@{}p{1cm}|!{\quad}p{\textwidth-1cm-2em-4\tabcolsep }}
      \ifNoChapNumber\relax\else\chapnumfont \thechapter\fi
      & \chaptitlefont ##1
    \end{tabular}
    \NoChapNumberfalse
  }
  \renewcommand\printchapternonum{\NoChapNumbertrue}
}
\end{showchapterstyle}
Variation over VZ39, contributed by Danie Els.\label{BlueBox}
\begin{showchapterstyle}{BlueBox}
\usepackage{fourier} % or what ever
\usepackage[scaled=.92]{helvet}%. Sans serif - Helvetica
\usepackage{color,calc}
\newsavebox{\ChpNumBox}
\definecolor{ChapBlue}{rgb}{0.00,0.65,0.65}
\makeatletter
\newcommand*{\thickhrulefill}{%
  \leavevmode\leaders\hrule height 1\p@ \hfill \kern \z@}
\newcommand*\BuildChpNum[2]{%
  \begin{tabular}[t]{@{}c@{}}
    \makebox[0pt][c]{#1\strut}  \\[.5ex]
    \colorbox{ChapBlue}{%
      \rule[-10em]{0pt}{0pt}%
      \rule{1ex}{0pt}\color{black}#2\strut
      \rule{1ex}{0pt}}%
  \end{tabular}}
\makechapterstyle{BlueBox}{%
  \renewcommand{\chapnamefont}{\large\scshape}
  \renewcommand{\chapnumfont}{\Huge\bfseries}
  \renewcommand{\chaptitlefont}{\raggedright\Huge\bfseries}
  \setlength{\beforechapskip}{20pt}
  \setlength{\midchapskip}{26pt}
  \setlength{\afterchapskip}{40pt}
  \renewcommand{\printchaptername}{}
  \renewcommand{\chapternamenum}{}
  \renewcommand{\printchapternum}{%
    \sbox{\ChpNumBox}{%
      \BuildChpNum{\chapnamefont\@chapapp}%
      {\chapnumfont\thechapter}}}
  \renewcommand{\printchapternonum}{%
    \sbox{\ChpNumBox}{%
      \BuildChpNum{\chapnamefont\vphantom{\@chapapp}}%
      {\chapnumfont\hphantom{\thechapter}}}}
  \renewcommand{\afterchapternum}{}
  \renewcommand{\printchaptertitle}[1]{%
    \usebox{\ChpNumBox}\hfill
    \parbox[t]{\hsize-\wd\ChpNumBox-1em}{%
      \vspace{\midchapskip}%
      \thickhrulefill\par
      \chaptitlefont ##1\par}}%
}
\end{showchapterstyle}
Style inspired by VZ43
\begin{showchapterstyle}{VZ43}
\usepackage{calc,color}
\newif\ifNoChapNumber
\newcommand\Vlines{%
  \def\VL{\rule[-2cm]{1pt}{5cm}\hspace{1mm}\relax}
  \VL\VL\VL\VL\VL\VL\VL}
\makeatletter
\setlength\midchapskip{0pt}
\makechapterstyle{VZ43}{
  \renewcommand\chapternamenum{}
  \renewcommand\printchaptername{}
  \renewcommand\printchapternum{}
  \renewcommand\chapnumfont{\Huge\bfseries\centering}
  \renewcommand\chaptitlefont{\Huge\bfseries\raggedright}
  \renewcommand\printchaptertitle[1]{%
    \Vlines\hspace*{-2em}%
    \begin{tabular}{@{}p{1cm} p{\textwidth-3cm}}%
      \ifNoChapNumber\relax\else%
      \colorbox{black}{\color{white}%
        \makebox[.8cm]{\chapnumfont\strut \thechapter}}
      \fi
      & \chaptitlefont ##1
    \end{tabular}
    \NoChapNumberfalse
  }
  \renewcommand\printchapternonum{\NoChapNumbertrue}
}
\makeatother
\end{showchapterstyle}

\begin{thebibliography}{9}
\bibitem{memman} Peter Wilson, \emph{The Memoir Class for Configurable
  Typesetting -- User Guide}, 2005.
\bibitem{memmanadd} Peter Wilson, \emph{ADDENDUM -- The Memoir Class for Configurable
  Typesetting -- User Guide}, 2006. Latest version to be released
  in August 2006.
\bibitem{VZ} Vincent Zoonekynd. On-line list of different chapter
  styles for \LaTeX. Available at
  \url{http://zoonek.free.fr/LaTeX/LaTeX_samples_chapter/0.html}. 
\end{thebibliography}

\end{document}

%%% Local Variables: 
%%% mode: latex
%%% TeX-master: t
%%% End: 
