\NeedsTeXFormat{LaTeX2e}[2005/12/01]
%%    2010/01/01 v1.7 IMTEK-Diplomarbeitsvorlage
%% Template fuer Diplom-, Bachelor- und Masterarbeiten
%% am IMTEK (c) Simon Dreher
%% Verbesserungsvorschlaege bitte an dreher@imtek.de

%%%%%%%%%%%%%%%%%%%%%%%%%%%%%%%%%%%%%%%%%%%%%%%%%%%%%%%%%%
%%%%%%%%%% Bitte vor dem Veraendern umbenennen! %%%%%%%%%%
%%%%%%%%%%%%%%%%%%%%%%%%%%%%%%%%%%%%%%%%%%%%%%%%%%%%%%%%%%

%% Moegliche Optionen: diejenigen der Klasse scrbook ausser titlepage

%% deutsche DA:
\documentclass[diplom,         %% Typ der Arbeit: diplom, bachelor oder master
               12pt,           %% Schriftgroesse
               twoside,        %% zweiseitiges Layout
               BCOR10mm,       %% Bindekorrektur 10 mm
%               liststotoc,nomtotoc,bibtotoc, %% Aufnahme der div. Verzeichnisse
                                              %% ins Inhaltsverzeichnis
%               pointlessnumbers, %% Ueberschriftnummer. ohne angehaengtem Punkt
               english,ngerman, %% Alternativspr. Englisch, Dokumentspr. Deutsch
%               final,          %% Endversion; draft fuer schnelles Kompilieren
               ]{IMTEKda}
%% Englisch mit dt. Vorspann:
% \documentclass[diplom,12pt,twoside,BCOR10mm,pointlessnumbers,ngerman,english,noenglishpreamble]{IMTEKda}

%% Labels anzeigen zur Korrektur
%\usepackage{showkeys} %% Labels verschwinden mit der Klassenoption final

\usepackage{babel}     %% Sprachen-Unterstuetzung
\usepackage{calc}      %% ermoeglicht Rechnen mit Laengen und Zaehlern
\usepackage[T1]{fontenc}
\usepackage[latin1]{inputenc}
%% in aktuellem Linux & MacOS X wird standardmaessig UTF8 kodiert!
% \usepackage[utf8]{inputenc}

\usepackage{amsmath,amssymb} %% zusaetzliche Mathe-Symbole

\usepackage{lmodern} %% type1-taugliche CM-Schrift als Variante zur
                     %% "normalen" EC-Schrift
%% Variante: Schriftumschaltung auf URW Garamond und Bitstream Vera
%\usepackage[garamond,sfscaled=false,ttscaled=false]{mathdesign}
%\usepackage[scaled=0.9]{berasans,beramono}

%% Paket fuer bibtex-Datenbanken
\usepackage[comma,numbers,sort&compress]{natbib}
%\usepackage{babelbib}         %% korrekte Sprache in Bibliographieeintraegen
\bibliographystyle{plainnat}  %% Formatierung Bibliographie ohne babelbib
%\bibliographystyle{babplain}  %% Formatierung Bibliographie mit babelbib

\newcommand{\tabheadfont}[1]{\textbf{#1}} %% Tabellenkopf in Fett
\usepackage{booktabs}  %% Befehle fuer besseres Tabellenlayout
\usepackage{longtable} %% umbrechbare Tabellen
%\usepackage{array}     %% zusaetzliche Spaltenoptionen

%% umfangreiche Pakete fuer Symbole wie \micro, \ohm, \degree, \celsius etc.
\usepackage{textcomp,gensymb}

%\usepackage{SIunits} %% Korrektes Setzen von Einheiten
\usepackage{units}   %% Variante fuer Einheiten

%\usepackage{icomma}  %% Abstandskorrektur fuer , als Dezimaltrenner

%% Hyperlinks im Dokument; muss als eines der letzten Pakete geladen werden
\usepackage[pdfstartview=FitH,      % Oeffnen mit fit width
            breaklinks=true,        % Umbrueche in Links, nur bei pdflatex default
            bookmarksopen=true,     % aufgeklappte Bookmarks
            bookmarksnumbered=true, % Kapitelnummerierung in bookmarks
            pdfprintscaling=None,   % Default-Einstellung zum Drucken: nicht skaliert
            pdfduplex=DuplexFlipLongEdge, % Default-Druck-Einstellung: Duplex
            ]{hyperref}

%% um keine SANSSERIF Schriften fuer Ueberschriften zu verwenden:
%\setkomafont{sectioning}{\normalfont\normalcolor\bfseries}
%% fuer kleinere Bild- und Tabellenunterschriften:
%\addtokomafont{caption}{\footnotesize}

%% um abgekuerzte Abbildungs- und Tabellenbezeichnung mit \autoref zu erhalten:
%\addto{\extrasngerman}{\renewcommand*{\figureautorefname}{Abb.}}
%\addto{\extrasngerman}{\renewcommand*{\tableautorefname}{Tab.}}

\begin{document}

\author{Max Mustermann}
\title{Diplomarbeitsthema -- hier steht das Thema der Diplomarbeit}
\hypersetup{pdfkeywords={IMTEK, Diplomarbeit, diploma thesis}}

%% Einfuegen eines Titelbilds (optional)
\titlepic{\includegraphics[height=10cm]{figures/bild}}
\titlepicdesc{Dieses Bild zeigt einen xyz-Sensor beim Messen der ABC-Kon"-zentration.}

%% Die jeweils auskommentierte Variante ist bei englischer Praeambel zu verwenden
\dpoversion{20.\,7.~2001}
%\dpoversion{July 20, 2001}
%\dpoversion{28.\,9.~2000}       %% DPO 2000
%\dpoversion{September 28, 2000} %% DPO 2000
\chair{Lehrstuhl f"ur \dots}
%\chair{Micro-optics}
\referees{Prof.\ Dr.\ \dots, Lehrstuhl f"ur \dots\\
Prof. Dr. \dots, Lehrstuhl f"ur \dots, Universit"at \dots (wenn nicht FR)}
%\referees{Prof.\ \dots of \dots}
\supervisor{Prof.\ Dr.\ \dots, Lehrstuhl f"ur \dots\\
Prof.\ Dr.\ \dots, Lehrstuhl f"ur \dots, Universit"at \dots (wenn nicht FR)}
\thesistime{1.\ Januar 2006 bis 31.\ Mai 2006}
%\thesistime{January~1, 2006 to May~31, 2006}

\frontmatter
\maketitle
\cleardoublepage\phantomsection\pdfbookmark{\abstractname}{abstract} %% fuegt ersten Abstract in die Bookmarks ein
%\begin{otherlanguage}{ngerman}
\begin{abstract}
  Hier werden auf einer halben Seite die Kernaussagen der Diplomarbeit
  auf Deutsch zusammengefasst.
  \bigskip\par
  \textbf{Stichw"orter:} IMTEK, Diplomarbeit
\end{abstract}
%\end{otherlanguage}
\begin{otherlanguage}{english}
\begin{abstract}
  Hier werden auf einer halben Seite die Kernaussagen der Diplomarbeit
  auf Englisch zusammengefasst.
  \bigskip\par
  \textbf{Keywords:} IMTEK, Diploma thesis
\end{abstract}
\end{otherlanguage}

%% fuegt Inhaltsverzeichnis in die Bookmarks ein
\cleardoublepage\phantomsection\pdfbookmark{\contentsname}{toc}
%% setzt Inhaltsverzeichnis
\tableofcontents

\begin{nomenclature}
%% Fuer die Berechnung der Spaltenbreiten muss \usepackage{calc}
%% geladen sein!
\section*{Lateinische Buchstaben}
\noindent
\begin{longtable}[l]{p{0.2\textwidth}p{0.7\textwidth-6\tabcolsep}p{0.1\textwidth}}
  \tabheadfont{Variable}&\tabheadfont{Bedeutung}&\tabheadfont{Einheit}\\\midrule\endhead
  $A$ & Querschnittsfl"ache & $\unit{m^2}$\\
  $c$ & Geschwindigkeit & $\unitfrac{m}{s}$
\end{longtable}
\section*{Griechische Buchstaben}
\begin{longtable}[l]{p{0.2\textwidth}p{0.7\textwidth-6\tabcolsep}p{0.1\textwidth}}
  \tabheadfont{Variable}&\tabheadfont{Bedeutung}&\tabheadfont{Einheit}\\\midrule\endhead
  $\alpha$  & Winkel & $\unit{\degree}$; --\\
  $\varrho$ & Dichte & $\unitfrac{kg}{m^3}$
\end{longtable}
\section*{Indizes}
\begin{longtable}[l]{p{0.2\textwidth}p{0.8\textwidth-4\tabcolsep}}
  \tabheadfont{Index}&\tabheadfont{Bedeutung}\\\midrule\endhead
  m & Meridian\\
  $r$ & Radial
\end{longtable}
\section*{Abk"urzungen}
\begin{longtable}[l]{p{0.2\textwidth}p{0.8\textwidth-4\tabcolsep}}
  \tabheadfont{Abk"urzung}&\tabheadfont{Bedeutung}\\\midrule\endhead
  2D & zweidimensional\\
  3D & dreidimensional\\
  max & maximal
\end{longtable}
\end{nomenclature}

%% die Klassenoption liststotoc uebernimmt das Abbildungs- und Tabellen-
%% verzeichnis in den TOC
%% \listoftables und \listoffigures sollten nur bei genuegender Anzahl Tabellen
%% verwendet werden
\listoffigures
\listoftables

\mainmatter   %% Anfang Hauptteil

\chapter{Einleitung}
Diese Diplomarbeitsvorlage ist eine Neu"uberarbeitung der Vorlage von Jan
Lienemann durch Simon Dreher. Wichtige Hinweise finden sich in der beigef"ugten
Dokumentation \verb|IMTEKda.pdf|, bitte diese sorgf"altig lesen!

\chapter{Grundlagen}
In diesem Kapitel werden die theoretischen Grundlagen erl"autert.

Wichtige Gleichungen, die in der Arbeit h"aufiger zitiert werden,
sollten eine Gleichungsnummer erhalten.
\begin{equation}
  \label{eq:pythagoras}
  a^2+b^2=c^2
\end{equation}
Zum Beispiel wird in Gleichung~\ref{eq:pythagoras} der Satz des Pythagoras
angegeben.

Gerade im Bereich der Grundlagen wird viel Literatur zitiert, z.\,B.\
\cite{Menz97}. Falls
mehrere Literaturzitate auf einmal zitiert werden, ist folgendes
z.\,B.\ m"oglich \cite{Horn90,DINEN6232,Menz97,Knuth84}.

\section{Unterkapitel Gliederungsebene 2}
Hier sollte etwas Text stehen.
\subsection{Unterkapitel Gliederungsebene 3}
Noch ein paar Beispiele zu Abbildungen und Tabellen:

Abbildung~\ref{fig:bildplatzhalter} verdeutlicht \dots

Wie die Abb.~\ref{fig:bildplatzhalter} und
Tab.~\ref{tab:tabellenplatzhalter} verdeutlichen \dots

\begin{figure}
  \centering
  \includegraphics[width=0.5\linewidth]{figures/bild}
  \caption{Bildbeschreibung}
  \label{fig:bildplatzhalter}
\end{figure}

Text\dots
\begin{table}
  \centering
  \begin{tabular}{llll}
    \toprule
    $A$-Wert&$B$-Wert&$C$-Wert&$D$-Wert\\
    \midrule
    aaaaaa&bbbbbbb&cccccc&ddddddd\\
    aaaaaa&bbbbbbb&cccccc&ddddddd\\
    \bottomrule
  \end{tabular}
  \caption{Tabellenbeschreibung}
  \label{tab:tabellenplatzhalter}
\end{table}

Text\dots

\chapter{Experimentelle Vorgehensweise}
Text\dots
\chapter{Ergebnisse}
Text\dots
\chapter{Diskussion}
Text\dots
\section{Unterkapitel}
Text\dots
\subsection{Unterkapitel}
Text\dots
\chapter{Zusammenfassung}
Text\dots

\appendix
\chapter{erster Anhang}
Text\dots
\chapter{zweiter Anhang}
Text\dots

%% fuegt Literaturverzeichnis in die Bookmarks ein
\cleardoublepage\phantomsection\pdfbookmark{\bibname}{bib}
\bibliography{diplarb} %% Bibliographie; unbedingt umbenennen!

\chapter*{Danksagung}
Dank\dots

\end{document}
