% makeindex < acroflex_man.idx > acroflex_man.ind
\documentclass{article}
\usepackage[fleqn]{amsmath}
\usepackage[
    web={centertitlepage,designv,
         nodirectory,forcolorpaper,
         usesf,latextoc,pro},
    eforms,
    linktoattachments,
    aebxmp
]{aeb_pro}
\usepackage[dvipsone,showembeds]{graphicxsp}
\usepackage{array}
\usepackage{myriadpro}

\usepackage{makeidx}
\makeindex
\usepackage{acroman}

\usepackage[active]{srcltx}

\urlstyle{rm}

\def\expath{../examples}

\DeclareDocInfo
{
    university={\AcroTeX.Net},
    title={\texorpdfstring{\AcroFLeX\\[1em]
    The {\AcroTeX} and FLEX/Flash Connection\\[1em]Applications to Graphing}{The AcroFLeX
    Graphing System}},
    author={D. P. Story},
    email={dpstory@acrotex.net},
    subject={Documentation for AcroFLeX},
    talksite={\url{www.acrotex.net}},
    version={0.5},
    keywords={SWF, Adobe FLEX, Adobe Acrobat, JavaScript, ActionScript, AcroTeX},
    copyrightStatus=True,
    copyrightNotice={Copyright (C) \the\year, D. P. Story},
    copyrightInfoURL={http://www.acrotex.net}
}

\def\dps{$\hbox{$\mathfrak D$\kern-.3em\hbox{$\mathfrak P$}%
   \kern-.6em \hbox{$\mathcal S$}}$}

\setWindowOptions{showtitle}

\universityLayout{fontsize=Large}
\titleLayout{fontsize=LARGE}
\authorLayout{fontsize=Large}
\tocLayout{fontsize=Large,color=aeb}
\sectionLayout{indent=-62.5pt,fontsize=large,color=aeb}
\subsectionLayout{indent=-31.25pt,color=aeb}
\subsubsectionLayout{indent=0pt,color=aeb}
\subsubDefaultDing{\texorpdfstring{$\bullet$}{\textrm\textbullet}}

%\pagestyle{empty}
\parindent0pt\parskip\medskipamount

\begin{docassembly}
\addWatermarkFromFile({
    bOnTop:false,
    cDIPath:"/C/AcroTeX/AcroTeX/ManualBGs/Manual_BG_Print_AeB.pdf"
});
\executeSave();
\end{docassembly}

\begin{document}

\maketitle

\selectColors{linkColor=black}
\tableofcontents
\selectColors{linkColor=webgreen}

\section{Introduction}

The \textbf{{\AcroFLeX}} Graphing Bundle is used to create a
\emph{graphing screen} that can be incorporated into a PDF document
and viewed within Adobe Reader, version 9.0 or later. The graphing
screen can be interactive or non-interactive.

For the interactive graphing screen, the user can enter an
expression representing a function of a single variable $x$, a polar
function of $t$, or a set of parametric equations that are functions
of $t$. Various controls are provided to change the viewing window,
for shifting horizontally and vertically, and for zooming in or out.

For the non-interactive graphing screen, the screen is populated
when the user clicks a link created by \cs{sgraphLink}.
\cs{sgraphLink} passes such information as the function,
domain, and range to the graphing routines of {\AcroFLeX}.

In this version of {\AcroFLeX}, up to four functions can be graphed
and four sets of plotted points can be displayed, on one graphing
screen.

The graph screen itself is actually a SWF file, named
\texttt{acroflex.swf}. This SWF file is part of the {\AcroFLeX}
distribution. This package uses the \texttt{rmannot} package, also
written by this author, to create rich media annotations, to embed
\texttt{acroflex.swf} in the PDF document, and to display the SWF through
the rich media annotation.

\subsection{Background}

Version 9 of Acrobat/Adobe Reader introduces the \textit{rich media
annotation} which plays FLV movies, and SWF animations, and MP3 files.\footnote{%
The rich media annotation is introduced in
\textsl{Adobe Supplement to the ISO 32000}, which documents
BaseLevel 1.7, ExtensionLevel 3, the Adobe's extensions to PDF~1.7.}
Acrobat/Adobe Reader also provides a scripting bridge between
JavaScript for Acrobat, and ActionScript, the scripting language of
Flash player. This bridge enables the PDF and the Flash widget,
embedded in the rich media annotation, to communicate. The scripting
bridge opens up wonderful opportunities for applications to the
education sector. The {\AcroFLeX} Graphing Bundle is one such
application of the new PDF-Flash connection to education.

\textbf{{\AcroFLeX}} uses the commercial product Adobe FLEX
Builder~3 and FLEX~3 SDK to produce Flash widgets, and the AeB to
create PDF documents with appropriate JavaScript to communicate with
the Flash widget. FLEX Builder~3 is currently free for students and
educators, the FLEX 3 SDK is free to all.

% This style file defines some basic controls for
% \textbf{{\AcroFLeX} Graphing} for controlling graphical SWF files
% for graphing.
%
% These commands are general {\LaTeX} commands, that require no special driver, so
% they can be used by, for example, pdftex; however, we use SWF files to produce
% the graphing screen with the built-in ActionScript to communicate with the PDF.
% For this, you will need Acrobat 9 Pro. In that case, you might as well use the
% Acrobat Distiller to produce your PDF, but it is not necessary.

\subsection{What is \texorpdfstring{{\AcroFLeX}}{AcroFLeX}?}

The word \textbf{\AcroFLeX} is meant to convey a merging of two computer technologies:
\begin{itemize}
\item \textbf{Acro}: connotes both \textbf{Adobe Acrobat} (\textbf{Adobe Reader})
and \textbf{{Acro\negthinspace\TeX}} (as in the
\textbf{{Acro\negthinspace\TeX} eDucation Bundle} or, just \textbf{AeB}).
\item \textbf{F\kern-.1667em\lower.5ex\hbox{L}\kern-.3eme\kern-.125emX\@}: connotes
   \textbf{Adobe FLEX 3}. FLEX 3 is used to create SWF files to interact with the user.
   In the case of graphing, plotting information is passed from Acrobat, via JavaScript,
   to the Flash widget. ActionScript takes the data, and plots the points provided, and
   connects them with a smooth curve.
\end{itemize}


\section{Requirements}

In this section we list the requirements for this package.

\subsection{{\LaTeX} Package Requirements}

The preamble of the demo file afgraph.tex lists:
\begin{verbatim}
    \usepackage[%
        driver=dvipsone,
        web={nodirectory,pro,tight,usesf},
        eforms,exerquiz,dljslib={ImplMulti},
        graphicxsp={showembeds}
    ]{aeb_pro}
    \usepackage{acroflex}
\end{verbatim}
Let me comment on each of these lines.
\begin{itemize}
  \item \verb!\usepackage{acroflex}!: Of course, we use the
  \texttt{acroflex} package. The \texttt{acroflex} package
  \textbf{requires} the \texttt{rmannot} package. The latter package
  is the one that creates the rich media annotation, embeds the
  graphing Flash widget, acroflex.swf, and displays it. The
  \texttt{rmannot} package is the only one listed in
  \texttt{acroflex.dtx} as required, however, more packages are
  really required, as discussed in the next items.

 \item[] \texttt{graphicxsp}: A \textbf{required} package.
  The \texttt{rmannot} package requires \texttt{graphicxsp}. The
  \texttt{graphicxsp} package, part of the AeB Pro Bundle,
  provides embedding of poster graphics for a rich media
  annotation. A poster graphic is the appearance you see when
  the annotation is not activated.

  \item \verb!\usepackage{aeb_pro}!: A \textbf{required} package.
  The {\AcroFLeX} Graphing Bundle uses the \texttt{willClose}
  environment to assure the document will behave properly
  when the user closes the document.

  Its options, the so-called \textbf{AeB Control Central},
  represent a convenient way to input the other required
  packages (in optimal order) needed by the \texttt{acroflex} package.

  \item[] If \texttt{aeb\_pro} is not used, then the individual
  required packages must be input using the \cs{usepackage}
  mechanism.

  \item[] We input the \texttt{aeb\_pro} package before the
  \texttt{acroflex} package.

  \item[] We now comment on each of the options used in the
  \texttt{aeb\_pro} package:
  \begin{itemize}
    \item \texttt{driver}: This system uses Acrobat Distiller, which
    distills a PostScript file. The driver values this package uses
    are \texttt{dvips} and \texttt{dvipsone}. Setting the driver is
    important because the dvi-to-ps application (\texttt{dvips} and
    \texttt{dvipsone}) consumes the dvi file produced by the {\TeX}
    compiler and writes a PostScript file that Acrobat Distiller
    consumes.

    \item \texttt{web}: The \texttt{web} package is not really
    required. It is used to create a PDF page size convenient to
    view on a computer monitor. This package has many options and
    features for the document author to design a document for
    screen---or for paper---viewing.

    \item[]The web package brings in the \texttt{hyperref} package,
    which is a \textbf{required package}. If \texttt{web} is not
    used, hyperref needs to be input.

    \item \texttt{eforms}: A \textbf{required} package. This package
    provides form field and link support for the {\AcroFLeX}
    Graphing Bundle. The \texttt{eforms}, in turn, inputs the
    \texttt{insdljs} package, which provides support for document-level
    JavaScript. It is the document-level JavaScript where much of the work
    is done: parsing input, calculating graphing data, and sending
    this data off to the \texttt{acroflex.swf} widget for display.

    \item \texttt{exerquiz}: A \textbf{required} package. The
    \texttt{exerquiz} package has several function parsing methods defined
    in its document-level JavaScript. {\AcroFLeX} uses these parsing
    routines.  One of these days, I'll separate out the parsing
    routines from \texttt{exerquiz}, but not now.

    \item \texttt{dljslib}: An optional, but recommended package.
    We use this package for its \texttt{ImplMulti} option.
    This option simplifies the problem of entering
    functional expressions. Without the \texttt{ImplMulti},
    to enter $2x\sin^2(2x)$, the user would have to type
    explicit multiplications, \verb!2*x*(sin(2*x))^2!, with
    the \texttt{ImplMulti} option, the user needs only enter
    \verb!2xsin^2(2x)!.

    \item \texttt{graphicxsp}: The  \texttt{graphicxsp} is a
    \textbf{required} package of \texttt{rmannot}, but we input it
    earlier so we can set its options through the \textbf{AeB
    Control Central} (part of AeB Pro).

  \end{itemize}
\end{itemize}

\subsection{PDF Creator Requirements}

This package requires Acrobat Distiller 9.0 (or later) to convert
PostScript files to PDF. Because this package uses \texttt{rmannot}
to create rich media annotations, there is also a requirement that
the Distiller must be opened using the \texttt{-F} command line
flag. See the documentation of the \texttt{rmannot} package for more
details.

\subsection{Installation}

The installation of the \texttt{acroflex} package is
straightforward. Place \texttt{acroflex.zip} in the search path of
your {\TeX} system and unzip. Unzipping creates a folder named
\texttt{acroflex}. Refresh your filename database, if your system
requires it.

Accompanying the distribution is a file named \texttt{acroflex.cfg}.
Open this file in your favorite text editor and you see the
following lines.
\begin{verbatim}
    %
    % AcroFleX Graphing Bundle Configuration File
    % D. P. Story, dpstory@acrotex.net
    %
    \pathToAcroFlex{C:/acrotex/aebpro/acroflex/swf}
\end{verbatim}
Edit the argument of \cs{pathToAcroFlex} (defined in the
\texttt{acroflex} package) to the path of the folder that contains
the \texttt{acroflex.swf} Flash file. Save and close
\texttt{acroflex.cfg}.

Of course, you need to install the latest versions of AeB (the
{\AcroTeX} eDucation Bundle), AeB Pro, \texttt{graphicxsp}, and
\texttt{rmannot}. Follow the package documentation closely for
installation, some of the packages require that certain JavaScript
file be installed.


\section{The \texorpdfstring{\AcroFLeX}{AcroFLeX} Graphing System}

This package defines several document-level scripts, the two primary
ones are \texttt{Graph\_xy()} and \texttt{Graph\_xyt()}, the others
support these two. \texttt{Graph\_xy()} and \texttt{Graph\_xyt()}
take the data passed to it, parse it, create plot data, and send it
off to the {\AcroFLeX} graphing widget to graph the data by way of
the infamous scripting bridge. Details of these functions can be
found in the documentation in the \texttt{acroflex.dtx} file.

In the {\AcroFLeX} graphing system there are three modes of
operation: interactive, populate, and silent.
\begin{itemize}
\item \textbf{Interactive:} This occurs when the user enters a
function through the UI.

The following controls are \emph{required}:
\begin{verbatim}
    \funcInputField, \graphBtn, \numPoints
    \domMin, \domMax, \rngMin, \rngMax,
\end{verbatim}
If parametric or polar graphs are to be used,
then \cs{domMinP} and \cs{domMaxP} are also required. The other
controls are \emph{optional}:
\begin{verbatim}
    \graphClrBtn
    \amtShift (\hShiftL, \hShiftR, \vShiftD, \vShiftU)
    \zoomInOut, \savedelSelBtn, \functionSelect
\end{verbatim}
The \cs{graphClrBtn} button is recommended, though not required.
All these commands will be discussed in detail in the pages that follow.

\item \textbf{Populate:} This mode occurs when the graphing parameters
are passed to \texttt{Graph\_xy} (or \texttt{Graph\_xyt}) by
\cs{sgraphLink} (or some other command). All the essential
information is passed as arguments. The target graphing screen
has all the required controls, as listed above. The command
initiating the graphing must set the
\texttt{graph\_props.populate} property to \texttt{true}. In
this case the graphing data populate the required fields and the
graph will be drawn. It is the document author's responsibility
to only use populate on graphing screens that have all the
required control fields.

\item[] Populate behaves exactly like interactive, but the graphing
data is passed to the graphing routines in pre-packaged form,
prepared by the document author; the user, however, can
manipulate the curve once it appears.

\item[] The required controls are the same as the interactive mode.

\item \textbf{Silent:} In the non-interactive mode, there must be no
controls other than \cs{graphClrBtn}.  Basically, the document
author prepares some pre-packaged graphs to be displayed to the
user, without interaction. These may go along with a tutorial
discussion symmetry, periodicity, tangent lines, etc.

\item[] If the document author wants the user to interact with the
graph, the required controls need to be supplied and the
\texttt{graph\_props.populate} property needs to be set to
\texttt{true}. That is, use the populate mode.

\end{itemize}

\subsection{Setting up the Graphing Screen}

It should be a hard and fast rule that all content concerning a
graphing screen should occur on the same page as the rich media
annotation that displays the graphing screen. Should discussion
cross page boundaries, create another graphing screen for that page.
Never fear, the {\AcroFLeX} graphing widget is only embedded once,
so adding more graphing screens does not bloat the size of the file.

There are three commands to set up an {\AcroFLeX} graph screen, these are
\Com{dimScreenGraph}, \Com{graphName} and \Com{graphScreen}. The use of
the command \cs{dimScreenGraph} is not required, but recommended.

\begin{dCmd}{.8\linewidth}
\dimScreenGraph{<width>}{<height>}
\graphName{<unique_name>}
\graphScreen[<rmannot_options>]{<width>}{<height>}
\end{dCmd}

\CmdDescription We describe each of these three, and their parameters.
\begin{itemize}
  \item \cs{dimScreenGraph}: This command is a convenient way of setting the dimensions
  of the graphing screen. You specify the width of the screen using the \texttt{<width>}
  parameter and the height of the screen using the \texttt{<height>} parameter. These values
  are passed through a \cs{setlength}, so simple calculations on the dimension can be performed
  on the parameters. (The \texttt{calc} package is used by the web package.) This command then
  defines macros \Com{hScreenGraph} and \Com{vScreenGraph} to hold these two dimensions, respectively.
  \cs{hScreenGraph} and \cs{vScreenGraph} can be used in \cs{graphScreen}, or in setting up
  \texttt{minipage}s based on these lengths, for example.

  \item[] If the aspect ratio of all your graphing screens is going to be the same, then
  it suffices to use \cs{dimScreenGraph} only once in the document.

  \item \cs{graphName}: Use this command to define a unique name for this graphing screen. Each
  screen must have a different name. This command defines the text macro \Com{afgraphName}, which
  expands to the given name.

  \item \cs{graphScreen}: This is the main command of this package,
  it's the one that creates a rich media annotation and associates
  it with the {\AcroFLeX} Graphing widget. It has three parameters:
  \begin{enumerate}
  \item \texttt{[<rmannot\_options>]} is optional and just passes to the underlying
      command \Com{rmAnnot} (defined in the \texttt{rmannot}
      package) that actually creates the rich media annotation.
      The most ``important'' key-value pair, for this package,
      is the \texttt{poster} key, through this key, a poster can
      be associated with the annotation.
  \item \texttt{<width>} is the width of the graph screen, if \cs{dimScreenGraph} was used,
  just use \cs{hScreenGraph} as this value.
  \item \texttt{<height>} is height the screen of the graph screen, if \cs{dimScreenGraph} was used,
  just use \cs{vScreenGraph} as this value.
  \end{enumerate}
\end{itemize}
The \cs{graphScreen} can be resized using \cs{resizebox} or
\cs{scalebox} (from the \texttt{graphicx} package) to obtain a
larger or smaller graph screen with the same aspect ratio.


The following is an example of the usage of each of these three commands. Note that the
height is three-fourths that of the width.

\begin{dCmd*}{\linewidth}
\dimScreenGraph{186bp}{186bp*3/4}
\graphName{graph1}
\graphScreen[poster=aflogo]{\hScreenGraph}{\vScreenGraph}
\end{dCmd*}

\paragraph{Graphing Screen in a Floating Window.} The graphing
window can appear in a floating window as well. The
\Com{iconFloatGraphScreen} command is used to create such a screen.

\begin{dCmd}{.9\linewidth}
\iconFloatGraphScreen[<key_values>]{<width>}{<height>}
\end{dCmd}
\PD The command has three parameters. The first optional one is
passed as the first optional parameter of the underlying
\cs{graphScreen} command. The \cs{graphScreen} command uses the two
parameters \Com{hScreenGraph} and \Com{vScreenGraph}, defined
through the \cs{dimScreenGraph} command, to set the dimensions of
the graph screen. The graph screen is then resized using
\Com{resizebox} from the \texttt{graphicx} package. The other two
parameters, \texttt{<width>} and \texttt{<height>}, are simply
passed to \cs{resizebox}. See the documentation on \cs{resizebox}
for details on these parameters.

For example,
\begin{dCmd*}{.7\linewidth}
\iconFloatGraphScreen[poster=aflogo]{40bp}{!}
\end{dCmd*}
The first parameter is used to define a poster of the icon, the
second parameter is \texttt{40bp} which means to resize the graphic
to a width of \texttt{40bp}, the third parameter of exclamation
point (!) signals \cs{resizebox} to maintain the aspect ratio of the
graphic.

The \cs{iconFloatGraphScreen} command is implemented by creating a
rich media annotation for the {\AcroFLeX} Graphing widget, with a
form field button on top of it that is transparent. Pressing on the
icon is actually pressing on the button. The button action activates
the graphing screen if it is not activated, and deactivates it if it
is activated. The graphing screen might be the target of graphing
data sent to it by the \cs{sgraphLink} command, see
\hyperref[sgraphlink]{Section~\ref*{sgraphlink}},
page~\pageref*{sgraphlink}, or through the graphing screen controls,
these are explained next.

\subsection{Graphing Screen Controls}

The controls described in this section are used for interactive and
populate modes.

\subsubsection{Required Controls}

For interactive or populate mode, in addition to \cs{graphScreen},
several controls are required so the user can manipulate the graph.

\begin{dCmd}[\sCom{funcInputField}\sCom{fileInputField}]{.75\linewidth}
\funcInputField[<key_values>]{<width>}{<height>}
\end{dCmd}
\CmdDescription The field created by
\cs{funcInputField}\footnote{The command was originally misnamed
\cs{fileInputField} and is still recognized by the {\AcroFLeX}
package; however, document authors should use the command
\cs{funcInputField}.} is used to enter a function or a set of points
to be graphed. The function is parsed by the \texttt{exerquiz}
routines, so the same syntax that is used for \texttt{exerquiz}
quizzes and short quizzes is used. The \texttt{<key\_values>} are
passed to the underlining text field and can be used to change the
appearance of the field, see the \texttt{eformman.pdf} for more
information. The \texttt{<width>} and \texttt{<height>} are the
width and height, respectively, of the text field.

\begin{dCmd}[\sCom{graphBtn}]{.75\linewidth}
\graphBtn[<key_values>]{<width>}{<height>}
\end{dCmd}
\CmdDescription The graph button. Once the user has entered a
required data into the required fields, the user press this button
and the graph appears in the graph screen. The parameters are the
same as for \cs{funcInputField}, the descriptions are the same.

\begin{dCmd}[\sCom{numPoints}]{.75\linewidth}
\numPoints[<key_values>]{<width>}{<height>}
\end{dCmd}
\CmdDescription This text field displays the number of points to be
plotted. It is editable, the user can change this value. The
parameters are the same as for \cs{funcInputField}, the descriptions
are the same.


\begin{dCmd}{.75\linewidth}
\domMin[<key_values>]{<width>}{<height>}
\domMax[<key_values>]{<width>}{<height>}
\rngMin[<key_values>]{<width>}{<height>}
\rngMax[<key_values>]{<width>}{<height>}
\end{dCmd}
\CmdDescription The graphing window is set by these four text
fields. When the curve is graphed, only the rectangular window
$[\,\cs{domMin}, \cs{domMax}\,] \times [\,\cs{rngMin},
\cs{rngMax}\,]$ is displayed.\footnote {By this notation, I mean the
intervals determined by the values of these intervals.} The
parameters are the same as for \cs{funcInputField}, the descriptions
are the same.

If parametric and polar graphing is required of the user, then
\cs{domMinP} and \cs{domMaxP} are required as well.

\begin{dCmd}{.75\linewidth}
\domMinP[<key_values>]{<width>}{<height>}
\domMaxP[<key_values>]{<width>}{<height>}
\end{dCmd}
\CmdDescription The interval $[\,\cs{domMinP}, \cs{domMaxP}\,]$ is
the interval over which a set of parametric equations is traced; in
the case of polar functions, this interval is used for the domain of
the polar function. The parameters are the same as for
\cs{funcInputField}, the descriptions are the same.

\paragraph{Setting the default values.} Whereas it is possible to
set the default values of the fields just described, a more
convenient method is used.

\begin{dCmd}{.8\linewidth}
\defaultFunction{<function|points>}
\defaultNumPoints{<postive_integer>}
\defaultDomRng{<x_min>}{<x_max>}{<y_min>}{<y_max>}
\defaultDomP{<t_min>}{<t_max>}
\end{dCmd}

These can be executed, along with \cs{graphName}, just before the
\cs{graphScreen} command. The values of their parameters will then
populate the corresponding fields as default values.

The following are the default values of all the required fields, as
defined by the \texttt{acroflex} package. Note that all of these are
parsed (with the exception of the number of points) using
\texttt{exerquiz}'s parsing routines; consequently, a value such as
\texttt{2*PI} is perfectly legal.

\begin{dCmd*}{.5\linewidth}
\defaultFunction{x^2}
\defaultNumPoints{40}
\defaultDomRng{-2}{2}{0}{4}
\defaultDomP{0}{2*PI}
\end{dCmd*}

\subsubsection{Optional Controls}

There are several other optional controls that may be useful in manipulating
a graph.

\begin{dCmd}{.75\linewidth}
\graphClrBtn[<key_values>]{<width>}{<height>}
\end{dCmd}
On clicking this button, the current graphing screen is cleared of
all graphs and plotted points. Shift-clicking this button
deactivates the graphing screen, and the annotation's poster
appears.

\paragraph{Multiple Plots.} By using the \Com{functionSelect} combo
box, the user can graph multiple curves.

\begin{dCmd}{.75\linewidth}
\savedelSelBtn[<key_values>]{<width>}{<height>}
\functionSelect[<key_values>]{<width>}{<height>}
\end{dCmd}
The \cs{functionSelect} combo box serves several purposes. It
consists of eight items that appear as \texttt{Curve 1},
\texttt{Curve 2}, \texttt{Curve 3}, \texttt{Curve 4}, \texttt{Plot
1}, \texttt{Plot 2}, \texttt{Plot 3}, \texttt{Plot 4}. When this
combo box is present, the user is able to graph multiple curves and
plots.  Changing the combo box to \texttt{Curve 2}, for example, and
pressing the \cs{graphBtn} button, the function will be graphed on
\texttt{Curve 2}. There are four curves possible, and four sets of
plotted points. The different curves and plots are color coded.

When the \cs{savedelSelBtn} is also present, the user can click on
it and save the function definion under that curve or plot. These
expressions will only be saved during the current viewing session in
Adobe Reader, but if the user is on Acrobat, the PDF can be saved
and the values added to the combo list will be saved as well.

The parameters are the same as for
\cs{funcInputField}, the descriptions are the same.

\paragraph{Horizontal and Vertical Shifting.} There are several
controls that shift the graphing window vertically or horizontally.

\begin{dCmd}{.75\linewidth}
\amtShift[<key_values>]{<width>}{<height>}
\hShiftL{<text>}
\hShiftR{<text>}
\vShiftU{<text>}
\vShiftD{<text>}
\end{dCmd}
The \cs{amtShift} is a text field, its value is a positive number
that will be used to shift the graphing window horizontally or
vertically. The user can change this value. The parameters are the
same as for \cs{funcInputField}, the descriptions are the same.

The other four commands are implemented as links, then clicked, the
graphing window moves the amount specified in \cs{amtShift} field
left (\cs{hShiftL}), right (\cs{hShiftR}), up (\cs{vShiftU}) or down
(\cs{vShiftD}). The argument \texttt{<text>} is the text to be used
to identify the link.

\paragraph{Zoom, zoom, zoom.}

The user can be allowed to optionally zoom the graph out or in by providing the
control \Com{zoomInOut}.

\begin{dCmd}{.75\linewidth}
\zoomInOut[<key_values>]{<width>}{<height>}
\end{dCmd}
Click the \cs{zoomInOut} button zooms out by an amount shown in the
\cs{amtShift} field; shift-clicking will zoom in by the amount shown
in the \cs{amtShift} field. The parameters are the same as for
\cs{funcInputField}, the descriptions are the same.

\paragraph{Setting the default values.} As with the required controls, the optional ones
can be given default values through convenience macros.


\begin{dCmd*}{.75\linewidth}
\defaultShiftAmt{<positive_number>}
\defaultShiftAmt{1}
\end{dCmd*}
The \cs{defaultShiftAmt} is used to set the default value of the \cs{amtShift} field; the default
value is \verb!\defaultShiftAmt{1}!.

The \cs{functionSelect} lists four curves and four plots. The text can be changed by
through the following text macros. Each command is followed by its default definition.
\begin{dCmd*}{.75\linewidth}
\afCurve{<name_for_curve>}
\afCurve{Curve}
\afPoint{<name_for_point>}
\afPoint{Point}
\afUnused{<unused>}
\afUnused{--unused}
\end{dCmd*}
The definitions values of \cs{afCurve} and \cs{afPoint} are the
target of several search using regular expressions. If the values of
\cs{afCurve} and \cs{afPoint} are too complex, the regular
expression search may fail. Try to keep these definitions to ASCII
characters.

\subsection{Populate and Silent Linking}

The previous section details the interactive mode, where the
\cs{graphScreen} is present with all its required controls, and
possibly some optional controls. Curves are generated purely through
the user interface, that is, the user enters data into the various
form fields, clicks the \cs{graphBtn}, and \textsl{voil\`{a}}, the
graph is drawn!

In this section, the populate mode is discussed as well as silent mode.

\subsection{Graphing with \texorpdfstring{\protect\cs{sgraphLink}}{\CMD{sgraphLink}}}\label{sgraphlink}

The document author can prepare function/points to be graphed, along
with all the essential data needed to view the graph. For populate,
the graphing data populate the required text field, and is available
for the user then to manipulate. The population of an interactive
graphing screen is done though a special link, the \cs{sgraphLink}.
(The ``s'' in \cs{sgraphLink} stands for ``silent,'' but that was
before I made the design decision to have a populate mode.)

The syntax for \cs{sgraphLink} is
\begin{dCmd}{\linewidth}
\sgraphLink[<appr>]{<graph_key_vals>}{<func|points>}{<text>}
\end{dCmd}
\PD The command takes four parameters, the first is the usual
optional parameter that can be used to change the appearance of the
link. The others we present in detail.

\begin{enumerate}
  \item \texttt{[<appr>]}: Key-value pairs that are used to change the appearance of the link.
  \item \texttt{<graph\_key\_vals>}: Key-value pairs, some of which are used on the {\LaTeX} side,
    some on the PDF side, while others on SWF side.
    \begin{itemize}

    \item\texttt{graph}: The value of this key determines which
        \emph{chart series} (FLEX terminology) the data will appear
        on. The values of this key are
        \begin{itemize}
          \item \texttt{c1}, \texttt{c2}, \texttt{c3}, and \texttt{c4}:
          Use one of these values to graph a function, a polar function,
          or a set of parametric functions. Up to four curves can be displayed
          on the graphing screen at once. These values are displayed using
          the \texttt{LinearSeries} (FLEX terminology).
          \item \texttt{a1}, \texttt{a2}, \texttt{a3}, and \texttt{a4}:
          Same as above, but the region between the horizontal axis, and
          the graph is shaded in. These values are displayed using
          the \texttt{AreaSeries} (FLEX terminology).
          \item \texttt{p1}, \texttt{p2}, \texttt{p3}, and \texttt{p4}:
          Use one of these values to plot points. These values are displayed using
          the \texttt{PlotSeries} (FLEX terminology).
        \end{itemize}
        Thus, \texttt{graph=c2} tells the graphing
        routines of {\AcroFLeX} and the {\AcroFLeX} Graphing widget
        to display this data on series \texttt{c2}.

    \item[] If two curves or plots have the same value for
        \texttt{graph}, then the the one graphed last will
        overwrite the earlier one. If you want both curves
        or plots to appear on the graph together, give them
        different \texttt{graph} values.

    \item[] When this key is not given a value, the default is
        \texttt{c1}.

    \item\texttt{type}: This key declares the type of curve,
        possible values are \texttt{cart}, \texttt{para}, and
        \texttt{polar}. This key is used mostly internally, and is
        normally not used. There is one situation that it is used. When
        defining a polar function, use \texttt{type=polar}. Thus, to
        define a polar function, type something like this:

\medskip
\begin{Verbatim}[fontsize=\footnotesize]
    \sgraphLink{type=polar,xInterval={[-1.5,1.5]},yInterval={[-1,2]},
            tInterval={[0,2*PI]},points=40,populate}
            {1+sin(t)}{$r = 1 + \sin(\theta)$ }
\end{Verbatim}
\smallskip

    Note the explicit use of \texttt{type=polar}; the
    parsing can identify a function of $x$ and a set of
    parametric equations that are function of $t$, but help
    is needed for polar.

    \item\texttt{populate}: Possible values are \texttt{true} or
        \texttt{false}, typing \texttt{populate} is the same as
        \texttt{populate=true}. This switch signals the graphing
        routines on the PDF side to populate the required fields
        with the graphing data. The default is \texttt{populate=false}, do not
        populate, use silent mode.

    \item[] \textbf{Populate versus Silent Modes:} The \texttt{populate}
        key is how populate mode is distinguished from silent mode:
        \texttt{populate=true} is populate mode,
        \texttt{populate=false} (or the populate parameter not
        listed) is silent mode. In populate mode, the target
        graphing screen must have all required control fields; in
        silent mode, the only control should be the \cs{graphClrBtn}
        button.

    \item\texttt{connectwith}: The method used to connect
        consecutive points on the graph, possible values are
        \texttt{curve} and \texttt{segment}. This value is passed to the
        {\AcroFLeX} graphing widget. For function of $x$, the default is
        \texttt{curve}; otherwise, the default is \texttt{segment}. This
        value is ignored when the \texttt{graph} property signals
        plotting (\texttt{p1}--\texttt{p4}).

    \item\texttt{points}: The number of points to generate for
        plotting the current function. When the \texttt{graph}
        property signals plotting (\texttt{p1}--\texttt{p4}), the
        \texttt{points} property is ignored. If the graph property
        signals graphing (\texttt{c1}--\texttt{c4};
        \texttt{a1}--\texttt{a4}), and argument \texttt{\#3} is a
        set of rectangular points, the \texttt{points} property must
        either not be present, or set to zero (\texttt{points=0}).

    \item \texttt{xInterval}: (Required) An interval on the $x$-axis, the
        interval must be in the form \verb!{[a,b]}!, for example,
        \verb!xInterval={[0,1]}!. For functions of $x$, this interval
        represents the domain over which the function is graphed. It
        also represents the left and right boundaries of the graphing
        window.

    \item[] \textbf{Important:} The \texttt{xkeyval} package
        parses these parameters. Because the interval notation
        contains a comma (,), the whole interval must be enclosed in
        braces so the parsing will be correct, as illustrated above.

    \item\texttt{yInterval}: (Required) An interval on the $y$-axis, the
        interval must be in the form \verb!{[a,b]}!, for
        example, \verb!yInterval={[0,1]}!. It represents
        the upper and lower boundaries of the graphing window.

    \item[] As with \texttt{xInterval}, the interval needs to be
        enclosed in braces.

    \item\texttt{tInterval}: When plotting a set of parametric equations, or
        a polar function, this interval is required as a parameter. The
        interval is of the form \verb!{[a,b]}!, including the braces,
        and represents the domain of the parameter. The
        \texttt{tInterval} must not be included otherwise, that is, for
        graphing a function of $x$. Some early {\LaTeX} parsing tests
        whether the value of \texttt{tInterval} is empty (the default)
        or not. If nonempty, we assume the graphing is parametric or
        polar. For point plotting, \texttt{tInterval} must not be
        included in the parameter list.

    \item\texttt{xPlot}: The parameter \texttt{xInterval} determines
        the left and right boundaries of the graphing window; it also
        determines the interval over which the function is to be plotted.
        The \texttt{xPlot} separates these two functionalities; the value
        of \texttt{xPlot} is an interval \verb![a,b]!, over which the
        function will be plotted. Thus, \verb!xInterval={[-2,2]}! specifies
        the scaling of the x-axis; while \verb!xPlot={[0,1]}! defines
        the interval to plot the given function. If \texttt{xPlot} is not
        specified, then \texttt{xInterval} will be used.

    \item\texttt{noquotes}: When argument \texttt{\#3} is parsed, it
        is, by default, placed in double quotes, for example,
        \verb!"x^2"!; however, there are some situations where the
        double quotes should not be used. (See the \texttt{afgraph.tex}
        file for one such example.) Possible values for
        \texttt{noquotes} are \texttt{true} and \texttt{false}.
        Including \texttt{noquotes} in the option list is equivalent to
        \texttt{noquotes=true}. The default is \texttt{noquotes=false}.

    \item\texttt{wait}: Possible values for \texttt{wait} are
        \texttt{true} and \texttt{false}. Including \texttt{wait} in the
        option list is equivalent to \texttt{wait=true}. The default is
        \texttt{wait=false}. When using \cs{defineGraphJS} to create a
        JavaScript action that will execute multiple calls to
        \texttt{Graph\_xy} or \texttt{Graph\_xyt}, list \texttt{wait} in
        the option list. This will cause a slight delay that allows the
        graphing screen to become activated, (if not already activated)
        before the graphing data is created and sent to the {\AcroFLeX}
        graphing widget. See the example below in
        \hyperref[defineGraphJS]{Section~\ref*{defineGraphJS}},
        page~\pageref*{defineGraphJS}.
    \end{itemize}
  \item \texttt{<func\string|points>}: This argument can be a function or a set of points.
  \begin{itemize}
    \item A function can be three types: (1) a function of $x$; (2)
        a function of $t$; and a pair of function of $t$.  If there is a
        single function of $t$, case (2), that is interpreted as a polar
        function, and graphed accordingly. The pair of functions must be
        functions of $t$ and separated by a semi-colon (;); these are
        then interpreted as a set of parametric equations. For example,
        \verb!x^2! would be graphed as a parabola; \verb!1+sin(t)! would
        be graphed as a Cardioid in the polar coordinate system; and
        \verb!cos(t);sin(t)! would be graphed as a circle.

    \item Points can be input as a semi-colon-delimited list of
        rectangular coordinates. For example, \verb!(1,2);(2,3);(5,6)!.
        Points can be plotted discretely, or plotted and connected with
        either a smooth curve, or line segments.
  \end{itemize}
  \item \texttt{<text>}: The text that the link is attached to, when
      this text is clicked, the defined action of populating the graph
      occurs.
\end{enumerate}

\subsection{Graphing with \texorpdfstring{\protect\cs{defineGraphJS}}{\CMD{defineGraphJS}}}\label{defineGraphJS}

The \Com{defineGraphJS} is a command that expands to either
\texttt{Graph\_xy()} or \texttt{Graph\_xyt()}, and is essentially
the code used by \cs{sgraphLink}. Use \cs{defineGraphJS} to create a
custom link action or form field action to graph pre-packaged
functions.


\begin{dCmd}{\linewidth}
\defineGraphJS{<graph_key_vals>}{<func|points>}{<command>}
\end{dCmd}

\CmdDescription \cs{defineGraphJS} defines a new command
\cs{<command>} what will expand to \texttt{Graph\_xy()} or
\texttt{Graph\_xyt()} fully populated by its arguments. This command
can be used to create new actions that involve multiple calls to the
{\AcroFLeX} graphing routines.

\PD There are three required parameters.
\begin{enumerate}
  \item \texttt{<graph\_key\_vals>}: The same key-value pairs as
      described for \cs{sgraphLink}.
  \item \texttt{<func\string|points>}: An expression representing a
      function of $x$, a polar function of $t$, a set of parametric
      equations, or a set of points.
  \item \texttt{<command>}: A command that this JavaScript code will
      be saved under.
\end{enumerate}

An example of usage can be found in \texttt{afgraph.tex}, we
present another example here, also included in \texttt{afgraph.tex},  that
might suggest the value of this command. We construct a link that
graphs a function and plots discrete points.

\begin{Verbatim}[fontsize=\footnotesize]
\def\DomX{[0,2PI]}\def\DomY{[-1,1]}
\defineGraphJS{graph=c1,wait,xInterval={\DomX},yInterval={\DomY},
    points=40}{sin(x)}{\mySineCurve}
\defineGraphJS{graph=p1,wait,xInterval={\DomX},yInterval={\DomY}}
    {(0,sin(0));(PI/2,sin(PI/2));(PI,sin(PI));(3PI/2,sin(3PI/2));
    (2PI,sin(2PI))}{\mySinePoints}
\setLinkText[\A{\JS{%
    \clearGraphJS\r
    \mySineCurve\r
    \mySinePoints
}}]{Consider the sine function and indicated points}%
\end{Verbatim}
Note the use of the \texttt{wait} key in both the definitions to give the graphing screen
time to be activated and ready to receive data. Observe also the list of points is given
in symbolic form, we let JavaScript calculate the values for us.

The command \Com{clearGraphJS} is used to clear the graphing screen before new curves are written
to the screen.  \cs{clearGraphJS} expands to the document-level JavaScript function that clears
the graphing screen.

\section{Customizations}

There are a number of English phrases that appear as tooltips or as
messages in alert dialog boxes, as a result, the acroflex package has
a language option.
\begin{dCmd}{.75\linewidth}
\usepackage[lang=english|german]{acroflex}
\end{dCmd}
Specifying \texttt{english} as the value of \texttt{lang} inputs the
file \texttt{afcustom\_us.def}, which normally does nothing; the
definitions made in this file are the hard-wired defaults of the
package. As an English speaker, you can edit this file, and improve
the phrasing, if you wish.  Specifying \texttt{german} as the value
of \texttt{lang} inputs the file \texttt{afcustom\_de.def}; you can,
of course, edit this file to get a preferred phrasing. If not lang
key-value pair is specified, the \textsf{acroflex} package inputs
the file \texttt{afcustom.def} (found in the examples folder). This
file is intended for local use. Place it in the folder where the
source file resides, modify it as desired to get custom messages.
The file \texttt{afcustom.def} contains some instructions and guidelines for editing.

\redpoint If the file \texttt{afcustom.def} is placed on the latex search
path, it will be found and input for each source file; if
\texttt{afcustom.def} is in the source file folder, it is this
version that is found first and input.

Should the phrases entered in \texttt{afcustom.def} require special
accents, use the \texttt{unicode} option of {\Web} (which just
passes the \texttt{unicode} option on to hyperref), and enter any
special characters using {\LaTeX} notation. For example, to address
my formerly favorite friend, J\"{u}rgen, we can write,
\begin{verbatim}
\ttgraphBtn{J\"{u}rgen, press to graph the function}
\end{verbatim}
For the alert box messages, use JavaScript unicode notation, for example
\begin{verbatim}
\defineJSStr{\af@badNumberMsg}{%
    J\u00FCrgen, the value input does not appear to be a number,
    please enter a number, or an expression that evaluates to a
    number. \dps}
\end{verbatim}
This latter example does not require the \texttt{unicode} option.

\newtopic Note that \cs{defineJSStr} is a new command (defined in \textsf{eforms}) that enables you
to enter unicode, for example, \cs{u00FC} is the u-umlaut (\verb!\"{u}! or \"{u}). Also, within the
argument string, you can use \cs{r} (carriage return) and \t (tab) to format your lines as
needed. Double back slash \verb!\\! is converted into single backslash \verb!\!,
so for example, \verb!\\defineJSStr! appears in the dialog box as
\cs{defineJSStr}. The string argument is immediately expanded, so a
command like \cs{dps} (in the above definition) gets
expanded at definition time. Use \cs{protect} to delay the expansion
until the tex compiler finally expands the JS command string (useful here, if
\cs{dps} gets redefined).



\bigskip
That's all for now, I simply must get back to my retirement. \dps

\newpage
\leftskip20pt\rightskip20pt\small
\addcontentsline{toc}{section}{\protect\numberline{}Index}
\markright{Index}
\printindex

\end{document}
