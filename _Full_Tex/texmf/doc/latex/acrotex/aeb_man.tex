%
% makeindex < aeb_man.idx > aeb_man.ind
% Things to do
% Explain \AnswerField, and some of the others as needed.
% Talk about committing the data before End Quiz
%
% C:\yandy\aeb cfg;C:\Users\Public\Documents\My TeX Files\tex\latex\\;C:\YANDY\TEXINPUT;C:\YANDY\TEX\BASE\\;C:\yandy\TEX\LATEX\REQUIRED\PSNFSS\\;C:\Program Files (x86)\MiKTeX 2.8\tex\\;C:\Users\D. P. Story\AppData\Roaming\MiKTeX\2.8\tex\\
%
\documentclass{article}
%\usepackage{makeidx,showidx}
\usepackage[fleqn]{amsmath}
\usepackage[%
    web={centertitlepage,designv,
        forcolorpaper,latextoc,pro},
    exerquiz={nocorrections},aebxmp
]{aeb_pro}
%\usepackage{myriadpro}
\usepackage[altbullet]{lucidbry}
\usepackage{aeb_mlink}
\usepackage{longtable,colortbl}
\usepackage{graphicx}
\usepackage{acroman}
\usepackage[active]{srcltx}

\makeatletter
\renewcommand*\l@subsection{\@dottedtocline{2}{2em}{3.5em}}
\renewcommand*\l@subsubsection{\@dottedtocline{4}{5.5em}{1.5em}}
\renewcommand\@pnumwidth{2.55em}
\renewcommand*\l@section[2]{%
  \ifnum \c@tocdepth >\z@
    \addpenalty\@secpenalty
    \addvspace{1.0em \@plus\p@}%
    \setlength\@tempdima{2em}%{1.5em}%
    \begingroup
      \parindent \z@ \rightskip \@pnumwidth
      \parfillskip -\@pnumwidth
      \leavevmode \bfseries
      \advance\leftskip\@tempdima
      \hskip -\leftskip
      #1\nobreak\hfil \nobreak\hb@xt@\@pnumwidth{\hss #2}\par
    \endgroup
  \fi}
\makeatother


%\usepackage[usecmtt]{myriadpro}

%\margins{.25in}{.25in}{24pt}{.25in} % left,right,top, bottom
%\screensize{4.19in}{4.67in} % height, width

%\margins{.5in}{.25in}{24pt}{.25in} % left,right,top, bottom
%\screensize{4.19in}{4.92in} % height, width

%\hoffset=20pt

\setlongtables
\LTpre=0pt %\medskipamount
\LTpost=-2\medskipamount
%\LTpost=0pt %\medskipamount

%\usepackage{makeidx}
%\makeindex
%\usepackage[active]{srcltx}

%\DeclareFontFamily{U}{wi}{}
%\DeclareFontShape{U}{wi}{m}{n}{<-> wingding}{}
%\DeclareFontFamily{U}{webd}{}
%\DeclareFontShape{U}{webd}{m}{n}{<-> webdings}{}

%\font\zqacr=zqacr at 8pt

\makeatletter

\renewcommand\exlabelformat{{\exlabel\ \theeqexno.}}
\renewcommand\exlabelformatwp{{\exlabel\ \theeqexno.}}

%\def\changelinkcolorto#1{\def\@linkcolor{#1}}
%\useFullWidthForPaper
\newdimen\totaltextwidth
\totaltextwidth=\fullscreenwidth
\advance\totaltextwidth\oddsidemargin
\renewcommand{\webheadwrapper}[1]{%
    \hspace{-\oddsidemargin}%
    \makebox[\totaltextwidth][s]{#1}\hss
}

%\sectionLayout{indent=-62.5pt,fontsize=large}
%\subsectionLayout{indent=-31.25pt}
%\subsubsectionLayout{indent=0pt}

\renewcommand\hproportionwebtitle{.75}
\universityLayout{fontsize=Large,fontfamily=sffamily}
\titleLayout{fontsize=LARGE,fontfamily=sffamily}
\authorLayout{fontsize=Large,fontfamily=sffamily}
\tocLayout{fontsize=Large,color=aeb,fontfamily=sffamily}
\sectionLayout{indent=-62.5pt,fontsize=large,color=aeb,fontfamily=sffamily}
\subsectionLayout{indent=-31.25pt,color=aeb,fontfamily=sffamily}
\subsubsectionLayout{indent=0pt,color=aeb,,fontfamily=sffamily}
\subsubDefaultDing{\texorpdfstring{$\bullet$}{\textrm\textbullet}}
\renewcommand{\paragraph}
    {\@startsection{paragraph}{4}{0pt}{12pt}{-3pt}{\bfseries}}
\newcommand{\aebCng}{\@ifstar{\def\aeb@checked{*}\@aebCng}{\def\aeb@checked{x}\@aebCng}}%
\newcommand\@aebCng[1][]{}
%\newcommand{\aebCng}{\@ifstar{\def\aeb@checked{*}\@aebCng}{\def\aeb@checked{x}\@aebCng}}%
%\newcommand{\@aebCng}[1][]{%
%    \if\aeb@checked*\def\aeb@marker{\textcolor{blue}{aeb}~\textcolor{red}{\ding{052}}\goodbreak}\else
%    \def\aeb@marker{\textcolor{red}{aeb}}\fi
%    \def\argi{#1}\ifx\argi\@empty
%        \marginpar{\scriptsize\aeb@marker}%
%    \else\marginpar{\scriptsize\aeb@marker\goodbreak\raggedright#1}\fi}

%\makeindex

%\widestNumber{00.00.}

%\hfuzz=1pt

\pdfstringdefDisableCommands{\let\!\empty}

\setlongtables

\def\AcroT{Acro\!\TeX}\def\cAcroT{\textcolor{blue}{\AcroT}}
\def\AcroEB{\AcroT{} eDucation Bundle}\def\cAcroEB{\textcolor{blue}{\AcroEB}}
\def\AcroB{\AcroT{} Bundle}\def\cAcroB{\textcolor{blue}{\AcroB}}
\def\AcroFLeX{%
    AcroF\kern-.1667em\lower.5ex\hbox{L}\kern-.3eme\kern-.125emX}
\def\APB{\textsf{APB}}
\def\cAPB{\textcolor{blue}{\APB}}
\def\AEB{\textsf{AeB}}
\def\cAEB{\textcolor{blue}{\AEB}}
\def\Web{\textsf{Web}}
\def\PDF{\textsf{PDF}}
\def\bUrl{http://www.math.uakron.edu/~dpstory}

\hypersetup{linktocpage}

\def\AmS{{\protect\usefont{OMS}{cmsy}{m}{n}%
  A\kern-.1667em\lower.5ex\hbox{M}\kern-.125emS}}

\def\latex/{{\protect\LaTeX}}
\def\amslatex/{{\protect\AmS-\protect\LaTeX}}
\def\tex/{{\protect\TeX}}
\def\amstex/{{\protect\AmS-\protect\TeX}}

\definecolor{aeb}{rgb}{0.24,0.38,0.68}%bleu
\universityColor{aeb}
\tocColor{aeb}

\makeatother

\def\Y&Y{Y\kern-.2em{\footnotesize\&}\kern-.2emY}
\def\parss{\par\smallskip}
\def\parms{\par\medskip}
\def\dps{$\mbox{$\mathfrak D$\kern-.3em\mbox{$\mathfrak P$}%
   \kern-.6em \hbox{$\mathcal S$}}$}


\title{\texorpdfstring{\LARGE The \AcroEB\space(AeB)}
    {The AcroTeX eDucation Bundle for LaTeX, Manual of Usage}}
\author{D. P. Story}
\subject{How to create on-line exercises and quizzes}
\keywords{LaTeX, hyperref, PDF, exercises, quizzes}
\university{{\AcroT} Software Development Team}
\email{dpstory@acrotex.net}
\version{6.3}
\copyrightyears{1999-\the\year}
\nocopyright
\revisionLabel{Prepared:}

\copyrightStatus{True}
\copyrightNotice{Copyright (C) 1999-\the\year, D. P. Story}
\copyrightInfoURL{http://www.acrotex.net}

\noPartNumbers

\DesignTitlePageTrailer
{%
    ul=Copyright \copyright\ \webcopyrightyears,
    ur=\thewebemail,
    lr=\url{http://www.acrotex.net},
    raise=-25pt
}
\def\CMD#1{\textbackslash#1}
%\newcommand{\exPDF}[2][\expath/]{\par\ifdim\lastskip>0pt\relax\vskip-\lastskip\fi
\newcommand{\exPDF}[2][\bUrl/webeq_ex.html\#]{\par\ifdim\lastskip>0pt\relax\vskip-\lastskip\fi
\vskip\medskipamount\noindent\makebox[0pt][r]
    {\raisebox{-2pt}{\color{red}\href{#1#2}{{\zqacr
    b\hspace{9.5pt}}}}\enspace}\ignorespaces}

\def\makeExBox#1{\href{run:\tutpathi/apb_#1.tex}{\scriptsize\fbox{#1}}}
\renewcommand\expoint[1]{\par\ifdim\lastskip>0pt\relax\vskip-\lastskip\fi
\vskip\medskipamount\noindent\makebox[0pt][r]{\makeExBox{#1}\enspace}\ignorespaces}

\def\tutpath{doc/tutorial}
\def\tutpathi{tutorial}
\def\expath{../examples}

\renewcommand{\theaebEx}{\arabic{aebEx}}
\renewcommand{\theHaebEx}{\arabic{aebEx}}

\def\insMargin{}
\def\Ex{\par\medskip\noindent\refstepcounter{aebEx}\insMargin\gdef\insMargin{}\textbf{Example~\theaebEx:}}
\def\CmdLoc{\par\medskip\noindent\textbf{Command Location: }}
\def\EnvLoc{\par\medskip\noindent\textbf{Environment Location: }}
\def\KVP{\par\goodbreak\medskip\noindent\textbf{Key-Value Pairs: }}
\def\PD{\par\goodbreak\medskip\noindent\textbf{Parameter Description: }}
\def\Important{\par\medskip\noindent\textbf{Important: }}
\def\CmdDescription{\par\medskip\noindent\textbf{Command Description: }}

\begin{insDLJS}[indefCompare]{compare}{Comparisons}
function indefCompare(a,b,c,F,G) {
    var C, eqx;
    with(Math) {
        eqx = a;
        if(isNaN(C = eval(F)-eval(G))) return -1;
        eqx = c;
        F = eval(F);
        if(isNaN(G = eval(G)));
        return abs( F - G - C );
    }
}
\end{insDLJS}

\frenchspacing

\definePath\bgPath{"C:/Users/Public/Documents/ManualBGs/Manual_BG_Print_AeB.pdf"}
\begin{docassembly}
\addWatermarkFromFile({
    bOnTop: false,
    cDIPath: \bgPath
})
\executeSave()
\end{docassembly}
\begin{document}

\maketitle

\newpage

\changelinkcolorto{black}

\tableofcontents

\changelinkcolorto{webgreen}

\part{Preface}

%\section{Introduction}

The {\cAcroEB} (abbreviated {\cAEB} and read ``{\AcroT} Education Bundle'') is a collection of {\LaTeX} macro files (briefly described below) along with various support and sample files.
The theme of {\cAEB} is \textit{ePublication} in the \textit{education} sector using {\LaTeX} as the authoring application and Adobe's Portable Document Format (\textsf{PDF}) as the file format of the output document.

\newtopic Currently, there are seven components to the bundle:

\begin{questions}

\item The \hyperref[s:web]{\textsf{web} package} is used to create an attractive, easy-on-the-eye page layout suitable for the \textsf{WWW} (or classroom/conference presentations).
\item  The \hyperref[s:exerquiz]{\textsf{exerquiz} package} makes
    it very easy to create interactive exercises and quizzes.
\item The \href{eformman.pdf}{eforms} package provides support for
    PDF form fields.
\item The \textsf{insdljs} package allows for the automatic
    insertion of document level JavaScript. Document authors can use
    \textsf{insdljs} to customize the processing of the
    \textsf{exerquiz} quizzes.  See the documentation that accompanies
    the package (\texttt{insdljs.dtx}) and see also the sample file
    \texttt{jqzspec.tex} for an extensive example of how to modify the
    \textsf{exerquiz} macros. The \textsf{insdljs} package also has an
    \texttt{execJS} environment which can be used to create executable
    and ``discardable'' JavaScript; see the DTX file for
    details.
%    and the demo file \texttt{execJStst.tex}, which features
%    an animation built completely from {\LaTeX} source lines.
    Documentation for the \textsf{insdljs} package appears in the
    \href{eformman.pdf}{eforms} reference manual
    (\href{\bUrl/eformman.pdf}{eformman.pdf}).
\item The \textsf{taborder} package supports the creation of the order
    of tabbing between Acrobat annotations (fields, links, and comments).
    See the part \textbf{Setting the Tab Order} in \href{eformman.pdf}{eforms}
    reference manual (\href{\bUrl/eformman.pdf}{eformman.pdf}) for more details.
\item The \textsf{dljslib} package is used as a library of
    JavaScript functions.  Some types of question require special
    processing. A JavaScript function written to process a
    particular function can be stored in the library, then
    retrieved as needed.  See the documentation contained in the
    file \texttt{\href{\bUrl/dljslib.dtx}{dljslib.dtx}}.% and try
%    the test file for this package,
%    \texttt{\href{\bUrl/jslib_ex.tex}{jslib\_ex.tex}}.
\item The \textsf{\href{\bUrl/eqexam.html}{eqExam}} Package is a
    stand alone {\LaTeX} package for creating exams, quizzes, homework
    assignments. It can also be used to create surveys, questionnaires,
    teach\-er evaluations, etc. The package has an \texttt{email} mode and
    comes with a server-side script that is used to send the data entered
    into the form fields to the instructor. See my
    \textsf{\href{\bUrl/eqExam/tex_survey.pdf}{\TeX/\LaTeX{} Survey}} for
    a demo of this feature.
\end{questions}

\noindent The \cAcroB{} should be useful to educators who want to post
interactive materials for their students on the \textsf{WWW}.

\begin{comment}
To play it safe, you should use the most recent version of
\textsf{hyperref} by Sebastian Rahtz and now Heiko Oberdiek. The
\textsf{exerquiz} package makes extensive use of forms; Adobe
Reader 3.0 with Forms 3.5 at least is needed, Adobe Reader 4.0 or
above is preferred.
\end{comment}

Here is an important point that should be emphasized early in this
manual. \cAcroT{} only supports three ways of producing a
\textsf{PDF} file: (1) the \textcolor{blue}{Adobe Acrobat
Distiller} (version~5.0 or higher \emph{required}); (2)
\textsf{pdftex}; or (3) \textsf{dvipdfm}. In the case of (1), you
probably use \texttt{dvips} to create a postscript file before
distilling. Some users have tried to use
\textcolor{blue}{GhostScript} to produce a \textsf{PDF}
from \cAcroT; this will not work! (You will get the \textsf{PDF}
file, but not much functionality.)

Please contact me at \url{dpstory@acrotex.net} should you encounter
any problems, or have suggestions to make.

\redpoint See \Nameref{s:start} for instructions on how to get up
and running.

\section{Getting Started}\label{s:start}

The distribution for the {\cAcroB} comes in a single ZIP file
\begin{itemize}
  \item \texttt{acrotex\_pack.zip}: This ZIP file contains program
  files (web, exerquiz, eforms, insdljs, dljslib, taborder, and aeb.js),
  this manual, and the eForms manual.
\end{itemize}

\subsection{Installing the Distribution}

The following are the instructions for installing the \cAEB.

\begin{questions}

\item \textbf{Placement.} The ZIP file installs in a folder called \texttt{acrotex}, so place
the ZIP file in a directory in the search path of your \LaTeX{} system
where you want the \texttt{acrotex} folder to reside. If you already have
an \texttt{acrotex} folder, place the ZIP files so that they unzip into this
\texttt{acrotex} folder, \emph{unless you are using MiK\TeX~2.8}.

\item[] \textbf{MiK\TeX~2.8 Users.} MikTeX 2.8 provides problems for
    {Acro\!\TeX} installations in that it is more particular about where you
    install packages by hand. MiKTeX 2.8 requires that you install the
    distribution in a local root TDS tree. Review the MiKTeX help page on this
    topic\\[3pt]\hspace*{20pt}
%
    \url{http://docs.miktex.org/manual/localadditions.html}\\[3pt]
%
Create a new {MiK\TeX} root folder, in my case, I created\\[3pt]\hspace*{20pt}
%
    \verb!C:\Users\Public\Documents\My TeX Files!\\[3pt]
%
Within the \texttt{My TeX Files} folder (you can name this folder anyway
you wish), create a \texttt{tex} folder, within the \texttt{tex} folder,
create a \texttt{latex} folder. Now copy the ZIP file to the
\texttt{latex} folder, for example, copy them into
\\[3pt]\hspace*{20pt}
%
    \verb!C:\Users\Public\Documents\My TeX Files\tex\latex!\\[3pt]
%
Finally, you need to register your new root folder with MiK\TeX. Open the
{MiK\TeX} Options dialog box (accessible through the WinEdt toolbar, or
through\\[3pt]\hspace*{20pt}
%
\texttt{Start > All Programs > MiKTeX 2.8 > Maintenance > Settings}\\[3pt]
%
Within this dialog box, select the \textbf{Roots} tab, then press
\textbf{Add}, and browse for your newly create \texttt{My TeX Files}. When
finished, the dialog should look like my own dialog box, as shown
in \hyperref[fig:rootstab]{Figure~\ref*{fig:rootstab}}, page~\pageref*{fig:rootstab}.

\begin{figure}[hbt]
\begin{center}
\includegraphics[scale=.6]{figures/miktex_roots}
\caption{Roots tab of MiK{\TeX} Options}\label{fig:rootstab}
\end{center}
\end{figure}

\item \textbf{Unzipping the Distribution.} Once you have found the
location to unzip the ZIP file, unzip it!

% C:\Users\Public\Documents\My TeX Files\tex\latex

%\item Unpack the \textsf{exerquiz} package by latexing \texttt{exerquiz.ins}.
%However, see \nameref{s:unpack} for important comments.

\item \textbf{Unpacking the distribution.} The whole distribution can be
unpacked by latexing \texttt{acrotex.ins}. (The other \texttt{*.ins} files
are the installation files for the individual packages,
\texttt{acrotex.ins} is the combined installation file.)

\item[] {MiK\TeX} users should not forget to refresh the file name database.

% \item Unpack \texttt{insdljs.dtx} by latexing \texttt{insdljs.ins}.

\end{questions}



\subsection{Installing \texttt{aeb.js}}

The JavaScript file \texttt{aeb.js} is only needed if you use
Acrobat Pro 8.1 or later. Increased security in that version has
made it necessary to install a folder JavaScript file.
\texttt{aeb.js} raises the ``trust'' of certain JS methods used to
import document JavaScript into the document just after
installation. The installation of this file is not needed if you use
pdftex or dvipdfm, or use a version of Acrobat prior to 8.1.

Start \textbf{Acrobat Pro~8.1}\footnote{The file will cause no harm
if you install it in an earlier version of Acrobat, as early as
version 7.} or later, and open the console window \textsf{Advanced
\texttt> JavaScript \texttt> Debugger} (\texttt{Ctrl+J}). Copy and
paste the following code into the window.
\begin{verbatim}
    app.getPath("user","javascript");
\end{verbatim}
Now, this the mouse cursor on the line containing this script, press
the \texttt{Ctrl+Enter} key. This will execute this JavaScript. This
JavaScript method returns the path to where \texttt{aeb.js} should
be placed. For example, on my system, the return string is
\begin{Verbatim}[fontsize=\small]
   /C/Documents and Settings/story/
        Application Data/Adobe/Acrobat/8.0/JavaScripts
\end{Verbatim}
Follow the path to this folder. If the \texttt{JavaScripts} folder
does not exist, create it.  Finally, copy \texttt{aeb.js}
into this folder. The file is read each time Acrobat starts.

\subsection{Language Localizations}

Should you wish to customize {\cAEB} to a language other than the ones
already supported, open the file \texttt{template.def} and add in your
language customizations.

%After reading the manual you are then ready to write your own set
%of tutorials, exams, exercises and quizzes.\enspace\dps


\subsection{Sample Files}\label{s:samplefiles}

Test, example, and demo files have been moved to the \href{\bUrl/webeq_ex.html}{Example Files for
AeB} web page. References seen in this manual reside on this page; each
sample file is a PDF with the source file attached to the PDF. Files one
the \href{\bUrl/webeq_ex.html}{Example Files for AeB} web page are
referenced with the icon \raisebox{-1.5pt}{\color{red}\zqacr b\hspace{12pt}} is the margin.

There is another more recent collection of examples on \href{\bUrl/aebblog.html}{AeB
Blog}, these will be referenced in the margin using the icon %\exAeBBlogPDF
\mbox{\makebox[0pt][l]{\textcolor{blue}{\Pisymbol{webd}{254}}}%
    \raisebox{-2pt}{\color{red}{{\zqacr b\hspace{9.5pt}}}}}\,.

%\exPDF{indexofex} There are numerous sample/demo files that illustrate various
%features of the {\cAcroB}. View the \textsf{PDF} file
%\texttt{\href{indexofex.pdf}{indexofex.pdf}}, which is an ``Index of
%{\AcroT} Examples''. This document contains a list of all
%examples, a short description of each, and links to the \textsf{PDF}
%file and source file.

\subsection{Package Requirements}

If you use the Acrobat Distiller, as I do, to create a PDF document, the {\cAcroB}
now requires the use of version 5.0 or later. I've given up on trying to support
prior version of Acrobat.

\newtopic In terms of \LaTeX, the following is a listing of package requirements:

\begin{enumerate}
    \item The \textsf{Web} Package
        \begin{itemize}
            \item \texttt{color}: \texttt{color} is distributed with \LaTeX, but {\Web}
            will use \texttt{xcolor}, if available, unless the
            \texttt{noxcolor} global option is used.
            \item \texttt{amssymb}: standard with \amslatex/
            \item \texttt{hyperref}: available from CTAN, get newer version
            \item \texttt{eso-pic} and \texttt{everyshi}: available from CTAN
        \end{itemize}
    \item The \textsf{Exerquiz} Package
        \begin{itemize}
            \item \texttt{color}: distributed with \LaTeX
            \item \texttt{verbatim}: distributed with \LaTeX
            \item \texttt{hyperref}: available from CTAN, get newer version
            \item \texttt{insdljs}: distributed with \AcroT
        \end{itemize}
    \item The \textsf{insdljs} Package
        \begin{itemize}
            \item \texttt{hyperref}: available from CTAN, get newer version
            \item \texttt{verbatim}: distributed with \LaTeX
            \item \texttt{everyshi}: available from CTAN
        \end{itemize}
    \item The \textsf{dljslib} Package
        \begin{itemize}
            \item \texttt{insdljs}: distributed with \AcroT
        \end{itemize}
\end{enumerate}

\subsection{\LaTeX ing Your First File}

The functionality of the \texttt{shortquiz} and \texttt{quiz}
environments depends on JavaScript code that is placed at the
``document level'', to use Adobe's terminology.  The applications
\textsf{pdftex} and \textsf{dvipdfm} offer direct support for
writing to this document level. For those who use the
\textcolor{blue}{Adobe Distiller}, things aren't quite so easy.

In this section, we describe how to insert document level
Java\-Scripts into a \textsf{PDF} file, prepared from a
\LaTeX{} source that uses the \texttt{exerquiz} package. Even
though the handling and insertion of document level JavaScript is
done with the package \textsf{insdljs}, a little care must be
taken, at least in the Distiller case, when building your
{\PDF} document.

\exPDF{webeqtst} Open \texttt{webeqtst.tex} in your favorite text editor. The top
lines read:
\begin{sverbatim}
    \documentclass{article}
    \usepackage{amsmath}
    \usepackage[tight,designi]{web}
    \usepackage{exerquiz}
\end{sverbatim}

\subsubsection{For \textsf{pdftex} and \textsf{dvipdfm} Users}

Edit the third line by inserting your driver; the choices are
\texttt{pdftex} and \texttt{dvipdfm}. For example, if you use \textsf{dvipdfm}, the
lines should read:
\begin{sverbatim}
    \documentclass{article}
    \usepackage{amsmath}
    \usepackage[dvipdfm,tight,designi]{web}
    \usepackage{exerquiz}
\end{sverbatim}

\noindent For the \textsf{pdftex} application, you simply call
\texttt{pdflatex}, and you have your nice \textsf{PDF} file,
ready for review. The insertion of the document level JavaScript
is automatic.

For \textsf{dvipdfm}, you \LaTeX{} the document, then hit it with \textsf{dvipdfm},
and your ready to review your \textsf{PDF} file.

\subsubsection{For Distiller Users}

If you use the distiller, as I do, the sophisticated features of
{\AcroB} require Acrobat 5.0 or higher.

%I've discontinued my attempt at supporting Acrobat 4.0.

Edit the optional parameters of the \textsf{web} package by inserting your driver; the choices are
\texttt{dvips} and \texttt{dvipsone}.\footnote{The choice of driver can be placed in he \texttt{web.cfg} configuration file, in which
case, you need not specify the driver. See \Nameref{sss:driver}.} For example, if you use \textsf{dvips}, the
lines should read:
\begin{sverbatim}
    \documentclass{article}
    \usepackage{amsmath}
    \usepackage[dvips,tight,designi]{web}
    \usepackage{exerquiz}
\end{sverbatim}

\begingroup\raggedright
\noindent When you latex the source file, you create a DVI file and one or more FDF files.\footnote{The FDF files (for example, \texttt{exerquiz.fdf}) contain the document level JavaScript that needs to be imported into your document. They are imported when the newly
created document is opened is Acrobat for the first time.}
You then convert your \texttt{.dvi} to \texttt{.ps} using either
\textsf{dvips} or \textsf{dvipsone}, and distill.
\par\endgroup

\newtopic\textcolor{red}{Important:} When you distill, \emph{save the
\emph{\texttt{.pdf}} back to the same folder} in which your source file
(\texttt{.tex}) resides as this is where the \texttt{.fdf} files
reside too. Insertion of document level JavaScripts automatically
takes place when you open your newly distilled document
in the \textsf{Acrobat} application. (It is actually
\textsf{Acrobat} that imports the scripts, not the
\textsf{Distiller}.)

\handpoint  When your document is opened in \textsf{Acrobat} for
the first time, the Java\-Script contained in the \texttt{.fdf} files
(for example, \texttt{exerquiz.fdf}) is imported into the document
and is stored at the document level.

\redpoint \textcolor{red}{Important:} \emph{Save your document}. When you
save, the JavaScripts you just imported are also saved with the
document. At this point you can move your \textsf{PDF} to another
folder, or to the web. The document does not need the
\texttt{.fdf} files any more.

\begin{comment}
\redpoint \textbf{For Distiller~4.0--4.05 Users.} Versions prior
to~5.0 of the Acrobat product cannot import document level
JavaScript contained in a \texttt{.fdf} file. The JavaScript needs
to be inserted ``by hand''.

The procedure is as follows: Modify the preamble and insert the \texttt{acrobativ}
option for the \textsf{exerquiz} package:
\begin{sverbatim}
\documentclass{article}
\usepackage{amsmath}
\usepackage[dvips,tight,designi]{web}
\usepackage[acrobativ]{exerquiz}      %<- acrobativ option
\end{sverbatim}
\noindent This suppresses all the JavaScript code generation that is used in the case
of \textsf{pdftex}, \textsf{dvipdfm} or \textsf{Distiller~5.0$^+$}. After that
change, \LaTeX{} the document, and convert
the \texttt{.dvi} file to PostScript (using \texttt{dvips} or
\texttt{dvipsone}), and distill. Now, open the new \textsf{PDF} file
 in \textsf{Acrobat} (formerly known as
\textsf{Exchange}). Click on \texttt{Document > Insert Pages},
browse, and choose the \textsf{PDF} file \textsf{eq\_dljs.pdf},
which comes with the \cAcroB. The file is now inserted. Next, click
on \texttt{Document > Delete Pages} and delete the page you just
inserted! \textcolor{red}{Important:} Do a ``\texttt{Save As}''; now you are done!

The \textsf{PDF} file \textsf{eq\_dljs.pdf} contains all the
standard JavaScript that goes in at the document level.  In all
the other situations discussed above, you can modify the
JavaScript from the preamble, for example, in the \textsf{exerquiz}
source code we have
\begin{sverbatim}
\newcommand\checkColor{["RGB", 0, .6, 0]}
\end{sverbatim}
This command is expanded when the document level JavaScript is imported into the
\textsf{PDF} file. If \cs{checkColor} has been redefined
\begin{sverbatim}
\renewcommand\checkColor{["RGB", 1, 0, 0]}
\end{sverbatim}
it is this definition that is used in the expansion. This
convenience is lost for $\text{4.0} \le\text{\textsf{distiller}}\le\text{4.05}$.
You can edit \textsf{eq\_dljs.pdf} and make whatever changes you
please to the script, in terms of changing color.
\end{comment}

There are several ways the Distiller workflow can be accomplished:
%\begin{itemize}
%    \item \textbf{Through the {\LaTeX} Editor.} Some editors, WinEdt, in particular, have toolbars that can distill
%    a PostScript file. (The default is to use  Ghostscript, you'll have to change to to Distiller.) From WinEdt, you
%    can latex, convert to Postscript, and distill with a few presses of toolbar buttons. Very smooth and easy.
%    \item \textbf{Do it yourself.} Some systems, most notably on Mac OS, do not have a convenient way of distilling.
%    A Mac user sent in the following steps.
%    \begin{enumerate}
%        \item Drop the \texttt{.ps} file on Distiller.
%        \item Drag the resulting .pdf file out of the folder it was
%            compiled in (for example, to the desktop)
%        \item Open the \texttt{.pdf} file in Acrobat.
%
%        At this stage readers will be prompted for the location of the existing .fdf file for inclusion.
%
%        Now, for contrast, try this:
%
%        1. Drop the .ps file on Distiller.
%        2. Open the .pdf file in Acrobat.
%        3. Save the .pdf file.
%        4. Drag the .pdf file out of the folder it was compiled in (for example, to the desktop)
%        5. Open the .pdf file.
%
%        This will allow readers the opportunity to see what will happen if they fail to save the file before moving it to another location, like the web.
%
%
%    \end{enumerate}
%\end{itemize}

\newtopic For distiller users, the
\href{http://www.math.uakron.edu/~dpstory/aeb_pro.html}{\AEB\ Pro package}
has many exciting features, functionality requires the document author use
Acrobat~7.0 or higher.



\addtocontents{toc}{\protect\vspace{\medskipamount}}
\part{The Web Package}\label{s:web}

\section{Introduction}

The purpose of the \textsf{web} package is to create a page layout
for documents meant for screen presentation, whether over the
\textsf{WWW} or classroom/conference presentations, in
\textsf{PDF}. Such documents are \emph{not \emph{(necessarily)}
intended to be printed}; consequently, the page layout is, in some
sense, optimized for screen viewing.

\subsection{Overview}

The \textsf{web} package redefines \cs{maketitle} and
\cs{tableofcontents} in a more web friendly way; it colors the
section headings, and inserts \cs{bullet}s ($\bullet$) at the
\cs{subsubsection} level.  This, to my eyes, is very attractive.
Additionally, certain navigation devices (a navigation bar and
some direction icons) are included in the package.

There are options for a small collection of drivers:
\texttt{dvipsone}, \texttt{dvips}. \texttt{dvipdfm} and \texttt{pdftex}.  The language
option redefines certain language dependent elements of the package
to other languages. Currently, the following options are supported:

\medskip
\begin{tabular}{lll}
\texttt{dutch}   & \texttt{french} & \texttt{german}\\
\texttt{italian} & \texttt{norsk}  & \texttt{russian}\\
\texttt{spanish} & \texttt{dansk}  & \texttt{polish}\\
\texttt{finnish} & \texttt{czech}  & \texttt{catalan}\\
\texttt{brazil}  & \texttt{turkish}
\end{tabular}
\medskip

\noindent There is even an option for reformatting the \texttt{web}
style to a print format!

The capabilities of the \textsf{web} package and its options are
discussed below. Any comments and suggested improvements (new
features) would be greatly appreciated.

\subsection{Package Requirements}

The \textsf{web} package was designed for screen presentations
tutorials, such as classroom or conference lectures, short technical
articles, etc.; consequently, the \texttt{article} class of \LaTeX{}
seems to be sufficient for these purposes.  Though you can use
\textsf{web} with any of the standard classes that define the
\cs{section}, \cs{subsection} and \cs{subsubsection} commands, the
package is really meant to be used with the \texttt{article}
class.  It is \textbf{strongly} suggested!

The package heavily depends on Sebastian Rahtz' \textsf{hyperref}
package (now maintained and developed by Heiko Oberdiek). The
\textsf{web} package was developed using version 6.56 of
\textsf{hyperref}. Using prior versions of \textsf{hyperref}
\emph{may} lead to successful compilation, no guarantees offered.
It is best to work with the most recent version of
\textsf{hyperref}.

The \textsf{color} and \textsf{amssymb} packages are also
required. The former is for obvious reasons, the later is to
provide certain navigation symbols when the \texttt{navibar}
option is invoked.

Finally, to create quality PDF documents, type 1 fonts \emph{must}
be used. Fortunately, type 1 fonts in the Computer Modern font set
are freely available, and come with all the major
freeware, shareware and commercial \TeX{} systems. If you haven't
done so already, learn how to use the type 1 fonts.

In this regard, I have written an article that may be of interest
to you entitled ``\textit{Using \LaTeX{} to Create Quality PDF Documents
for the WWW}'', see reference \cite{article:Story:LaTeX}.

\section{Basic Package Options}

To use the \textsf{web} package, insert into the preamble of
your document the following:
\begin{sverbatim}
   \usepackage[<driver_option>,<other_options>]{web}
\end{sverbatim}
\noindent Replace \verb+<other_options>+ with any of the options recognized by
\textsf{web}; see \hyperref[s:ListofOpts]{Section~\ref*{s:ListofOpts}}
for a complete list of options. The first and optional argument
\verb+<driver_option>+ above defines the driver to be used; for
example,
\begin{sverbatim}
   \usepackage[dvipsone]{web}
\end{sverbatim}
\noindent Currently, the \textsf{web} package supports five drivers:
\texttt{dvipsone}, the dvi-to-ps converter by \Y&Y, Inc.;
\texttt{dvips}, the freeware dvi-to-ps converter; \texttt{pdftex},
the \texttt{tex}-to-\texttt{pdf} application; and \texttt{dvipdfm},
the \texttt{dvi}-to-\texttt{pdf} application by Mark Wicks,\footnote{\url{http://odo.kettering.edu/dvipdfm/}} and the commercial \TeX{} system for the Mac, \texttt{textures} and \href{http://www.uoregon.edu/~koch/texshop/texshop.html}{{\TeX}shop}.

\redpoint The package has been tested using
\verb+\documentclass{article}+ and it is \emph{strongly}
recommended that this class be used; however, {\Web} does support the
\texttt{book} class as well.

%\begin{comment}
%
%\redpoint The default paper setting for \textsf{dvips} is
%\texttt{A4}, but the cropping parameters assumes \texttt{letter}
%paper size.  To get the cropping of the page correct, there are
%several procedures you can follow.
%
%\begin{questions}
%\item  Use the command line switch \texttt{-t letter}.
%For example,
%\begin{sverbatim}
%dvips.exe -j0 -Pcm -Pams -t letter -o <foo>.ps <foo>.dvi
%\end{sverbatim}
%\item Change the default paper setting of \textsf{dvips} to
%\texttt{letter} by placing the paper size for \texttt{letter}
%first in the \texttt{config.ps} file of \textsf{dvips}:
%
%\smallskip
%\leavevmode\begin{minipage}{.45\linewidth}
%Change these lines \dots
%\begin{sverbatim}
%@ A4size 210mm 297mm
%@+ %%PaperSize: A4
%
%@ letterSize 8.5in 11in
%
%@ letter 8.5in 11in
%@+ %%BeginPaperSize: Letter
%@+ letter
%@+ %%EndPaperSize
%\end{sverbatim}
%\end{minipage}
%\hfil
%\begin{minipage}{.45\linewidth}
%\dots to these.
%\begin{sverbatim}
%@ letter 8.5in 11in
%@+ %%BeginPaperSize: Letter
%@+ letter
%@+ %%EndPaperSize
%
%@ A4size 210mm 297mm
%@+ %%PaperSize: A4
%
%@ letterSize 8.5in 11in
%
%\end{sverbatim}
%\end{minipage}
%
%\end{questions}
%
% If you find changing the command line inconvenient, you can
%include the command
%\begin{sverbatim}
%   \hypersetup{pdfpagescrop=53 486 389 754}
%\end{sverbatim}
%in the preamble of your document.
%
%\end{comment}

%\makeatletter\hy@bookmarksnumberedfalse\makeatother
%\subsubsection{\texorpdfstring{}{\textbullet\space}Setting the Driver Option}
%\makeatletter\hy@bookmarksnumberedtrue\makeatother

\subsection{Setting the Driver Option}\label{sss:driver}

You can set your driver option in one of three ways:
\begin{itemize}
  \item Pass as a local option:\\ %[3pt]
        \verb+\usepackage[<driver_option>]{web}+
  \item Pass as a global option:\\ %[3pt]
        \verb+\documentclass[<driver_option>]{article}+
  \item Create the file \texttt{web.cfg} with the single command
       in it:\\ %[3pt]
       \verb+\ExecuteOptions{<driver_option>}+\\
       Place the file \texttt{web.cfg} in any folder where \LaTeX\
       looks for input files.  Then, you need only type
       \verb+\usepackage{web}+.
\end{itemize}
Note that \texttt{<driver\_option>} is any of the following options:
\texttt{dvipsone}, \texttt{dvips}, \texttt{pdftex}, \texttt{dvipdfm}
or \texttt{textures}

The macros of the \textsf{web} package have been extensively
tested using the \Y&Y\ \TeX\ System for the
\texttt{dvipsone} option and a MiK\TeX\ System
(\url{www.miktex.org}) for the \texttt{dvips}, \texttt{pdftex}
and \texttt{dvipdfm} options.

%\subsubsection{The \texorpdfstring{\protect\texttt{pdftex}}{pdftex} Option}
%
%
%The \texttt{pdftex} option requires \textsf{hyperref} version 6.60
%or greater. (The most recent version is preferred.) The
%\textsf{web} and \textsf{exerquiz} packages will probably run
%correctly with slightly earlier versions, but you may see
%`underfull hbox' error messages.
%
%The \textsf{web} package uses the AMS Font set for the
%black triangle that appears in the navigation bar, see
%\hyperref[ss:naviaids]{Section~\ref*{ss:naviaids}}.
%
%The AMS Font sets comes with Mik\TeX, if you don't have
%\textsf{pdftex} configured the use them, here are some
%instructions that work for the Mik\TeX{} system.
%\begin{questions}
%\item Copy the file \texttt{amsfonts.map} from the folder
%\texttt{texmf/dvips/ams} to the folder \texttt{texmf/pdftex/base}.
%(Mik\TeX{} may have restructured the directory tree since the time
%of this documentation.)
%\item Edit the file \texttt{pdftex.cfg} to include the lines:
%\begin{sverbatim}
%map  +amsfonts.map     % to get AMSFonts
%map   standard.map     % to get dings
%\end{sverbatim}
%
%\item The  \texttt{pdftex.cfg} file now looks like this:
%\begin{sverbatim}
%output_format 1
%compress_level 9
%decimal_digits 2
%page_width 210mm
%page_height 297mm
%horigin 1in            % <-- important, see 4.
%vorigin 1in            % <-- important, see 4.
%map standard.map
%map +cm.map
%map +amsfonts.map
%\end{sverbatim}
%\item Take note of the settings of \texttt{horigin} and
%\texttt{vorigin}. Set these two to 1\,in\@.  This will align the
%text on the page where the cropping numbers expects the text to
%be; the result will be a properly cropped page.
%\end{questions}


\subsection{The \texttt{tight} Option}\label{sss:webtight}

In an effort to compact more material per page, I've introduced a
\texttt{tight} option.  When this option is used, many of the list
parameters are redefined so that there is not so much space around these
environments, and between items.
\begin{sverbatim}
   \usepackage[<driver_option>,tight,<other_options>]{web}
\end{sverbatim}
\noindent This screen version of this manual was typeset with the
\texttt{tight} option, the print version was typeset without it.

\subsection{The \texttt{usesf} Option}\label{usesf}

For presentations, some people like a sans serif font. Using this option
make san serif the default font style.

\subsection{The \texttt{draft} Option}\label{draft}

When you take this option, graphic overlays are not allowed.  This
is useful when you rely heavily on graphic overlays, but during the
development phase, don't need to read and re-read your overlays. The
defined background colors will be used instead. Remove this option
to build the final version of your document.


\subsection{The \texttt{nobullets} Option}\label{nobullets}

Traditionally, the subsubsections have been denoted by a bullet, by
using this option, you can force the use of numbers for the
subsubsections, This option automatically executes the
\texttt{latextoc} option.

There is another option \texttt{forceSubSubNumbers}, which does the same thing
as nobullets. This option is included for compatibility with the {\AcroT}
Presentation Bundle (\textsf{APB}).

\subsection{The \texttt{unicode} Option}\label{web:unicode}

This option is passed to \textsf{hyperref}, and is placed in {\Web} as a
convenience. If \textsf{eforms} is subsequently loaded, \textsf{eforms} (and
\textsf{exerquiz}) will accept {\LaTeX} markup in the optional argument
of the form fields. See the \href{eformman.pdf}{eforms manual}
for details.

\subsection{The \texttt{useui} Option}\label{web:useui}

This option is passed to \textsf{eforms}, and is placed in {\Web} as
a convenience. If \textsf{eforms} is subsequently loaded,
\textsf{eforms} loads the xkeyval package, and key-value pairs are
defined for use in the optional argument of form and link commands.
See the \href{eformman.pdf}{eforms manual} for details.

\section{Setting Screen Size}\label{s:setscreensize}

Beginning with version 2.0, the screen size can be set by the
author. There are two ways to do this: (1) use the macros
\cs{screensize} and \cs{margins} (These are the same
macros, slightly redefined, for setting the screen size used by
Radhakrishnan in his fine screen package \textsf{pdfscreen}.); or (2) use
a screen design option. The next two sections address each of
these in turn.

\subsection{Custom Design}\label{cusDesign}

There are six dimensions that need to be specified. As with
\textsf{pdfscreen},  the two commands \cs{screensize} and
\cs{margins} are used for doing so.

The command \cs{screensize} takes two length parameters:
\begin{sverbatim}
    \screensize{<height>}{<width>}
\end{sverbatim}
\noindent The \texttt{<width>} and \texttt{<height>} parameters are the desired
screen size of the page. The screen version of this manual uses
\begin{sverbatim}
    \screensize{3.72in}{4.67in}
\end{sverbatim}
\noindent The other command, \cs{margins}, which determines the desired
margins, takes four length parameters:
\begin{sverbatim}
    \margins[<panel_width>]{<left>}{<right>}{<top>}{<bottom>}
\end{sverbatim}
\newtopic The values of \cs{textheight} and \cs{textwidth} are computed
based on the screen size and the margins. The margin settings for
this document are given below:
\begin{sverbatim}
    \margins{.25in}{.25in}{30pt}{.25in}
\end{sverbatim}
\newtopic The optional first parameter \texttt{<panel\_width>} is used to set
the width of the panel, if there is one.  The default is 1in.


\redpoint An important comment about the third parameter
\texttt{<top>} is the following: as with \textsf{pdfscreen}, we put
\verb+\@Topmargin=<top>+.  The running header fits within the top
margin (this varies from standard \LaTeX{} practice).  The
\textsf{web} package dimension \cs{web@Topmargin} is the distance
from the top of the screen down to the top of the running header. Thus,
\begin{sverbatim}
    \@Topmargin = \web@Topmargin + \headheight + \headsep
\end{sverbatim}
\noindent Also, \cs{web@Topmargin} can be used to adjust the
positioning of a running header, which is specified in the
\cs{margins} command. The default value of \cs{headheight} is
\texttt{8pt}, so the value of \cs{headsep} is determined by the
above equation. See the \textsf{web.dtx} file for more details.


\subsection{Screen Design Options}\label{sss:designoptions}

For your convenience, I've included six options, \texttt{designi}, \texttt{designii},\dots \texttt{designviii}.  The first one roughly corresponds to the original screen dimensions of \textsf{web}. The dimensions of these designs options are
\begingroup\par\noindent\hfuzz=2pt\small
\begin{tabular}{>{\ttfamily}l>{\ttfamily}lc|>{\ttfamily}l>{\ttfamily}lc}
\multicolumn{1}{c}{Design Option}&\multicolumn{1}{c}{width $\times$ height}&\multicolumn{1}{c|}{AR}&%
\multicolumn{1}{c}{Design Option}&\multicolumn{1}{c}{width $\times$ height}&\multicolumn{1}{c}{AR}\\\hline
designi &  {4.67in} $\times$ {3.736in}&5:4&designv &{6in} $\times$ {4.5in}&4:3\\
designii &{5in} $\times$ {4.5in}&10:9&designvi &{4.67in} $\times$ {4.17in}&28:25\\
designiii &{6in} $\times$ {5in}&6:5&designvii &{10in} $\times$ {7.5in}&4:3\\
designiv &{5in} $\times$ {4in}&5:4&designviii&{6.67in} $\times$ {3.75in}&16:9
\end{tabular}
\endgroup
\newtopic Most of these design values were chosen for their visual appeal, or for some particular need.
The designs \texttt{designv} and \texttt{designvii} have an aspect ratio of 4:3, this is a standard for video displays,
while \texttt{designviii} is another one with a standard aspect ratio for wide screen computer monitors.

\newtopic To select a particular design for your document, simply include its name amongst the option
list for the web package; for example,
\begin{sverbatim}
    \usepackage[designv,pdftex]{web}
\end{sverbatim}
\noindent creates a document with dimensions of \texttt{designv},  $6\,\text{in} \times 4.5\,\text{in}$.

\redpoint When you specify a screen design, the macros
\cs{screensize} and \cs{margins} are redefined to gobble up
their parameters. To define a custom screen size, therefore, do
not specify a screen design option for \textsf{web}.

\subsection{\texorpdfstring{\protect\cs{setScreensizeFromGraphic}}{\CMD{setScreensizeFromGraphic}}}\label{setScreensizeFromGraphic}

If a design is \emph{not specified} in the option list of \Web, you
can use a graphic to set the screen size with
\Com{setScreensizeFromGraphic}. The command takes two parameters
that correspond to the parameters of \cs{includegraphics}, and are
just passed to \cs{includegraphics}. This is useful if you have a
graphic to be used for a template but cannot be deformed to fit one
of the standard designs. The solution is to create a screen size
matching the graphic, then using that graphic as a template. Example
usage,
\begin{verbatim}
    \margins{1in}{1in}{24pt}{.25in}
    \setScreensizeFromGraphic{acro_30}
    \template{acro_30}
\end{verbatim}
It may become necessary with some graphics to use the
\texttt{hiresbb} option for the command \cs{includegraphics}, both in
\cs{setScreensizeFromGraphic} and \cs{template} like so,
\begin{verbatim}
    \margins{1in}{1in}{24pt}{.25in}
    \setScreensizeFromGraphic[hiresbb]{acro_30}
    \template[hiresbb]{acro_30}
\end{verbatim}
\newtopic This command should not be used with a panel option for the text screen
size will be adjusted for the panel, and the graphic will be
rescaled appropriately.


\subsection{Using a panel option}

When the \texttt{rightpanel} or \texttt{leftpanel} option is used
the page is divided into the text screen and a panel (either on the
left or right of the page).  The default width of the panel is 1in
and the minimum width of the panel is 1in.

When designing the screen using \cs{screensize} and \cs{margins},
the panel width of the panel can be set by the first optional
parameter of \cs{screensize}, as described in
section~\ref{cusDesign}, page~\pageref*{cusDesign}. When using one
of the standard design options, the panel width is set to the
default value of 1in. To set the panel width to another value, use
\cs{panelwidth} in the preamble. \verb!\panelwidth{1.25in}! sets the
panel width to 1.25 inches.

If you set the panel width to a length less than the minimal panel
width, the minimal panel width (1in) will be used. You can reset the
minimal panel width using \Com{minPanelWidth}. See
section~\ref{panelscreentemplate},
page~\pageref*{panelscreentemplate}, for more details on this command.

\section{Hyperref Options}

The \textsf{web} package loads \textsf{hyperref} into the document
and sets some selected options of that package;  therefore,
including the \textsf{hyperref} package is not needed in the
preamble of your own document.

Any additional \textsf{hyperref} options that are needed can be
introduced into the package using \textsf{hyperref}'s
\cs{hypersetup} macro, for example,
\begin{sverbatim}
\documentclass{article}
\usepackage[dvipsone]{web}   % or dvips or pdftex

% Declare additional hyperref options using \hypersetup
\hypersetup{pdfpagemode=None,bookmarksopen=false}
\end{sverbatim}

\noindent Documentation of the options that \textsf{hyperref}
recognizes can be had by either \LaTeX ing the file
\texttt{hyperref.dtx}, or by getting a copy of the \textsl{The
\LaTeX\ Web Companion} \cite{book:Goossens3} by Michel Goossens,
{\textsl{et al}}.

\section{Running Headers and Footers}

There are convenience commands setting  the running headers and
footers; they are \cs{lheader}, \cs{cheader}, \cs{rheader},
\cs{lfooter}, \cs{cfooter}, and \cs{rfooter}. Each of these takes a
single argument, which is the text for the header or footer in the
position desired. In the  standard setup for {\Web}, \cs{lheader},
\cs{rheader} and \cs{cfooter} are used for, respectively, the
current section title, the current page and the navigation bar (if
activated).

You can set some basic formatting styles for the header and footer using
\Com{header\-format} and \cs{footerformat}. For example, executing
\begin{verbatim}
    \headerformat{\bfseries\color{red}}
\end{verbatim}
causes the running header to appear in bold red color.

There are a couple of hooks for the header and footers, these are
\Com{webheadwrapper} and \Com{webfootwrapper}. These commands can be
use to stylize the entire header or footer. They can be defined to have
one argument. The argument is the content of the header or footer. For example,
\begin{Verbatim}[numbers=left,xleftmargin=20pt,fontsize=\small]
\headerformat{\bfseries\color{red}}
\footerformat{\bfseries\color{red}}
\renewcommand{\webheadwrapper}[1]{\fcolorbox{red}{yellow}{%
    \makebox[\linewidth-2\fboxsep-2\fboxrule][s]{#1}}}
\renewcommand{\webfootwrapper}[1]{\fcolorbox{red}{yellow}{%
    \makebox[\linewidth-2\fboxsep-2\fboxrule][s]{#1}}}
\end{Verbatim}
\newtopic Lines~(1) and~(2) makes declares bold font for the header and footer.
Lines~(3) and~(4) defines a \cs{fcolorbox} with hideous red and yellow colors;
lines~(5) and~(6). Note that adjustment in the width to account for the \cs{fcolorbox}
rule width and the space surrounding the text in the \cs{fcolorbox}.

\newtopic On occasion you may want to have no running headers at all, this can
be accomplished by saying in the preamble, or anywhere,
\verb!\lheader{}!, \verb!\cheader{}!, and/or \verb!\rheader{}!.
Another solution is to use the commands \cs{clearHeaders} and
\cs{restoreHeaders}. The command \cs{clearHeaders} first saves the current
definitions of the headers then sets the headers to the empty
header. The command \cs{restoreHeaders} restores the definitions that were in
effect when the last \cs{clearHeaders} was executed.

\newtopic
One user did not like a running header that includes a section title
on the same page where the section begins. The default behavior is
to create a running header where the section title does not appear
in the header on the same page as the section is inserted, on
subsequent pages, the section head appears, I hope.

The two commands control this behavior, the default is \emph{to not
show} the section heading (\cs{lheader}) on the page where the
section begins. Use \Com{headersOnSectionPage} to place headers on
the same page as the beginning of a section. You can also change back to
the default behavior using \Com{noHeadersOnSectionPage} (the default).


\section{The Title Page and TOC}

The title page is constructed from the values of the macros:
\cs{title}, \cs{author}, \cs{uni\-ver\-sity}, \cs{email}, and
\cs{version}.  The values of some of the macros \cs{title} and
\cs{author} are also transferred to the PDFDocInfo section of the
Acrobat/Adobe Reader. (Press \texttt{Ctrl+D} while viewing a \PDF{} document
to see this information.)

Additionally, the values of \cs{subject} and \cs{keywords}---which
do not appear on the title page---are inserted into the PDFDocInfo
section.

\newtopic The colors of the text corresponding to \cs{title}, \cs{author} and \cs{university} on the title
page can be set by using \cs{titleColor}, \cs{authorColor} and \cs{universityColor}.
Each takes a single argument, a named color. The defaults are
\begin{sverbatim}
\universityColor{blue}
\titleColor{black}
\authorColor{black}
\end{sverbatim}
For more information on defining named \textsf{colors}, see the documentation
of the color or \textsf{xcolor} packages. A simple example would be
\begin{sverbatim}
\definecolor{MyOwnColor}{rgb}{0.945,0.87,0.518}
\end{sverbatim}

\redpoint When the \texttt{pro} option is used, all colors can be specified through the \cs{selectColors}
command, see \Nameref{designColor}.

\subsection{Basic Information Commands}\label{ss:BIM}

The basic information commands allows you a convenient way to enter the title and author
of the document, as well as other information. To enter this basic information,
just fill in the values of all the basic macros briefly described below.

For example, the following is a copy of the title information for this document:
\begin{sverbatim}
% \title,\author,\subject,\keywords are sent to DocInfo
\title{The Web and Exerquiz Packages Manual of Usage}
\author{D. P. Story}
\subject{How to create on-line exercises and quizzes}
\keywords{LaTeX,hyperref,PDF,exercises,quizzes}
\end{sverbatim}

\begin{sverbatim}
% \university,\email,\version are used only on title page
\university{Northwest Florida State College\\
   Department of Mathematics}
\email{dpstory@acrotex.net}
\version{1.0}
\copyrightyears{1999-2010}
\end{sverbatim}
You can optionally specify the \cs{date}. The web packages uses the value of \cs{date}
at the bottom of the title page. There is says, ``Last Revision Date: <date>''. If
the \cs{date} command is not used, the current date is used.

\redpoint The \cs{title}, \cs{author}, \cs{subject}, \cs{keywords}
are a convenient way of entering information in the Document
Information fields---see\parss
%
\texttt{File > Document Info > General... (Ctrl+D)}\parss
%
\noindent in the Acrobat/Adobe Reader.

If \cs{title} contains control sequences that do not expand to the
Standard PDFDocEncoding character set, Distiller will be thrown
into a tailspin; \textsf{hyperref} defines the \cs{texorpdfstring}
macro\footnote{The code for handling PDFDocEncoding for
\textsf{hyperref} is due to Heiko Oberdiek} to avoid these kinds
of problems.  For example,
\begin{sverbatim}
\title{The \texorpdfstring{$e^x$}{EXP} Function}
\end{sverbatim}
\noindent The first argument is the one that is typeset (on the title page,
the title of the document will be `The $e^x$ Function'); the
second argument is the one that is sent to the title field of
DocInfo in the Adobe Reader (and will read `The EXP Function').

Of all the \nameref{ss:BIM}, as listed on page~\pageref*{ss:BIM},
use \cs{texorpdfstring} only with the \cs{title} and \cs{author};
these are the only two that are both typeset and placed in the
DocInfo field of the Adobe Reader.

\redpoint \cs{texorpdfstring} works for \cs{section},
\cs{subsection}, etc\@. as well.\par\medskip

\newtopic Having entered the values of the basic information
commands, you can now type the standard sort of \LaTeX\ commands of
\cs{maketitle} and \cs{tableofcontents}:
\begin{sverbatim}
\begin{document}
\maketitle
\tableofcontents
...
...
\end{document}
\end{sverbatim}

\exPDF{webeqtst} Use the file \texttt{webeqtst.tex}, which comes with the
distribution, as a prototype or template for your own document.

\redpoint When the \texttt{pro} option is in effect, the basic document
information just described can be entered through the
\cs{DeclareDocInfo} command, as described on page~\pageref{docinfo}.

\subsection{Title Page Structure}

The title page is divided into three parts: top, middle and bottom.
\begin{itemize}
\item \textbf{Title Top:} The content of the top is determined by the \Com{top\-Titl\-ePage}
command. (This command can be redefined, but it is not recommended.) The
\cs{topTitlePage} command expands to three elements: the university (affiliation), the title,
and the author(s), in that order vertically. These are the values of the commands \cs{university},
\cs{title} and \cs{author} discussed in the previous section.

\item \textbf{Title Middle:} The \Com{optionalPageMatter} command
is used to enter content into this part of the title page. This
middle part is optional; if \cs{optionalPageMatter} does
not appear in the preamble, this part of the title page is
empty. Here is an example of usage:
\begin{sverbatim}
\optionalPageMatter{\vfill
    \begin{center}
    \fcolorbox{blue}{webyellow}{
    \begin{minipage}{.67\linewidth}
    \noindent\textcolor{red}{\textbf{Abstract:}} This
    file attempts to teach you how to create a simple
    \LaTeX\ document.
    \end{minipage}}
    \end{center}}
\end{sverbatim}
The above definition will create the framed box seen below.

    \begin{center}
    \fcolorbox{blue}{webyellow}{
    \begin{minipage}{.67\linewidth}
    \noindent\textcolor{red}{\textbf{Abstract:}} This file
    attempts to teach you how to create a simple \LaTeX\ document.
    \end{minipage}}
    \end{center}

The \cs{optionalPageMatter} appears just below the \cs{webauthor}
and above the directory listing if there is any.

\item \textbf{Title Bottom:} Bottom of the title page is controlled
by the contents of the command \cs{titlepageTrailer} and consists of
the values of the commands \cs{email}, \cs{version}, \texttt{copyrightyears}
Below is a rough figure depicting the location of the elements found
in the title page trailer.

{\setlength{\tabcolsep}{0pt}
\begin{tabular*}{\linewidth}{@{\extracolsep{\fill}}lr}
\texttt{<copyright notice>}  &  \cs{thewebemail}\\
\texttt{<revision date>}     &  \cs{webversion}
\end{tabular*}}
\newtopic The definition of these elements are as follows:
\begin{sverbatim}
\def\maketitle@trailer@ul{\web@copyright\ \web@copyright@symbol\
    \webcopyrightyears}
\def\maketitle@trailer@ll{\web@revision\ \@date}
\def\maketitle@trailer@ur{\thewebemail}
\def\maketitle@trailer@lr{\webversion}
\end{sverbatim}

\begin{description}
%
% ?? The ul corner not accurate, need to address this.
%
\item The \texttt{ul} corner: The macro \cs{web@copyright} expands to `Copyright', it can be redefined, or removed by executing \Com{nocopyright}. \cs{webcopyrightyears} expands to the argument of \cs{copyrightyears}.

%the final element in the upper left is the email address.
%To make everything go away except for the email address, executed \Com{nocopyrightNotice}.

\item The \texttt{ll} corner: The \texttt{<revision>} expands to \cs{web@revision} followed by \cs{@date}. The \cs{web@revision} expands to ``Last Revision Date:'', and can be changed by using \cs{revisionLabel} (for example \verb!\revisionLabel{Published:}!). The \cs{@date} expands to the argument of the \cs{date} command, or if this was  not used, expands to the current date.

\item The \texttt{ur} corner: The upper-right corner goes the email address, as enter through
the \cs{email} command.

\item The \texttt{lr} corner: The \cs{webversion} command goes into the lower-right corner. The command expands to
\verb!\web@versionlabel\ <version_number>!. The command \cs{web@versionlabel} expands to ``Version''. This label
can be changed by using \Com{versionLabel} (for example, \verb!versionLabel{Attempt}!). Finally,
the variable \texttt{<version\_number>} is the number entered through the \cs{version} command.
\end{description}
\end{itemize}

\newtopic The font size of the title page trailer
 is set by the command \cs{trailerFontSize}, the default is
\cs{footnotesize}. This command must be re-defined in the usual
way: for example, \verb!\renewcommand{\trailerFontSize}{\small}!.


\redpoint When using the pro option, the elements of the title page trailer
can be entered by using \Com{DesignTitlePageTrailer}, see
\Nameref{trailer}.

\subsection{Redefining
\texorpdfstring{\protect\cs{maketitle}}{\textbackslash maketitle}}

The arguments of the \Nameref{ss:BIM} macros, as just discussed,
are used to define text macros with no parameters; for example,
when you type \verb+\title{Web Package}+, the macro \cs{title}
takes its argument and defines a macro \cs{webtitle} that expands
to `Web Package'.

You can redesign the title page to suit your needs simply by
redefining \cs{maketitle}: rearrange the macros listed in the
second column of \hyperref[table:BIM]{Table~\ref*{table:BIM}} on
the page, or include a graphic, or change the background color.
To redefine \cs{maketitle}, use the commands:
\begin{sverbatim}
    \renewcommand\maketitle{...your design...}
\end{sverbatim}
\noindent See the definition of \cs{maketitle} in the \texttt{web.sty} file
for an example.

\begin{table}[hbt]
\begin{center}
\begin{tabular}{|l|l|}
\hline
\multicolumn1{|l|}{This macro}   &
\multicolumn1{l|}{defines this macro} \\
\hline
\cs{title}          &   \cs{webtitle}  \\
\cs{author}         &   \cs{webauthor}  \\
\cs{subject}        &   \cs{websubject} \\
\cs{keywords}       &   \cs{webkeywords}\\
\cs{university}     &   \cs{webuniversity} \\
\cs{email}          &   \cs{webemail} \\
\cs{version}        &   \cs{webversion} \\
\cs{copyrightyears} &   \cs{webcopyrightyears} \\
\hline
\end{tabular}
\caption{The Basic Information Macros}\label{table:BIM}
\end{center}
\vspace{-\baselineskip}
\end{table}

When making the design, it is useful to know that the {\Web}
package uses the command \cs{hypertarget} to create a named destination,
`\texttt{webtoc}', in the table of contents.  Use this
\texttt{webtoc} to jump to the table of contents using
the macro \cs{hyperlink}.

\redpoint When the \texttt{pro} option is in effect, a slight redefinition
of \cs{maketitle} occurs. See the details in section~\ref{maketitlepro},
\Nameref{maketitlepro}.

\subsection{The Title Page Directory}\label{titlepagedir}

When the \texttt{usedirectory} option is taken, a directory listing
appears on the title page. The default structure of the directory is

\makeatletter
    \vbox{\begin{flushleft}{\web@formatWordDirectory\web@directory}%
    \vspace{-3pt}%
    \begin{itemize}\setlength{\itemsep}{-3pt}\web@formatDirectoryItems
        \if\web@removeDirTOC y\else\item\relax\dirTOCItem\fi
        \if\web@removeDirArticle y\else\item\relax\dirArticleItem\fi
        \web@addtoDirList
    \end{itemize}
    \end{flushleft}}
\makeatother
There are several commands that can be used to modify the directory.

\begin{dCmd}{.85\linewidth}
\directoryName{Directory}
\tocName{Table of Contents}
\dirContentLink{Begin \hyperlink{\web@Start.1}{Article}}
\end{dCmd}
\newtopic These three define the default text strings
that appear. In the case of \cs{dirContentLink},
there is a built-in \cs{hyperlink} to the start of
the article.

\begin{dCmd}{.85\linewidth}
\formatWordDirectory{\bfseries\large}
\formatDirectoryItems{\bfseries}
\end{dCmd}
\newtopic The formatting of the \cs{directoryName} can be set by \cs{formatWordDirectory},
the default definition is given above.  The (global) formatting for the listing of the
items can be set using \cs{formatDirectoryItems}, the default is shown above.

\begin{dCmd}{.5\linewidth}
\removeDirTOC
\removeDirArticle
\end{dCmd}
\newtopic These two commands remove the ``Table of Contents'' and
the ``Begin Article'' items from the list, respectively.


\begin{dCmd}{.5\linewidth}
\addtoDirList{<new item>}
\dirTOCItem
\dirArticleItem
\end{dCmd}
\newtopic To add a new item to the end of the directory list, use \cs{addtoDirList}, the argument
is the new item. The other two commands \cs{dirTOCItem} and \cs{dirArticleItem} are commands
representing the first two default items. For example, typing
\begin{verbatim}
    \addtoDirList{My home page is \url{www.example.com}}
\end{verbatim}
in the preamble, adds this item to the end of the list. Should you want your home page listed first, you could do the following

\begin{Verbatim}
\removeDirTOC
\removeDirArticle
\addtoDirList{My home page is \url{www.example.com}}
\addtoDirList{\dirTOCItem}
\addtoDirList{\dirArticleItem}
\end{Verbatim}

\newtopic The following are for creating some special effects around the directory,
such as putting the directory into a multicolumn format.

\begin{dCmd}{.5\linewidth}
\priorDirMatter{<text/commands>}
\afterDirMatter{<text/commands>}
\priorDirList{<text/commands>}
\afterDirList{<text/commands>}
\end{dCmd}

Below is a partial listing of the definition of \cs{webdirectory}, the internal name
for the directory. We present the code so you can see where the prior/after commands
write to.

\begin{Verbatim}[xleftmargin=20pt]
\web@priorDirMatter
\begin{flushleft}{\web@formatWordDirectory\web@directory}%
\web@priorDirList
\vspace{\vspaceAfterDirName}%
\begin{itemize}
<directory listing>
\end{itemize}
\web@afterDirList
\end{flushleft}
\web@afterDirMatter
\end{Verbatim}
\newtopic Contained within the code is another command, \Com{vspaceAfterDirName}, not mentioned before.
It is used to adjust vertical spacing after the directory name and
the beginning of the itemize list.

A final command that may be useful is \Com{directoryhook}. It is used for manipulating
the directory as a whole.

\begin{dCmd}{.5\linewidth}
\directoryhook{
    <code that manipulates #1>
}
\end{dCmd}
\newtopic The argument of \cs{directoryhook} is some \LaTeX{} code that manipulates \texttt{\#1}, within
the argument \texttt{\#1} refers to \cs{webdirectory}, the internal name of the directory.
The following example places a colored box around the directory, and centers it on the title page.

\begin{Verbatim}[xleftmargin=20pt]
\directoryhook{%
    \begin{center}
    \setlength{\topsep}{0pt}%
    \fcolorbox{red}{webyellow}{%
    \parbox{.4\linewidth}{#1}}%
    \end{center}
}
\end{Verbatim}

\exAeBBlogPDF{multicol_dir} See the file
\href{http://www.math.uakron.edu/~dpstory/aebblog/multi_dir.pdf}{multicol\_dir.pdf}
for examples of many of these commands. This file is posted on the
\mlhref{http://www.math.uakron.edu/~dpstory/aebblog.html}{AeB Blog}.


%\directoryhook


\subsection{The TOC for Web}

The \textsf{Web} style comes with its own table of contents format,
as seen in the table of contents for the screen version of this document.
The amount of indentation can be adjusted using \cs{tocindent}. The default is
\begin{sverbatim}
\tocindent{20pt}
\end{sverbatim}
\noindent There is another relevant parameter, \cs{widestNumber}. The value of the argument
of this command sets the amount of indentation for the subsection numbers. The default
is
\begin{sverbatim}
\widestNumber{0.0.}
\end{sverbatim}
\noindent This is a template for the subsection numbers, the
default is a one digit section number and a one digit subsection
number. In the preamble of this document, I've set
\verb+\widestNumber{0.00.}+, since some subsection numbers have
two digits.

\newtopic The color of the heading of the table of contents is set through
\begin{dCmd}{.5\linewidth}
\tocColor{<named_color>}
\end{dCmd}
Specifying a color through this command at any time before the creation of the table of contents
will change the color of the heading. The default is blue.


\redpoint If you prefer the standard {\LaTeX}, the
\texttt{\hyperref[sss:latextoc]{latextoc}} option can be used.

\subsection{The
\texorpdfstring{\protect\texttt{usedirectory}}{usedirectory}
Option}\label{sss:nodir}

When the \texttt{usedirectory} option is specified, a short
directory appears on the first page, see \Nameref{titlepagedir} for
more details.

\redpoint \textbf{\textcolor{red}{Important Note:}} In previous versions
of {\Web}, the directory automatically appeared, and there was a
\texttt{nodirectory} option to make it go away. Now, the directory
does not appear by default, and the option \texttt{usedirectory}
makes it appear. The \texttt{nodirectory} option is still defined,
but does nothing.


\subsection{The \texorpdfstring{\protect\texttt{latextoc}}{latextoc}
Option}\label{sss:latextoc}

If you don't like the default design for the table of contents,
you can always recover the standard \LaTeX\ table of contents by
using the \texttt{latextoc} option with the \textsf{web}
package:
\begin{sverbatim}
\usepackage[latextoc]{web}
\end{sverbatim}
\noindent Should you want to go with this option, you might consider
including
\begin{sverbatim}
\hypersetup{linktocpage}
\end{sverbatim}
\noindent Look at the table of contents with and without this
\textsf{hyperref} option to decide which you prefer.

\subsection{The \texttt{centertitlepage} Option}\label{centertitle}

Beginning with version 5.0, there is a \texttt{centertitlepage} option
that attempts to center the title page better when the \texttt{forpaper} option
is taken. Seems to work best with \texttt{designv}. Will try to improve.

\subsection{The \texorpdfstring{\protect\cs{makeinlinetitle}}{\CMD{makeinlinetitle}} Command}\label{inlinetitle}
    \aebCng[makeinlinetitle]

For some short documents,\footnote{Short documents such as white
papers, homework assignments, for example.} a formal title page may
not be needed or desirable. In this case, use \cs{makeinlinetitle},
the ``in-line'' title.

\begin{dCmd}[\sCom{makeinlinetitle}]{.5\linewidth}
\makeinlinetitle
\end{dCmd}

\newtopic The command, which can be redefined to your needs, has the following design:
\begin{center}\setlength{\tabcolsep}{0pt}
\begin{tabular*}{\linewidth}{@{\extracolsep{\fill}}lcr}
                    & \cs{webtitle}  &\\
\cs{webuniversity}  & \cs{webauthor} & \cs{webemail}\\
\texttt{date}       &                & \cs{webversion}
\end{tabular*}
\end{center}
The table above extends the entire \cs{linewidth}. The values of \cs{webtitle},
\cs{web\-uni\-ver\-sity}, \cs{webauthor}, \cs{webmail} and \cs{web\-ver\-sion}
are populated by their command counterparts, see \hyperref[table:BIM]
{Table~\ref*{table:BIM}}, page~\pageref*{table:BIM}. The \texttt{date},
as appears in the lower left row, will expand to the current date,
or the value specified by the \cs{date} command.

\newtopic Below are notes in the case the \texttt{pro} option is in effect:
\begin{itemize}
    \item The basic information above can be entered through
    \cs{DeclareDocInfo} command as described in
    section~\ref*{docinfo}, entitled \Nameref{docinfo}.
    \item If the value of the \texttt{prepared} key is not specified, then
    \texttt{date} is the current date, otherwise, \texttt{date} is the one specified
    as the value of the \texttt{prepared} key.
    \item If the value of the \texttt{talksite} key is not specified, then
    \cs{webversion} is used; but if  \texttt{talksite} is specified, the value of this
    key is used instead of \cs{webversion}.
\end{itemize}

\section{Template Options}\label{ss:templates}

The \textsf{Web} Package has three options (and supporting
commands) for creating colored backgrounds, graphics backgrounds,
and various overlays.

\subsection{The \texttt{usetemplates} Option}

The \texttt{usetemplates} option activates the mechanism for
creating colored backgrounds and graphic overlays. A complete
discussion of the commands related to this option can be found
in the section entitled \Nameref{ss:templagemagement}.

\redpoint See the demo file \texttt{bgtest.tex} for examples.

\subsection{The \texttt{leftpanel} and \texttt{right\-panel} Options}\label{ss:lrpanel}

When either of these two options is specified, a vertical panel
is created. See the section entitled \Nameref{ss:templagemagement}
for a complete discussion of the commands related to these options.

\redpoint See the demo file \texttt{bgtest.tex} for examples.

\subsection{The \texttt{leftpanelprt} and \texttt{right\-panelprt} Options}\label{ss:lrpanelprt}

These two options are the same as \texttt{leftpanel} and
\texttt{right\-panel} \emph{when there is no paper option} (\texttt{forpaper},
\texttt{forcolorpaper}). When a paper option is taken, these options do
nothing. The options are useful for creating a paneled PDF for the
screen, but the paper version that has no panel.

\section{Navigation Aids}\label{ss:naviaids}

The \textsf{web} package offers a couple of navigation aids to
help you move around: the \texttt{navibar} Option, and some
\hyperref[sss:diricons]{direction icons}.

\subsection{A Navigation Bar}\label{sss:navibar}

Use the \texttt{navibar} option of \textsf{web} to add a
navigation toolbar, as seen at the bottom of this page.
Usage:
\begin{sverbatim}
   \usepackage[<driver_option>,navibar]{web}
\end{sverbatim}
\noindent The result is the navigation bar you see at the bottom of the
page.

\NaviBarOn\AddToShipoutPicture*{\NaviBarOff}

\begin{comment}
\redpoint Here is (almost) the preamble of this document.

\begin{sverbatim}
\documentclass{article}
\usepackage[dvipsone,navibar]{web}  % <- navibar option

\hypersetup{pdfpagemode=None,bookmarksopen=false}

\title{The Web and Exerquiz Packages Manual of Usage}
\author{D. P. Story}
\subject{How to create on-line exercises and quizzes}
\keywords{LaTeX, hyperref, PDF, exercises, quizzes}
\version{1.30}\copyrightyears{1999-2000}
\email{dpstory@uakron.edu}
\university{THE UNIVERSITY OF AKRON\\
   Mathematics and Computer Science}
\end{sverbatim}
\end{comment}

\redpoint The toolbar can be turned on or off by the following
commands: \cs{Navi\-Bar\-On}\sCom{NaviBarOn} and
\cs{Navi\-Bar\-Off}\sCom{NaviBarOff}. The navigation toolbar at
the bottom of the page was generated by the \cs{NaviBarOn}.
\cs{NaviBarOff} was placed on the next page to turn off the bar. The
exact code used on this page is
\begin{sverbatim}
\NaviBarOn\AddToShipoutPicture*{\NaviBarOff}
\end{sverbatim}

\newtopic The colors for the navigation bar can be set with commands
\Com{navibarTextColor} and \Com{navibarBgColor}. Each take a single
argument, a named color. The default colors are \texttt{webblue} and
\texttt{webgray}, respectively.


\subsection{\texorpdfstring{\protect\cs{newNaviIcon}}
    {\textbackslash newNaviIcon}}\label{ss:newnaviicon}

The \cs{newNaviIcon} can be used to define a navigation icon.
The action of the icon can be to execute a menu item, perform a
hyper-jump, or execute the JavaScript code. It takes six
parameters:

\medskip\noindent Parameters
\begin{sverbatim}
   #1 = m, j, or l
   #2 = command name of the new navigation icon
   #3 = width of icon
   #4 = height of icon
   #5 = text to appear in the center of the icon.
   #6 = if m: named menu action, e.g., NextPage, PrevPage, etc.
        if j: execute JavaScript
        if l: \hyperlink{arg} or \href{arg}
\end{sverbatim}
\noindent Once the \cs{newNaviIcon} command is executed, a new icon is
defined. The name of this new icon is the value of parameter \#2.

\redpoint \textcolor{blue}{Example:}
\begin{sverbatim}
\newNaviIcon[m]{\myNext}{34pt}{10pt}{Next}{NextPage}
\newNaviIcon[j]{\jsWarning}{34pt}{10pt}{Hi}{app.alert("Hi there")}
\newNaviIcon[l]{\linkJump}{34pt}{10pt}{Go}{\hyperlink{page.1}}
\end{sverbatim}

    \newNaviIcon[m]{\myNext}{34pt}{10pt}{Next}{NextPage}\tracingmacros=0
    \newNaviIcon[j]{\jsWarning}{34pt}{10pt}{Hi}{app.alert("Hi there")}
    \newNaviIcon[l]{\linkJump}{34pt}{10pt}{Go}{\hyperlink{page.1}}

\noindent By typing \verb+\myNext \ \jsWarning\ \linkJump+, we get
$$
    \myNext \ \jsWarning\ \linkJump
$$
The colors for the navigation bar can be set with commands
\cs{navibarTextColor} and \cs{navibarBgColor}.

\subsection{Direction Icons}\label{sss:diricons}

The up arrow you see in the upper right-hand corner was
constructed using colored rules and the AMS symbol font,
\texttt{amssymb}. The uparrow icon was produced by the command:
\begin{sverbatim}
\insertnaviiconhere{\ArrowUp{\hyperlink{webtoc}}}
\end{sverbatim}
The definition of \cs{ArrowUp}, which be found in the \textsf{web.dtx} file is
\begin{sverbatim}
\newcommand\ArrowUp[1]
{%
    \setlength{\fboxsep}{6pt}\normalsize
    \raisebox{-\depth}[0pt][0pt]{%
    #1{\web@colorbox@w@transparency{\web@directionIconBgColor}%
    {\textcolor{\web@directionIconTextColor}{$\bigl\Uparrow$}}}}%
}
\end{sverbatim}

\noindent Or, more generally,
\begin{sverbatim}
\insertnaviiconhere{\ArrowUp{link_command}}
\insertnaviiconhere{\ArrowDown{link_command}}
\end{sverbatim}
\noindent This will insert direction icons on the current page (I
hope).

\insertnaviiconhere{\ArrowUp{\hyperlink{toc.1}}}

\medskip

If you want a running direction icon you can use
\begin{sverbatim}
\insertnaviiconhereafter{\ArrowUp{link_command}}
\end{sverbatim}
\noindent or
\begin{sverbatim}
\insertnaviiconhereafter{\ArrowDown{link_command}}
\end{sverbatim}


\redpoint To discontinue a running arrow icon type\parss
\verb+\defaultpageheader+\parss\noindent
on the page you want the arrow(s) to disappear.

\newtopic The colors of the direction icons by using
the commands \Com{directionIconTextColor} and
\Com{directionIconBgColor}. Each command takes a named color as its argument;
the default values are \texttt{webblue} and \texttt{webgray}, respectively.

\subsection{\texorpdfstring{\protect\cs{panelNaviGroup}}
    {\textbackslash panelNaviGroup}}\label{ss:panelnavigroup}

When the \texttt{\hyperref[ss:lrpanel]{leftpanel}} or
\texttt{\hyperref[ss:lrpanel]{rightpanel}} options are chosen, a
(navigation) panel is created. The command \cs{panelNaviGroup}
can be used to create the standard navigation panel.

\exPDF{bgtest} See the sample file \texttt{bgtest.tex} for an example of usage.


\section{The Language Options}\label{web:langopts}

The language options redefine all of the language dependent text
macros that appear on the title page, in the table of contents, and
in the running headers. Invoke these options in the usual
way:
\begin{sverbatim}
   \usepackage[<driver_opt>,<lang_opt>]{web}
\end{sverbatim}
\noindent Here, \texttt{<lang\_opt>} is one of the following values:
\texttt{dutch}, \texttt{french}, \texttt{german}, \texttt{italian},
\texttt{norsk}, \texttt{russian}, \texttt{spanish}, \texttt{polish}
\texttt{finnish},  \texttt{czech}, \texttt{catalan},
\texttt{brazil}, and \texttt{turkish}.

The \textsf{web} and \textsf{exerquiz} packages seem to be
compatible with the \textsf{babel} package;  you can use
\begin{sverbatim}
\documentclass{article}
\usepackage[french]{babel}
\usepackage[dvips,french]{web}
\usepackage{exerquiz}
\end{sverbatim}
\noindent subject to the usual restrictions on these language packages.
(Don't use characters declared active by these languages within a
\cs{label}, or as a field name for a quiz.


The translations for the \texttt{french} option is due to the
tremendous efforts of Jean-Michel Sarlat, and Michael Wiedmann did
the translations for the \texttt{german} option.

\section{Paper Related Options and Commands}\label{paperoptions}

In this section, several paper options are discussed, and one command.

\subsection{The
\texorpdfstring{\protect\texttt{forpaper}}{forpaper}
Option}\label{sss:webforpaper}

Some people may want to create exercises using the
\texttt{exercise} environment for a paper document. The
\texttt{forpaper} option can be used to remove the color from the
document, and to restore the standard \cs{textheight} of a standard
\texttt{article} class \LaTeX{} document.

The \cs{textwidth} is determined from the \cs{screensize} and
\cs{margins} parameters or by the design option (see
\nameref{sss:designoptions}), \emph{if any are given}; consequently,
the line breaks are the same for the ``web'' version and the
``print'' version. If the screen dimensions are not set (by a design
option or by the \cs{screensize} and \cs{margins} commands) when
using the \texttt{forpaper} option, the standard {\LaTeX} page
layout dimensions are used for the class.

Using the \texttt{forpaper} option with the \texttt{latexlayout}
option will give you the standard \LaTeX{} \cs{textwidth}.

The \texttt{forpaper} option also changes the \cs{newpage} command
to \cs{par}\cs{medskip} at the end of each solution---we don't
want to waste paper now, do we?

Finally, there is a Boolean switch \verb+\ifeqforpaper+, which you
are free to use to refine the look your \texttt{forpaper} version.

\subsection{The
\texorpdfstring{\protect\texttt{forcolorpaper}}{forcolorpaper}
Option}\label{forcolorpaper}

Same as the \hyperref[sss:webforpaper]{\texttt{forpaper}} option,
but the color operators are not turned off.


\subsection{The
\texorpdfstring{\protect\texttt{latexlayout}}{latexlayout}
Option}\label{sss:weblatexlayout}

For those who want to go ``totally native,'' use the
\texttt{latexlayout} and \texttt{forpaper} options together.  When the
\texttt{latexlayout} option is used, the page layout redefinitions of
\textsf{web} are bypassed, leaving the original layout values of the
\texttt{article} class of \LaTeX.

\redpoint If the \texttt{latexlayout} option is taken, all templates
are turned off, and the \texttt{for\-color\-option} is executed. To remove
color, you need to explicitly take the \texttt{forpaper} option.

\subsection{The
\texorpdfstring{\protect\texttt{uselatexparts}}{uselatexparts}
and \texorpdfstring{\protect\texttt{uselatexchapters}}{uselatexchapters}
Options}

As described in \hyperref[partsChaptersDots]{section~\ref*{partsChaptersDots}},
page~\pageref*{partsChaptersDots}, the {\Web} package redefines the \cs{part}
and \cs{chapter} commands for the purpose of allowing easier access to redefining
how these section elements are to appear in the document and in the table of contents.
You can bypass these definitions by using \texttt{uselatexparts} and
the \texttt{uselatexchapters} options. This may be useful if you want to use AeB
to create a paper document and you want to use that standard definitions, or perhaps
want to use another package to define the look of the table of contents or
of the \cs{part} and \cs{chapter} commands. (The other sectioning comments can be overridden
as well, but any controls provided by the \texttt{pro} option will be lost.)

\subsection{The \texorpdfstring{\protect\cs{useFullWidthForPaper}}
    {\CMD{useFullWidthForPaper}}  Command}

We introduce a command of resetting the page layout paper \emph{when
the \emph{\texttt{forpaper}} option is in effect}. The primary use
is to create standard documents for academics such as homeworks,
syllabuses or any handout to the student.

This command is used to set the page layout to its maximum width, given
a 1 inch margin. The height is maximized after taking into account
other page parameters that effect it.  The parameters \cs{marginparwidth}
and \cs{marginparsep} are set to zero; for this layout, there are no
marginal comments. The command may be redefined as desired.

\newtopic The command should be used in the preamble, otherwise, it
has no effect.

\section{Formatting for screen and paper}

As we learned in \hyperref[paperoptions]
{Section~\ref*{paperoptions}}, {\Web} can format a document in a for
screen or for paper. The {\Web} package provides several commands and
environments for changing content as needed.

When the \texttt{forpaper} option is used, the switch
\cs{ifeqforpaper}\sCom{ifeqforpaper} is set to true. You can use
this switch, to get different content for the printed page or the
screen page, or you can use the convenience command \cs{prtscr}:

\begin{dCmd}[\sCom{prtscr}]{.65\linewidth}
\prtscr{<print_text>}{<screen_text>}
\end{dCmd}
\CmdDescription This command expands to \texttt{<print\_text>} if the \texttt{forpaper} option
is taken and to \texttt{<screen\_text>} if not.

The \cs{prtscr} command is useful for small changes in content, not
involving verbatim text or paragraph breaks. {\Web} defines two additional environments,
\texttt{forpaper} and \texttt{forscreen}
\begin{dCmd}[\index{forscreen Environment\AT\texttt{forscreen} Environment}]{.5\linewidth}
\begin{forscreen}
....
....
....
\end{forscreen}
\end{dCmd}

\EnvLoc Anywhere.

\newtopic Use the \texttt{forscreen} environment to insert commands or content meant only for
the screen.

\begin{dCmd}[\index{forpaper Environment\AT\texttt{forpaper} Environment}]{.5\linewidth}
\begin{forpaper}
....
....
....
\end{forpaper}
\end{dCmd}

\EnvLoc Anywhere.

\newtopic Use the \texttt{forpaper} environment to insert commands
or content meant only for the paper document.

\newtopic Another command, \cs{NewPage}, is useful for formatting
the same document for print as well as screen

\begin{dCmd}[\sCom{NewPage}]{.3\linewidth}
\NewPage
\end{dCmd}

\CmdLoc Anywhere.

\CmdDescription This command expands to \cs{newpage} if the
\texttt{forpaper} option is \emph{not taken}, and does nothing,
otherwise. Useful for inserting page breaks in the screen version
that are not needed in the paper version.

\section{Template Building and Management}\label{ss:templagemagement}

The {\Web} package has a template building capability.
You can conveniently create backgrounds for your page, insert an
arbitrary number of graphic overlays, create a left or right
(navigation) side panel, define your own navigation icons that
appear in the panel, and write material that will appear in a
panel.

\exPDF{bgtest} The demo file for the template feature is
\texttt{bgtest.tex}.

\subsection{Template options}

As with pdfscreen by Radhakrishnan C. V., we shall have the two
options, \texttt{leftpanel} and \texttt{rightpanel}. In addition to
these two, there is the \texttt{usetemplates} option.  Use the
option \texttt{usetemplates} if you want to use colored backgrounds
or overlays without a left or right panel.

The template, or overlay, capability of the \textsf{Web Package}
requires the use of two \LaTeX{} Packages: \texttt{everyshi.dtx},
by Martin Schr\"oder, and \texttt{eso-pic.dtx}, by  Rolf
Niepraschk.  If any of the three template options
(\texttt{usetemplates}, \texttt{leftpanel} or
\texttt{rightpanel}) are used, the \textsf{eso-pic package} is
automatically included by \textsf{web}. The \textsf{eso-pic
package}, in turn, inputs the \textsf{everyshi} package. These two
packages need to be present on your system,  unpacked, and in the
search path of \LaTeX.

Templates, or overlays, are available for the \texttt{dvipsone},
\texttt{dvips}, \texttt{pdftex}, and \texttt{dvipdfm} options.

\subsection{Text Screen Template}

You can specify a graphic that will be overlaid onto the text
screen, that portion of the screen to which {\LaTeX} content is
written.  If a panel option has not been specified, this is the
whole screen; otherwise, it is that portion of the screen outside
the panel.

If one of the options \texttt{usetemplates}, \texttt{leftpanel} or
\texttt{rightpanel} is specified, the commands
\begin{sverbatim}
\template{<graphics_file_name>}
\textBgColor{<named_color>}
\end{sverbatim}
\noindent insert a background graphic and a background color,
respectively, onto the text screen region. The \cs{template} command
will rescale the graphic to cover the entire text screen region.

Additional graphics can be overlayed with the \cs{AddToTemplate}
command.
\begin{sverbatim}
    \AddToTemplate{<template_name>}
\end{sverbatim}
\noindent The command takes one argument, the \textit{template\_name}.
Define an overlay,
\begin{sverbatim}
\newcommand\myTemplate
{%
    < commands to insert an overlay >
}
\end{sverbatim}
\noindent the \textit{template\_name} for this template is \texttt{myTemplate}. (Note that there
is no backslash.) To add this template to the list graphics to be overlayed onto the page,
we would type
\begin{sverbatim}
\AddToTemplate{myTemplate}
\end{sverbatim}

\redpoint \textcolor{blue}{Example:} Insert the ``\AcroT'' logo in lower-left corner, offset by~36pt
in the $x$ and $y$ directions.
\begin{sverbatim}
\newcommand\AEBLogo
{%
    \put(36,36){\includegraphics{acrotexlogo}}%
}
\AddToTemplate{AEBLogo}
\end{sverbatim}
\noindent Because the \textsf{Web Package} uses \textsf{eso-pic}, the
commands will be executed within a \texttt{picture} environment.
Within the \texttt{picture} environment, the reference point of the text
screen is the lower-left corner. The above code puts the
``\AcroT'' logo at coordinates of \texttt{(36,36)} relative to the
lower-left corner. The units are measured in (\TeX) points.

\redpoint \textcolor{blue}{Example:} Center the logo within the text screen region.
\begin{sverbatim}
\newcommand\AEBLogoCenter
{%
    \ifnum\arabic{page}>1\relax
        \parbox[b][\paperheight][c]{\textscreenwidth}
        {\centering\includegraphics{acrotexlogo}}%
     \fi
}
\AddToTemplate{AEBLogoCenter}
\end{sverbatim}

\noindent
See the section titled \Nameref{sss:templatemanage} for details of
how to manage your templates.

\subsection{Panel Screen Template}\label{panelscreentemplate}

When the \texttt{leftpanel} or \texttt{rightpanel}
option is specified, a (navigation/logo) panel is created. The commands
\begin{sverbatim}
\paneltemplate{<graphics_file_name>}
\panelBgColor{<named_color>}
\end{sverbatim}
\noindent set the overlay graphic and the background color, respectively.
The graphic is rescaled to fit the panel region.

Once the panel and its background have been defined, contents and
form elements can be placed on top of the panel. The command \cs{buildpanel} can
be used for this purpose. For example, from the sample file
\texttt{bgtest.tex},
\begin{sverbatim}
\buildpanel
{%
    \href{http://www.math.uakron.edu/}
        {\includegraphics[scale=.4]{uakron}}
    \par\vspace{\stretch{1}}
    \href{http://www.math.uakron.edu/~dpstory/acrotex.html}
        {\rotatebox{-90}{\aebLogo}}
    \par\vspace{\stretch{1}}
    \panelNaviGroup             % defined in web
}
\end{sverbatim}
\noindent The content of the panel is stacked from top to bottom.\footnote{The command \cs{aebLogo}
is defined in the file, \texttt{bgtest.tex}.}

\redpoint Additional overlays can be added with
\cs{AddToPanelTemplate}. The command may not be as useful as the panel overlay can
always be rebuilt using \cs{buildpanel}.

\newtopic The minimal width for the panel is set through the            \aebCng[minPanelWidth]
\Com{minPanelWidth} command, the default size is 1in; that is,
\verb!\minPanelWidth{1in}! is executed by the package at start up.

\begin{dCmd}{.5\linewidth}
\minPanelWidth{<length>}
\end{dCmd}

\CmdLoc You can reset the minimal width by executing the
command in the preamble before the \cs{screenwidth} and \cs{margins}
commands. Or, place it in the \texttt{web.cfg}.

\subsection{Template Management}\label{sss:templatemanage}

In order to change backgrounds or templates, on any page, re-issue
any one of the commands \cs{template} or \cs{textBgColor} (for the
screen text region), or \cs{paneltemplate} or \cs{textBgColor}
(for the panel region).

The panel overlay can be redesigned with \cs{buildpanel}, or some of the
command components that make up the panel overlay can be redefined.

Templates that are inserted into the output stream using
either \cs{AddToTemplate} or \cs{AddToPanelTemplate} can also be
redefined on any page.

Templates, created by either \cs{AddToTemplate} or
\cs{AddTo\-Panel\-Template}, can also be \emph{disabled} or \emph{enabled}
individually. For example, if the \texttt{AEBLogoCenter} template
has been overlayed using the command
\begin{sverbatim}
\AddToTemplate{AEBLogoCenter}
\end{sverbatim}
\noindent the template can be disabled (turned off) by typing
\begin{sverbatim}
\disableTemplate{AEBLogoCenter}
\end{sverbatim}
\noindent on any page. (\textcolor{red}{Note:} The effects of this command
may be not be seen until the following page.) Turn the template
on by typing
\begin{sverbatim}
\enableTemplate{AEBLogoCenter}
\end{sverbatim}
\noindent on any page.

For the panel region, there are commands for \emph{disabling} (\cs{disablePanelTemplate}) and \emph{enabling}
(\cs{enablePanelTemplate}) as well. Each of these takes a
\textit{template\_name} as an argument.

There are a number of commands for \emph{clearing} backgrounds and
templates.
\begin{sverbatim}
\ClearTextTemplate
\ClearPanelTemplate
\end{sverbatim}
\noindent These two clear background colors and background graphics.
\begin{sverbatim}
\ClearBuildPanel
\end{sverbatim}
\noindent This command will clear the build panel as well as the graphics
and field elements that lay on top of the panel created by the
\cs{buildpanel} command.
\begin{sverbatim}
\ClearAllTemplates
\end{sverbatim}
\noindent The command \cs{ClearAllTemplates} is equivalent to executing \cs{ClearTextTemplate} and \cs{ClearPanelTemplate}.
\begin{sverbatim}
\ClearTextTemplateBuffer
\ClearPanelTemplateBuffer
\end{sverbatim}
\noindent The commands will clear all overlays, including overlays created  by
\cs{Add\-To\-Template} and \cs{Add\-To\-Panel\-Template}.

\redpoint See the documentation file, \texttt{web.dtx}, for exact definitions of the
commands in this section.

% \subsubsection{Parameters used in Template Building}\label{sss:partemp}

\section{The \texttt{pro} Option}\label{webpro}

Version 5.0 of {\Web} introduces the \texttt{pro}, which
encompasses many new features. The \texttt{pro} option uses the very fine
package \texttt{xkeyval} by Hendri Adriaens, which enables
developers to create commands with key-value pairs.

\newtopic The following new features are provided by the \texttt{pro} option:
\begin{questions}
    \item \cs{DeclareDocInfo}: A data structure for setting various document properties.
    \item A slightly re-designed \cs{maketitle}. Also, increased control over
    the placement of the elements of the title page.
    \item Extensive control over the display of \cs{section}, \cs{subsection},
    \cs{subsubsection} headings. Now color, fonts, size and so one can easily
    be set.
    \item A choice of using section numbers (the default) or no sections, or dings.
    \item A single data structure to set virtually all colors.
\end{questions}
Details provided in subsequent sections.

\subsection{\texorpdfstring{\protect\cs{DeclareDocInfo}}{\CMD{DeclareDocInfo}}}\label{docinfo}

The preamble of your document should contain a number of keys that
identify the document, including the title and author of the
document. Some of this information is used to construct the
title page, some is placed in the {\PDF}, to be displayed in the
`Description' tab of the \textbf{Document Properties} dialog box, which
is accessed through the \texttt{Ctrl+D} accelerator key, or through
the menu system \texttt{File > Document Properties...} (for version 8,
this is \texttt{File > Properties...}).

Information is passed through the \cs{DeclareDocInfo} command which
takes a number of key-value pairs. This is a simple \textsf{xkeyval}
interface to many of the text macros that are defined in \Web.

\begin{dCmd}[\sCom{DeclareDocInfo}]{.65\linewidth}
\DeclareDocInfo{<key-value pairs>}
\end{dCmd}

\CmdLoc Place in the preamble.

\KVP The following is a description of the key value pairs.
\begin{enumerate}
    \item \texttt{title}: The title of the document.
    \item \texttt{author}: The author or authors of the document.
    \item \texttt{subject}: The subject of the presentation. Optional, this appears only in the
    `Description' tab of the \textbf{Document Properties}.
    \item \texttt{keywords}: A list of keywords that describe your document. Optional, this appears only in the
    `Description' tab of the \textbf{Document Properties}. Some search engines use this field.
    \item \texttt{university}: The university or company the author represents.
    \item \texttt{email}: The email address of the author. This appears on the title page, and becomes
    an email link.
    \item \texttt{version}: The version number of the document.
    \item \texttt{versionLabel}: Text that precedes the version number. The default is \texttt{Version}.
    \item \texttt{talkdate}: Date of the presentation.

           The fields \texttt{version} and \texttt{talkdate} occupy the same position on the title page.
           If the \texttt{talkdate} is specified, then the talk date will appear on the title page, of
           \texttt{talkdate} if not specified, the \texttt{version} will appear. Both
           \texttt{version} and \texttt{talkdate} can be specified, but in this case,
           it is \texttt{talkdate} that will appear. The \texttt{version} can be used for version management.

    \item \texttt{talkdateLabel}: Text that precedes the date of the document. The default text for the talkdate label is \texttt{Presented:}.
    \item \texttt{talksite}: Site of the presentation. This field can be used for generally anything, for
    example, you could specify your web site.
    \item \texttt{copyrightyears}: Year(s) of the copyright of this publication, defaults to this year.
    \item \texttt{prepared}: The date of preparation of the
          document, defaults to the day the file was compiled (\LaTeX ed).
          This was formerly the last revision date.
    \item \texttt{preparedLabel}: The label that precedes the date \texttt{prepared}. The default string
            value is \texttt{Prepared:}. The old default was \texttt{Last Revision Date:}, and will remain so
            if the \texttt{pro} option is not taken.
    \item \texttt{copyrightStatus}: If the \textsf{aebxmp} package is loaded for advanced metadata,
        this key allows you to set the copyright status. Possible
        values \texttt{True}, \texttt{False}, or blank (no value, or the key not listed at all)
        corresponding to \textsf{Copyrighted}, \textsf{Public Domain} and
        \textsf{Unknown}, respectively. The values of this key are case sensitive, so you must enter
        \texttt{True} and \texttt{False}, with the first letter capitalized. If \textsf{aebxmp} is
        not loaded, specifying this key does nothing.
    \item \texttt{copyrightNotice}: If the \textsf{aebxmp} package is loaded for advanced metadata,
        this key allows you to set copyright notice, short text describing the copyright, perhaps,
\begin{sverbatim}
    copyrightNotice={Copyright D. P. Story, 2009}
\end{sverbatim}
        If \textsf{aebxmp} is not loaded, specifying this key does nothing.
    \item \texttt{copyrightInfoURL}: If the \textsf{aebxmp} package is loaded for advanced metadata,
        this key allows you to set the copyright info url, a url to
        a page on the web containing a more detailed description of the copyright. If \textsf{aebxmp} is
        not loaded, specifying this key does nothing.
\end{enumerate}

\Ex\label{exDeclareDocInfo} Example of usage of \Com{DeclareDocInfo} can be found
in \hyperref[fig:declaredocinfo]{Figure~\ref*{fig:declaredocinfo}}, page~\pageref*{fig:declaredocinfo}.
\begin{figure}[ht]
\footnotesize
\begin{dCmd*}{.9\linewidth}
\DeclareDocInfo
{
    title=My First Presentation,
    author=D. P. Speaker,
    university=My University,
    email=dpspeaker@myu.edu,
    talkdate={Dec.\ 17, \the\year},
    talksite=The Talking University,
    subject=On the Theory of AcroTeX,
    keywords={LaTeX, PDF, AcroTeX},
    copyrightStatus=False,
    copyrightNotice={Copyright D. P. Story, \the\year},
    copyrightInfoURL=http://www.acrotex.net
}
\end{dCmd*}
\caption{The \protect\cs{DeclareDocInfo} Command}\label{fig:declaredocinfo}
\end{figure}

\handpoint Two tricks of importance: When the value contains a
comma, then the whole value should be delimited by matching braces,
as in the \texttt{talkdate} key-value in the
\textbf{Example~\ref*{exDeclareDocInfo}} above. The \textsf{hyperref}
command \Com{texorpdfstring} is handy for giving alternate wording, when
some of the {\LaTeX} commands do not transfer to the {\PDF}'s \textbf{Document Properties}.
For example, the title might have been ``A Discussion of $e^x$''; this title should appear
in \cs{DeclareDocInfo} as follows:
\begin{Verbatim}
    title=A Discussion of \texorpdfstring{$e^x$}{exp(x)},
\end{Verbatim}
\noindent Now the phrase, ``A Discussion of exp(x)'' will appear in
the `Title' field of the \textbf{Document Properties}.

\exPDF{web_pro} The demo file \texttt{web\_pro} illustrates this command.

\subsection{\texorpdfstring{\protect\cs{DeclarePageLayout}}{\CMD{DeclarePageLayout}}}\label{pagelayout}

The \Com{DeclarePageLayout} command gathers together many page
layout parameters, both for {\LaTeX} and for {\Web}, into one
convenient package of key-value pairs.

\CmdLoc Place in the preamble.

\KVP The following is a description of the key value pairs.
\begin{enumerate}
\item \texttt{design}: This key sets the \texttt{screensize}
    according to preset dimensions, permissible values are
    \texttt{designi}, \texttt{designii}, \texttt{designiii},
    \texttt{designiv}, \texttt{designv}, \texttt{designvi},
    \texttt{designvii}, and \texttt{designviii} These correspond to the
    package options of the same names. This key is ignored if one of the
    design options was taken as a {\Web} option. This key sets both the
    \cs{screensize} and \cs{margins} commands.
\item \texttt{screensizeOf}: This key sets the \texttt{screensize}
    only (does not change the margins), possible values are
    \texttt{designi}, \texttt{designii}, \texttt{designiii},
    \texttt{designiv}, \texttt{designv}, \texttt{designvi},
    \texttt{designvii}, and \texttt{designviii}. Use this key, in
    combination with the \texttt{margins} key, to use a standard design
    \texttt{screensize}, but with different margins.
\item \texttt{screensize}: Sets the \texttt{screensize} by executing
    the \cs{screensize} command. The key takes two dimensions as
    parameters:
\begin{verbatim}
screensize={<height>}{<width>}
\end{verbatim}
The \texttt{screensize} key sets the screen to a height of
\texttt{<height>} and a width of \texttt{<width>}. This key is
ignored if a design option has been taken in the {\Web} option list,
or if the \cs{screensize} command has already been executed.
\item \texttt{margins}: Sets the dimensions of the margins of the
    screen page. The key takes four dimensions as parameters:
\begin{verbatim}
margins={<left>}{<right>}{<top>}{<bottom>}
\end{verbatim}
There \texttt{<left>}, \texttt{<right>}, \texttt{<top>}, and
\texttt{<bottom>} are the dimensions of the margins of the screen
page. These are the same arguments as the \cs{margins} command. This
key is ignored if a design option has been taken in the {\Web}
option list, or if the \cs{margins} command has already been
executed.
\item \texttt{headheight}: Sets the standard page layout parameter
    \cs{headheight}. The default is \texttt{8pt}.
\item \texttt{topmargin}: Sets the position of the running header.
    The default is 8pt for screen sizes, and the {\LaTeX} default is the
    \texttt{forpaper} option is taken. You can say
\begin{verbatim}
topmargin=\prtscr{<paper_dim>}{<screen_dim>}
\end{verbatim}
To get a \texttt{topmargin} value that works for paper and screen.
(Perhaps you have a border running around the page, and you need to
reposition the header below the border. This repositioning differs
depending on whether you compile with the \texttt{forpaper} option,
or not.

\item \texttt{additionalheadsep}: Additional separation between the
    running header and the beginning of the text. This parameter
    modifies \cs{headsep}, which is computed automatically by the
    \textsf{Web} package. Use \texttt{additionalheadsep} to add in a
    little bit more, if the calculation does not yield good results. The default
    value is 0pt.

\item \texttt{marginparsep}: The horizontal distance between body and marginal notes.
    The default is 11pt. When the page size is designed for viewing on a computer,
    the margins are often times quite small, and marginal notes are typically not used.
    You can use a conditional to give two values one for screen and one for paper.
    See the \texttt{topmargin} key.

\item \texttt{marginparwidth}: The width of marginal notes. The default is .25in.
    When the page size is designed for viewing on a computer,
    the margins are often times quite small, and marginal notes are typically not used.
    You can use a conditional to give two values one for screen and one for paper.
    See the \texttt{topmargin} key.

\item \texttt{marginparpush}: The minimal vertical space between
    successive marginal notes. The default is 5pt.

\item \texttt{footskip}: Vertical distance separating the baseline
    of the last line and the baseline of the footer. The default is
    30pt.

\item \texttt{webfootskip}: Same as \texttt{footskip}, but the
    \texttt{webfootskip} is measured up from the bottom edge of the
    document. The default is 4pt.

\item \texttt{panelwidth}: Sets the panel width. If not specified,
    the minimal panel width of 1in is used. Panels appear when either
    \texttt{leftpanel} or \texttt{rightpanel} is taken in the option
    list of \Web. This key executes the \cs{panelwidth} command.

\item \texttt{panelsep}: Sets the separation between the text screen
    and the panel. The default is 10pt.
\end{enumerate}

\subsection{The Title Page: \texorpdfstring{\protect\cs{maketitle}}{\CMD{maketitle}}}\label{maketitlepro}

In this section, we described the enhanced features for controlling the title page.

\subsubsection{The Title Page Structure}

The title page is divided into three parts: top, middle and bottom.
\begin{itemize}
\item \textbf{Title Top:} The content of the top is determined by the \Com{top\-Titl\-ePage}
command. (This command can be redefined, but it is not recommended.) The
\cs{topTitlePage} command expands to three elements: the university (affiliation), the title,
and the author(s), in that order vertically. These are the values of the keys \texttt{university},
\texttt{title} and \texttt{author} that appear in the \cs{Declare\-Doc\-Info} command.

\item \textbf{Title Middle:} The \Com{optionalPageMatter} command
is used to enter content into this part of the title page. This middle
part is optional; if \cs{optionalPageMatter} does not
appear in the preamble, then this part of the title page is empty.

\item \textbf{Title Bottom:} Bottom of the title page is controlled by the contents
of the command \cs{titlepageTrailer} and consists of some of the
document information entered in the \cs{Declare\-DocInfo} command.
By default, \cs{titlepageTrailer} lists the values of the
\cs{DeclareDocInfo} keys \texttt{email}, \texttt{talkdate},
\texttt{talksite}, and \texttt{copy\-right\-years}, as described above. The
font size of this bottom part is set by the command \cs{trailerFontSize}, the
default is \cs{footnotesize}. This command can be re-defined in the
usual way, for example,
\begin{verbatim}
    \renewcommand{\trailerFontSize}{\scriptsize}
\end{verbatim}
\end{itemize}

\newtopic \hyperref[fig:titlepage]{Figure~\ref*{fig:titlepage}}, page~\pageref*{fig:titlepage}, shows the basic composition of the title page of an {\AEB} document. The title page elements are described as they relate to the key-values of \nameref{docinfo}, described on page~\pageref{docinfo}.
\begin{itemize}
    \item At the very top is the value of the \Indextt{university} key.  The color of this
    element can be set using the \Indextt{universityColor} key of \hyperref[designColor]{\Com{selectColors}},
    page~\pageref{designColor}.
    \item Next comes the value of \Indextt{title}, its color is controlled using the \Indextt{titleColor}
    key  of \hyperref[designColor]{\cs{selectColors}}.

    \item The author follows, which is the value of \Indextt{author}. The color is set by
    \Indextt{authorColor} of \hyperref[designColor]{\cs{selectColors}}.
    \item The {\AEB} logo is inserted using \Com{optionalPageMatter}. Normally,
    this macro does nothing unless it is defined. In this example, we have
\begin{Verbatim}[fontsize=\small]
\optionalPageMatter
{%
    \begin{center}
        \begin{minipage}{.67\linewidth}
        \centering\includegraphics[scale=.5]{AeB_Logo}
        \end{minipage}
    \end{center}
}
\end{Verbatim}
    \item Finally comes the \Com{titlepageTrailer}, a macro that can
    be redefined (see the package file \texttt{web.dtx} for its definition. This
    macro places the other elements at the bottom of the page:
    \begin{itemize}
        \item The copyright year and email address, as given by
        \Com{email}. The color of the email address is set by
        \Indextt{urlColor} through the command
        \hyperref[designColor]{\cs{selectColors}}. This color
        actually sets the color of all external \textsf{URL}s.
        \item To the right, on the first line of the title page trailer is the value of \Com{talksite}.
        The color is the default color for text.
        \item In the lower left is the date of the last revision of the document. The color is the default color for text.
        \item In the lower right is the date the talk was given. The color is the default color for text.
    \end{itemize}
\end{itemize}
Not shown in Figure~\ref*{fig:titlepage} is the title page directory, which is turned off by default.

\begin{figure}[ht]
\setlength{\fboxsep}{0pt}
\centering\fbox{\includegraphics[width=.5\linewidth]{figures/titlepage1}}
\caption{The Title Page}\label{fig:titlepage}
\end{figure}

\newtopic The title page layout is, of course, defined by the standard
\Com{maketitle} command, which has been redefined by the \texttt{pro} option.
In {\Web}, the \cs{maketitle} has different behaviors, one for screen and one for paper.

\subsubsection{Greater Control of the Top Title Page}\label{top}

The three elements of the top title page are the values of the keys
\texttt{university}, \texttt{title} and \texttt{author}. These keys
are defined in the command \cs{DeclareDocInfo}. The \texttt{pro} option defines three
commands \Com{universityLayout}, \Com{titleLayout}, and
\Com{authorLayout}, to format these three keys in a variety of ways.

\handpoint A working example of the commands that follows can be found
in \texttt{aebpro\_ex2.tex}.


\begin{dCmd}{.7\linewidth}
\universityLayout{<key-value_pairs>}
\titleLayout{<key-value_pairs>}
\authorLayout{<key-value_pairs>}
\end{dCmd}

\CmdLoc Place these optional commands in the preamble.

\KVP Each of these commands have a number of key-value pairs. The
first thirteen are the same that appear in the description of
\cs{sectionLayout}, \mlNameref{secHead}. The rest are unique to
these three commands. In the descriptions below, the word `element'
refers to the values of the keys \texttt{university},
\texttt{title} and \texttt{author}.

\begin{enumerate}
\item \texttt{fontfamily}: Font family to use with this element,
    permitted values are \texttt{rmfamily}, \texttt{sffamily}, \texttt{ttfamily}.

\item \texttt{fontseries}: Font series to  use, possible values are \texttt{bfseries} and \texttt{mdseries}.

\item \texttt{fontshape}: Font shape to use: \texttt{upshape}, \texttt{itshape}, \texttt{scshape}, \texttt{slshape}.

\item \texttt{fontsize}: Font size to use with this element, recognized values are \texttt{tiny}, \texttt{script\-size},
\texttt{footnotesize}, \texttt{small}, \texttt{normalsize},
    \texttt{large}, \texttt{Large}, \texttt{LARGE}, \texttt{huge}, \texttt{Huge}.

\item \texttt{halign}: Alignment of this element within its enclosing \cs{parbox}, permissible
    values are \texttt{l} (left aligned), \texttt{c} (centered), \texttt{r} (right aligned). See \hyperref[tutex17]
    {Example~\ref*{tutex17}} for a visualization of the effects of the \texttt{halign} key.

\item \texttt{color}: The color of the section title, this can be any named color. The default is blue
for \texttt{title}, and black otherwise.

\item \texttt{special}: Through this key, you can a specify predefined layout for the title elements. Permissible
    values are \texttt{shadow}, \texttt{framebox}, \texttt{colorbox}, \texttt{fcolorbox},
    \texttt{frame\-box\-fit}, \texttt{color\-box\-fit}, \texttt{fcolor\-box\-fit}, \texttt{custom}, \texttt{default}.

\item[] Custom titles can be created by specifying a value of
    \texttt{custom}. In this case, {\Web} uses
    \Com{customUniversity}, \Com{customTitle} and \Com{customAuthor}.
    These are macros that take one argument, the code for designing the
    title. The title is referred to as \texttt{\#1}. Depending on how
    these custom titles are defined, the other keys may not be
    obeyed.

\item \texttt{framecolor}: The color of the frame surrounding the subject when the \texttt{special}
    key has a value of \texttt{framebox}, \texttt{fcolorbox} \texttt{frameboxfit} or \texttt{fcolorboxfit}.

\item \texttt{bgcolor}: The background (fill color) of the box enclosing this element, when
    \texttt{special} has a value of \texttt{colorbox}, \texttt{fcolorbox}, \texttt{colorboxfit}
    or \texttt{fcolorboxfit}.

\item \texttt{shadowcolor}: The color of the shadow, when \texttt{special} has a value of \texttt{shadow}.

\item \texttt{beforeskip}: The amount of skip before the title element.

\item \texttt{afterskip}: The amount of skip after the title element.

\item \texttt{usefont}: Through this key it is possible to specify
an arbitrary font and font size. The key takes five paramters, for
example,
\begin{verbatim}
    usefont={OT1}{cmdh}{m}{n}{{16}{16pt}}
\end{verbatim}
The first four are the arguments of the {\LaTeX}'s \cs{usefont}, encoding, family, series and shape. The last argument are the arguments of the {\LaTeX}'s \cs{fontsize}, size and baselineskip.

\item[] If the fifth parameter is empty, no font size is specified, the current default sizes are used.


%\item \texttt{reset}: This key attempts to reset changes to their defaults. Permissible values are
%    \texttt{font} (to reset the font changes only), or \texttt{all} (to reset all the keys).

\item \texttt{hproportion}: Each of the three elements (university, title, author) lie in their own
\cs{parbox}, the width of this box is determined by the value of this key, as a proportion of the
total \cs{linewidth}. The default for all three is .7. This value can be set to get more or less
``natural'' line breaks, without having to insert a new line with a `\verb!\\!'. See \hyperref[tutex17]
{Example~\ref*{tutex17}} for a visualization of the effects of the \texttt{hproportion} key.

\item \texttt{xhalign}: The \cs{parbox} of each of the three elements are also placed in a \cs{makebox}, additional
control over positioning can be had by setting this key, which sets the positioning parameter of \cs{makebox}.
The default value for \texttt{xhalign} is \texttt{c}, the element is centered. See \hyperref[tutex17]
{Example~\ref*{tutex17}} for a visualization of the effects of the \texttt{xhalign} key.
\end{enumerate}

\exPDF{web_pro} The demo file \texttt{web\_pro} illustrates this command.

\exPDF{webpro_titlepg} The file \texttt{webpro\_titlepg} shows the structure of the title page.

\newtopic Just above the \textbf{Title Top} is a skip that can be    \aebCng[aboveTopTitleSkip]
used to push the top down. The default value of this skip is
\texttt{0pt}, but this can be changed through
\cs{aboveTopTitleSkip}.

\begin{dCmd}[\sCom{aboveTopTitleSkip}]{.6\linewidth}
\aboveTopTitleSkip{<skip>}
\end{dCmd}

\newtopic There is one other title page parameter that effects the layout.

\begin{dCmd}[\sCom{topTitlePageProportion}]{.6\linewidth}
\topTitlePageProportion{<0..1>}
\end{dCmd}

\CmdLoc Place this (optional) command in the preamble.

\PD The top part of the title page is enclosed in a big \cs{parbox}
with depth set to a proportion of \cs{textheight}.
\cs{topTitlePageProportion} is used to set this proportion. The
argument of this command should be a number between 0 and 1,
obviously a value of 0 makes no sense. The default value is set by
the {\Web} package to .33, i.e., the default is
\verb!\topTitlePageProportion{.33}!.

\begin{figure}[ht]
\setlength{\fboxsep}{0pt}
\centering\fbox{\includegraphics[width=.5\linewidth]{figures/titlepage}}
\caption{The Title Page Layout Structure}\label{fig:titlepage1}
\end{figure}

\Ex\label{tutex17} \hyperref[fig:titlepage1]{Figure~\ref*{fig:titlepage1}}
gives a representation of the page layout of the title page. The
big \cs{parbox} of depth equal to the proportion of
\cs{textheight} set by \cs{topTitlePage\-Proportion} is
shown as a blue box. Each of the three top title elements are
enclosed in a \cs{makebox}, shown in yellow.  Inside this
\cs{makebox}, the top title elements are placed in a \cs{parbox},
shown in gray. The image shown in
\hyperref[fig:titlepage1]{Figure~\ref*{fig:titlepage1}} came about
as a result of the following commands in the preamble:
\begin{pCmd*}{fontsize=\small}{.75\linewidth}
\topTitlePageProportion{.5}
\universityLayout{halign=l,color=red}
\titleLayout{halign=r,xhalign=r,hproportion=.4}
\authorLayout{color=webbrown}
\end{pCmd*}
\noindent In \hyperref[fig:titlepage1]
{Figure~\ref*{fig:titlepage1}}, notice that the
\texttt{university} is left aligned within its \cs{parbox}. For
\texttt{title}, the proportion is changed from the default of .7
to .4, this is manifested by the shorter gray box (which
represents the underlying \cs{parbox}); \texttt{halign} and
\texttt{xhalign} are both set to \texttt{r}, so the title appears
right aligned within its \cs{parbox}, and the \cs{parbox} is right
aligned within its \cs{makebox}, understand? Finally, for the
\texttt{author} key, we change only the color.  Swave!

\subsubsection{Greater Control over Title Page Trailer}\label{trailer} \aebCng[trailer]

The pro option defines the command \Com{DesignTitlePageTrailer} that takes a
number of key-value pairs to describe the title page trailer. The use of this
command is not needed unless you want something in the trailer other than the default
information.
\begin{dCmd}{.6\linewidth}
\DesignTitlePageTrailer{<key_values>}
\end{dCmd}

\KVP The key-value pairs are described below:
\begin{enumerate}
    \item \texttt{ul}: The text in the upper left corner of the trailer, the default is
\begin{verbatim}
\web@copyright\ \web@copyright@symbol\
    \webcopyrightyears\ \thewebemail
\end{verbatim}
    \item \texttt{ur}: The text in the upper right corner of the trailer, the default is \cs{aeb@talksite},
    as defined by the \cs{talksite} command or the \texttt{talksite} key of the \cs{DeclareDoc\-Info} command.
    \item \texttt{ll}: The text in the lower left corner of the trailer, the default is
    \cs{aeb@Prepared}, as defined by the \cs{prepared} command or the \texttt{prepared} key of the
    \cs{De\-clare\-Doc\-Info} command.
    \item \texttt{lr}: The text in the lower right corner of the trailer, the default is
    \cs{webversion}, as defined by the \cs{version} command, by the \texttt{version} key, or by
    the \texttt{talkdate} key. If \texttt{version} and \texttt{talkdate} both have values, \texttt{talkdate} is used.
    \item \texttt{textColor}: The default color of the text. The default is black.
    \item \texttt{bgColor}: The background of the trailer. This is a named color. The default is there is no color.
    \item \texttt{borderColor}: The border color of the trailer. This is a named color. If specified, then
    \texttt{bgColor} must have a named color too, otherwise, this
    key does nothing. The default is that there is no border color.
    \item \texttt{fboxsep}: The previous two keys use \cs{colorbox} and \cs{fcolorbox}. You can adjust
        the parameters \cs{fboxsep} through this key. The default is 3pt.
    \item \texttt{fboxrule}: The width of the rule created when there are values for
    \texttt{bgColor} and for \texttt{borderColor}. This key sets the value of \cs{fboxrule}. The default
    is .4pt.
    \item \texttt{graphic}: The value of this key is the base name of a graphic to be used as the background
    of the trailer. The graphic is rescaled to the width and the height of the trailer. It also reduces
    \cs{linewidth} so that the text is \cs{fboxsep} from the borders. The default is no graphic.
    If the key graphic is specified, then \texttt{bgColor} and \texttt{borderColor} are ignored.

    \item[] If the \textsf{graphicxbox} package is loaded, AeB uses this package to place the graphic.
    In this case, the \texttt{borderColor} key is obeyed; whereas, it is not obeyed without the
    \textsf{graphicxbox} package.

    \item \texttt{namedgraphic}: This key requires the use of \textsf{graphicxsp}, and Acrobat Distiller.
    Use \textsf{graphicxsp} to embed a graphic and give it a symbolic name \texttt{<mygraphic>}.
    If you then say \texttt{namedgraphic=<mygraphic>}, the graphic is used without re-embedding the graphic.
    The advantage of this method is that the graphic can be used and re-used in the document, without
    significantly increasing file size.

    \item[] If the \textsf{graphicxbox} package is loaded, AeB uses this package to place the named graphic.
    In this case, the \texttt{borderColor} key is obeyed; whereas, it is not obeyed without the
    \textsf{graphicxbox} package.

    \item \texttt{raise}: A key that takes a dimension as its value. The trailer will be raised or
    lowered by the amount specified. For example, \texttt{raise=-10pt} lowers the trailer 10pt from its natural
    position.

    \item \texttt{fontSize}: Set the font size of the text, the default is \cs{footnotesize}. This key
    simply redefines \cs{trailerFontSize}.

    \item \texttt{formatting}: Any additional formatting you wish
    that does not cause trouble; for example,
    \verb!formatting=\bfseries!.
\end{enumerate}

\exPDF{web_pro} The demo file \texttt{web\_pro} illustrates this command.

\subsection{The Table of Contents}

The pro option also defines \Com{tocLayout} for controlling the
title of the table of contents.


\begin{dCmd}{.7\linewidth}
\tocLayout{<key-value_pairs>}
\end{dCmd}
\noindent The key-value pairs are the same as layout commands for the section titles, see
\mlNameref{secHead}.

\CmdLoc Place this command in the preamble.

\begin{dCmd}[\sCom{selectTocDings}]{.7\linewidth}
\selectTocDings{<key-value_pairs>}
\end{dCmd}
\noindent When the \cs{noSectionNumbers} command is used (\Nameref{secHead}), the entries in the table of
contents have no section number.  You can define a ding for each of the three
levels of the table of contents with this command. The key-value entries in the
\cs{selectTocDings} command are ignored if \cs{noSectionNumbers} is not in effect.

\CmdLoc Place this (optional) command in the preamble.

\KVP The following are the recognized key-value pairs:

\begin{enumerate}
\item \texttt{dDing}: The ding for the first (section) level entry.\\Example,
\verb!dDing=\ding{072}!.
\item \texttt{ddDing}: The ding for the second (subsection) level entry.
\item \texttt{dddDing}: The ding for the third (subsubsection) level entry.
\item \texttt{dDingColor}: The (named) color for the ding of the first level entry. The default is red.
\item \texttt{ddDingColor}: The (named) color for the ding of the first level entry. The default is blue.
\item \texttt{dddDingColor}: The (named) color for the ding of the first level entry. The default is webgreen.

\end{enumerate}

\exPDF{web_pro} The demo file \texttt{web\_pro} illustrates this command.

\subsection{Selecting Colors}\label{designColor}

The \cs{selectColors} command can be used to set the color of most all elements of
the document that have a color attribute.

\begin{dCmd}[\sCom{selectColors}]{.5\linewidth}
\selectColors{<key-values>}
\end{dCmd}

\CmdLoc There are no restrictions on this command.  In the preamble,
it will set the defaults for the entire document.

\KVP Each of the following keys take a named color. Many of these keys have a
command interface as well, these are also noted.
\begin{enumerate}
\item  \texttt{universityColor}: The color of the value of the \cs{university}
    declaration (refer to page~\pageref{docinfo}). The value of
    \cs{university} appears on the title page. The default is blue.
\item  \texttt{titleColor}: The color of the value of the \cs{title} declaration
    (see page~\pageref{docinfo}).
    The value of \cs{title} appears on the title page. The default is black.
\item  \texttt{authorColor}: The color of the value of the \cs{author} declaration
    (see page~\pageref{docinfo}).
    The value of \cs{author} appears on the title page. The default is black.
\item  \texttt{textBgColor}: The background color of the text screen. The default is white.
\item  \texttt{panelBgColor}: The background color of the navigation panel. The default is white.
\item  \texttt{urlColor}: The color of an \textsf{URL} link. The default is webbrown.
\item  \texttt{linkColor}: The color of a link. The default is webgreen.
\item  \texttt{fileColor}: The color of a link to a local file. The default is webbrown.
\end{enumerate}
There are several other colors as well, these can be set through their interface, see
\hyperref[secHead]{Section~\ref*{secHead}}, page~\pageref*{secHead}, for a discussion of
the commands \cs{sectionLayout}, \cs{sub\-sec\-iion\-Layout} and \cs{sub\-sub\-sec\-tion\-Layout}.

\exPDF{web_pro} The demo file \texttt{web\_pro} illustrates this command.


\subsection{Section Headings}\label{secHead}

The {\Web} package uses the section heading commands \cs{section}, \cs{subsection}, and \cs{sub\-sub\-section}, but the \texttt{pro} option modifies their definitions so that the document author can easily design how the section titles look.

\begin{dCmd}[\sCom{noSectionNumbers}]{.5\linewidth}
\noSectionNumbers
\end{dCmd}
\noindent The command \Com{noSectionNumbers} causes {\Web} to remove the usual
numbering system for a {\LaTeX} document.  This may be useful when
preparing a document for presentation, and a numbering system is not
needed. The default is to use section numbers.

\CmdLoc The use of this command is restricted to the preamble. The
decision must be made for the whole document at the beginning of the
document.

\begin{dCmd}[\sCom{sectionLayout}\sCom{subsectionLayout}\sCom{subsubsectionLayout}]{.7\linewidth}
\sectionLayout{<key-values>}
\subsectionLayout{<key-values>}
\subsubsectionLayout{<key-values>}
\end{dCmd}

\CmdLoc No restriction on the use of this command. In the preamble, you set the layout
for sections for the whole document. In the body of the document, changes occur immediately
starting with the next relevant section.

\KVP Each of these takes the same key-value pairs.
\begin{enumerate}
\item \texttt{fontfamily}: Font family to use for section titles,
    permitted values are \texttt{rmfamily}, \texttt{sffamily}, \texttt{ttfamily}.
\item \texttt{fontseries}: Font series to use for the section title, values are \texttt{bfseries} and \texttt{mdseries}.
\item \texttt{fontshape}: Font shapes to use for the section title, values are \texttt{upshape}, \texttt{itshape}, \texttt{scshape}, \texttt{slshape}.
\item \texttt{fontsize}: Font size of the section title, permissible values are \texttt{tiny}, \texttt{scriptsize},
\texttt{foot\-note\-size}, \texttt{small}, \texttt{normalsize}.
    \texttt{large}, \texttt{Large}, \texttt{LARGE}, \texttt{huge}, \texttt{Huge}.
\item \texttt{halign}: Alignment of the section title, permissible
    values are \texttt{l} (left aligned), \texttt{c} (centered), \texttt{r} (right aligned).
\item \texttt{ding}: You can specify a ding to display where the section number is usually placed; e.g.,
    \verb+ding=\ding{066}+.

\item \texttt{color}: The color of the section title, this can be any named color. The default is blue.

\item \texttt{special}: Through this key, you can a specify predefined layout for the section titles. Permissible
    values are \texttt{shadow}, \texttt{framebox}, \texttt{colorbox}, \texttt{fcolorbox},
    \texttt{frame\-box\-fit}, \texttt{color\-box\-fit}, \texttt{fcolorbox\-fit}, \texttt{custom}, \texttt{default}.

\item[] Custom section heads can be used by specifying a value of \texttt{custom}. In this
    case, {\Web} uses the values of \Com{customSecHead}, \Com{custom\-Sub\-sec\-Head} and
    \Com{custom\-Sub\-sub\-sec\-Head}. Each of these macros take one argument, the code for designing
    the title. The title is referred to as \texttt{\#1}. Depending on how these custom section titles
    are defined, the other keys may not be obeyed.

%    See an \hyperref[customHeadEx]
%    {Example~\ref*{customHeadEx}} below; there are other examples in the demo files.

\item \texttt{framecolor}: The color of the frame surrounding the subject when the \texttt{special}
    key has a value of \texttt{framebox}, \texttt{fcolorbox} \texttt{frameboxfit} or \texttt{fcolorboxfit}.

\item \texttt{bgcolor}: The background (fill color) of the box enclosing the section title, when
    \texttt{special} has a value of \texttt{colorbox}, \texttt{fcolorbox}, \texttt{colorboxfit}
    or \texttt{fcolorboxfit}.

\item \texttt{shadowcolor}: The color of the shadow, when \texttt{special} has a value of \texttt{shadow}.

\item \texttt{beforeskip}: The amount of skip before the section title.

\item \texttt{afterskip}: The amount of skip after the section title.

\item \texttt{usefont}: Through this key it is possible to specify
an arbitrary font and font size. The key takes five parameters, for
example,
\begin{verbatim}
    usefont={OT1}{cmdh}{m}{n}{{16}{16pt}}
\end{verbatim}
The first
four are the arguments of the {\LaTeX}'s \cs{usefont}, encoding,
family, series and shape. The last argument are the arguments of the
{\LaTeX}'s \cs{fontsize}, size and baselineskip.

\item[] If the fifth parameter is empty, no font size is specified, the current default sizes are used.

\item \texttt{numdingcolor}: The color of the section number or ding, if defined. The default is blue.

\item \texttt{reset}: This key attempts to reset changes to their defaults. Permissible values are
    \texttt{font} (to reset the font changes only), or \texttt{all} (to reset all the keys).
\end{enumerate}

\Ex\label{tutex7} The following example sets the section
title with shadows: text is red and shadow is blue. We center the
title horizontally, and specify an after skip of \texttt{12pt}.
\begin{dCmd*}{.5\linewidth}
\sectionLayout{%
    afterskip=12pt,
    halign=c,
    color=red,
    shadowcolor=blue,
    special=shadow
}
\end{dCmd*}
\noindent Similarly for \Com{subsectionLayout} and \Com{subsubsectionLayout}.

% \expoint{ex7} Use \texttt{apb\_ex7.tex} to explore the section layout commands.

\begin{dCmd}[\sCom{shadowhoffset}\sCom{shadowvoffset}]{.5\linewidth}
\shadowhoffset{<length>}
\shadowvoffset{<length>}
\end{dCmd}
When \texttt{$\text{special} = \text{shadow}$}, shadowing is
obtained by typesetting the text twice with different colors. The
amount of horizontal and vertical offset are controlled by these two
commands. The defaults are \verb!\shadowhoffset{.2ex}! and
\verb!\shadowvoffset{-.2ex}!.

\newtopic The interface for writing custom section layouts are the following three
commands.

\begin{dCmd}[\sCom{customSecHead}\sCom{customSubsecHead}\sCom{customSubsubsecHead}]{.6\linewidth}
\customSecHead{<tex_code>}
\customSubsecHead{<tex_code>}
\customSubsubsecHead{<tex_code>}
\end{dCmd}
\CmdLoc Use in the preamble or before a section head.

\PD These three commands are used to write custom section,
subsection and subsubsection layouts. Here, the parameter \texttt{<tex\_code>} is
{\TeX} code for laying out the section titles, and should use
\texttt{\#1} to represent the section title.

\Ex\label{customHeadEx} The following definitions can appear
anywhere, they are global unless appearing in a group.
This puts a colored box around the section title, which
is assumed to appear in one line. We use the internal color names
for the \texttt{color} (\cs{aeb@sectioncolor}) and \texttt{bgcolor}
(\cs{aeb@sectionbgcolor}) keys, in this way, the title will obey the
values of these keys. This particular custom head obeys the
\texttt{halign} key as well.
\begin{dCmd*}{\linewidth}
\makeatletter
\customSecHead{\vbox{\colorbox{\aeb@sectionbgcolor}
    {\color{\aeb@sectioncolor}#1}}}
\sectionLayout{%
    special=custom,
    halign=l,
    bgcolor=red,
    color=white
}
\makeatother
\end{dCmd*}
\noindent The above code is a simplified version of the \texttt{colorboxfit},
a value of the \texttt{special} key.

\exPDF{web_pro} The demo file \texttt{web\_pro} illustrates these commands.

%\expoint{ex7} See \texttt{apb\_ex7.tex} for a working example of custom section
%head layout.

%\Important It is important to remark that between slides \texttt{@}
%is made into a `letter', and reset to `other' at the beginning of
%the next slide. This makes it convenient to make these definitions
%between slides. In the preamble, however, such a definition must be
%between  \cs{makeatletter} and \cs{makeatother}.

\section{On Parts, Chapters, and Final Dots}\label{partsChaptersDots}

The {\Web} package has several commands for customizing the \cs{part} and
\cs{chapter} commands.

\subsection{The \texorpdfstring{\protect\cs{part}}{\CMD{part}} command}

The appearance of the part title can be controlled by \cs{formatPartTitle}.

\begin{dCmd}[\sCom{formatPartTitle}\sCom{noPartNumbers}]{.6\linewidth}
\formatPartTitle{<tex_code>}
\noPartNumbers
\end{dCmd}

\newtopic The argument of \verb!<tex_code>! consists of formatting commands; within the
argument \texttt{\#1} represents the title.  The command \cs{noPartNumbers}
is a convenience macro to turn off all part number, if you want to create
a document that has no part numbers, such as this document.

The default argument for \cs{formatPartTitle}

\begin{Verbatim}[xleftmargin=20pt,fontsize=\small]
\large\bfseries
 \ifnum \c@secnumdepth >\m@ne
   \partname~\thepart:\space
\fi#1
\end{Verbatim}
\newtopic This typesets as {\large\bfseries Part I: The Web Package}, for example. You can
modify the basic definition to introduce colors, fonts, sizes of your choice, and
even special effects; for example, the declaration
\begin{Verbatim}[xleftmargin=20pt,fontsize=\small]
\formatPartTitle{\fbox{\large\bfseries
 \ifnum \c@secnumdepth >\m@ne
   \partname~\thepart:\space
\fi#1}}
\end{Verbatim}
gives a part title like this, {\large\bfseries\fbox{Part I: The Web Package}}

\newtopic
A companion comment for changing the format of a part title, is \cs{tocPartTitle},
which modifies how the part title appears in the table of contents.

\begin{dCmd}[\sCom{tocPartTitle}]{.6\linewidth}
\tocPartTitle{<tex_code>}
\tocPartTitle{\thepart\hspace{1em}#1}
\end{dCmd}

\newtopic Again, within the argument of \cs{tocPartTitle}, \texttt{\#1} refers to the
text of the part title. The default definition is given above also.

\redpoint Finer control over the formatting of the \cs{part} command can be obtained by redefining
\cs{web@makeparthead} or \cs{web@makesparthead}, the latter one for \cs{part*}.

\subsection{The \texorpdfstring{\protect\cs{chapter}}{\CMD{chapter}} command}

The \cs{chapter} command is available in the \texttt{book} and \texttt{report} classes.
There are several defined in the {\Web} package for getting a little control over these.


\long\def\formatChapterNumber#1{\def\web@formatChapterNumber{#1}}
\formatChapterNumber{\large\bfseries
    \@chapapp\space\thechapter\par\nobreak}
\long\def\formatChapterTitle#1{\def\web@formatChapterTitle##1{#1}}
\formatChapterTitle{%
    \interlinepenalty\@M
    \noindent\hspace{1em}\large\bfseries#1\par\nobreak
}

\begin{dCmd}[\sCom{formatChapterNumber}\sCom{formatChapterTitle}]{.6\linewidth}
\formatChapterNumber{<tex_code>}
\formatChapterTitle{<tex_code>}
\end{dCmd}

\newtopic The argument of \verb!<tex_code>! consists of formatting commands; within the
argument \texttt{\#1} represents the title.

The default definitions for these are

\begin{dCmd*}{.85\linewidth}
\formatChapterNumber{\large\bfseries
    \@chapapp\space\thechapter\par\nobreak}
\formatChapterTitle{%
    \interlinepenalty\@M
    \noindent\hspace{1em}\large\bfseries#1\par\nobreak
}
\end{dCmd*}
\newtopic These are slighly modified versions of what is found in \texttt{book.cls}.

\redpoint A finer control can be obtained over the \cs{chapter} command formatting by redefining
\cs{web@makechapterhead} or \cs{web@makeschapterhead}, the latter one is used for \cs{chapter*}.

\subsection{The \texorpdfstring{\protect\cs{noFinalDot}}{\CMD{noFinalDot}} Command}\label{nofinaldot}

The default behavior for the {\Web} package is to place a dot (.) at the end of the section
number in both the body of the document and in the table of contents. Thus,
this section is numbered as \textcolor{blue}{\ref*{nofinaldot}.} (note the final dot following
the section number). This section will appear in the table of contents the same way. The default
behavior of all (that I know of) standard {\LaTeX} classes is not to place this final
dot (.) at the end of the section number. To revert to this behavior, use the
\Com{noFinalDot} command in the preamble.

%---------------------- Exerquiz -----------------------------------%

\addtocontents{toc}{%
    \par\vskip6pt\vskip0pt plus .25\vsize\penalty-1000 \vskip0pt plus-.25\vsize}

%\addtocontents{toc}{\par\protect\vspace{\bigskipamount}\kern0pt\vspace{-\medskipamount}}
\part{The Exerquiz Package}\label{s:exerquiz}

%The \textsf{exerquiz} package contains a series of macros for
%creating on-line exercises, short quizzes (with or without
%solutions), and longer quizzes that are graded using JavaScript.

%\section{Overview}

The \textsf{exerquiz} package provides environments for creating
the following interactive elements in a \textsf{PDF} file.

\begingroup
\def\mylabel#1{\hfil\hyperref[#1]{\large$\bullet$}\ }
\begin{list}{}
{\usecounter{questionno}\setlength{\parsep}{0pt}\setlength{\topsep}{0pt}
\renewcommand{\makelabel}{\mylabel}
\settowidth{\labelwidth}{\normalfont\bfseries00.\ }%
\setlength{\itemindent}{0pt}\setlength{\labelsep}{0pt}}%

\item[ss:online] \begin{NoHyper}\nameref{ss:online}\end{NoHyper}:
Macros for creating on-line exercises.

\item[ss:QuizIR] \begin{NoHyper}\nameref{ss:QuizIR}\end{NoHyper}:
Macros for creating interactive quizzes with immediate feedback.


\item[ss:QuizS] \begin{NoHyper}\nameref{ss:QuizS}\end{NoHyper}:
Macros for creating quizzes with immediate feedback
and a link to the solutions to the quizzes.

\item[ss:GradedQuizzes]
\begin{NoHyper}\nameref{ss:GradedQuizzes}\end{NoHyper}:
Macros for creating quizzes graded by Java\-Script, with an option
to have the quizzes corrected using JavaScript.

\end{list}
\endgroup

\noindent In each of the quiz environments, you can pose multiple
choice, math fill-in, or text fill-in questions.

The \textsf{exerquiz} provides the above listed environments for
the \texttt{dvips\-one}, \texttt{dvips}, \texttt{textures}, \texttt{pdftex} and
\texttt{dvipdfm} options.

There are options for reformatting the exercises to a print
format,  for excluding the solutions to the exercises, for writing
the solutions to the exercises so they follow the question, and
for different languages, and much more.

The \textsf{exerquiz} also allows you to rearrange the order and
location of the solutions to the exercises and quizzes,  to
redefine many running headers, to customize the exercises and
quizzes, and to use the \texttt{exercise} environment to create
a new environment with its own counter, or with no counter at all.

All the above mentioned macros and the options of the package are
discussed in this section.

\section{\protect\textsf{Exerquiz} and Acrobat JavaScript}

\textsf{Exerquiz} now uses the \textsf{insdljs} Package to insert
Document level Java\-Scripts into the \textsf{PDF} file. The quizzes
created using the \texttt{shortquiz} or \texttt{quiz} environment are
graded, marked and scored using these inserted JavaScript functions.

Because the package \textsf{insdljs} is already loaded, it is easy
for the document author to develop JavaScripts that can be called
from the standard \textsf{Exerquiz} commands. The ability to write
JavaScript, therefore, right in the \LaTeX{} document gives a
unique programming flair to \textsf{Exerquiz}.

\section{Package Requirements}

The \textsf{exerquiz} package is independent of the \textsf{web}
package;  however, \textsf{exerquiz} utilizes \textsf{hyperref}
just as \textsf{web} does.  Use the latest version of
\textsf{hyperref}.  In addition to the \textsf{color} package,
also used by \textsf{web}, \textsf{exerquiz} also uses the
\textsf{verbatim} package. This is used to write verbatim
solutions to exercises and quizzes to certain auxiliary files.

Results from the quizzes created by the \texttt{shortquiz} and
\texttt{quiz} environments are evaluated using document level
JavaScripts. These JavaScripts are inserted into the final
\textsf{PDF} file using the \textsf{insdljs} package. This package
makes it easy for the package writer or document author to write
JavaScripts.

The \textsf{exerquiz} package uses \emph{form features} of
\textsf{PDF} that \textsf{web} does not use. For the interactive
features to properly work, use \textcolor{blue}{Adobe Reader
5.0} or higher.

\section{Basic Package Options}\label{ss:exbasicusage}

Place in the preamble of your document
\begin{sverbatim}
\usepackage{exerquiz}
\end{sverbatim}

\redpoint Use \textsf{exerquiz} with the \textsf{web} package:
\begin{sverbatim}
\usepackage[<driver_option>,<more_options>]{web}
\usepackage[<options>]{exerquiz}
\end{sverbatim}
\noindent A complete list of the options recognized by \textsf{exerquiz} can be
found in \hyperref[s:ListofOpts]{Section~\ref*{s:ListofOpts}}, page~\pageref*{s:ListofOpts}; they are
also discussed below.

The driver options for the \textsf{web} package are \texttt{dvipsone}, \texttt{dvips}, \texttt{pdftex}, \texttt{dvipdfm} and \texttt{textures}.
No driver option with \textsf{exerquiz} is needed if you are using
the \textsf{web} package,  \textsf{exerquiz} inherits the driver option from
the \textsf{web} package.

For the \texttt{dvipdfm} option to work properly you
will need \textsf{dvipdfm}, version 0.12.7b or later, and
\textsf{hyperref}, version 6.68a or later.

\redpoint Use \textsf{hyperref} and \textsf{exerquiz} with
either \texttt{dvipsone} or \texttt{dvips}:
\begin{sverbatim}
\usepackage[<driver_options>,<more_options>]{hyperref}
\usepackage{exerquiz}
\end{sverbatim}
\noindent Permissible driver options are \texttt{dvipsone} and \texttt{dvips}.

\redpoint Use \textsf{hyperref} and \textsf{exerquiz} with
\texttt{pdftex} or \texttt{dvipdfm}
\begin{sverbatim}
\usepackage[<driver_options>,<more_options>]{hyperref}
\usepackage[<driver_option>]{exerquiz}
\end{sverbatim}
\noindent See the next few paragraphs for more details.

\subsection{The \texorpdfstring{\protect\texttt{pdftex}}{pdftex}
Option}\label{ss:pdftexerquiz}

The \textsf{exerquiz} package is independent of the \textsf{web}
package.  Therefore, you can create your own page layout package
and use \textsf{exerquiz} to help you create exercises and
quizzes. Of course, \textsf{hyperref} must be used.

%\redpoint Using \textsf{exerquiz} with \textsf{hyperref} or
%\textsf{pdfscreen}

Should you want to use the \textsf{exerquiz} package using
\textsf{pdftex} without the \textsf{web} package, use the
\texttt{pdftex} option:
\begin{sverbatim}
\usepackage[pdftex,<more options>]{hyperref}
\usepackage[pdftex]{exerquiz}
\end{sverbatim}
\noindent In particular,
\textsf{pdfscreen}\footnote{CTAN:macros/latex/contrib/supported/pdfscreen},
a screen design package written for \textsf{pdftex} by C. V.
Radhakrishnan, has been tested and works correctly with
\textsf{exerquiz}.  For example,
\begin{sverbatim}
\usepackage[screen,article,sidebar]{pdfscreen}
\usepackage[pdftex]{exerquiz}
\end{sverbatim}
\noindent See the sample file \texttt{eq\_pdfs.tex} already set up for use
with \textsf{pdfscreen}, obtained by downloading the zipped file
\texttt{\href{\bUrl/tutorial/webeq/eq_pdfs.zip}{eq\_pdfs.zip}}.

\subsection{The \texttt{dvipdfm} Option}\label{ss:dvipdfm}

Should you want to use the \textsf{exerquiz} package without
the \textsf{web} package, in this case, the usage is
\begin{sverbatim}
   \usepackage[dvipdfm,<more_options>]{hyperref}
   \usepackage[dvipdfm]{exerquiz}
\end{sverbatim}

%\subsection{The \texttt{dvi\-windo} Option}\label{ss:dviwindo}
%
%Beginning with version 1.3 of \textsf{web} and \textsf{exerquiz},
%\textsf{dviwindo} (the \texttt{.dvi} previewer by
%\href{http://www.yandy.com/}{\Y&Y, Inc.}) is supported.  This
%means that hypertext links will be active from within the
%\textsf{dviwindo} previewer---as well as from within the Adobe
%Reader after the file has been converted to \textsf{pdf}.
%
%Should you want to use the \textsf{exerquiz} package without
%the \textsf{web} package, in this case, the usage is
%\begin{sverbatim}
%   \usepackage[dviwindo,<more_options>]{hyperref}
%   \usepackage[dviwindo]{exerquiz}
%\end{sverbatim}


\redpoint \textbf{\textcolor{red}{Important Note:}} \emph{Only}
the \texttt{exercise} environment (this is the material described in
\hyperref[ss:online]{Section~\ref*{ss:online}}) is supported by
these two options.  None of the quiz environments can be used with
these two options at this time. \Y&Y{} users need to use the
\texttt{dvipsone} option if the quiz environment is needed.


\subsection{The Language Option}\label{sss:exerquizlangopts}

The \hyperref[web:langopts]{language option}, available in the
\textsf{web} package, can be invoked even when the \textsf{web}
package is not used.\footnote{Otherwise, the language option is
introduced as an option of the \textsf{web} package.} Currently,
\texttt{dutch}, \texttt{french}, \texttt{german},
\texttt{italian}, \texttt{norsk}, \texttt{russian},
\texttt{spanish}, \texttt{polish}, \texttt{finnish},  \texttt{czech}, \texttt{catalan},
\texttt{brazil}, and \texttt{turkish}.
are the supported options. For example, with \textsf{hyperref}, you could use:
\begin{sverbatim}
   \usepackage[<driver_option>,<more_options>]{hyperref}
   \usepackage[<driver_option>,french]{exerquiz}
\end{sverbatim}
\noindent \texttt{<driver\_option>} is any of the drivers:
\texttt{dvips\-one}, \texttt{dvips}, \texttt{pdftex}, or
\texttt{dvipdfm}.  \emph{Note}:  the \texttt{<driver\_option>} is
not needed with the \textsf{exerquiz} package with \texttt{dvipsone}
or \texttt{dvips}.

\redpoint\textbf{\textcolor{red}{Important Note:}} When using the
\texttt{turkish} option, without using the \textsf{Web}
package, use the \texttt{unicode} option for hyperref, like so,\footnote{The
\textsf{Web} package passes the \texttt{unicode} option to \textsf{hyperref} when using the \texttt{turkish} option
through the Web option list so there is no need to explicitly use the \texttt{unicode} option.}
\begin{sverbatim}
   \usepackage[<driver_option>,<more_options>,unicode]{hyperref}
   \usepackage[<driver_option>,turkish]{exerquiz}
\end{sverbatim}
This is because the word for ``Table of Contents'' is ``{\.I}{\c
c}indekiler'' in Turkish. For {\.I}{\c c}indekiler to appear
correctly in the bookmarks, hyperref needs the \texttt{unicode}
option.\footnote{The glyph {\.I} (latin capital letter I with dot above) is in the Latin Extended-A group,
unicode \texttt{U+0130}.}

\subsection{The
\texorpdfstring{\protect\texttt{forpaper{\protect\normalfont\protect\space and }forcolorpaper}}{forpaper and forcolorpaper}
 Options}\label{sss:exquizforpaper}

The \texttt{forpaper} or \texttt{forcolorpaper} option, also available in the
\textsf{\hyperref[sss:webforpaper]{web}} package, is needed in the
\textsf{exerquiz} package if your are using \textsf{exerquiz}
without \textsf{web}.  The option is invoked in the usual way.
\begin{sverbatim}
\usepackage[<options>]{hyperref}  % or pdfscreen
\usepackage[forpaper]{exerquiz}
\end{sverbatim}
\noindent See the discussion of the \texttt{\hyperref[sss:webforpaper]{forpaper}} on
page~\pageref{sss:webforpaper} given earlier.

\subsection{The \texttt{preview} Option}\label{sss:preview}

The \textsf{exerquiz} package can generate a large number of form
fields: buttons, check boxes, radio buttons, and text fields. These
are PDF objects and cannot be seen in a dvi previewer.  By using
the \texttt{preview} option, the bounding rectangles of the form
objects are surrounded with rules, which outline the form fields
and make their positions visible.

This option may help you to fine tune the positions of the form
fields. The option is for developmental use only. When you are
satisfied with the positioning and are ready to publish, remove
this option.

\redpoint This option is not useful with the \texttt{pdftex} option, as
\textsf{pdftex} does not (normally) produce a dvi file.

\subsection{The \texttt{nodljs} Option}\label{sss:nodljs}

If you are creating a document that is meant to be printed or your
document only has exercises and solutions in it (which do not
require JavaScript), the size of the document can be reduced
significantly by using the \texttt{nodljs} option. This option is
just passed on to the \textsf{insdljs} package.

% \subsubsection{The \texttt{acrobativ} Option}\label{sss:acrobativ}
%
% If the document author is using the \texttt{dvips} or the \texttt{dvipsone} option
% but has only Acrobat~4.0 or~4.05, then the document level JavaScripts need
% to be inserted manually. Therefore, we need to turn off the automatic inclusion of JavaScript.
% This option does exactly that; it is equivalent to the \texttt{nodljs} option.

\subsection{The \texttt{exercisesonly} Option}\label{sss:noquizzes}

If the document author only uses the \texttt{exercise} environment, then all
the document level JavaScripts of \texttt{exerquiz} are not needed.  Use either
one of these two equivalent options to exclude the insertion of the JavaScripts.

This is a convenience option that simply calls the \texttt{nodljs} option described above.

\subsection{The \texttt{de\-bug} Option}\label{sss:debug}

Developing JavaScript functions can be tricky. Quite often, it is
useful to insert some code lines that will help you in debugging a
particular function or a set of functions. For example, you might
want to verify that the parameters being passed to a function are
the correct ones, or that the return value is correct. You can
have Acrobat write the values to its console like so:
\begin{sverbatim}
console.println("Function myFunc");
console.println("Parameters: x = " x + ", y = " + y );
console.println("Return Value: retnValue = " + retnValue);
\end{sverbatim}
\noindent In the above code, I have used the
\texttt{console.println()} method, which is only available in the
Acrobat application, not the Reader. For the Reader,  one could
use \texttt{app.alert()}, but this method is not well-suited for
monitoring values of a large number variables as the script
executes. If you don't have the full Acrobat, the \texttt{debug}
option will not be useful.

\textsf{Exerquiz} just passes this option on to the
\textsf{insdljs} package. Additional details on the debug option
can be found there. Within the \texttt{insDLJS} environment, you
can place debugging code lines as follows:
\begin{sverbatim}
function myFunc(x,y)
{
    retnValue = x + y;
\db console.println("Function myFunc");\db%
\db console.println("Parameters: x = " x + ", y = " + y );\db%
\db console.println("Return Value: retnValue = " + retnValue);\db%
    return retnValue;
}
\end{sverbatim}
\noindent Any line that begins with \cs{db} and ends with \cs{db} is a
debugging line. These lines will be included if the \texttt{debug} option
is taken; otherwise they are removed. The `\texttt{\%}', is
the comment character within the \texttt{insDLJS} environment, and prevents,
in this case, the introduction of a carriage return.

\subsection{The \texttt{allowrandomize} Option}\label{ss:allowrandomize}

Use this option to load in the macros to randomize the choices
of a multiple choice question. See \hyperref[s:random]{Section~\ref*{s:random}},
\Nameref{s:random} for details.

\subsection{The \texttt{unicode} Option}\label{eq:unicode}

This option is passed to \textsf{hyperref}, and is placed in {\Web} as a
convenience. If \textsf{eforms} is subsequently loaded, \textsf{eforms} (and
\textsf{exerquiz}) will accept {\LaTeX} markup in the optional argument
of the form fields. See the \href{eformman.pdf}{eforms manual}
for details.

\subsection{The \texttt{useui} Option}\label{eq:useui}

This option is passed to \textsf{eforms}, and is placed in {\Web} as
a convenience. If \textsf{eforms} is subsequently loaded,
\textsf{eforms} loads the xkeyval package, and key-value pairs are
defined for use in the optional argument of form and link commands.
See the \href{eformman.pdf}{eforms manual} for details.

\subsection{The \texttt{usesumrytbls} Option}\label{eq:usesumrytbls}

When this option is taken, the code for creating quiz summary tables is input.
See \hyperref[s:sumrytbl]{Section~\ref*{s:sumrytbl}}, page~\pageref*{s:sumrytbl} for details.


\section{The \texttt{exercise} Environment}\label{ss:online}

The \textsf{exerquiz} package defines \texttt{exercise} and
\texttt{solution} environments, the latter being nested inside the
former. With these environments, you can create questions
(exercises) with solutions. Solutions are written
\texttt{verbatim} to the auxiliary file \cs{jobname.sol}, then
input back in near the end of the document. A hypertext link is
created to connect the exercise with the solution.

An exercise with \hyperref[sss:exwparts]{multiple parts} can also
be defined, with hypertext links to the solutions of the
individual parts of the exercise.

The \texttt{exercise} environment has its own counter
(\texttt{eqexno}), but there is an \hyperref[sss:eqexno]{option}
for using another counter, or no counter at all.  This may be
useful for creating a numbered example environment.

There is an option for placing the
\hyperref[sss:solutionsafter]{solutions immediately after} the
statement of the problem.  This, again, may be useful for an
example environment where you want the solution to the example to
follow the statement, rather than being hypertext-linked to the
solution.

Finally, there is an option for \hyperref[sss:hide]{hiding
solutions}, in the following sense: When the hidden option is
used, the solutions are commented out rather then being written to
the \verb+\jobname.sol+ file. Additionally, there is a global
option, \texttt{nohiddensolutions}; in this case, when you
re-\LaTeX, the solutions are written to \verb+\jobname.sol+, and
input back into the document.

\subsection{Basic Usage}\label{ss:basicexer}

The syntax for the \texttt{exercise} and \texttt{solution}
environments is as follows:
\begin{sverbatim}
\begin{exercise}
Your Question.
\begin{solution}
The Solution to Your Question
. . . . . . . . . . . . . . .
. . . . . . . . . . . . . . .
. . . . . . . . . . . . . . .
\end{solution}
\end{exercise}
\end{sverbatim}
\noindent Here is an example of the usage.

\begin{exercise}\label{ex:int}%
Evaluate the integral \(\displaystyle\int x^2 e^{2x}\,dx\).
\begin{solution}
We evaluate by \texttt{integration by parts}:
\begin{alignat*}{2}
 \int x^2 e^{2x}\,dx &
   = \frac12 x^2 e^{2x} - \int x e^{2x}\,dx &&\quad
           \text{$u=x^2$, $dv=e^{2x}\,dx$}\\&
   = \frac12 x^2 e^{2x} -
           \left[\frac12 x e^{2x}-\int \frac12 e^{2x}\,dx\right] &&\quad
           \text{integration by parts}\\&
   = \frac12 x^2 e^{2x} - \frac12 x e^{2x} + \frac12\int e^{2x}\,dx &&\quad
           \text{$u=x^2$, $dv=e^{2x}\,dx$}\\&
   = \frac12 x^2 e^{2x} - \frac12 x e^{2x} + \frac14 e^{2x} &&\quad
           \text{integration by parts}\\&
   = \frac14(2x^2-2x+1)e^{2x} &&\quad
           \text{simplify!}
\end{alignat*}
\end{solution}
\end{exercise}

The code for this is

\begin{sverbatim}
\begin{exercise}\label{ex:int}%
Evaluate the integral \(\displaystyle\int x^2 e^{2x}\,dx\).
\begin{solution}
We evaluate by \texttt{integration by parts}:
\begin{alignat*}{2}
 \int x^2 e^{2x}\,dx &
   = \frac12 x^2 e^{2x} - \int x e^{2x}\,dx &&\quad
           \text{$u=x^2$, $dv=e^{2x}\,dx$}\\&
... lines removed ...
   = \frac14(2x^2-2x+1)e^{2x} &&\quad
           \text{simplify!}
\end{alignat*}
\end{solution}
\end{exercise}
\end{sverbatim}
\noindent See the demo file \textsf{webeqtst.tex} for a complete listing of this
exercise.

\redpoint Questions and solutions are kept together \textit{\`a la
Knuth}. The solutions are written  to the file
\cs{jobname.sol} verbatim then input back using
\cs{includeexersolutions}.


\redpoint You can redefine the counter to include the section
number.  For example, the code
\begin{sverbatim}
\renewcommand{\theeqexno}{\thesection.\arabic{eqexno}}
\end{sverbatim}
\noindent can be placed in the preamble of your document. In this
case, the above exercise would appear as
\textsf{Exercise~\thesection.\ref*{ex:int}}.

\redpoint The usual cross-referencing mechanisms for \LaTeX, i.e.,
using \cs{ref} and \cs{pageref}, work as expected.

\medskip\noindent
For example, the label `\verb+\label{ex:int}+' was placed
just after the \verb+\begin{exercise}+ on the previous page. Let us now
reference Exercise~\ref{ex:int}, on page~\pageref{ex:int}.
\begin{quote}
\begin{sverbatim}
let us now reference Exercise~\ref{ex:int},
on~\pageref{ex:int}.
\end{sverbatim}
\end{quote}
\noindent Of course, the nicer looking variations can be done as well.
For example, see \hyperref[ex:int]{\textsf{Exercise~\ref*{ex:int}}}.
\begin{center}
\verb+\hyperref[ex:int]{\textsf{Exercise~\ref*{ex:int}}}+
\end{center}
\noindent The \texttt*-form of \cs{ref} was used to turn off the
redundant link creation. (\textsf{hyperref} would normally make
the \cs{ref} macro into a link.)

\redpoint An `\textsf{Exercise}' that is also a hypertext link
appears in the default color \textcolor{webgreen}{green}; if an
`\textsf{Exercise}' is not a link, it appears in \textcolor{blue}
{blue}. (The word `\textsf{Exercise}' is not a link if it is an
\hyperref[sss:exwparts]{exercise with parts}, or if the
\texttt{\hyperref[sss:nosolutions]{nosolutions}} option is used.
Finally, if the \texttt{web} option
\texttt{\hyperref[sss:webforpaper]{forpaper}} is used, color is
turned off and `\textsf{Exercise}' appears in black.

\redpoint\textbf{\textcolor{red}{Caveat:}} There is one problem
you might watch for.  There is an optional argument to the
\texttt{solution} environment. When \LaTeX{} searches the source
looking for the optional parameter, which may not exist, it expands
macros looking for a `\texttt['.  This causes problem when you have
a solution that begins with a math display environment and \LaTeX{}
prematurely expands such an environment.

\begin{exercise}
Write an equation of a line that crosses the $x$- and $y$-axes at 1.
\begin{solution}
\relax\begin{equation*}
    \boxed{x+y=1}
\end{equation*}
\end{solution}
\end{exercise}

\noindent To prevent \LaTeX{} errors that will stop the compilation,
just place a \cs{relax} prior to the math environment. The code for
the previous exercise is
\begin{sverbatim}
\begin{exercise}
Write an equation of a line that crosses
the $x$- and $y$-axes at 1.
\begin{solution}
\relax\begin{equation*}
    \boxed{x+y=1}
\end{equation*}
\end{solution}
\end{exercise}
\end{sverbatim}


\noindent This is only necessary if the solution \textbf{does not} begin with text.

\subsubsection{Exercises with Parts: The \texttt{exercise*} Environment}\label{sss:exwparts}

Beginning with version 6.07, the \texttt{exercise*} environment is used to create exercises with multiple parts.\footnote{For users of exerquiz, this is equivalent
to the \texttt*-option with the \texttt{exercise} environment. The \texttt{exercise*} environment is the preferred form
but the \texttt*-option works as before.}

A companion environment to \texttt{exercise*} is the
\texttt{parts} environment, use use to enclose the multiple parts of
the question. The \texttt{parts} environment takes one optional
argument, the number of columns to be used. The argument must be a
positive integer greater than 1, in this case, a tabular
environment is used, with the number of columns equal to the
specified argument. If no optional parameter is given, then a list
environment is used.

\begin{sverbatim}
\begin{exercise*}
Preamble for your multi-parted question.
\begin{parts}                 % begin listing of the parts
\item First question.
\begin{solution}
Solution to first question.
\end{solution}
...
...
\item Final question.
\begin{solution}
Solution to the final question.
\end{solution}
\end{parts}                   % end listing of parts
\end{exercise*}
\end{sverbatim}


\noindent The following exercise illustrates this option.  This example appears in
\texttt{\href{\bUrl/webeq_ex.html\#webeqtst}{webeqtst.tex}}.

\begin{exercise*}
Suppose a particle is moving along the $s$-axis, and that its position
at any time $t$ is given by $s=t^2 - 5t + 1$.
\begin{parts}
\item Find the velocity, $v$, of the particle at any time
$t$.\label{item:part}
\begin{solution}
Velocity is the rate of change of position with respect to time. In
symbols:
$$
                    v = \frac{ds}{dt}
$$
For our problem, we have
$$
        v = \frac{ds}{dt} =\frac d{dt}(t^2 - 5t + 1) = 2t-5.
$$
The velocity at time $t$ is given by $\boxed{v=2t-5}$.
\end{solution}
\item Find the acceleration, $a$, of the particle at any time $t$.
\begin{solution}
Acceleration is the rate of change of velocity with respect to time.
Thus,
$$
                    a = \frac{dv}{dt}
$$
For our problem, we have
$$
        a = \frac{dv}{dt} =\frac d{dt}(2t-5)=2.
$$
The acceleration at time $t$ is constant: $\boxed{a=2}$.
\end{solution}
\end{parts}
\end{exercise*}



\noindent There is also an option for listing multiparts questions in tabular form.

\begin{exercise*}\relax\label{ex:tabular}
Simplify each of the following expressions in the complex number system.
\textit{Note}: $\bar z$ is the conjugate of $z$; $\operatorname{Re} z$ is the real
part of $z$ and $\operatorname{Im} z$ is the imaginary part of $z$.
\begin{parts}[2]
\item $i^2$
\begin{solution}
$i^2 = -1$
\end{solution}
&
\item $i^3$
\begin{solution}
$i^3 = i i^2 = -i$
\end{solution}
\\
\item $z+\bar z$
\begin{solution}
$z+\bar z=\operatorname{Re} z$
\end{solution}
&
\item $1/z$
\begin{solution}
$\displaystyle\frac 1z=\frac 1z\frac{\bar z}{\bar z}=\frac z{z\bar z}=\frac z{|z|^2}$
\end{solution}
\end{parts}
\end{exercise*}



\noindent The syntax is the same as an exercise with multiparts.
\begin{sverbatim}
\begin{exercise*} % <- star indicates multipart
Simplify each...
\begin{parts}[2]  % <- optional argument indicates tabular
\item $i^2$
\begin{solution} $i^2 = -1$ \end{solution}
&
\item $i^3$ \begin{solution} $i^3 = i i^2 = -i$\end{solution}
\\
\item $z+\bar z$
\begin{solution} $z+\bar z=\operatorname{Re} z$\end{solution}
&
...
\end{solution}
\end{parts}
\end{exercise*}
\end{sverbatim}

\redpoint This problem style does not obey the
\texttt{solutionsafter} option. (See the section entitled
\Nameref{sss:solutionsafter}).

\exPDF{webeqtst} The sample file
\texttt{\href{\bUrl/webeq_ex.html\#webeqtst.tex}{webeqtst.tex}}
contains this particular example.

\subsection{Options of the \texttt{exercise} Environment}

\subsubsection{Leaving Vertical Space instead of a
Solution}\label{sss:leavevspace}

The \texttt{exercise} environment can be used for test
construction. Initially, you may want to pose a questions and
leave space beneath for the student to write in an answer.

\newtopic The \texttt{solutions} environment has an optional parameter for
inserting a vertical space.
\begin{sverbatim}
    \begin{exercise}
    This is the question.
    \begin{solution}[1in]    % <-- optional vertical skip
    This is the solution.
    \end{solution}
    \end{exercise}
\end{sverbatim}

\noindent This vertical space only appears when the
\texttt{\hyperref[sss:nosolutions] {nosolutions}} option is in
effect.

Within the context of test construction, write the test (including
the solutions), then publish it with the \texttt{nosolutions}
option (leaving vertical spaces as appropriate), then publish the
key with the \texttt{\hyperref[sss:solutionsafter]
{solutionsafter}} option.\footnote{If \texttt{solutionsafter} and
\texttt{nosolutions} both appear in the option list,
\texttt{solutionsafter} overrides \texttt{nosolutions}.}

\redpoint The optional parameter for the solution is ignored for
exercises with parts having a tabular format
(\hyperref[ex:tabular]{Example~\ref{ex:tabular}} is an example of
a tabular multipart exercise).


\subsubsection{Hiding some Solutions}\label{sss:hide}

A subset of the solutions can be hidden by using the `\texttt{h}'
option.  This option is an option of the \texttt{exercise}
environment, as well as an option of \cs{item}, when there is an
exercise with parts. For example, the following code
\begin{sverbatim}
\begin{exercise}[h]  % <- hide solution
Give an example of a set that is \textit{clopen}.
\begin{solution}
The real number line is both closed and open in the
usual topology of the real line.
\end{solution}
\end{exercise}
\end{sverbatim}
\noindent
yields the following exercise.

\begin{exercise}[h]  % <- hide solution
Give an example of a set that is \textit{clopen}.
\begin{solution}
The real number line is both closed and open in the
usual topology of the real line.
\end{solution}
\end{exercise}

Notice that there is no hypertext link to the solution; indeed,
the solution was not even written to the \cs{jobname.sol} file.

The `\texttt{h}' option works with exercises with parts as well.
Just apply the `\texttt{h}' option to the \cs{item}:

\begin{sverbatim}
\begin{exercise*}
A particle has position $s=t^2 - 5t + 1$ at time $t$.
\begin{parts}

\item Find the velocity, $v$, at time $t$.
\begin{solution}
$v = 2t-5$.
\end{solution}

% This solution will not be included in the solutions
% section at the end of the document.
\item[h] Find the acceleration, $a$, at time $t$.
\begin{solution}
$a = 2$.
\end{solution}
\end{parts}
\end{exercise*}
\end{sverbatim}

\noindent The results of this code follow:

\begin{exercise*}
A particle has position $s=t^2 - 5t + 1$ at time $t$.
\begin{parts}

\item Find the velocity, $v$, at time $t$.
\begin{solution}
$v = 2t-5$.
\end{solution}

\item[h] Find the acceleration, $a$, at time $t$.
\begin{solution}
$a = 2$.
\end{solution}
\end{parts}
\end{exercise*}

Part (a) is hypertext linked to its solution, whereas part (b) is
blue, indicating there is no link there.

\redpoint Multipart exercises in the tabular format behave the same way; use
\cs{item[h]} to ``hide'' a solution.

\redpoint There is also an `\texttt{H}' option as well. Specifying
`\texttt{H}' also hides the solutions. See the next two sections for a discussion
of revealing the solutions marked by either `\texttt{h}' or `\texttt{H}' to understand
the distinction between the two.


\subsubsection{The \texttt{nohiddensolutions} Option}\label{sss:nohideopt}

Hidden solutions can be included in the document by either
removing the `\texttt{h}' option everywhere and re-\LaTeX ing, or
by simply using the \texttt{nohiddensolutions} of
\textsf{exerquiz}.
\begin{sverbatim}
   \usepackage[nohiddensolutions]{exerquiz}
\end{sverbatim}

\noindent This option overrides the local `\texttt{h}' option throughout the
document.

\redpoint When the \texttt{solutionsafter} option of \textsf{exerquiz} is invoked,
the hidden solutions are also revealed.  To keep the solutions hidden, in this case,
you should use `\texttt{H}' option instead of `\texttt{h}'. See the next section.

\subsubsection{The \texttt{noHiddensolutions} Option}\label{sss:noHideopt}

In addition to the `\texttt{h}', you can also use the `\texttt{H}'
option with exercises. The solution will be hidden with
`\texttt{H}', but will not be revealed when either the
\texttt{nohiddensolutions} or the \texttt{solutionsafter}
options are used.

The `\texttt{H}' option can be overridden by using the
\texttt{noHiddensolutions} of \textsf{exerquiz}.
\begin{sverbatim}
   \usepackage[noHiddensolutions]{exerquiz}
\end{sverbatim}

\noindent This option overrides the local `\texttt{h}' option throughout the
document.


\subsubsection{The \protect\texttt{exercise} environment Counter}\label{sss:eqexno}

The counter for the \texttt{exercise} environment is \texttt{eqexno},
and will number your exercises consecutively throughout the
document.  Should you want the counter to be reset after each
\texttt{section}, place in the preamble of your document
the following lines:
\begin{sverbatim}
\makeatletter
\@addtoreset{eqexno}{section}
\makeatother
\end{sverbatim}

\subsubsection{The
\texorpdfstring{\protect\texttt{nosolutions}}{nosolutions}
Option}\label{sss:nosolutions}

Some educators may initially want to post a series of exercises on the
Web without the solutions.  Then, at a later date, repost the
exercises with the solutions included.  For this application there
is the \texttt{nosolutions} option for the \textsf{exerquiz}
package.
\begin{sverbatim}
\documentclass{article}
\usepackage[pdftex]{web}  % dvipsone, dvips or dvipdfm
\usepackage[nosolutions]{exerquiz}
\end{sverbatim}
\noindent For this kind of application, it might make sense to publish the exercises
with the \texttt{forpaper} option.


\subsubsection{The
\texorpdfstring{\texttt{solutions\-after}}{solutionsafter}
Option}\label{sss:solutionsafter}


For additional flexibility with how you want the solutions to the
exercises presented, there is a \texttt{sol\-u\-tions\-af\-ter} option
with \textsf{exerquiz}.  Should you invoke this option,
\begin{sverbatim}
\documentclass{article}
\usepackage[dvipsone]{web}  % dvips or pdftex
\usepackage[solutionsafter]{exerquiz}
\end{sverbatim}
\noindent the solutions to the exercises appear just \emph{after} the
exercise question.  For example,

\SolutionsAfter
\begin{exercise}
Let $V$ be a vector space, show that the zero vector, $\mathbf 0$, is
unique.
\begin{solution}
Let $\mathbf 0'$ be a vector that satisfies the axiom of being a
zero of the vector space $V$.  We want to show $\mathbf 0=\mathbf
0'$.  Since $\mathbf 0$ is a zero, we have $\mathbf 0 + \mathbf 0'=\mathbf
0'$.  But we are assuming $\mathbf 0'$ is a zero vector as well,
hence, $\mathbf 0' + \mathbf 0=\mathbf 0$.  Finally,
\[
   \mathbf0' = \mathbf0 + \mathbf0' = \mathbf0' + \mathbf0 = \mathbf0
\]
and this completes the proof.
\end{solution}
\end{exercise}
\SolutionsAtEnd

As you can see in the above example, the word \textit{Solution:} is typeset
at the beginning of the solutions. The command \Com{renameSolnAfterTo} can be used
for conveniently changing the solution after label, for example,
\verb!\renameSolnAfterTo{\textbf{Proof:}}! changes the label to \textbf{Proof:},
and \verb!\renameSolnAfterTo{}! produces no label. These changes will be local
to the group in which they are made, or global of there they are not made in a group.

The command sets \cs{resetSolnAfterToDefault} the label text back to
the default. The default label is
\begin{verbatim}
\newcommand{\eq@exsolafterDefault}{\textit{Solution}:}
\end{verbatim}


\newtopic The option \texttt{solutionsafter} is global; all exercises are
typeset this way---unless you change it within the document using
the macros \cs{SolutionsAfter} and \cs{Solutions\-At\-End}.  This
manual was typeset without the \texttt{solutionsafter} option. The
above example was typeset as follows:
\begin{sverbatim}
\SolutionsAfter  % show solution following exercise
\begin{exercise}
Let $V$ be a vector space, show ...
\begin{solution}
............
\end{solution}
\end{exercise}
\SolutionsAtEnd  % turn back on solutions at of document
\end{sverbatim}

\noindent Normally, a typical document might have all solutions at the end
of the document (the default behavior), or all solutions following
each exercise (\texttt{solutionsafter} option).  Mixtures of these two
types can be obtained by using the two commands
\cs{Solutions\-After} and \cs{Solutions\-At\-End}.

\newtopic This feature might be an easy way of typsetting examples.  See the example
in the section entitled \Nameref{sss:redesignex}.

\redpoint The \texttt{solutionsafter} option has no effect on
multipart exercises in \textit{tabular form};  I haven't been
able to find a convenient way of displaying the solutions after
the questions when the questions are in tabular form.

\exPDF{webeqtst} See the file \texttt{\href{\bUrl/webeq_ex.html\#webeqtst}
{webeqtst.pdf}} for examples.

\subsubsection{Moving the Solution Set}

The solution set, by default, comes last in the file.  You can
move its positioning by including the command
\cs{includeexersolutions} at any point \emph{after} the last
exercise.  You'll note that I have moved the solutions in this
file before the \nameref{references} section, as indicated, for
example, by its position in the table of contents.


\subsection{Redesigning the
\texorpdfstring{\protect\texttt{exercise}}{exercise}
Environment}\label{sss:redesignex}

You can customize the \texttt{exercise} environment to suit your
own needs.  To customize, you need to change some or all of the
following  six commands. In the listing below, the \LaTeX{} definition
of each follows a short description.
\begin{questions}
   \item \cs{exlabel}: This command expands to the name of the exercise label,
   the default string is `Exercise'.
   \begin{sverbatim}\newcommand\exlabel{Exercise}\end{sverbatim}
   \item \cs{exlabelformat}: Typesets the exercise
   label; use it to introduce additional type style such as
   boldface, italic, small caps, etc.
   \begin{sverbatim}\newcommand\exlabelformat{%
      {\scshape\exlabel\ \theeqexno.}}\end{sverbatim}
   \item \cs{exlabelsol}: Expands to the name of the exercise
   label in the solutions section. Usually its value is the same
   as \cs{exlabel}.
   \begin{sverbatim}\newcommand\exlabelsol{\exlabel}\end{sverbatim}
   \item \cs{exsllabelformat}: The format of the solutions label,
   \verb+\bfseries\exlabel+ is the default.
   \begin{sverbatim}\newcommand\exsllabelformat
      {\noexpand\textbf{\exlabelsol\ \theeqexno.}}\end{sverbatim}
   \item \cs{exrtnlabelformat}: This is the label you click on to
   return from the solution of the exercise.
   \begin{sverbatim}\newcommand\exrtnlabelformat{\exlabelsol\ \theeqexno}\end{sverbatim}
   \item \cs{exsectitle}: The section title of the solutions to
   the exercises.
   \begin{sverbatim}\newcommand\exsectitle{Solutions to \exlabel s}\end{sverbatim}
   \item \cs{exsecrunhead}: The running header for the solution
   section for the exercises.
   \begin{sverbatim}\newcommand\exsecrunhead{\exsectitle}\end{sverbatim}
\end{questions}

\redpoint The counter \texttt{eqexno} is used to count exercises.
When an \texttt{exercise} environment starts, this counter is
incremented.  The value of this counter is used in
the definitions of \cs{exlabelformat}, \cs{exsllabelformat} and
\cs{exrtnlabelformat}.  Still, the use of \texttt{eqexno} is
optional; for example, you might want to state a problem just as `Special
Exercise', without an associated exercise number.

Below is an example of redefining the \texttt{exercise}
environment.  We define a \texttt{problem} environment based on
the \texttt{exercise} environment.

\begin{sverbatim}
\newenvironment{problem}{%
\renewcommand\exlabel{Problem}
\renewcommand\exlabelformat{\textbf{\exlabel\ \theeqexno.}}
\renewcommand\exsllabelformat
   {\noexpand\textbf{\exlabel\ \theeqexno}}
\renewcommand\exrtnlabelformat{$\blacktriangleleft$}
\renewcommand\exsecrunhead{\exsectitle}
\begin{exercise}}%
{\end{exercise}}
\end{sverbatim}

\newenvironment{problem}{%
\renewcommand\exlabel{Problem}
\renewcommand\exlabelformat{\textbf{\exlabel\ \theeqexno.}}
\renewcommand\exsllabelformat
   {\noexpand\textbf{\exlabel\ \theeqexno.}}
\renewcommand\exrtnlabelformat{$\blacktriangleleft$}
\renewcommand\exsecrunhead{Solutions to Problems}
\begin{exercise}}%
{\end{exercise}}

\noindent See any standard \LaTeX{} reference on how to define a new
environment, for example \cite{book:Goossens1}.

\newtopic Here is an example of the new \texttt{problem} environment:\medskip

\begin{problem}
This is a question.
\begin{solution}
This is the solution.
\end{solution}
\end{problem}

\noindent
The code for this problem was simply:
\begin{sverbatim}
\begin{problem}
This is a question.
\begin{solution}
This is the solution.
\end{solution}
\end{problem}
\end{sverbatim}

\redpoint Two of these commands, \cs{exsllabelformat} and \cs{exrtnlabelformat}, must be handled with special care. Formatting such as \cs{textbf} or \cs{scseries} must be preceded by a \cs{noexpand}. These commands are written to a file, and must be prevented from expanding.

When you use the \texttt{exercise} environment, the counter
\texttt{eqexno} is automatically incremented by default.  The
\texttt{exercise} does have an optional argument for inserting
your own counter.
\begin{sverbatim}
\begin{exercise}[<ctr>]
.......................
\end{exercise}
\end{sverbatim}
\noindent Where \texttt{<ctr>} is a counter already defined.  This option is
useful if you want to use the \texttt{exercise} environment to
create a new environment with its own numbering scheme, as the
following example illustrates.

This example demonstrates how to define an \texttt{example}
environment with its own counter. For examples, we don't want the
solutions to appear at the end of the file, so we'll use
the commands \cs{SolutionsAfter} and \cs{SolutionsAtEnd}.  All changes are
local.
\begin{sverbatim}
% put a counter in preamble
\newcounter{exampleno}
\newenvironment{example}{%
\renewcommand\exlabel{Example}
\renewcommand\exlabelformat
  {\textbf{\exlabel\ \theexampleno.}}
\renewcommand\exrtnlabelformat{$\square$}
\SolutionsAfter
\begin{exercise}[exampleno]}%
{\end{exercise}
\SolutionsAtEnd}
\end{sverbatim}


\newcounter{exampleno}
\newenvironment{example}{%
\renewcommand\exlabel{Example}
\renewcommand\exlabelformat{\textbf{\exlabel\ \theexampleno.}}
\renewcommand\exrtnlabelformat{$\square$}
\SolutionsAfter
\begin{exercise}[exampleno]}%
{\end{exercise}
\SolutionsAtEnd}

\noindent Now we simply type
\begin{sverbatim}
\begin{example}
What is $2+2$?
\begin{solution}
It is well known that $2+2=4$.
\end{solution}
\end{example}
\end{sverbatim}

\noindent to obtain\parms

\begin{example}
What is $2+2$?
\begin{solution}
It is well known that $2+2=4$.
\end{solution}
\end{example}

\noindent The changes are local to the new \texttt{example} environment. If
we have another exercise, we get a correctly numbered exercise.\parms


\begin{exercise}
What is $2+2$?
\begin{solution}
It is well known that $2+2=4$.
\end{solution}
\end{exercise}

\redpoint The command \cs{exsolafter} typesets the solution label
to the exercise in the case the \texttt{solutionsafter} option is in effect.
The default definition of \cs{exsolafter} is `\verb+\textit{Solution}:+'.
You can redefine it as follows:
\begin{verbatim}
    \renewcommand\exsolafter{\textit{L\"osung}:}
\end{verbatim}
\noindent The command \cs{renameSolnAfterTo} is a convenience macro for changing this label.
We can redefine the label above like so:
This redefinition yields:
\begin{verbatim}
    \renameSolnAfterTo{\textit{L\"osung}:}
\end{verbatim}

\renewenvironment{example}{%
\renewcommand\exlabel{Example}
\renewcommand\exsolafter{\textit{L\"osung}:}
\renewcommand\exlabelformat{\textbf{\exlabel\ \theexampleno. }}
\renewcommand\exrtnlabelformat{$\square$}
\SolutionsAfter
\begin{exercise}[exampleno]}%
{\end{exercise}
\SolutionsAtEnd}

\begin{example}
What is $2+2$?
\begin{solution}
It is well known that $2+2=4$.
\end{solution}
\end{example}

\redpoint There is a special option to the \texttt{exercise}
environment as well, \parss
%
\verb+\begin{exercise}[0]+\\\indent
\verb+.......................+\\\indent
\verb+\end{exercise}+\parss\noindent
%
When the optional argument is \texttt{0} rather than a counter. In
this case, no counter is associated with the environment.  For
example,
\begin{sverbatim}
\newenvironment{project}{%
\renewcommand\exlabel{Project}
\renewcommand\exlabelformat{\textbf{\exlabel. }}
\renewcommand\exsllabelformat
   {\noexpand\textbf{\exlabel\ Hint:}}
\renewcommand\exrtnlabelformat{$\blacktriangleleft$}
\begin{exercise}[0]}%
{\end{exercise}}
\end{sverbatim}

\newenvironment{project}{%
\renewcommand\exlabel{Project}
\renewcommand\exlabelformat{\textbf{\exlabel. }}
\renewcommand\exsllabelformat
   {\noexpand\textbf{\exlabel\ Hint:}}
\renewcommand\exrtnlabelformat{$\blacktriangleleft$}
\renewcommand\exsecrunhead{Hints to Projects}
\begin{exercise}[0]}%
{\end{exercise}}

\noindent Thus, we obtain,

\begin{project}
Find a shorter proof of \textsf{Fermat's Last Theorem}. Do not
look at the project hints until you have finished the
project.
\begin{solution}
There, you didn't need my help after all.
\end{solution}
\end{project}

\noindent The code:
\begin{sverbatim}
\begin{project}
Find a shorter proof of \textsf{Fermat's Last Theorem}. Do not
look at the project hints until you have finished the project.
\begin{solution}
There, you didn't need my help after all.
\end{solution}
\end{project}
\end{sverbatim}

\noindent Note that the solutions are typeset at the end of the file in the
`Solutions to Exercises' section.  At this time, there is no
feature for sorting out these different types of environments;
they are all \texttt{exercise} environments, which is what they are.

\section{The \texttt{shortquiz} Environment}\label{ss:QuizIR}

The \texttt{shortquiz} environment is used to create multiple
choice question and math/text fill-in questions with immediate
response. The discussion of math and text fill-in questions is
post-phoned to \hyperref[s:oQ]{Section~\ref*{s:oQ}}, entitled
\nameref{s:oQ}.  The environment allows redefinition to customize
the look of your quizzes. (See
\mlNameref{sss:redefinesq}.)

% The presentation of the answers will either be in a \texttt{list}
% or a \texttt{tabular} environment, depending on the parameter
% \texttt{num\_cols}. (A \texttt{list} is used if \texttt{num\_col}
% is set to~1.)

\subsection{Basic Usage}\label{BasicUsage}

The syntax for the environment (\texttt{tabular} version) is as
follows:

\begin{sverbatim}
\begin{shortquiz}                % begin shortquiz
...Question goes here...
\begin{answers}{num_cols}        % begin proposed answers
...
\Ans0 <an incorrect answer> &    % a wrong answer
...
\Ans1 <a correct answer> &       % the right answer
...
\end{answers}                    % end listing of answers
\end{shortquiz}                  % end shortquiz
\end{sverbatim}
\noindent The parameter \texttt{num\_cols} is the number of
columns you want to typeset for your multiple choice responses. The
environment sets up a \texttt{tabular} environment if
\texttt{num\_cols} is greater than~1, and a \texttt{list}
environment if \texttt{num\_cols} is~1.

This type of quiz is suitable for asking a short series of
question of the reader, perhaps after explaining some concept.
Quizzes can be used to direct the reader's attention to an
important point.

\redpoint Here is an example of the \texttt{shortquiz}
environment.  Responses are graded without comment using
JavaScript.

\begin{shortquiz}
Which of the following is the $\dfrac{d}{dx}{\sin(x^3)}$?
\begin{answers}{4}
\Ans0 $\sin(3x^2)$ & \Ans0 $\cos(x^3)$ & \Ans1 $3x^2\cos(x^3)$ &
\Ans0 $3x^2\cos(3x^2)$ \\
\end{answers}
\end{shortquiz}

\noindent The verbatim listing follows:
\begin{sverbatim}
\begin{shortquiz}         % begin shortquiz environment
Which of the following is the $\dfrac{d}{dx}{\sin(x^3)}$?
\begin{answers}{4}        % 4 columns of answers
  \Ans0 $\sin(3x^2)$ &    % \Ans0 is a false answer
  \Ans0 $\cos(x^3)$ &
  \Ans1 $3x^2\cos(x^3)$ & % \Ans1 is the correct answer
  \Ans0 $3x^2\cos(3x^2)$
\end{answers}             % end answers environment
\end{shortquiz}           % end shortquiz environment
\end{sverbatim}

\noindent If \texttt{num\_cols} is greater than 1, the
\texttt{answers} sets up a \texttt{tabular} environment,
\verb+p{<width>}+ to sets up the column widths. The \cs{parbox}es are
typeset ragged right.


\redpoint Below is a two-column example in which the posed
alternatives are rather long.   The \texttt{answers} environment
produces is a nicely aligned set of paragraphs.

\goodbreak
\begin{shortquiz}
Which of the following best describes Augustin Cauchy?
\begin{answers}{2}
\Ans0 He developed the Calculus while his University was closed
for the plague. &  % Newton
\Ans0 Given credit for first using the functional notation
$f(x)$.\\ % Euler
\Ans0 He created the ``bell-shaped curve'' and first used the
method of
least squares. & % Guass
\Ans1 He first formulated a precise
definition of the limit and continuity of a function. \\ % Cauchy
\Ans0 Gave a rigorous definition of the definite integral---an
integral
that now bears his name. & % Riemann
\Ans0 His notation for the derivative and the integral is used
even to this day. % Leibniz
\end{answers}
\end{shortquiz}

\noindent Here is the same example in which the \texttt{num\_cols} is set
to~1; in this case, a \texttt{list} environment is used.

\begin{shortquiz}
Which of the following best describes Augustin Cauchy?
\begin{answers}{1}
\Ans0 He developed the Calculus while his University was closed
for the plague.   % Newton
\Ans0 Given credit for first using the functional notation
$f(x)$. % Euler
\Ans0 He created the ``bell-shaped curve'' and first used the
method of
least squares.  % Guass
\Ans1 He first formulated a precise
definition of the limit and continuity of a function.  % Cauchy
\Ans0 Gave a rigorous definition of the definite integral---an
integral
that now bears his name.  % Riemann
\Ans0 His notation for the derivative and the integral is used
even to this day. % Leibniz
\end{answers}
\end{shortquiz}

\exPDF{webeqtst} See the sample file
\texttt{\href{\bUrl/webeq_ex.html\#webeqtst} {webeqtst.tex}} for
examples.

\redpoint The space between the label and the text of a multiple
choice question change be adjusted using two commands,
\Com{setMClabelsep} and \Com{resetMClabelsep}. The default
horizontal space between the label and text is a hard space `\verb!\ !'.
The first command takes as its argument
anything that creates a horizontal white space, for example,
\verb!\setMClabelsep{\quad}! or \verb!\setMClabelsep{\kern20pt}!. If
\cs{setMClabelsep} is set in a group, its value will revert to the
default on exit from the group. The second command
\cs{resetMClabelsep} resets the space back to its default `\verb!\ !'.

The space above the beginning of the \texttt{answer} environment can be
adjusted using the length \Com{aboveanswersSkip}. The default value for
this is
\begin{verbatim}
\setlength\aboveanswersSkip{3pt}
\end{verbatim}
and this can be reset to any desired vertical skip.


\subsubsection{\texttt{shortquiz} with Radio Buttons}\label{sqwRadio}

The short quizzes (with multiple choices) can also be laid out using
radio buttons rather than typeset lettering. Use a
\texttt{shortquiz*} environment, with an optional argument the value
of which is a unique name (which will be used to construct the
titles of the radio buttons).\footnote{In previous versions of \textsf{exerquiz}, there was a
\texttt*-option that was use to signal the use of radio buttons.  This option is still
available, but the shortquiz* environment is preferred.} If you do not provide an optional
argument name, one will programmatically be provided for you.

\newtopic For example, the following code

\begin{sverbatim}
\begin{shortquiz*}[KublaKhan1]
Was it in Xanadu did Kubla Kahn a stately pleasure dome decree?
\begin{answers}{4}
\Ans1 True & \Ans0 False
\end{answers}
\end{shortquiz*}
\end{sverbatim}
\noindent yields the following question:

\begin{shortquiz*}[KublaKhan1-1]
Was it in Xanadu did Kubla Kahn a stately pleasure dome decree?
\begin{answers}{4}
\Ans1 True & \Ans0 False
\end{answers}
\end{shortquiz*}
\noindent Check the functionality of this question, and contrast it with
the same question.

\sqTurnOffAlerts \sqCorrections

\begin{shortquiz*}[KublaKhan2]
Was it in Xanadu did Kubla Kahn a stately pleasure dome decree?
\begin{answers}{4}
\Ans1 True & \Ans0 False
\end{answers}
\end{shortquiz*}
\noindent We have inserted two new commands prior to this last
short quiz \cs{sqTurnOffAlerts} and \cs{sqCorrections} to change
response feedback. The former turns off the alerts and the latter
turns on the corrections: check for a correct answer and an cross
for an incorrect answer. (It doesn't make sense to
\cs{sqTurnOffAlerts} without \cs{sqCorrections};
\cs{sqCorrections} can be used without turning off the alerts.)

\redpoint These two commands only apply to a short quiz that uses radio buttons.
You can reverse these two commands with \cs{sqTurnOnAlerts} and
\cs{sqNoCorrections}, respectively. These settings are the
defaults of the \texttt{shortquiz} with check boxes.

\subsubsection{\texttt{shortquiz} with Solutions}\label{ss:QuizS}

Another type of quiz that is easy to implement in PDF is the
multiple choice quiz with immediate response with solution given.
This too is a \texttt{shortquiz} environment:
\begin{sverbatim}
\begin{shortquiz}
...Question goes here...
\begin{answers}[[<name>]|*]{<num_cols>}
...
\Ans0 <an incorrect answer> &
...
\Ans1 <a correct answer> &
...
\end{answers}
\begin{solution}
...Solution to correct answer goes here...
\end{solution}
\end{shortquiz}
\end{sverbatim}

\newtopic The \texttt{<name>} is a name used to create a hypertext jump to
the solution; \texttt{<name>} will be the ``named destination.'' If \texttt{*}
is used, \textsf{exerquiz} automatically generates the name. Use \texttt{<name>}
when you want to refer to a solution in another part of the document; otherwise,
use \texttt{*}, for convenience.

\newtopic As before, \texttt{<num\_cols>} is the number of columns to
typeset the answers in.

\newtopic The following example illustrates the quiz with
solution.

%\renewcommand\sqslrtnlabel{\string\raisebox{1pt}{$\blacktriangleleft$}Return to Quiz}

\begin{shortquiz}
Define a function $f(s)=4s^3$ and another function
$F(t)=t^4$. Is $F$ an antiderivative of $f$?
\begin{answers}[quiz:anti]{4}
\Ans1 Yes &\Ans0 No
\end{answers}
\begin{solution}
The answer is `Yes'.  The definition requires that
$$
           F'(x) = f(x) \quad\text{for all $x$,}
$$
well, let's check it out.

The definition of $f$ is $f(s)=4s^3$ and so $f(x)=4x^3$.

The definition of $F$ is $F(t)=t^4$ and so, by the rules of
differentiation, $F'(t)=4t^3$.  Thus, $F'(x)=4x^3$. Therefore,
$$
       F'(x) = 4x^3 = f(x)\quad\text{for all $x$,}
$$
as required by the definition.
\end{solution}
\end{shortquiz}

\newtopic The verbatim listing:
\begin{sverbatim}
\begin{shortquiz}
Define a function $f(s)=4s^3$ and another
function $F(t)=t^4$. Is $F$ an antiderivative of $f$?
\begin{answers}[quiz:anti]{4}
\Ans1 Yes &\Ans0 No
\end{answers}
...solution not shown...
\end{solution}
\end{shortquiz}
\end{sverbatim}

\newtopic Here is a listing of the same quiz using the \texttt{*} to
automatically assign a destination label.
\begin{sverbatim}
\begin{shortquiz}
Define a function $f(s)=4s^3$ and another
function $F(t)=t^4$. Is $F$ an antiderivative of $f$?
\begin{answers}*{4}
\Ans1 Yes &\Ans0 No
\end{answers}
...solution not shown...
\end{solution}
\end{shortquiz}
\end{sverbatim}



\subsubsection{The \texorpdfstring{\protect\cs{bChoices}/\-\protect\cs{eChoices}}
    {\textbackslash bChoices/\textbackslash eChoices} Commands}\label{bChoices}


Beginning with \textsf{exerquiz} version~6.03, the command pair
\cs{bChoices} and \cs{eChoices} are defined to help lay out
the list of choices for a multiple choice question. This pair of
macros is not really an environment, the \cs{eChoices} really
does nothing other than to act as an ending marker. The idea
behind this pair of macros is twofold:
\begin{questions}

\item To provide a convenient way of listing alternate choices for
a multiple choice questions, the ``environment'' makes it easy to
change the number of columns, or to change from tabular to list
(or visa-versa);

\item To lay out the alternatives in a uniform manner, which will
make it easy to develop techniques of randomizing the
alternatives.

\end{questions}
Here is an example of usage:
\begin{sverbatim}
\begin{answers}{4}                  %<- 4 columns
\bChoices[2]                        %<- use only first 2 columns
  \Ans0 a choice\eAns               %<- choices delimited by
  \Ans1 another choice\eAns         %   \Ans...\eAns
  \Ans0 still another choice\eAns
  \Ans0 another\eAns
  \Ans0 incoming\eAns
  \Ans0 more choices\eAns
  \Ans0 another still\eAns
  \Ans0 too many\eAns
  \Ans0 choices\eAns
\eChoices
\end{answers}
\end{sverbatim}
\noindent Note that the choices begin with the \cs{Ans} macro
followed by a 0 or 1, as described in the previous pages.  The
end of the choice is delimited by the \cs{eAns} command (which is
undefined, but is used as a argument delimiter). Note that it is
not necessary (and would be a mistake) to place the column
delimiter~`\texttt\&', or the end of line
command~`\texttt{\string\\}', this is done by the \cs{bChoices}
command.

The commands \cs{bChoices} and \cs{eChoices} can be used within
the \texttt{answers} environment as part of a \texttt{shortquiz} or a
\texttt{\hyperref[ss:GradedQuizzes]{quiz}} environment.

Concerning the example above, the argument of the \texttt{answers}
environment is 4 which means we are going to use the tabular
environment with 4 columns. Now within those 4 columns we are
going to only use the first 2 columns (this is the optional
argument of \cs{bChoices}). If the optional argument is removed
from \cs{bChoices}, the choices are typeset in the with 4
columns.

If the argument of \cs{answers} is changed to 1, the
optional argument of \cs{bChoices} is ignored, and the
alternatives are typeset in a one column list environment.

By changing the two parameters (the one for \cs{answers} and the
one for \cs{bChoices}) you can easily modify how the
alternatives are typeset.

\redpoint \cs{rowsep}: Space between rows (or items in list environment)
can be adjusted with the \cs{rowsep} command. If you
say \verb+\rowsep{3pt}+, an additional \texttt{3pt} space is added
between rows or items in a list environment. The default is \texttt{0pt}. The default is
controlled through the command \cs{rowsepDefault}, like so,
\begin{sverbatim}
\rowsepDefault{0pt}
\end{sverbatim}
\noindent For example,
\begin{sverbatim}
\begin{answers}{4}
\rowsep{3pt}
\bChoices[2]
  \Ans0 a choice\eAns
  \Ans1 another choice\eAns
  \Ans0 still another choice\eAns
  \Ans0 another\eAns
  \Ans0 incoming\eAns
\eChoices
\end{answers}
\end{sverbatim}
\noindent Do not place this command inside the
\cs{bChoices}/\cs{eChoices} pair, the above example shows
the optimal placement, just before \cs{bChoices}.  This
separation {\cs{rowsep}} is reset to the default at the end of the
listing. To get a uniform separation between rows across the
document, redefine \cs{rowsepDefault} in the preamble, like so,
\begin{sverbatim}
\rowsepDefault{3pt}
\end{sverbatim}


\redpoint No commands should appear \textit{between} the pair
\cs{bChoices} and \cs{eChoices} and \textit{outside} an
\cs{Ans} and \cs{eAns} pair.  All such tokens are gobbled up.

\newtopic\textcolor{red}{Limitations:} The command \cs{eAns} acts as
a macro parameter delimiter: Everything between \cs{Ans} and \cs{eAns}
is taken in as the argument of a macro; therefore, verbatim content
generated by \cs{verb} is not permitted by {\LaTeX}. If you want
your choices to contain verbatim text, use the style as described in
\Nameref{BasicUsage}.

\subsubsection{The
\texorpdfstring{\protect\texttt{questions}}{questions} Environment}

The \texttt{questions} environment was designed to work with the
\texttt{quiz} environment---taken up in
\hyperref[ss:GradedQuizzes]{Section~\ref*{ss:GradedQuizzes}}
below---but it works equally well with \texttt{shortquiz}.

Using the \texttt{questions} environment, quizzes defined by
\texttt{shortquiz}, with/without solutions, can be
mixed together and combined to make a ``mini-quiz''.  For example,

\begin{shortquiz} Determine the LCD for each of the following.
\begin{questions}
\item  $\dfrac{3x}{2y^2z^3}-\dfrac2{xy^3 z^2}$.
\begin{answers}{2}
\bChoices
    \Ans0 $\text{LCD}=2xy^5z^5$\eAns
    \Ans0 $\text{LCD}=2y^3 z^3$\eAns
    \Ans1 $\text{LCD}=2x y^3 z^3$\eAns
    \Ans0 $\text{LCD}=2 x y^3 z^5$\eAns
\eChoices
\end{answers}
\item  $\dfrac{x+y}{3 x^{3/2}y^2}-\dfrac{x^2+y^2}{6 x y^4}$.
\begin{answers}[quiz:LCB]{2}
\bChoices
    \Ans0 $\text{LCD}=18 x^{3/2}y^4$\eAns
    \Ans1 $\text{LCD}=6x^{3/2}y^4$\eAns
    \Ans0 $\text{LCD}=18 x y^4$\eAns
    \Ans0 $\text{LCD}=6 x y^4$\eAns
\eChoices
\end{answers}
\begin{solution}
If you erred on this one, more than likely it was on the
appropriate multiplicative constant: 6 not 18.  At least
that's what I'm betting on.

The instructions of the LCD Algorithm said to \textit{completely
factor the denominator}.  Here's a list of the factors
$$\def\us#1#2{\underbrace{#2}_{\text{#1 term}}}
         \us{first}{3, x^{3/2}, y^2}\,,\,\us{second}{2, 3, x, y^4}
$$
Let's rearrange them
$$
                 2, 3, 3, x, x^{3/2}, y^2, y^4
$$
Now drop duplicate factors---that's the 3.  Oops! I did mention
dropping identical factors, didn't I?
$$
                   2, 3, x, x^{3/2}, y^2, y^4
$$
Now, group together all terms which have the same base, then drop,
from each of these groups all terms but the one with the highest
power.  We obtain then,
$$
                      2, 3, x^{3/2}, y^4
$$
The LCD is the product of same:
$$
    \text{LCD}=(2)(3) x^{3/2} y^4 = 6 x^{3/2} y^4.
$$
{\it Solution Notes\/}: Alternative (a) will work as a common
denominator, but it is not the least common denominator.  If you
use (a), you will be working with larger numbers than is really
necessary.
\end{solution}
\end{questions}
\end{shortquiz}

\noindent The first question is given without a solution, the second has a
solution attached to it. An abbreviated verbatim listing follows.

\begin{sverbatim}
\begin{shortquiz}
Determine the LCD for each of the following.
\begin{questions}
\item  $\dfrac{3x}{2y^2z^3}-\dfrac2{xy^3 z^2}$.
\begin{answers}{2}
...
\end{answers}
\item  $\dfrac{x+y}{3 x^{3/2}y^2}
          -\dfrac{x^2+y^2}{6 x y^4}$.
\begin{answers}[quiz:LCB]{2}
...
\end{answers}
\begin{solution}
If you erred on this one, ... ...
\end{solution}
\end{questions}
\end{shortquiz}
\end{sverbatim}

\Important The questions environment can be nested three deep.

\newtopic Between items, you can say \Com{pushquestions}, make a comment, to continue
with more questions, type \Com{popquestions}. Here is an example without using \texttt{shortquiz} environment.

\begin{questions}
  \item This is my first question.
  \item This is my second question.
\pushquestions
We now begin a multipart question.
\popquestions
    \item Answer each of the following:
    \begin{questions}
        \item First part of the question.
\pushquestions
Be sure to concentrate on this next question, its a hard one; recall that $1+1=2$.
\popquestions
        \item Second part of the question.
        \begin{questions}
            \item This is a subpart of this part.
\pushquestions
The questions environment can be nested up to three deep.
\popquestions
            \item This is a second subpart of this part.
        \end{questions}
    \end{questions}
    \item This is the fourth question.
\end{questions}

\newtopic A partial listing of the above environment is presented below.
\begin{sverbatim}
\begin{questions}
  \item This is my first question.
  \item This is my second question.
\pushquestions
We now begin a multipart question.
\popquestions
    \item Answer each of the following:
...
\end{questions}
\end{sverbatim}

\subsection{Options of the \texttt{shortquiz} Environment}\label{sq:opts}

\subsubsection{The
\texorpdfstring{\protect\texttt{forpaper}}{forpaper} Option}

The \texttt{forpaper} option has already been described. The
solutions to \texttt{short\-quiz} questions are not typeset on
separate pages, but are separated by a \cs{medskip}.

Following up on the pre-test angle discussed in
``\mlnameref{sss:redefinesq}'', page~\pageref*{sss:redefinesq}, a single
document can be constructed that can be published on-line, or
published for paper distribution.  This feature may be useful to
some educators.



\subsubsection{The \texttt{solutionsafter} Option}

The \texttt{solutionsafter} option works as described for the
\texttt{exercise} environment.  The option just sets a Boolean
switch.  This switch can be controlled  with
\cs{Solutions\-After} and \cs{SolutionsAtEnd}.  Here is a simple
example.

\SolutionsAfter
\begin{shortquiz}
In what year did Columbus sail the ocean blue?
\begin{answers}[qz:columbus2]{4}
\Ans0 1490 &\Ans0 1491 &\Ans1 1492 &\Ans0 1493
\end{answers}
\begin{solution}
Columbus sailed the ocean blue in 1492.  Some say he discovered
San Salvatore, others say he first sited Cat Island in the
Bahamas.
\end{solution}
\end{shortquiz}
\SolutionsAtEnd
In the example above, \cs{SolutionsAfter} was placed before the \texttt{shortquiz} environment and
the \cs{SolutionsAtEnd} was placed just after.

This option may be useful in publishing an answer key to a multiple
choice quiz.  The quiz and solutions can be created together.  The
quiz can be published, then later, the quiz with complete
solutions.


\subsubsection{The \texorpdfstring{\texttt{proof\-ing}}{proofing}
Option}\label{sss:proofing}

For proofreading, use the \texttt{proofing} option of
\textsf{exerquiz}.
\begin{sverbatim}
    \usepackage[proofing]{exerquiz}
\end{sverbatim}

\noindent When used, a symbol, defined by the command \cs{proofingsymbol},
will mark the correct answers, as defined in your source file. The
command \cs{proofingsymbol} may be redefined, its definition is
\begin{sverbatim}
\newcommand\proofingsymbol{\textcolor{webgreen}{$\bullet$}}
\end{sverbatim}
\noindent
This option works for the
\texttt{\hyperref[ss:GradedQuizzes]{quiz}} environment defined
below
(\hyperref[ss:GradedQuizzes]{page~\pageref*{ss:GradedQuizzes}}),
as well.

\subsubsection{The \texttt{showgrayletters} Option}\label{showgraylettersOpt}  \aebCng[showgrayletters]

When \texttt{showgrayletters} is used, the form field check boxes
that appear in a \texttt{quiz} or in a \texttt{shortquiz} will have a gray capital letter
\texttt{A}, \texttt{B}, \texttt{C}, etc. underneath it. This letter
can then be referred to in the text or the solution.

This option is global and is controlled by a switch, \cs{ifaebshowgrayletters}. The gray letter feature can be turned on and off locally: To turn on this feature, insert the command \Com{graylettersOn} at some appropriately chosen point in the document; to turn off the gray letter feature insert \Com{graylettersOff}.

\newtopic In support of the \texttt{showgrayletters} option is a new command \Com{REF}.
\cs{REF} acts like the {\LaTeX} command \cs{ref} with the hyperref modifications, but it converts
the reference to uppercase. When \cs{ref} would typeset the letter `a', for example, \cs{REF}
would typeset the letter `A'. \cs{REF}, like \cs{ref}, typesets a hyperref link. Hyperref defines a
\texttt{*} version of \cs{ref}; \cs{ref*} typesets the reference, but does not create a link;
\cs{REF*} does the same.  When \cs{aebshowgraylettersfalse} is in effect, \cs{REF} does not capitalize
the reference.

The \texttt{showgrayletters} feature and the \cs{REF} command work equally
well for the \texttt{quiz} environment.

\redpoint Below is an example of this.

\graylettersOn

\begin{shortquiz*}[TestSQ]
Answer this if you can!
\begin{answers}{2}
\bChoices
    \Ans0\label{testsqFirst}  This is a possible answer.\eAns

    \Ans1\label{testsqSecond} Try this one (the correct one)\eAns

    \Ans0  This is an answer.\eAns

    \Ans0  Another alternative.\eAns
\eChoices
\end{answers}
\end{shortquiz*}

\noindent Reference both (\REF{testsqFirst}) and (\REF*{testsqSecond}).

\graylettersOff

\newtopic The code is

\begin{sverbatim}
\graylettersOn
\begin{shortquiz*}[TestSQ]
Answer this if you can!
\begin{answers}{2}
\bChoices
    \Ans0\label{testsqFirst} This is a possible answer.\eAns
    \Ans1\label{testsqSecond} Try this one (the correct one).\eAns
    \Ans0 This is an answer.\eAns
    \Ans0 Another alternative.\eAns
\eChoices
\end{answers}
\end{shortquiz*}
\noindent Reference both (\REF{testsqFirst}) and (\REF*{testsqSecond}).
\graylettersOff
\end{sverbatim}
\newtopic Notice that the gray letters was  not turned to off
until after the usage of \cs{REF}.

\textcolor{red}{\Important} The gray letters are typeset into the document. Do not
use a background color for the checkboxes as this color will cover
up the gray letters. The default background color checkboxes is
transparent, keep it that way.

\newtopic The command \Com{insertGrayLetters} is used to insert the gray letters;
this command can be redefined as desired. See the source file \texttt{exerquiz.dtx}.
Also see the definition of \Com{bottomOfAnsfStack}.

\subsubsection{Moving the Solution Set}

The solution set, by default, comes last in the file.  You can move
its positioning by including the command \cs{includequizsolutions}
at any point \emph{after} the last exercise.  You'll note that I
have moved the solutions in this file before the
\nameref{references} section, as indicated, for example, by its
position in the table of contents.

\subsection{Redesigning the
\texorpdfstring{\protect\texttt{shortquiz}}{shortquiz}
Environment}\label{sss:redefinesq}

The \texttt{shortquiz} environment can be redesigned to better suit
your needs.  In the paragraphs below, we describe how you can change
titles and form elements.


\subsubsection{Changing Titles}

You can temporarily change the title  for the \texttt{shortquiz}
environment by redefining the macro \cs{sqlabel}; for example, the
default definition of this macro is
\begin{sverbatim}
    \newcommand\sqlabel{\textcolor{red}{Quiz.}}
\end{sverbatim}
\noindent The syntax for redefining \cs{sqlabel} is
\begin{sverbatim}
    \renewcommand\sqlabel{...new code goes here...}
\end{sverbatim}
\noindent You can redefine the \textit{default} label as well; the
default label is the title label that \texttt{short\-quiz} uses when
\cs{sqlabel} is \textit{not present}.  The default label is
\verb+\eq@sqlabel+ and must be redefined using the macro
\cs{renewcommand}. The best place for this to be done is the
preamble. The syntax:
\begin{sverbatim}
\makeatletter    % make `at'=@ a normal letter
\renewcommand\eq@sqlabel{...new code goes here...}
\makeatother     % make `at'=@ something special(other)
\end{sverbatim}
\noindent To change the entire document to use `Exam' instead of `Quiz',
make the following changes in the preamble:
\begin{sverbatim}
\makeatletter
% change default quiz title to 'Exam'
\renewcommand\eq@sqlabel{\textcolor{red}{Exam.}}
% change quiz solutions return label
\renewcommand\eq@sqslrtnlabel{End Exam}
% change solutions label
\renewcommand\eq@sqsllabel{%
   \string\textbf{Solution to Exam:}}
\renewcommand\eq@sqslsectitle{Solutions to Exams}
% change default running header for solutions
\renewcommand\eq@qslsecrunhead{Solutions to Exams}
\makeatother
\end{sverbatim}

\redpoint The above commands are `global'---they are in effect
throughout the entire document. You can temporarily change these
labels using the \cs{sqlabel}, \cs{sqslrtnlabel}, \cs{sqsllabel} and
\cs{sqslsectitle}.  Note that you cannot temporarily change  the running label
\cs{eq@qslsecrunhead}, this should be set in the
preamble.

Should you want to make a series of multiple choice questions
(using the \texttt{questions} environment) and combine them into a
sort of review or pretest, a useful idea would be to number the
solutions.  The counter that maintains the question number is
called \texttt{questionno}. You can then, for example, define
\begin{sverbatim}
\renewcommand\eq@sqsllabel{%
   \string\textbf{Solution to Question \thequestionno:}}
\end{sverbatim}

\exPDF{webeqtst} See the sample file
\texttt{\href{\bUrl/webeq_ex.html\#webeqtst}{webeqtst.tex}}
for examples.

\subsubsection{Modifying Form Elements}\label{sqMFE1}

For quizzes that use radio buttons (see page~\pageref{sqwRadio}
above), the appearance of the radio buttons can be controlled
using the ``every'' mechanism as described in the document
\href{eformman.pdf}{eFormMan.pdf} on
\textit{\href{eformman.pdf}{eForm Support}} for the \cAcroB.
The radio buttons can be modified using the command \cs{everysqRadioButton}.

Prior to the short quiz below, the following command was executed

\begin{sverbatim}
\everysqRadioButton{\BC{.690 .769 .871}\BG{.941 1 .941}}
\end{sverbatim}

\sqTurnOffAlerts \sqCorrections
\everysqRadioButton{\BC{.690 .769 .871}\BG{.941 1 .941}}

\begin{shortquiz*}[KublaKhan3]
Was it in Xanadu did Kubla Kahn a stately pleasure dome decree?
\begin{answers}{4}
\Ans1 True & \Ans0 False
\end{answers}
\end{shortquiz*}

\everysqRadioButton{}
\sqTurnOnAlerts\sqNoCorrections

\noindent Return to the defaults, if desired, by then emitting
\begin{sverbatim}
\everysqRadioButton{}
\end{sverbatim}

\redpoint The short quiz can also have fill-in questions and
various other controls, these are described in
Section~\ref{s:oQsq}, \nameref{s:oQsq}. There too, methods of
modifying the appearance of the form elements are discussed.

\subsection{\texorpdfstring{\protect\cs{titleQuiz}}{\CMD{titleQuiz}}
    and \texorpdfstring{\protect\cs{fancyQuizHeaders}}{\CMD{fancyQuizHeaders}}}\label{titleQuizSQ}

\exPDF{title_quiz}The demo file for these features is \texttt{title\_quiz.tex}.


\subsubsection{\texorpdfstring{\protect\cs{titleQuiz}}{\CMD{titleQuiz}}}

To assign a title to your short quiz use \Com{titleQuiz}. This title
appears just after the typeset word ``Quiz''.

\fancyQuizHeaders
\titleQuiz{Two Question MC}
\begin{shortquiz}\label{myCoolSQ}
Answer each of these to the best of your ability.
\begin{questions}
\item Math is fun.
\begin{answers}*{4}
\Ans1 True & \Ans0 False\\
\end{answers}
\begin{solution}
  The answer is true, of course, math is fun!
\end{solution}

\item Statistics is fun.
\begin{answers}*{4}
\Ans0 True & \Ans1 False\\
\end{answers}
\begin{solution}
  The answer is false, statistics is interesting but not fun.
\end{solution}
\end{questions}
\end{shortquiz}

\noindent The partial verbatim code for the above quiz follows:

\begin{sverbatim}
\titleQuiz{Two Question MC}
\begin{shortquiz}\label{myCoolSQ}
Answer each of these to the best of your ability.
\begin{questions}
...
\end{questions}
\end{shortquiz}
\end{sverbatim}

\restoreDefaultQuizHeaders

\noindent The format of the title is controlled by the command \Com{titleQuizfmt},
the default definition is \verb!\newcommand\titleQuizfmt{\bfseries}!. The value
of the \cs{titleQuiz} command can act as an argument for \cs{titleQuizfmt} if you redefine
\cs{titleQuizfmt} to have an argument. For example,
\begin{sverbatim}
\renewcommand\titleQuizfmt[1]{\bfseries(#1)}
\end{sverbatim}
Now, the title will appear in a set of parentheses. The title and formatting are enclosed
in a \cs{mbox}, so font and style changes are localized to that  \cs{mbox}.

\newtopic If a quiz does not have a title, as defined by \cs{titleQuiz}, the title defaults to
an empty title.

\newtopic The definition of \cs{titleQuiz} is such that the command \cs{makeatletter} is executed
before the argument for \cs{titleQuiz} is read. This means that the special character
`\texttt{@}' can be used within the argument of \cs{titleQuiz}.  For example, the \texttt{shortquiz}
environment maintains a text macro \cs{@shortquizCnt} which contains the current short quiz
count number. Thus you can say \verb!\titleQuiz{Short Quiz \@shortquizCnt}!, the title of
the short quiz will include a short quiz number as well.

The \cs{titleQuiz} is used for quizzes as well as with short quizzes. It is possible to have
different formatting for each of these two environments by using the \Com{ifQuizType}
command. The command takes two arguments:
\begin{sverbatim}
\ifQuizType{<quiz_code>}{<shortquiz_code>}
\end{sverbatim}
For example, we could define \cs{titleQuizfmt} as follows:
\begin{sverbatim}
\renewcommand\titleQuizfmt{\ifQuizType{\bfseries}{\slshape}}
\end{sverbatim}
or with an argument, as mentioned above, you could define \cs{titleQuizfmt} with an argument,
like so,
\begin{sverbatim}
\renewcommand{\titleQuizfmt}[1]{\ifQuizType{\bfseries(#1)}{\slshape[#1]}}
\end{sverbatim}

\subsubsection{\texorpdfstring{\protect\cs{titleQuiz*}}{\CMD{titleQuiz*}}}

For \texttt{shortquiz} environment---not for the \texttt{quiz}
environment---there is a `\texttt*' optional first parameter. When
the `\texttt*' option is taken, the quiz label (\cs{sqlabel}) is
redefined to be the title of the quiz text.

\exPDF{title_quiz} The \texttt{title\_quiz.tex} demo file illustrates the `\texttt{*}' option.

\newtopic Thus, if we title a quiz with the `\texttt{*}', we get

\titleQuiz*{My Cool Quiz}
\begin{shortquiz}
Answer each of these to the best of your ability.
\begin{questions}
\item Math is fun.
\begin{answers}{4}
\Ans1 True & \Ans0 False\\
\end{answers}
\item Statistics is fun.
\begin{answers}{4}
\Ans0 True & \Ans1 False\\
\end{answers}
\end{questions}
\end{shortquiz}
The formatting is that defined by the \cs{titleQuizfmt}, which is \cs{bfseries}. In addition
to the \Com{ifQuizType} command, there is also the \cs{ifstaroption} command. It takes two
parameters
\begin{sverbatim}
\ifstaroption{<code if \titleQuiz*>}{<otherwise>}
\end{sverbatim}
This allows us to not only format for quizzes, but for short quizzes, with or without
the `\texttt*' option. If we redefine \cs{titleQuizfmt} to be
\begin{verbatim}
\renewcommand{\titleQuizfmt}{\ifQuizType{\bfseries}%
    {\ifstaroption{\color{red}}{\bfseries}}}
\end{verbatim}
we format the quiz title as \cs{bfseries} is either a quiz or a
short quiz without `\texttt*' option, and as \verb!\color{red}! with
a short quiz with the `\texttt*' option. The results are seen in the next quiz,
with this redefinition:

\renewcommand{\titleQuizfmt}{\ifQuizType{\bfseries}%
    {\ifstaroption{\color{red}}{\bfseries}}}

\titleQuiz*{My Swave Quiz}
\begin{shortquiz}
Answer each of these to the best of your ability.
\begin{questions}
\item Math is fun.
\begin{answers}{4}
\Ans1 True & \Ans0 False\\
\end{answers}
\item Statistics is fun.
\begin{answers}{4}
\Ans0 True & \Ans1 False\\
\end{answers}
\end{questions}
\end{shortquiz}

\subsubsection{Cross-Referencing}

Quizzes and shortquizzes conform to the cross-referencing system
of {\LaTeX}. For example, we used the label \verb!\label{myCoolSQ}! in the last quiz, the one
on page~\pageref{myCoolSQ}. The title of the quiz is \nameref{myCoolSQ}. The code for the past
two sentences is
\begin{sverbatim}
    For example, we used the label \verb!\label{myCoolSQ}! in the last
    quiz, the one on page~\pageref{myCoolSQ}. The title of the quiz is
    \nameref{myCoolSQ}.
\end{sverbatim}
\noindent If the shortquiz does not have a title, as defined by the \cs{titleQuiz} command, a generic label of
`Quiz' is generated. The default label name is
\begin{sverbatim}
    \newcommand\eq@defaultShortQuizLableName{Quiz}
\end{sverbatim}
\noindent This definition can be change using \cs{renewcommand}.


\subsubsection{\texorpdfstring{\protect\cs{fancyQuizHeaders}}{\CMD{fancyQuizHeaders}}}

The \Com{fancyQuizHeaders} command, a companion feature to \Com{titleQuiz}, enhances
the labeling of the quiz in the solutions section of the document. If
\cs{fancyQuizHeaders} is executed in the preamble, or prior to a quiz, the solution header has the
following default format:

\newtopic{\color{blue}Two Question MC: Question 1.} The answer is true, of course, math is fun!

\newtopic The above assumes the title of the quiz is the one given in the previous section.

\newtopic When a quiz has no title, the header reverts to
\textcolor{blue}{Solution to Quiz:} followed by the question number.

\redpoint You can turn off fancy quiz headers by using
\Com{restoreDefaultQuizHeaders}.

\newtopic You can modify the fancy headers with a combination of setting \Com{setsolnspace}
and/or redefining the command \Com{eq@fancyQuizHeadersfmt}. First
\cs{setsolnspace}: This is a hook into the space immediately
following the header. To insert a \cs{medskip} after the header, for example, you
can say \verb!\setsolnspace{\par\medskip\noindent}! a
\cs{medskip} is created after the solution header and the solution
above appears as

\newtopic{\color{blue}Two Question MC: Question 1.}\par\medskip\noindent The answer is true, of course, math is fun!

\newtopic The definition of \cs{eq@fancyQuizHeadersfmt} is
\begin{verbatim}
\def\eq@fancyShrtQuizHeadersfmt{\eq@fancyQuizHeadersfmt}
\end{verbatim}
The \Com{eq@fancyQuizHeadersfmt} is the \texttt{quiz} counterpart to
this \texttt{shortquiz} construct. The fancy short quiz solutions header
defaults to the same header used for a quiz. You can copy
and paste the definition of \cs{eq@fancyQuizHeadersfmt} into the definition
of the command \cs{eq@fancyShrtQuizHeadersfmt}, then modify as described in
\Nameref{fancyQuizHeadersQ}.

%See the section \Nameref{fancyQuizHeadersQ} for a full discussion of this topic.

% \subsection{Graded Quizzes using JavaScript}
\section{The \texttt{quiz} Environment}\label{ss:GradedQuizzes}

Use the \texttt{quiz} environment to create graded quizzes.  In
this case, several (many) questions are bundled together.  The
student takes the quiz and responses are recorded by
JavaScript.  Upon completion of the quiz, the total score is
reported to the student.

The \texttt{quiz} environment can generate multiple choice
questions and math/text fill-in questions. The discussion of math
and text fill-in questions is postponed to \hyperref[s:oQ]
{Section~\ref*{s:oQ}} on page~\pageref*{s:oQ}


There are two types of quizzes, the
\hyperref[sss:QuizwLinks]{link-style} and
\hyperref[sss:QuizwChecks]{form-style}.
In \hyperref[ss:correct]{Section~\ref*{ss:correct}}, we see that
the \texttt{quiz} environment can also correct the quizzes.

The quiz environment takes one required parameter, the
\texttt{quizfieldname}, which is a string.
\begin{sverbatim}
\begin{quiz}{quizfieldname}
...
\end{quiz}
\end{sverbatim}
\textcolor{red}{\Important} \textsf{Exerquiz} uses the
\texttt{quizfieldname} to build Acrobat form fields, and also JavaScript
variables; therefore, there are restrictions on the choice of the
name. The name must be a (JavaScript) \emph{identifier}, the rules to
be an identifier are these.
\begin{quote}
The first character shall be a letter, an underscore (\texttt{\_}),
or a dollar sign (\texttt{\$}); subsequent characters may be any
letter or digit, or an underscore or a dollar sign.
\end{quote}
Do not use other symbols, like a colon (\texttt{\:}) or a period as part of the field  name,
these will confuse the JavaScript interpreter, and may cause the quiz not to function as
designed.


The \texttt{quiz} environment consists of a series of nested
environments.  Inside the \texttt{quiz} environment is the
\texttt{questions} environment (an enumerated list), and within
that environment is the \texttt{answers} environment.
Symbolically, we can express this as
$$
    \mbox{\texttt{quiz}} \supseteq
    \mbox{\texttt{questions}} \supseteq
    \mbox{\texttt{answers}}
$$
The term `answers' is, perhaps, not sufficiently descriptive;
`alternatives' would be more appropriate, but it requires more
typing.  \verb!{:-{)!

\redpoint The \texttt{answers} environment requires one parameter,
the \texttt{num\_cols}. If \texttt{num\_cols} is~1, a
\texttt{list} environment is created; otherwise, a
\texttt{tabular} environment is used.

This (\texttt{tabular}) environment has the following syntax:

\begin{sverbatim}
\begin{quiz}{quizfieldname}
The preamble to the questions goes here.
\begin{questions}
\item State first question....
\begin{answers}{4} % <- num_cols = 4
\Ans0 ... &\Ans1 ... &\Ans0 ... &\Ans0 ...
\end{answers}
...
\item n th question....
\begin{answers}{4}                     % <- 4 column format
\Ans0 ... &\Ans1 ... &\Ans0 ... &\Ans0 ...
\end{answers}
\end{questions}
\end{quiz}
\end{sverbatim}

\redpoint Following the quiz, or anywhere in the document, place
the macro \cs{ScoreField}, defined in \textsf{exerquiz}, to
display the results of the quiz:
\begin{sverbatim}
\ScoreField{quizfieldname}
\end{sverbatim}
\noindent\textcolor{red}{\bfseries Important.} The value of the parameter of
the macro \cs{ScoreField} must match the \texttt{quiz\-field\-name}
defined in the argument of the \texttt{quiz} environment.

\redpoint There is a convenience macro, \cs{currQuiz}, that holds the name
of the current quiz. Thus, we could have instead typed:
\begin{sverbatim}
\ScoreField\currQuiz
\end{sverbatim}

\noindent Read the paragraph entitled \Nameref{sss:scorefield} for more
details on this macro.

%An alternative to \cs{ScoreField} is to use \cs{TextField}.
%a macro defined by \textsf{hyperref}.  See more details later. %??

\begin{comment}
The \textsf{exerquiz} macro \cs{eqScore} is a text macro that
expands to the phrase `Score:'. (It expands to `Punkte:' with the
\texttt{german} language option and `Score :' with the
\texttt{french}.)


The \cs{TextField} macro takes two arguments, we need only the
first---the second is left empty. This \cs{TextField} can be
modified by changing the options; the layout of the field can be
changed as well using the \cs{LayoutTextField}.  For more
information see the paragraphs on \nameref{ss:modoutputfield}
beginning on page~\pageref{ss:modoutputfield}.   There are
extensive examples on how to set up the \cs{TextField} in the
paragraph \nameref{sss:modtextfield} on
page~\pageref{sss:modtextfield}, with examples following on
page~\pageref{sss:exampltextfield}.
\end{comment}

\subsection{Options of the \texttt{quiz} Environment}

In addition to the options listed in the section \ref{sq:opts}, page \pageref*{sq:opts}, the following is
designed for quizzes.

\subsubsection{The option
\texorpdfstring{\protect\texttt{noquiz\-solutions}}{noquizsolutions}}\label{sss:noquizsolutions}

For online quizzing, where results are stored in some way (database,
e-mail, text file) the presence of the solutions in the same file as the questions
is a breach in security of the quiz.  Using the \texttt{noquizsolutions} removes
the solutions to a \texttt{quiz} and \texttt{shortquiz} from the document under construction.


\subsection{The \texttt{answers} Environment}\label{answersQuiz}

The simplest type of question that can be posed is the multiple choice question.
For this question type, the \texttt{answers} environment is used. The basic syntax
is

\begin{small}
\begin{verbatim}
  \item A question
  \begin{answers}[[<name>]|*]{<num_cols>}
  \Ans[<partial_pts>]{0|1}....
  \end{answers}
\end{verbatim}
\end{small}

\newtopic The \texttt{[<name>]}, an optional argument, is a name used to create a hypertext jump to
the solution; \texttt{[<name>]} will be the ``named destination.'' If \texttt{*}
is used, \textsf{exerquiz} automatically generates the name. Use \texttt{<name>}
when you want to refer to a solution in another part of the document; otherwise,
use \texttt{*}, for convenience.

If either the \texttt{[<name>]} or \texttt{*} is used, a
\texttt{solution} environment should follow.

\newtopic The \texttt{<num\_cols>} is the number of columns to
typeset the answers in. When \texttt{<num\_cols>} is 1, the answers are set
in a list environment; otherwise, a tabular environment is used. In the latter case,
the usual table syntax is used. (Use \texttt{\&} to separate columns, and \verb!\\! to
terminate a row.)

\newtopic The \cs{Ans} command has one required argument, a 1 to mark that this
answer is a correct answer, and a 0 to mark that this answer is not true.  \cs{Ans}
also has an optional argument to pass partial points rewarded for a wrong answer.

\newtopic This environment is illustrated extensively in this section. See also the
\texttt{manswers} environment, \Nameref{manswersQuiz}, for posting questions with multiple
correct answers.

\subsection{Basic Usage}

In this section we discuss the two  \texttt{quiz} styles:
\nameref{sss:QuizwLinks} and \nameref{sss:QuizwChecks}.

A paragraph is devoted to some modifications that can be made at
the beginning and end of the quiz.  In addition, a
\texttt{\hyperref[sss:proofing]{proofing}} option is also
described.

\subsubsection{Link-Style Quiz}\label{sss:QuizwLinks}

%\renewcommand\bqlabel{Begin Exam}
%\renewcommand\eqlabel{End Exam}

%There are actually two variations on this type of quiz, one uses
%links, the other uses radio button fields.

This style uses links to record the choices to the alternatives.
The link method takes up less space in the pdf file than does the
form-style.

Below is an example of a link-style quiz.  Instructions should be
given to guide the student in operating the quiz correctly.

\medskip\noindent\textcolor{red}{Instructions.} You must click on
`Begin Quiz' to  initialize the quiz.  Not doing so brings forth
an error message.  When finished, click on `End Quiz'.

\begin{quiz}{qzdiscrl}
Using the discriminant, $b^2-4ac$, respond to each of the
following questions.
\begin{questions}
\item Is the quadratic polynomial $x^2-4x + 3$ irreducible?
\begin{answers}{4}
\Ans0 Yes &\Ans1 No
\end{answers}
\item Is the quadratic polynomial $2x^2 - 4x + 3 $ irreducible?
\begin{answers}{4}
\Ans1 Yes &\Ans0 No
\end{answers}
\item How many solutions does the equation $2x^2 - 3x - 2= 0$ have?
\begin{answers}{4}
\Ans0 none &\Ans0 one &\Ans1 two
\end{answers}
\end{questions}
\end{quiz}\par
\ScoreField\currQuiz
%\TextField[name=qz:discr,width=3in,default=\eqScore]{}

\redpoint While you are taking the test, and before you click on
`End Quiz', you can change your answers.

%A message box comes out,
%gives you your original choice, and asks you whether you really
%want to change your answer.

\begin{sverbatim}
\begin{quiz}{qzdiscrl}  % qz:discr=quiz field name
Using the discriminant, $b^2-4ac$, respond to each of the
following questions.
\begin{questions}
\item Is the quadratic polynomial $x^2-4x + 3$ irreducible?
\begin{answers}{4}
\Ans0 Yes &\Ans1 No
\end{answers}
\item Is the quadratic polynomial $2x^2-4x+3$ irreducible?
\begin{answers}{4}
\Ans1 Yes &\Ans0 No
\end{answers}
\item How many solutions does the equation $2x^2-3x-2=0$ have?
\begin{answers}{4}
\Ans0 none &\Ans0 one &\Ans1 two
\end{answers}
\end{questions}
\end{quiz}\par
\ScoreField\currQuiz % matching quiz field name
\end{sverbatim}

\redpoint The convenience text macro, \cs{currQuiz}, contains the name
of the current quiz. This macro can be used as the argument of
\cs{ScoreField}.

\subsubsection{Form-Style Quiz}\label{sss:QuizwChecks}

You may be thinking that such a quiz format---one in which the
student cannot see the choices made---is not very good.  It is
perhaps adequate for two or three quick questions.  For a longer
quiz format, one would like to see a ``checkbox'' format.
A quiz with a checkbox format can be obtained using the
\texttt{quiz*} environment:\footnote{In previous versions of \textsf{exerquiz}, there was a \texttt*-option
for this purpose. The \texttt{quiz*} has replaced the \texttt*-option and is the preferred method, though
the \texttt*-option is still defined so as not to break previous code.}
\begin{sverbatim}
   \begin{quiz*}{quizfieldname}
   ...
   ...same format as before...
   ...
   \end{quiz*}
\end{sverbatim}

\noindent Here is the same sample quiz with the form-style option.  The only
change in the code is the insertion of the \texttt*-option.

\medskip\noindent\textcolor{red}{Instructions.} You must click on
`Begin Quiz' to  initialize the quiz.  Not doing so, brings forth
an error message.  When finished, click on `End Quiz'.


\begin{quiz*}{qzdiscrf}
Using the discriminant, $b^2-4ac$, respond to each of the
following questions.
\begin{questions}
\item Is the quadratic polynomial $x^2-4x + 3$ irreducible?
\begin{answers}4
\Ans0 Yes &\Ans1 No
\end{answers}
\item Is the quadratic polynomial $2x^2 - 4x + 3 $ irreducible?
\begin{answers}4
\Ans1 Yes &\Ans0 No
\end{answers}
\item How many solutions does the equation $2x^2 - 3x - 2= 0$ have?
\begin{answers}4
\Ans0 none &\Ans0 one &\Ans1 two
\end{answers}
\end{questions}
\end{quiz*}

\ScoreField{qzdiscrf}

\redpoint Before completing the quiz, a student can easily change
alternatives.

\redpoint This type is more suitable for longer quizzes.  The
choices the student makes are visually recorded for the student to
review and change before clicking on `End Quiz'. A partial
verbatim listing:
\begin{sverbatim}
\begin{quiz*}{qzdiscrf}
Using the discriminant, $b^2-4ac$, respond to each of the
following questions.
\begin{questions}
.............
.............
\end{questions}
\end{quiz*}\par
\ScoreField{qzdiscrf}
\end{sverbatim}

\exPDF{webeqtst} See the sample file
\texttt{\href{\bUrl/webeq_ex.html\#webeqtst} {webeqtst.tex}}
and for examples.

\subsubsection{Using Circular Radio Buttons with MC Questions}

As seen with the examples of the previous section, the form-type quizzes
have a rectangular check box for the user to make his selection. Acrobat also
offers a circular radio button that can be used to make a multiple choice.

To create circular radio buttons for a MC question execute \Com{useMCCircles} before
the quiz, or, perhaps, in the preamble of your document.

To return to the default rectangular choice boxes, execute \Com{useMCCRects}. This
setting is the default, there is no need to executed it unless
\cs{useMCCircles} has already been executed.

Executing \cs{useMCCircles} does not affect the \texttt{manswers} environment, page~\pageref*{manswersQuiz}), so you can have circular radio buttons for multiple choice, and rectangular choice boxes for multiple selection problems.

\exPDF{quiz_w_circ} The sample file \texttt{quiz\_w\_circ.tex} illustrates
these two commands.

\subsubsection{Overriding the `quiztype' Parameter}

You can globally declare that all quizzes to be a link-type or
form-type by using the command \Com{quiztype}. Placing
\verb+\quiztype{f}+  in the preamble (or prior to any quiz) will
cause all quizzes following that command to be form-type  quizzes.
Similarly, executing \verb+\quiztype{l}+ will produce all link-type quizzes.

You can make the environments revert back to their normal behavior by using
the command \Com{defaultquiztype}, which cancels out the \cs{quiztype} command.

\exPDF{quizpts} The sample file \texttt{\href{\bUrl/webeq_ex.html\#quizpts}
{quizpts.tex}} illustrates these collections of commands.

\newtopic \emph{Within a quiz}, you can say \Com{useForms}, to change the style to a visible checkbox
for multiple choice or multiple selection questions; use \Com{useLinks} to change
the style to links with alphabetized alternatives. Using \cs{restoreFLTypeDefault}
restores the next multiple choice/selection question back to the default determined
by the quiz environment (\texttt{quiz} versus \texttt{quiz*}), or as set by the
\cs{quiztype} command.

\subsubsection{The BeginQuiz and EndQuiz Form Buttons}\label{sss:FormButton}

The default setup the \texttt{quiz} environment is to have
hypertext links for the `Begin Quiz' and `End Quiz'. You can also
redefine this linking and use a form button instead
Prior to your quiz, use the following code, if desired.
\begin{sverbatim}
\useBeginQuizButton
\useEndQuizButton
\end{sverbatim}

\useBeginQuizButton\useEndQuizButton
%\makeatletter
%\renewcommand\eq@beginQuiz{\eq@BeginQuizButton}
%\renewcommand\eq@endQuiz{\eq@EndQuizButton}
%\makeatother


\begin{quiz}{qzTeXf} Answer each of the following. Passing
is 100\%.
\begin{questions}
\item Who created \TeX?
\begin{answers}4
\Ans1 Knuth &\Ans0 Lamport &\Ans0 Carlisle &\Ans0 Rahtz
\end{answers}
\item Who originally wrote \LaTeX?
\begin{answers}4
\Ans0 Knuth &\Ans1 Lamport &\Ans0 Carlisle &\Ans0 Rahtz
\end{answers}
\end{questions}
\end{quiz}\quad\ScoreField{qzTeXf}

\medskip
\noindent Revert back to link-style as follows:
\begin{sverbatim}
\useBeginQuizLink
\useEndQuizLink
\end{sverbatim}

\useBeginQuizLink\useEndQuizLink

\noindent The commands \cs{useBeginQuizButton} and \cs{useEndQuizButton} each
have an optional argument that can be used to modify the appearance
of the buttons.
\begin{sverbatim}
\useBeginQuizButton[\textColor{0 0 1}]
\end{sverbatim}
\noindent would create a `Begin Quiz' button with blue text for the button label.

\subsubsection{The \texorpdfstring{\texttt{proof\-ing}}{proofing}
Option}\label{ss:proofing}

For proofreading, use the \texttt{proofing} option of
\textsf{exerquiz}.
\begin{sverbatim}
\usepackage[proofing]{exerquiz}
\end{sverbatim}
\noindent When used, a symbol, defined by the command \cs{proofingsymbol},
will mark the correct answers, as defined in your source file. The
command \cs{proofingsymbol} can be redefined, its definition is
\begin{sverbatim}
\newcommand\proofingsymbol{\textcolor{webgreen}{$\bullet$}}
\end{sverbatim}
\noindent
This option works for the \texttt{\hyperref[ss:QuizIR]{shortquiz}}
environments defined above
(\hyperref[ss:QuizIR]{page~\pageref*{ss:QuizIR}}), as well.

\subsubsection{The \texttt{showgrayletters} Option}             \aebCng[showgrayletters]

The showgrayletters option is valid for the \texttt{quiz} and \texttt{shortquiz} environments.
See the section entitled \Nameref{showgraylettersOpt} for a detailed explanation
of this option.

\subsubsection{Setting the Threshold}\label{threshold}

The default behavior of the \texttt{quiz} environment is that a
student can begin the quiz and finish the quiz without answering
any or all of the questions. This is called a \texttt{lowThreshold}
and is the default behavior.

The document author can set a \texttt{highThreshold} by re-defining
the \cs{minQuizResp} macro. The default definition is
\begin{sverbatim}
\newcommand\minQuizResp{lowThreshold}
\end{sverbatim}
\noindent However, if you make the definition
\begin{sverbatim}
\renewcommand\minQuizResp{highThreshold}
\end{sverbatim}
\noindent the student is required to answer all the questions of a
quiz.

Actually, \texttt{lowThreshold} and \texttt{highThreshold} are
JavaScript functions that are called when the ``End
Quiz'' button is clicked. If the threshold is not met, an alert box
appears informing the user of this.

The document author can write a custom threshold function and place
its name in the \cs{minQuizResp} macro. See the \textsf{exerquiz}
source code---in either \textsf{exerquiz.dtx} or
\textsf{exerquiz.sty}---for the \texttt{highThreshold()} function
for an example of how to do this.

\subsection{The \texttt{manswers} Environment}\label{manswersQuiz}

Beginning with \textsf{Exerquiz} v.6.2, the \texttt{manswers}
environment is defined to support multiple selection questions.
(\texttt{manswers} stands for multiple answers). This is in contrast
with the \texttt{answers} environment in which there is only one
correct answer; for \texttt{manswers}, there are several correct
answers.

\exPDF{manswers} See the example file for this feature is \texttt{manswers.tex}.

\newtopic The syntax is the same as that of the \texttt{answers} environment,
\Nameref{answersQuiz}. Below is example fragment, taken from \texttt{manswers.tex}

\begin{small}
\begin{verbatim}
\item\PTs{5} Select which people were the President
             of the United States.
\begin{manswers}{4}
    \bChoices[2]
        \Ans[-1]{0} Henry Clay\eAns
        \Ans[-1]{0} Ben Franklin\eAns
        \Ans[1.5]{1} Andrew Jackson\eAns
        \Ans[1.5]{1} Ronald Reagan\eAns
        \Ans[-1]{0} George Meade\eAns
        \Ans[2]{1} Grover Cleveland\eAns
        \Ans[-1]{0} John Jay\eAns
        \Ans[-1]{0} Paul Revere\eAns
    \eChoices
\end{manswers}
\end{verbatim}
\end{small}

\newtopic The arguments of the \cs{Ans} command is the same as
those of \cs{Ans} in the \texttt{answers} environment.

\newtopic When the \cs{ScoreField} is used to display results, a manswers
question is correct if and only if all correct answers are checked, and none
of the incorrect answers are checked.

When the \cs{PointsField} is used (see \Nameref{sss:asignPts}), and
points are assigned, as in the above code fragment, the optional
argument of \cs{Ans} is used to assign points for each response (if
there is no optional argument, it defaults to 0). When a student
selects a choice, the point value in the optional argument is added
to the point score. In the case of choosing \texttt{Ronald Reagan},
1.5 points are added to the point score, while -1 point is added to
the point score for choosing \texttt{John Jay}. The minimal point
score for each question is zero, unless \cs{negPointsAllowed} command
is executed in the preamble, see \Nameref{negPtsAllowed}.


\subsection{Correcting the Quizzes with JavaScript}\label{ss:correct}

%\makeatletter\eq@nocorrectionsfalse\makeatother
\CorrectionsOn

Beginning with \textsf{exerquiz}, version 1.2, you can now correct
quizzes created by the \texttt{quiz} environment.  To correct the
quizzes, simply include an additional element into your quiz, a
correction button. The correction button is installed using the
macro \cs{eqButton}.

\redpoint The following is a link-style quiz.

%     \begin{center}
%     \noindent\fcolorbox{blue}{webyellow}{%
%     \begin{minipage}{.965\linewidth}%
%     \noindent\textcolor{red}{\textbf{Instructions:}} Click on
%     `\textcolor{webgreen}{Begin Quiz}' to  initialize the quiz.
%     When finished, click on `\textcolor{webgreen}{End Quiz}'.
%     Then, click on the `Correct' button.
% %    To grade your quiz, press the `Correct' button.
%     \end{minipage}
%     }%
%     \end{center}

\begin{quiz}{qzdiscr1l} Using the discriminant, $b^2-4ac$,
respond to each of the following questions.
\begin{questions}
\item Is the quadratic polynomial $x^2-4x + 3$ irreducible?
\begin{answers}4
\Ans0 Yes &\Ans1 No
\end{answers}
\item Is the quadratic polynomial $2x^2 - 4x + 3 $ irreducible?
\begin{answers}4
\Ans1 Yes &\Ans0 No
\end{answers}
\item How many solutions does the equation $2x^2 - 3x - 2= 0$ have?
\begin{answers}4
\Ans0 none &\Ans0 one &\Ans1 two
\end{answers}
\end{questions}
\end{quiz}\par
\ScoreField{qzdiscr1l}\eqButton{qzdiscr1l}%

\medskip
%\fcolorbox{blue}{webyellow}{%
%    \begin{minipage}{.975\linewidth}%
    \noindent\textcolor{red}{\textbf{Legend:}}
   A \textcolor{red}{\ding{52}} indicates a
    correct response; a \textcolor{red}{\ding{56}}, indicates an
    incorrect response, in this case, the correct answer is marked
    with a \textcolor{webgreen} {\ding{108}}.
%  \end{minipage}}

\begin{comment}
    \begin{center}
    \fcolorbox{blue}{webyellow}{%
    \begin{minipage}{.95\linewidth}
    \noindent\textcolor{red}{\textbf{Legend:}}
    A \textcolor{red}{\ding{52}} indicates that the student gave the
    correct response; a \textcolor{red}{\ding{56}}, indicates an
    incorrect response, in this case, the correct answer is marked
    with a \textcolor{webgreen} {\ding{108}}.
    \end{minipage}}
    \end{center}
\end{comment}

\medskip
A partial verbatim listing of this quiz follows:
\begin{sverbatim}
\begin{quiz}{qzdiscr1l} Using the discriminant, $b^2-4ac$,
respond to each of the following questions.
\begin{questions}
............................
............................
............................
\end{questions}
\end{quiz}

\ScoreField{qzdiscr1l}\eqButton{qzdiscr1l}
\end{sverbatim}


\redpoint The macro \cs{eqButton} is used to create a nice
``correction'' button.  JavaScript is used to correct the quiz.
The only required argument is the field label that uniquely
defines the field in which the total score is placed. See the
section entitled \mlNameref{sss:eqButton} for more details on how
to use this macro.

\redpoint The \cs{eqButton} will not work until the user has
clicked on `End Quiz'.  The user can re-take the quiz simply by
clicking on `Begin Quiz', the form fields and JavaScript variables
will be cleared.

\redpoint It is possible to take this form data and submit it to a
CGI script for processing. (The data can be saved to a database,
for example.)  However, there is no built-in capability for this
in the \textsf{exerquiz} package.

\symbolchoice{check}

\medskip

\noindent The same quiz can be written in form-style simply by inserting the
\texttt*-option.

\medskip\noindent\textcolor{red}{Instructions.} You must click on
`Begin Quiz' to  initialize the quiz.  Not doing so, brings forth
an error message.  When finished, click on `End Quiz'.

\begin{quiz*}{qzdiscr1f} Using the discriminant, $b^2-4ac$,
respond to each of the following questions.
\begin{questions}
\item Is the quadratic polynomial $x^2-4x + 3$ irreducible?
\begin{answers}4
\Ans0 Yes &\Ans1 No
\end{answers}
\item Is the quadratic polynomial $2x^2 - 4x + 3 $ irreducible?
\begin{answers}4
\Ans1 Yes &\Ans0 No
\end{answers}
\item How many solutions does the equation $2x^2 - 3x - 2= 0$ have?
\begin{answers}4
\Ans0 none &\Ans0 one &\Ans1 two
\end{answers}
\end{questions}
\end{quiz*}\quad\ScoreField\currQuiz\eqButton\currQuiz

\redpoint In the partial verbatim listing that follows, notice the
field name has been changed from \texttt{qz:discr1-l} to
\texttt{qzdiscr1}.  Different quizzes must have unique
field names.

\begin{sverbatim}
\begin{quiz*}{qzdiscr1f} Using the discriminant, $b^2-4ac$,
respond to each of the following questions.
\begin{questions}
............................
............................
............................
\end{questions}
\end{quiz*}\quad\ScoreField\currQuiz\eqButton\currQuiz
\end{sverbatim}

\redpoint Notice that in this example, the \cs{ScoreField} and the
\cs{eqButton} are positioned following the `End Quiz'; this makes
the design more compact and nicer looking.

\subsubsection{The
\texorpdfstring{\protect\texttt{nocorrections}}{nocorrections}
Option}\label{sss:nocorrections}

Including the corrections adds quite a bit more JavaScript code to
the \texttt{.pdf} document, this feature is `on' by default. If
you have a document in which you do not want to have the option of
offering corrected quizzes, then just specify
\texttt{nocorrections} in the option list of \textsf{exerquiz}.

There are also a couple of macros you can use to override the option switch, they are
\cs{CorrectionsOn} and \cs{CorrectionsOff}.  Each remains in
effect until the other is invoked.

\redpoint If the \texttt{nocorrections} option is taken, then the \cs{eqButton}
does not appear for a quiz.

\subsubsection{Showing Partial Credit Markup}\label{showcreditmarkup}

When a quiz is corrected, it is possible to show how much credit was given.

\exPDF{pc_test} The demo file for this feature is \texttt{pc\_test.tex}.

\exPDF{pc_test1} The example file \texttt{pc\_test1.tex} illustrates
how to have a multiple selection problem and a grouped problem with
no partial credit. The student must get all parts correct to get
full credit for that problem, otherwise, no credit is assigned.

\newtopic Place the command \Com{showCreditMarkup} before a quiz you want the
partial credit markup. The command \Com{hideCreditMarkup} hides this
markup (it switches off the previous \cs{show\-Credit\-Mark\-up}); place
this command before a quiz for which you don't want the markup. The
default is not to show any partial credit markup.

When a question item introduces a series of question, but poses no question itself,
place \Com{multipartquestion} prior to that item, for example
\begin{verbatim}
    \multipartquestion
    \item\PTs{6} Answer each of these questions
    \begin{question}
      \item\PTs{3} Answer this
      \item\PTs{3} Answer that
    \end{question}
\end{verbatim}
Other relevant commands are \cs{aeb@creditmarkup} and
\cs{aeb@creditmarkupfmt}; the first one defines a little text field
to hold the partial-credit data and the second one sets the
placement of the text field. These can be redefined if the default
definitions are not to your liking. The markup in red that appears
in the left margin shows the number of points the student received
for that problem, for example \texttt{\textcolor{red}{3 pts}}. The
string \texttt{pts} (and its singular counterpart, \texttt{pt}) is language dependent, in
English it refers to ``points'' (or ``point''). You can change this
string to an abbreviation of points (point) using the commands
commands \Com{ptsLabel} and \Com{ptLabel}. The default definitions are
\verb!\ptsLabel{pts}! and \verb!\ptsLabel{pt}!, respectively.

\newtopic\textbf{\textcolor{red}{Important:}} Adobe Reader 5.0 or
later is required for this feature.

\subsection{Quizzes with Solutions}

In addition to scoring and marking the quizzes, you can also
(optionally) provide solutions as well. To enter a solution to a
multiple choice question, use a \texttt{solution} environment, and
attach a named destination to the \texttt{answers} environment.
A partial verbatim listing follows the next example.

%\ifeqforpaper\else\NoPeeking\fi


\begin{quiz*}{qzTeXl} Answer each of the following. Passing
is 100\%.

\begin{questions}

\hfuzz=2pt

\item Who created \TeX?
\begin{answers}[knuth]{4}%
\Ans1 Knuth &\Ans0 Lamport &\Ans0 Carlisle &\Ans0 Rahtz
\end{answers}
\begin{solution}
Yes, Donald Knuth was the creator of \TeX.
\end{solution}

\item Who originally wrote \LaTeX?
\begin{answers}[lamport]{4}
\Ans0 Knuth &\Ans1 Lamport &\Ans0 Carlisle &\Ans0 Rahtz
\end{answers}
\begin{solution}
Yes, Leslie Lamport was the creator of \LaTeX.
\end{solution}
\end{questions}
\end{quiz*}\quad\ScoreField\currQuiz\eqButton\currQuiz

\AllowPeeking

After the quiz is completed and the corrections button is pressed,
the corrections appear.  The correct answer has a green filled
circle or a green check; this circle is now outlined by a green
rectangle to indicate that this is a link to the solution. Click on
the green dot and jump to the solution!

Solutions do not have to appear. Some problems can have solutions,
while others do not. The ones with the solutions have the green
boundary to indicate a link to the solution.

Here is a partial listing of the above example.

\begin{sverbatim}
\begin{quiz*}{qzTeXl} Answer each of the following.
Passing is 100\%.
\begin{questions}
\item Who created \TeX?
\begin{answers}[knuth]4
\Ans1 Knuth &\Ans0 Lamport &\Ans0 Carlisle &\Ans0 Rahtz
\end{answers}
\begin{solution}
Yes, Donald Knuth was the creator of \TeX.
\end{solution}
....
\end{questions}
\end{quiz*}\quad\ScoreField\currQuiz\eqButton\currQuiz
\end{sverbatim}

\redpoint Notice that in the \texttt{answers} environment, an
optional parameter \texttt{[knuth]} appears. The value of this
parameter is a unique name for the solution to the quiz. Notice as well that
the \texttt{solution} environment follows, and is not nested
within the \texttt{answers} environment.


\subsection{How to Modify the \texttt{quiz} Environment}

There are four ways the appearance of the quizzes can change:

\begingroup
\def\mylabel#1{\hfil\hyperref[#1]{\large$\bullet$}\ }
\begin{list}{}
%{\normalfont\bfseries\arabic{questionno}.\ }%
{\usecounter{questionno}\setlength{\parsep}{0pt}\setlength{\topsep}{0pt}
\renewcommand{\makelabel}{\mylabel}
\settowidth{\labelwidth}{\normalfont\bfseries00.\ }%
\setlength{\itemindent}{0pt}\setlength{\labelsep}{0pt}}%
\item[sss:quiztitles] change the titles
\item[sss:ckappear] change the `check' appearance
\item[sss:scorefield] change the text field in which the score appears,
\item[sss:eqButton] change the appearance of the `Correction' button.
\end{list}
\par\endgroup

\noindent This section discusses each of these four in turn.

\subsubsection{The Quiz Titles}\label{sss:quiztitles}

It is possible to redefine the quiz titles and other labels if
desired.

\redpoint Locally:
\begin{sverbatim}
\renewcommand\bqlabel{Begin Exam}
\renewcommand\eqlabel{End Exam}
\end{sverbatim}

\redpoint Globally:
\begin{sverbatim}
\makeatletter
\renewcommand\eq@bqlabel{Begin Exam}
\renewcommand\eq@eqlabel{End Exam}
\makeatother
\end{sverbatim}

\subsubsection{The check appearance}\label{sss:ckappear}

The appearance of the `check' can be chosen using the
\cs{symbolchoice} macro of the \textsf{exerquiz} package. The
permissible values for the argument of \cs{symbolchoice} are
\texttt{check} (the default), \texttt{circle}, \texttt{cross},
\texttt{diamond}, \texttt{square}, and \texttt{star}.

This quiz was generated by inserting
\verb+\symbolchoice{diamond}+ before the quiz. The multiple choice field
is actually a radio button field. The appearance of these radio buttons can
be modified using the command \cs{everyqRadioButton}.

\begin{sverbatim}
\symbolchoice{diamond}
\everyqRadioButton{\textColor{0 0 1}
    \BC{.690 .769 .871}\BG{.941 1 .941}}
\end{sverbatim}

\symbolchoice{diamond}
\everyqRadioButton{\symbolchoice{diamond}\textColor{0 0 1}\BC{.690 .769 .871}\BG{.941 1 .941}}


\begin{quiz*}{qzTeX} Answer each of the following. Passing
is 100\%.
\begin{questions}
\item Who created \TeX?
\begin{answers}4
\Ans1 Knuth &\Ans0 Lamport &\Ans0 Carlisle &\Ans0 Rahtz
\end{answers}
\item Who originally wrote \LaTeX?
\begin{answers}4
\Ans0 Knuth &\Ans1 Lamport &\Ans0 Carlisle &\Ans0 Rahtz
\end{answers}
\end{questions}
\end{quiz*}\quad\ScoreField{qzTeX}\eqButton{qzTeX}

\noindent If desired, we can return to the defaults:
\begin{sverbatim}
\symbolchoice{check} % restore the default
\everyqRadioButton{}
\end{sverbatim}

\symbolchoice{check} % restore the default
\everyqRadioButton{}

\redpoint The \cs{symbolchoice} can also be introduced into the flow of the code through
the \cs{every\-qRadio\-Button}:
\begin{sverbatim}
\everyqRadioButton{\symbolchoice{diamond}\textColor{0 0 1}
    \BC{.690 .769 .871}\BG{.941 1 .941}}
\end{sverbatim}
\noindent After the quiz, we could return to the defaults by
\begin{sverbatim}
\everyqRadioButton{}
\end{sverbatim}

\redpoint See the document \href{eformman.pdf}{eFormMan.pdf} on
\textit{\href{eformman.pdf}{eForm Support}} for complete
documentation on how to modify a field using the optional first
argument, and how to use the ``every'' command.

\subsubsection{Change color of Correction Marks}

The colors used to mark the quiz can be changed by redefining the
commands \cs{check\-Color}, \cs{crossColor} and \cs{correctColor} in
the \textit{preamble} or before. Below are the
defaults:
\begin{sverbatim}
\renewcommand\checkColor{["RGB", 0, .6, 0]}
\renewcommand\crossColor{color.red}
\renewcommand\correctColor{["RGB", 0, .6, 0]} % webgreen
\end{sverbatim}
\noindent The colors are inserted into the field using JavaScript, so the color definitions
are in the color space of the JavaScript object model.


\subsubsection{The `Correction' Button}\label{sss:eqButton}

The `Correction' button is defined by the \cs{eqButton} has two
parameters.
\begin{sverbatim}
\eqButton[mod_appear]{field_name}
\end{sverbatim}
\noindent The second parameter is the field name that contains the
total score for the quiz (see the above examples).  It also has one
optional argument that can be used to modify the appearance of the
button.

In addition to the optional parameter to modify the appearance of
\cs{eqButton}, there is also a ``global'' mechanism for
modifying the appearance of the button field. These are

\paragraph*{Global Modification:} \cs{everyButtonField} and \cs{everyeqButton}

\newtopic The first one modifies the appearance of every quiz button field, the second can be used
to modify all \cs{eqButton}s.

\redpoint See the document \href{eformman.pdf}{eFormMan.pdf} on
\textit{\href{eformman.pdf}{eForm Support}} for complete
documentation on how to modify a field using the optional first
argument, and how to use the ``every'' command.


% \begin{table}[hbt]
% \setlength{\extrarowheight}{1mm}
% \setlength{\tabcolsep}{2mm}
% \begin{center}
% \makeatletter
% \begin{tabular}{|l|l|l|l|}
% \hline
% \multicolumn1{|l|}{Local}   &
% \multicolumn1{l|}{Global}   &
% \multicolumn1{l|}{Default}  &
% \multicolumn1{l|}{Description}\\
% \hline
% \cs{BC} &\cs{eq@BC} & 1 0 0 & border color\\
% \cs{BG} &\cs{eq@BG} & .7529 .7529 .7529 & face color\\
% \cs{CA} &\cs{eq@CA} & Correct & button text\\
% \cs{RC} &\cs{eq@RC} & My Answers! & rollover\\
% \cs{AC} &\cs{eq@AC} & Please! & pushed text \\
% \cs{DA} &\cs{eq@DA} & /Helv 10 Tf 0 g & text format\\
% \cs{BS} &\cs{eq@BS} & /W 1  /S /B &  button spec\\
% \hline
% \end{tabular}
% \caption{\protect\cs{eqButton} Parameters}\label{table:eqButton}
% \end{center}
% \vspace{-\baselineskip}
% \end{table}
%
%
% The macros listed in the first column of \autoref{table:eqButton}
% are permitted in the optional parameter field of
% \cs{eqButton}---there is not parameter checking, \TeX\ or
% Distiller/Reader will find the errors.
%
% The meaning of these values is beyond the scope of this manual.
% Refer to the internet article ``Pdfmarks: Links and Forms'',
% \cite{article:Story:Pdfmarks}, for details; in particular, see the
% ``Forms'' article.

\subsubsection{The Score Field}\label{sss:scorefield}

The score field is the text field to which the quiz (and its
underlying JavaScript) reports the score. This field can be
constructed using the \cs{ScoreField} macro
\begin{sverbatim}
\ScoreField[mod_appear]{field_name}
\end{sverbatim}
\noindent In the simplest case, \cs{ScoreField} takes one argument, as above,
the \texttt{field\_name} of the associated quiz.  The expansion of \cs{ScoreField}
produces a \texttt{read-only} text field  that is 1.5 inches in
width with a red border.  The initial text that appears in the
field is the expansion of the macro \cs{eqScore}. The expansion of
\cs{eqScore} depends on the language option: \cs{eqScore} expands
to `Score:' by default, to `Punkte:' for the \texttt{german}
option and to `Score :' for the \texttt{french} option.

The macro \cs{ScoreField} also has an optional parameter that can
be used to modify the appearance of the text field. Should the
document author want to change the basic look of the text field produced
by \cs{ScoreField}, just introduce the changes through this
optional parameter.

In addition to the optional parameter for modifying the appearance of
the text field, \cs{Score\-Field}, there is also a ``global'' mechanism for
modifying the appearance of the button field. These are
\parss
\noindent\textbf{Global Modification:} \cs{everyeqTextField} and \cs{everyScoreField}
\parss\noindent
The first one modifies the appearance of every quiz text field, the second can be used
to modify all \cs{ScoreField}s.

\redpoint See the document \href{eformman.pdf}{eFormMan.pdf} on
\textit{\href{eformman.pdf}{eForm Support}} for complete
documentation on how to modify a field using the optional first
argument, and how to use the ``every'' command.


\begin{quiz*}{qzTeXc} Answer each of the following. Passing
is 100\%.
\begin{questions}
\item What \TeX\ System does Thomas Esser maintain?
\begin{answers}4
\Ans0 MiK\TeX &\Ans0 cs\TeX &\Ans1 te\TeX &\Ans0 fp\TeX
\end{answers}
\item What \TeX\ System does Fabrice Popineau maintain?
\begin{answers}4
\Ans0 MiK\TeX &\Ans0 cs\TeX &\Ans0 te\TeX &\Ans1 fp\TeX
\end{answers}
\item What \TeX\ System does Christian Schenk maintain?
\begin{answers}4
\Ans1 MiK\TeX &\Ans0 cs\TeX &\Ans0 te\TeX &\Ans0 fp\TeX
\end{answers}
\end{questions}
\end{quiz*}\quad%
\ScoreField[\BC{0 0 1}]{qzTeXc}%
   \eqButton[\BC{0 0 1}%    blue border color
   \CA{TeX}%                Button text
   \RC{Users}%              rollover text
   \AC{Group}%              pushed text
   \textFont{TiRo}%         text font: Times Roman
   \textSize{10}%           text size: 10 point
   \textColor{0 0 1}%       times roman, 10 pt, blue text
   \W{1}\S{I}%              border width 1, inset button
   ]{qzTeXc}

\noindent The new part is the customized scoring and correction button. Here
is a verbatim listing of the \cs{ScoreField} and \cs{eqButton}
macros.
\begin{sverbatim}
\ScoreField[\BC{0 0 1}]{qz:TeXc}%
   \eqButton[\BC{0 0 1}     % blue border color
   \CA{TeX}                 % Button text
   \RC{Users}               % rollover text
   \AC{Group}               % pushed text
   \textFont{TiRo}          % text font: Times Roman
   \textSize{10}            % text size: 10 point
   \textColor{0 0 1}        % blue text
   \W{1}\S{I}               % border width 1, inset button
   ]{qz:TeXc}
\end{sverbatim}

\exPDF{webeqtst} This example---as well as others---appears in
\texttt{\href{\bUrl/webeq_ex.html\#webeqtst}{webeqtst.tex}}, a
test file that accompanies the \cAcroB.

\subsection{\texorpdfstring{\protect\cs{titleQuiz}}{\CMD{titleQuiz}}
    and \texorpdfstring{\protect\cs{fancyQuizHeaders}}{\CMD{fancyQuizHeaders}}}\label{titleQuiz}

\exPDF{title_quiz}The demo file for these features is \texttt{title\_quiz.tex}.


\subsubsection{\texorpdfstring{\protect\cs{titleQuiz}}{\CMD{titleQuiz}}}

To assign a title to your quiz use \Com{titleQuiz}. This title appears just after
the ``Begin Quiz'' button

\fancyQuizHeaders
\titleQuiz{Two Question MC}
\begin{quiz}{MyMCQuiz}\label{MyMCQuiz}
Answer each of these to the best of your ability.
\begin{questions}
\item Math is fun.
\begin{answers}*{4}
\Ans1 True & \Ans0 False\\
\end{answers}
\begin{solution}
  The answer is true, of course, math is fun!
\end{solution}

\item Statistics is fun.
\begin{answers}*{4}
\Ans0 True & \Ans1 False\\
\end{answers}
\begin{solution}
  The answer is false, statistics is interesting but not fun.
\end{solution}
\end{questions}
\end{quiz}\quad\ScoreField\currQuiz\eqButton\currQuiz

\noindent The partial verbatim code for the above quiz follows:

\begin{sverbatim}
\titleQuiz{Two Question MC}
\begin{quiz}{MyMCQuiz}\label{MyMCQuiz}
Answer each of these to the best of your ability.
\begin{questions}
...
\end{questions}
\end{quiz}\quad\ScoreField\currQuiz\eqButton\currQuiz
\end{sverbatim}

\restoreDefaultQuizHeaders

\noindent The format of the title is controlled by the command \Com{titleQuizfmt},
the default definition is \verb!\newcommand\titleQuizfmt{\bfseries}!. The value
of the \cs{titleQuiz} command can act as an argument for \cs{titleQuizfmt} if you redefine
\cs{titleQuizfmt} to have an argument. For example,
\begin{sverbatim}
\renewcommand\titleQuizfmt[1]{\bfseries(#1)}
\end{sverbatim}
Now, the title will appear in a set of parentheses. The title and formatting are enclosed
in a \cs{mbox}, so font and style changes are localized to that  \cs{mbox}.

\newtopic If a quiz does not have a title, as defined by \cs{titleQuiz}, the title defaults to
an empty title.

\newtopic The definition of \cs{titleQuiz} is such that the command \cs{makeatletter} is executed
before the argument for \cs{titleQuiz} is read. This means that the special character
`\texttt{@}' can be used within the argument of \cs{titleQuiz}.  For example, the \texttt{quiz}
environment maintains a text macro \cs{\@quizCnt} which contains the current quiz
count number. Thus you can say \verb!\titleQuiz{Quiz \@quizCnt}!, the title of
the quiz will include a short quiz number as well.


\newtopic The \cs{titleQuiz} is used for short quizzes as well as
with quizzes. It is possible to have different formatting for each
of these two environments by using the \Com{ifQuizType} command. The
command takes two arguments:
\begin{sverbatim}
\ifQuizType{<quiz_code>}{<shortquiz_code>}
\end{sverbatim}
For example, we could define \cs{titleQuizfmt} as follows:
\begin{sverbatim}
\renewcommand\titleQuizfmt{\ifQuizType{\bfseries}{\slshape}}
\end{sverbatim}
or with an argument, as mentioned above, you could define \cs{titleQuizfmt} with an argument,
like so,
\begin{sverbatim}
\renewcommand{\titleQuizfmt}[1]{\ifQuizType{\bfseries(#1)}{\slshape[#1]}}
\end{sverbatim}

\subsubsection{Cross-Referencing}

Quizzes and shortquizzes conform to the cross-referencing system
of {\LaTeX}. For example, we used the label \verb!\label{MyMCQuiz}! in the last quiz, the one
on page~\pageref{MyMCQuiz}. The title of the quiz is \nameref{MyMCQuiz}. The code for the past
two sentences is
\begin{sverbatim}
    For example, we used the label \verb!\label{MyMCQuiz}! in the last
    quiz, the one on page~\pageref{MyMCQuiz}. The title of the quiz is
    \nameref{MyMCQuiz}.
\end{sverbatim}
\noindent If the quiz does not have a title, as defined by the \cs{titleQuiz} command, a generic label of
`Quiz' is generated. The default label name is
\begin{sverbatim}
    \newcommand\eq@defaultQuizLableName{Quiz}
\end{sverbatim}
\noindent This definition can be change using \cs{renewcommand}.

\subsubsection{\texorpdfstring{\protect\cs{fancyQuizHeaders}}{\CMD{fancyQuizHeaders}}}\label{fancyQuizHeadersQ}

The \Com{fancyQuizHeaders} command, a companion feature to \Com{titleQuiz}, enhances
the labeling of the quiz in the solutions section of the document. If
\cs{fancyQuizHeaders} is executed in the preamble, or prior to a quiz, the solution header has the
following default format:

\begin{quote}
\newtopic{\color{blue}Two Question MC: Question 1.} The answer is true, of course, math is fun!
\end{quote}


\newtopic The above assumes the title of the quiz is the one given in the previous section.

\newtopic When a quiz has no title, the header reverts to
\textcolor{blue}{Solution to Quiz:} followed by the question number.

\redpoint You can turn off fancy quiz headers by using
\Com{restoreDefaultQuizHeaders}.

\newtopic You can modify the fancy headers with a combination of setting \Com{setsolnspace}
and/or redefining the command \Com{eq@fancyQuizHeadersfmt}. First
\cs{setsolnspace}: This is a hook into the space immediately
following the header. To insert a \cs{medskip} after the header, for example, you
can say \verb!\setsolnspace{\par\medskip\noindent}! a
\cs{medskip} is created after the solution header and the solution
above appears as

\begin{quote}
\newtopic{\color{blue}Two Question MC: Question 1.}\par\medskip\noindent The answer is true, of course, math is fun!
\end{quote}


\newtopic The definition of \cs{eq@fancyQuizHeadersfmt} is
\begin{footnotesize}
\begin{verbatim}
\def\eq@fancyQuizHeadersfmt{%
 {\protect\color{blue}\protect\textbf{%
 \ifx\aeb@@titleQuiz\@empty\ifnum\@eqquestiondepth>0Solution to Quiz:\fi\else
     \aeb@@titleQuiz:\protect\ \ifnum\@eqquestiondepth=0\else Question\fi\fi\space
 \ifcase\@eqquestiondepth\ifx\aeb@@titleQuiz\@empty Solution to Quiz:\fi\or
 \arabic{eqquestionnoi}.\or\arabic{eqquestionnoi}(\alph{eqquestionnoii})\or
 \arabic{eqquestionnoi}(\alph{eqquestionnoii})(\roman{eqquestionnoiii})\fi}}}
\end{verbatim}
\end{footnotesize}
Ugly isn't it.  We need to set the header in all cases: with or without a title;
multiple questions within a \texttt{questions} environment, or a single question
with no \texttt{questions} environment.

\newtopic A simple re-definition might be
\begin{footnotesize}
\begin{verbatim}
\def\eq@fancyQuizHeadersfmt{%
 {\protect\color{blue}\protect\textbf{%
 \ifx\aeb@@titleQuiz\@empty\ifnum\@eqquestiondepth>0Solution to Quiz\fi\else
     \aeb@@titleQuiz\protect\ \ifnum\@eqquestiondepth=0\else Question\fi\fi\space
 \ifcase\@eqquestiondepth\ifx\aeb@@titleQuiz\@empty Solution to Quiz\fi\or
 \arabic{eqquestionnoi}.\or\arabic{eqquestionnoi}(\alph{eqquestionnoii})\or
 \arabic{eqquestionnoi}(\alph{eqquestionnoii})(\roman{eqquestionnoiii})\fi}}}
\end{verbatim}
\end{footnotesize}

\noindent The colons (\texttt{:}) are removed and a \verb!\\! inserted in the fourth
line; that line now reads, in part, \verb!\else\\Question\fi!, instead of
\verb!\else Question\fi!.

\newtopic The results of these changes, in combination with the earlier
change of
\begin{sverbatim}
\setsolnspace{\par\medskip\noindent}
\end{sverbatim}
\noindent are

\begin{quote}
\newtopic{\color{blue}Two Question MC\\Question 1.}\par\medskip\noindent The answer is true, of course, math is fun!
\end{quote}

\newtopic Other (minor) changes such as color, style, font can be done in this way.

\newtopic As a reminder, you can use the web package's switch, \Com{ifeqforpaper} to further
refine these headers. You can have one style of solution headers for paper, and another
style for the screen.

\subsection{Adding Points to a Quiz}\label{sss:asignPts}

%The discussion of this topic can be found in \Nameref{sss:asignPts}.

%\subsubsection{Assigning Points}\label{sss:asignPts}

The questions on a quiz, especially a quiz meant for credit, may
not have the same weight. A point scheme, therefore, has been
created; several additional text fields in support have also been
defined.

Here is a simple two question example to illustrate:

\begin{quiz*}{mathquiz} Answer each of the following. Passing
is 100\%.\hfill\smash{\GradeField[\rectH{30bp}\rectW{30bp}\textSize{0}]\currQuiz}

\begin{questions}

\PTsHook{($\eqPTs^{\text{pts}}$)}

\item\PTs{4} If $\lim_{x\to a} f(x) = f(a)$, then we say that $f$ is\dots
\begin{answers}{4}
\Ans0 differentiable &\Ans1 continuous &\Ans0 integrable
\end{answers}

\item\PTs{6} Name \emph{one} of the two people recognized as a founder of
Calculus.\\[1ex]
\RespBoxTxt{0}{0}{4}{Isaac Newton}{Newton}{Gottfried Leibniz}{Leibniz}%
\CorrAnsButton{Isaac Newton or Gottfried Leibniz}

\end{questions}
\end{quiz*}\quad\ScoreField\currQuiz\eqButton\currQuiz

\noindent
Answers: \AnswerField\currQuiz

\medskip\noindent
Points: \PointsField\currQuiz\ Percent: \PercentField\currQuiz

\exPDF{quizpts} See the sample file
\href{\bUrl/webeq_ex.html\#quizpts}{quizpts.tex} for a more
elaborate version of this question, as well as the source code.

\begin{questions}
\item \verb+\PTs#1+: This macro takes one argument, the number of
points to be assigned to the current problem. Place this command
immediately after the \cs{item} in the \textsf{questions} environment.
For example, in the above quiz we had
\begin{sverbatim}
\item\PTs{6} Name \emph{one} of the two people recognized
             as a founder of Calculus.
\end{sverbatim}
\item \verb+\PTsHook#1+: This macro, which takes one argument, can
be used to typeset the points assigned. and is called by
\cs{PTs}. The argument is what is to be typeset. The value
assigned the current problem by \cs{PTs} is contained within the
macro \cs{eqPTs}. In the quiz above, we had
\begin{sverbatim}
\PTsHook{($\eqPTs^{\text{pts}}$)}
\end{sverbatim}

\item There are three other commands that create text fields to
display results from a quiz with points assigned:
\begin{itemize}
    \item \verb+\PointsField[#1]#2+: The number of points earned for the quiz, the
    total points are also reported. The parameter \texttt{\#2} is the base name
    of the quiz.
    \item \verb+\PercentField[#1]#2+: The percentage score for the
    quiz. The parameter \texttt{\#1} is the base name of the quiz.
    \item \verb+\GradeField[#1]#2+: The letter grade of the
    performance on the quiz. The parameter \texttt{\#2} is the
    base name of the quiz. The values placed in this field are determined
    by the macro \cs{eqGradeScale}.
\end{itemize}
\item \cs{eqGradeScale}: This macro sets the grade scale of a quiz, the default
definition is
\begin{sverbatim}
\newcommand\eqGradeScale{"A",[90, 100],"B",[80,90],
    "C",[70,80],"D",[60,70],"F",[0,60]}
\end{sverbatim}
\noindent The ways things are defined now, there can be only one grade scale
per document.  The value of \cs{eqGradeScale} is a matrix with an
even number of elements. The odd numbered elements are the grades;
the even number elements are intervals of percentages (percentages
of the total number of points on the quiz). If the percentage of
the score falls into a particular range, the corresponding grade
is assigned.

\item[] Note, obviously, you can redefine this command. The
letter grades do not actually have to be grades, they can be little
messages to the student upon completion of the quiz.

\begin{sverbatim}
\renewcommand\eqGradeScale{%
    "Excellent Work.",[90, 100],
    "Solid Effort.",[80,90],
    "Fair.",[70,80],
    "Needs improvement, better work expected.",[60,70],
    "Learning still in progress.",[0,60]
}
\end{sverbatim}
\end{questions}

\subsubsection{\texorpdfstring{\protect\cs{negPointsAllowed}}{\CMD{negPointsAllowed}}}\label{negPtsAllowed}

The \texttt{answers} and \texttt{manswers} environments allow for partial credit,
and even assigning of negative points for wrong answers (see \Nameref{manswersQuiz});
consequently, it is possible for a student taking a quiz to get a
negative points for the quiz. By default \textsf{exerquiz} reports a point score of zero
to \cs{PointsField} if the points are negative.

If you want exerquiz to report the true point score, use the command
\cs{neg\-Points\-Allowed} in the preamble. This will allow the reporting
of a negative point total.

\exPDF{manswers} See the example file for this feature is \texttt{manswers.tex}.
Compare the results of when the default behavior is used, versus
the results when \cs{negPointsAllowed} is used in the preamble.


\subsection{Floating a Quiz}\label{floatQuiz}

The quiz structure is to have a `Begin Quiz' button or link at the
beginning of the quiz and an `End Quiz' button or quiz at the end.
In some settings, you might want to place the `Begin Quiz' button on
another page, with, perhaps, some instructions. For this purpose
\textsf{exerquiz} defines three commands \Com{DeclareQuiz},
\Com{floatQuiz}, \Com{startQuizHere}, \Com{endQuizHere} and
\Com{dockQuiz}.

The \cs{DeclareQuiz} command takes one argument, which is the name
of the quiz, for example \verb!\DeclareQuiz{myQuiz}!. This command
takes it argument and creates a text macro \cs{currQuiz}, which
expands to the quiz name. \cs{currQuiz} can then be used in the
\texttt{quiz} environment command argument for the quiz name. This
makes it easy to change the names without having to search through
and make all changes.

The command \cs{floatQuiz} declares the current quiz, as set by the
\cs{DeclareQuiz} command, is floating, which means the `Start Quiz' and
`End Quiz' link-text or buttons are separated from their usual placement
and are allowed to ``float.'' The third command \cs{startQuizHere}
inserts the `Begin Quiz' link text or form button, while the
\cs{endQuizHere} command inserts the `End Quiz' link or form button.

\newtopic For example,

\DeclareQuiz{myQuiz}
\useBeginQuizButton[\CA{Start}\RC{the}\AC{Quiz}\BC{}\W0]
\useEndQuizButton[\CA{End}\RC{the}\AC{Quiz}\BC{}\W0]
\floatQuiz

\begin{flushleft}
\fcolorbox{blue}{webyellow}{\begin{minipage}{.67\linewidth}
Solve each of the following problems. To begin the quiz,
click on the \startQuizHere\space button.
\end{minipage}}
\end{flushleft}

\begin{quiz}{\currQuiz}
Answer these questions truthfully.
\begin{questions}
  \item Do you understanding the concept of \cs{floatQuiz}?
  \begin{answers}{4}
    \Ans1 Yes & \Ans0 No
  \end{answers}
  \item Do you understanding the concept of \cs{dockQuiz}?
  \begin{answers}*{4}
    \Ans0 Yes & \Ans1 No
    \begin{solution}
      You can't understand it until I explain it!
    \end{solution}
  \end{answers}
\end{questions}
\end{quiz}\ScoreField\currQuiz\eqButton\currQuiz

\begin{flushleft}
\fcolorbox{blue}{webyellow}{\begin{minipage}{.67\linewidth}
After you have finished the quiz, click on the
\endQuizHere\space button. Click the\ \smash[b]{\eqButton[\CA{Correct}\RC{the}\AC{Quiz}\S{S}\BC{}\W0]\currQuiz}\ button if you
wish and your quiz will be marked. Click on the choice with
the green border to jump to the solution.
\end{minipage}}
\end{flushleft}
\dockQuiz

\newtopic To return to the default behavior of the quiz environment
place  \cs{dockQuiz} following the quiz. Quizzes that come
after the \cs{dockQuiz} will then have the standard quiz format.

\exPDF{floatquiz} See the example file \texttt{floatquiz.tex} for the {\LaTeX} code
of the above example.


\section{Objective Style Questions}\label{s:oQ}

Beginning with version~2.0 of \textsf{exerquiz}, objective style
questions can be posed. Single questions can be posed in the
\texttt{\hyperref[ss:oquestion]{oQuestion}} environment,
multiple questions can be placed in either the \texttt{shortquiz}
or the \texttt{quiz} environments. This section discusses this
type of question and all of its supporting commands.


\subsection{Math and Text Questions}

\textsf{Exerquiz} distinguishes between two types of open ended or objective
questions:
\begin{questions}
\item A mathematical question that requires a mathematical expression
as the answer.
\item A question that requires a text answer.
\end{questions}

\redpoint The demo file \texttt{jquiztst.tex} is an important
source of examples and instruction for the mathematical type
question; additionally, the file \texttt{jtxttst.tex} has many
examples for the text type question.


\subsubsection{The Mathematical Question}

At this stage in the development of \textsf{exerquiz}, a
(mathematical) question can be posed that requires an answer that
is a function of one or more declared variables $x$, $y$, $z$, etc.
Thus, when the declared variables $x$, $y$, $z$ are given a value,
the answer is reduced to a number.
%\begin{questions}
%\item A question that has a numerical value.
%\item A question that has a symbolic answer in one variable, $x$.
%The symbolic answer must reduce, however, to a numerical value
%when $x$ is given a numerical value.
%\end{questions}
%Thus, only questions whose answers can be reduced to numerical
%values can be posed.

For example, the answer to the question ``Differentiate $\frac
d{dx} \sin^2(x)$'', is a function in one variable $x$, it can be
evaluated numerically and can, therefore, be posed:

\begin{oQuestion}{ex1}
\redpoint Differentiate $\dfrac d{dx} \sin^2(x) =
$\space\RespBoxMath{2*sin(x)*cos(x)}{4}{.0001}{0}{1}
\end{oQuestion}

\smallskip
See \Nameref{ss:respbox} for details.

In contrast, consider the question: ``Name the probability
distribution popularly referred to as the `bell-shaped curve' ''.
The answer to this question cannot be reduced to a numerical
value. This question can be posed as an \hyperref[sss:oTxtQ]{text
objective question}, or, it does lend itself to a multiple choice
question, however.

\subsubsection{The Text Question}\label{sss:oTxtQ}

You can also pose questions that require a text answer:

\redpoint\begin{oQuestion}{exTxt2}
Name the probability
distribution popularly referred to as the ``bell-shaped curve''.\\
\RespBoxTxt{0}{0}{4}{Normal}{Normal Distribution}{Gaussian}{Gaussian Distribution}
\end{oQuestion}

See \Nameref{ss:respboxtxt} for details.

\subsection{The \texttt{oQuestion} Environment}\label{ss:oquestion}

The \texttt{oQuestion} environment is a very simple environment
for posing a \textit{single} question and will be used in this
section to discuss in detail the macros for posing mathematical
and text open questions.

The syntax for the \texttt{oQuestion} environment is
\begin{sverbatim}
\begin{oQuestion}{<field_name>}
<A math or text open ended question.>
\end{oQuestion}
\end{sverbatim}
\noindent The environment takes one required argument, a unique name for the question.
This name, \texttt{field\_name}, is used by other supporting
macros.

\subsubsection{\texorpdfstring{\protect\cs{RespBoxMath}}
{\textbackslash RespBoxMath}: The Math Question}\label{ss:respbox}

%\subsubsection{\texorpdfstring{\protect\cs{RespBox},
%\protect\cs{Resp\-BoxNT}}{\textbackslash RespBox, \textbackslash
%RespBoxNT}}

The \cs{RespBoxMath} command is used for posing an objective
question. This command must appear in the \texttt{oQuestion},
\texttt{short\-quiz} or \texttt{quiz} environments. In this
section we discuss only the \texttt{oQuestion} environment.

The following is a \hypertarget{minExample}{minimal example}.
Additional enhancements will be discussed in subsequent sections.

\begin{oQuestion}{sine1}
\redpoint Differentiate $\dfrac d{dx} \sin^2(x) =
\RespBoxMath{2*sin(x)*cos(x)}{4}{.0001}{[0,1]}$
\end{oQuestion}
\par\medskip\noindent
The code for the above example is
\begin{sverbatim}
\begin{oQuestion}{sine1}
\redpoint Differentiate $\dfrac d{dx} \sin^2(x) =
\RespBoxMath{2*sin(x)*cos(x)}{4}{.0001}{[0,1]}$
\end{oQuestion}
\end{sverbatim}
\noindent The \cs{RespBoxMath} need not appear in math mode. The definition
of the \cs{redpoint} command, a command written for this document follows.
\begin{sverbatim}
\newcommand\redpoint{\par\removelastskip\vskip\medskipamount\noindent
  \makebox[\parindent][l]{\large\color{red}$\blacktriangleright$}}
\end{sverbatim}

\newtopic You can also pose multivariate questions as well, for example

\begin{oQuestion}{multivariate}
\redpoint $\dfrac{\partial}{\partial y} {4 x^2 y^3 }
    = \RespBoxMath{12*x^2*y^2}(xy){4}{.0001}{[0,1]x[0,1]}$
\end{oQuestion}
\par\medskip\noindent
The code for the above example is
\begin{sverbatim}
\begin{oQuestion}{multivariate}
\redpoint $\dfrac{\partial}{\partial y} {4 x^2 y^3 }
    = \RespBoxMath{12*x^2*y^2}(xy){4}{.0001}{[0,1]x[0,1]}$
\end{oQuestion}
\end{sverbatim}
\noindent See the file \texttt{multivar.tex} for more examples
quizzes involving multivariate problems.


The algorithm used for determining the correctness of the answer
entered by the user is very simple: The user's answer and the
correct answer are evaluated at randomly selected points in an
interval, then compared. If any of the comparisons differ by more
than a pre-selected amount, an $\epsilon$ value, if you will, the
user's answer is declared incorrect; otherwise, it is considered
correct.\footnote{The idea for evaluating user input in this way
comes from  Drs.\ Wlodzimierz Bryc and Stephan Pelikan of The
University of  Cincinnati.}

\newtopic The command \cs{RespBoxMath} takes ten parameters,
five optional and five required:%\hypertarget{respboxargs}{}
\begin{dCmd}{.67\linewidth}
\RespBoxMath[#1]#2(#3)[#4]#5#6#7#8[#9]*#10
\end{dCmd}
\noindent\textbf{\hypertarget{paraRespBox}{Parameters:}}
\begin{description}
\item[\ttfamily\#1 :] Optional parameter used to modify the appearance of the
    text field. See the section entitled \nameref{sss:eqButton} for examples, and \textsf{exerquiz.dtx}
    for a listing of all controlling macros.
\item[\ttfamily\#2 :] The correct answer to the question.
    This must be a numerical value, or a function of one or more variable(s).

\item[\ttfamily\hypertarget{paraRespBoxThree}{\#3} :] An optional parameter, \textit{delimited by
parentheses}, that defines the independent variable; \texttt{x},
is the default value. Note that this parameter is set off by
parentheses.  For a multivariate question, just list the variables
in juxtaposition, \texttt{(xyz)}.

Beginning with version 5.5 of \textsf{exerquiz}, an alternate
method is to delimit with commas \texttt{(x,y,n)} and include the
type of the variables \texttt{(r:x,r:y,i:n)}, where \texttt{"r"}
means a real variable and \texttt{"i"} means an integer variable.
When a type is not specified explicitly, \texttt{"r"} is assumed.
The variables must be either of the old style (no commas, no
typing) or the new style. Do  not mix the styles.

See the \hyperlink{exofT}{example} in \Nameref{enhancements}
of the section below and see the demo file \texttt{integer\_test.tex} to
demonstrate the new method for specifying variables, found in the
\texttt{dljslib\_examples} folder.

%\item[\ttfamily\#3 :] An optional parameter, \textit{delimited by parentheses},
%that defines the independent variable; \texttt{x}, is the default value. Note
%that this parameter is set of by parentheses.

\item[\ttfamily\#4 :] Optional, a named destination to the solution to the
question. There are two forms: the \texttt{[mydest]} an explicit destination for the solution to the problem, or
a `\texttt*', in which case, the name is
\texttt{[\string\curr@quiz.\string\thequestionno]} is automatically assigned.

If this parameter appears, a solution must follow the
question, enclosed in a \texttt{solution} environment.


\item[\ttfamily\#5 :] The number of samples points to be used, usually 3 or 4 is
sufficient.
\item[\ttfamily\#6 :] Precision required, a \emph{non-negative} ``$\epsilon$'' value, if you will. If the precision
is set to zero, \texttt{reldiffCompare} is used as the default compare, and the precision
is set to a small positive value, \text{1E-14}, by default. The small value is may reset with
\cs{defaultRDPrecision}. The default definition, \verb!\defaultRDPrecision{1E-14}!.

\item[\ttfamily\#7 :] Parameters \#7 and \#8 are  used to define the interval from
which to draw the sample points. There are two forms: (1) \#7 is the left-hand endpoint
of the interval and \#8 is the right-hand endpoint (the use of \#7 and \#8 in this form
is deprecated); (2) the interval is defined by standard interval notation, \texttt{[a,b]}.
For a multivariate question---one where parameter \#2 lists more than one variable,
separate the intervals for each variable by a `\texttt{x}', \texttt{[0,2]x[1,2]x[3,4]}.
\item[\ttfamily\#8 :] (1) \#8 is the right-hand endpoint of the interval (the use of this
parameter is deprecated); (2) in the second
case, \#8 is not used.

\item[\ttfamily\hypertarget{paraRespBoxNine}{\#9} :] This optional parameter is the name of a customized
comparison function.

Beginning with version 5.5 of \textsf{exerquiz}, this argument can
also be a JavaScript object with at most two properties:
\texttt{priorParse} and \texttt{comp}. \texttt{priorParse} is used to insert
additional JavaScript into \texttt{ProcResp} prior to processing the user's answer;
this allows additional ``filtering'' of the user's response. The value of
\texttt{priorParse} can either be a single function, or an array of functions. These
functions take \texttt{UserAns} as its argument and return either \texttt{null}, if \texttt{UserAns}
is not acceptable, or \texttt{true}, if it is ok for processing. The value of \texttt{comp} is
the name of the function to be used to compare answers.

See the demo file \texttt{integer\_tst.tex} for examples of usage.

\item[\ttfamily\#10:] (Only detected if following an asterisk, `\texttt*')
The name of a JavaScript function that is to be used to process the user input.
\end{description}

\redpoint For the above example,
\begin{sverbatim}
\RespBoxMath{2*sin(x)*cos(x)}{4}{.0001}{[0,1]}
\end{sverbatim}
\noindent no optional parameter is specified. The correct answer written in
valid JavaScript is given by \verb+2*sin(x)*cos(x)+; evaluation of the
user's answer is done by randomly selecting 4 points from the
interval $[\,0,1\,]$; if the evaluation at any of the 4 points
differs from the evaluation of the correct answer at the same
point by more than $\epsilon=0.0001$, the user's answer is
considered wrong.

Once you choose the question to ask, you must then select the
values of the parameters for \cs{RespBoxMath}.

\noindent\redpoint \textbf{Some Comments}:
\begin{questions}
\item The correct answer can be written either with valid JavaScript,
or in the same syntax a user would enter the answer with. The
functions and operators are pretty much as expected. See the
demo file \textsf{jquiztst.tex} for some discussion how authors
and users should enter their answers.

\item The interval from which the sample points are taken
needs to be chosen with care. The interval must, obviously, be a
subset of the domain of the answer function. Choose an interval
away from any singularities the answer may have.

%\item The JavaScript of Acrobat 5.0 does have exception
%handling, but this has not been incorporated into the code yet.
%Taking advantage of this new capability will be my next project.
%Exception handling will give the code protection against user's
%entering spurious answers. For example, based on the correct
%answer, the author chooses the interval $[\,0,1\,]$, but the user
%enters a function whose domain does not contain the interval, such
%as \verb=(x-1)^(1/2)=.
\end{questions}

\exPDF{jquiztst} See the file \href{\bUrl/webeq_ex.html\#jquiztst}
{jquiztst.pdf} for various examples of the math questions.

By using the optional first parameter, you can modify the appearance
of the field ``locally.'' There is also a ``global'' mechanism as well:
\parss
\noindent\textbf{Global Modification:} \cs{everyeqTextField}, \cs{everyRespBoxMath}
\parss\noindent
The first one modifies the appearance of every quiz text field,
and the second can be used to modify all fields created using
\cs{RespBoxMath}.

\redpoint See the document \href{eformman.pdf}{eFormMan.pdf} on
\textit{\href{eformman.pdf}{eForm Support}} for complete
documentation on how to modify a field using the optional first
argument, and how to use the ``every'' command.

\subsubsection{\texorpdfstring{\protect\cs{RespBoxTxt}}%
    {\textbackslash RespBoxTxt}: The Text Question}\label{ss:respboxtxt}

You can also pose a question that takes a simple text response. The
basic command for posing this type of question is
\cs{RespBoxTxt}. Consider the example given
earlier:
\begin{oQuestion}{exTxt1}
\redpoint Name the probability
distribution popularly referred to as the ``bell-shaped curve''.\\
\mbox{\strut\RespBoxTxt{0}{0}{4}{Normal}{Normal Distribution}%
{Gaussian}{Gaussian Distribution}}
%\raisebox{-3bp}{\RespBoxTxt{0}{0}{4}{Normal}{Normal Distribution}%
%{Gaussian}{Gaussian Distribution}}
\end{oQuestion}

The underlying JavaScript compares the user's response against
acceptable alternatives, as supplied by the author of the
question. If there is a match, the response is deemed correct.

The code for this example is
\begin{sverbatim}
\begin{oQuestion}{exTxt1}
\redpoint Name the probability distribution popularly
referred to as the ``bell-shaped curve''.\\
\RespBoxTxt{0}{0}{4}{Normal}{Normal Distribution}%
{Gaussian}{Gaussian Distribution}
\end{oQuestion}
\end{sverbatim}

\noindent The command \cs{RespBoxTxt} takes five or more parameters.
\begin{sverbatim}
\RespBoxTxt[#1]#2#3[#4]#5<plus listing of alternatives>
\end{sverbatim}
\noindent\textbf{\hypertarget{paraRespBoxTxt}{Parameters:}}
\begin{description}
\item[\ttfamily\#1 :] Optional parameter used to modify the appearance of the
    text field. See the section entitled \nameref{sss:eqButton},
    page~\pageref*{sss:eqButton} for examples, and
    \textsf{exerquiz.dtx} for a listing of all controlling macros.
\item[\ttfamily\#2 :] This required parameter is a number that indicates
the filtering method to be used. Permissible values of this parameter are
\begin{description}
    \item[\ttfamily -1: ] (The default) The author's and user's answers are not filtered
    in any way. (Spaces, case, and punctuation are preserved.)
    \item[\ttfamily 0: ] The author's and user's answers are converted to
    lower case, any white space and non-word characters are removed.
    \item[\ttfamily 1: ] The author's and user's answers are converted to
    lower case, any white space is removed.
    \item[\ttfamily 2: ] The author's and user's answers are stripped of any
    white space.
\end{description}
See the JavaScript function \texttt{eqFilter} in \texttt{exerquiz.dtx} for the
program code details.  Additional filtering options may be added.

\item[\ttfamily\#3 :] This parameter a number that indicates the compare
method to be used. Permissible values of this parameter are
\begin{description}
    \item[\ttfamily 0: ] (The default) The author's and user's answers are
    compared for an exact match. (These answers are filtered before they are
    compared.)
    \item[\ttfamily 1: ] The user's response is searched in an attempt to
    get a substring match with the author's alternatives. Additional comparison
    methods may be added.
\end{description}
See the JavaScript function \texttt{compareTxt} in \texttt{exerquiz.dtx} for the
program code details.

\item[\ttfamily\#4 :] Optional, a named destination to the solution
to the question. If this parameter appears, then a solution must
follow the question, enclosed in a \texttt{solution} environment. If
the fourth parameter is a `\texttt*', then an automatic naming scheme
is used instead.
\item[\ttfamily\#5 :] This required parameter is the number of alternative
answers that are acceptable. The alternative answers are listed
immediately after this parameter. (The example above specified
that 4 alternatives follow.)
\end{description}

\exPDF{jtxttst} See the file \href{\bUrl/webeq_ex.html\#jtxttst}
{jtxttst.pdf} for examples of the differences between various combinations
of filtering rules and comparison methods.

\newtopic By using the optional first parameter, you can modify the appearance
of the field ``locally''. There is also a ``global'' mechanism as well:
\parss
\noindent\textbf{Global Modification:} \cs{everyeqTextField} and \cs{everyRespBoxTxt}
\parss\noindent
The first mechanism modifies the appearance of every quiz text
field, the second can be used to modify all fields created using
\cs{RespBoxTxt}.

\redpoint See the document \href{eformman.pdf}{eFormMan.pdf} on
\textit{\href{eformman.pdf}{eForm Support}} for complete
documentation on how to modify a field using the optional first
argument, and how to use the ``every'' command.

\subsubsection{\texorpdfstring{\protect\cs{RespBoxTxtPC}}%
    {\textbackslash RespBoxTxtPC}: The Text Question with Partial Credit}\label{ss:respboxtxtpc}

\textsf{Exerquiz} can now create text fill-in questions that awards
credit each time one of the key words are found in the student's input string.

\newtopic The command \cs{RespBoxTxtPC} is the one that creates a text fill-in question. Its
syntax is
\begin{verbatim}
\RespBoxTxtPC[#1]#2[#3]#4[<num1>]{<word1>}...[<num_n>]{<word_n>}
\end{verbatim}
The new command \cs{RespBoxTxtPC} behaves differently than \cs{RespBoxTxt}.
JavaScript performs a regular expression search for each word listed (the word,
can be a regular expression), if the word is found (the search is successful), the total credit
for this problem is incremented by the amount associated with the word.

\medskip\noindent\textbf{\hypertarget{paraRespBoxTxtPC}{Parameters:}}
\begin{enumerate}
\item[\ttfamily\#1:]Optional parameter used to modify the appearance of the
    text field.
\item[\ttfamily\#2:]This required parameter is a number indicating
the filtering method to be used. Permissible values of this parameter are
\begin{enumerate}
    \item[\ttfamily-1:] (The default) The author's and user's answers are not filtered
    in any way. (Spaces, case, and punctuation are preserved.)
    \item[\ttfamily0:] The author's and user's answers are converted to
    lower case, any white space and non-word characters are removed.
    \item[\ttfamily1:] The author's and user's answers are converted to
    lower case, any white space is removed.
    \item[\ttfamily2:] The author's and user's answers are stripped of
    white space.
    \item[\ttfamily 3:] Same as -1, but a case insensitive search is performed.
    \emph{This is the recommended value for this function.}
\end{enumerate}
See the JavaScript function \texttt{eqFilter} in \texttt{exerquiz.dtx} for
program code details.  Additional filtering options may be added.

\textbf{Recommendation:} For \cs{RespBoxTxtPC}, use either~-1
(case sensitive search) or~3 (case insensitive search).

\item[\ttfamily\#3:] Optional, a named destination to the
solution to the question. If this parameter appears, then a
solution must follow the question, enclosed in a \texttt{solution}
environment.  If the third parameter is a `\texttt*', then an automatic
naming scheme is used instead.
\item[\ttfamily\#4:] This required parameter is the
number of alternative answers that are acceptable. The alternative
answers are listed immediately after this parameter.
\item[]The parameters that follow \texttt{\#4} are of the form
\verb![<num>]{<word>}!. The \texttt{[<num>]} is the amount of credit
the user gets if his answer contains \texttt{<word>}. Actually,
\texttt{<word>} can be a regular expression so that a more sophisticated
search criteria can be set up. This is illustrated in the examples above,
and explained below.

A \cs{RespBoxTxtPC} question is deemed correct (and marked with a green rectangle,
in the case of a quiz) if at least one of the words is found.

If a word does not have a partial credit \texttt{[<num>]} before it, that word has a partial
credit of zero, and may as well not be included in the list of words.
\end{enumerate}

\exPDF{jtxttst_pc} The demo file for this section is \texttt{jtxttst\_pc.tex}.
The file contains extensive examples, and discusses techniques of searching
using regular expressions.

\subsection{Some Enhancements}\label{enhancements}

There are several enhancements to the math (using \cs{RespBoxMath})
and text (using \cs{Resp\-Box\-Txt}) open-ended question beyond the
minimal examples given earlier. These enhancements can be used
within the \texttt{oQuestion}, the \texttt{shortquiz}, and the
\texttt{quiz} environments.

\subsubsection{Including an Answer Key with \texorpdfstring
{\protect\cs{CorrAnsButton}}{\textbackslash CorrAnsButton}}

The correct solution can be included in the question as well; just include
the command \cs{Corr\-Ans\-Button}. This command takes one parameter, the correct
answer that will be viewed when the user clicks on the button.

The example below also illustrates the (optional) third parameter of
\cs{RespBoxMath}. Here we pose the question in the variable $t$ rather
than the default variable of $x$.
\begin{oQuestion}{sine2}
\redpoint Differentiate\\[1ex]
$\dfrac d{dt} \sin^2(t) =$
\RespBoxMath{2*sin(t)*cos(t)}(t){4}{.0001}{0}{1}\kern1bp\CorrAnsButton{2*sin(t)*cos(t)}
\end{oQuestion}

\medskip\noindent
The listing follows:
\begin{sverbatim}
\begin{oQuestion}{sine2}\\[1ex]
\redpoint Differentiate $\dfrac d{dt} \sin^2(t) =$
\RespBoxMath{2*sin(t)*cos(t)}(t){4}{.0001}{0}{1}\kern1bp
\CorrAnsButton{2*sin(t)*cos(t)}
\end{oQuestion}
\end{sverbatim}

\noindent The \cs{CorrAnsButton} takes one parameter, the correct answer.
This answer is (usually) the same as the one given as the second
argument (the optional argument is the first) in the \cs{RespBoxMath}
command.

\redpoint The \cs{CorrAnsButton} also controls access to the (optional) solution, see
the next section.

\subsubsection{Including a Solution}\label{inclsoln}

In addition to a correct answer, you can also include a solution
to the question. Insert the optional fourth parameter---fourth for both
\cs{RespBoxMath} and \cs{RespBoxTxt}---into the parameter list giving the name of
the destination to the solution.  Follow the question by a \texttt{solution}
environment containing the solution.

The user \texttt{Shift-Clicks} on the \cs{CorrAnsButton} to jump
to the solution.

\begin{oQuestion}{sine3}
\redpoint Differentiate\\[1ex]
$\dfrac d{dt} \sin^2(t) =$
\RespBoxMath{2*sin(t)*cos(t)}(t)[sine3]{4}{.0001}{0}{1}\kern1bp\CorrAnsButton{2*sin(t)*cos(t)}
\begin{solution}
$$
    \frac d{dx}\sin^2(x) = 2\sin(x)\cos(x) = \sin(2x)
$$
\end{solution}
\end{oQuestion}

\medskip\noindent
The listing follows:
\begin{sverbatim}
\begin{oQuestion}{sine3}\\[1ex]
\redpoint Differentiate $\dfrac d{dt} \sin^2(t) =$
\RespBoxMath{2*sin(t)*cos(t)}(t)[sine3]{4}{.0001}{0}{1}\kern1bp
\CorrAnsButton{2*sin(t)*cos(t)}
\begin{solution}
$$
    \frac d{dx}\sin^2(x) = 2\sin(x)\cos(x) = \sin(2x)
$$
\end{solution}
\end{oQuestion}
\end{sverbatim}

\redpoint The \cs{CorrAnsButton} works the same way for the \texttt{shortquiz} and
the \texttt{quiz} environments.

\newtopic Actually, \cs{CorrAnsButton} has an optional argument. The full syntax is
\begin{dCmd}{.75\linewidth}
\CorrAnsButton{<display_answer>}[*{<JS_function>}]
\end{dCmd}
\newtopic When the \texttt* is detected following the \texttt{<display\_answer>}, the \texttt{<JS\_function>}
is used to evaluate \texttt{<display\_answer>}, before displaying it.  There is one built-in
function called \texttt{EvalCorrAnsButton}, the definition of which is
\begin{sverbatim}
function EvalCorrAnsButton(fieldname,theanswer)
{
    theanswer = eval(theanswer);
    DisplayAnswer(fieldname,theanswer);
}
\end{sverbatim}
Other JavaScript functions may be defined using the \texttt{insDLJS} environment, then
passed to \cs{CorrAnsButton} with the optional argument.

An example of this optional parameter is

\begin{oQuestion}{add1}
\redpoint Perform the indicated operation, and enter
result into the box.\\[1ex]
$ \text{9} + \text{8} =$
\RespBoxMath{9 + 8}{1}{.0001}{[0,1]}\kern1bp\CorrAnsButton{9 + 8}*{EvalCorrAnsButton}\kern1bp
\sqTallyBox
\end{oQuestion}

\medskip\noindent
The listing follows:
\begin{sverbatim}
\begin{oQuestion}{add1}
\redpoint Perform the indicated operation, and enter
result into the box.\\[1ex]
$ 9 + 8 = $
\RespBoxMath{9 + 8}{1}{.0001}{[0,1]}\kern1bp
    \CorrAnsButton{9 + 8}*{EvalCorrAnsButton}\kern1bp
\sqTallyBox
\end{oQuestion}
\end{sverbatim}


\subsubsection{Including a Tally Box}

The macro \cs{sqTallyBox} is used to keep a running total of the
number of wrong answers a user has entered into the response box.

For example,

\begin{oQuestion}{sine4}
\redpoint \hypertarget{exofT}{Differentiate}\\[1ex]
$\dfrac d{dx} \sin^2(x) =$
\RespBoxMath{2*sin(x)*cos(x)}{4}{.0001}{0}{1}\kern1bp\CorrAnsButton{2*sin(x)*cos(x)}\kern1bp
\sqTallyBox
\end{oQuestion}

\medskip\noindent
The listing follows:
\begin{sverbatim}
\begin{oQuestion}{sine4}
\redpoint Differentiate\\[1ex]
$\dfrac d{dx} \sin^2(x) =$
\RespBoxMath{2*sin(x)*cos(x)}{4}{.0001}{0}{1}\kern1bp
\CorrAnsButton{2*sin(x)*cos(x)}\kern1bp
\sqTallyBox
\end{oQuestion}
\end{sverbatim}

\redpoint The tally box can be used within the \texttt{oQuestion}
and \texttt{shortquiz} environments; in the \texttt{quiz}
environment, no tally box is used.

\subsubsection{Clearing the Fields}

For the \texttt{oQuestion} and the \texttt{shortquiz}
environments, you can clear the response box fields by placing
insert \cs{sqClearButton}.


\begin{oQuestion}{sine5}
\redpoint Differentiate\\[1ex]
$\dfrac d{dx} \sin^2(x) =$
\RespBoxMath{2*sin(x)*cos(x)}{4}{.0001}{0}{1}\kern1bp\CorrAnsButton{2*sin(x)*cos(x)}\kern1bp
\sqTallyBox\kern1bp\sqClearButton
\end{oQuestion}


\medskip\noindent
The listing follows:
\begin{sverbatim}
\begin{oQuestion}{sine5}
\redpoint Differentiate\\[1ex]
$\dfrac d{dx} \sin^2(x) =$
\RespBoxMath{2*sin(x)*cos(x)}{4}{.0001}{0}{1}%
\CorrAnsButton{2*sin(x)*cos(x)}\kern1bp
\sqTallyBox\kern1bp\sqClearButton
\end{oQuestion}
\end{sverbatim}

\newtopic Notice that I've inserted a \cs{kern1bp} to separate the
two fields \cs{sqTallyBox} and \cs{sq\-Clear\-Button}. This is to keep
their borders from overlapping.

\subsection{More on Math Fill-in Questions}\label{s:moremath}

This section covers rules for author and user input into a math fill-in question.

\subsubsection{Comments on Authoring}\label{s:mathauthoring}

When authoring a math fill-in, that is, when using \cs{RespBoxMath} to
construct a question, several things must be kept in mind.

\begin{itemize}
    \item The \hyperlink{paraRespBox}{second parameter} of \cs{RespBoxMath},
\begin{sverbatim}
\RespBoxMath[#1]#2(#3)[#4]#5#6#7#8[#9]*#10
\end{sverbatim}
    is the correct answer parameter. You need not compute the
    correct answer, you can let the JavaScript of exerquiz do that.
    For example, if the answer is $\sqrt{2}$ you can enter
    \texttt{sqrt(2)}. All the rules of entering user input outlined
    in the next section on
    \begin{NoHyper}`\nameref{ss:userinput}'\end{NoHyper} holds for
    this second parameter as well.

    \item[] If you choose to restrict user input in some way, perhaps using the \texttt{limitArith}
    option of \textsf{dljslib}, these limitations only apply to the user input, not to the author's
    input in parameter \texttt{\#2}. See \Nameref{limitArith} for more information on how to limit
    arithmetic operations.

    \item The question must posed so as to make it clear to the student what is expected
    by way of input into the text field, an expression, an equation, and so on.

\end{itemize}

\subsubsection{Comments on User Input}\label{ss:userinput}

When responding to a Math Fill-in question, the student responds by typing
in an answer.  The following notation is expected by \textsf{exerquiz}.

\begin{itemize}
\item Use \texttt{*} to indicate multiplication: Type \texttt{4*x} for $4x$.

\item[] If the \texttt{ImplMulti} option of the \textsf{dljslib} package is taken, then the
    student need only type in \texttt{4*x} for $4x$. Exerquiz attempts to
    insert the multiplication symbol \texttt{*} by parsing the user input.
    For information on, see the section on \Nameref{ImplMulti}.

\item Use \verb|^| to indicate powers: Type \verb|4*x^3| for
$4x^3$; \verb|12*x^-6| for $12x^{-6}$. Again, if the \texttt{ImplMulti} option is taken,
entering \verb|12x^-6| is sufficient.

\item Use parentheses to delimit the argument of a function; that is, type
\texttt{sin(x)}, rather than \texttt{sin~x}.

\item Use parentheses to define the \textit{scope} of an operation: Type \verb|4*x*(x^2+1)^3| for $4x(x^2+1)^3$; \verb|4^(2*x+1)| for $4^{2x+1}$; \verb|(sin(x))^2| for $(\sin(x))^2$. \textit{Do not} type an expression like \verb|sin^2(x)| for $\sin^2(x)$, type \verb|(sin(x))^2| instead.

\item[] If the \texttt{ImplMulti} option of the \textsf{dljslib} package is taken,
    the use of \texttt{*} is unnecessary, but also the simplified notation for powers
    of functions is supported, now you can  type \verb|sin^2(x)| for $\sin^2(x)$.

\item Brackets \texttt{[ ]} or braces \texttt{\{ \}} can be used
to delimit a mathematical expressions.

\item Functions that are recognized:
\begin{itemize}
    \item Trig: \texttt{sin}, \texttt{cos}, \texttt{tan}, \texttt{cot}, \texttt{sec}, \texttt{csc}.
    \item  Inverse Trig: \texttt{arcsin} (or \texttt{asin}) , \texttt{arccos} (or \texttt{acos}),
           \texttt{arctan} (or \texttt{atan}).
    \item  Logarithms: \texttt{ln} for natural logs, and  \texttt{log} for common logs.
           For example, \texttt{ln(x)} or \texttt{log(x)}.
    \item  Exponential: The natural exponential function, $e^x$,
           can be entered as \texttt{exp(x)} or as \verb|e^x|.
    \item  The absolute function, \texttt{abs($\cdot$)} can also be written
           in the usual way \texttt{|$\cdot$|}; thus, you can type either
           \texttt{abs(x)} or \texttt{|x|}.
    \item  Misc.: \texttt{sqrt}, usage \texttt{sqrt(x)} for $\sqrt{x}$ (or, use
           exponential notation: \verb|x^(1/2)|).
    \item[] When the \texttt{combinatorics} option of \textsf{dljslib} is taken, the binomial coefficient function
            \texttt{C(n,r)}, the permutation function \texttt{P(n,r)}, and the factorial function
            \texttt{fact}, are defined. See \Nameref{combinatorics} for additional details.
\end{itemize}
\end{itemize}
A variation of these listings should appear as instructions to the student on how to enter
math fill-in questions.

\newtopic When a user enters a response, some attempt is made to
determine whether the response is a valid mathematical expression.
For example, if \texttt{san(x)} is entered, the function `\texttt{san}' will
not be recognized as a valid mathematical function; an error
message is generated, and the user is not penalized for a possible
typing error. The JavaScript routines will also check for
unbalanced parentheses; thus, \verb|((x^4+1) + sin(x)^2| will be
flagged as a syntax error.

\Important The student must use the variables declared for the
problem when entering mathematical expressions. If the problem
statement involves the variable $x$, then \texttt{x} must be used;
if the problem statement uses $t$, then \texttt{t} must be used.
To enter an expression using $t$ when an expression with variable $x$ is expected will,
no doubt, result in missed problem.

\subsection{The \texttt{shortquiz} Environment}\label{s:oQsq}

The objective question (with or without the presence of a
correction box, \cs{corr\-Ans\-Button} or a tally box \cs{sqTallyBox})
can be mixed in with multiple choice questions.

Solutions to the questions can also be included using a
\texttt{solution} environment. Click on the ``Ans'' button to get
the answer to a question; shift-click on the ``Ans'' button to get
the solution.

\begin{shortquiz}[oQsq]
Answer each of the following. Passing is 100\%.

\begin{questions}
\item If $f$ is differentiable, then $f$ is continuous.
\begin{answers}{4}
\Ans1 True & \Ans0 False
\end{answers}\hfill\sqTallyBox

\item $\displaystyle\frac d{dx} \sin^2(x) =$
\RespBoxMath{2*sin(x)*cos(x)}[sinsqx]{4}{.0001}{0}{1}%
\hfill\CorrAnsButton{2*sin(x)*cos(x)}%
\kern1bp\sqTallyBox
\begin{solution}
$$
    \frac d{dx}\sin^2(x) = 2\sin(x)\cos(x) = \sin(2x)
$$
\end{solution}

\item Name \emph{one} of the two people recognized as a founder of
Calculus.\vadjust{\kern3pt}\newline
\RespBoxTxt{2}{0}[newton]{5}%
{Isaac Newton}{Newton}{I. Newton}{Gottfried Leibniz}{Leibniz}\hfill
\CorrAnsButton{Isaac Newton or Gottfried Leibniz}\kern1bp\sqTallyBox
\end{questions}
\begin{solution}
Yes, Isaac Newton and Gottfried Leibniz are considered founders of
Calculus.
\end{solution}
\end{shortquiz}
\begin{flushright}
\sqClearButton\kern1bp\sqTallyTotal
\end{flushright}

\redpoint When using objective questions within a
\texttt{shortquiz} environment, you must give a unique field name
as an optional argument of the environment. The listing of this
example follows:

\begin{sverbatim}
\begin{shortquiz}[oQsq]  % <-- unique field name
Answer each of the following. Passing is 100\%.
\begin{questions}

\item If $f$ is differentiable, then $f$ is continuous.
\begin{answers}{4}
\Ans1 True & \Ans0 False
\end{answers}\hfill\sqTallyBox

\item $\displaystyle\frac d{dx} \sin^2(x) =$
\RespBoxMath{2*sin(x)*cos(x)}[sinsqx]{4}{.0001}{0}{1}%
\hfill\CorrAnsButton{2*sin(x)*cos(x)}%
\kern1bp\sqTallyBox
\begin{solution}
$$
    \frac d{dx}\sin^2(x) = 2\sin(x)\cos(x) = \sin(2x)
$$
\end{solution}

\item Name \emph{one} of the two people recognized
as a founder of Calculus.\vadjust{\kern3pt}\newline
\RespBoxTxt{2}{0}[newton]{5}{Isaac Newton}{Newton}{I. Newton}%
{Gottfried Leibniz}{Leibniz}\hfill
\CorrAnsButton{Isaac Newton or Gottfried Leibniz}%
\kern1bp\sqTallyBox
\begin{solution}
Yes, Isaac Newton and Gottfried Leibniz are considered
founders of Calculus.
\end{solution}
\end{questions}
\end{shortquiz}
\begin{flushright}
\sqClearButton\kern1bp\sqTallyTotal %<-- total tally
\end{flushright}
\end{sverbatim}

\noindent\textbf{Example Notes:}
\begin{itemize}
\item Note the optional argument gives this collection of
questions a common base name. All supporting macros use this name.
\item The named destination to the solution is entered with parameter
\texttt{\#5} of the command \hyperlink{paraRespBox}{\cs{RespBoxMath}}, and with parameter
\texttt{\#4} of \hyperlink{paraRespBoxTxt}{\cs{RespBoxTxt}}.
\item In this example, another built-in macro,
\cs{sqTallyTotal} was used. This macro creates a text field that
accumulates the totals of all the tally boxes.
\end{itemize}

\redpoint The \texttt{shortquiz} environment can also be used for
a single objective question. Just don't use the
\texttt{questions} environment within.

\begin{sverbatim}
\begin{shortquiz}[anExample]
< an objective style question >
\end{shortquiz}
\end{sverbatim}


\subsection{The \texttt{quiz} Environment}\label{s:oQquiz}

Objective questions can be mixed in with multiple choice questions
within the \texttt{quiz} environment.  When posing an objective
style question in the \texttt{quiz} environment, use
\cs{RespBoxMath} and \cs{RespBoxTxt}, and optionally include
the \cs{CorrAnsButton}.

Since the evaluation of the quiz is delayed until the user has
finished the quiz, the \cs{sqTally\-Box} macro is not needed.

\begin{quiz*}{oQq}
Answer each of the following. Passing is 100\%.

\begin{questions}
\item If $f$ is differentiable, then $f$ is continuous.
\begin{answers}{4}
\Ans1 True & \Ans0 False
\end{answers}

\item $\displaystyle\frac d{dx} \sin^2(x) =$
\RespBoxMath{2*sin(x)*cos(x)}[sinsqx2]{4}{.0001}{0}{1}%
\hfill\CorrAnsButton{2*sin(x)*cos(x)}%
\begin{solution}
$$
    \frac d{dx}\sin^2(x) = 2\sin(x)\cos(x) = \sin(2x)
$$
\end{solution}

\item Name \emph{one} of the two people recognized as a founder of
Calculus.\vadjust{\kern3pt}\newline
\RespBoxTxt{2}{0}[leibniz]{5}{Isaac Newton}{Newton}{I. Newton}{Gottfried Leibniz}{Leibniz}\hfill
\CorrAnsButton{Isaac Newton or Gottfried Leibniz}
\begin{solution}
Yes, Isaac Newton and Gottfried Leibniz are considered
founders of Calculus.
\end{solution}

\end{questions}
\end{quiz*}\quad\ScoreField{oQq}\eqButton{oQq}

\noindent Answers: \AnswerField{oQq}

\redpoint The buttons created by \cs{CorrAnsButton} are hidden
until the user ends the quiz (and gets scored) and clicks on the
corrections button (\cs{eqButton}).  The \cs{Corr\-Ans\-Button} should
not be included if there is no \cs{eqButton}.

\redpoint If there is a solution to the problem, the ``Ans'' button
is outlined in green. Shift-click on the ``Ans'' button to jump to
the solution.

\redpoint The \texttt{quiz} environment requires a field name.
This same name is used by the objective style question as well.

\medbreak

The listing for the above example follows.

\begin{sverbatim}
\begin{quiz*}{oQq}
Answer each of the following. Passing is 100\%.
\begin{questions}

\item If $f$ is differentiable, then $f$ is continuous.
\begin{answers}{4}
\Ans1 True & \Ans0 False
\end{answers}

\item $\displaystyle\frac d{dx} \sin^2(x) =$
\RespBoxMath{2*sin(x)*cos(x)}{4}{.0001}{0}{1}%
\hfill\CorrAnsButton{2*sin(x)*cos(x)}%

\item Name \emph{one} of the two people recognized
as a founder of Calculus.\vadjust{\kern3pt}\newline
\RespBoxTxt{2}{0}[leibniz]{5}{Isaac Newton}{Newton}{I. Newton}%
{Gottfried Leibniz}{Leibniz}\hfill
\CorrAnsButton{Isaac Newton or Gottfried Leibniz}
\begin{solution}
Yes, Isaac Newton and Gottfried Leibniz are considered
founders of Calculus.
\end{solution}
\end{questions}
\end{quiz*}\quad\ScoreField{oQq}\eqButton{oQq}

\noindent Answers: \AnswerField{oQq}
\end{sverbatim}

\redpoint There are some additional grade reporting and statistical fields
defined: \textsf{exerquiz} defines \cs{PointsField}, \cs{PercentField}, and \cs{GradeField}.
The demo file \texttt{quizpts.tex} illustrates these commands; see also
the section below entitled \nameref{sss:asignPts}.

\subsubsection{The Prompt Button}

In addition to the \cs{CorrAnsButton}, the document author can
provide a prompt button (probably not the best descriptive term).

For some quizzes, the author might want to ask a series of
questions where the answer to one question depends on the correct
answer of a previous question. In this situation, you'd like to
provide the correct answer so the student can make a good run at
the next question. The \cs{@PromptQuestion} not only provides the
answer to the question but it also makes the corresponding math
fill-in read only, so that the student cannot change the answer
already provided.

Ideally, the student first enters an answer, and once satisfied
with the answer, can then get the correct answer for future
questions.

\redpoint See the demo file \texttt{prompt\_tst.tex} for an example of usage.

\subsubsection{Grouped Math/Text Fill-in Questions}\label{grouped}

\textsf{Exerquiz} defines a grouping environment, \texttt{mathGrp},
for math fill-in and \emph{text fill-in} questions where the response to the question
might require entering text into multiple math fill-in fields.

When you use the \texttt{mathGrp} environment to enclose a set of
related math questions, you need to the \cs{CorrAnsButtonGrp}
button, instead of the \cs{CorrAnsButton} button. The required
argument for this button is a comma delimited list of the answers
that appear within the grouped questions. The answers should be
listed in the same order {\TeX} processes the math (or text)
questions. The group of questions is processed as if it were a
single question.

\redpoint For example\dots

\noindent \begin{minipage}{\linewidth}
\useBeginQuizButton[\textColor{0 0 1}\CA{Begin}\AC{}\RC{}\rectW{}]
\useEndQuizButton[\textColor{0 0 1}\CA{End}\AC{}\RC{}\rectW{}]
\begin{quiz*}{grpExample}\PTsHook{($\eqPTs^{\text{pts}}$)}\PTs{3}
Compute the following cross product:
\begin{equation*}
\def\vi#1{\setlength{\fboxrule}{0pt}\setlength{\fboxsep}{1bp}
\fbox{\RespBoxMath[\rectW{11bp}\Q{1}\textSize{0}]{#1}{1}{.0001}{[0,1]}}}
%
    (3\vec i - 2\vec j) \times (\vec i + 5\vec k) = \begin{mathGrp}\PTs*{1}
            \vi{-10}\,\vec i + \vi{-15}\, \vec j + \vi{2}\,\vec k
    \end{mathGrp}\tag*{\CorrAnsButtonGrp{-10,-15,2}}
%
\end{equation*}
\end{quiz*}\ $\underset{\text{ScoreField}}{\underbrace{\ScoreField[\DV{}\V{}\rectW{1in}]\currQuiz}}$%
    \kern1bp$\underset{\text{PointsField}}{\underbrace{\PointsField[\rectW{1in}]\currQuiz}}$\eqButton[\CA{Correct}\AC{}\RC{}]\currQuiz
    \hfill Ans: \AnswerField[\rectW{.5in}]\currQuiz
\end{minipage}

\medskip\noindent
\textbf{Notes:}
\begin{questions}
\item[\color{blue}\ding{042}] If you miss any one of the three
answers, the \texttt{ScoreField} reports back `\texttt{Score: 0
out of 1}'. There is only one question there, to get it correct,
you must answer all three inputs correctly.

\item[\color{blue}\ding{042}] Points can be assigned to the
individual responses and a score is given based on the validity of
the inputs and the corresponding points. There is a default
JavaScript function that scores the results. The document author
can define a custom JavaScript function to have a more ``exotic''
method of evaluating the group.  See the test file
\texttt{grp\_test.tex} for details.

\item[\color{blue}\ding{042}]Notice that after you take the quiz
and click on ``Correct'' button, the ``Ans'' button appears (as
usual). If you click repeatedly on this ``Ans'' button, you can
cycle through all answers to this question; the response box is
highlighted (or put in focus) and the answer appears in the answer
field provided.
\end{questions}

\redpoint See the demo file \texttt{grp\_test.tex} for the source
code of this example, as well as more technical details of the
\texttt{mathGrp} environment.

\subsection{Modifying Form Elements}

All form elements have a first optional parameter for modifying their appearance,
and they have an associated ``every'' command for global modifications as well.

\begingroup
\setlength{\extrarowheight}{1mm}
\setlength{\tabcolsep}{2mm}
\begin{longtable}{|>{\ttfamily}l|>{\PBS\raggedright}p{.5\linewidth}|}
\hline
\multicolumn2{|c|}{\bfseries Global Modification Commands} \\
\hline
\endfirsthead
\hline
\multicolumn2{|c|}{\bfseries Global Modification Commands} \\
\hline
\endhead
\hline
\endfoot
\multicolumn2{|c|}{For the \texttt{shortquiz} Environment} \\
\hline
\cs{Ans} & \cs{everysqRadioButton}\\
\cs{sqTallyBox} &\cs{everysqTallyBox}\\
\cs{sqTallyTotal}& \cs{everysqTallyTotal}\\
\cs{sqClearButton}&\cs{everysqClearButton}\\\hline
\multicolumn2{|c|}{For the \texttt{quiz} Environment} \\ \hline
\cs{useBeginQuizButton}&\cs{everyBeginQuizButton}\\
\cs{useEndQuizButton}&\cs{everyEndQuizButton}\\
\cs{Ans} & \cs{everyqRadioButton}\\
\cs{ScoreField}&\cs{everyScoreField}\\
\cs{eqButton}&\cs{everyeqButton}\\
\cs{AnswerField}&\cs{everyAnswerField}\\
\cs{PointsField}&\cs{everyPointsField}\\
\cs{PercentField}&\cs{everyPercentField}\\
\cs{GradeField}&\cs{everyGradeField}\\\hline
\multicolumn2{|c|}{For Both Environments} \\ \hline
\cs{RespBoxMath}&\cs{everyRespBoxMath}\\
\cs{RespBoxTxt}&\cs{everyRespBoxTxt}\\
\cs{CorrAnsButton}&\cs{everyCorrAnsButton}
\end{longtable}
\par\endgroup

\bigskip
\handpoint In additional to these, there are other ``every'' commands that effect
the appearance of the various buttons and text fields. The two commands
\cs{everyeqButtonField} and \cs{every\-eqText\-Field} are executed before every
\texttt{exerquiz} button and text field (respectfully).  These can be used to give
a general uniform appearance for all the short quiz or quiz form elements; use
the more specific version, as listed in the above table, to make additional refinements
in appearance.

\redpoint See the document \href{eformman.pdf}{eFormMan.pdf} on
\textit{\href{eformman.pdf}{eForm Support}} for complete documentation
on how to modify a field using the optional first argument, and how to use the
``every'' mechanism.

\section{Randomizing the Multiple Choices}\label{s:random}

Beginning with version~6.1 of \textsf{exerquiz}, the choices of a
multiple choice question can be randomized. The \texttt{random.tex}
macro file by Donald Arseneau is used for this purpose.

\exPDF{randomize} The demo file for this section is \texttt{randomize.tex}.

\newtopic The randomization is only allowed if the \texttt{allowrandomize}
option of \textsf{exerquiz} is used; otherwise, no randomization
can occur.

The randomization is only defined for choices listed between the
pair \cs{bChoices} and \cs{eChoices} (see \Nameref{bChoices}). The
command \cs{bChoices} now takes two optional key-value arguments:
\begin{itemize}
  \item \texttt{nCols=<number>}: The number of columns to create, as described
  on~\pageref{bChoices}. You can also use the old style by specifying just
  \texttt{<number>}. Thus, the commands \cs{bChoices[nCols=2]} and \cs{bChoices[2]}
  are equivalent.
  \item \texttt{random=<true|false>}: Specify this option to randomize choices.
  You can use the key word \texttt{random} instead of
  \texttt{random=true}. For example, the following commands will randomize the
  choices, \cs{bChoices[random]}, \cs{bChoices[nCols=2,random]}, or
  \cs{bChoices[2,random]}. The default is to not randomize the
  choices.
\end{itemize}

\newtopic The following is an example of the \texttt{random} option of \cs{bChoices}.
\begin{sverbatim}
\begin{shortquiz}
This is a quiz.
\begin{questions}
    \item Try to guess the correct answer.
    \begin{answers}{3}
    \bChoices[nCols=2,random]
        \Ans0 1 a choice\eAns
        \Ans1\label{eq} 2 another choice\eAns
        \Ans0 3 still another choice\eAns
        \Ans0 4 another\eAns
        \Ans0 5 incoming\eAns
        \Ans0 6 more choices\eAns
        \Ans0 7 another still\eAns
        \Ans0 8 too many\eAns
        \Ans0 9 choices\eAns
    \eFreeze
        \Ans0 10 None of these\eAns
    \eChoices
    \end{answers}
\end{questions}
\end{shortquiz}
\end{sverbatim}

\newtopic Note the presence of the command \Com{eFreeze}.  Any of
the items listed after \cs{eFreeze} are not randomized, and are
placed at the end of the list. So, for the example above, the first
nine items will be randomized, whereas, the last item (None of
these) will placed at the end of the list.

\newtopic Additional, there are five other commands that support the
randomization feature.
\begin{dCmd}[\sCom{saveRandomSeed}\sCom{inputRandomSeed}]{.5\linewidth}
\saveRandomSeed
\inputRandomSeed
\end{dCmd}

\newtopic A pseudo-random sequence of numbers requires an initial ``seed
value.'. The the example file \texttt{random.tex} creates, by default, a
seed value based on the data and time (the number of minutes since
midnight); consequently, after every minute, the random sequence
will change. By setting the value of the count register \cs{randomi},
the document author can also set the initial seed of the pseudo-random
sequence.

The command \cs{saveRandomSeed} will write the last seed used in the
source file to an auxiliary file (\cs{jobname\_ran.sav}), while the
command \cs{inputRandomSeed} inputs the seed stored in the
\cs{jobname\_ran.sav} back into the beginning of the source file.
These two commands should be placed in the preamble.

By invoking both of these commands, a new pseudo-random sequence will be generated
each time the source file is latexed.

Assuming a \cs{jobname\_ran.sav} has already been created, by
invoking the command \cs{inputRandomSeed} only (and not
\cs{saveRandomSeed}), the seed already saved will be used for every
subsequent compiling of the source document. Using the same seed is
necessary in two situations:
\begin{enumerate}
  \item When the quiz contains one or more \cs{label} commands, using the same
  seed gives you the same sequence every time you latex the document. This will
  give the auxiliary files a chance to come up to date so that any referencing of the label
  will be accurate.

  \item When creating an online (or paper) quiz with randomization, which later you publish
  the solutions to, it is important that the randomization for the
  quiz document is the same as that for the solution document. By
  using \cs{inputRandomSeed} (rather than \cs{save\-Ran\-dom\-Seed}), you
  get the same sequence for the solution document (unless you modify the
  source file, adding or removing questions that have randomization).
\end{enumerate}

\newtopic\textcolor{red}{Things to look for:} If \textsf{exerquiz} is not rearranging the order
of the choices as you expect it to, it could be that
\textsf{exerquiz} is reading an old \texttt{.sav} file. Either delete that file
in your source folder, or comment out \cs{inputRandomSeed} in your document.

\begin{dCmd}[\sCom{\useRandomSeed}]{.5\linewidth}
\useRandomSeed{<number>}
\end{dCmd}

You may have several sections of the same class take a quiz with the
questions rearranged for each.  Save the seed value used by
\textsf{exerquiz} to randomized the choices (open the \texttt{.sav}
and copy and paste line you see into your document, for example, it
could read \cs{randomi=132088850}. Then use \cs{useRandomSeed} to
use that seed value for that class, for example
\begin{verbatim}
\useRandomSeed{132088850}   % 11:00 class
% \useRandomSeed{634952429} % 12:30 class
\end{verbatim}
Of course comment out \cs{inputRandomSeed}.

\begin{dCmd}[\sCom{turnOnRandomize}\sCom{obeyLocalRandomize}]{.5\linewidth}
\turnOnRandomize
\obeyLocalRandomize
\end{dCmd}

The command \cs{turnOnRandomize} overrides all local settings of \cs{bChoices}
and causes all choice lists to be randomized. While \cs{obeyLocalRandomize}
returns control to the local settings. For example,
\begin{verbatim}
\turnOnRandomize
...
\bChoices
    \Ans...\eAns
    \Ans...\eAns
    ...
\eChoices
\end{verbatim}
will cause the choice list to be randomized, even though the
\texttt{random} option was not specified. Whereas, in this code
\begin{verbatim}
\turnOnRandomize
...
\obeyLocalRandomize
...
\bChoices
    \Ans...\eAns
    \Ans...\eAns
    ...
\eChoices
\end{verbatim}
the choices will not be randomized, because the \texttt{random}
option was not specified; or they will be randomized if the
\texttt{random} option is used.

\newtopic\textcolor{red}{Limitations:} There are natural limitations on the use
of \cs{bChoices} and \cs{eChoices} and consequently, there are
limitations on the randomization.  The content between \cs{Ans} and
\cs{eAns} cannot have any verbatim text. This is usually not a
problem for mathematical content, but could be a limitation for
computer science where questions about syntax may be posed. I have
in mind a work-around, but haven't pursued the problem as of yet.

\section{Creating a Quiz Summary Table}\label{s:sumrytbl}

This feature is designed for a \texttt{quiz} (as opposed to a
\texttt{shortquiz}); and is probably most useful for a long quiz, spanning
several pages. The \cs{displaySumryTbl} command creates a table that summarizes
the user's effort for the quiz; \cs{displaySumryTbl} must follow the final
\verb!\end{questions}! and \emph{must be on a separate page} from the last question of the quiz.
An example of the generated table is shown below in \hyperref[fig:qzSumryTbl]{Figure~\ref*{fig:qzSumryTbl}}.

\exAeBBlogPDF{sumry_tbl} The demo file for this section is
\href{http://www.math.uakron.edu/~dpstory/aebblog/sumry_tbl.pdf}{sumry\_tbl.pdf}, found
in the \mlhref{http://www.math.uakron.edu/~dpstory/aebblog.html}{AeB Blog}.


\Important The code for these commands are only input when the \textsf{exerquiz} option
\texttt{usesumrytbls} is taken.
\begin{sverbatim}
\begin{quiz*}{<quiz_name>}
Solve each of the following.
\begin{questions}
\item...
...
\item...
\end{questions}

\newpage    %<--start new page
Before you complete your quiz by pressing `End Quiz', review the list
below. A check mark indicates that you responded to that question;
otherwise, you did not respond to the question.

\begin{center}
\displaySumryTbl{\currQuiz}
\end{center}

If you are satisfied with your answers, press `End Quiz' to
complete the quiz.\medskip

\end{quiz*}\quad\ScoreField\currQuiz\eqButton\currQuiz
\end{sverbatim}

\begin{figure}[htb]
\begin{center}
\includegraphics{figures/sumrytbl}
\caption{Quiz Summary Table}\label{fig:qzSumryTbl}
\end{center}
\end{figure}


The table (as seen in \hyperref[fig:qzSumryTbl]{Figure~\ref*{fig:qzSumryTbl}}) shows a check mark for each question to which the student gave a response. The student
can review the table, and optionally return to questions that had no response.

\begin{dCmd}{.67\linewidth}
\displaySumryTbl[<key_vals>]{<quiz_name>}
\end{dCmd}
\newtopic\textbf{Parameter Description:} The second parameter is the quiz name of the quiz this table represents,
in most cases, \cs{currQuiz} is used. The first parameter consists of key-value pairs, these
are described below.

\KVP The first parameter takes key-value pairs
\begin{enumerate}
       \item\texttt{ntables} is a key that has a value of 1 or 2, the
           default is determined by command \cs{smrytbl@ntables}, which is set to 2 in
           this package. If \texttt{ntables=1} only one table is created, and
           if \texttt{ntables=2} two tables are created, each containing half
           the information, as seen in \hyperref[fig:qzSumryTbl]{Figure~\ref*{fig:qzSumryTbl}}.
       \item \texttt{showmarkup}: If this switch is true, then markup fields are
             created that will hold the number of points the student received for
             each problem. The default is that there is no markup. Entering
             \texttt{showmarkup} in the option list will generate the markup fields.
             The code
\begin{verbatim}
    \displaySumryTbl[ntables=1,showmarkup]{\currQuiz}
\end{verbatim}
            produces a single table with the points markup.
       \item \texttt{nocorrections}: If this switch is true, corrections will not
             be shown when the user presses the `End Quiz' button. Because of the
             way this key is implemented, if \texttt{nocorrections} is true, then
             it will be true for any ``duplicate table,'' that is, any table that
             references the same quiz.
\end{enumerate}

\newtopic When the user presses the `Corrections' button, the borders of the
are colored, and if \texttt{showmarkup} is used, the points earned for each
problem are shown as well.

When corrections are made, a red border indicates the answer is
wrong, a green border means the answer is correct, and a
blue border the question was answered incorrectly, but partial credit was
given. The colors are determined by the commands \cs{correctColor}, \cs{wrongColor},
and \cs{partialColor}. These are ``global color'' that need to set in the preamble;
the quiz and shortquiz system uses the first two, change them will yield changes everywhere.
However, you can change the \cs{partialColor} as this is the only place it is used.
\begin{verbatim}
    \newcommand\partialColor{color.blue}
\end{verbatim}
This is a JavaScript color, if you redefined it to be a nonstandard color, for example,
this redefinition makes the partial credit color yellow.
\begin{verbatim}
    \renewcommand\partialColor{["RGB", 1, 1, 0]}
\end{verbatim}

\Important The table must be on a different page from the questions
because there is an open page action that activates and populates the table
based on the user's responses.

\newtopic Other related commands are
\begin{verbatim}
    \renewcommand{\sumryTblQ}{Question}
    \renewcommand{\sumryTblR}{Responded}
    \renewcommand{\sumryTblP}{Page}
\end{verbatim}
The above three commands form the column titles of the table.
\begin{verbatim}
    \renewcommand{\sumrytablesep}{\space}
\end{verbatim}
Sets the separation between the tables, in the case that two tables are generated.
\begin{verbatim}
    \newcommand{\sumryTblProbFmt}[1]{\textbf{\textcolor{blue}{#1}}}
\end{verbatim}
The above command sets the {\LaTeX} formatting of the problems, the
default is bold and blue, but may be redefined.

\subsection{Placing the Quiz Summary Table Elsewhere}

You may want the Quiz Summary Table in a
place other than the default location (after the final
\verb!\end{questions}! and before the `End Quiz'). There is no problem
putting it after the quiz and\emph{ before the next quiz};
however, if you want to put the table up front (that is, in front of the quiz) for
whatever reasons, you need additional commands.

\newtopic In the area just after \verb!\end{questions}! and before `End Quiz,' place the command
\begin{dCmd}{.33\linewidth}
\writeProListAux
\end{dCmd}
This command saves to the auxiliary file the two pieces of information the
\cs{dis\-playSumryTbl} needs to build the table. Then, in the location of
your choice---usually outside of any quiz environment---enclose
\cs{displaySumryTbl} in the \texttt{sumryTblAux} environment.
\begin{dCmd}{.67\linewidth}
\begin{sumryTblAux}{<quiz_name>
\displaySumryTbl[<key-vals>]{<quiz_name>}
\end{sumryTblAux}
\end{dCmd}
Here, \texttt{<quiz\_name>} is the name of the quiz that this table represents. An example
of this environment follows.
\begin{sverbatim}
\begin{sumryTblAux}{\currQuiz}\small %<--make a little smaller
\renewcommand{\sumryTblQ}{\sffamily\bfseries Question}
\renewcommand{\sumryTblR}{\sffamily\bfseries Answered}
\renewcommand{\sumryTblP}{\sffamily\bfseries Page}
\begin{center}
    {\normalsize\sffamily\bfseries Quiz Summary Table}\\[3pt]
    \displaySumryTbl[showmarkup]{\currQuiz}
\end{center}
\end{sverbatim}


%sumryTblAux

\section{Bookmarking Exercises and Quizzes}\label{bmExQz}

It is unlikely that anyone would use this feature: Bookmarking exercises
and quizzes.

\exPDF{bkmrk_eq} The demo file for this feature is  \texttt{bkmrk\_eq.tex}.

\subsection{For Exercises}

This ``intelligent'' bookmarking system requires
the minimum of effort to use. For exercises, the command for bookmarking
is \Com{expdfbookmark},
\begin{verbatim}
    \expdfbookmark{<bookmark_title>}
\end{verbatim}
and should be placed immediately following the beginning of an
exercise environment
\begin{verbatim}
    \begin{exercise}\expdfbookmark{<bookmark_title>}
    \begin{exercise*}\expdfbookmark{<bookmark_title>}
\end{verbatim}
or, in the case of exercises with parts, and you want to bookmark an
individual part to a question, place \cs{expdfbookmark} immediately
after the \cs{item}, like so
\begin{verbatim}
    \item\expdfbookmark{<bookmark_title>}
\end{verbatim}

\newtopic Every bookmark title, \texttt{<bookmark\_title>}, has a
suffix defined with \Com{exbookmarkfmt} or
\Com{partbookmarkfmt}, depending on whether
the entry is for the top-level exercise, or one of its parts.
The default definitions are
\begin{verbatim}
    \newcommand{\exbookmarkfmt}{\exlabel\space\theeqexno.\space}
    \newcommand{\partbookmarkfmt}{(\thepartno)\space}
\end{verbatim}
The command \cs{exlabel} expands to the language
dependent word for `Exercise', while \cs{theeqexno} expands to
the exercise number, and \cs{thepartno} to the part
letter. These commands can be redefined.

\subsection{For Quizzes}

To bookmark a \texttt{shortquiz} or \texttt{quiz}, use \Com{quizpdfbookmark}.
\begin{verbatim}
    \quizpdfbookmark{<bookmark_title>}
\end{verbatim}
It should be placed immediately following the beginning of the environments,
 \begin{verbatim}
    \begin{shortquiz}\quizpdfbookmark{<bookmark_title>}
    \begin{quiz}{<quiz_name>}\quizpdfbookmark{<bookmark_title>}
\end{verbatim}

\redpoint Unlike bookmarking for exercises, the bookmark command for quizzes does not
support bookmarking individual questions within the quiz.

\newtopic Every bookmark title, \texttt{<bookmark\_title>}, for a
\texttt{shortquiz} environment has a suffix defined with the
\Com{sqbookmarkfmt} command, the default definition is

\begin{verbatim}
    \newcommand{\sqbookmarkfmt}{Short Quiz \@shortquizCnt.\space}
\end{verbatim}

\newtopic The command \cs{@shortquizCnt} is a marco (not a counter)
that contains the current short quiz count. This command can be redefined.

\newtopic Every bookmark title, \texttt{<bookmark\_title>}, for a \texttt{quiz} environment  has a
suffix defined with the \Com{qzbookmarkfmt} command, the
default definition is

\begin{verbatim}
    \newcommand{\qzbookmarkfmt}{Quiz \@quizCnt.\space}
\end{verbatim}

\newtopic The command \cs{@quizCnt} is a macro (not a counter)
that contains the current quiz count. This command can be redefined.

\redpoint \cs{quizpdfbookmark} interacts with the command \Com{titleQuiz}.
(for more information on this, see \hyperref[titleQuiz]{Section~\ref*{titleQuiz}}
and \hyperref[titleQuizSQ]{Section~\ref*{titleQuizSQ}}, pages~\pageref*{titleQuiz}
and~\pageref*{titleQuizSQ}, to learn about \cs{title\-Quiz} for quizzes and short quizzes, respectively.) When a quiz has a defined title, as defined
by the \cs{titleQuiz}, and the argument of \cs{quizpdfbookmark} is left empty,
the value passed to \cs{titleQuiz} is used as the bookmark text. Thus, the code

\begin{verbatim}
\titleQuiz{Differentiation Quiz}
\begin{quiz}{MyDiffQuiz}\quizpdfbookmark{}
...
\end{quiz}
\end{verbatim}
will bookmark the quiz under the name ``Differentiation Quiz.''


\subsection{Final Note}

If the bookmarks for exercises, or parts of the exercises, or quizzes do not appear,
you may have to increase the \texttt{tocdepth}. Exerquiz sets this at
\verb!\setcounter{tocdepth}{5}!.

\section{Extending \AcroT{} with  \texttt{dljslib} and \texttt{insdljs}}

The \textsf{exerquiz} Package, especially the math fill-in question, is quite
programmable. In this section, we discuss two methods of extending the
capabilities of the \AcroB: (1) through the use of the package \texttt{dljslib},
which is a JavaScript library of extensions; (2) by writing your own custom
extensions using the \texttt{insdljs} package for inserting JavaScripts into the
PDF document.

\subsection{Using the \texttt{dljslib} Package}\label{sss:dljslib}

The \texttt{dljslib} Package is actually a ``library'' of JavaScript
functions. At the time of this writing, the library has JavaScripts that can
process answers to math fill-in questions where an \textit{equation}
or a \textit{vector} answer is expected. There is also a JavaScript compare
function that properly evaluates an answer when an indefinite integral is
expected.  See the documentation that accompanies the package (by latexing
\texttt{dljslib.dtx}) for details of how to use the library.

\redpoint For full documentation of \textsf{dljslib} package, see \Nameref{part:dljslib}.

%\redpoint \textbf{Equation handling.}  See the sample file \texttt{jqzequat.tex} for examples
%of posing and evaluating questions that expect an equation as the response.  Below is
%a portion of the preamble of that file; basically, to use one or more of the JavaScripts in
%the JavaScript library, you specify that option \cs{usepackage} command for \texttt{dljslib}
%package. In this case, we want to process equations so type\dots
%\begin{sverbatim}
%\documentclass{article}
%\usepackage{amsmath,amscd}
%\usepackage[tight,pdftex,designi,nodirectory]{web}
%\usepackage{exerquiz}
%\usepackage[equations]{dljslib}  % <--choose equations
%\end{sverbatim}
%
%\redpoint \textbf{Vector Handling}  There are also JavaScript functions for processing
%vector answers.  See the sample file \texttt{jqzspec.tex}. Actually this file does not
%use the JavaScript library, but is more of a tutorial on how to use \cs{insdljs} to
%write custom JavaScripts to process \textsf{exerquiz} math fill-in questions.
%
%The preamble of that document could actually be replaced with\dots
%\begin{sverbatim}
%\documentclass{article}
%\usepackage{amsmath,amscd}
%\usepackage[tight,dvipdfm,designi,nodirectory]{web}
%\usepackage{exerquiz}
%\usepackage[vectors,indefIntegral]{dljslib}
%\end{sverbatim}
%\noindent The \texttt{vectors} option specifies JavaScripts for processing
%vector questions. The option \texttt{in\-def\-In\-te\-gral} is also
%specified. This is because that in the file \texttt{jqzspec.tex} a
%comparison function is developed for properly evaluating questions
%in which an indefinite integral is expected.
%
%\redpoint \textbf{Comma Delimited Answers and Sets} The
%\texttt{setSupport} option of the \textsf{dljslib} Package
%provides basic support for math fill-in questions requiring a
%(unordered) list of comma delimited numerical or symbolic answers,
%or for a set. The demo and tutorial file is
%\texttt{set\_test.tex}. See that file for more details.
%
%\redpoint In addition to the two above mentioned sets of JavaScripts there are
%a couple of comparison functions, one for processing indefinite integrals (see
%\texttt{dljs\_ex.tex}), and the other for using relative absolute error rather
%than absolute error. Again, see the documentation of \texttt{dljslib.dtx} and
%the sample file \texttt{jqzspec.tex}.

\subsection{Using the \texttt{insdljs} Package}

With the \texttt{insdljs} package you can write your own JavaScript functions
right in the \LaTeX{} source file.  These custom JavaScripts are then inserted
into the section of the PDF document where the document level JavaScripts
reside. This package is a stand-alone package, and does not need \textsf{exerquiz},
though \textsf{exerquiz} now uses this package to insert its JavaScripts into
the document.

\redpoint Refer to \href{\bUrl/eformman.pdf}{eformman.pdf}
(relative link \href{eformman.pdf}{eformman.pdf}) for the \textsf{insdljs} and the \textsf{eforms}
documentation. Also, see the sample file \texttt{insdljs\_ex.tex}
for examples that do not use \textsf{exerquiz}, and the file
\texttt{jqzspec.tex}, for examples that does use \textsf{exerquiz}.

\section{Submitting a \protect\texttt{quiz} to a Web Server}

Quizzes created by the \texttt{quiz} environment are entirely self-contained.
They function within the Web browser (or from within the Adobe Reader) and
do not communicate with any server. This kind of quiz is ideal for a
do-it-yourself tutorial system, read by a well-motivated student who has the
discipline to read the material and to take the quizzes in the spirit in which
they are given.

However, some educators, myself included, may wish to use the quizzes created
by the \texttt{quiz} environment for classroom credit. It is necessary,
therefore, for the student to be able to submit quiz results to a Web server
which, in turn, should store the results to a database.

In this section we discuss techniques of turning the quiz into something that
can be submitted to a server.

\redpoint I have released the \textsf{eq2db} Package, a {\LaTeX} macro package and
server-side script to process \textsf{exerquiz} quizzes. See \Nameref{eq2db}.

\subsection{Technical Info for ``Do It Yourself''}\label{DIY}

All one really has to do is to redefine the ``End Quiz'' link or button to
submit the results of the quiz to the Web server and CGI of your choice. Since
the quiz itself is scored, (optionally) marked, with (optional) answers and
solutions provided, the CGI simply stores the quiz results to a database.

\subsubsection{Redefining ``End Quiz''}

I've written the ``End Quiz'' link (button) to have various
programming hooks available to the developer.
The following code is common to both \cs{eq@EndQuizLink} and
\cs{eq@EndQuizButton}, the macros that control the action of
the end quiz link and button, respectively.
\begin{sverbatim}
if (\minQuizResp(\thequestionno)) {\r\t
    var f = this.getField("ScoreField.\curr@quiz");\r\t\t
    if ( f != null )\r\t\t\t
        this.getField("ScoreField.\curr@quiz").value
            =(\eq@QuizTotalMsg);\r\t\t
    \eq@submitURL
    resetQuiz("\curr@quiz")\r\t
}
\end{sverbatim}

\redpoint The code is a mixture of \LaTeX{} macros and JavaScript.
You can see from this code, that there is a submit hook macro
provided, \cs{eq@submitURL}. Normally, this macro has a definition
of \cs{@empty}. A developer needs only redefine this macro
accordingly; one would use the Acrobat JavaScript method
\texttt{this.submitForm()} to do this. See the \textit{Acrobat JavaScript
Scripting Reference} \cite{tech:AcroJS} for more detail about this
method.

\redpoint The code flow above is as follows: (1) Execute this code if the
threshold has been met. (See \nameref{threshold}.) The text macro
\cs{curr@quiz} holds the base name of the current quiz.

(2) If the field \texttt{"ScoreField.\cs{curr@quiz}"} exists, then
write the student's score to that field. (This is the ``Score: 2
out of 3'' that you see in the demo quizzes.)

(3) We then submit with the macro \cs{eq@submitURL}. (This would
do nothing if its value is \cs{empty}, the default value.) At this
point we call a \texttt{resetQuiz("\cs{curr@quiz}")} which
sets some values in an array to indicate the state of this quiz.

\subsubsection{Gathering ID Information with
\texorpdfstring{\protect\cs{textField}}{\textbackslash textField}}

\redpoint What kind of information would one submit to a CGI?
Well, there is the usual information concerning the identity of
the student (Name, SSN, etc.) and the course, section and so on.

This basic information can be gathered from the student by
inserting text fields into the document to be filled in. \textsf{Exerquiz}
provides the macro \cs{textField}\footnote{You can also use hyperref's
\cs{TextField} command for this purpose as well.} for this purpose.
For example,
\begin{sverbatim}
\newcommand\FirstName[2]{\textField
    [\DV{First Name}\textFont{TiRo}\textSize{10}\textColor{0 0 1}]
    {IdInfo.Name.First}{#1}{#2}}
\end{sverbatim}
\noindent This defines a text field with a name of \texttt{"IdInfo.Name.First"}, the two
arguments are the width and height of the field that you want to create. E.g.,
\begin{sverbatim}
\FirstName{100pt}{10pt}
\end{sverbatim}
\noindent creates a text field 100pt wide and 10pt high.

The \cs{textField} macro takes four parameters.
\begin{sverbatim}
\textField[#1]#2#3#4
\end{sverbatim}
\noindent The first (optional) parameter can be used to custom
design the field; the second is the name of the field; the third
and fourth are the width and height of the field desired.

\redpoint See the file \texttt{eqformman.tex} on {\cAcroT} \textsf{eForm}
support for complete documentation on
\cs{textField}.

\subsubsection{Gathering Quiz Specific Information with
\texorpdfstring{\protect\cs{eqSubmit}}{\textbackslash eqSubmit}}

In addition to ID information on the one taking the quiz,
specific information about what quiz is being taken and where the
results of the quiz are to be stored are needed as well.

\textsf{Exerquiz} provides a basic macro, called \cs{eqSubmit}
that can be used to gather basic formation of this type. The definition
of it and related commands are given below:
\begin{sverbatim}
\newcommand\databaseName[1]{\def\db@Name{#1}}\def\db@Name{}
\newcommand\tableName[1]{\def\db@Table{#1}}\def\db@Table{}
\newcommand\eqCGI[1]{\def\eq@CGI{#1}}\def\eq@CGI{}
\newcommand\eqSubmit[3]
    {\eqCGI{"#1"}\databaseName{#2}\tableName{#3}}
\end{sverbatim}

\noindent The meaning of the parameters are self-explanatory.

Just prior to the quiz you can type:
\begin{sverbatim}
\eqSubmit{http://www.myschool.edu/cgi-bin/myCGI.cgi}%
    {CalcIII}{Quizzes}
\begin{quiz*}{Quiz3} Answer each of the following.
\begin{questions}
...
...
\end{questions}
\end{quiz*}\quad\ScoreField\currQuiz\eqButton\currQuiz

\noindent
Answers: \AnswerField\currQuiz
\end{sverbatim}

\redpoint Any redefinition of \cs{eq@submitURL} would then
include the values of some or all of these text parameters:
\begin{sverbatim}
\eq@CGI, \db@Name, \db@Table, \curr@quiz
\end{sverbatim}
\noindent The last text macro is not gathered by \cs{eqSubmit}, it
is known, however, at the time \cs{eq@submit\-URL} is expanded.

\subsubsection{Some Variables to Submit}

When you submit a quiz to a server, the values of \textit{all}
fields are also submitted, unless you define specifically which
fields are to be submitted.

In addition to the ID info, you would like also to submit the results
of the quiz itself. The relevant variables are as follows:
\begin{questions}
    \item The JavaScript variable \texttt{Score} has the number of
    correct responses as its value.
    \item The \LaTeX{} counter variable \cs{thequestionno} has the count
    of the total number of questions in the quiz.
    \item The JavaScript array \texttt{Responses} contains the responses of
    the student: multiple choice and fill-in responses. The contents of this
    array can be converted to a comma-delimited string by using the
    \texttt{toString()} method, \texttt{Responses.toString()}.
\end{questions}

\noindent Now, how does one submit these values?  The \cs{eq@submitURL}
command can be used not only to submit the data, but to also
populate certain \textit{hidden} fields with this information. The
hidden data is submitted along with the ID info to be processed.
You can use the \cs{textField} to create hidden text fields for
this purpose. See the next section for a discussion of how to create
hidden text fields.

\subsection{Features \protect\textit{apropos} to Submitting}

%\subsubsection{\texorpdfstring{\protect\cs{RespBoxEssay}}
%{\textbackslash RespBoxEssay}: The Essay Question}
%
%Some disciplines make heavy use of the essay question. In an effort to
%expand \textsf{exerquiz} to this question type, the \cs{RespBoxEssay}
%question has been defined.
%
%The syntax is
%\begin{sverbatim}
%\RespBoxEssay[<commands to modify the appearance>]
%\end{sverbatim}
%The default width of is \cs{linewidth} and the default height is
%\verb+6\baselineskip+.  For example,
%
%\begin{oQuestion}{bioNewton}
%\redpoint Sketch the biography of Isaac Newton:\par
%\RespBoxEssay[\rectH{3\baselineskip}]\par\smallskip
%\end{oQuestion}
%
%This type of question is not evaluated by JavaScript, but is meant
%to be submitted to a CGI for storage in a database. Later, the
%instructor can review the essay answer.

\subsubsection{Assigning Points}

See \Nameref{sss:asignPts} for information on this subject.

\subsubsection{\texorpdfstring{\protect\cs{NoPeeking}}
{\textbackslash NoPeeking}}\label{sss:nopeek}

If you execute the command \cs{NoPeeking} in the preamble of your
document, or prior to a \texttt{quiz}, then any quiz question with
solution will be protected somewhat from prying eyes.

In this case, an open page action is placed on the first page of
each solution. If the user (student) tries to view a quiz solution
before doing the quiz, the \textcolor{blue}{Adobe Reader} will
automatically change the page to the page containing the quiz and
place an alert box on the screen saying that viewing the solution
before taking the quiz is not permitted.

To resort to the default behavior, use the \cs{AllowPeeking}
command.

The previous \texttt{quiz} has been surrounded with a
\cs{NoPeeking}/\-\cs{AllowPeeking} pair. If you go to one
of the \hyperlink{knuth}{solutions} to that quiz, you will see
what happens.  If nothing interesting happens, read the next
red point.

\redpoint Protection is removed when you click on ``End Quiz'' and
restored when you click on some ``Begin Quiz''.

\section{Functions and Syntax supported by \textsf{Exerquiz}}

In addition to the JavaScript built in math functions, exerquiz defines a number
of other functions useful in mathematical work. These functions are typically input
in a math question created by \cs{RespBoxMath}, for example, or in the function input
field of an {\AcroFLeX} document.\footnote{The {\AcroFLeX} Graphing Bundle,
\url{http://www.math.uakron.edu/~dpstory/acroflex.html}}

Exerquiz also has its own syntax for entering algebraic expression in the
\cs{Resp\-Box\-Math} field, these are document here as well.

\subsection{Functions supported by \textsf{Exerquiz}}

The following functions are built-in to JavaScript:

\setlength{\extrarowheight}{1mm}
\setlength{\tabcolsep}{2mm}

\noindent\begin{longtable}{@{}>{\ttfamily}lp{2.8in}>{\PBS\raggedright}p{1.5in}}
\hline
\multicolumn{1}{@{}l}{Function}&\multicolumn{1}{l}{Description}&\multicolumn{1}{l}{Usage}\\
\hline
\endfirsthead
\hline
\multicolumn{1}{@{}l}{Function}&\multicolumn{1}{l}{Description}&\multicolumn{1}{l}{Usage}\\
\hline
\endhead
\hline
\endfoot
abs\textrm{, }|$\cdot$| & absolute value function, preferred form, \texttt{|$\cdot$|} & \texttt{|-3|=3}, \texttt{|x|}\\
sin & the sine function & \texttt{sin(pi/2)=1}, see the definition of \texttt{pi} below\\
cos & the cosine function & \texttt{cos(1.5)}, \texttt{cos(2*x)}\\
tan & the tangent function & \texttt{tan(pi/4)=1}\\
asin& inverse sine function   & \texttt{asin(x)}\\
acos& inverse cosine function & \texttt{acos(.2)}\\
atan& inverse tangent function& \texttt{atan(-4)}\\
exp & the natural exponential function, this function is also implemented as a \verb!e^x!,
      where \texttt{e} is the natural number & \texttt{exp(-.5)}, \texttt{exp(-x\string^2)}\\
log & the common logarithm. In JavaScript log is the natural logarithm, \textsf{exerquiz}
    redefines this to be the common logarithm, use \texttt{ln} for natural logarithm & \texttt{log(10)=1}\\
sqrt & the square root function & \texttt{sqrt(4.5)}\\
ceil & the \texttt{ceil} function, \texttt{ceil(x)} is the smallest integer greater than or equal to \texttt{x} & \texttt{ceil(4.5)=5}\\
floor&the \texttt{floor} function, where \texttt{floor(x)} is the largest integer less than or equal to \texttt{x} & \texttt{floor(4.5)=4}\\
max & the maximum between two numbers & \texttt{max(2, 3.4) = 3.4}\\
min & the minimum between two numbers & \texttt{max(2, 3.4) = 2}\\
random & a pseudo-random number generator, this function is used internally by \textsf{exerquiz}, and is
    available to users, \texttt{random()} returns a number between  0.0 and 1.0 & \texttt{random()}\\
round & a rounding function, rounds a number to the nearest integer
      & \texttt{round(4.5)= 5}, \texttt{round(-3.3)=-3}
\end{longtable}

\bigskip

\newtopic The next set of functions are defined by \textsf{exerquiz}.

\noindent\begin{longtable}{@{}>{\ttfamily}lp{2.8in}>{\PBS\raggedright}p{1.5in}}
\hline
\multicolumn{1}{@{}l}{Function}&\multicolumn{1}{l}{Description}&\multicolumn{1}{l}{Usage}\\
\hline
\endfirsthead
\hline
\multicolumn{1}{@{}l}{Function}&\multicolumn{1}{l}{Description}&\multicolumn{1}{l}{Usage}\\
\hline
\endhead
\hline
\endfoot
cot & the \texttt{cotangent} function & \texttt{cot(1)}\\
sec & the secant function & \texttt{sec(2)}\\
csc & the \texttt{cosecant} function & \texttt{csc(2)}\\
arcsin & the inverse sine function, same as \texttt{asin} & \texttt{arcsin(.5)}\\
arccos &  the inverse cosine function, same as \texttt{acos} & \texttt{arccos(-.5)}\\
arctan & the inverse tangent function, same as \texttt{atan} & \texttt{arctan(1)}\\
ln & the natural logarithm & \texttt{ln(3.4)}\\
sinh & the hyperbolic sine function & \texttt{sinh(0)}, \texttt{sinh(x)}\\
cosh & the hyperbolic cosine function & \texttt{cosh(0)}, \texttt{cosh(x)}\\
tanh & the hyperbolic tangent function & \texttt{tanh(x)}\\
coth & the hyperbolic cotangent function & \texttt{coth(x)}\\
sech & the hyperbolic secant function & \texttt{sech(x)}\\
csch & the hyperbolic cosecant function & \texttt{csch(x)}\\
asinh & the inverse hyperbolic sine function & \texttt{asinh(0)}, \texttt{asinh(x)}\\
acosh & the inverse hyperbolic cosine function & \texttt{acosh(0)}, \texttt{acosh(x)}\\
atanh & the inverse hyperbolic tangent function & \texttt{atanh(x)}\\
acoth & the inverse hyperbolic cotangent function & \texttt{acoth(x)}\\
asech & the inverse hyperbolic secant function & \texttt{asech(x)}\\
acsch & the inverse hyperbolic cosecant function & \texttt{acsch(x)}\\
sgn & the sign function, \texttt{sgn(x)} is -1 if \texttt{x} is negative, 1 if \texttt{x} is positive, and 0 of \texttt{x} is zero
    & \texttt{sgn(-3.2) = -1} \\
cis & the polar representation of a complex number,
      $\operatorname{cis}(\theta)=\cos(\theta)+\imath\sin(\theta)$. This function
      is only defined if the \texttt{complex} option of \textsf{dljslib} is used.
      & \texttt{cis(pi/3)}
\end{longtable}

\bigskip

\newtopic There are several other \textsf{exerquiz}-defined
functions that are available through options of the
\textsf{dljslib}, these are \texttt{C} (combinations), \texttt{P}
(permutations), and \texttt{fact} (permutations). See the section
titled \Nameref{extNumerFuncs} for details.

\paragraph*{Recognized Constants.} \textsf{Exerquiz} recognizes
\texttt{PI} and \texttt{pi} as symbols for the mathematical constant
$\pi\approx \text{3.14159}$. The letter \texttt{e} is recognized as
the natural number, which is approximately 2.7182818285.

\subsection{Syntax supported by \textsf{Exerquiz}}

\textsf{Exerquiz} attempts to make entering mathematics as easy and
as natural as using a modern calculator. The following paragraphs
outline the syntax of \textsf{exerquiz}. Authors should reproduce
some form of these rules to their students, and perhaps some
practice documents for them to test themselves with.

\paragraph*{Grouping.} Algebraic expressions are grouped with
matching parentheses~\texttt{(\,)}, brackets~\texttt{[\,]}, or
braces~\texttt{\{\,\}}.

\paragraph*{Spaces.} When the user enters a math expression, all
spaces are removed before parsing of the expression begins.
Students can write \texttt{1+(2+3)} or \texttt{1 + ( 2 + 3 )}, these
are the same.

\paragraph*{Addition and Subtraction.} These operations are
performed using \texttt{+} and \texttt{-}, as expected. For example,
\texttt{(x+1)-(x-2)}, \texttt{x+sin(x-pi)}

\paragraph*{Multiplication and Division.} These operations are
performed using \texttt{*} and \texttt{/}, as expected. For example,
\texttt{2*x*(3*x+1)} and \texttt{(2*x-1)/(2*x+1)}.

The use of the multiplication (\texttt{*}), is required, by default,
and its use can be very tedious to the student. If you include the
\textsf{dljslib} package with the \texttt{ImplMulti} option,
\texttt{exerquiz} automatically inserts the multiplication symbol
(\texttt{*}). The examples above, can be written as
\texttt{2x(3x+1)} and \texttt{(2x-1)/(2x+1)}; it is recommended,
therefore, that the \texttt{ImplMulti} option should normally be
used. (Note: The use of \texttt{*} is still correct even when
\texttt{ImplMulti} option is in effect.)

\paragraph*{Powers and Exponents.} \textsf{Exerquiz} uses
\texttt{\string^} to indicate exponents, for example,
\texttt{x\string^2} means $x^2$. Other examples are \verb!17^(x+1)!,
\verb!(x^2+1)^3!, \verb!(sin(x))^3!, and \verb!(x+1)^(2x+1)!.

Recall that the letter \texttt{e} is recognized as the natural
number, so \verb!e^{-x^2}! is the same as \verb!exp(-x^2)!.
Normally, the use of \texttt{e} is preferred by students over the
use of \texttt{exp}. For work in elementary mathematics, the
existence of \texttt{exp} need not be even mentioned to the
students.

The difficulty of entering powers of functions is another issue for
students. As illustrated above, in entering \verb!(sin(x))^3! or
more complex expression, students find a source of difficulty, pain,
and anguish. \textsf{Exerquiz} allows entering powers in the
calculator style, the student can enter \verb!sin(x)^3! for
$(\sin(x))^3$ (often written as $\sin^3(x)$) or \verb!3^sin(x)! to
mean $3^{\sin(x)}$. Expressions like \verb!2^2^3! are interpreted as
\verb!(2^2)^3! (the typeset version is $(2^2)^3$), which evaluates
to 64. The use of parentheses is needed if the desired expression is
$2^{2^3}$; this expression needs to be typed as \verb!2^(2^3)!,
which evaluates to 256.

If the \texttt{ImplMulti} option of the \texttt{dljslib} package is
in effect, powers of functions can be entered in a more
mathematically traditional, for example \verb!sin^2(x)! is
recognized as the same as \verb!sin(x)^2!, or \verb!(sin(x))^2!;
consequently, \verb!sin^2(x)^2! is the same as \verb!sin(x)^4!!


\part{The \protect\textsf{eq2db} Package}\label{eq2db}

 As the name suggests, this package facilitates submitting an
\texttt{Exerquiz} quiz to a CGI for storage in a database. The
{\LaTeX} package itself does very little other than to define some
useful commands that make it easy to convert a self-contained quiz
into one that is submitted to server-side script.

The \texttt{eq2db} currently has two options, \texttt{eqRecord} and
\texttt{custom}:
\begin{sverbatim}
\usepackage[eqRecord]{eq2db}
\end{sverbatim}
\noindent The option \texttt{eqRecord} sets up the quiz to use an
ASP (Active Server Page) that I have written. This ASP, named
naturally, \texttt{eqRecord.asp}, takes the data and stores it to
a database, such as Microsoft Access.

%The other option
%\begin{sverbatim}
%\usepackage[quaFDF]{eq2db}
%\end{sverbatim}
%\noindent sets things up to use ``The quaFDF Package'', a series
%of ASP that I wrote during the summer, 2001, with the support of
%Adobe Systems, Inc. This ASP is much more powerful and flexible
%than \texttt{eqRecord.asp}. It can be used to store data into
%different tables from the same database. You can also use it to
%access and edit (update) the values of the database, all in PDF.

There is also a \texttt{custom} option. With this option, a
developer can write \LaTeX{} code to set the quiz up for
submission to a CGI used or written by the developer.

\redpoint For more details, see \href{eq2dbman.pdf}{eq2dbman.pdf},\footnote{The absolute URL is \url{\bUrl/acrotex/eq2dbman.pdf}}
the documentation for the distribution. The \textsf{eq2db} package is available for download
at the {\cAcroT} home page:
$$
\text{\url{\bUrl/webeq.html}}
$$

\part{{\AcroT} \protect\textsf{eForm} Support}

In this document, we describe the support for form elements in an
{\cAcroT} document. The PDF Specifications indicate there are four
different categories of fields for a total of seven types of
fields.

\begin{questions}
\setbox0=\hbox{(d)}
\def\I#1{\makebox[\wd0]{\color{blue}(\hfil#1\hfil)}\ }

    \item \textbf{Button Fields}
        \begin{description}
            \item[\I{a}] \textbf{Push Button}
            \item[\I{b}] \textbf{Check Box}
            \item[\I{c}] \textbf{Radio Button}
        \end{description}
    \item \textbf{Choice Fields}
        \begin{description}
            \item[\I{a}] \textbf{List Box}
            \item[\I{b}] \textbf{Combo Box}
        \end{description}
    \item \textbf{Text Fields}
    \item \textbf{Signature Fields}

\end{questions}
The {\cAcroB} does not support \textit{signature fields}, this
leaves six types of fields. Commands for creating each of the
remaining six types will be discussed.

The \textsf{hyperref} Package (Rahtz, Oberdiek \textit{et al})
provides support for the same set of form fields; however, not all
features of these fields can be accessed through the
\textsf{hyperref} commands.  I was determined to write my own set
of commands which would be sufficiently comprehensive and
extendable to suit all the needs of the \cAcroB.  All the quiz
environments have been modified to use this new set of form
commands, in this way, there is a uniform treatment of all form
fields in \cAcroB.

\redpoint The documentation for \textsf{eForm} support is too voluminous
to include in this already voluminous document. See
\href{\bUrl/eformman.pdf}{eformman.pdf} (relative link
\href{eformman.pdf}{eformman.pdf}) for complete details.


\part{The AeB JavaScript Library}\label{part:dljslib}


The \textsf{dljslib} package---document level JavaScript
library---is a companion to \textsf{exerquiz} and utilizes
\textsf{insdljs} package. It is a library of JavaScript code that
can be referenced in the preamble of a {\LaTeX} document, the code
is then extracted from the package files, and inserted into the {\LaTeX}
source using the \texttt{insDLJS} environment.

The code belonging to the library supports the processing
of the quizzes and short quizzes produced by \textsf{exerquiz}; they
extend the capabilities of the core features of the
\textsf{exerquiz} quizzes. Many of the JavaScript functions in the
library were contributed by  users of AeB.

\newtopic My deep appreciation goes out to the contributors to the AeB JavaScript library.
\begin{itemize}
    \item The options \texttt{unordered}, \texttt{factors},
        \texttt{point} and \texttt{intervals} were written by Dr. Bruce Wagner
        for an online grading system developed by Drs.\ Bruce Wagner and David
        Arnold, and Mr.~Jacob Miles-Prystowsky. Dr. Robert Marik contributed to the \texttt{intervals} option, as well.
    \item The options \texttt{nodec}, \texttt{noBinFact} and \texttt{combinatorics} were contributed by
    Ross Moore and Frances Griffin, and were developed for their
    \href{http://rutherglen.ics.mq.edu.au/~macqtex/}{MacQ\TeX}
    online testing system.\footnote{\url{http://rutherglen.ics.mq.edu.au/~macqtex/}}
\end{itemize}
The AeB JavaScript library continues to accept contributions from AeB developers.

\section{Usage and Options}

As with all {\LaTeX} package, you use \textsf{dljslib} as follows,
\begin{verbatim}
    \usepackage[<options>]{dljslib}
\end{verbatim}
With each of the \texttt{<options>}, the associated JavaScript code is brought into the
{\LaTeX} source document. The JavaScript functions, then, are available to your
\textsf{exerquiz} quizzes.

A list of the \texttt{<options>} of the \textsf{dljslib} package are listed below,
and each are described in detail in the sections that follow.
\begin{enumerate}
\item \texttt{ImplMulti}: Use this option to allow implied multiplication, e.g., the user can input
    \texttt{2xsin(x)} instead of \texttt{2*x*sin(x)}. See \Nameref{ImplMulti}.
\item \texttt{equations}: Use this option to process questions that take an equation as an answer.
    See \Nameref{equations}.
\item \texttt{vectors}: Use this option to process questions that take a vector as an answer.
    See \Nameref{vectors}.
\item \texttt{setSupport}: Use this option to process questions that take a set as an answer.
    See \Nameref{setSupport}.
\item \texttt{unordered}: Use this option to process questions that take an unordered list as an answer.
    See \Nameref{unordered}.
\item \texttt{factors}: Use this option to process questions that take a factored form of a polynomial.
    See \Nameref{factors}.
\item \texttt{point}: Use this option to process questions that take an ordered pair $(x,y)$ as an answer.
    See \Nameref{point}.
\item \texttt{intervals}: Use this option to process questions that take an interval, or a union of intervals
    as an answer. See \Nameref{intervals}.
\item \texttt{indefIntegral}: Use this compare function when evaluating the response to an indefinite integral.
    See \Nameref{indefIntegral}.
\item \texttt{nodec}: Use this filter function to deny the use of decimal numbers in an answer.
    See \Nameref{nodec}.
\item \texttt{noBinFac}: User this filter function to deny the use of \texttt{combinatorial} functions.
    See \Nameref{noBinFac}.
\item \texttt{limitArith}: Use any of this collection of filter functions to limit the use of arithmetic operations
    in an answer. See \Nameref{limitArith}.
\item \texttt{combinatorics}: Use this option to add combinatorial functions. The user may use these in an answer,
    unless denied by the \texttt{noBinFact} filter.
    See \Nameref{combinatorics}.
\end{enumerate}

\newtopic For a document author that is building a system of quizzes and who regularly
uses a subset of these options, sets of options can be grouped
together into a custom option. Use the \texttt{libcusopt.opt} file
to declare your own sets of options. These declarations should be a
combination of existing options. For example, The declarations in the
\texttt{libcusopt.opt} file
\begin{sverbatim}
\DeclareOption{procrespgrp}{\includeOptions{%
        vectors,setSupport,equations,intervals,limitArith,%
        unordered,point,factors,ImplMulti}
}
\end{sverbatim}
creates a new option for \textsf{dljslib} called
\texttt{procrespgrp} which bundles together all frequently used
options. The file \texttt{libcusopt.opt} can be placed in the
{\LaTeX} search path, but it's perhaps best to put it in your source
file, so that the options are local to that folder. The command
\cs{includeOptions} is defined in the \textsf{insdljs} package.

\exPDF{dljslib_apw} A sample \texttt{libcusopt.opt} is distributed with the {\AEB},
this sample file consists of the example given above, and is used
by the sample file \texttt{dljslib\_apw.tex}, which can be found in the folder \texttt{dljslib\_examples}.

\newtopic The JavaScript functions are organized into five categories:
\begin{questions}
    \item \nameref{piExt} are support functions whose capabilities are built
    into the core \textsf{exerquiz} package. When \textsf{exerquiz} detects the presence of these
    support functions, it utilizes them seamlessly.
    \item \nameref{respfunc} are functions the take over the task of processing the user input. Discussion
    of these functions begins on page~\pageref*{respfunc}.
    \item \nameref{compareFuncs} are functions that perform the comparison between the author's answer
    and the student's answer. Based on this comparison, the student's answer is judged correct or not.
    Refer to page~\pageref*{compareFuncs} for the compare functions.
    \item \nameref{filterUserResp} are functions that are called immediately after the user
    commits the response. A filter function can examine the response before passing it on to
    the response function.  The filter function can cancel the user input if the input does not
    meet certain criterion. The filter functions begin on page~\pageref*{filterUserResp}.
    \item \nameref{extNumerFuncs} extend the collection of built-in functions that perform
    numerical calculations. These numerical functions are discussed beginning on page~\pageref*{extNumerFuncs}.
\end{questions}

\def\respBoxPara{%
\texttt{\hyperlink{paraRespBox}{\string\RespBoxMath[\#1]\#2(\#3)[\#4]\#5\#6\#7\#8[\#9]*\#10}}}

\subsection{Parse Input Extensions}\label{piExt}

This type of JavaScript functions are support functions whose
capabilities are built into the core \textsf{exerquiz} package.
When \textsf{exerquiz} detects the presence of these support
functions, it utilizes them seamlessly.

\subsubsection{\texttt{ImplMulti}}\label{ImplMulti}

The default behavior for processing a math fill-in question is to
require the student to insert `\texttt*' for multiplication and to
enclose any function being raised to a power in parentheses, e.g.
\verb+x*(sin(x))^2+. This becomes quite tiresome if the expression to be entered
is complicated. The \texttt{ImplMulti} option enables the student to use
\emph{implied multiplication} and \emph{simplified exponentiation}.

The \texttt{ImplMulti} option loads two JavaScript functions, \texttt{Ck4Products}
and \texttt{Ck4Expo\-nents}. The latter implements the notation, \verb+sin^2(x)+
for $\sin^2(x)$, and is equivalent to $(\sin(x))^2$. The a more complex exponent must be enclosed enclosed
in parentheses, for example \verb-ln^(x+1)(x)-. The former function, \texttt{Ck4Products},
inserts the multiplication symbol, `\texttt{*}', wherever multiplication is implied. For example,
\verb+xysin(xy)+ becomes \verb+x*y*sin(x*y)+.

\newtopic\textcolor{red}{Important:} If loaded, \texttt{exerquiz}
will automatically use these two functions in any math fill-in question, no further action is needed.

\exPDF{jquiztst} Many of the demo files that come with AeB use the \texttt{ImplMulti} option, one such
representative file is \texttt{jquiztst.tex}.

\subsection{Response Functions} \label{respfunc}

For fill-in questions, a response function is one that is called immediately after the user
enters and commits an answer. The default response function that processes the user's response
is \texttt{ProcResp}. The $\text{10}^{\text{th}}$ argument of the \cs{RespBoxMath} command,
\begin{equation*}
  \text{\respBoxPara}
\end{equation*}
is used to specify another response function. This section discusses the alternate
response functions available in the library, and gives examples of each.

\subsubsection{\texttt{equations}}    \label{equations}

Use this option to process questions for which an equation is the
expected answer. This option defines the JavaScript function
\texttt{ProcRespEq}. When using a math fill-in question (the
\cs{RespBoxMath} command), this function is placed in the optional
$\text{10}^{\text{th}}$ argument, the one that follows \texttt{*}.
See \Nameref{ss:respbox} for a description of the parameters of
\cs{RespBoxMath}

\newtopic The following question goes inside a \texttt{quiz} or
\texttt{shortquiz} environment.
\begin{sverbatim}
\RespBoxMath{y = 4 * x - 3}(xy){4}{.0001}{[0,1]x[0,1]}*{ProcRespEq}
\end{sverbatim}

\exPDF{jqzequat} The demo file \texttt{jqzequat.tex} includes the
above example as well as other in the context of a working
\texttt{shortquiz}, found in the \texttt{dljslib\_examples} folder.

\newtopic Alert box message: May be redefined.
\begin{sverbatim}
    \newcommand\equationsAlertMsg{"An equation is expected"}
\end{sverbatim}

\subsubsection{\texttt{vectors}}\label{vectors}

Use this option to process questions for which a vector is the
expected answer. This option defines the JavaScript function
\texttt{ProcVec}. When using a math fill-in question (the
\cs{RespBoxMath} command), this function is placed in the optional
$\text{10}^{\text{th}}$ argument,
the one that follows \texttt{*}. See \Nameref{ss:respbox} for a
description of the parameters of \cs{RespBoxMath}.

\newtopic The following question goes inside a \texttt{quiz} or
\texttt{shortquiz} environment. The premise of these two questions
is that the definitions of $\vec a$, $\vec b$ and $f(t)$ are given:
\begin{sverbatim}
\item $\vec a + \vec b = \RespBoxMath{<4, 4, 4>}{1}{.0001}24*{ProcVec}$
\item $\vec f'(t) =
    \RespBoxMath{<e^t, 2*t, cos(t)>}(t){3}{.0001}{0}{1}*{ProcVec}$
\end{sverbatim}
The first question takes a three component vector of numerical values, the second
takes a three component vector of function values.

\exPDF{jqzvector} These two examples were taken from the demo file \texttt{jqzvector.tex},
which includes both \texttt{quiz} and \texttt{shortquiz} environments, found in the \texttt{dljslib\_examples} folder.

\newtopic Alert box messages: May be redefined.
\begin{sverbatim}
    \newcommand\vectorsErrorMsgi{"I'm looking for a vector.
        You need to use proper vector notation,
        try using angle brackets <....>."}
    \newcommand\vectorsErrorMsgii{"Angle brackets are not balanced.
        Check the expression you typed in."}
    \newcommand\vectorsErrorMsgiii{"Incorrect number of components.
        The answer requires " + aCorrAns.length + " components."}
\end{sverbatim}

\subsubsection{\texttt{setSupport}}   \label{setSupport}

We introduce three JavaScript (response) functions
for handling a set of answers or a comma delimited list of answers, these
are \texttt{ProcRespSetNum}, \texttt{ProcRespSetSym}, and
\texttt{ProcRespListFormula}. The first function is for numerical
answers, the second for simple symbolic answers. These JavaScript response
functions are passed to \cs{RespBoxMath} as the $\text{10}^{\text{th}}$ argument.

\exPDF{set_test} The demo file for these three
response functions is \texttt{set\_test.tex}; the examples below
were taken from that file, found in the \texttt{dljslib\_examples}
folder.

The functions \texttt{ProcRespSetNum}, \texttt{ProcRespSetSym} can
handle (math fill-in) questions whose answers are a \textit{set of
numbers (symbols)} or a comma delimited \emph{list of numbers
(symbols)}.

\paragraph*{ProcRespSetNum:} The function can handle (math fill-in)
questions whose answers are a \textit{set of numbers} or a
comma delimited \textit{list of numbers}.

\newtopic This example takes a comma-delimited
list as its expected response.  After the student enters the
response, the answer is formatted as a set, with enclosing braces,
like so \verb!{ ... }!. The student does not enter the braces, the
JavaScript does after the comma-delimited list is entered and
committed.
\begin{sverbatim}
 $(x+1)^2 (x+3) = 0$, $S = \RespBoxMath[\AddAAFormat{\formatAsSet}
    \rectW{.75in}\textSize{0}]{-1,-3}{1}{.0001}{[0,1]}*{ProcRespSetNum}$
\end{sverbatim}
\noindent The formatting of answer is accomplished by passing
\verb!\AddAAFormat{\formatAsSet}! as an optional argument of
\cs{RespBoxMath}. The command \cs{formatAsSet} expands to a
JavaScript function that is defined when the \texttt{setSupport}
option is taken. The little-known key \cs{AddAAFormat} is
defined in \textsf{exerquiz} and is specifically designed for inserting
special formatting into a math fill-in response. The use of the set
formatting is optional.

\newtopic This next example again uses \texttt{ProcRespSetNum}. The
equation has repeated roots. To require the student to enter the
repeated roots, just list them as part of the answer argument.
%Here,the set notation is not used (since it would be incorrect
%\verb~{-1, 1/2, 2, 2, 2} = {-1, 1/2, 2}~).
\begin{sverbatim}
 $(x+1)(2x-1)(x-2)^3 = 0$, $x = \RespBoxMath[\rectW{.75in}\textSize{0}]
    {-1, 1/2, 2, 2, 2}{1}{.0001}{[0,1]}*{ProcRespSetNum}$
\end{sverbatim}

\paragraph*{ProcRespSetSym:} The function can handle (math fill-in)
questions whose answers are a \textit{set of symbols}.

This function can handle simple symbolic answers. The variable list should
be the ``universal set'' of the problem.
\begin{sverbatim}
\def\U{a,b,c,d,e,f,g} % define a universal set
$A \cap B = \RespBoxMath[\AddAAFormat{\formatAsSet}
  \rectW{.75in}\textSize{0}]{c,d}(\U){1}{.0001}{[0,1]}*{ProcRespSetSym}$
\end{sverbatim}

\paragraph*{ProcRespListFormula:} This function can handle a (math fill-in)
question whose answer is a comma-delimited list of expressions. Similar to
\texttt{ProcVec}, but the angle brackets are not used to specify the vector.

\newtopic In this example, taken from \texttt{set\_test.tex}, a
vector response is required, but without the enclosing angle
brackets.
\begin{sverbatim}
$\vec f'(t) = \RespBoxMath{e^t, 2*t,cos(t)}(t){3}
    {.0001}{0}{1}*{ProcRespListFormula}$
\end{sverbatim}
You can also define your own formatting function, to format
the input.

\subsubsection{\texttt{unordered}}    \label{unordered}

This option defines the JavaScript (response) function
\texttt{ProcRespSetFormula} that will grade an \emph{unordered list} of
formulas, such as \texttt{x}, \verb!x^2!, \verb!x^3!.  For example,
\begin{sverbatim}
Some question requiring an unordered list of responses.
\RespBoxMath{x,x^2,x^3}{10}{.0001}{[1,2]}*{ProcRespSetFormula}
\end{sverbatim}
A correct answer is a list of the monomials \verb!x!,\verb!x^2!,\verb!x^3! in any order.
As usual, this JavaScript response function is passed to
\cs{RespBoxMath} as the $\text{10}^{\text{th}}$ argument.

\exPDF{dljslib_apw}The demo file that uses the \texttt{unordered} option is
\texttt{dljslib\_apw.tex}, which can be found in the \texttt{dljslib\_examples} folder

\subsubsection{\texttt{factors}}      \label{factors}

The factors option defines \texttt{ProcRespFactors}, passed to
\cs{RespBoxMath} as the $\text{10}^{\text{th}}$ argument. The
response function \texttt{ProcRespFactors} is used for grading
polynomial factoring questions. For example, if a polynomial factors
as \verb!-5x^2(x-4)(x+2)!, then the answer is only unique up to the
order of factors and a change of sign on an even number of factors.
The student could respond with \verb!-5x^2(x+2)(x-4)! or
\verb!5x^2(-x+4)(x+2)! or \verb!-5(x-4)(x+2)x^2!. The function
should grade all of these variations correctly; however, it will not
grade \verb!-(x-4)(x+2)5x^2! as correct. The leading coefficient, if
there is one, must be placed at the beginning, which conforms to our
usual conventions.

\begin{sverbatim}
Completely factor the polynomial $x^3-4x^2 + 2x - 8$.
\RespBoxMath{(x^2+2)(x-4)}{10}{0.0001}{[0,1]}*{ProcRespFactors}
\end{sverbatim}
Note that \texttt{ProcRespFactors} is passed to
\cs{RespBoxMath} as the $\text{10}^{\text{th}}$ argument.

\exPDF{dljslib_apw}The demo file that uses the \texttt{factors} option is
\texttt{dljslib\_apw.tex}, which can be found in the \texttt{dljslib\_examples} folder

\subsubsection{\texttt{point}}        \label{point}

The response function \texttt{ProcPoint} is an almost exact copy of your
\texttt{ProcVec}, but uses parentheses instead of angle brackets as
delimiters. Used with questions that have a point $(x,y)$ as the
answer.
\begin{sverbatim}
Calculate the derivative of $\vec f(t) = (e^t,e^(t^2))$.
\RespBoxMath{(e^t,2te^t)}(t){10}{0.0001}{[0,1]}*{ProcPoint}
\end{sverbatim}
\texttt{ProcPoint} is passed to
\cs{RespBoxMath} as the $\text{10}^{\text{th}}$ argument.


\exPDF{dljslib_apw}The demo file that uses the \texttt{point} option is
\texttt{dljslib\_apw.tex}.

\newtopic Alert box messages: May be redefined.
\begin{sverbatim}
    \newcommand\pointErrorMsgi{"I'm looking for a point.
        You need to use proper point notation."}
    \newcommand\pointErrorMsgii{"Parentheses are not balanced."}
    \newcommand\pointErrorMsgiii{"Incorrect number of components.
        The answer requires " + aCorrAns.length+" components."}
\end{sverbatim}

\subsubsection{\texttt{intervals}}    \label{intervals}

\texttt{ProcRespIntervals} is for grading questions that have an
interval or union of intervals as the answer. Example of answers
supported by this function are listed below:
\begin{verbatim}
    (2,3)
    [2,3]
    (-inf,4)
    (-inf,4]
    (6,inf)
    [6,inf)
    (2,3)U[8,9)
    (-inf,6)U(9,inf)
    (-inf,6]U[7,8)U(9,inf)
\end{verbatim}
Here \texttt{inf} for infinity and the capital letter \texttt{U} for
the union symbol. Note also that intervals in a union can appear
in any order. The

The \texttt{ProcRespIntervals} function has been modified by Robert Marik
to allow intervals with end points \texttt{ln(2)}, \texttt{exp(4)} or \texttt{sqrt(3)}.
Thus, intervals of the form
\begin{sverbatim}
    (ln(2),ln(3)]U(exp(10),inf)
\end{sverbatim}
are recognized by this procedure.

\begin{verbatim}
\RespBoxMath{(-inf,4]U[5,inf)}{10}
    {.0001}{[0,1]x[0,1]x[0,1]x[0,1]}*{ProcRespIntervals}
\end{verbatim}
\Important Problems with unbalanced parentheses while
authoring and building the PDF file. The code below
\begin{sverbatim}
    \intervalbox{(-inf,4)U[5,inf)}
\end{sverbatim}
the right parenthesis of the second interval does not have a
matching left parenthesis. This causes problems with Acrobat
Distiller. As a work around, escape the odd parenthesis (using
\cs{eqbs}), like so \verb!\intervalbox{(-inf,4)U[5,inf\eqbs)}!,
 where \cs{eqbs} is defined in the \textsf{insdljs} package.

\exPDF{dljslib_apw}The demo file that uses the \texttt{factors} option is
\texttt{dljslib\_apw.tex}.

\subsubsection{\texttt{complex}}\label{complex}

This option extends the ability of \textsf{exerquiz} to process questions
that are expecting a complex number as an answer.

\paragraph*{ProcRespComplex:} The \texttt{complex} option defines
the function \texttt{ProcRespComplex}, which is specified in the usual way, as seen below.
\begin{sverbatim}
$(3+3i)+(4+5i) = \RespBoxMath{7+8i}{3}{0.0001}{[0,1]}*{ProcRespComplex}$
\end{sverbatim}
The \texttt{complex} option also defines the \texttt{cis} function,
the definition of which is
$\operatorname{cis}(x)=\cos(x)+\imath\sin(x)$. When \texttt{complex}
is not loaded, this function is undefined, and will be caught by the
parser of exerquiz as a syntax error. If \texttt{complex} is loaded
and the user enters the \texttt{cis} function in a non-complex
problem, the user's answer is not flagged as a syntax error and will
most probably be scored as \emph{wrong}. The exerquiz implementation of the \texttt{cis}
function, does not, at this time, support powers of the \texttt{cis} function, the parser
attempts to find expressions of the form \texttt{cis\string^}, flag it, and emit
an alert message, see below. This implementation does not detect powers of the form
\texttt{(cis(x))\string^2}, however.

The next two functions are due to Bruce Wagner, with modifications by D. P. Story.

\paragraph*{ProcRespListComplex:} This response function evaluates an \emph{ordered} list
of comma-delimited complex numbers. For example,
\begin{sverbatim}
If $z=4(\cos x+i\sin x)$, compute $z^2$ and $z^3$, in that order.
\RespBoxMath{16*cis(2x),64*cis(3x)}{4}{0.0001}{[0,1]}*{ProcRespListComplex}
\end{sverbatim}


\paragraph*{ProcRespSetComplex:} This response function evaluates an \emph{unordered} list
of comma-delimited complex numbers. For example,
\begin{sverbatim}
 Find all real and complex solutions of the equation $x^2=-9$.
 Express your answer(s) in rectangular form $a+bi$.
 \RespBoxMath{3i,-3i}{4}{0.0001}{[0,1]}*{ProcRespSetComplex}
\end{sverbatim}


\exPDF{dljslib_complex}The demo file that uses the \texttt{complex} option is
\texttt{dljslib\_complex.tex}.

\newtopic Alert box messages: May be redefined.
\begin{sverbatim}
\newcommand{\complexPowerAlertMsg}{%
    "Powers of i (for example, i^2, i^3) are not supported,
    replace powers of i with their complex equivalents."
}
\newcommand{\complexCisAlertMsg}{%
    "The cis function does not support exponents. Write,
    for example, cis^3(x) as cis(3*x), instead."}
\newcommand{\alertNotComplexMsg}{%
    "The expression is not in the form of a complex
    number, a+bi"}
\def\emptyCompComplexMsg(#1){%
    "You entered nothing for the component "
     +(#1+1)+" of your answer. Please enter a complex number."
}
\end{sverbatim}
The latter definition takes a delimited argument, necessary since it executes
within a verbatim environment where braces \texttt{\{} and \texttt{\}} are
not grouping delimiters.

\subsection{Compare Functions} \label{compareFuncs}

Compare functions are used to compare the author's answer with the
student's answer. Based on the comparison, the student's answer is
judged correct, or not. The \textsf{exerquiz} package has two
built-in compare functions: \texttt{diffCompare} (the default) and
\texttt{reldiffCompare}. The former makes the comparison based on absolute differences
$$
    \texttt{diffCompare} = |\,\text{Author Ans} - \text{Student Ans}\,|
$$
while the latter uses absolute relative differences to make the comparison
$$
    \texttt{reldiffCompare} = \frac{|\,\text{Author Ans} - \text{Student Ans}\,|}
                                    {|\text{Author Ans}|}
$$
Comparison functions are passed through the $\text{9}^{\text{th}}$ argument of \cs{RespBoxMath},
if no such argument is passed, \texttt{diffCompare} is used.

\subsubsection{\texttt{indefIntegral}}\label{indefIntegral}

The answer to an indefinite integral is non-unique; however all
answers differ by a constant. This compare function is used with
indefinite integrals. Note the name of the function is
\texttt{indef\-Com\-pare}, this is the name you use to call the
function. To use this compare function, insert its name in optional
$\text{9}^{\text{th}}$ argument,
\begin{equation*}
  \text{\respBoxPara}
\end{equation*}
that's the argument that follows the interval specification.


An example of usage of this function can be found in the
sample file \texttt{jslib\_ex.tex} that comes with the Acro\TeX{}
Bundle; from that file we have:
\begin{verbatim}
$\displaystyle\int\sin(x)\,dx =
    \RespBoxMath{-cos(x)}{4}{.0001}{[0,1]}[indefCompare]$
\end{verbatim}

\exPDF{jslib_ex} See the demo file
\texttt{jslib\_ex.tex} for an example of indefinite integration
comparison.  Numerous other files use indefinite integration as well.

\subsection{Filter User's Responses} \label{filterUserResp}

Filters are JavaScript functions that process the user input before being passed to the response function.
Filters are passed to \cs{RespBoxMath} through the $\text{9}^{\text{th}}$ argument,
\begin{verbatim}
$\displaystyle\int\sin(x)\,dx =
    \RespBoxMath{-cos(x)}{4}{.0001}{[0,1]}[indefCompare]$
\end{verbatim}
This $\text{9}^{\text{th}}$ argument can take on two formats: (1) the name of the compare function,
as illustrated in `\nameref{indefIntegral}' above; (2) a JavaScript object. Use a JavaScript object
as the $\text{9}^{\text{th}}$ argument to specify a filter.

The JavaScript object that can be passed in the $\text{9}^{\text{th}}$ argument has two properties
\texttt{comp} and \texttt{priorParse}. The object has the form
\begin{sverbatim}
    { comp: <compare>, priorParse: <JS_func> | <array_JS_funcs> }
\end{sverbatim}
The first property is used to specify a compare function, the second
property specifies a filter or filters. The value of
\texttt{priorParse} may be a JavaScript function to filter the
user's response, or an array of JavaScript functions to filter the
user input. (The array must be specified using the \Com{Array}
command, illustrated below, see the paragraph
`\nameref{NoProducts}', on page~\pageref*{NoProducts}.)

\subsubsection{\texttt{nodec}}        \label{nodec}

The \texttt{nodec} option defines the JavaScript function
\texttt{nodec} that attempts to prohibit the use of decimal numbers
in the response. The function simply search the user response for
any occurrence of the decimal point. (The German language uses
the comma as a decimal point, you can set the decimal point to some
other symbol that \texttt{nodec} searches for through the command
\Com{setdecimalpoint}, the default definition is
\verb!\setdecimalpoint{.}!.)

\begin{sverbatim}
$ 0.75 = \RespBoxMath{3/4}{1}{.0001}{[0,1]}[{priorParse: nodec }] $
\end{sverbatim}

\newtopic Alert box message: May be redefined.
\begin{sverbatim}
\newcommand\nodecAlertMsg{%
    "A decimal answer is not acceptable here
     Please express your answer using fractions, square roots,
     e, log, etc."}
\end{sverbatim}

\subsubsection{\texttt{noBinFac}}     \label{noBinFac}

The \texttt{noBinFac} option defines the JavaScript filter function
\texttt{noBinFac}, that disallows binomial coefficients and
factorials in math fill-in questions. Binomial and factorial
functions are defined through the \texttt{combinatorics} option,
page~\pageref*{combinatorics}.

\begin{sverbatim}
The number of ways to choose four object from a set of ten is
    \RespBoxMath{C(10,4)}{1}{.0001}{[0,1]}[{priorParse: noBinFac}]
        \CorrAnsButton{C(10,4)}*{EvalCorrAnsButton}
\end{sverbatim}
The \texttt{noBinFac} filter is used, so the student is not allowed
to enter binomial coefficients; however, the document author can use
binomial coefficients, note the use in the answer (first argument)
and in the \cs{CorrAnsButton}. The special syntax used for
\cs{CorrAnsButton}, \Nameref{inclsoln}.

\exPDF{dljs_comb} Illustrations of this filter can be
found in the \texttt{dljs\_comb.tex}.

\newtopic Alert box messages: Mayb be redefined.
\begin{sverbatim}
    \newcommand\noBinFactBinCoeffAlertMsg{%
        "You may not use this notation here.
         Please evaluate the binomial coefficient.
         You may present your answer as a product rather
         than calculating a very large number."}
    \newcommand\noBinFactPermAlertMsg{%
        "You may not use this notation here.
         Please evaluate the permutation.
         You may present your answer as a product rather
         than calculating a very large number."}
    \newcommand\noBinFactFactAlertMsg{%
        "You may not use this notation here.
         Please evaluate the factorial.
         You may present your answer as a product rather
         than calculating a very large number."}
\end{sverbatim}

\subsubsection{\texttt{limitArith}}   \label{limitArith}

The \texttt{limitArith} option defines a number of filter functions:
\begin{verbatim}
DecimalsOnly, NoProducts, NoDivision, NoAddOrSub,
NoArithAllowed,NoTrigAllowed, NoTrigLogAllowed
\end{verbatim}

\exPDF{limarth} Examples of these filtering functions
can be found in the demo file \texttt{limarth.tex}.

\paragraph*{DecimalsOnly} The \texttt{DecimalsOnly} is a function
that takes only numbers, either integer or decimal. Usage
\begin{sverbatim}
 $ 2.3 + 4.5 = \RespBoxMath{6.8}{1}{.0001}{[0,1]}
    [{priorParse: DecimalsOnly }]$
\end{sverbatim}

\paragraph*{NoProducts}\label{NoProducts} The function
\texttt{NoProducts} requires that \texttt{ImplMulti} is taken. An
example of usage is
\begin{sverbatim}
$ 3/4 = \RespBoxMath{.75}{1}{.0001}{[0,1]}
    [{priorParse: \Array(NoDivision, NoProducts) }]$
\end{sverbatim}
In the above example, we first call \texttt{NoDivision} then \texttt{NoProducts}. Several
filtering functions can be executed by putting them into an array, using the \cs{Array}
command.\footnote{The \cs{Array} command  was originally named \cs{array}, it has been renamed
to avoid clashing with the \texttt{amsmath} package.}

\paragraph*{NoAddOrSub} This function, as the name implies, denies the use of both addition and subtraction.
The algorithm I use here is that if a plus or minus follows anything but a left parenthesis,
we assume it is a binary operation of addition or subtraction, respectively.
\begin{sverbatim}
$ 5.1 - 3.2 =
    \RespBoxMath{1.9}{1}{.0001}{[0,1]}[{priorParse: NoAddOrSub}]$
\end{sverbatim}


\paragraph*{NoArithAllowed} Deny user the use of all arithmetic functions, including exponents.
\begin{sverbatim}
$ 6/5 = \RespBoxMath{1.25}{1}{.0001}{[0,1]}[{priorParse: NoArithAllowed}]
\end{sverbatim}

\paragraph*{NoTrigAllowed} Deny user the use all trig functions, as
well as the constants \texttt{PI} and \texttt{pi}, which are aliases
for $\pi$. Example:
\begin{sverbatim}
$ \sin(\pi/4) = \RespBoxMath{sqrt(2)/2}{1}{.0001}{[0,1]}
    [{priorParse: NoTrigAllowed }]$
\end{sverbatim}


\paragraph*{NoTrigLogAllowed} Deny the use of trig and log functions.

\begin{sverbatim}
$ \ln(e^2)\sin(\pi/4) = \RespBoxMath{sqrt(2)}{1}{.0001}{[0,1]}
    [{priorParse: NoTrigLogAllowed }]\CorrAnsButton{sqrt(2)}$
\end{sverbatim}


\newtopic Alert box messages: Mayb be redefined.
\begin{sverbatim}
    \newcommand\DecimalsOnlyErrorMsg{"Enter only an integer,
        e.g., 17, or a decimal number, e.g. 12.4.
        Using arithmetic operations or built in function is
        not acceptable for this problem."}
    \newcommand\NoProductsErrorMsg{"Multiplication is not allowed
        for this problem."}
    \newcommand\NoDivisionErrorMsg{"Division is not allowed
        for this problem."}
    \newcommand\NoAddOrSubErrorMsg{"Neither addition nor subtraction
        is not allowed for this problem."}
    \newcommand\NoExpAllowedErrorMsg{"The use of exponents is not
        allowed for this problem."}
    \newcommand\NoTrigAllowedErrorMsg{"The use of trig functions
        in this problem is not allowed."}
    \newcommand\NoTrigLogAllowedErrorMsg{"The use of trig and log
        functions is not allowed in this problem."}
\end{sverbatim}

\subsection{Extending Numerical Functions}  \label{extNumerFuncs}

These JavaScript functions extend the collection of built-in
functions that perform numerical calculations.

\subsubsection{\texttt{combinatorics}}\label{combinatorics}

This option defines three JavaScript functions: a binomial
coefficient function \texttt{ch(n,r)}; a number of permutations
function \texttt{perm(n,r)}, and a factorial function
\texttt{fact(n)}.

\paragraph*{User Accessible Functions} The user should use the names \texttt{C}, \texttt{P} and \texttt{fact}
to access the functions are \texttt{ch}, \texttt{perm} and \texttt{fact}, respectively:
\begin{itemize}
    \item $ \texttt{C(n,r)} = \texttt{ch(n,r)} = \dbinom{n}{r} = \dfrac{n\,!}{r\,!(n-r)\,!}$
    \item $ \texttt{P(n,r)} = \texttt{perm(n,r)} = n\cdot(n-1)\cdots(n-r)$
    \item $ \texttt{fact(n)} = n\cdot(n-1)\cdots3\cdot2\cdot1$
\end{itemize}
\textbf{Note: }If you place \verb!\replaceExclPt{true}! in the preamble, the user is allowed to enter factorials using the
traditional exclamation notation: for example, \texttt{17!} is the same as fact(17), and \texttt{(17)!}, \texttt{[17]!} or \verb|{17}!|
are all interpreted as \texttt{fact(17)}.

\exPDF{dljs_comb} An illustrations of the \texttt{combinatorics} option can be
found in the \texttt{dljs\_comb.tex}.


%------------------------ List of Options ------------------------------------
\newpage
\section{List of Options}\label{s:ListofOpts}

%\begin{tabular}{|>{\bfseries\color{red}}l|>{\PBS\raggedright}p{.5\linewidth}|}\hline
%  \multicolumn1{|>{\bfseries\columncolor{light-blue}}c|}{Component} &
%  \multicolumn1{>{\bfseries\columncolor{light-blue}}c|}{Brief Description}\\  \hline \rowcolor[gray]{.9}

\def\currentPackage{Web}

\begin{center}
\setlength{\extrarowheight}{1mm}
\setlength{\tabcolsep}{2mm}
\begin{longtable}{|>{\ttfamily}l|>{\PBS\raggedright}p{.5\linewidth}|}
\hline
\multicolumn2{|c|}{Options of the Web/Exerquiz Packages} \\
\hline
\endfirsthead
\hline
\multicolumn2{|c|}{Options of the Web/Exerquiz Packages (cont.)} \\
\hline
\endhead
\hline
\endfoot
\multicolumn2{|c|}{Options of the Web Package} \\
\hline
\hyperref[sss:driver]{dvipsone}         & dvi-to-ps driver by {\Y&Y}, Inc. \\
\hyperref[sss:driver]{dvips}            & dvi-to-ps driver \\
\hyperref[sss:driver]{pdftex}           & tex-to-pdf application \\
\hyperref[sss:driver]{dvipdfm}          & dvi-to-pdf application \\
\hyperref[sss:driver]{textures}         & the Textures System for Mac \\
\hyperref[sss:designoptions]
    {\parbox[t]{.75in+4pt}{designi,\newline
    designii,\newline designiii,\newline
    designiv,\newline designv}}         & these set screen design parameters \\
\hyperref[ss:templates]{usetemplates}   & this option activates mechanism for
                                            creating colored backgrounds and overlays\\
\hyperref[ss:templates]{leftpanel}      & creates a left navigation panel\\
\hyperref[ss:templates]{rightpanel}     & creates a right navigation panel\\
\hyperref[sss:navibar]{navibar}         & inserts a menu bar at the bottom or each
           page\\
\hyperref[sss:latextoc]{latextoc}       & displays the standard toc \\
%\hyperref[sss:nodir]{nodirectory}       & eliminates the directory listing on the title page\\
\hyperref[sss:nodir]{usedirectory}      & causes the title page directory to appear on the title page\\
\hyperref[sss:webforpaper]{forpaper}    & this turns off color, and does not put solutions on
           separate pages. \\
\hyperref[forcolorpaper]{forcolorpaper} & Same as \texttt{forpaper}, but
                                            does not turn off color, useful for color printers.\\
\hyperref[sss:weblatexlayout]%
    {latexlayout}                       & \texttt{web} uses page
                                            layout for \texttt{article} class.
                                            For use with \texttt{forpaper}.\\
\hyperref[sss:webtight]{tight}          & redefines list environment parameters so
                                            lists don't take up so much space \\
\hyperref[webpro]{pro}                  & causes web to input enhanced control
                                            over various web elements\\
\hyperref[nobullets]{nobullets}         & forces the use of numbers for subsubsections,  also executes the \texttt{latextoc} option\\
\hyperref[usesf]{usesf}                 & switches to sans serif as the default font, good for presentations\\
\hyperref[web:unicode]{unicode}         & Passes the \texttt{unicode} option to \textsf{hyperref}.\\
\hyperref[web:useui]{useui}             & Passes the \texttt{useui} option to \textsf{eforms}.\\
\hyperref[web:langopts]{dutch}          & Dutch for web, passed to exerquiz. (Thanks to Henny Wilbrink)\\
\hyperref[web:langopts]{french}         & French for web, passed to exerquiz. (Thanks to Jean-Michel Sarlat)\\
\hyperref[web:langopts]{german}         & German for web, passed to exerquiz. (Thanks to Michael Wiedmann)\\
\hyperref[web:langopts]{italian}        & Italian for web, passed to exerquiz. (Thanks to PierLuigi Zezza)\\
\hyperref[web:langopts]{norsk}          & Norwegian for web, passed to exerquiz. (Thanks to Hans Fredrik Nordhaug)\\
\hyperref[web:langopts]{russian}        & Russian for web, passed to exerquiz. (Thanks to Sergei V. Znamenskii)\\
\hyperref[web:langopts]{spanish}        & Spanish for web, passed to exerquiz. (Thanks to Pedro Luis Luque)\\
\hyperref[web:langopts]{polish}         & Polish for web, passed to exerquiz. (Thanks to Jerzy Mycielski)\\
\hyperref[web:langopts]{finnish}        & Finnish for web, passed to exerquiz. (Thanks to P\"aivi Porras)\\
\hyperref[web:langopts]{catalan}        & Catalan for web, passed to exerquiz. (Thanks to Ramon Ballester)\\
\hyperref[web:langopts]{czech}          & Czech for web, passed to exerquiz. (Thanks to Robert Marik)\\
\hyperref[web:langopts]{brazil}         & Brazilian Portuguese for web, passed to exerquiz. (Thanks to Koichi Sameshima)\\
\hyperref[web:langopts]{turkish}        & Turkish for web, passed to exerquiz. (Thanks to Mahmut Ko\c{c}ak)\\
\hline
\multicolumn2{|c|}{Options of the Exerquiz Package\gdef\currentPackage{Exerquiz}} \\ \hline
\hyperref[ss:pdftexerquiz]{pdftex}      & tex-to-pdf application \\
\hyperref[ss:exbasicusage]{dvipdfm}     & dvi-to-pdf application \\
\hyperref[sss:nosolutions]{nosolutions} & removes the solutions to the exercises \\
\hyperref[sss:noquizsolutions]%
    {noquizsolutions}                   & removes the solutions to the quizzes \\
\hyperref[sss:nohideopt]%
    {nohiddensolutions}                 & overrides the `\texttt{h}' (hidden)
                                            option for the exercises.\\
\hyperref[sss:noHideopt]%
    {noHiddensolutions}                 & overrides the `\texttt{h}' and `\texttt{H}'
                                            (hidden) options for the exercises.\\
\hyperref[sss:nocorrections]%
    {nocorrections}                     & removes the ability to correct the quizzes\\
\hyperref[sss:solutionsafter]%
    {solutionsafter}                    & solutions to exercises are typeset just after the question\\
\hyperref[sss:exquizforpaper]{forpaper} & same function as in \textsf{web}.
                                          Needed when \textsf{exerquiz} is not
                                          used with \textsf{web}\\
\hyperref[sss:exquizforpaper]%
    {forcolorpaper}                     & same function as in \textsf{web}.
                                          Needed when \textsf{exerquiz} is not
                                          used with \textsf{web}\\
\hyperref[sss:preview]{preview}         & shows the outline of all form fields
                                            in the dvi previewer \\
\hyperref[sss:nodljs]{nodljs}           & turns off the insertion of DLJS \\
%\hyperref[sss:acrobativ]{acrobativ}    & equivalent to \texttt{nodljs}\\
\hyperref[sss:noquizzes]{exercisesonly} & if document has only exercises, no doc level JS needed\\
\hyperref[sss:debug]{debug}             & this option is passed on to the \texttt{insdljs} package\\
\hyperref[sss:proofing]{proofing}       & mark the correct answers for
                                            \texttt{shortquiz} \&
                                            \texttt{quiz} for
                                             proof reading. \\
\hyperref[showgraylettersOpt]%
    {showgrayletters}                   & Show gray letters under check boxes\\
\hyperref[ss:allowrandomize]%
    {allowrandomize}                    & Allow the randomization of choices in multiple choie questions\\
\hyperref[eq:unicode]{unicode}          & Passes the \texttt{unicode} option to \textsf{hyperref}.\\
\hyperref[eq:useui]{useui}              & Passes the \texttt{useui} option to \textsf{eforms}.\\
\hyperref[eq:usesumrytbls]{usesumrytbls}& Inputs the code used to create quiz summary tables,
                                         \hyperref[s:sumrytbl]{Section~\ref*{s:sumrytbl}}, page~\pageref*{s:sumrytbl}.\\
\hyperref[sss:exerquizlangopts]{dutch}  & JavaScript messages in Dutch (Thanks to Henny Wilbrink)\\
\hyperref[sss:exerquizlangopts]{french} & JavaScript messages in French (Thanks to Jean-Michel Sarlat)\\
\hyperref[sss:exerquizlangopts]{german} & JavaScript messages in German (Thanks to Michael Wiedmann)\\
\hyperref[sss:exerquizlangopts]{italian}& JavaScript messages in Italian (Thanks to PierLuigi Zezza)\\
\hyperref[sss:exerquizlangopts]{norsk}  & JavaScript messages in Noregian (Thanks to Hans Fredrik Nordhaug)\\
\hyperref[sss:exerquizlangopts]{russian}& JavaScript messages in Russian (Thanks to Sergei V. Znamenskii)\\
\hyperref[sss:exerquizlangopts]{spanish}& JavaScript messages in Spanish (Thanks to Pedro Luis Luque)\\
\hyperref[sss:exerquizlangopts]{polish} & JavaScript messages in Polish (Thanks to Jerzy Mycielski)\\
\hyperref[sss:exerquizlangopts]{finnish}& JavaScript messages in Finnish. (Thanks to P\"aivi Porras)\\
\hyperref[sss:exerquizlangopts]{catalan}& JavaScript messages in Catalan. (Thanks to Ramon Ballester)\\
\hyperref[sss:exerquizlangopts]{czech}  & JavaScript messages in Czech. (Thanks to Robert Marik)\\
\hyperref[sss:exerquizlangopts]{brazil} & JavaScript messages in Brazilian Portuguese. (Thanks to Koichi Sameshima)\\
\hyperref[sss:exerquizlangopts]{turkish}& JavaScript messages in Turkish. (Thanks to Mahmut Ko\c{c}ak)\\
\hline
\end{longtable}
\end{center}

\newpage
\markright{References}

\begin{thebibliography}{[1]}\label{references}
\addcontentsline{toc}{section}{\protect\numberline{}References}

\bibitem{tech:AcroJS}\hypertarget{references}{}%
   Acrobat JavaScript Scripting Refernce, Technical Note \#5431, Version~6.0.,
   Adobe Systems, Inc., 200\\ \url{http://partners.adobe.com/asn/acrobat/docs.jsp}

\bibitem{book:Lamport}
   Leslie Lamport, \textit{\LaTeX: A Document Preparation}
   System (2nd ed.), Addison-Wesley Publishing Company,
   1994, ISBN 0-201-52983-1.

\bibitem{book:Goossens1}
   Michel Goossens, Frank Mittelbach and Alexander
   Samarin, \textit{The \LaTeX{} Companion}, Addison-Wesley
   Publishing Company, 1994, ISBN 0-201-54199-8.

\bibitem{book:Goossens2}
   Michel Goossens, Sebastian Rahtz, and Frank
   Mittelbach, \textit{The \LaTeX{} Graphics Companion},
   Addison-Wesley Publishing Company, 1997, ISBN
   0-201-85469-4.

\bibitem{book:Goossens3} Michel Goossens, and Rahtz, Sebastian,
   with Gurari, Eitan,  Moore, Ross, and Sutor, Robert,
   \textit{The \LaTeX\ Web Companion},
   Addison Wesley Longman, Reading, Massachusetts, USA, 1999.
   ISBN:  0-201-43311-7.

\bibitem{book:Kopka}
   Helmut Kopka and Patrick W. Daly, \textit{A Guide to
  \LaTeX2e} (2nd ed.), Addison-Wesley Publishing Company,
  1995, ISBN 0-201-43777-X.

\bibitem{book:Knuth}
  Donald E. Knuth, \textit{The \TeX book}, Addison-Wesley
  Publishing Company, 1987, ISBN 0-201-13448-9.

\bibitem{book:Merz}
  Thomas Merz, \textit{Web Publishing with Acrobat/PDF},
  Springer-Verlag Berlin Heidelberg New York, 1998,
 ISBN 3-540-63762-1.

\bibitem{article:Story:Pdfmarks}
 D.P. Story, \textit{Pdfmarks: Links and Forms}, AcroTeX Web
 Site, 1998,\\\url{www.math.uakron.edu/~dpstory/acrotex.html}

\bibitem{article:Story:LaTeX}
 D.P. Story, \textit{Using \LaTeX\ to Create Quality PDF Documents for the
 WWW}, Acro\!\TeX\ Web Site, 1998,\\
 \url{www.math.uakron.edu/~dpstory/acrotex.html}

\end{thebibliography}


% \printindex
\end{document}

\newpage

\makeatletter
\renewcommand{\section}{%
   \renewcommand{\@seccntformat}[1]{}%
   \@startsection{section}{1}{\z@}%
   {-3ex\@plus -1ex \@minus-.2ex}%
   {1ex\@plus .2ex}% 6pt
   {\noindent\normalfont\normalsize\bfseries\color{aeb}}}%
\renewcommand\subsection{%
   \renewcommand{\@seccntformat}[1]{\textbullet\enspace}%
   \@startsection{subsection}{2}{\z@}%
   {-2.5ex\@plus -1ex \@minus -.2ex}%
   {1ex \@plus .2ex}%
   {\noindent\normalfont\normalsize\bfseries\color{aeb}}}
\renewcommand\subsubsection{%
   \renewcommand{\@seccntformat}[1]{\color{aeb}\large$\bullet$\enspace}
   \@startsection{subsubsection}{3}{\z@}%
     {-2ex\@plus -1ex \@minus -.2ex}%
     {.5ex \@plus .2ex}%
     {\noindent\normalfont\normalsize\bfseries}}
\renewcommand*\thesection{}
\renewcommand*\thesubsection{\textbullet}
%\renewcommand*{\thesubsubsection}{\textbullet}
\renewcommand*\l@subsection{\@dottedtocline{2}{2.8em}{1.2em}}
%\l@subsection{\@dottedtocline{3}{3.8em}{1.2em}}
\makeatother

\addtocontents{toc}{\string\renewcommand*\string\l@subsection{\string\@dottedtocline{2}{2.8em}{1.2em}}}

\section{What's New}

The following is a brief enumeration of some of the major new
features of the \textsf{web} and \textsf{exerquiz} packages.

\subsection{Web}

\begin{questions}

    \item (Version 4.1) Added the option \texttt{useui}, which is passed
    \textsf{eforms}. If \textsf{eforms} is subsequently loaded,
    \textsf{eforms} loads the xkeyval package, and key-value pairs are
    defined for use in the optional argument of form and link commands.
    See the \href{eformman.pdf}{eforms manual} for details.

    \item (Version 4.1) Added a unicode option. This option is passed to hyperref, if eforms
    is subsequently loaded, eforms (and exerquiz) will accept {\LaTeX} markup in the optional
    argument of the form fields. See the \href{eformman.pdf}{eforms manual} for details.

    \item (Version 4.1) Added the \texttt{turkish} language option. I extend my appreciation
    to Mahmut Ko\c{c}ak for this translation.

    \item (Version 4.0g) Added the command \cs{useFullWidthForPaper}, see the discussion
    in \Nameref{paperoptions}.
    \item (Version 4.0f) Added \cs{setScreensizeFromGraphic}, see
    \hyperref[setScreensizeFromGraphic]{Section~\ref*{setScreensizeFromGraphic}},
    page~\pageref*{setScreensizeFromGraphic}.
    \item (Version 4.0d) \cs{makeinlinetitle}, see \Nameref{inlinetitle}.
    \item (Version 4.0c) Added \cs{aboveTopTitleSkip}, the amount of skip above the first
    line on the title page, so the whole title can be move downwards, if
    desired. See the section \Nameref{top}.
    \item (Version 4.0) The pro option with many user-convenient features,
    see \Nameref{webpro}.
    \item (Version 4.0) New \Web{} options of \texttt{usesf},
    \hyperref[usesf]{page~\pageref*{usesf}}  and \texttt{center\-title\-page},
    \hyperref[centertitle]{page~\pageref*{centertitle}}.

\item[] \rule[2pt]{1in}{.4pt} Older New Stuff \rule[2pt]{1in}{.4pt}

    \item (Version 3.3) Made many internal changes for my new package, the
    \textsf{\textcolor{blue}{Acro\!\TeX{} Presentation Bundle}} (\cAPB), yet
    to be released, but see \url{http://www.acrotex.net} for details and demos.
    \item (Version 3.3) Added some commands for running headers
    and footers, these are \cs{lheader}, \cs{cheader}, \cs{rheader},
    \cs{lfooter}, \cs{cfooter}, \cs{rfooter}, \cs{clearHeaders} and \cs{re\-store\-Head\-ers}.
     \item (Version 3.2) Added support for the Dr. Uwe Kern's \textsf{xcolor}
        package. Unrecognized options are passed to the color package,
        in particular, the \textsf{(x)color} option \texttt{dvipsnames} is passed
        from \Web{} to the color package.
    \item[] If available, \textsf{xcolor} is used, otherwise
        \textsf{web} uses the standard \textsf{color} package.
     \item[] There is a \texttt{noxcolor} which should be used as a
        global option, that is passed as an option of the class. Thus,
\begin{verbatim}
    \documentclass[noxcolor]{article}
\end{verbatim}
        will cause \textsf{web} (and other packages such as
        \textsf{PSTricks}) to use the \textsf{color} package, even
        though \textsf{xcolor} may be present.
     \item (Version 3.2) Added a \texttt{nobullets} option. When taken, the bullets
        that traditionally appear for the \cs{subsubection} now appear
        with subsubsection numbers.

     \item (Version 3.2) Added a \texttt{draft} option: When you take this option,
        graphic overlays are not allowed.  This is useful when you rely
        heavily on graphic overlays, but during the development phase,
        don't need to read and re-read your overlays. The defined
        background colors will be used instead. Remove this option to
        build the final version of your document.

     \item (Version 3.2) Defined macros for conveniently
         setting the color of some of the title page elements:
         \cs{titleColor}, \cs{authorColor} and \cs{universityColor}.
        Each takes a single argument, a named color. The defaults are
\begin{verbatim}
    \universityColor{blue}
    \titleColor{black}
    \authorColor{black}
\end{verbatim}

    \item (Version 3.2) Changed optional argument for the template commands
        \cs{template} and \cs{panel\-template}:
\begin{sverbatim}
    \template[<includegraphics\_opts>]{<path\_to\_template>}
    \paneltemplate[<includegraphics\_opts>]{<path\_to\_template>}
\end{sverbatim}
        The optional argument \texttt{[<includegraphics\_opts>]} can be used
        to pass options to the underlying \cs{includegraphics} command. Do
        not use the \texttt{width} and \texttt{height} keys; probably, it is
        safest to use only the Boolean keys.  The Boolean key
        \texttt{hiresbb} is quite useful when stretching a small graphic to
        fit over a larger area. The one required argument is the path to the
        graphic file.

    \item (Version 3.2) Added a number of new commands to improve template management. See
    the \texttt{web.dtx} for documentation.
     \item (Version 3.2) Added language option \texttt{czech}, the translation due
         to Robert Marik.
     \item (Version 3.2) Added language option \texttt{catalan}, the translation due
         to Ramon Ballester.
     \item (Version 3.2) Added language option \texttt{brazil}, the translation due to
         Koichi Sameshima.
     \item Added language \texttt{finnish} Options. Translation due
     to Paivi Porras.
     \item Added a \texttt{\hyperref[forcolorpaper]{forcolorpaper}} option. This is the same as \texttt{forpaper}
     but the color operators are not turned off.
     \item Added three new options for creating color backgrounds
     and overlays, \texttt{usetemplates}. \texttt{rightpanel} and
     \texttt{leftpanel}, see the section \Nameref{ss:templates}.
     The various template options requires the \textsf{eso-pic}
     package by  Rolf Niepraschk and the \textsf{everyshi} package
     by Martin Schr\"oder. The \textsf{web package} supports graphics overlays
     for the \texttt{dvipsone}, \texttt{dvips}, \texttt{pdftex} and
     \texttt{dvipdfm} options.
     \item Added a command to create navigation icons,
     \nameref{ss:newnaviicon}, page~\pageref{ss:newnaviicon}.
     \item Added a \texttt{textures} option, due to Ross Moore.
    \item Made some minor changes to make \textsf{web} work better
    with the \texttt{book} document class.
    \item The introduction of the \textsf{polish} option. This
    makes a total of nine localizations of \textsf{web/exerquiz}:
    french, german, norsk, dutch, spanish, italian, russian,
    dansk and polish.
    \item  Set an alias between \cs{marginsize} and \cs{margins}.
    The latter is used by pdfscreen. This will make it easier for
    people to switch between using web and pdfscreen. The use of
    \cs{marginsize} is now deprecated.
\end{questions}

\normalcolor

\subsection{Exerquiz}


\begin{questions}

    \item (Version 6.3) Added the option \texttt{useui}, which is passed
    \textsf{eforms}. If \textsf{eforms} is subsequently loaded,
    \textsf{eforms} loads the xkeyval package, and key-value pairs are
    defined for use in the optional argument of form and link commands.
    See the \href{eformman.pdf}{eforms manual} for details.

    \item (Version 6.3) Added a unicode option. This option is passed to hyperref, if eforms
    is subsequently loaded, eforms (and exerquiz) will accept {\LaTeX} markup in the optional
    argument of the form fields. See the \href{eformman.pdf}{eforms manual} for details.

    \item (Version 6.3) Added the \texttt{turkish} language option. I extend my appreciation
    to Mahmut Ko\c{c}ak for this translation.

    \item (Version 6.2d) Added \cs{titleQuiz} and \cs{fancyQuizHeaders},
    see \hyperref[titleQuiz]{Section~\ref*{titleQuiz}}, page~\pageref*{titleQuiz}
    and \hyperref[titleQuizSQ]{Section~\ref*{titleQuizSQ}}, page~\pageref*{titleQuizSQ}.
    \item (Version 6.2c) Added a feature that I have used for years to \textsf{exerquiz}:
    see the section titled \Nameref{showcreditmarkup}.
    \item (Version 6.2a) Added a feature for bookmarking exercises and quizzes,
    see \hyperref[bmExQz]{Section~\ref*{bmExQz}} on page~\pageref*{bmExQz}.
    \item (Version 6.2) Added the manswers environment for multiple answer
    questions, that is, multiple choice questions that have several correct
    answers. Multiple selection is permitted. See the \Nameref{manswersQuiz}.
    \item (Version 6.1) Added the ability to randomize the multiple
    choice alternatives, see the two sections \Nameref{ss:allowrandomize} and
    \Nameref{s:random} for details.
    \item (Version 6.05f) For inputting math fill-in responses. \texttt{log} now means common log,
    and as before \texttt{ln} means natural log.  Also,
    re-engineered things so that math fill-in now accepts, and
    correctly interprets, scientific notation (using E-notation)
    $\texttt{<decimal\_number>}\text{E}\texttt{<integer>}$. The
    capital letter E represents ``times ten raised to the power'',
    the lower case e is reserved for use as the natural number.
    \item (Version 6.05d) Added option \texttt{showgrayletters},
    see \Nameref{showgraylettersOpt}.
    \item (Version 6.05e) Added \cs{setMClabelsep} and
        \cs{resetMClabelsep} to adjust the horizontal distance between
        the label and text for multiple choice questions. See the
        discussion in \Nameref{BasicUsage}.

    \item (Version 6.05g)  Added \cs{renameSolnAfterTo} for conveniently
        changing the solution after label, and
        \cs{resetSolnAfterToDefault} for resetting back to the default.
        These two commands are available for use in \textsf{exerquiz} or
        in \textsf{eqexam}. There is a brief discussion of these
        commands in \Nameref{sss:solutionsafter}.

    \item (Version 6.05c) Added the commands \Com{DeclareQuiz}, \Com{floatQuiz},
        \Com{startQuizHere}, \Com{endQuizHere}, and \Com{dockQuiz}. to
        start a quiz in an arbitrary place prior to the beginning of the
        quiz. See \Nameref{floatQuiz}.

\item[] \rule[2pt]{1in}{.4pt} Older New Stuff \rule[2pt]{1in}{.4pt}

\item (Version 6.03) Added the \hyperref[bChoices]{\cs{bChoices}}/\cs{eChoices} command
    pair, this makes it easy to lay out multiple choice alternatives.

\item (Version 6.0) Improved the \texttt{mathGrp} environment so
    that it treats grouped question truly as a unit.  Added two
    optional parameters to the environment. The document author
    can specify a JavaScript function to grade the grouped
    questions.  Despite its name, \texttt{mathGrp} environment
    works for text fill-in as well as math fill-ins (but not for
    multiple choice).  The \texttt{mathGrp} environment also now
    works for the \texttt{shortquiz} environment (in addition to
    the \texttt{quiz} environment). See \Nameref{grouped}.

    \item Separated the \href{eformman.pdf}{eforms} support from
    the \textsf{exerquiz} package. \textsf{eforms} is now a
    stand-alone package that can be used by people who may want to
    use PDF forms outside the context of a document that contains
    quizzes. See the demo file \texttt{eforms\_tst.tex}.

    \item \textcolor{red}{Important:} There is one required
    parameter for the \texttt{quiz} environment, a base name used
    to build the names of the form fields used in the quiz, e.g.,
    \verb+\begin{quiz}{qzQuiz1}+. Now this base name of the quiz
    should only contain ascii characters or numbers.  This name is
    now used to construct a JavaScript object, so it must be a
    valid JavaScript object name. For example \texttt{qzQuiz1} is
    correct, but \texttt{qz:Quiz-1} is not.

    \item Created \cs{@PromptButton} a new correct answer button for quizzes. Quiz
        questions might have a series of parts in which the answer
        to one part depends on the answer to the previous part.
        With \cs{@PromptButton}, the answer can be optionally
        provided while the student is taking the quiz. See the section on the
        Prompt Button for details.

    \item Created a grouping environment for math fill-in questions.  See the section
        entitled Grouping Math Fill-in Questions.

    \item Added a \texttt{noquizsolutions} option -- mostly for online quizzing --
    see page~\pageref{sss:noquizsolutions}.
    \item Changed the behavior of \cs{eqButton}: If \hyperref[sss:nocorrections]{\texttt{nocorrections}}
            is used, the \cs{eqButton} does not appear. This is reasonable,
            \cs{eqButton} is used to display corrections to the quiz.
    \item Modified the \texttt{questions} environment so that it behaves more like
            standard {\LaTeX} list environments. You can now nest the \texttt{questions}
            environment three deep. See \texttt{pc\_test.tex} for examples.
    \item Added an optional argument to the \cs{Ans} command for multiple-choice questions
        for assigning \textit{partial credit}. See the demo file \texttt{pc\_test.tex}.
    \item Modified the behavior of the link-type multiple choice questions  When you select
            an alternative, a check appears superimposed over the letter. I've removed the
            alert dialog informing the user that he has change his selection. The file
            \texttt{pc\_test.tex} illustrates this.
    \item The \hyperlink{paraRespBoxNine}{ninth parameter} of \cs{RespBoxMath} has been modified to take an JS object.
         This object can have two properties:
\begin{sverbatim}
    { [ priorParse: <js function>  | <array of js functions> ],
      [ comp: <compare function> ] }
\end{sverbatim}
        The \texttt{priorParse} property is used to filter the user's answer prior to be fully parsed,
        while the \texttt{comp} property is used to specify a custom compare function.
        See the demo file \texttt{integer\_tst.tex} for examples.
    \item Again for the math fill-in (\cs{RespBoxMath}), the way in which you specify variables (the \hyperlink{paraRespBoxThree}{third parameter})
        has changed: Specify the variables using a comma-delimited list, for example, \texttt{(r:x,i:n,y)}.
        Typing the variables has been introduced: \texttt{r:x} means that \texttt{x} is a real variable (the default)
        and \texttt{i:n} means that \texttt{n} is an integer variable. See the demo file \texttt{integer\_tst.tex}.

     \item Added complete support with \textsf{eForms}. Integrated
         this support into the quiz environments. See the \texttt{\href{eformman.pdf}{eForms Manual}}
         documentation file for complete details.
     \item Documented the use of form check boxes with the \texttt{shortquiz}
     environment and introduced the commands
 \begin{sverbatim}
 \sqTurnOffAlerts,\sqTurnOnAlerts,\sqNoCorrections,\sqNoCorrections
 \end{sverbatim}
     to control response feedback.
     \item Added language \texttt{finnish} Options. Translation due
         to Paivi Porras.
    \item Eliminated the need for the \cs{ifforAcroV} switch in the
    \texttt{*.ins} files. I've written new code that detects whether Acro4 or
    Acro5 is being used; in the latter case, a \texttt{try/catch} strategy is
    used to catch exceptions.
    \item Added a \texttt{textures} option, due to Ross Moore.
    \item Wrote same JavaScripts and placed them in the \texttt{dljslib} to
    process questions in which an equation is the expected answer. See the
    sample file \texttt{jqzequat.tex}.
    \item Wrote the \texttt{dljslib} Package, this is a library of JavaScript
    functions that extends the capabilities of \textsf{exerquiz}.
    \item Extended math fill-in questions to include arbitrary multivariate
    questions.  See the sample file \texttt{multivar.tex}.
    \item  Added several macros: \cs{defaultquiztype} and
    \cs{quiztype}. The first command takes no arguments, the
    second on takes one: either l or f; e.g. \verb+quiztype{f}+
    (\verb+quiztype{l}+) causes the quiz environment to ignore the
    optional `quiztype' parameter (\texttt* or no \texttt*) and to
    use a form-type (resp.\ link-type).  Placing
    \cs{defaultquiztype} reverts the quiz environment back to its
    default behavior (obeying the first optional parameter).
    \item Added the convenience commands for the \texttt{quiz} environment: \cs{useBeginQuizButton},
    \cs{use\-EndQuizButton}, \cs{useBeginQuizLink} and \cs{useEndQuizLink}.
    See \mlNameref{sss:FormButton} for details.\footnote{This link requires Adobe Reader 7.0 or later.}
    \item Added the \texttt{noHiddensolutions} for \textsf{exerquiz}, and
    added an `\texttt{H}' option for exercises to hide solutions, see
    \nameref{sss:hide} and \nameref{sss:noHideopt}, for details.
    \item Added a method of assigning points to a quiz question. Useful
    for submitting questions to a CGI for recording in a database.
    \item Added a \texttt{noPeeking()} JS function and supporting macros.
    When the \cs{NoPeeking} command is executed, the student cannot
    see the solutions to the quizzes (not shortquizzes) by browsing
    through the file. See \Nameref{sss:nopeek}.
    \item Added new JS functions \texttt{lowThreshold()} and
    \texttt{highThreshold()}. I modified the end of quiz macros to
    incorporate the calling of one of these two (or calling a
    document author defined routine). See \Nameref{threshold}.
    \item  Added a `\texttt{debug}' option, which gets passed on
    to \texttt{insdljs} package. This can be used to write some
    debugging commands within your JS. (\Nameref{sss:debug})
    \item For math fill-ins, the author does not use the
    JavaScript syntax; the author can use simplified notation, e.g,
    \verb+2*x*e^(x^2)+ instead of \texttt{2*x*exp(pow(x,2))}.
    Author's answer now passes through the ParseInput JS routine.
    \item Now the document author can define a custom
    JS function to process an answer. Also, when writing
    Math fill-in questions, you can also define your
    own variable (no longer restricted to just `$x$').
    \item Added in a \texttt{solutions} environment to
    \texttt{quiz} environment. Added additional optional parameter
    into the \cs{RespBox}, \cs{RespBoxTxt} macros to indicate the
    presence of a solution. The solution is viewed by
    shift-clicking on the ``Ans'' Button (\cs{CorrAnsButton}). A
    button or checkbox that has a solution has a boundary color of
    \cs{solutionColor}.
    \item \texttt{Exerquiz} now uses the package \texttt{insDLSJ} to
    insert document level JavaScripts; this gives the document author
    a chance to write custom JavaScripts. I've also modified many macros
    that enable the document author to ``hook'' into.
    \item New command \cs{RespBoxEssay} that can be used to pose
    Essay-type questions. The question is not evaluated by JavaScript
    within the document; rather, this question should be submitted
    to a CGI for later review by the instructor.
    \item \cs{RespBoxMath} has been defined to be the same as
    \cs{RespBox} to give a little more consistency in naming.
    \item The use of \cs{RespBoxNT} has been deprecated. Added in
    feature to customize the comparison of two answers.
    \item Added a text fill-in question type that can be used in the
    \texttt{short\-quiz} and \texttt{quiz} environments.
    (\Nameref{sss:oTxtQ})
    \item A \texttt{preview} option has been added. When this
    option is used, the bounding rectangles of all form fields
    are framed so their positions can be seen in a dvi previewer.
    See \mlNameref{sss:preview}.\footnote{This link requires Adobe Reader 7.0 or later.}
 \end{questions}

 \subsection{insdljs}

 The \texttt{insdljs} Package is a general purpose \LaTeX{} package
 for inserting Acrobat JavaScript into the document level section
 of a PDF document. The package features the \texttt{insDLJS}
 environment.  This environment typically goes in the preamble of a
 \LaTeX{} source file, or in the style files of a \LaTeX{} package.
 See the documentation contained within the \texttt{insdljs.dtx}
 file for additional details.  There is a sample file,
 \texttt{insdljs\_ex.tex} that can be used as a startup test file.

\begin{questions}
    \item (Version 2.0f) Added a command \cs{previewMiKTeX} which redefines \cs{jsR} and
    \cs{jsT} that enables, I hope, the \textsf{yap} previewer to preview
    an \textsf{AeB} document.
     \item Added the \texttt{defineJS} environment. Use this environment to
     write JavaScript for buttons and other form fields.
     \item Added the \texttt{execJS} environment. Use this environment to write
     ``discardable'' JavaScript that will execute when the newly created document is opened
     in the Acrobat Viewer.  (The Acrobat Viewer, not the Reader, is required.)
     See the demo file \texttt{execjstst.tex}.
\end{questions}

\subsection{dljslib}

The \texttt{dljslib} Package acts as a library of JavaScript
functions. Due to the increased programmability of
\texttt{exerquiz} and its new-found flexibility, it is possible to
write a number of different routines to handle various kinds of
math fill-in questions. These JavaScript functions can be stored
in the library and retrieved when needed.  This package requires
the \texttt{insdljs} package.
 \begin{questions}

    \item There are some major user-contributed JavaScript routines.
        The following functions, which are \texttt{ProcResp}-types, were
        written for an online grading system being developed by Drs.\
        Bruce Wagner and David Arnold, and Mr. Jacob Miles-Prystowsky.
        The descriptions given below were provided by the authors.
        I want to extend my deep thanks for their contribution! See the
        documentation of \textsf{dljslib.dtx} for extended details.
    \begin{questions}
        \item The JS function \texttt{ProcRespSetFormula} grades an
            unordered list of formulas, such as \texttt{x}, \verb!x^2!, \verb!x^3!.
        \item \texttt{ProcRespFactors} is for grading polynomial factoring
            questions. If a polynomial factors as
            \verb!-5x^2(x-4)(x+2)!, the answer is only unique
            up to the order of factors and a change of sign on an
            even number of factors. For example, it could also be
            written as \verb!-5x^2(x+2)(x-4)! or
            \verb!5x^2(-x+4)(x+2)! or \verb!-5(x-4)(x+2)x^2!, etc.
            This script should grade all of these correctly.
        \item The JS function \texttt{ProcPoint} is almost an exact copy of
            the JS function \texttt{ProcVec}, but uses parentheses instead of angle brackets as
            delimiters. This is for questions that have a point $(x,y)$ as the
            answer.
        \item  \texttt{ProcRespIntervals} is for grading questions that have an
                interval or union of intervals as the answer.
    \end{questions}
    \item[] The sample file for this contribution is \texttt{dljslib\_apw.tex}.

    \item Added a feature whereby you can create a file
    \texttt{libcusopt.opt} to declare your own options for this
    package. These declarations should be a combination of existing options.
    For example,
\begin{sverbatim}
\DeclareOption{procrespgrp}{\includeOptions{%
        vectors,setSupport,equations,intervals,%
        unordered,point,factors,ImplMulti}
}
\end{sverbatim}
    creates a new option for \textsf{dljslib} called \texttt{procrespgrp} which bundles
    together all frequently used JS functions frequently used. The file
    \texttt{libcusopt.opt} can be placed in the {\LaTeX} search path, but it's perhaps
    best to put it in your source file, so that the options are local to that folder.
    The command \cs{includeOptions} is defined in \texttt{dljslit.dtx}.

    \item[] A sample \texttt{libcusopt.opt} is distributed with the
    {\AEB}, this sample file consists of the example given above,
    and is used by the sample file \texttt{dljslib\_apw.tex}.

\item[] \rule[2pt]{1in}{.4pt} Older New Stuff \rule[2pt]{1in}{.4pt}

    \item  Added an option \texttt{limitArith} to the
    \textsf{dljslib}, which limits the use of arithmetic operations
    and built-in functions. The demo file is
    \texttt{limarth.tex}, found in the \texttt{dljslib\_examples}
    folder. \item  Added \texttt{setSupport} option to
    \textsf{dljslib}. This gives support to math fill-in questions
    for processing (1) a set of numerical answers or a comma
    delimited list of answers; (2) a symbolic set of answers. The
    demo file is \texttt{set\_test.tex}.

     \item Added \texttt{combinatorics}, \texttt{nodec} and
     \texttt{noBinFac} to the JavaScript library. These additions
     are due to Ross Moore and Frances Griffin and were developed
     for their
     \href{http://rutherglen.ics.mq.edu.au/~macqtex/}{MacQ{\TeX}}
     online testing system. Some of the changes listed in the
     \textsf{exerquiz} section above were to accommodate these new
     functions (for example, having integer variables).  Many, many
     thanks for generously sharing their JavaScript with the
     \TeX/PDF community. The test file for these features is
     \texttt{integer\_tst.tex}.
\end{questions}
% See the documentation contained within the \texttt{dljslib.dtx} document, and
% see \Nameref{sss:dljslib}.

\section{What's Old}

\paragraph*{History.} The \textsf{web} and \textsf{exerquiz} packages were written in
preparation for a two-day workshop on \LaTeX/PDF that I gave at
the \textcolor{red}{College of the Redwoods}, April~30-May~1,
1999, at the invitation of David Arnold.  The workshop forced me
to take many of the basic macros that I had developed in plain
\TeX\ and convert them to \LaTeX.

Significant additions to the \textsf{exerquiz} were made immediately
following the $\text{20}^{\text{th}}$ Annual Conference of the \TeX\
User's Group (\textsf{tug}), in August, 1999, Vancouver, British
Columbia, which I attended.

The \textsf{insdljs} package was written for the $\text{22}^{\text{nd}}$
Annual Conference of the \TeX\ User's Group (\textsf{tug}), in
August 2001, at the University of Delaware, Newark, Delaware.

The \texttt{execJS} environment was created as a result of some
work I did for the Seybold SF Convention, 2002 (see article at
\href{http://www.planetpdf.com/mainpage.asp?webpageid=2130}{planetPDF});
the techniques were first presented to the PDF public at this
convention.  My complete presentation from that conference is
again available from
\href{http://www.planetpdf.com/mainpage.asp?webpageid=2310}{planetPDF}.

\paragraph*{Acknowledgements.} Noel Vaillant, \url{www.probability.net}, deserves my thanks for
his enthusiasm for the \textsf{web} style file and his initial
work on it inspired me to make a serious effort at writing a
\LaTeX\ package.

Thanks also goes out to Jean-Michel Sarlat for writing a French
version of the \textsf{web} and \textsf{exerquiz} packages. See
his \href{http://melusine.eu.org/syracuse/}{Syracuse} Web site. He
urged me to include a language option.  Thanks also goes to
Michael Wiedmann who suggested a language option many months
earlier, but I'm afraid it landed on deaf ears at the time.
These two provided the translations for the \texttt{french} and
\texttt{german} options. (January 1, 2000)

My thanks  to Heiko Oberdiek, who took a close look at
\textsf{insdljs}. He made several suggestions, and urged me to
make some significant improvements to this package.

\medbreak\noindent
Now, I really must get back to retirement.\enspace\dps

\addtocontents{toc}{\protect\vspace{2ex}}
\includeexersolutions
\includequizsolutions


\newpage
\markright{References}

\begin{thebibliography}{[1]}\label{references}
\addcontentsline{toc}{section}{\protect\numberline{}References}

\bibitem{tech:AcroJS}\hypertarget{references}{}%
   Acrobat JavaScript Scripting Refernce, Technical Note \#5431, Version~6.0.,
   Adobe Systems, Inc., 200\\ \url{http://partners.adobe.com/asn/acrobat/docs.jsp}

\bibitem{book:Lamport}
   Leslie Lamport, \textit{\LaTeX: A Document Preparation}
   System (2nd ed.), Addison-Wesley Publishing Company,
   1994, ISBN 0-201-52983-1.

\bibitem{book:Goossens1}
   Michel Goossens, Frank Mittelbach and Alexander
   Samarin, \textit{The \LaTeX{} Companion}, Addison-Wesley
   Publishing Company, 1994, ISBN 0-201-54199-8.

\bibitem{book:Goossens2}
   Michel Goossens, Sebastian Rahtz, and Frank
   Mittelbach, \textit{The \LaTeX{} Graphics Companion},
   Addison-Wesley Publishing Company, 1997, ISBN
   0-201-85469-4.

\bibitem{book:Goossens3} Michel Goossens, and Rahtz, Sebastian,
   with Gurari, Eitan,  Moore, Ross, and Sutor, Robert,
   \textit{The \LaTeX\ Web Companion},
   Addison Wesley Longman, Reading, Massachusetts, USA, 1999.
   ISBN:  0-201-43311-7.

\bibitem{book:Kopka}
   Helmut Kopka and Patrick W. Daly, \textit{A Guide to
  \LaTeX2e} (2nd ed.), Addison-Wesley Publishing Company,
  1995, ISBN 0-201-43777-X.

\bibitem{book:Knuth}
  Donald E. Knuth, \textit{The \TeX book}, Addison-Wesley
  Publishing Company, 1987, ISBN 0-201-13448-9.

\bibitem{book:Merz}
  Thomas Merz, \textit{Web Publishing with Acrobat/PDF},
  Springer-Verlag Berlin Heidelberg New York, 1998,
 ISBN 3-540-63762-1.

\bibitem{article:Story:Pdfmarks}
 D.P. Story, \textit{Pdfmarks: Links and Forms}, AcroTeX Web
 Site, 1998,\\\url{www.math.uakron.edu/~dpstory/acrotex.html}

\bibitem{article:Story:LaTeX}
 D.P. Story, \textit{Using \LaTeX\ to Create Quality PDF Documents for the
 WWW}, Acro\!\TeX\ Web Site, 1998,\\
 \url{www.math.uakron.edu/~dpstory/acrotex.html}

\end{thebibliography}


% \printindex
\end{document}

I copied the file amsfonts.map from texmf/dvips/ams
to texmf/pdftex/base.  I then edited the file
pdftex.cfg and included the line
map  +amsfonts.map.
map standard.map   % to get dings

output_format 1
compress_level 9
decimal_digits 2
page_width 210mm
page_height 297mm
horigin 1in
vorigin 1in
map standard.map
map +cm.map
map +amsfonts.map
% map +logo.map
% map +lucida.map
% map +misc.map
% map +cs.map
% map +cmttf.map
% map +test.map
% map +ttftest.map
% map +mono.map


\subsubsection{An Example}\label{sss:quizexample}


%\renewcommand\bqlabel{Begin Quiz}
%\renewcommand\eqlabel{End Quiz}

The following example is used at the Acro\!\TeX\ Web Site in a
demonstration program on how to use \emph{Adobe's FDF Toolkit} to
process form data from a pdf file.  (Form data exported from a pdf
file is in the \emph{Forms Data Format}, or  FDF.)  The url for
this demonstration program is \url{\bUrl/demo/quiz1_q.pdf}.


\symbolchoice{diamond}

\begin{quiz*}{QuizTotal} Answer each of the following. Passing is 100\%.

\begin{questions}

\item Calculate the limit, if it exists, otherwise, state that it does not
exist (d.n.e.).
$$
    \lim_{x\to3}\frac{\sqrt{3x}(x^2-4)^2}{x^2-2x-1}.
$$
\begin{answers}4
\Ans0 $75$  &\Ans1 $\dfrac{75}2$ &\Ans0 $25$ &\Ans0 d.n.e.
\end{answers}

\item Calculate the limit, if it exists, otherwise, state that it does not
exist (d.n.e.).
$$
    \lim_{t\to2}\frac{t^2-4}{t-2}.
$$
\begin{answers}4
\Ans0 $2$ &\Ans0 $+\infty$ &\Ans1 4 &\Ans0 d.n.e.
\end{answers}

\item Calculate the limit,
$\displaystyle\lim_{t\to2^-}\frac{t^2}{t-2}$, if it exists, otherwise,
state that it does not exist (d.n.e.).
\begin{answers}4
\Ans1 $-\infty$ &\Ans0 $+\infty$ &\Ans0 $4$ &\Ans0 d.n.e.
\end{answers}

\item Using one-sided limit techniques, determine whether the limit
$\lim_{x\to2}f(x)$ exists and state its value, where
$$
       f(x) = \cases
             |x^2-5|  & \text{for }x\le2 \\
            \dfrac{x-2}{x-2} & \text{for }x>2
          \endcases
$$
\begin{answers}4
\Ans0 $-1$& \Ans0 0 &\Ans1 1 &\Ans0 d.n.e.
\end{answers}

\item Calculate the limit,
$\displaystyle\lim_{x\to-2}\frac{\sqrt{x+8}-\sqrt6}{x+2}$,
if it exists, otherwise, state that it does not
exist (d.n.e.).
\begin{answers}4
\Ans0 $\dfrac1{\sqrt6}$ &\Ans1 $\dfrac{\sqrt6}{12}$ &
  \Ans0 $\dfrac12$ &\Ans0 d.n.e.
\end{answers}

\item There is some function $f$ and some number $a$ in the domain of
$f$ such that
$$
    f'(a)=\lim_{x\to-2}\frac{\sqrt{x+8}-\sqrt6}{x+2}
$$
Name the function $f$ and the number $a$.
\begin{answers}2
\Ans1 \strut$f(x)=\sqrt{x+8}$; $a=-2$ &\Ans0 $f(x)=\sqrt{x}$; $a=-2$\\[3pt]
\Ans0 \strut$f(x)=\sqrt{x+8}$; $a=0$  &\Ans0 $f(x)=\sqrt{x+8}$; $a=2$
\end{answers}
\end{questions}
\end{quiz*}\quad\raisebox{-3pt}
{\TextField[width=3in,name=QuizTotal,bordercolor=0 0 1,default=\eqScore]{}}

\medskip\noindent
The partial code for this quiz looks like this.
\begin{sverbatim}
\symbolchoice{diamond}  % change to diamond `check'
\begin{quiz*}{QuizTotal} Answer each of the following....
\begin{questions}
.........
\end{questions}
\end{quiz*}\quad\raisebox{-3pt}{\TextField[name=QuizTotal,
   default=\eqScore,bordercolor=0 0 1,width=3in]{}}
\end{sverbatim}


\begin{quiz*}{qzTeXc} Answer each of the following. Passing
is 100\%.
\begin{questions}
\item What \TeX\ System does Thomas Esser maintain?
\begin{answers}4
\Ans0 MiK\TeX &\Ans0 cs\TeX &\Ans1 te\TeX &\Ans0 fp\TeX
\end{answers}
\item What \TeX\ System does Fabrice Popineau maintain?
\begin{answers}4
\Ans0 MiK\TeX &\Ans0 cs\TeX &\Ans0 te\TeX &\Ans1 fp\TeX
\end{answers}
\item What \TeX\ System does Christian Schenk maintain?
\begin{answers}4
\Ans1 MiK\TeX &\Ans0 cs\TeX &\Ans0 te\TeX &\Ans0 fp\TeX
\end{answers}
\end{questions}
\end{quiz*}\quad
\raisebox{-2pt}{\renewcommand\LayoutTextField[2]{% label, field
\medskip\noindent#2}%
\TextField[name=qzTeXc,width=1.25in,align={0 /Ff 1},
   default=\eqScore]{}\raisebox{3.5pt}{%
   \eqButton[\BC{0 0 1}%  blue border color
   \CA{TeX}%              Button text
   \RC{Users}%            rollover text
   \AC{Group}%            pushed text
   \DA{/TiRo 10 Tf 0 0 1 rg}% times roman, 10 pt, blue text
%   \W{1}\S{I}%
   \BS{/W 1 /S /I}%      % border width 1, inset button
   ]{qzTeXc}}}


%\noindent\begin{minipage}{\linewidth}
\begin{quiz}*{qzTeX} Answer each of the following. Passing
is 100\%.
\begin{questions}
\item Who created \TeX?
\begin{answers}4
\Ans1 Knuth &\Ans0 Lamport &\Ans0 Carlisle &\Ans0 Rahtz
\end{answers}
\item Who originally wrote \LaTeX?
\begin{answers}4
\Ans0 Knuth &\Ans1 Lamport &\Ans0 Carlisle &\Ans0 Rahtz
\end{answers}
\end{questions}
\end{quiz}


\subsection{Greater Customization with
  \texorpdfstring{\protect\cs{TextField}}
  {\textbackslash TextField}}\label{ss:modoutputfield}

The \cs{ScoreField} macro is defined as
\begin{sverbatim}
\newcommand{\ScoreField}[2][width=1.5in]{%
  \TextField[name=#2,default=\eqScore,align={0 /Ff 1},#1]{}}
\end{sverbatim}

\noindent You can see the command takes one required argument and one
optional argument (the required one is label \texttt{\#2} and the
optional one is \texttt{\#1}).  The required argument becomes the
\texttt{name} of the field, and the optional parameter is just
added into the parameter list of \cs{TextField}.  You can also see
the \texttt{value} of the field is set to \cs{eqScore}, which
expands to `Score:' in English.

More sophisticated score fields can be obtained by manipulating
\cs{TextField} directly.

We can modify the appearance of the text field in a couple of
ways: (1) by changing some of the options in \cs{TextField}; (2)
changing the macro definition of \cs{LayoutTextField}.

\subsubsection{Admissible key-value pairs of
\texorpdfstring{\protect\cs{TextField}}
  {\textbackslash TextField}}\label{sss:modtextfield}

In this section we will outline methods of modifying the
appearance of \cs{TextField} text box, which is a macro defined in
the \textsf{hyperref} package.

The following table lists several useful options for
\cs{TextField}.

%Numerous examples on how to use these options
%follow the table.


\begin{longtable}{>{\ttfamily}lll>{\PBS\raggedright}p{1.5in}}
\multicolumn{4}{l}{\large\color{red}$\blacktriangleright$
\color{blue} \cs{TextField} Options: Dimension Related} \\ [3pt]
height      & dimen  &       & height of box \\
width       & dimen  &       & width of field box \\ [12pt]
%
\multicolumn{4}{l}{\large\color{red}$\blacktriangleright$
\color{blue} \cs{TextField} Options: Text Related} \\ [3pt] align
&name &0 & $0=\text{left-aligned}$,\newline
                 $1=\text{centered}$\newline
                 $2=\text{right-aligned}$ \\
charsize    & dimen  & 10pt & Font size of text \\ [12pt]
%
\multicolumn{4}{l}{\large\color{red}$\blacktriangleright$
\color{blue} \cs{TextField} Options: Color Related} \\ [3pt]
backgroundcolor & RGB &1 1 1& Color of background color.\\
bordercolor & RGB & 1 0 0 & Color of the border.\\ [12pt]
color       & RGB    & 0 0 0 & Color of text within field\\
borderstyle & name &S& S (solid), B (beveled),
                      I (insert), U (underlined) \\ [12pt]
%
\multicolumn{4}{l}{\large\color{red}$\blacktriangleright$
\color{blue} \cs{TextField} Options: Style Related} \\ [3pt]
borderwidth & number & 1 & Width of the border of the
                           box (in points)\\[12pt]
%
\multicolumn{4}{l}{\large\color{red}$\blacktriangleright$
\color{blue} \cs{TextField} Options: Field Related} \\ [3pt]
default     &        &       & default value of the field \\
name        & name   &       & name of the field\\
readonly    & Boolean& false & set read-only switch
\end{longtable}
%\end{flushleft}

A complete list of options comes with the \textsf{hyperref}
documentation; or see Michel Goossens \textit{et
al}~\cite{book:Goossens3}, entitled \textit{The \LaTeX\ Web
Companion}

We next present several examples that illustrate the use of these
options.

\subsubsection{Some  Examples of
\texorpdfstring{\protect\cs{Textfield}}
  {\textbackslash Textfield}}\label{sss:exampltextfield}

\def\LayoutTextField#1#2{\redpoint\raisebox{-3pt}{#2}}

\TextField[width=2in,name=QuizTotal,bordercolor=0 0 1,
    default=\eqScore]{}

\begin{sverbatim}
\TextField[width=2in,name=QuizTotal,
   bordercolor=0 0 1,default=\eqScore]{}
\end{sverbatim}

\noindent The \texttt{readonly} is an important attribute to have---the
students cannot modify their score.  Unfortunately,
\texttt{readonly} \emph{does not work} with \textsf{hyperref}. To
overcome this ``bug'' in \textsf{hyperref}, you have to be a
little sneaky.

\TextField[width=2in,name=Example,bordercolor=0 0 1,
   backgroundcolor=0.98 0.92 0.73,color=1 0 0,align={0  /Ff 1},
   default=Text Field with Various Options]{}

\begin{sverbatim}
\TextField[width=2in,name=Example,bordercolor=0 0 1,
   backgroundcolor=0.98 0.92 0.73,color=1 0 0,
   align={0  /Ff 1}, % realonly is /Ff 1 in pdf code
   default=Text Field with Various Options]{}
\end{sverbatim}

\noindent The read only key-value (\texttt{/Ff 1}) can ``ride'' into the
code on the ``back'' of the \texttt{align} parameter.
(\texttt{align=0} means left-aligned in the text field.)

\TextField[width=2in,name=Example,bordercolor=1 1 1,color=0 0 1,
   backgroundcolor=.753 .753 .753,borderstyle=B,align={1 /Ff 1},
   default=Your Score goes here]{}

\begin{sverbatim}
\TextField[width=1.25in,name=Example,bordercolor=1 1 1,
   backgroundcolor=.753 .753 .753,color=0 0 1,
   borderstyle=B,align={1 /Ff 1}, % /Ff 1 = readonly
   default=Your Score goes here]{}
\end{sverbatim}

\subsubsection{Modifying
\texorpdfstring{\protect\cs{LayoutTextField}} {\textbackslash
LayoutTextField}}\label{sss:modlayout}

\renewcommand\LayoutTextField[2]{#1 #2}

The macro \cs{LayoutTextField} allows you to design how your text
box is formated by \TeX.  The default definition for
\textsf{hyperref} is
\begin{sverbatim}
   \newcommand\LayoutTextField[2]{#1 #2}
\end{sverbatim}
\noindent The first argument is the \textsf{label}, the second is the
textfield.  For example,  above definition yields:\\
\TextField[width=2.4in,name=Name]{\bfseries Enter your name: }
\begin{sverbatim}
   \TextField[width=2.5in,name=Name]{Enter your name: }
\end{sverbatim}

\medskip
The \textsf{exerquiz} package redefines this to
\begin{sverbatim}
\renewcommand\LayoutTextField[2]{% label, field
   \medskip\noindent#2}
\end{sverbatim}

Finally, the field examples given earlier had the following
definition for \cs{LayoutTextField}:
\begin{sverbatim}
\newcommand\redpoint{\par\medskip\noindent
  \makebox[\parindent][l]{\large\textcolor{red}
  {$\blacktriangleright$}}}
\renewcommand\LayoutTextField[2]{\redpoint
  \raisebox{-3pt}{#2}}
\end{sverbatim}

\noindent I lowered the text field to line up better with the
\texttt{amssymb} symbol \cs{blacktriangleright}
($\blacktriangleright$), which is painted red. In this way we
obtain

\def\LayoutTextField#1#2{\redpoint\raisebox{-3pt}{#2}}
\TextField[width=2in,name=QuizTotal,bordercolor=0 0 1,
    default=\eqScore]{}
\begin{comment}
\begin{sverbatim}
\TextField[width=2in,name=QuizTotal,bordercolor=0 0 1,
    default=\eqScore]{}
\end{sverbatim}
\end{comment}

\subsubsection{Custom Comparisons}\label{cusComp}

This section is devoted to customizing the method of comparison
for the math type objective question using \cs{RespBoxMath} command.

When the user enters an answer into the response box, a
document level JavaScript function, \texttt{ProcResp}, is called.
As discussed earlier, the value of the expression entered by the
user is compared the corresponding value of the answer given by
the document author.  The key word in the last sentence is ``compared''. There
is a compare function that compares the author's answer with the user's answer.

In general, the \texttt{ProcResp} passes five parameters to a compare function.
The compare function can be be customized by the document author; there is, however,
a standard, or default, compare function. It is defined as follows:

\begin{sverbatim}
function diffCompare(a,b,c,F,G) {
    var x;
    with(Math) {
    x = c;
    F = eval(F);
    G = eval(G);
    return abs ( F - G );
    }
}
\end{sverbatim}
\noindent The meaning of these five parameters of this comparison function are

\begin{description}
    \item[\ttfamily a:] The left-hand endpoint of the interval.
    \item[\ttfamily b:] The right-hand endpoint of the interval.
    \item[\ttfamily c:] The point at which both the user and author answers are to
    be evaluated.
    \item[\ttfamily F:] The author's answer, a function in $x$.
    \item[\ttfamily G:] The user's answer, a function in $x$.
\end{description}

The function \texttt{diffCompare} evaluates each of the two
functions \texttt F and \texttt G at \texttt{x = c}, and returns
the absolute difference in the two. For this simple comparison, the parameters
\texttt a and \texttt b are not used.

The \texttt{diffCompare} will be sufficient for most, but not
all  mathematical questions. For example, consider the following question:

\medskip
\begin{oQuestion}{indefInt1}
$\displaystyle\int \sin(x)\,dx = $\space
\RespBoxMath{-cos(x)}{4}{.0001}01\kern1bp\CorrAnsButton{-cos(x)}
\end{oQuestion}

\medskip\noindent
There are infinitely many answers, all of them of the form $-\cos(x)+C$. Note that
the only input that is marked as correct is $-\cos(x)$, the JavaScript does not judge
$-\cos(x)+1$, to be correct, for example. The problem is, of course, the method of
comparison, \texttt{diffCompare}, is not appropriate to this particular problem.

The above question needs a specialized method of comparison. Consider
the compare function:

\begin{sverbatim}
\begin{insDLJS}[indefCompare]{compare}{Comparisons}
function indefCompare(a,b,c,F,G) {
    var C, eqx;
    with(Math) {
        eqx = a;
        C = eval(F)-eval(G);
        eqx = c;
        F = eval(F);
        G = eval(G);
        return abs( F - G - C );
    }
}
\end{insDLJS}
\end{sverbatim}
\noindent This environment was placed in the preamble of this document, and uses the
\texttt{insdljs} package, which comes with the \texttt{exerquiz} distribution.
This will add the function \texttt{indefCompare} function to the document level
JavaScript section of the PDF file when it is created.

\redpoint It is \emph{important} to note that when you are writing your
own compare functions, the independent variable must be named
`\texttt{eqx}'. The functions \texttt F and \texttt G have been
searched and their original independent variable has been replaced with
`\texttt{eqx}'. To get a correct evaluation, therefore, any compare functions
must use the `\texttt{eqx}' to evaluate the functions.

Let's try that same question, but with this compare function instead of the
default compare function

\medskip
\begin{oQuestion}{indefInt2}
$\displaystyle\int \sin(x)\,dx = $\space
\RespBoxMath{-cos(x)}{4}{.0001}01[indefCompare]\kern1bp\CorrAnsButton{-cos(x)}
\end{oQuestion}

\medskip\noindent
Now, if you enter $-\cos(x) + 10$, for example, that answer will be judged as
correct by \texttt{ProcResp}.

The code for the above response box is

\begin{sverbatim}
\begin{oQuestion}{indefInt2}
$\displaystyle\int \sin(x)\,dx = $\space
\RespBoxMath{-cos(x)}{4}{.0001}01[indefCompare]\kern1bp
\CorrAnsButton{-cos(x)}
\end{oQuestion}
\end{sverbatim}

\noindent The optional ninth parameter is present in the
\cs{\hyperlink{paraRespBox}{RespBoxMath}}. Use this parameter to
pass the name of the compare function that is to process the
question. Here, we specify \texttt{indefCompare}, which was
defined in the preamble of this document using the
\texttt{insDLJS} environment.
