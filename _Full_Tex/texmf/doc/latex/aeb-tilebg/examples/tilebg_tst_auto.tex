\documentclass{article}
\usepackage[tight,usetemplates,nodirectory]{web} % dvipsone, dvips, pdftex
\usepackage{aeb_tilebg}

\title{Creating Tiled Backgrounds}
\author{D. P. Story}
\subject{Test file for the AcroTeX eDucation Bundle}
\keywords{LaTeX, Web package, tiled backgrounds, Adobe Acrobat}
\university{Acro\negthinspace\TeX.Net}
\email{dpstory@acrotex.net}
\version{1.0}
\revisionLabel{Prepared:}


\graphicspath{{tile/}}

\setTileBgGraphic[scale=.25]{./bike_10}
% n_rows, n_columns, left,right,top, bottom
\autosetScreensizeWithMargins{7}{5}{.25in}{.25in}{24pt}{.25in}

\textBgColor{cornsilk}

\parindent0pt
\parskip\medskipamount

\begin{document}

\maketitle

\setTileBgGraphic[scale=.75]{./brewtile1}

This newest package is a simple application of established packages
\textsf{graphicx}, \textsf{multido} and \textsf{web}.

Download your favorite tiled background swatch from the Internet,
convert it to an \texttt{.eps} or a \texttt{.pdf} format (depending
if you use \textsf{distiller} or \textsf{pdftex}), place that image
in the same folder as your source document. Anywhere in your
document, use the command \verb!\setTileBgGraphic! to bring in your
tiled background using the \textsf{graphicx} package; for example,
\begin{verbatim}
\setTileBgGraphic[scale=.75]{brewtile1}
\end{verbatim}
The \textsf{multido} package places these images in the background,
starting from the upper-left corner (the lower left corner if the
\texttt{uselltiling} option is taken) going across and up
(respectively, going across and down), one row at a time. The web
package is used to manage the graphical template created.

\newpage

\setTileBgGraphic[scale=.5]{./math-bkgrd}

Inserting a second file will create a new tiled background, like so
\begin{verbatim}
\setTileBgGraphic[scale=.5]{./math-bkgrd}
\end{verbatim}

\end{document}
