\documentclass{article}
\usepackage
[%
    driver=dvips,
    nopro,
    web={
        pro,
        designiv,
%        navibar,
        tight,
        nodirectory,
%        forcolorpaper,
        centertitlepage,
        usesf
    },
    aebxmp
]{aeb_pro}

\begin{comment}
Experiment with various combinations of \noSectionNumber, \forceSubSubNumbers, Commenting
and uncommenting the \selectDings data structure (for the toc).  Comment and uncomment
the dings key-value pair for the layout sections data structure.
\end{comment}
\noSectionNumbers
%\forceSubSubNumbers

\begin{comment}
\selectDings
{
    dDingToc=\ding{082},
    ddDingToc=\ding{079},
    dddDingToc=\ding{254}
}
\end{comment}

\definecolor{orange}{rgb}{0.97,0.65,0.00}
\definecolor{indianbrown}{rgb}{0.35,0.24,0.11}


\sectionLayout
{%
    ding=\ding{082},                        % This ding will appear when \noSectionNumbers is in effect, comment out to get no ding
    indent=\prtscr{-\oddsidemargin}{0pt},   % Use \prtscr to have different values depending on for print or screen
    halign=l,                               % align left, the default
    color=\prtscr{webbrown}{red},           % text color of red
    special=\prtscr{default}{shadow},       % shadow special effects
    shadowcolor=blue                        % shadow color blue
}
\subsectionLayout{%
    ding=\ding{079},
    indent=\prtscr{-.5\oddsidemargin}{0pt},
    halign=\prtscr{l}{c},
    color=red,
    special=\prtscr{default}{fcolorbox}
}
\subsubsectionLayout{%
    ding=$\bullet$,
    halign=\prtscr{l}{r},
    color=indianbrown,
    special=\prtscr{default}{fcolorboxfit},
    framecolor=red,
    bgcolor=orange
}
\tocLayout
{%
    ding=\ding{082},
    indent=\prtscr{-\oddsidemargin}{0pt},
    halign=\prtscr{l}{c},
    color=red,
    shadowcolor=blue
}

\DeclareDocInfo
{
    title=The AeB Pro Package
        \texorpdfstring{\\[1ex]}
        {: }Highlighting the \texttt{pro} option of Web,
    author=D. P. Story,
    university=Acro\negthinspace\TeX.Net,
    email=dpstory@acrotex.net,
    subject=Test file for the AeB Pro package,
    keywords={Adobe Acrobat, JavaScript},
    talksite=\url{http://www.acrotex.net},
    talkdate={January 12, 2007},
    copyrightStatus=True,
    copyrightNotice={Copyright (C) \the\year, D. P. Story},
    copyrightInfoURL=http://www.acrotex.net
}
\talkdateLabel{Published:}

\selectColors
{
    universityColor = blue,
    titleColor = red,
    authorColor = blue,
    urlColor = webbrown,
    linkColor = webgreen,
    fileColor = webbrown
}

\newcommand{\cs}[1]{\texttt{\char`\\#1}}
\newcommand\newtopic{\par\ifdim\lastskip>0pt\relax\vskip-\lastskip\fi
\vskip\medskipamount\noindent}
\newenvironment{sverbatim}
{\par\footnotesize\verbatim}{\endverbatim}


\begin{document}

\maketitle

\tableofcontents

\section{AeB Control Central}

The AeB family of software, {\LaTeX} packages all, are for the most
part stand alone; however, usually they are used in combination with
each other, at least that is the purpose for which they were
originally designed.  When several members of family AeB are used,
they should be loaded in the optimal order. With AeB Pro, you can
now list the members of the AeB family you wish to use, along with
their optional parameters you wish to use.

\newtopic For example,

\begin{sverbatim}
\usepackage[%
    driver=dvipsone,
    web={pro,designv,tight,nodirectory,centertitlepage,usesf},
    exerquiz={<optional parameters>},
    ...,
    aebxmp,
]{aeb_pro}
\end{sverbatim}
Yes, yes, I know this is not necessary, you can always load the
packages earlier than AeB Pro, but please, humor me.

By default, the code for supporting features that require the use of
Distiller and Acrobat~Pro are included; there is a \texttt{nopro} option that excludes
these features. Use the \texttt{nopro} if you wish to use the AeB Control Center
to load the various members of the Acro\negthinspace\TeX{} family.

\newtopic See the AeB Pro documentation for the list of supported options.

\section{Document Information:
    \texorpdfstring{\protect\cs{DeclareDocInfo}}
        {\textbackslash DeclareDocInfo}}

The \texttt{web} package and the \texttt{hyperref} package both have
several data strings used in the Description tab of the Document
Properties of a PDF document. Below is the \cs{DeclareDocInfo}
``data structure'', the meanings of the keys are clear. Some of the
keys are used in the Description tab of the PDF document, others are
used in the title page, while others true play a dual role.

\begin{sverbatim}
\DeclareDocInfo{
    title=The AeB Pro Package
        \texorpdfstring{\\[1ex]}
        {: }Highlighting the \texttt{pro} option of Web,
    author=D. P. Story,
    university=Acro\negthinspace\TeX.Net,
    email=dpstory@acrotex.net,
    subject=Test file for the AeB Pro package,
    keywords={Adobe Acrobat, JavaScript},
    talksite=Okaloosa-Walton College,
    talkdate={January 12, 2007},
    copyrightStatus=True,
    copyrightNotice={Copyright (C) \the\year, D. P. Story},
    copyrightInfoURL=http://www.acrotex.net
}
\end{sverbatim}
The last three keys are particular to the \textsf{aebxmp} package,
which has been input into this document. These three keys populate
the Advanced Metadata dialog box; Acrobat~8 Pro is required for
\textsf{aebxmp} package to work correctly.

\newtopic See the AeB Pro documentation for a detailed description of \cs{DeclareDoc\-In\-fo}
and the \textsf{aebxmp} package documentation.

\section{Control over Headings}

The \texttt{pro} option of web introduces the use of the xkeyval package and with it comes a
complex choice for setting up your section headings and various elements of your title page.

\subsection{Designing your Section Headings}

When the \texttt{pro} option of the \textsf{web} package is used, the commands
\cs{section\-Layout}, \cs{sub\-section\-Layout} and \cs{subsubsectionLayout} become available.
Look in the preamble of this document, there, I've set the look of the format of each of these
three section levels. My design choices are meant to illustrate the variety of choices you have,
I myself have no sense of color, or design, for that matter.

The normal setting is to show section numbers, when you execute \cs{no\-Section\-Numbers} in the preamble,
obviously, no section numbers will be shown. (Useful for presentations, with no cross-references.) If
you specify a value for the \texttt{ding} key, then a ding appears where the section number was.

When showing section numbers, the default, one of the famous features of the \textsf{web} package was to have
a bullet for the subsubsection heading, rather than the subsubsection number. By executing
\cs{forceSubSubNumbers} in the preamble, you force the appearance of the subsubsection numbers. To get
the appearance of these numbers in the table of contents, use the \texttt{latextoc} option of \textsf{web}.

The commands can be placed in the preamble or anywhere. They take
effect at the next encountered section that is applicable.

\subsection{Designing your Initial Pages}

The same mechanism used for the formatting of the section headings is also used for the
title page and the table of contents.

\subsubsection{The Title Page}

Certain components of the title page can be controlled using the same mechanism as the
section headings.  Corresponding to the university, the title and the author are
\cs{universityLayout}, \cs{titleLayout} and \cs{authorLayout}. These three have the same key-value
pairs as the section layout commands, but they do now obey the \texttt{ding} key.

\newtopic See \texttt{aebpro\_titlepg.tex} for an interesting illustration of the key-values
of the layout for the title page.

\subsubsection{The Table of Contents}

The \cs{tocLayout} command is like the layout commands on the title page, it does not obey the
\texttt{ding} key. With it you can adjust color of the text and horizontal positioning. Special
effects can also be specified.

When \cs{noPageNumbers} is in effect, you can specify the values of the \cs{selectDings} structure
to have dings showing up in the table of contents listing. These dings may or may not match the dings
of the section labeling.  Specifying no ding for a particular section level displays no ding for that
toc entry.

\end{document}
