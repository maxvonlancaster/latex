\NeedsTeXFormat{LaTeX2e}
\documentclass[%
foldmarks,
%bankhigh,
banklow,
%draft,
refdate,
%rightdate,
%subjectdate,
twoside,
%reverseaddr,
%letterpaper]
%a4offset]
a4paper]
{akletter}

\usepackage{german}
%\usepackage[dvips]{graphicx}

% If you want to use PostScript fonts it may be a good idea
% to enable the following. This should get rid of any remaining
% cm-fonts.
%\usepackage[TS1,T1]{fontenc}
%\usepackage{textcomp}

% Options:
% a4paper:     ISO A4 sized paper
% a4offset:    ISO A4 sized paper with offset to the right, for 
%              presentations folders
% letterpaper: US letter size
% bankhigh:    Bankconnections top right
% banklow :    Bankconnections in footer
% foldmarks:   Enable printing of foldmarks
% reverseaddr: Exchange adrfield and rightfield

% Specify the name of the .cfg-file
% If not given akletter.cfg will be used

\usename{akletter}
%\usename{akfax}  % Use Sans-Serif for better readability
%\usename{akfaxps}% Use PostScript fonts (Times/Courier)

% Create an extra page with labels
%
\makelabels

%% This document will use the english page-layout throughout
%% the file (only visible in the footer) even if it contains
%% one English and two German letters.
%% The rightfield will be ok for all three letters.
%%
%% There is no way to circumvent this, since the height of the
%% footer is needed at the \begin{letter} to set the pagestyle.
%% Well, there is: don't use language-dependent stuff inside a box.
%%
%\selectlanguage{english}

\begin{document}
\selectlanguage{english}
\begin{letter}{DANTE e.V.\\Postfach 101840\\69120 Heidelberg}
\invoice{123/95}
\customer{178}
%%
%% This should really go into a cfg-file
%% It's here to show you what is possible
%% 
\renewcommand*\pnumfont   {\bfseries}
\renewcommand*\headfont   {\slshape} 
\renewcommand*\reffont    {\normalfont}
\renewcommand*\reftextfont{\normalfont}
\pagestyle{headings} % to make these settings useful
% \telephone{ }
% \telefax{  }
% \email{A.Kielhorn@web.de}
% \mailbox{ }
\name{A. Kielhorn }
% \signature{\vspace{-1cm} Extra large Sig.\\ Could be a picture 
% with your \emph{real} signature.\\(Axel Kielhorn) }
\refdatename{Braunschweig,}
%%
%% If you want to use refdatename you have to set it first and
%% define yref & co later! (Yes, this is ugly, still thinking
%% about a better way)
%%
% \yref{ }
% \ymail{1.6.1994}
% \myref{ak/AK}
% \mymail{1.6.1994}
% \reftrue
% \specialmail{ Einschreiben }
\subject{LaTeX2e Letter-style}
\opening{Dear friends,}
%
this is a demonstration-letter for my new akletter-class. It is 
based on letter.cls and liletter (Stefan Lindner Letter Style, 
distributed with Atari \TeX) with some additions from 
script\_l.sty. (Now scrlettr.cls)

It supports everything needed like your-ref, your-mail, my-ref, 
my-mail, subject and of course configurable name and signature.

The new version even supports footnotes.\footnote{This is a footnote 
in letter one.}

Everything site-depending like header, footer or returnaddress 
went into a .cfg-file. The \verb|\usename| command selects a 
configuration file and enables you do have different 
information for business and privat letters. You may even set up 
a .cfg-file for your husband.

Someone showed me how to use \verb|\makelabels| and now they are 
supported. They are hardcoded to the labels I'm using, but you may 
change them in the cfg-file.

Language-dependend captions are now available in English, German, 
French and Danish. Other languages are supported using the cfg-file.

Up to version 1.5 f only A4 paper was supported. But I finally got 
some patches for letter paper and a special modification usefull for 
presentation binders where about 15 mm of the left margin are 
invisible (\texttt{a4offset}). Both formats are now supported.

New commands:

ISO 8601 defines a date format similar to the one used by the \LaTeX 
3 Team. Version 1.5 h of \texttt{akletter.cls} supports this new 
format with the \verb+\dateiso+ command. You can simply use 
\verb+\dateiso+ in the preamble and everywhere the date is used (e.g. 
using the \verb+\today+ command) the new ISO format is used. Please 
note that changing the language will reset the date format. Today is 
the {\dateiso\today} in ISO notations.

This format is recommended by the Chicago Manual of Style.

Obsolete Commands:

The \verb|\address| and \verb|\locations| commands which 
used to set \verb|\fromaddress| and \verb|\fromlocation| are no 
longer defined because they were never used. Define your address 
and location in the \verb|.cfg| file which is more flexible.

\closing{Keep \TeX ing}

\cc{de.comp.tex\\comp.text.tex}
\encl{akletter.cls\\aketter.cfg}

\ps P.S. see akletter.cfg for more information
\end{letter}

%%%%%%%%%%%%%%%%%%%%%%%%%%%%%%%%%%%%%%%%%%%%%%%%%%%%%%%%%

\selectlanguage{german}

\begin{letter}{DANTE e.V.\\Postfach 101840\\69120 Heidelberg}
\telephone{}
\email{A.Kielhorn@web.de}
\name{A. Kielhorn }
\signature{}
%%
%% this will produce an empty signature
%%
\specialmail{ Einschreiben }
% \refdatename{Braunschweig, den}
% \yref{ }
% \ymail{1.6.1994}
 \myref{ak/AK}
% \mymail{1.6.1994}
% \reftrue
\subject{LaTeX2e Letter-class}
\opening{Liebe \TeX-Gemeinde,}
%
dies ist ein Demonstrationsbrief f"ur meine neue Letter-Klasse 
f"ur \LaTeXe. Sie basiert auf \texttt{letter.cls} und 
\texttt{liletter.sty} (Stefan Lindner) sowie einigen 
Erweiterungen aus \texttt{scrlettr.cls}.

Unterst"utzt werden folgende Einstellungen:

\begin{itemize}
\item Ihre Zeichen \verb|\yref{}|
\item Ihr Schreiben vom \verb|\ymail{}|
\item Unsere Zeichen \verb|\myref{}|
\item Unser Schreiben vom \verb|\mymail{}|
\item Sachbearbeiternamen \verb|\name{}|
\item Unterschrift \verb|\signature{}|
\item Kundennummer \verb+\customer{}+
\item Rechnungsnummer \verb+\invoice{}+
\end{itemize}

Im Gegensatz zu "alteren Stilen wurde alles ver"anderliche in eine 
\verb|.cfg|-Datei ausgelagert. Somit ist ein Update der Klasse 
m"oglich ohne alles neu anpassen zu m"ussen.\footnote{Zumindest in 
der Theorie:-)} Ich rate daher von einer Ver"anderung der Datei 
\texttt{akletter.cls} ab. Begr"undete Erg"anzungs- bzw. 
"Anderungsw"unsche werden in sp"atere Versionen "ubernommen und 
stehen damit allen Anwendern zur Verf"ugung.

Es k"onnen mehrere Konfigurationdateien angelegt werden. Sie 
werden durch den Befehl \verb|\usename{}| im Vorspann 
ausgew"ahlt. (z.B. f"ur interne Schreiben, private Post etc.)

Einige Befehle der \verb|letter|-Klasse wurden entfernt, da die 
L"osung mit Konfigurationsdatein flexibler ist. Es handelt sich 
dabei um die Befehle \verb|\address| und \verb|\location|.

Da die Bundespost inzwischen nur noch Umschl"age im Format 
DIN~680 und DIN~C4 transportiert und letztere in der 
Fenster-Ausf"uhrung selten sind, habe ich auch den 
\verb|\makelabels|-Befehl "uberarbeitet. Geplant ist noch eine 
Anpassung an verschiedene Etikettenformate.

Neben Deutsch werden auch Englisch, Franz"osich und D"anisch unterst"utzt. 
Weiter Sprachen lassen sich in der Konfigurationdatei zuf"ugen.

Seit Version 1.5 h ist es auch m"oglich, auf letter-Papier zu drucken,
allerdings d"urfte es hier in Deutschland nur sehr schlecht
erh"altlich sein. Weitaus interessanter d"urfte die Option
\texttt{a4offset} sein, die das Layout nach rechts verschiebt. Dadurch
lassen sich die Briefe jetzt problemlos als Anschreiben in einer
Bewerbungsmappe verwenden. Die Verschiebung ist zur Zeit fest kodiert,
evtl. werde ich den Wert aber konfigurierbar machen.

Version 1.5 h bietet au"serdem ein neues Datumsformat nach ISO 8601
\verb+\dateiso+ das folgenderma"sen aussieht: {\dateiso\today} f"ur
den {\dategerman\today}. Da einfach der \verb+\today+ Befehl ge"andert
wird, "andert sich auch die Datumsschreibweise bei jedem
Sprachwechsel. (Diese Schreibweise wird "ubrigens in der DIN 5008
(Schreibmaschinenschreiben) von 1996 empfohlen)

\closing{Mit \TeX nischen Gr"u"sen}

\encl{akletter.cls\\akletter.cfg}

\ps P.S. Weitere Informationen in der Datei \verb|akletter.cfg|
\end{letter}

%%%%%%%%%%%

\begin{letter}{Erika Mustermann\\Um die Ecke 2\\[0.5ex] 27182 
Berlin}
\invoice{95/123}
\name{A. Kielhorn }
 \signature{\vspace{-1cm} Extra large Sig.\\ Could be a picture 
 with your \emph{real} signature.\\(Axel Kielhorn) }
\subject{Dummy Brief}
\opening{Hallo Erika,}

dies ist ein Test-Brief.

\closing{Bleib musterg"ultig}
\end{letter}

%
% Das sollte normalerweise die option erverseaddr machen!
%
\makeatletter
\@reverseaddrtrue
\makeatother

\begin{letter}{grebledieH 02196\\048101 hcaftsoP\\ .V.e ETNAD}
\telephone{}
\email{A.Kielhorn@web.de}
\name{A. Kielhorn }
\signature{}
%%
%% this will produce an empty signature
%%
\specialmail{ Einschreiben }
% \refdatename{Braunschweig, den}
% \yref{ }
% \ymail{1.6.1994}
 \myref{ak/AK}
% \mymail{1.6.1994}
% \reftrue
\subject{LaTeX2e Letter-class}
\opening{Liebe \TeX-Gemeinde,}
%
auf besonderen Wunsch einiger Anwender gibt es jetzt die M"oglichkeit, das
Anschriftenfeld auf die rechte Seite des Blattes zu verschieben. Dies
geschieht normalerweise mit der Option \verb+reverseaddr+, zur Demonstration
wurde es hier vom Hand aktiviert.

Damit das Anschriftenfeld auch in der H"ohe anpa"sbar ist, habe ich drei
neue L"angen definiert:

\begin{description}
	\item[\textbackslash addrfieldsep] Der Abstand oberhalb des
		Anschriftfeldes.
	\item[\textbackslash datefieldsep] Der Abstand zwischen Anschriftfeld und
		Datumszeile.
	\item[\textbackslash openingsep] Der Abstand zwischen der Datumszeile und
		der Gru�zeile.
\end{description}

\closing{Mit \TeX nischen Gr"u"sen}

\encl{akletter.cls\\akletter.cfg}

\ps P.S. Weitere Informationen in der Datei \verb|akletter.cfg|
\end{letter}

\end{document}
