\documentclass[11pt]{article}
\usepackage[utf8]{inputenc}
\usepackage[T1]{fontenc}
\usepackage{lmodern}
\usepackage{xkeyval,array,multirow,amsmath,amssymb}
\usepackage{fullpage,longtable}
\usepackage[]{alterqcm}
\usepackage[frenchb]{babel}

\begin{document}
\begin{alterqcm}[lq=90mm,pre=true,long]

\AQquestion{Parmi les propositions suivantes, quelle est celle qui permet d'affirmer que la fonction exponentielle  admet pour asymptote la droite d'équation $y = 0$ ?}
{{$\displaystyle\lim_{x \to +\infty} \text{e}^x = + \infty$},
{$\displaystyle\lim_{x \to -\infty} \text{e}^x = 0$},
{$\displaystyle\lim_{x \to +\infty} \dfrac{\text{e}^x}{x} = + \infty$}}

\AQquestion{Parmi les propositions suivantes, quelle est celle qui permet d'affirmer que l'inéquation $\ln (2x + 1) \geqslant \ln (x + 3)$ admet l'intervalle $\big[2~;~+\infty\big[$ comme ensemble de solution ? }
{{\begin{minipage}{5cm}la fonction ln est positive sur $\big[1~;~+\infty\big[$\end{minipage}},
{$\displaystyle\lim_{x \to +\infty} \ln x = + \infty$},
{\begin{minipage}{5cm}la fonction $\ln$ est croissante sur $\big]0~;~+\infty\big[$\end{minipage}}
}

\AQquestion{Parmi les propositions suivantes quelle est celle qui permet d'affirmer qu'une primitive de la fonction $f$ définie sur $\mathbb{R}$ par $x \mapsto (x + 1)\text{e}^x$ est la fonction $g~:~x~ \mapsto~ x~ \text{e}^x$~? }
{{pour tout réel $x,~f'(x) = g(x)$},
{pour tout réel $x,~g'(x) = f(x)$},
{\begin{minipage}{5.5cm} pour tout réel $x,~g(x) = f'(x) + k$, $k$ réel quelconque \end{minipage}}}

\AQquestion[pq=2mm]{ L'équation $2\text{e}^{2x} - 3\text{e}^x + 1 = 0$ admet pour ensemble solution}
{{$\left\{\dfrac{1}{2}~;~1\right\}$},
{$\left\{0~;~\ln \dfrac{1}{2}\right\}$},
{$\big\{0~;~\ln 2\big\}$}
}

\AQquestion[pq=2mm]{Pour tout $n \in \mathbb{N}$ }
{{$\displaystyle\lim_{x \to +\infty} \frac{\text{e}^x}{x^n} = 1$},
{$\displaystyle\lim_{x \to +\infty} \frac{\text{e}^x}{x^n} = +\infty$},
{$\displaystyle\lim_{x \to +\infty} \frac{\text{e}^x}{x^n} = 0$}}

\AQquestion[pq=1pt]{Soit $f$ la fonction définie sur $\big]0~;~+\infty\big[$ par $f(x) = 2\ln x - 3x + 4$. Dans un repère, une équation de la tangente à la courbe représentative de $f$ au point d'abscisse 1 est :}
{{$y = - x + 2$},
{$y = x + 2$},
{$y = - x - 2$}
}

\AQquestion[pq=2mm]{La valeur moyenne sur $\big[1 ; 3\big]$ de la fonction $f$ définie par : $f(x) = x^2 + 2x$ est :}
{{$\dfrac{50}{3}$},
{$\dfrac{25}{3}$},
{$6$}
}
\AQquestion{ exp$(\ln x) = x$ pour tout $x$ appartenant à }
{{$\mathbb{R}$},
{$\big]0~;~+ \infty\big[$},
{$\big[0~;~+\infty\big[$}
}
\AQquestion[pq=1pt]{Soit $f$ la fonction définie sur $\big]0~;~+\infty\big[$ par $f(x) = 2\ln x - 3x + 4$. Dans un repère, une équation de la tangente à la courbe représentative de $f$ au point d'abscisse 1 est :}
{{$y = - x + 2$},
{$y = x + 2$},
{$y = - x - 2$}
}

\AQquestion[pq=2mm]{La valeur moyenne sur $\big[1 ; 3\big]$ de la fonction $f$ définie par : $f(x) = x^2 + 2x$ est :}
{{$\dfrac{50}{3}$},
{$\dfrac{25}{3}$},
{$6$}
}
\AQquestion{ exp$(\ln x) = x$ pour tout $x$ appartenant à }
{{$\mathbb{R}$},
{$\big]0~;~+ \infty\big[$},
{$\big[0~;~+\infty\big[$}
}
\end{alterqcm}
\end{document}

% AntillesESjuin2006

% encoding : utf8
% format   : pdflatex
% engine   : pdfetex
% author   : Alain Matthes