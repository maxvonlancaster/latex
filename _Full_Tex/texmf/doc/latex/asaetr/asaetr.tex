% asaetr.tex v1.0 01 Jan 92
\documentstyle{asaetr}
\title{\LaTeX\ and  
       B\kern-.05em{\large I}\kern-.025em{\large B}\kern-.08em\TeX\ 
       for ASAE Papers\thanks{Written for presentation as the 1992 
               International WinterMeeting of ASAE.}
      }
\author{J.~D.~McCauley\thanks{USDA Fellow, Department of Agricultural
       Engineering, Purdue University. Formerly, 
       Graduate Assistant--Research, Department of Agricultural
       Engineering, Texas A\&M University. ({\tt jdm5548@diamond.tamu.edu}).}
       \student
       %\and
       %A.~D.~Whittaker \member
       }
       % the format is: name \membership_grade, where membership_grade
       % is one of ( \member, \associate, \student, \affiliate, \fellow) 
\begin{document}
\bibliographystyle{asaetr}
\maketitle

\begin{abstract}
A \LaTeX\ style for the American Society for Agricultural Engineers
(ASAE) is discussed. This is not official and not (yet) an accepted
way to submit articles to ASAE.  {\em This is NOT an ASAE Transcations
article.} \keywords{\LaTeX,\ \BibTeX,\ typesetting, ASAE Transactions.}
\end{abstract}

\section{Introduction}
\drop{T}his document, when compared to source code,
instructs someone on the use of {\tt asaetr.sty}, a \LaTeX\ style file
for writing documents that look like {\it ASAE Transactions} articles.
Examples of section headers, itemized lists, tables, and figures are
given. Hopefully, a document written using the default {\tt article}
style in \LaTeX\ could be transformed to look like a {\em
Transactions} article by only a few changes.

The American Society of Agricultural Engineers editorial staff
encourages authors to submit electronic manuscripts in the following
formats: MacWrite, MS--Word, MS--Works (preferably Version 2.0),
WordPerfect (Version 5.0 or later), and WriteNow \cite{asaeins} (ASAE,
1991).  Though \LaTeX\ is not an accepted format to submit articles to
ASAE, it does have utility when predicting page length and appearance
of an article that you submit. Predicting page length is especially
important when submitting ``Technical Notes'' (which cannot exceed two
pages) and when trying to avoid unnecessary page charges.

It should be emphasized that this document was written as an example
of use of the style files; it's value is greatly enhanced if you
compare it with source code (see a later section for instructions) and
a copy of {\em ASAE Transactions}.

\subsection{Objective}

The objective of this work was to develop \LaTeX\ and \BibTeX\
style files for ASAE members.

\subsubsection{Finer Objectives}
Well, that sounds pretty noble, but I also wanted to
\begin{enumerate}
\item Make it easier on myself (because I'm a programmer, and programmers
are lazy),
\item Encourage a friend to switch to \TeX\, and
\item Show you the use of a  \verb#\subsubsection# heading and the 
{\tt enumerate} environment.
\end{enumerate}

\paragraph{Test Heading} 
This is a test of level four headings.

\section{Getting Started}

If you're unfamilar with \LaTeX, I would suggest picking up a copy of
the manual (Lamport, 1986) \cite[note]{ll:86} and putting this paper
aside for a while. If you're already familiar, read on.

\subsection{The Preamble}
The preamble is where tell \LaTeX\ that you are going to use {\tt
asaetr.sty}.  It's also where you list the authors and ASAE membership
grades. Here's an example:
\small \begin{verbatim}
\documentstyle{asaetr}
\title{Boring Title}
\author{U.\ B.\ Boring \fellow \and 
       I.\ M.\ Young \student \and
       R.\ U.\ Happy \nonmember   }
\begin{document}
\maketitle
\end{verbatim} \normalsize
I have used up to four authors and still got it fit on one line.  Five
authors may fit, depending on the lengths of the names. If they don't
all fit, two rows of authors will be formed. Membership grades can be
any of
\small \begin{verbatim}
 \member, \associate, \student, 
 \affiliate, or \fellow. 
\end{verbatim} \normalsize
\noindent
You can also use \verb#\nonmember#, but it has the same effect
as leaving the membership grade off. The \verb#\maketitle# command
simply tells \LaTeX\ to use this author and title information to
compose the title of the paper.

\subsection{The Abstract}

After the preamble comes the abstract. Here's an example:
\small\begin{verbatim}
\begin{abstract}
This is going to be short. See, I told you.
\keywords{brevity, terseness, words.}
\end{abstract}
\end{verbatim}\normalsize
This should be straightforward enough.

\subsection{The Body}
The commands that you should be most familiar with to typeset the
body of your paper are the sectioning commands. They are
\begin{description}
\item[section:] Same  level as the Introduction.
\item[subsection:] Secondary headings, such as 
``Objectives.'' 
\item[subsubsection:] Third level headings.
\item[paragraph:] Avoid fourth level headings.
\end{description}
The usage of these commands can be best described by an example:
\begin{verbatim}
\subsection{The Body}
\end{verbatim}
This is the sectioning command for the section you are now reading.  

You may want to avoid fourth level headings because it requires an
additional font for your \TeX\ installation: {\tt cmcscsl10}.  This is
a slanted, small caps font. If you have \MF\ working on your system,
this is not a problem.  Source code for this font is available with
the style files. Ask you local \TeX\ guru for help, should you need
it.

\subsection{Figures and Tables}

If you have a \PS\ printer available, it's highly recommended that you
use the \verb# \psfig# macros written by Trevor Darrell to include
high quality figures. Another useful utility for including figures is
{\tt fig} (or {\tt xfig} if you use X Windows).  Figure 1 was created
in about 30 seconds using {\tt xfig}. You can get {\tt fig} from {\tt
cayuga.cs.rochester.edu} by anonymous ftp. Remember that in {\em
Transactions of the ASAE}, captions for figures go {\em below} the
figures.

\renewcommand{\footnoterule}{} % no line
\begin{table}[hbp]
\footnotesize
\caption{Comparison of Publishing Tools}
\begin{center}
\begin{minipage}{\columnwidth}
\renewcommand{\footnoterule}{} % no line
\begin{center}
\renewcommand{\thefootnote}{\fnsym{footnote}}
\begin{tabular}{lrrrrrr}  \thickhline  
Tool & \multicolumn{3}{c}{Learning Curve\footnote{1.0 being easiest.} 
  }& \multicolumn{3}{c}{Support\footnote{10.0 being the best.}}\\ 
& \multicolumn{3}{c}{($units$)} & \multicolumn{3}{c}{($units$)}\\ \thinhline 
FrameMaker && 5.0 && 6.0 \\  
Troff && 10.0 && 1.0 \\
\TeX \footnote{\TeX\ is the winner!} && 7.0 && 10.0  \\ \thickhline
\end{tabular}
\linethickness{0pt}
\end{center}
\end{minipage}
\end{center}
\end{table}

\begin{figure}[htb]
  \setlength{\unitlength}{0.1mm} %{0.00625in}%{0.0125in}%
  \begin{center}
    \begin{picture}(181,181)(0,0)
      \thinlines \multiput(80,80)(-20,-20){4}{\framebox(80,80){}}
      \thicklines \put(0,0){\framebox(180,180){}} 
      \put(60,60){\line( 1, 1){ 60}}
    \end{picture}
  \end{center}
  \caption{Primitive figure.} 
\end{figure}

If you plan to include tables, and if you want to have footnotes
within these tables, use the {\tt minipage} environment.  Contact
your local \LaTeX\ guru or see your {\em local guide} for more
information about how to code tables. The source code for this paper
also provides good information.  You'll notice in {\em Transactions of
the ASAE} or in {\em Applied Engineering in Agriculture} they use
thicker lines for the top and bottom rules in tables. Instead of
having to change line thickness yourself (as you do in using document
style ``article'' and others), you can use two macros that come with
this style: \verb#\thickhline# and \verb#\thinhline#.  See the example
table in this document (Table 1).  Don't forget to put the caption
{\em above} the table instead of below it.

\subsection{The References}

\BibTeX\ automatically generates the ``References'' section of your
paper from an external database.  Style files govern to appearance of
your ``References'' section.  In principle, you could change a paper,
which met the requirements of one professional society, to that of
another by simply changing the style file that you use.  The style
file {\tt asaetr.bst} (for ASAE) is currently being developed.  Other
style files exist for IEEE, ACM, APA, etc.

To use \BibTeX, you normally process your file with \LaTeX, then with
\BibTeX, then twice more with \LaTeX. The \BibTeX\ style file, which 
is unfinished, comes close to the citation style used by ASAE. You may
have to edit some entries by hand. To do this, edit the {\tt *.bbl}
file after you have processed the file with \BibTeX.  See Appendix B
in \cite{ll:86} (Lamport, 1986) for more information about \BibTeX.

\subsection{Submission}
When you submit to ASAE, they want a double--spaced, single--column
document with figures and tables on separate pages.  If you've written
you document using

\noindent{\verb#\documentstyle{asaetr}#,}\par
\noindent simply change that first line to read\par
\noindent{\verb#\documentstyle[asaesubmit]{asaetr}#.}\par
\noindent This makes all the changes necessary.

\section{Where to Find}
This collection of files discussed in this paper include:
\begin{description}
\item[asaetr.sty:] \LaTeX\ style 
\item[asaesubmit.sty:] for paper submissions to ASAE
\item[asaetr.bst:] \BibTeX\ style 
\item[asaetr.tex:] example usage of and documentation for above (source for
           this document)
\item[asaetr.bib:] part of the above example
\item[cmcscsl10.mf:] \MF\ source for level four heading font
\end{description}

A copy of this collection of style files can be obtained via
anonymous ftp:
\small \begin{verbatim}
% ftp sun.soe.clarkson.edu
Connected to sun.soe.clarkson.edu.
Name (sun.soe.clarkson.edu:jdm5548): anonymous
331 Guest login ok, send ident as password.
Password: jdm5548@diamond.tamu.edu
230 Guest login ok, access restrictions apply.
ftp> cd pub/tex/latex-style
250 CWD command successful.
ftp> mget asae*
...
ftp> quit
\end{verbatim}\normalsize
\noindent
To retrieve this collection via electronic mail, send a MAIL message to

\centerline{\tt FILESERV@SHSU.BITNET}
\noindent with the command:

\centerline{\tt SENDME ASAETR}

\section{Conclusion}

This file should serve as an excellent example of the use of the style
files.  If you still can't figure things out, hunt up your local guru
and ask him/her to explain \LaTeX\ and \BibTeX\ style files.

\bibliography{asaetr}
\end{document}
