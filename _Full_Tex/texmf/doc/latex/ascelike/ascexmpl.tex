%
\documentclass[Proceedings]{ascelike}
%
% NOTE: Don't include the \NameTag{<your name>} if you have selected 
%       the NoPageNumbers option: this leads to an inconsistency.
\NameTag{Kuhn}
%
\begin{document}
%
% You will need to make the title all-caps
\title{STYLE FILES FOR ASCE-LIKE DOCUMENTS}
%
\author{
Matthew R. Kuhn%
\thanks{
Dept. of Civ. and Env. Engrg.,
School of Engrg., Univ. of Portland, 5000 N. Willamette Blvd.,
Portland, OR  97203. E-mail: kuhn@up.edu.},
\ Member, ASCE
%
% I've found that the \and command doesn't quite work, so just use "and"
%  such as the following (and don't forget the ending curly brace `}').
%\\
%and
%Younyee Kuhn%
%\thanks{Flourishing wife of same.},%
%\ Not a Member, ASCE
}
%
\maketitle
%
\begin{abstract}
This document was produced with the \LaTeX\ typsetting program
using the document class ``\texttt{ascelike.cls}'' and the
example document ``\texttt{ascexmpl.tex}''.
The objective is manuscripts that roughly comply with the
guidelines of the American Society of Civil Engineers.
The document class
produces either double-spaced manuscripts for journal submissions or
camera-ready manuscripts
for conference proceedings.
This document serves as a brief guide to \texttt{ascelike.cls},
as well as a test of the output that is produced
by the input file \texttt{ascexmpl.tex}.
The package is freely available under the LaTeX 
Project Public License, version 1.1
\end{abstract}
%
% Some keywords, using a new command: \KeyWords{}
%
\KeyWords{\LaTeX, ASCE, document class.}
%
\section{Introduction}
The experimental document class ``\texttt{ascelike.cls}''
produces manuscripts that roughly comply with the
guidelines of the American Society of Civil Engineers.
However, it was \emph{not} produced by ASCE, its agents,
or employees;
nor is it in any way sanctioned or approved by that organization.
The program \texttt{ascelike.cls} is distributed under the
terms of the LaTeX Project Public License Distributed,
available from the CTAN archives;
either version 1.1 of the License, or any later version.
If you modify
\texttt{ascelike.cls}, you should rename it so that ``contaminated''
copies are not later disseminated.
\par
The document class ``\texttt{ascelike.cls}'' requires the following
supplementary packages:
\texttt{ifthen.sty}, \texttt{setspace.sty}, and \texttt{endfloat.sty}
\emph{Without these files,} \texttt{ascelike.cls} \emph{won't work}.
These files are typically included in \LaTeX\ distribution, such as the
\textsf{tetex} and \textsf{MikTex} distributions.
All three files are also freely available
from the Comprehensive \TeX\ Archive Network (CTAN) archive,
through \verb+//http:www.tug.org+, but they may need to be unbundled from
a \verb+*.dtx+ form.
In addition, the file \texttt{ascelike.bst} can be used with
the tool \textsc{Bib}\TeX\ to produce ASCE-like 
reference citations and entries (with the weird use of
quotation marks around titles, etc.).
An example bibliographic data base is given in \texttt{ascexmpl.bib}.
\par
In addition to these important files,
we have found the following packages most useful (and we'll use this
opportunity to illustrate a \LaTeX\ itemized list):
%
\begin{itemize}
\item
\texttt{epsfig.sty} and its companion files for incorporating
encapsulated post\-script (figure) files into the document
\item
\texttt{subfigure.sty} for arranging and numbering subfigures
\item
\texttt{amstex.sty} and its companion files for the AMS extensions
to mathematical formatting.
\end{itemize}
%
All of these packages are also freely 
available from the CTAN archive, 
but they are included in most \LaTeX\ distributions.
%
\section{Input and Options}
Prepare your \verb+*.tex+ input file as a regular
\LaTeX\ file using the standard \texttt{article.cls} constructs,
but, of course, substitute \texttt{ascelike} for \texttt{article}
as the document class.
You will, however, likely need to specify a number of options.
In addition, I've provided two new commands: \verb+KeyWords+ and
\verb+NameTag+, both of which are described further below.
Document class \texttt{ascelike.cls} has a the options given
below (and we'll also use this opportunity to illustrate an enumerated list).
The \verb+Proceedings|+\-\verb+Journal+ options are the most important;
the other options are largely incidental.
%
\begin{enumerate}
\item
\verb+Journal|+\verb+Proceedings+ specify the overall format
of the output man\-u\-script.  
\par
\texttt{Journal} produces double-spaced manuscripts for ASCE journals.
It places tables and figures at the end of the manuscript, 
and produces lists of tables and figures.  
It numbers the appendices with Roman numerals and produces 
proper headings for
sections, subsections, subsubsections, appendices, and abstract.
It produces the proper page margins and numbers the pages.
\par
\texttt{Proceedings} produces camera-ready single-spaced manuscripts
for ASCE conference proceedings.  
It produces the proper page margins as
given on the old shiny, 
camera-ready paper (with the light blue lines) 
supplied by ASCE. 
It places figures and tables
within the text.  It produces proper headings for
sections, subsections, subsubsections, appendices, and the abstract.
Pages are numbered, and the bottom left corner can be ``tagged'' with
the author's name (this can be done by inserting the command
\verb+\NameTag{<+\emph{your name}\verb+>}+ within the preamble of your
document.
\item
\verb+BackFigs|InsideFigs+ can be used to override 
the default placement of tables
and figures in the \texttt{Journal} and \texttt{Proceedings} formats.
\item
\verb+SingleSpace|DoubleSpace+ can be used to override 
the default text spacing in the 
\texttt{Journal} and \texttt{Proceedings} formats.
\item
\verb+10pt|11pt|12pt+ can be used to override the 
default text size (12pt).
\item
\texttt{NoLists} suppresses the inclusion of the lists of tables
and figures that would normally be included in the \texttt{Journal}
format.
\item
\texttt{NoPageNumbers} suppresses the printing of page numbers.
\end{enumerate}
%
\section{Sections, subsections, equations, etc.}
I've included this section to test the formating of sections, subsections,
subsubsections, equations, tables, and figures.
Section heads are automatically made uppercase, which is great unless
your section heading contains mathematics, \verb+$<math stuff>$+.
If your head does contain mathematics, you will need to modify
\texttt{ascelike.cls}, in particular the line containing the
\verb+\uppercase+ command.
%
\subsection{An Example Subsection}
No automatic capitalization occurs with subsection headings; 
you'll need to capitalize the first letter of each word,
as in ``An Example Subsection.''
%
\subsubsection{An example subsubsection}
No automatic capitalization occurs with subsubsections; 
you'll need to capitalize only the first letter of the subsubsection heading.
And now we include an example of a displayed equation (Eq.~\ref{eq:Einstein})
%
\begin{equation} \label{eq:Einstein}
E = m c^{2} \;,
\end{equation}
%
a figure (Fig.~\ref{fig:box_fig}), 
%
\begin{figure}
\centering
\framebox[3.00in]{\rule[0in]{0in}{1.00in}}
\caption{An example figure (just a box).  
This particular figure has a caption with more information 
than the figure itself, a very poor practice indeed.}
\label{fig:box_fig}
\end{figure}
%
and a table (Table~\ref{table:assembly}).
%
\begin{table}
\caption{An example table}
\label{table:assembly}
\centering
\small
\renewcommand{\arraystretch}{1.25}
\begin{tabular}{l | l}
\hline\hline
\multicolumn{1}{c|}{Assembly Attribute} &
\multicolumn{1}{c}{Values} \\
\multicolumn{1}{c|}{(1)} &
\multicolumn{1}{c}{(2)} \\
\hline
Number of particles & 4008 \\
Particle sizes & Multiple  \\
Particle size range & $0.45D_{50}^{\:\ast}$ to $1.40D_{50}$ \\
Initial void ratio, $e_{\mathrm{init}}$ & $0.179$ \\
Assembly size & $54D_{50} \times 54D_{50} \times 54D_{50}$ \\
\hline
\multicolumn{2}{l}{$\ast$ $D_{50}$ represents the median particle diameter} \\
\hline\hline
\end{tabular}
\normalsize
\end{table}
%
\par
I've added a new command \verb+\KeyWords{<your key words>}+ for
a labeled list of key words.  
It can be placed anywhere in the document and produces an unindented
paragraph.
%
\section{Citations and bibliographic entries}
When used together, \texttt{ascelike.cls} and \texttt{ascelike.bst}
produce APA~/ \emph{Chica\-go Manual of Style} citations in
name-date format.
The code for this format is a modification of the \texttt{chicago.sty} and
\texttt{chicago.bst} packages.
I've made available the following citation options:
\begin{itemize}
\item
\verb+\cite{key}+ produces citations with full author 
list and year \cite{Ireland:1954a}.
\item
\verb+\citeNP{key}+ produces citations with full author list and year, 
but without enclosing parentheses: eg. \citeNP{Ireland:1954a}.
\item
\verb+\citeA{key}+ produces citations with only the full 
author list: e.g. \citeA{Ireland:1954a}
\item
\verb+\citeN{key}+ produces citations with the full author list and year, but
which can be used as nouns in a sentence; no parentheses appear around
the author names, but only around the year: eg. \citeN{Ireland:1954a}
states that \ldots
\item
\verb+\citeyear{key}+ produces the year information only, within parentheses,
as in \citeyear{Ireland:1954a}.
\item
\verb+\citeyearNP{key}+ produces the year information only,
as in \citeyearNP{Ireland:1954a}.
\end{itemize}
%
\par
The example bibliographic data base \texttt{ascexmpl.bib}
gives examples of bibliographic entries for different document types.
These entries are from the canonical set in the
ASCE web document ``Instructions For Preparation Of Electronic Manuscripts'':
an anonymous book \cite{Moody:1988a}, 
an anonymous report \cite{FHWA:1991a}, 
an anonymous newspaper story ("Educators" 1993), 
an ASCE journal paper \cite{Pennoni:1992a}, 
a book with editors \cite{Zadeh:1981a}, 
a building code \cite{ICBO:1988a}, 
a discussion \cite{Vesilind:1992a}, 
a doctoral thesis \cite{Chang:1987a}, 
a paper in a foreign journal \cite{Ireland:1954a}, 
a paper in a proceedings \cite{Eshenaur:1991a}, 
a standard \cite{ASTM:1991a}, 
a translated book \cite{Melan:1913a}, 
a two-part paper \cite{Frater:1992a,Frater:1992b}, 
a university report \cite{Duan:1990a}, 
an untitled item in the Federal Register \cite{FR:1968a}, 
works in a foreign language \cite{Duvant:1972a,Reiffenstuhl:1982a},
and software \cite{Lotus:1985a}.  
%
\section{Miscellany}
Many ASCE conference proceedings are now published on CD ROM media.
I've noticed that instructions on paper formats issued by
conference organizers often differ from the
standard ASCE instructions.  
Fortunately most of the differences can be easily accomodated, such as
changes in the margins and placement of the authors' addresses.
As for margins, these can, of course, be altered by using
\verb+\setlength{<length>}+ commands within the preamble to a document without
making any changes to \texttt{ascelike.cls}.  
(See the \LaTeX\ book
\cite{Lamport:1994a}, its companion \cite{Goossens:1994a}, or
online web documentation.)
Authors' addresses can be placed below the title (instead of
in a footnote) by \emph{not} using the \verb+\thanks+ command.
%
%
% Now we start the appendices, with the new section name, "Appendix", and a 
%  new counter, "I", "II", etc.
%
\appendix
%
% Here's the first appendix, the list of references:
%
\bibliography{ascexmpl}
%
% And now for some pretty impressive notation.  In this example, I have used
%   the tabular environment to line up the columns in ASCE style.
%   Note that this and all appendices (except the references) start with 
%   the \section command
%
\section{Notation}
\emph{The following symbols are used in this paper:}%\par\vspace{0.10in}
\nopagebreak
\par
\begin{tabular}{r  @{\hspace{1em}=\hspace{1em}}  l}
$D$                    & pile diameter (m); \\
$R$                    & distance (m);      and\\
$C_{\mathrm{Oh\;no!}}$ & fudge factor.
\end{tabular}
%
\end{document}
