\documentclass[twocolumn]{ltxguide}
\usepackage[cp1252]{inputenc}

\raggedbottom

\usepackage{accents}
\usepackage{tensind}

\tensordelimiter{?}
\tensorformat{lrb}

\usepackage{dotlessi}

\title{The \textsf{Bezos} bundle of packages}
\author{Javier Bezos\\\normalsize\texttt{www.tex-tipografia.com}}
\date{2009-06-12}

\newenvironment{sample}{\small\begin{tabbing}%
  \hskip14pc\=\hskip6pc\\}
  {\end{tabbing}}

\renewenvironment{decl}%
    {\par\small
     \addvspace{4.5ex plus 1ex}%
     \vskip-\parskip
     \noindent
     \begin{tabular}{\string|l\string|}\hline\ignorespaces}%
    {\\\hline\end{tabular}\par\nobreak
     \vspace{2.3ex}%
     \vskip-\parskip}

\usepackage{mathptmx}

\begin{document}
\maketitle

This document expains briefly the packages written by me and grouped
under the Bezos directory in CTAN. They are in some sense small
packages and this is the very reason they have been grouped here.
Large packages like \textsf{gloss} (written with Jos\'e Luis D\'\i
az), \textsf{titlesec} (including \textsf{titletoc}) and
\textsf{enumitem} are under their own directories.

\section{Accents}

This package provides some miscellaneous tools for mathematical 
accents, with the following goals in mind:
\begin{itemize}
\item Creating faked accents from non-accent symbols, like
$\accentset{\star}{s}$.
\item Grouping them, perhaps including actual accents, like 
$\hat{\accentset{\star}{\hat h}}$. That
has the side effect that multiple actual accents can be built
as well.
\item Putting them below the main symbol instead of above.
\end{itemize}

That is done so that the faked accents behave in quite similar fashion 
to actual accents do; i.e., the skew of the letter is taken into 
account (compare $\accentset{\ast}{d}$ with 
$\accentset{\ast}{h}$) and the sub and superscripts attached to 
it aren't misplaced (look carefully at $\accentset{\star}{f}_1^1$).

The two basic commands are the following:
\begin{decl}
|\accentset{<accent>}{<symbol>}|
\end{decl}
Builds a faked accent, as for instance
\begin{sample}
|\accentset{\star}{d}|    \> $\accentset{\star}{d}$\\
|\accentset{\diamond}{h}| \> $\accentset{\diamond}{h}$\\
|\tilde{\accentset{\circ}{\phi}}| \>
    $\tilde{\accentset{\circ}{\phi}}$
\end{sample}

\begin{decl}
|\underaccent{<accent>}{<symbol>}|
\end{decl}

This command puts the |<accent>| under the |<symbol>|. Both real
and faked accents are allowed. For instance,
\begin{sample}
|\underaccent{\hat}{x}|      \> $\underaccent{\hat}{x}$\\
|\underaccent{\bar}{\gamma}| \> $\underaccent{\bar}{\gamma}$\\
|\underaccent{\triangleright}{q}|  \>
   $\underaccent{\triangleright}{q}$\\
|\underaccent{\tilde}{\mathcal{A}}| \>
    $\underaccent{\tilde}{\mathcal{A}}$
\end{sample}

Typeset \texttt{accents.tex} for further details.

\section{Esindex}

This package defines the command \verb|\esindex| which
easies writing Spanish index entries:
\begin{verbatim}
\esindex{ca��n}
\end{verbatim}
is equivalent to
\begin{verbatim}
\index{can^^ffon@ca��n}
\end{verbatim}
This way, entries with diacritical marks are properly
sorted. It has options to ignore spaces (letter-by-letter) and 
particles, and to predefine sort keys.

The manual is \texttt{esindex.tex} and is in Spanish.

\section{Tensind}

This package provides typographically fine tensorial notation, with
the following features:
\begin{itemize}
\item Dots filling gaps. 

\item Symbol subtitution to easy typing (if you are using greek
letters, for example).

\item Corrected position of indexes: horizontally, to compensate the
small displacement in letters like $f$ (look carefully at $f_*^*$) and
vertically, to avoid superscripts too raised.

\item Aditional minute corrections are also allowed.
\end{itemize}

After setting a \textit{tensor delimiter} with \verb|\tensordelimiter|
you can write tensorial indexes in a very compact way.  In subsequent
examples we will assume
\begin{verbatim}
\tensordelimiter{?}
\end{verbatim}
and every instance of |?| will actually mean |<tensor-delim>|.

Tha basic commands are the following:

\begin{decl}
|?[<format>]{<nucleous>}{<special-index>}...|\\
|                       <super-or-sub>...?|
\end{decl}

Creates a tensor. A few examples will be illustrative:
\begin{sample}
|?R_ij^kl_\alpha^\beta?|     \>        $?R_ij^kl_\alpha^\beta?$\\[2pt]
|?R^ij_kl^\alpha_\beta?|     \>        $?R^ij_kl^\alpha_\beta?$\\[2pt]
|?R**_ij^kl_\alpha^\beta?|   \>        $?R**_ij^kl_\alpha^\beta?$\\[2pt] 
|?R**^ij_kl^\alpha_\beta?|   \>        $?R**^ij_kl^\alpha_\beta?$
\end{sample}

The following letters may be used in format:
\begin{itemize}
\item[\texttt{l}] Gaps to the left of the last index are filled with dots.
\item[\texttt{r}] Gaps to the right of the first index are filled.
\item[\texttt{e}] If there is no index (empty), gaps are filled.
\item[\texttt{b}] Only gaps in subscripts are filled.
\end{itemize}
(There are further options to set the vertical spacing.)
\begin{sample}
|?[]f\prime_ij^kl?|    \>  $?[]f\prime_ij^kl?$\\[2pt]
|?[e]f^ij?|            \>  $?[e]f^ij?$ \\[2pt] 
|?[l]f*_ij^kl?|        \>  $?[l]f*_ij^kl?$\\[2pt]
\end{sample}

Please, typeset \verb|tensind.tex| for further datails.

\section{Subdocs}

This is a tentative package aiming to provide a different approach for
multidocument works (mainly, books with a document per chapter).
Unlike the \verb|\include| mechanism, every subdocument is a complete
normal \LaTeX{} document and can be typeset separately.  What the
package does is sharing the \texttt{aux} files.  It's still an alpha version,
and no attempt has been done yet to make it to work with, say,
\textsf{hyperref}.

This approach has the following advantages: 1) packages can be loaded
just when a part requires them;  2) generated pdf/dvi/ps files are
kept.

You need a master tex file containing a \verb|\subdocuments| command
listing the files.  With an optional argument you can set its
behaviour:
\begin{sample}
|\subdocuments[clear=double]{file1, file2, file3}|
\end{sample}

The option \texttt{clear=double} adds an empty page at the end to give
a even number of pages.  \texttt{clear=single} doesn't and it's the
default.  Spaces and new lines before and after commas are ignored, as
well as empty items.  It relies on the \textsf{keyval} package, so
names containing \texttt{=} must be given within braces.  This file
can be used as a shared preample, with common settings for all the
documents.

Documents must have the folowing line just after \verb|\documentclass|:
\begin{sample}
|\usepackage[master]{subdocs}|
\end{sample}
where \texttt{master} is the name of the file described above.

Since every subdocument reads all the aux files, \verb|\tableofcontents|
may be used where desired. 


\section{Dotlessi}

This package defines two commands: \verb|\dotlessi| and
\verb|\dotlessj|, which give you access to dotless i's and j's in math
mode.  They are intended for symbols in non English languages.  If the
default math font is used, they are equivalent to \verb|\imath| and
\verb|\jmath|; if a different font is used (like \verb|\mathrm|,
\verb|\mathbf|, \verb|\mathsf|) these commands look in the
corresponding font tables to find their codes.  In text mode, they
expand to \verb|\i| and \verb|\j|.  For example:
\begin{sample}
|f_{\mathrm{m\acute{\dotlessi}n}}|   \>  
$f_{\mathrm{m\acute{\dotlessi}n}}$
\end{sample}


\section{Checkend}

When an environment is left open, \LaTeX{} gives an error at the end
of the document.  However it only informs about the first of them,
while the rest are shown with meaningless errors:
\begin{verbatim}
(\end occurred inside a group at level N)
\end{verbatim}

This package replaces them with more useful messages
which show which environments (in reverse order) were
not closed.

That's all. There are no user macros. Just load the package.

\section{Arabicfront}

This package modifies the definitions of \verb|\frontmatter| and
\verb|\mainmatter| so that  page numbering starts in arabic style from the
front matter while preserving the rest of the original definitions.
For it to work, \verb|\pagenumbering| has to be inside these macros --
most of classes do that, but there are exceptions like
\textsf{memoir}.

To use it, just \verb|\usepackage{arabicfront}|.

% \section{Esbst}
% 
% This file was formerly included in the Spanish style for
% babel but from version 5.0 on (not yet released) it will
% be a separate package.

\end{document}