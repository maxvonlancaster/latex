%%
%%
%%  Dies ist die Dokumentation zu bibarts 1.3.
%%       (c) Timo Baumann  <28.Mar.1998>
%%
%%
%%
%%   siehe unten:  BEARBEITUNG DIESES TEXTES
%%
%%
%%
%%  ENGLISH ABSTRACT:
%%  =================
%%
%%  bibarts helps you to make a bibliography (literature, sources & index).
%%  - bibarts switches from english to german, if \newumlaut is defined
%%    (in german.sty, \newumlaut is defined).
%%  - bibarts switches from LaTeX 2.09 to LaTeX 2e, if \LaTeX2e is defined
%%    (see definition of \@footnotetext in bibarts.sty).
%%  - If you are no DOS user, you have to compile gbibsort.c using your own
%%    compiler, or, you have to use your own sorting program. If you do not
%%    use gbibsort, you may not use commands \sort, \male, nor \female.
%%  - bibarts.sty defines the following commands:
%%     1  \makebar: Prepares bibliography FILE.bar.
%%        You have to perform the following steps:
%%          1.1 latex FILE.tex
%%          1.2 gbibsort FILE.bar FILE.phy
%%          1.3 makeind[e]x FILE.bar
%%          1.4 latex FILE.tex
%%          1.5 latex FILE.tex
%%        FILE.bar contains three kinds of items: literature, sources, and
%%        index. They are created by commands 2, 3, 4, 5, 9, 10 and 12.
%%     2  \vli{Donald}{Knuth}{The TeX-Book, Reading (Mass.) 1984}
%%        - prints "Donald Knuth: The TeX-Book, Reading (Mass.) 1984".
%%        - writes "Knuth, Donald: The TeX-Book, Reading (Mass.) 1984"
%%          into list of literature. "Knuth" will be underlined.
%%     3  \vqu does the same, but writes into list of sources.
%%     4  \addtovli{a}{b}{c}: Writes "b, a: c" into list of literature.
%%     5  \addtovqu{a}{b}{c}: Writes "b, a: c" into list of sources.
%%     6  \printonlyvli{a}{b}{c}: Prints "a b: c". (2 <=> 4 + 6)
%%     7  \printonlyvqu{a}{b}{c}: Prints "a b: c". (3 <=> 5 + 7)
%%     8  \sort{A} in \vli's, \vqu's, \addtovli's or \addtovqu's arguments
%%        will be sorted like if there is an A, but A will not be printed.
%%     9  \bib{HINT}: Prints HINT and writes HINT to index, indexing page
%%        number and, if \bib is in a footnote, an exponent after the page
%%        number will also show the number of this footnote.
%%    10  \addtobib{HINT} writes HINT to index. (Do not use \sort{A}, but A@)
%%    11  \printonlybib{HINT} prints HINT.         (9 <=> 10 + 11)
%%    12  \kurz{SHORT} prints "(in the following SHORT)" and behaves like
%%        \addtobib{SHORT}, too. If it is in \vli's, \vqu's, \addtovli's or
%%        \addtovqu's arguments, it writes "[SHORT]" into the bibliography.
%%    13  \printvli: Prints list of literature (extracting FILE.phy).
%%    14  \printvqu: Prints list of sources (extracting FILE.phy).
%%    15  \printnumvli: Prints list of literature, but adds to every item
%%        the page (and the footnote) number from which the item came from:
%%        "TITLE >> PAGENO" or "TITLE >> PAGENO^FOOTNOTENO"
%%        (^FOOTNOTENO, if \vli or \addtovli have been in a footnote).
%%    16  \printnumvqu: Prints the list of sources like \printnumvli.
%%    17  If you start gbibsort with option -K, and there is an author in the
%%        list of literature or the list of sources more than one time, the
%%        name of this author will be printed only the first time, and then
%%        instead of his name it will be printed 'similar'. You may change
%%        'similar' to "---" by saying "\renewcommand{\killname}{---}".
%%        At the end of \vli's, \vqu's, \addtovli's or \addtovqu's first
%%        argument, you may say \female, or \male. Instead of \killname,
%%        \femalename ("Dies.") or \malename ("Ders.") will be printed.
%%    18  \printind: Prints the index, extracting FILE.ind. If \bib was
%%        in a footnote, footnote number is added as an exponent to page
%%        number. All entries coming from footnotes are printed first.
%%          18.1 article: "HINT: PAGENO^FOOTNOTENO, ..., PAGENO, ...,"
%%          18.2 report:  "HINT
%%                           Chapter CHAPTERNO: PAGENO^FOOTNOTENO, PAGENO,"
%%          18.3 book:    "HINT
%%                           Part PARTNO:
%%                             Chapter CHAPTERNO: PAGENO^FOOTNOTENO, PAGENO,"
%%          Write yourself an index style file with the following lines:
%%                          delim_0 " "
%%                          delim_1 " "
%%                          delim_2 " "
%%                          delim_n "\\komma "
%%                          delim_t " "
%%          Use this style in MakeIndex' command line (-s STYLE).
%%    19  \bibmark{LABEL}: Writes "in the following LABEL" and sets LABEL.
%%          LABEL may be Authors' name. (To be used in footnotes.)
%%    20  \bibref{LABEL}:  Writes "LABEL (see n. XX on p. YY)".       *19
%%    21  \xbibmark{TEXT}{LABEL}: Writes "in the following TEXT" and sets
%%          LABEL. TEXT may be an Authors' name, if this name contains
%%          commands, which are not transformed in macro \@markprotect.
%%    22  \xbibref{TEXT}{LABEL}:  Writes "TEXT (see n. XX on p. YY)". *21
%%
%%  =================
%%
%%
%%
%%  TableOfFiles: BIBARTS
%%  =====================
%%
%%  FILE NAME     ?
%%  ------------  -----------------------------------------
%%  BIBARTS.STY   LaTeX style file.
%%  BIBARTS.TEX   This file (LaTeX).
%%  GBIBSORT.C    Ansi C source file of sorting program.
%%  GBIBSORT.EXE  Sorting program.
%%  GBIBARTS.IST  MakeIndex style file. English users have
%%                  to delete the line re-defining "quote".
%%  GBIB2E.BAT    DOS batch file to make all files. It
%%                  assumes the existence of the DOS batch
%%                  file GLTX2E.BAT whitch starts LaTeX.
%%  GBIB209.BAT   DOS batch file to make all files. It
%%                  assumes the existence of the DOS batch
%%                  file GLATEX.BAT whitch starts LaTeX.
%%  ------------  -----------------------------------------
%%  \sum       7
%%
%%  =====================
%%
%%
%%
%%  BEARBEITUNG DIESES TEXTES:
%%  ==========================
%%
%%  1. Zum Ausprobieren von BiBarts ist es ausreichend, die dazugeh"origen
%%     sieben Dateien in ein(em) Verzeichnis zu kopieren (zu entpacken).
%%
%%  2. Je nach dem, ob LaTeX 2.09 oder LaTeX 2e verwendet wird, m"ussen hier
%%     im File bibarts.tex die Befehle \documentclass und \usepackage oder
%%     \documentstyle auskommentiert werden.
%%
%%  3. DOS & LaTeX 2e: Dreimal in die DOS-Befehls-Zeile
%%        gbib2e bibarts
%%     eingeben!
%%
%%     DOS & LaTeX 2.09: Dreimal in die DOS-Befehls-Zeile
%%        gbib209 bibarts
%%     eingeben!
%%
%%     Falls das nicht funktioniert oder nicht DOS verwendet wird:
%%     UNIX: gbibsort.c in ausf"uhrbare Datei "ubersetzen durch Eingabe von
%%        gcc -O2 -c gbibsort.c -o gbibsort
%%     ALLGEMEIN: Dieses File bibarts.tex mit LaTeX bearbeiten, das dabei
%%     erzeugte bibarts.bar auf bibarts.phy sortieren durch Eingabe von
%%        gbibsort bibarts.bar bibarts.phy
%%     und gbibarts.bar mit
%%        makeind(e)x -s gbibarts.ist -g gbibarts.bar
%%     bearbeiten und danach bibarts.tex noch zweimal mit LaTeX bearbeiten!
%%
%%  4. END-INSTALLATION:
%%     ALLGEMEIN: bibarts.sty in das Verzeichnis f"ur Style-Files kopieren;
%%     in Version 2e ist dies wohl C:\EMTEX\TEXINPUT\LATEX2E, in Version 2.09
%%     vermutlich C:\EMTEX\TEXINPUT. Der Index-Style GBIBARTS.IST geh"ort ins
%%     Verzeichnis C:\EMTEX\IDXSTYLE.
%%     DOS: Falls Sie gbib209.bat oder gbib2e.bat st"andig benutzen wollen,
%%     kopieren sie diese Dateien in ein Verzeichnis im PATH. Gleiches gilt
%%     f"ur gbibsort.exe.
%%
%%  ==========================
%%



 \documentclass[11pt]{article}                    %% LaTeX 2e
 \usepackage{german}                              %%   ^
 \usepackage{a4}                                  %%   |
 \usepackage{bibarts}                             %%   |
                                                   %%   |  Kommentare
                                                   %%   |  richtig setzen!
                                                   %%   |
                                                   %%   v
% \documentstyle[11pt,german,a4,bibarts]{article}   %% LaTeX 2.09

%%
%%   german muss (wenn verwendet)  v o r  bibarts eingeladen werden!!
%%



                             \makebar



%%% ERLAUBT:
% \pagenumbering{roman}
% \stressing{bf}          %% "Andert Betonung des Autorennamens in fett.
% \def\thefootnote{\roman{footnote}}

%%% VERBOTEN: \def\thefootnote{\Roman{footnote}}




   \title{\sf Die \LaTeX-Dokumenten-Stil-Option \bibarts}

  \author{\sc Timo Baumann}

    \date{\sf Version 1.3\, vom 28.Mar.1998}



  \flushbottom
  \def\Befehl#1{{\tt$\backslash$#1}}
  \def\Doppelbox#1#2{\nopagebreak{\fbox{\parbox{.45\textwidth}{\footnotesize\tt
       #1}\hfill\parbox{.45\textwidth}{\small\sf\vspace{1ex}#2\vspace{1ex}}}\vspace{1ex}}}
  \def\Punkt#1{\begin{itemize}\item{#1}\end{itemize}}
  \def\thempfootnote{\thefootnote}
  \def\senkrecht{\hbox to .6em{\hfil\boldmath$|$\hfil}}
  \def\Klammeraffe{\hbox to .6em{\hfil\tt @\hfil}}
  \def\Ausrufezeichen{\hbox to .6em{\hfil\tt !\hfil}}
  \def\indexname{\hspace*{.04\textwidth}\fbox{\parbox{.9\textwidth}{\normalsize\tt
    \Befehl{renewcommand\{}\Befehl{indexname\}\{Register zu
    }\Befehl{bibarts\}}\\
    \Befehl{printind}}}\\[1cm]Register zu \bibarts}


\makeatletter

\def\@Kapitell#1#2{%
     \ifbibarts
%\@vli{\sort{\authorstress #1!!!!!} \protect\bibVkapit {\protect\KapSchr #2} \protect\bibAkapit}
%\@vqu{\sort{\authorstress #1!!!!!} \protect\bibVkapit {\protect\KapSchr #2} \protect\bibAkapit}
\@bib{#1 -- --@ \protect\bibVkapit {\protect\KapSchr #2} \protect\bibAkapit}
     \fi
}

\def\Kapitell{%
\@Kapitell{a}{A}
\@Kapitell{b}{B}
%\@Kapitell{c}{C}
%\@Kapitell{d}{D}
%\@Kapitell{e}{E}
\@Kapitell{f}{F}
\@Kapitell{g}{G}
%\@Kapitell{h}{H}
%\@Kapitell{i}{I}
%\@Kapitell{j}{J}
\@Kapitell{k}{K}
%\@Kapitell{l}{L}
\@Kapitell{m}{M}
%\@Kapitell{n}{N}
%\@Kapitell{o}{O}
\@Kapitell{p}{P}
%\@Kapitell{q}{Q}
%\@Kapitell{r}{R}
\@Kapitell{s}{S}
%\@Kapitell{t}{T}
%\@Kapitell{u}{U}
\@Kapitell{v}{V}
%\@Kapitell{w}{W}
\@Kapitell{x}{X}
%\@Kapitell{y}{Y}
%\@Kapitell{z}{Z}
}

\makeatother

\Kapitell %% Kapitelle (nach obiger Definition von \@Kapitell: nur im Index)

%% \KapSchr definiert die Schrift der Kapitelle





 \begin{document}


  \maketitle

  \noindent\hrulefill

  \begin{abstract}
  In den meisten geisteswissenschaftlichen Disziplinen ist es "ublich, die
  "Uberpr"uf"|barkeit von Aussagen mittels Angaben in Fu"snoten
  sicherzustellen. Wird ein bestimmtes Schriftwerk zum ersten Mal im Text
  erw"ahnt, ist es Standard, in einer Fu"snote dar"uber eine vollst"andige
  bib\-lio\-gra\-phische Angabe zu machen, also Vorname und Name des
  Autors, sowie Titel, Erscheinungsort und Erscheinungsjahr des Werkes
  anzugeben. Bei nochmaligem Verweis auf ein schon in dieser Weise erw"ahntes
  Werk wird nur noch der Autorenname und ein signifikanter Teil aus dem
  Titel genannt. Zuletzt sind noch alphabetisch geordnete Listen der
  verwendeten Schriftwerke abzudrucken. Solche Listen stimmen im wesentlichen
  mit den Angaben in den Fu"snoten "uberein, wie sie schon bei den erw"ahnten
  "`Erstzitaten"' zu machen waren. Ein st"andiger Abgleich dieser Listen mit
  den in Fu"snoten gemachten Angaben ist aber "au"serst m"uhsam --- von dem
  h"aufigen Fehler ganz zu schweigen, da"s Erstzitate und bibliographische
  Listen unbeabsichtigt Unterschiede aufweisen. Die \LaTeX-Befehle
  \Befehl{index} und \Befehl{glossary} l"osen das Problem nur schlecht,
  besonders, da sie in Fu"snoten Schwierigkeiten mit l"angeren Eintr"agen
  machen.

  \LaTeX-Anwender kennen die L"osung f"ur ein vergleichbares Problem
  bez"uglich des Inhaltsverzeichnisses. Hier wird einem Gliederungsbefehl
  --- etwa \Befehl{section} --- als Argument Text f"ur eine "Uberschrift
  "ubergeben, die sowohl am Ort des Befehls, als auch im Inhaltsverzeichnis
  ausgedruckt wird. Das Problem hier unterscheidet sich von dem des
  Inhaltsverzeichnisses nur in der Notwendigkeit der alphabetischen
  Reihenfolge. Wird die Dokumenten-Stil-Option {\tt bibarts.sty} in das
  \LaTeX-File eingeladen, steht der Befehl
  \Befehl{makebar}\addtobib{makebar@\protect\Befehl{makebar}} f"ur den
  Vorspann des \LaTeX-Files zur Verf"ugung. Dieser Befehl legt eine
  Bib\-lio\-gra\-phie-Datei mit dem Suffix {\tt$\!$.bar} an. Befehle namens
  \printonlybib{\Befehl{vli}} und
  \Befehl{vqu}\addtobib{vqu@\protect\Befehl{vqu}} drucken den Text, der ihnen
  in Argumenten "ubergeben wird, am Ort des Befehls aus, schreiben diesen
  Text aber auch in die Bib\-lio\-gra\-phie-Datei. Diese Datei kann danach
  mit dem f"ur dieses Sortier-Problem geschriebenen Programm {\tt gbibsort}
  bearbeitet werden. Die in einer Datei mit Suffix {\tt$\!$.phy} sortiert
  abgelegten Angaben k"onnen mit Befehlen ausgedruckt werden, die in {\tt
  bibarts.sty} bereitgestellt werden. Um diese Grundidee herum f"ugt sich
  noch einiges passendes Beiwerk.
  \end{abstract}

  \nobreak\noindent\hrulefill

  \newpage
  \noindent\hrulefill
  \tableofcontents
  \vspace{1ex}\nobreak\noindent\hrulefill\vspace{2ex}


  \section{\"Anderungen gegen"uber Vers. 1 und Vers. 1.1}

  F"ur diesen Abschnitt werden sich (abgesehen vom zweiten Punkt) nur
  diejenigen interessieren, die \bibarts{} schon in einer "alteren Version
  kennen. Diese neue Version ist teilweise mit den "alteren inkompatibel.
  Die neue Anwendungsweise kann aus dem "ubern"achsten Abschnitt entnommen
  werden.

  \begin{itemize}
  \item Den Befehlen \Befehl{vli},
  \Befehl{vqu},\addtobib{vqu@\protect\Befehl{vqu}}
  \Befehl{addtovli},\addtobib{addtovli@\protect\Befehl{addtovli}}
  \Befehl{addtovqu},\addtobib{addtovqu@\protect\Befehl{addtovqu}}
  \Befehl{printonlyvli}\addtobib{printonlyvli@\protect\Befehl{printonlyvli}}
  und
  \Befehl{printonlyvqu}\addtobib{printonlyvqu@\protect\Befehl{printonlyvqu}}
  m"ussen ab jetzt drei Argumente "ubergeben werden, statt bisher einem.

  \item Bisher gab es zwei Style-Files namens {\tt bibarts.sty} und {\tt
  gbibarts.sty}; das zweitgenannte mu"ste verwendet werden, wenn auch {\tt
  german.sty} verwendet wurde. Dies ist nichtmehr der Fall; es gibt jetzt nur
  noch ein Style-File mit dem Namen {\tt bibarts.sty}. Es erkennt jetzt
  selbst"andig, ob vor {\tt bibarts} das Paket {\tt german} eingeladen
  wurde.\footnote{Und zwar anhand der "Uberpr"ufung, ob \Befehl{newumlaut}
  ein bekannter Befehl ist.} Nach wie vor mu"s {\tt german} vor {\tt
  bibarts} eingeladen werden.\footnote{D.h. bei \LaTeX 2e:
  \Befehl{usepackage\{german\}} \Befehl{usepackage\{bibarts\}}, bzw. unter
  \LaTeX{} 2.09: \Befehl{documentstyle\{...,german,bibarts\}}.}

  \item Der Befehl
  \Befehl{makereportbar}\addtobib{makereportbar@\protect\Befehl{makereportbar}} ist nicht
  mehr notwendig und auch nicht mehr definiert. Der Befehl
  \Befehl{makebar}\addtobib{makebar@\protect\Befehl{makebar}} kann jetzt
  selbst"andig die Unterscheidungen vornehmen, die vorher der Anwender
  machen mu"ste, indem er entweder
  \Befehl{makebar}\addtobib{makebar@\protect\Befehl{makebar}} oder
  \Befehl{makereportbar} schrieb.

  \item Der Anwender soll ab jetzt der Literaturliste, dem Quellenverzeichnis
  und dem Stichwortverzeichnis selber eine "Uberschrift geben. Die fr"uher
  verwendeten \Befehl{section}-Befehle stehen in {\tt bibarts.sty} noch
  hinter Auskommentierungen, die der Anwender ggf.~entfernen
  kann.\footnote{Ich habe darauf verzichtet, anhand einer Abpr"ufung
  dar"uber, ob die Z"ahler {\tt part} oder {\tt chapter} oder keiner von
  beiden verwendet wurde, zu entscheiden, welche Gr"o"se der dazugeh"orige
  Gliederungsbefehl haben m"u"ste. Solche Definitionen vorwegzunehmen ist
  n"amlich f"ur die Anwender kein Service, die z.B.~unter {\tt book} eben
  keine "Uberschrift in Gr"o"se {\tt chapter} haben wollen und sich dann
  "uberlegen m"ussen, wie sie die Voreinstellung loswerden sollen.}

  \item Die Datei {\tt bibarts.sty} enth"alt die Definitionen von
  \Befehl{@footnotetext} aus \LaTeX{} 2.09 und \LaTeX{} 2e, die hier einen
  Zusatz erhalten haben, der bei der "Ubersetzung abpr"ufbar macht, ob man
  sich gerade innerhalb oder au"serhalb einer Fu"snote aufh"alt. \bibarts{}
  entscheidet sich selbst"andig f"ur die richtige
  \LaTeX-Version.\footnote{Und zwar anhand der "Uberpr"ufung, ob
  \Befehl{LaTeX2e} ein bekannter Befehl ist.}

  \item \Befehl{makebar}\addtobib{makebar@\protect\Befehl{makebar}} kann
  auskommentiert werden, wenn gerade keine Notwendigkeit besteht, "uber
  aktuelle bibliographische Listen zu verf"ugen.

  \item Folgendes betrifft nur die Dokumentengr"o"sen {\tt report} und {\tt
  book}: \bibarts{} "ubernimmt jetzt, wenn die Gliederungsbefehle
  \Befehl{part} und \Befehl{chapter} verwendet werden, in sein
  Stichwortverzeichnis den Gliederungsnamen, der tats"achlich am Ort des
  \Befehl{bib}-Befehls aktuell war. Denn \Befehl{partname} und
  \Befehl{chaptername} k"onnen sich im Laufe des Textes "andern, z.B. von
  \frq Teil\flq{} in \frq Anlagen\flq{} und von \frq Kapitel\flq{} in \frq
  Anhang\flq.

  \item In den Index werden Teil-, Kapitel-, Seiten- und Fu"snotenz"ahler
  jetzt so "ubernommen, wie sie im Text stehen, also z.B. {\tt roman} oder
  {\tt Alph}.\footnote{Aber nicht {\tt Roman}; solche Eintr"age werden von
  {\sc MakeIndex} zur"uckgewiesen!}

  \end{itemize}


  \section{\"Anderungen gegen"uber Vers. 1.2 und Vers. 1.2a}

  \begin{itemize}

  \item Ein \underline{I}ndex-\underline{St}yle-File in der Art von {\tt
  gbibarts.ist} ist nun notwendige Voraussetzung zum Erzeugen eines
  \bibarts-Index.\footnote{F"ur DOS: {\tt gbib2e.bat} bzw. {\tt gbib209.bat}
  suchen zun"achst im aktuellen Verzeichnis und dann in
  {\tt C:\Befehl{EMTEX}\Befehl{IDXSTYLE}} nach einer Datei {\tt
  gbibarts.ist}.} Als Gegenleistung sind die bisherigen Probleme mit dem
  Index gel"ost, die auftraten, wenn \Befehl{bib}-Befehle au"serhalb von
  Fu"snoten standen.

  \item Dies betrifft ebenfalls den Index. Die Eigenschaft des Befehls
  \Befehl{bib} hat jetzt der Befehl \Befehl{addtobib}, der bisher nicht
  existierte. \Befehl{addtobib} schreibt nur in den Index, \Befehl{bib}
  schreibt ebenfalls in den Index, druckt jetzt aber auch zus"atzlich das
  ihm "ubergebene Argument einfach aus. Diese "Anderung wurde vorgenommen,
  um einheitliche Verhaltensweisen von (1)~\Befehl{vli}, \Befehl{vqu} und
  \Befehl{bib}, sowie auch (2)~\Befehl{addtovli}, \Befehl{addtovqu} und
  \Befehl{addtobib} zu erreichen.

  \item Dies betrifft Literatur- und Quellen-Zug"ange. {\tt gbibsort}
  kann jetzt mit der Option {\tt -k} gestartet werden. Falls von {\it
  einem\/} Autor mehrere Werke zitiert werden, wird der entsprechende Vor-
  und Nachname auf der jeweiligen Liste nur einmal ausgedruckt. Alle weiteren
  Nennungen werden durch \killname{} ersetzt. Alternativ kann --- das mu"s
  aber durchgehend geschehen --- unmittelbar nach dem Vornamen
  \Befehl{female}\addtobib{female@\protect\Befehl{female}} oder
  \Befehl{male}\addtobib{male@\protect\Befehl{male}} stehen, um statt
  \killname{} geschlechts-spezifisch {\tt\femalename} oder {\tt\malename}
  ausgedruckt zu bekommen. Der Vorname ist das erste Argument, das
  \Befehl{vli}, \Befehl{vqu}, \Befehl{addtovli} oder \Befehl{addtovqu}
  "ubergeben werden soll.

  \item Dies betrifft nur Literatur- und Quellen-Zug"ange. \bibarts{}
  verf"ugt jetzt "uber ein Instrument, um die Sortier-Reihenfolge der
  bibliographischen Listen beeinflussen zu k"onnen. In einem der je drei
  Argumente der Befehle \Befehl{vli}, \Befehl{vqu}, \Befehl{addtovli} und
  \Befehl{addtovqu} kann jetzt der Befehl \Befehl{sort\{RELEVANT\}} verwendet
  werden. {\tt gbibsort} sortiert den Eintrag so, als ob nur {\tt RELEVANT}
  zwischen vorausgehendem und nachfolgendem Text im Argument steht; {\tt
  RELEVANT} wird aber nirgens ausgedruckt. Zu beachten ist, da"s erstes und
  zweites Argument (Vor- und Nachname) ausgetauscht werden, bevor der Eintrag
  in das Bibliographie-File geht. Innerhalb des Arguments des Befehls
  \Befehl{sort\{ARG\}} darf keine Konstruktion in geschweiften Klammern
  stehen. \Befehl{sort}\addtobib{sort@\protect\Befehl{sort}} ist f"ur die
  Argumente von \Befehl{bib} und \Befehl{addtobib} ungeeignet.\footnote{Die
  Sortier-Reihenfolge im Index ist aber {\sc MakeIndex}-spezifisch mit einem
  {\tt @}\addtobib{MakeIndex@{\protect\sc MakeIndex}!\protect\Klammeraffe}
  in der Art \Befehl{addtobib\{RELEVANT@TEXT\}} zu erreichen. Das Zeichen
  {\tt @}\addtobib{gbibsort@{\protect\tt gbibsort}!\protect\Klammeraffe} wird
  im Argument von \Befehl{bib} ausgedruckt; deswegen sollte \Befehl{bib}
  zerlegt werden in seine beiden Bestandteile
  \Befehl{printonlybib\{TEXT\}}\addtobib{printonlybib@\protect\Befehl{printonlybib}} und
  \Befehl{addtobib\{RELEVANT@TEXT\}}. {\tt gbibsort} hat kein
  \bib{Maskierungszeichen}; {\tt\{RELEVANT\}@} --- wozu \Befehl{sort}
  \addtobib{sort@\protect\Befehl{sort}} expandiert --- wird "`verschluckt"',
  {\tt @} ansonsten ausgedruckt.}

  \end{itemize}


  \section{Vorstellung der Optionen}

  \subsection{Die bibliographischen Listen}

  \subsubsection{Das Bib\-lio\-gra\-phie-File}

  Im folgenden werden bibliographische Angaben in Vorname, Name und restliche
  Angaben getrennt. Rest ist dabei Titel, Erscheinungsort und
  Erscheinungsjahr. W"ahrend die bibliographische Angabe in einer Fu"snote
  die Reihenfolge \frq Vorname, Name, Rest\flq{} aufweisen soll, wird in der
  bibliographischen Liste die Reihenfolge \frq Name, Vorname, Rest\flq{}
  erwartet. VORNAME, NAME und REST stehen ab jetzt symbolhaft f"ur die drei
  Argumente, die nachfolgend beschriebenen Befehlen "ubergeben werden
  m"ussen.

  Das Erzeugen der nachfolgend beschriebenen Listen setzt voraus, da"s
  \LaTeX{} w"ahrend der "Ubersetzung eine Datei angefertigt, in der die
  bibliographischen Daten gesammelt werden. \bibarts{} stellt Befehle bereit,
  die das Anfertigen einer solchen Datei erm"oglichen. Diese Datei hat den
  Namen der Datei, die mit \LaTeX{} bearbeitet wird und das Suffix {\tt.bar}.
  Das zu {\tt bibarts.tex} geh"orende Literaturdaten-File hei"st also
  {\tt bibarts.bar}. Sollen ein Literaturverzeichnis oder zus"atzlich auch
  ein Quellenverzeichnis zum Ausdruck bereitgestellt werden, mu"s die Datei
  {\tt NAME.bar} sortiert werden auf eine Datei {\tt NAME.phy}. Die erzeugte
  Datei mu"s also das Suffix {\tt.phy} aufweisen. Im Falle von {\tt
  bibarts.tex} mu"s der Inhalt von {\tt bibarts.bar} nach Zeilen sortiert
  unter dem Namen {\tt bibarts.phy} abgespeichert werden. Das
  Sortierprogramm {\tt gbibsort} l"ost das Sortierproblem. (F"ur DOS ist
  die ausf"uhrbare Datei {\tt gbibsort.exe} im Paket \bibarts{} mit
  enthalten. Anwender anderer Betriebsysteme m"ussen sich die Quell-Datei
  {\tt gbibsort.c} in eine ausf"uhrbare Datei "ubersetzen.\footnote{{\tt
  gbibsort.c} ist in ANSI-C geschrieben und l"a"st sich erfahrungsgem"a"s
  mit dem GNU-C-Compiler auch auf HPUX und PC-UNIX "ubersetzen.}) Der
  Befehl zum Sortieren lautet allgemein {\tt gbibsort QUELLE ZIEL}, also
  f"ur {\tt bibarts.tex}:

  \begin{center}{\tt gbibsort bibarts.bar bibarts.phy}\end{center}

  \bibarts{} stellt Befehle bereit, die die erw"unschten Daten aus dieser
  alphabetisch sortierten Datei an einer bestimmten Textstelle ausdrucken.
  Sie werden unten beschrieben.

  \Punkt{F"ur alle Listen gemeinsam gilt aber folgendes: Ein Befehl im
  Vorspann\footnote{Mit dem Begriff Vorspann bezeichne ich in einem
  \LaTeX-File den Bereich nach dem Befehl \Befehl{documentclass} bzw.
  \Befehl{documentstyle} und vor dem Befehl \Befehl{begin\{document\}}.} mu"s
  die Datei {\tt NAME.bar} anlegen. Dieser Befehl hei"st
  \Befehl{makebar}\addtobib{makebar@\protect\Befehl{makebar}}.}

  \Doppelbox{\Befehl{makebar}}{Bereitet die Datei vor, die die
                             bibliographischen Daten aufnehmen soll.}
                             \addtobib{makebar@\protect\Befehl{makebar}}

  Das Anfertigen dieser Datei kann m"oglicherweise die "Ubersetzung eines
  Textes mit \LaTeX{} unerw"unscht verlangsamen. Deswegen kann der Befehl
  auch auskommentiert (ein \% davorgeschrieben) werden, wenn eine
  "Ubersetzung nicht mit dem Ziel erfolgt, die bibliographischen Daten zu
  aktualisieren, sondern anderer "Anderungen wegen erfolgt. Dann bleibt alles
  "ubrige gleich, nur wird nicht in das Literaturdaten-File geschrieben. Die
  bibliographischen Listen sind lediglich nicht auf dem neuesten Stand, wenn
  sie ausgedruckt werden. (F"ur die letzten beiden "Ubersetzungen mu"s der
  Befehl \Befehl{makebar}\addtobib{makebar@\protect\Befehl{makebar}}
  nat"urlich wirksam sein, also ohne vorangestelltes Kommentarzeichen (\%)
  dastehen!)

  Zuletzt noch ein Hinweis auf die M"oglichkeiten von {\tt gbibsort}:
  Es versteht in der Kommando-Zeile die Option {\tt -K}, was {\sf author
  killing} andeuten soll. Falls ein Autor mehrfach im Literaturverzeichnis
  (oder Quellenverzeichnis) erscheint, bewirkt {\tt -K}, da"s Name und
  Vorname des Autors nur das erste Mal genannt und danach durch \killname{}
  ersetzt werden. Falls das {\sf author killing} benutzt werden soll, gilt:
  Die Befehlszeile f"ur die Bearbeitung eines bibliographischen Files zu
  einer Datei {\tt text.tex} lautet explizit
  \addtobib{gbibsort@{\protect\tt gbibsort}!author killing ({\protect\tt -K})}

  \begin{center}{\tt gbibsort -k text.bar text.phy}\end{center}

  \Punkt{Das Ersetzungszeichen \killname{} wird durch
  \Befehl{killname} erzeugt.}\addtobib{killname@\protect\Befehl{killname}} 

  \Doppelbox{\Befehl{renewcommand\{}\Befehl{killname\}\{---\}}}{
                           "Andert das Ersetzungszeichen in ---.}
                           \addtobib{killname@\protect\Befehl{killname}}

  \vspace{1ex}
  Oft m"ochte man aber geschlechtsspezifische Ersetzungen haben. "Ublich sind
  {\tt\femalename} und {\tt\malename}. Unten werden gleich die Befehle
  \Befehl{vli}, \Befehl{vqu}, \Befehl{addtovli} und \Befehl{addtovqu}
  erkl"art werden. Sie sollen drei Argumente "ubergeben bekommen
  (\Befehl{vli\{ARG1\}\{ARG2\}\{ARG3\}}). Das erste soll der Vorname eines
  Autors sein.

  \Punkt{Unmittelbar nach dem Vornamen im Voll-Zitat (vor \}) kann
  \Befehl{female} oder \Befehl{male} stehen, wenn statt \killname{}
  {\tt\femalename} oder {\tt\malename} ausgedruckt werden
  soll.}\addtobib{female@\protect\Befehl{female}}
        \addtobib{male@\protect\Befehl{male}}

  \Doppelbox{\Befehl{vli\{VORNAME}\Befehl{female\}\{NAME\}\{REST\}}\\
  \Befehl{vli\{VORNAME}\Befehl{male\}\{NAME\}\{REST\}}}{Mehrfachnennungen von
  Autoren werden durch \frq{\rm Dies.}\flq{} und \frq{\rm Ders.}\flq{}
  ersetzt.}

  \Punkt{{\tt\femalename} und {\tt\malename} werden durch \Befehl{femalename}
  und \Befehl{malename} definiert:}
                           \addtobib{femalename@\protect\Befehl{femalename}}
                           \addtobib{malename@\protect\Befehl{malename}}

  \Doppelbox{\Befehl{renewcommand\{}\Befehl{malename\}\{Derselbe\}}}{
                           "Andert \frq{\rm\malename}\flq{} in \frq{\rm
                           Derselbe}\flq.}

  \vspace{1ex}\noindent
  Zuletzt k"onnen noch F"alle auftreten, die eine "Anderung der
  Sortier-Reihen\-fol\-ge der Literaturliste oder des Quellenverzeichnisses
  erstrebenswert erscheinen lassen. Dies wird mittels eines Eintrags von
  \Befehl{sort} in eines der drei Argumente von \Befehl{vli}, \Befehl{vqu},
  \Befehl{addtovli} oder \Befehl{addtovqu} erreicht. \Befehl{sort\{ARG\}}
  wird wie {\tt ARG} sortiert, aber nichts ausgedruckt.

  \Punkt{Ein f"ur {\tt gbibsort} sortier-relevantes, aber nicht
  auszudruckendes Wort ist das Argument von \Befehl{sort}:
  \addtobib{sort@\protect\Befehl{sort}}}

  \Doppelbox{\Befehl{sort\{RELEVANT\}}}{Greift in die Reihenfolge im
                                        Literatur- oder Quellen-Verzeichnis
                                        ein. Es wird wie {\tt RELEVANT}
                                        sortiert, {\tt RELEVANT} wird aber
                                        nicht ausgedruckt.}

  \vspace{1ex}\noindent
  Unter den beiden n"achsten "Uberschriften werden das Literaturverzeichnis
  und das Quellenverzeichnis vorgestellt, wie sie mit \bibarts{} erzeugt
  werden k"onnen.


  \subsubsection{Die Literaturangaben}

  Literaturangaben der nachfolgenden Art sind in allen
  geisteswissenschaftlichen Disziplinen und auch weit dar"uber hinaus
  "ublich.

  \Punkt{Der Befehl \Befehl{vli} druckt die drei ihm "ubergebenen Argumente
  an dem Ort aus, an dem \Befehl{vli} steht, und "ubergibt diese Argumente
  auch dem Literaturdaten-File. \Befehl{vli} steht f"ur \underline{v}olle
  \underline{Li}teraturangabe.}
  \addtobib{vli@\protect\Befehl{vli}}

  \Doppelbox{\Befehl{vli\{VORNAME\}\{NAME\}\{REST\}}}{{\rm VORNAME,
                          \underline{NAME}: REST\/} wird \\ausgedruckt.\\
                          {\rm \underline{NAME}, VORNAME: REST\/} wird im\\
                          Literaturverzeichnis ausgedruckt.}

  \vspace{1ex}
  Der Befehl \Befehl{vli} setzt sich aus den beiden Teilbefehlen
  \Befehl{printonlyvli} und
  \Befehl{addtovli}\addtobib{addtovli@\protect\Befehl{addtovli}} zusammen,
  die auch statt \Befehl{vli} einzeln verwendet werden k"onnen, falls
  Unterschiede zwischen den Angaben in Fu"snoten und dem Quellenverzeichnis
  erw"unscht sind. Ein m"oglicher Unterschied ist, da"s der Titel einer
  Reihe, in der ein Buchtitel erschienen ist, im Literaturverzeichnis
  angegeben werden soll, nicht aber in der Fu"snote. Dann ist REST in
  beiden F"allen unterschiedlich. 

  \Punkt{Der Befehl \Befehl{printonlyvli} druckt die drei ihm "ubergebenen
  Argumente wie \Befehl{vli} an dem Ort aus, an dem der Befehl
  \Befehl{printonlyvli} steht. Es erfolgt aber keine Wiedergabe im
  Literaturverzeichnis.}\addtobib{printonlyvli@\protect\Befehl{printonlyvli}}

  \Doppelbox{\Befehl{printonlyvli\{VORNAME\}\{NAME\}\{REST\}}}{{\rm VORNAME,
                                  \underline{NAME}: REST\/} wird\\
                                  ausgedruckt.\\ Es erfolgt kein Eintrag
                                  ins\\ Literaturverzeichnis.}
                       \addtobib{printonlyvli@\protect\Befehl{printonlyvli}}

  \Punkt{Auf den Befehl \Befehl{addtovli} hin erfolgt lediglich ein Eintrag
  ins Literaturverzeichnis.}\addtobib{addtovli@\protect\Befehl{addtovli}}

  \Doppelbox{\Befehl{addtovli\{VORNAME\}\{NAME\}\{REST\}}}{{\rm\underline{NAME},
                               VORNAME: REST\/} wird dem\\
                               Literaturdaten-File "ubergeben.\\ Es erfolgt
                               kein Ausdruck am Ort des\\ Befehls.}
                               \addtobib{addtovli@\protect\Befehl{addtovli}}

   \Punkt{Jetzt mu"s das Literaturverzeichnis noch ausgedruckt werden. Die
   schon beschriebene Prozedur mu"s zuvor noch durchlaufen werden, um aus der
   Datei {\tt FILE.bar} eine alphabetisch sortierte Datei {\tt FILE.phy} zu
   erzeugen. Dann kann mittels des Befehls \Befehl{printvli} das
   Literaturverzeichnis ausgedruckt werden. Unmittelbar vor
   \Befehl{printvli} kann der Anwender, wenn er m"ochte, einen
   Gliederungsbefehl seiner Wahl setzen. M"oglich ist:}

   \Doppelbox{\Befehl{section\{Literaturverzeichnis\}}\\
              \Befehl{printvli}}{Druckt die Literaturliste unter der
              "Uberschrift \frq Literaturverzeichnis\flq{} aus.}

   \Punkt{Die Befehle \Befehl{vli} und
   \Befehl{addtovli}\addtobib{addtovli@\protect\Befehl{addtovli}} notieren in
   das Bib\-lio\-gra\-phie-File auch Informationen "uber den Ort, an dem
   sie stehen: Die Seitenzahl, weiter auch, ob sie in einer Fu"snote stehen
   --- und wenn ja: die Nummer der Fu"snote. So gibt es noch einen zweiten
   Befehl, um das Literaturverzeichnis ausdrucken zu k"onnen. Der Befehl
   hei"st \Befehl{printnumvli} und unterscheidet sich von \Befehl{printvli}
   dadurch, da"s hinter jedem Zugang die Seitenzahl steht, auf der der
   Zugang stattfand. Falls der Zugang in einer Fu"snote erfolgte, wird die
   Nummer der Fu"snote als Exponent zur Seitenzahl ausgedruckt. Die
   Gliederungs"uberschrift zu setzen ist wieder Sache des Anwenders ---
   Beispiel: eine \Befehl{subsection}.} 

   \Doppelbox{\Befehl{subsection\{Literaturverzeichnis\}}\\
              \Befehl{printnumvli}}{Druckt Literaturliste aus, derart, da"s
                                     auf den Ort zur"uckverwiesen wird, an
                                     dem das Erstzitat steht.}

   \vspace{1ex}\noindent
   Nat"urlich kann \Befehl{printnumvli} auch w"ahrend der Entstehung
   eines Textes verwendet werden, um bei Korrekturen, die an der
   Literaturliste ansetzen, schneller in den Text zur"uckzufinden. Vor der
   letzten "Ubersetzung mit \LaTeX{} kann dieser Befehl dann durch
   \Befehl{printvli} ersetzt werden, falls die R"uckverweise nicht
   erw"unscht sind.


  \subsubsection{Die Quellenangaben}

  In der Geschichtswissenschaft ist es dar"uberhinaus "ublich, zwischen
  Literatur und Quellen zu trennen. Die bibliographischen Angaben
  unterscheiden sich zwar nicht qualitativ voneinander, jedoch sollen zwei
  bibliographische Listen angefertigt werden: Ein Quellenverzeichnis und ein
  Literaturverzeichnis. (Quellen sind Schriften aus der Zeit, "uber die
  geschrieben wird, w"ahrend mit Literatur sp"ater verfa"ste Titel
  bezeichnet werden, die die behandelte Epoche anhand der Quellen
  beschreiben.) Auch hierf"ur h"alt \bibarts{} eine Option offen.

  \Punkt{Der Befehl \Befehl{vqu} druckt die drei ihm "ubergebenen Argumente
  sowohl an dem Ort aus, an dem \Befehl{vqu} steht, als auch in das
  Quellen-Verzeichnis. \Befehl{vqu} steht f"ur \underline{v}olle
  \underline{Qu}ellenangabe.}\addtobib{vqu@\protect\Befehl{vqu}}

  \vspace{-1ex}
  \Doppelbox{\Befehl{vqu\{VORNAME\}\{NAME\}\{REST\}}}{{\rm VORNAME,
                          \underline{NAME}: REST\/} wird \\ausgedruckt.\\
                          {\rm\underline{NAME}, VORNAME: REST\/} wird im\\
                          Quellenverzeichnis ausgedruckt.}

  \vspace{2ex}
  Der Befehl \Befehl{vqu}\addtobib{vqu@\protect\Befehl{vqu}} setzt sich aus
  den beiden Teilbefehlen \Befehl{printonlyvqu} und
  \Befehl{addtovqu}\addtobib{addtovqu@\protect\Befehl{addtovqu}} zusammen,
  die auch statt \Befehl{vqu}\addtobib{vqu@\protect\Befehl{vqu}} einzeln
  verwendet werden k"onnen, falls Unterschiede zwischen den Angaben in
  Fu"snoten und dem Quellenverzeichnis erw"unscht sind. Ein m"oglicher
  Unterschied ist, da"s der Titel einer Reihe, in der die Quellenedition
  erschienen ist, im Quellenverzeichnis angegeben werden soll, nicht aber
  in der Fu"snote. Dann ist REST in beiden F"allen unterschiedlich. 

  \Punkt{Der Befehl \Befehl{printonlyvqu} druckt die drei ihm "ubergebenen
  Argumente an dem Ort aus, an dem der Befehl \Befehl{printonlyvqu} steht.
  Es erfolgt aber keine Wiedergabe im
  Quellenverzeichnis.}\addtobib{printonlyvqu@\protect\Befehl{printonlyvqu}}

  \Doppelbox{\Befehl{printonlyvqu\{VORNAME\}\{NAME\}\{REST\}}}{{\rm VORNAME,
                                  \underline{NAME}: REST\/} wird
                                  \\ausgedruckt.\\ Es erfolgt kein Eintrag
                                  ins \\ Quellenverzeichnis.}
                      \addtobib{printonlyvqu@\protect\Befehl{printonlyvqu}}

  \Punkt{Auf den Befehl \Befehl{addtovqu} hin erfolgt lediglich ein Eintrag
  in das Quellenverzeichnis.}\addtobib{addtovqu@\protect\Befehl{addtovqu}}

  \Doppelbox{\Befehl{addtovqu\{VORNAME\}\{NAME\}\{REST\}}}{{\rm
                                  \underline{NAME}, VORNAME: REST\/} wird
                                  im\\ Quellenverzeichnis ausgedruckt.\\ Es
                                  erfolgt kein Ausdruck am Ort des\\
                                  Befehls.}
                               \addtobib{addtovqu@\protect\Befehl{addtovqu}}

   \Punkt{Jetzt mu"s das Quellenverzeichnis noch tats"achlich ausgedruckt
   werden. Die schon beschriebene Prozedur mu"s durchlaufen werden, um aus
   der Datei {\tt FILE.bar} eine alphabetisch sortierte Datei {\tt FILE.phy}
   zu erzeugen. Die Daten f"ur die Literaturliste und das
   Quellenverzeichnis entstammen der selben Datei. Aus ihr drucken die
   Befehle \Befehl{printvli}, \Befehl{printnumvli}, \Befehl{printvqu} und
   \Befehl{printnumvqu} selektiv spezifische Daten aus. Mit dem Befehl
   \Befehl{printvqu} kann das Quellenverzeichnis ausgedruckt werden.
   Unmittelbar vor den Befehl \Befehl{printvqu} kann der Anwender, wenn er
   m"ochte, einen Gliederungsbefehl seiner Wahl setzen. M"oglich ist:}

   \Doppelbox{\Befehl{section\{Quellenverzeichnis\}}\\
              \Befehl{printvqu}}{Druckt das Quellenverzeichnis unter der
                             "Uberschrift \frq Quellenverzeichnis\flq{} aus.}

   \Punkt{Die Befehle \Befehl{vqu}\addtobib{vqu@\protect\Befehl{vqu}} und
   \Befehl{addtovqu}\addtobib{addtovqu@\protect\Befehl{addtovqu}} notieren in
   das Bib\-lio\-gra\-phie-File auch Informationen "uber den Ort, an dem
   sie stehen: Die Seitenzahl, dann: ob sie in einer Fu"snote stehen ---
   und wenn ja: die Nummer der Fu"snote. So gibt es noch einen zweiten
   Befehl, um das Quellenverzeichnis ausdrucken zu k"onnen. Der Befehl
   hei"st \Befehl{printnumvqu} und unterscheidet sich von \Befehl{printvqu}
   dadurch, da"s hinter jedem Eintrag die Seitenzahl steht, auf der der
   Zugang stattfand. Falls der Zugang in einer Fu"snote erfolgte, wird die
   Nummer der Fu"snote als Exponent zur Seitenzahl ausgedruckt. Als
   Gliederungs"uberschrift wurde eine \Befehl{subsection} gew"ahlt.}

   \Doppelbox{\Befehl{subsection\{Quellenverzeichnis\}}\\
              \Befehl{printnumvqu}}{Druckt Quellenverzeichnis aus, derart,
                                     da"s auf den Ort zur"uckverwiesen
                                     wird, an dem das Erstzitat steht.}


  \subsubsection{Das Kurzzitate-Verzeichnis}

  \Punkt{"Ahnlich den Befehlen \Befehl{index} oder \Befehl{glossary} stellt
  \bibarts{} den Befehl \Befehl{bib} zur Verf"ugung. Als Argument mu"s diesem
  Befehl ein Stichwort "ubergeben werden. {\tt STICHWORT} kann speziell ein
  Kurz-Zitat (Marx, Kapital) sein; das Schriftwerk, auf das sich {\tt
  STICHWORT} bezieht, soll vorher schon vollst"andig zitiert worden sein.
  Ein Kurz-Zitat besteht aus dem Namen des Autors und (durch Komma
  getrennt) einem oder wenigen signifikanten Worten aus dem Titel, die eine
  eindeutige Feststellung der vollst"andigen bibliographischen Daten anhand
  des Literatur- oder Quellen-Verzeichnisses m"oglich machen. Gegen"uber
  dem \LaTeX-Befehl \Befehl{index} wird das Argument von \Befehl{bib}
  erweitert indexiert: Hier verweisen die Stichworte n"amlich nicht nur auf
  die Seite, sondern auch auf die Nummer der Fu"snote zur"uck, falls der
  Befehl in einer Fu"snote steht. Diese Informationen gehen ebenfalls in
  das {\tt FILE.bar}. Auch wenn die zuvor vorgestellten Befehle zum
  Erzeugen einer Literaturliste oder eines Quellenverzeichnisses nicht
  benutzt werden, so mu"s das {\tt FILE.bar} dennoch durch den schon
  beschriebenen Befehl
  \Befehl{makebar}\addtobib{makebar@\protect\Befehl{makebar}}
  bereitgestellt werden.}

  \nopagebreak
  \Doppelbox{\Befehl{bib\{STICHWORT\}}}{Stellt einen Eintrag f"ur das
                                        Stichwort-Verzeichnis zur Verf"ugung.
                                        Der Befehl druckt STICHWORT aus.}

  \Punkt{Falls STICHWORT nur im Index erscheinen soll, steht neu der Befehl
  \Befehl{addtobib} zur Verf"ugung.}

  \Doppelbox{\Befehl{addtobib\{STICHWORT\}}}{Stellt einen Eintrag f"ur das
                                             Stichwort-Verzeichnis zur Verf"ugung.
                                             Der Befehl druckt nichts aus an
                                             dem Ort, an dem er steht.}

  \vspace{1ex}
  Das mittels des Befehls \Befehl{bib} gef"utterte {\tt FILE.bar} mu"s mit
  dem Programm {\sc MakeIndex} bearbeitet werden. Zug"ange der Befehle, die
  zum Erzeugen der Literatur- und Quellen-Verzeichnisse ebenfalls vorhanden
  sein m"ogen, wird der Stichwortprozessor {\sc MakeIndex} als fehlerhaft
  einstufen und nicht verwenden. Das ist so erw"unscht. Um eine Datei {\tt
  FILE.bar} zu bearbeiten, mu"s unter DOS

  \vspace{-1ex}
   \begin{center}{\tt makeindx FILE.bar}\end{center}

  \vspace{-1ex}
   \noindent befohlen werden. Unter UNIX lautet der Aufruf

  \vspace{-1ex}
   \begin{center}{\tt makeindex FILE.bar}\end{center}

  \vspace{-1ex}
   F"ur deutsche Texte mu"s die Option {\tt -g} "ubergeben werden, um die
   Sortierreihenfolge auf \underline{g}erman umzuschalten. Das gen"ugt aber
   noch nicht: In Grundeinstellung ist \verb|"| das \bib{Maskierungszeichen}
   von {\sc MakeIndex}. Maskierungszeichen dienen dazu, \verb+!+, \verb+|+
   und \verb+@+ via \verb+"!+, \verb+"|+ und \verb+"@+ ausdrucken lassen zu
   k"onnen. Unmaskiert stellen sie Befehle dar, die das Verhalten von {\sc
   MakeIndex} steuern. \verb+"+ wird aber f"ur die deutschen Umlaute
   gebraucht, wie mit {\tt german.sty} definiert. Um das Maskierungszeichen
   zu "andern, mu"s ein Index-Style-File nach der Option {\tt -s} in der
   Befehlszeile stehen. In diesem Index-Style-File kann eine Zeile

   \begin{center}{\tt quote '='}\end{center}

   \noindent
   das \bib{Maskierungszeichen} in {\tt =} umbenennen. Erst dann wird die
   Option {\tt -g} akzeptiert. Dazu wurde zu \bibarts{} die Datei {\tt
   gbibarts.ist} beigef"ugt. Der restliche Inhalt dieser Datei ist
   sprachunabh"angig f"ur \bibarts{} notwendig. (In \Befehl{bib} ist \verb+|+
   \addtobib{MakeIndex@{\protect\sc MakeIndex}!\protect\senkrecht} "ubrigens~
   i~m~m~e~r~ schon besetzt!\footnote{Ebenso steht {\tt !}, das in
   \Befehl{index} zweimal gebraucht werden darf, in \Befehl{bib} nur einmal
   zur Verf"ugung, wenn {\tt chapter}-"Uberschriften, und garnicht, wenn
   dar"uber hinaus {\tt part}-"Uberschriften benutzt werden. In {\tt
   bibarts.tex} sind also zwei {\tt !} erlaubt.
   \addtobib{MakeIndex@{\protect\sc MakeIndex}!\protect\Ausrufezeichen}
   (Bedeutung von {\tt !} in Anm.\,\ref{elf}).\label{zehn}}) --- Unter DOS
   lautet die Befehlszeile des bei der Bearbeitung von {\tt bibarts.tex}
   erzeugten Files {\tt gbibarts.bar} also:

   \begin{center}{\tt makeindx -s gbibarts.ist -g bibarts.bar}\end{center}

   \Punkt{{\sc MakeIndex} fertigt eine Index-Datei {\tt FILE.ind} an. Der
   Befehl \Befehl{printind} druckt diese Datei zweispaltig unter der
   "Uberschrift Index aus. Der Ausdruck unterscheidet sich vom
   normalen \LaTeX-Index: An die Seitenzahlen ist als Exponent die Nummer der
   Fu"snote hinzugedruckt, falls der Zugang in einer Fu"snote erfolgte.}

   \Doppelbox{\Befehl{printind}}{Druckt das mit \Befehl{bib} gef"utterte
                                 Stichwortverzeichnis.}

   \Punkt{Falls jedes Zitat in das Kurzzitate-Verzeichnis aufgenommen werden
   soll, ist es sinnvoll, neben \Befehl{bib} einen Befehl zu besitzen, der
   (1)~innerhalb von \Befehl{vli} oder \Befehl{vqu} im voraus ank"undigt,
   wie der Titel kurzzitiert wird, (2)~gleich einen Eintrag ins
   Kurzzitate-Verzeichnis macht und (3)~hinter dem Eintrag auf der
   Literatur- oder Quellen-Liste in eckigen Klammern ausdruckt, wie
   kurz-zitiert wurde.}

  \vspace{-1ex}
  \Doppelbox{\Befehl{kurz\{STICHWORT\}}}{Stellt einen Eintrag f"ur das
                                        Stichwort-Verzeichnis zur Verf"ugung.
                                        Der Befehl druckt {\rm (im folgenden
                                        STICHWORT)\/} aus. Falls \Befehl{kurz}
                                        am Ende des dritten Arguments von
                                        \Befehl{vli}, \Befehl{vqu},
                                        \Befehl{addtovli} oder \Befehl{addtovqu}
                                        steht, wird \\
                                        {\rm [STICHWORT]\/} am Ende des
                                        Eintrags auf der bibliographischen
                                        Liste stehen.}
                                        \addtobib{kurz@\protect\Befehl{kurz}}

  \vspace{1ex}
   Erg"anzen m"ochte ich noch, da"s der Befehl \Befehl{subitem} in den
   {\tt FILEs.ind} stehen wird, die von {\sc MakeIndex} erzeugt werden,
   nachdem die Gliederungs-"Uberschrift \Befehl{chapter} zum ersten Mal
   verwendet wurde und dar"uber hinaus auch \Befehl{subsubitem}, falls
   \Befehl{part} verwendet wurde. Sie entsprechen Unter-Stichworten. Mit
   Befehlen der Art
   \Befehl{renewcommand\{\Befehl{subitem}\}\{\Befehl{hspace\{.3em\}}\}} kann
   erzwungen werden, da"s Text nach dem Index-Haupteintrag der Reihe nach
   heruntergeschrieben wird. Zur Sicherheit sollte auf den {\sc
   MakeIndex}-Befehl {\tt !}, der Stichwort und Unterstichwort trennt, im
   Argument von \Befehl{bib} verzichtet werden.\footnote{Zur Erzeugung von
   \Befehl{subitem}s verwendet \bibarts{} Ausrufezeichen, die f"ur {\sc
   MakeIndex} Befehlscharakter haben. Wenn z.B.
   \Befehl{index\{Maier!Arbeit\}} und \Befehl{index\{Maier!Recht\}}
   geschrieben wird, gibt es die beiden Unter-Eintr"age Arbeit und Recht
   unter dem Haupteintrag Maier. Maximal zwei~{\tt !}~k"onnen verwendet
   werden. Damit scheidet die beschriebene Untergliederung mit \Befehl{bib}
   aus, wenn \Befehl{part}-"Uberschriften verwendet werden. (Siehe auch
   Anm.\,\ref{zehn}.)\label{elf}\addtobib{MakeIndex@{\protect\sc
   MakeIndex}!\protect\Ausrufezeichen}}


  \subsection{Die Verweise}

   Was in diesem Unterabschnitt zu Sprache kommt, hat nichts mit dem
   Bib\-lio\-gra\-phie-File zu tun. "`Verweis"' wird hier eine zweite
   Zitier-Konvention genannt, die manche Verlage f"ur Aufs"atze verwenden.
   Hier werden ebenso wie zuvor anl"a"slich des Erstzitats vollst"andige
   bibliographische Angaben gemacht. Wird auf einen Titel ein zweites Mal
   eingegangen, wird dieser ebenfalls kurz-zitiert, aber dahinter kommt ein
   Verweis auf die vollst"andige erste Angabe. Zu solchen Aufs"atzen
   geh"oren keine bibliographischen Listen.

   \Punkt{Nach jedem Erstzitat kann \Befehl{bibmark\{MARKE\}} dazu verwendet
   werden, eine Marke zu erzeugen. Im Unterschied zu dem \LaTeX-Befehl
   \Befehl{label} wird aber ausgedruckt, wie k"unftig kurzzitiert wird.
   {\tt MARKE} und Kurzzitat sind identisch.}
   \addtobib{bibmark@\protect\Befehl{bibmark}}

   \Doppelbox{\Befehl{bibmark\{MARKE\}}}{Setzt die Referenz {\tt MARKE} und
                                          druckt aus: {\rm im folgenden
                                          MARKE\/}}
                                \addtobib{bibmark@\protect\Befehl{bibmark}}

   \Punkt{Mit dem Befehl \Befehl{bibref\{MARKE\}}
   \addtobib{bibref@\protect\Befehl{bibref}} kann man sich dann Seitenzahl
   und Fu"snotennummer in einer "ubliche Konstruktion zur"uckgeben lassen.}

   \Doppelbox{\Befehl{bibref\{MARKE\}}}{Druckt in Fu"snote aus:\\
                                        {\rm MARKE (wie Anm.~XX, S.~YY)\/},\\
                                        ansonsten:\\
                                        {\rm MARKE (siehe S.~YY)\/}.
                                        }
   \addtobib{bibref@\protect\Befehl{bibref}}

   \vspace{1ex}
   {\tt MARKE} kann dabei ohne weiteres ein Kurz-Zitat in der Art {\tt Marx,
   Kapital} sein. Auch \verb!B"ulow! funktioniert als Marke, denn Umlaute, "s,
   Akut, Zirkumflex und Gravis sind in einer Struktur namens
   \Befehl{@markprotect} gesch"utzt. Bei hier nicht gesch"utzten Befehlen
   m"ussen auszudruckender {\tt TEXT} und Schl"usselbegriff {\tt MARKE}
   voneinander getrennt werden.

   \Punkt{Der Befehl \Befehl{xbibmark\{TEXT\}\{MARKE\}} setzt die {\tt MARKE}
   und druckt den {\tt TEXT} aus.}
   \addtobib{bibmark@\protect\Befehl{bibmark}}
   \addtobib{xbibmark@\protect\Befehl{xbibmark}}

   \Doppelbox{\Befehl{xbibmark\{TEXT\}\{MARKE\}}}{Setzt die Referenz {\tt
                                           MARKE} und druckt aus: {\rm
                                           im folgenden TEXT\/}.}
                             \addtobib{xbibmark@\protect\Befehl{xbibmark}}

   \Punkt{Mit dem Befehl \Befehl{xbibref\{TEXT\}\{MARKE\}}
   \addtobib{xbibref@\protect\Befehl{xbibref}} kann man sich dann Seitenzahl
   und Fu"snotennummer von {\tt MARKE} zur"uckgeben lassen.}

   \Doppelbox{\Befehl{xbibref\{TEXT\}\{MARKE\}}}{Druckt in Fu"snote aus:\\
                                      {\rm TEXT (wie Anm.~XX, S.~YY)\/},\\
                                      ansonsten:\\
                                      {\rm TEXT (siehe S.~YY)\/}.
                                      }
   \addtobib{xbibref@\protect\Befehl{xbibref}}

   \vspace{1ex}\noindent
   Alle vier zuletzt angesprochenen Befehle \Befehl{bibmark}-\Befehl{bibref}
   und \Befehl{xbibmark}-\Befehl{xbibref} sollen paarweise entweder beide
   in Fu"snoten ober au"serhalb von Fu"snoten stehen. F"ur die
   Verweis-Schl"usselbegriffe MARKE aller vier Befehle ist das Makro
   \Befehl{@markprotect} zust"andig. Die \Befehl{x...}-Befehle sind
   eigentlich nichtmehr unbedingt notwendig, denn mit \Befehl{@markprotect}
   kann eine unliebsame Expansion von Befehls-Makros durch eine an dieser
   Stelle erw"unschte Expansion ausgetauscht werden. Haupts"achlich aus
   Gr"unden der Kompatibilit"at mit "alteren \bibarts-Versionen sind die
   \Befehl{x...}-Befehle erhalten geblieben.

   \vfill
   \noindent\hrulefill

   \vspace{1ex}\nopagebreak\noindent
   Auf der n"achsten Seite werden Beispiele zu den bibliographischen Listen
   folgen. Hierzu ein Hinweis f"ur DOS: {\tt gbibsort.exe} enth"alt einen
   \bib{go32}-Extender. Es ist oft notwendig, f"ur diesen ein Verzeichnis
   NAME anzulegen, das dessen {\it paging-files} aufnehmen kann (damit wird
   der Arbeitsspeicher virtuell vergr"o"sert). Auf dieses Verzeichnis ist
   dann mit {\tt set go32tmp=NAME} eine Umgebungsvariable zu setzen. Dieser
   Befehl sollte sinnvollerweise in der Datei {\tt autoexec.bat} stehen.

   \nopagebreak\noindent\hrulefill

  \newpage
  \section{Beispiele}

  Nat"urlich kann in die formatierte Ausgabe der Literaturangaben
  eingegriffen werden. Die Befehle \Befehl{stressing}
  \addtobib{stressing@\protect\Befehl{stressing}} und \Befehl{punctuation}
  \addtobib{punctuation@\protect\Befehl{punctuation}} definieren die Betonung
  des Autoren-Nachnamens und das Satzzeichen, das zwischen dem Autorennamen
  und dem REST steht. Dies betrifft einerseits die Befehle \Befehl{vqu},
  \Befehl{addtovqu} und \Befehl{printonlyvqu}, andererseits \Befehl{vli},
  \Befehl{addtovli} und \Befehl{printonlyvli}. Voreinstellung f"ur
  \Befehl{stressing} \addtobib{stressing@\protect\Befehl{stressing}} ist {\tt
  underline}. 

  \Punkt{\Befehl{stressing} \addtobib{stressing@\protect\Befehl{stressing}}
  kann im Vorspann und vor jedem Listen-Ausdruckbefehl stehen. Sein Argument
  kann z.B. {\tt sc}, {\tt large}, {\tt underline} oder {\tt fbox}
  enthalten.}

  \Doppelbox{\Befehl{stressing\{bf\}}}{"Andert die Betonung des Autorennamens
                                    von unterstreichen auf fett (bold face).}

  \Punkt{\Befehl{punctuation}
  \addtobib{punctuation@\protect\Befehl{punctuation}} sollte nur im
  Vorspann verwendet werden:}

  \Doppelbox{\Befehl{renewcommand\{}\Befehl{punctuation\}\{,\}}}{"Andert den
  Doppelpunkt zwischen VORNAME bzw. NAME und REST in Komma.}

   \vspace{1ex}\noindent
    Jetzt soll der Weg eines \Befehl{vli}-Befehls verfolgt
    werden:

    \Punkt{Hier steht (und man vergleiche dies mit der Fu"snote
    unter besonderer Beachtung der {\sf serifen--losen
    Schrift}):}

    \refstepcounter{footnote}
\framebox{\begin{minipage}[t]{.9\textwidth}
    \framebox{\parbox{.9\textwidth}{
    ... Text.\Befehl{footnote\{ ... mit Bezug auf}\\
    \Befehl{vli\{Helmut\}\{Kopka\}\{\Befehl{LaTeX}\{\} --}
                  {\tt Eine Einf"uhrung,\\
    \hspace*{1cm}Bonn 1993\},}\\
    \hspace*{1cm}{\tt S. 187 belegen.\}}
             }} \vspace{1ex}
    \addtobib{vli@\protect\Befehl{vli}} \vspace{2ex}

                                       \footnotetrue
    ... Text.\footnote{ Ich m"ochte meine Aussagen mit Bezug auf
    \vli{Helmut}{Kopka}{\LaTeX{} -- Eine Einf"uhrung, Bonn 1993},
    \addtobib{vli@\protect\Befehl{vli}}    \footnotefalse
                     S. 187 belegen.
                  \begin{center}\thepage\end{center}
                                    }
\end{minipage}}

    \Punkt{Der \Befehl{vli}-Eintrag "uber Kopka kommt ins
    Literaturverzeichnis:}

\framebox{\begin{minipage}[t]{.9\textwidth}
    \framebox{\parbox{.9\textwidth}{
    \Befehl{section*\{Literaturverzeichnis\} \Befehl{printvli}}
             }}
                      \section*{Literaturverzeichnis} \printvli
\end{minipage}}

    \Punkt{Im Literaturverzeichnis kann auch auf die Seite zur"uckverwiesen
    werden, auf der der Literatur-Zugang mit \Befehl{vli} stattfand:}

\framebox{\begin{minipage}[t]{.9\textwidth}
    \framebox{\parbox{.9\textwidth}{
    \Befehl{section*\{Literaturverzeichnis\} \Befehl{printnumvli}}
             }}
                     \section*{Literaturverzeichnis} \printnumvli
\end{minipage}}

    \vspace{2ex}\noindent
    Wie angegeben, l"a"st sich bei Kopkas Buch sowohl Seite, als auch
    Fu"snotennummer nachvollziehen.

    \Punkt{Die Angabe zu Norbert Schwarz steht nicht in einer Fu"snote. Hier
    steht:}

\framebox{\begin{minipage}[t]{.9\textwidth}
    \framebox{\parbox{.9\textwidth}{
    \Befehl{vli\{Norbert\}\{Schwarz\}\{Einf"uhrung in}\\
     \hspace*{1cm}{\tt\Befehl{protect}\Befehl{TeX}, Bonn
         1988\Befehl{kurz\{Schwarz\}}\}}
    \addtobib{vli@\protect\Befehl{vli}}
              }}\vspace{2ex}

    \vli{Norbert}{Schwarz}{Einf"uhrung in \protect\TeX, Bonn
    1988\kurz{Schwarz}}
    \addtobib{vli@\protect\Befehl{vli}}\addtobib{kurz@\protect\Befehl{kurz}}
\end{minipage}}

    \Punkt{Die Schriftart, in der hier das Erstzitat "`Kopka"' oder
    "`Schwarz"' ausgedruckt ist, kann einheitlich von {\sf sans serif} auf
    {\it italics\/} umgestellt werden.}

    \Doppelbox{\Befehl{renewcommand\{}\Befehl{schrift\}\{}\Befehl{it\}}}{
    "Andert die Schrift, in der Erstzitate\\ ausgedruckt werden, in {\it
    italics\/}.}

    \Punkt{Das Zeichen $\gg$ in den \Befehl{printnum...}-Listen ist durch den
    Befehl \Befehl{verw} definiert; es kann beispielsweise so ge"andert
    werden:}\addtobib{verw@\protect\Befehl{verw}}

    \addtobib{verw@\protect\Befehl{verw}}
    \Doppelbox{\Befehl{renewcommand\{\Befehl{verw}\}\{; Erstzitat:\~{}\}}}{
    "Andert das Zeichen, das vor den\\ Angaben des Zugangsorts steht.}

    \Punkt{Die Schriftart, in der Literaturverzeichnis und Quellenverzeichnis
    ausgedruckt werden, hei"st \Befehl{barschrift}. "Anderungsvorschlag:}

    \Doppelbox{\Befehl{renewcommand\{}\Befehl{barschrift\}\{}\Befehl{rm\}}}{
    "Andert die Schrift der bibliographischen Verzeichnisse in {\rm roman}.}

    \Punkt{Die Schriftart, in der das Stichwortverzeichnis ausgedruckt wird,
    hei"st \Befehl{indschrift}. "Anderungsvorschlag:}

    \Doppelbox{\Befehl{renewcommand\{}\Befehl{indschrift\}\{}\Befehl{sf}\}}{
    "Andert die Schrift, in der der Index\\ ausgedruckt wird in {\sf sans
    serif}.}

    \vspace{1ex}
    Jetzt ein Punkt f"ur Fortgeschrittene: Ein noch nicht erw"ahnter Befehl
    hei"st \Befehl{Kapitell}. Falls er genutzt wird, soll er im Vorspann
    direkt nach \Befehl{makebar} stehen. Er bewirkt, da"s Gro"s-Buchstaben
    A bis Z in gro"ser Schrift in das Quellenverzeichnis, das
    Literaturverzeichnis und das Kurzzitateverzeichnis (Index, Register)
    ausgedruckt werden. Diese Kapitelle stehen leicht abgesetzt am Anfang
    jeder Gruppe von Autoren mit einem bestimmten Anfangsbuchstaben.
    \Befehl{Kapitell} ist im Vorspann von {\tt bibarts.tex} modifiziert
    worden, denn es sollte dazu unter jeden Buchstaben mindestens ein
    Zugang mit dem gleichen Anfangsbuchstaben fallen. Das ist im Register
    dieses Textes nicht der Fall. Auf Kapitelle in den Beispielen f"ur
    Literatur- und Quellenverzeichnis wurde verzichtet. Der Vorspann von {\tt
    bibarts.tex} kann als Beispiel f"ur eigene Texte dienen. Notwendig sind
    hier die Befehle \Befehl{makeatletter} und \Befehl{makeatother} vor und
    nach Befehlen, die {\tt @} enthalten.

    \vspace{1ex}
    Alle vorausgegangenen Beispiele betrafen Einstellungen im Umfeld des
    bibliographischen Files und die Einstellungen dar"uber, wie in das
    bibliographische File schreibende Befehle ihre Argumente an Ort und
    Stelle ausdrucken.

    Doch jetzt zur zweiten Zitierkonvention, den R"uck-Verweisen nach
    Kurz-Zitaten auf das vollst"andige Erstzitat. (Es ist keine
    Literaturliste oder vergleichbares vorgesehen f"ur das Ende eines Textes,
    der diesen Teil der \bibarts-Optionen nutzt.) Es folgt eine
    \Befehl{bibref}-\Befehl{bibmark}-Kom\-bi\-na\-tion:
    \addtobib{bibref@\protect\Befehl{bibref}}
    \addtobib{bibmark@\protect\Befehl{bibmark}}

     \vspace{1ex}
\framebox{\begin{minipage}[t]{.9\textwidth}
    \framebox{\parbox{.9\textwidth}{\tt
  ... Text.\Befehl{footnote\{In dieser Fu"snote sei\\
     \hspace*{1cm}Lecl\Befehl{`e}rc erstzitiert:}\\
     \hspace*{1cm}Fran\Befehl{c}\{c\}ois Lecl\Befehl{`e}rc: Der kalte
         Sommerregen,\\
      \hspace*{1cm}Musterland 1899\\
      \hspace*{.5cm}(\Befehl{bibmark\{Lecl}\Befehl{`erc,
         Beruf\}).\}}\\
      Nicht in Fu\Befehl{ss}\{\}note:\\
     \hspace*{.72cm}\Befehl{bibref\{Lecl}\Befehl{`e}rc, Beruf\}\\
  ... Text.\Befehl{footnote\{ Wie schon }\\
      \hspace*{.72cm}\Befehl{bibref\{Lecl}\Befehl{`erc, Beruf\}
       sagte ... \}} }} \vspace{2ex}

       \refstepcounter{footnote}    \footnotetrue
  ... Text.\footnote{ In dieser Fu"snote sei Lecl\`erc erstzitiert:
      Fran\c{c}ois Lecl\`erc: Der kalte Sommerregen, Musterland 1899
         (\bibmark{Lecl\`erc, Beruf}).       }
         \addtobib{bibmark@\protect\Befehl{bibmark}}
                                    \footnotefalse
      Nicht in Fu\ss{}note: \bibref{Lecl\`erc, Beruf}
       \refstepcounter{footnote}    \footnotetrue
  ... Text.\footnote{ Wie schon
         \addtobib{bibref@\protect\Befehl{bibref}}
         \bibref{Lecl\`erc, Beruf} sagte ...
                  \begin{center}\thepage\end{center}
             }
                                    \footnotefalse
\end{minipage}}
     \addtobib{bibmark@\protect\Befehl{bibmark}}

    \vspace{1cm}\noindent
    Auf der nachfolgenden Seite soll als letztes Beispiel mit
    \Befehl{printind} der Index ausgedruckt werden. Diese n"achste Seite wird
    vollst"andig fehlen, falls kein Index existiert.
    DOS: Der Index wird gleich von {\tt gbib209.bat} oder {\tt gbib2e.bat}
    miterzeugt durch Eingabe von {\tt gbib2e bibarts}, falls die installierte
    \LaTeX-Version 2$\epsilon$ hei"st, oder {\tt gbib209 bibarts}, falls noch
    \LaTeX 2.09 verwendet wird.

    \newpage
    \printind


  \section{Grenzen der M"achtigkeit von \bibarts}

  Die Befehle \Befehl{vli} und
  \Befehl{vqu},\addtobib{vqu@\protect\Befehl{vqu}}
  \Befehl{addtovli}\addtobib{addtovli@\protect\Befehl{addtovli}} und
  \Befehl{addtovqu}\addtobib{addtovqu@\protect\Befehl{addtovqu}} sind einer
  wichtigen Beschr"ankung unterworfen: Befehle, die in ihren Argumenten
  auftauchen, k"onnen "`zerbrechen"', wenn Befehle Makros sind. Nicht der
  Befehl selbst wird in das {\tt FILE.bar} geschrieben, sondern die
  expandierten Makros. Solche "`zerbrochenen Befehle"' k"onnen beim
  "Ubersetzen mit \LaTeX{} oft nicht mehr verstanden, also nicht
  ausgedruckt werden. \TeX{} beschwert sich bei solchen Gelegenheiten auch
  oftmals dar"uber, da"s seine Kapazit"at "uberschritten sei ("`...
  Sorry"'). Wie bei den befehlen \Befehl{index} oder \Befehl{glossary}
  auch, mu"s ein solcher Befehl dann durch ein davorgestelltes
  \Befehl{protect} gesch"utzt werden. ODER: In {\tt bibarts.sty} wird
  dieser Befehl im Makro \Befehl{@sIcHerUnG} --- so wie dort vorgemacht ---
  mit \Befehl{string} gesch"utzt. Hier sind bereits die ganzen Umlaute und
  viele andere Akzente (wie Akut, Zirkumflex und Gravis) gegen Zerbrechen
  gesch"utzt. Ich empfehle allerdings, die Konstruktion
  \Befehl{@ifundefined\{newumlaut\}...} unver"andert zu lassen, denn sie
  erkennt den wirklich bedeutsamen Unterschied, ob {\tt german.sty}
  verwendet wird, oder nicht. (Das Paket {\tt german} mu"s --- wenn es
  verwendet wird --- vor {\tt bibarts} eingeladen werden!)

  Ebenfalls eine Sicherungsstruktur wurde f"ur
  \Befehl{bibmark}-\Befehl{bibref}-Kom\-bi\-na\-tio\-nen eingef"uhrt. Diese
  Befehle nutzen \Befehl{label}, \Befehl{pageref} und \Befehl{ref}, in deren
  Argumente keine Befehle stehen d"urfen. Die Sicherung, die hier z.B.
  Umlaute m"oglich macht, hei"st \Befehl{@markprotect}. Die hier
  bearbeiteten Befehle sind fast die selben, wie auch im Makro
  \Befehl{@sIcHerUnG}. Allerdings werden die Befehle hier nicht
  schreibgesch"utzt, sondern in \verb!|X|!-Konstruktionen "ubersetzt, wobei
  \verb!X! ein f"ur jeden Befehl anderer Buchstabe ist und \verb!|! dazu
  dient, da"s Verwechslungsm"oglichkeiten mit anderen Schl"usselbegriffen
  minimiert werden. Da das Makro \Befehl{@markprotect} gleicherma"sen in
  den Marken, wie in den Referenzen verwendet wird, kann der Anwender
  bedenkenlos Befehle in die Marken setzen --- {\it wenn\/} diese Befehle
  in \Befehl{@markprotect} zu Buchstaben expandiert werden. Da
  \Befehl{underline} und Umlaute derart bearbeitet werden, ist

  \verb!\bibmark{\underline{M"uller}, Hallo}!

  \verb!\bibref{\underline{M"uller}, Hallo}!\\m"oglich. In
  \Befehl{@markprotect} k"onnen --- wie dort vorgemacht --- weitere h"aufig
  verwendete Befehle zur Verwendung als Verweis-Schl"usselworte
  bereitgestellt werden. F"ur seltene Konstruktionen steht
  \Befehl{xbibmark}-\Befehl{xbibref} zur Verf"ugung, die {\tt MARKE} und
  auszudruckenden {\tt TEXT} trennen. Verweis-Marken ergeben "ubrigens
  \Befehl{newlabel}-Eintr"age im Auxiliary-File mit dem Suffix {\tt .aux}.

  Doch zur"uck zur Bearbeitung des bibliographischen Files. Das Programm {\tt
  gbibsort} kann so viele Zeilen sortieren, wie in der Gr"o"se {\tt MAXLINES}
  in der Datei {\tt gbibsort.c} zum Zeitpunkt der "Ubersetzung von {\tt
  gbibsort.c} in eine ausf"uhrbare Datei festgelegt war. Ebenso d"urfen
  Zeilen eines zu sortierenden Files nicht l"anger als {\tt MAXLEN} (1024)
  sein. L"angere Zeilen sollten gek"urzt werden, indem ein Rest des
  Eintrags in eine Datei (mit dem Namen {\tt DATEI-NAME}) verschoben wird;
  anstelle des Restes sollte \Befehl{input\{DATEI-NAME\}} eingetragen
  werden. Der Befehl \Befehl{input} mu"s in den \Befehl{vli}-,
  \Befehl{vqu}-,\addtobib{vqu@\protect\Befehl{vqu}}
  \Befehl{addtovli}-,\addtobib{addtovli@\protect\Befehl{addtovli}}
  \Befehl{addtovqu}-Befehlen\addtobib{addtovqu@\protect\Befehl{addtovqu}}
  durch ein davorgestelltes \Befehl{protect} oder durch Definition
  innerhalb des Makros \Befehl{@sIcHerUnG} vor dem "`Zerbrechen"'
  gesch"utzt werden.

  Die Kapazit"at des Befehls \Befehl{bib} hat etwas gr"o"sere
  Beschr"ankungen, als die anderen \bibarts-Befehle.

  \vspace{-1ex}
  \Punkt{Eintr"age im Stichwortverzeichnis werden sicher richtig sortiert,
  wenn sie von Seiten mit Seitenzahlen zwischen 1 und 999 stammen. Zwei
  \Befehl{bib}-Befehle mit gleichem Argument dagegen werden, wenn auch nur
  einer von beiden auf einer Seite mit einer Seitenzahl $<= 0$ oder $>=
  1000$ steht, nur durch Zufall richtig sortiert.}

  \vspace{-3ex}
  \Punkt{Folgendes gilt f"ur das Stichwortverzeichnis in Dokumentengr"o"se
  {\tt book} und {\tt part}: Die Sortierhilfen f"ur die Z"ahler {\tt part}
  und {\tt chapter} reichen von 1 bis 99. Zwei \Befehl{bib}-Eintr"age mit
  gleichem Argument aus Teilen oder Kapiteln mit Nummern ab einhundert
  werden nur durch Zufall richtig sortiert. Tats"achlich erh"alt man, wenn
  Zug"ange aus mehreren Kapiteln und sowohl aus dem Text als auch aus
  Fu"snoten kommen, Ausgaben in der Art}

  \vspace{-2ex}
  \framebox{\small\begin{minipage}[t]{.9\textwidth}
  \vspace*{1ex}
  {\rm STICHWORT}:\\
  \hspace*{1cm}{\it Kapitel 1\/}:\\
  \hspace*{2cm}2$^5$, 5$^8$, 1, 4,\\
  \hspace*{1cm}{\it Kapitel 3\/}:\\
  \hspace*{2cm}23$^5$, 31$^{18}$, 24, 25,
  \vspace{1ex}
                  \end{minipage}
          }

  \vspace{2ex}
  Zuerst werden kapitelweise alle Zug"ange aus Fu"snoten ausgedruckt und
  danach die anderen Zug"ange aus dem jeweiligen Kapitel.

  \vspace{-1ex}
  \Punkt{\Befehl{bib} kann nur verwendet werden, wenn ein
  \underline{I}ndex-\underline{St}yle-File in der Art von {\tt gbibarts.ist}
  verwendet wird. {\tt gbibarts.ist} mu"s neben Sprachspezifika wie {\tt
  quote~'='} folgende Zeilen enthalten:}

  \vspace{-3ex}
  \begin{verbatim}
             delim_0 " "
             delim_1 " "
             delim_2 " "
             delim_n "\\komma "
             delim_t " "
  \end{verbatim}

  \vspace{-3ex}
  \noindent
  Solche {\tt STYLE}s werden durch Aufrufe der Art

  \vspace{-1ex}
  \begin{center}{\tt makeind(e)x\/ -s\/ STYLE\, FILE}\end{center}

  \vspace{-1ex}
  \noindent
  bei der Bearbeitung der Datei {\tt FILE} ber"ucksichtigt.

  Der \bibarts-Index verweist zur"uck auf Stellen im Text. Falls der
  Index-Zugang \Befehl{bib\{STICHWORT\}} in den Fu"snoten 1, 2 und 3 auf
  einer Seite 12 steht, erscheint ein Eintrag {\sf STICHWORT~12$^{1-3}$\/},
  wenn der Index mit \Befehl{printind} ausgedruckt wird. Zusammenfassungen
  der Art {\tt 1-3} werden nicht erzeugt, falls \Befehl{bib} oder
  \Befehl{addtobib} au"serhalb von Fu"snoten (im normalen Text) verwendet
  werden. Falls Zug"ange von den Seiten 1, 2 und 3 kommen, wird {\tt 1}, {\tt
  2}, {\tt 3} dastehen. Es kann daher oft sinnvoll sein, sich auf dem
  herk"ommlichen Weg via \Befehl{index\{STICHWORT\}} einen seperaten Index
  f"ur Verweise in den Haupt-Text zu machen. Der \bibarts-Index ist insofern
  eine gute Grundlage f"ur ein Personenregister, das "ublicherweise von
  anderen Registern getrennt wird, die mit den \LaTeX-Befehlen \Befehl{index}
  und \Befehl{glossary} erzeugt werden k"onnen. Die Mehrfachausgabe einer
  Seitenzahl im \bibarts-Kurzzitate-Index wird immerhin unterdr"uckt, falls
  von einer Seite mehrere gleiche Index-Zug"ange kommen. Dies funktioniert
  auch dann, falls die Seitennummerierung mittels
  \Befehl{pagenumbering\{roman\}} auf kleine r"omische Zahlen umgestellt
  wird. \Befehl{pagenumbering\{Roman\}} funktioniert nicht.

  Eine wirklich elegante L"osung dieser Probleme wird erst eine Erweiterung
  des Programms {\sc MakeIndex} bringen, das dann nach mehreren Zahlen ---
  neben einem Eintrag des Z"ahlers {\tt page} zumindest noch nach dem des
  Z"ahlers {\tt footnote} --- sortieren k"onnen mu"s.

  \def\abstractname{Message}

  \vspace{3ex}
  \noindent\hrulefill

  \begin{abstract}
  Die Idee von \bibarts{} ist, die bibliographischen Angaben an einer
  zentralen Stelle {\it im Text\/} zu machen, und diese Informationen unter
  m"oglichst wenig Hinzutun des Anwenders an allen notwendigen Stellen
  ausgedruckt zu bekommen. So erh"alt man um den Preis des Markierens von
  Erstzitaten eine fertige Literaturliste. Sie erscheint an der Stelle, an
  der der Ausdruckbefehl plaziert wird. Jede und jeder, die oder der schon
  einmal tagelang am Erstellen einer Literaturliste sa"s, wird sofort die
  Arbeitsersparnis einsehen: Das Durchsuchen der Fu"snoten nach Erstzitaten
  entf"allt vollst"andig. Dieses "`Definieren an zentraler Stelle"' greift
  eigentlich eine Idee des Programmierens auf. Dort gilt es als guter Stil,
  mehrfach verwendete Gr"o"sen, die jeweils gleich sein sollen, einmal und
  endg"ultig an einer Stelle festzulegen. "Anderungen des Programmtextes
  lassen sich so besser beherrschen. Soll diese Idee auf das Schreiben von
  Texten "uber\-tragen werden, stellt sich die Frage, an welcher Stelle
  solche Definitionen sinnvollerweise zu machen sind.
 
  Sicherlich ist es heute "ublich, bei der Literatur-Recherche
  computergest"utzt zu arbeiten. Es existieren schon mehrere andere
  Programme oder Formatvorlagen, die die Idee zentralen Definierens in Form
  von Datenbanken umsetzen. Aus dem Text heraus kann auf einzelne Eintr"age
  in dieser Datenbank verwiesen werden, um sich so Titel ausdrucken lassen
  zu k"onnen. Die Frage ist, ob das Instrument zur Datenerfassung auch direkt
  zum Schreiben eines Textes benutzt werden soll.

  Mit \bibarts{} wird diese Frage mit {\it nein\/} beantwortet. Ich
  jedenfalls kann mir nicht tausende von {\it short cuts\/} oder
  Schl"usselbegriffen merken, die auf die einzelnen Titel verweisen.
  Die Literatur-Datenbanken m"ussen sowieso durchsucht werden. Die
  Optimierungs-Aufgabe dort stellt sich so: Wie schnell kann der Editor, mit
  dem der Text geschrieben wird, in einer Datenbank Informationen
  auf"|finden und in den Text transferieren? Mit dieser Frage habe
  ich mich hier nicht befa"st. Das wesentliche Problem im Rahmen
  geisteswissenschaftlicher Zitierkonventionen --- d.h.: der Punkt, wo
  Optimierung zuerst ansetzen mu"s --- liegt n"amlich in der Kombination
  Erstzitat--Literaturliste. Bibliographische Listen werden (neben ihrem
  Selbstzweck) mit \bibarts{} auch zu Kontrollausdrucken f"ur die Erstzitate.
  \bibarts{} wurde vor diesem Hintergrund ver"offentlicht, in der Hoffnung,
  da"s es als n"utzlich empfunden wird. \hfill {\tt
  Timo.Baumann@hist.unibe.ch}
  \end{abstract}

  \vspace{-1ex}
  \nopagebreak\noindent\hrulefill


 \end{document}
