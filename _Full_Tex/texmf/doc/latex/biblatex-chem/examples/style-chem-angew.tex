%
% This file presents the `chem-angew' style
%
\documentclass[a4paper]{article}
\usepackage[T1]{fontenc}
\usepackage[english,UKenglish]{babel}
\usepackage[babel]{csquotes}
\usepackage[
  style=chem-angew,
  % articletitle,     % To include article titles
  % biblabel=dot,     % Alter bibliography labels
  % chaptertitle,     % Include chapter titles for parts of books
  % doi,              % Always include DOI for articles
  % pageranges=false, % Only include first page of a range
  % subentry,         % For (a), (b), etc. in sets
  hyperref,
  ]{biblatex}
\usepackage[
  colorlinks,
  linkcolor=black,
  urlcolor=black,
  citecolor=black
  ]{hyperref}
\bibliography{biblatex-chem}

\begin{document}

\section*{The \texttt{chem-angew} style}

This style prints numeric citations with bibliography
formatting following the rules of the used in \emph{Angewandte
Chemie}. The same formatting applies to related journals
published by Wiley, for example \emph{Chemistry---A European
Journal} and \emph{European Journal of Organic Chemistry}. With
settings for citations as given, the citations will be
superscript and punctuation will be moved before citations, for
example \autocite{Kabbe1973} or \cite{Arduengo1991}.

\nocite{*}

\printbibliography

\end{document}
