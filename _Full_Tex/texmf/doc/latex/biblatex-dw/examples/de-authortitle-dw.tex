% v 1.3a
% This file presents the `authortitle-dw' style
%
\listfiles
\documentclass[a4paper]{article}
\usepackage[T1]{fontenc}
\usepackage[latin9]{inputenc}
\usepackage[ngerman]{babel}
\usepackage[autostyle]{csquotes}
\usepackage[
  style=authortitle-dw,
%% biblatex-dw-Optionen %%%
  %acronyms=true,
  annotation=true,
  %citedas=false,
	%citepages=suppress,%omit,%permit,%separate,
  %edbyidem=false,
  %editorstring=normal,%brackets,
  %edstringincitations=false,
  %edsuper=true,
  %firstfull=true,
  %firstfullname=true,
  %firstnamefont=smallcaps,%italic,%bold,
  %ibidemfont=smallcaps,%italic,%bold,
  %idembib=false,
  %idembibformat=dash,
  %idemfont=smallcaps,%italic,%bold,
  %journalnumber=afteryear,%date,%standard
  library=true,
  %namefont=smallcaps,%italic,%bold,
  %nolocation=true,
  %nopublisher=false,
  %oldauthor=false,
  %omiteditor=true,
  %origfields=false,
  %origfieldsformat=parens,%brackets,%punct,
  %pagetotal=true,
  %pseudoauthor=false,
  %series=afteryear,
  %shorthandibid=false,
	%shorthandinbib=true,
  %shorthandwidth=40pt,%3em,
  %shortjournal=true,
  %terselos=false,
  %xref=true,
%% biblatex-Optionen %%%
  %autocite=plain,
	%citetracker=false,
	%doi=true,
	%eprint=true,
	%ibidpage=true,
	%ibidtracker=false,
  %idemtracker=false,
	%isbn=true,
	%pagetracker=false,
  %hyperref
]{biblatex}
\usepackage{hyperref}
\bibliography{de-examples-dw}
% Some generic settings:
\newcommand{\cmd}[1]{\texttt{\textbackslash #1}}
\usepackage{xcolor}
\newcommand{\option}[1]{\textcolor{red}{#1}}
\setlength{\parindent}{0pt}

%\renewcommand*{\bibleftpseudo}{\bibleftbracket}
%\renewcommand*{\bibrightpseudo}{\bibrightbracket}

\begin{document}

\section*{Der Stil \texttt{authortitle-dw}}

\subsection*{\cmd{cite}-Beispiele}

\cite{aristotle:rhetoric}

\cite[59]{aristotle:rhetoric}

\cite[Siehe][]{springer}

\cite[Siehe][92--95]{springer}

\subsection*{\cmd{parencite}-Beispiele}

Dies ist nur F�lltext \parencite{aristotle:rhetoric}.

Dies ist nur F�lltext \parencite[59]{aristotle:rhetoric}.

Dies ist nur F�lltext \parencite[Siehe][]{springer}.

Dies ist nur F�lltext \parencite[Siehe][92--95]{springer}.

\subsection*{\cmd{citeauthor}- und \cmd{parencite*}-Beispiele}

\citeauthor{aristotle:rhetoric} zeigt, dass dies nur F�lltext
ist \parencite*{aristotle:rhetoric}.

\citeauthor{aristotle:rhetoric} zeigt, dass dies nur F�lltext
ist \parencite*[59]{aristotle:rhetoric}.

\citeauthor{springer} zeigt, dass dies nur F�lltext
ist \parencite*[Siehe][]{springer}.

\citeauthor{springer} zeigt, dass dies nur F�lltext
ist \parencite*[Siehe][92--95]{springer}.

\subsection*{\cmd{footcite}-Beispiele}

Dies ist nur F�lltext.\footcite{aristotle:rhetoric}

Dies ist nur F�lltext.\footcite[59]{aristotle:rhetoric}

Dies ist nur F�lltext.\footcite[Siehe][]{springer}

Dies ist nur F�lltext.\footcite[Siehe][92--95]{springer}

\subsection*{\cmd{textcite}-Beispiele}

\textcite{aristotle:rhetoric} zeigt, dass dies nur F�lltext ist.

\textcite[59]{aristotle:rhetoric} zeigt, dass dies nur F�lltext ist.

\textcite[Siehe][]{springer} f�r mehr F�lltext.

\textcite[Siehe][92--95]{springer} f�r mehr F�lltext.

\subsection*{\cmd{autocite}-Beispiele}

Dies ist nur F�lltext \autocite{springer}.

\subsection*{Mehrere Zitate}

\cite{aristotle:rhetoric,aristotle:physics,aristotle:poetics}

\subsection*{Shorthand-Beispiele}

\cite{kant:kpv}

\cite[noch einmal]{kant:kpv}

\cite{kant:ku}

\subsection*{\cmd{fullcite}-Beispiele}

Dies ist nur F�lltext. \fullcite{aristotle:rhetoric}

Dies ist nur F�lltext. \fullcite[59]{aristotle:rhetoric}

Dies ist nur F�lltext. \fullcite[Siehe][]{springer}

Dies ist nur F�lltext. \fullcite[Siehe][92--95]{springer}

\subsection*{\cmd{footfullcite}-Beispiele}

Dies ist nur F�lltext. \footfullcite{aristotle:rhetoric}

Dies ist nur F�lltext. \footfullcite[59]{aristotle:rhetoric}

Dies ist nur F�lltext. \footfullcite[Siehe][]{springer}

Dies ist nur F�lltext. \footfullcite[Siehe][92--95]{springer}

\clearpage

\printshorthands

\nocite{*}
\printbibliography[notkeyword=journalnumberdate]

\end{document}
