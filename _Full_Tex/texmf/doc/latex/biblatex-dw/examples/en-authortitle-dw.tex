% v 1.3a
% This file presents the `authortitle-dw' style
%
\documentclass[a4paper]{article}
\usepackage[T1]{fontenc}
\usepackage[british]{babel}
\usepackage[autostyle]{csquotes}
\usepackage[
  style=authortitle-dw,
%%% biblatex-dw options %%%
  %acronyms=true,
  annotation=true,
  %citedas=false,
	%citepages=suppress,%omit,%permit,%separate,
  %edbyidem=false,
  %editorstring=normal,%brackets,
  %edstringincitations=false,
  %edsuper=true,
  %firstfull=true,
  %firstfullname=true,
  %firstnamefont=smallcaps,%italic,%bold,
  %ibidemfont=smallcaps,%italic,%bold,
  %idembib=false,
  %idembibformat=dash,
  %idemfont=smallcaps,%italic,%bold,
  %journalnumber=afteryear,%date,%standard
  library=true,
  %namefont=smallcaps,%italic,%bold,
  %nolocation=true,
  %nopublisher=false,
  %oldauthor=false,
  %omiteditor=true,
  %origfields=false,
  %origfieldsformat=parens,%brackets,%punct,
  %pagetotal=true,
  %pseudoauthor=false,
  %series=afteryear,
	%shorthandinbib=true,
  %shorthandibid=false,
  %shorthandwidth=40pt,%3em,
  %shortjournal=true,
  %terselos=false,
  %xref=true,
%% biblatex options %%%
  %autocite=plain,
	%citetracker=false,
	%doi=true,
	%eprint=true,
	%ibidpage=true,
	%ibidtracker=false,
  %idemtracker=false,
	%isbn=true,
	%pagetracker=false,
  hyperref
]{biblatex}
\usepackage{hyperref}
\bibliography{examples-dw}
% Some generic settings:
\newcommand{\cmd}[1]{\texttt{\textbackslash #1}}
\usepackage{xcolor}
\newcommand{\option}[1]{\textcolor{red}{#1}}
\setlength{\parindent}{0pt}
\begin{document}

\section*{The \texttt{authortitle-dw} style}

\subsection*{\cmd{cite} examples}

\cite{aristotle:rhetoric}

\cite[59]{aristotle:rhetoric}

\cite[See][]{springer}

\cite[See][59--63]{springer}

\subsection*{\cmd{parencite} examples}

This is just filler text \parencite{aristotle:rhetoric}.

This is just filler text \parencite[59]{aristotle:rhetoric}.

This is just filler text \parencite[See][]{springer}.

This is just filler text \parencite[See][59--63]{springer}.

\subsection*{\cmd{citeauthor} and \cmd{parencite*} examples}

\citeauthor{aristotle:rhetoric} shows that this is just filler
text \parencite*{aristotle:rhetoric}.

\citeauthor{aristotle:rhetoric} shows that this is just filler
text \parencite*[59]{aristotle:rhetoric}.

\citeauthor{springer} shows that this is just filler
text \parencite*[See][]{springer}.

\citeauthor{springer} shows that this is just filler
text \parencite*[See][59--63]{springer}.

\subsection*{\cmd{footcite} examples}

This is just filler text.\footcite{aristotle:rhetoric}

This is just filler text.\footcite[59]{aristotle:rhetoric}

This is just filler text.\footcite[See][]{springer}

This is just filler text.\footcite[See][59--63]{springer}

\subsection*{\cmd{textcite} examples}

\textcite{aristotle:rhetoric} shows that this is just filler text.

\textcite[59]{aristotle:rhetoric} shows that this is just filler text.

\textcite[See][]{springer} for more filler text.

\textcite[See][59--63]{springer} for more filler text.

\subsection*{\cmd{autocite} examples}

This is just filler text \autocite{springer}.

\subsection*{Multiple citations}

\cite{aristotle:rhetoric,aristotle:physics,aristotle:poetics}

\subsection*{Shorthand examples}

\cite{kant:kpv}

\cite[again]{kant:kpv}

\cite{kant:ku}

\subsection*{\cmd{fullcite} examples}

This is just filler text. \fullcite{aristotle:rhetoric}

This is just filler text. \fullcite[59]{aristotle:rhetoric}

This is just filler text. \fullcite[See][]{springer}

This is just filler text. \fullcite[See][92--95]{springer}

\subsection*{\cmd{footfullcite} examples}

This is just filler text. \footfullcite{aristotle:rhetoric}

This is just filler text. \footfullcite[59]{aristotle:rhetoric}

This is just filler text. \footfullcite[See][]{springer}

This is just filler text. \footfullcite[See][92--95]{springer}

\clearpage

\printshorthands

\nocite{*}
\printbibliography[notkeyword=journalnumberdate]

\end{document}
