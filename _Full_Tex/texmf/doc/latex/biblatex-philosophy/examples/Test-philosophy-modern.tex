\documentclass[a4paper]{article}
\usepackage[latin1]{inputenc}
\usepackage[T1]{fontenc}
\usepackage[english,german,italian]{babel}
\usepackage[babel,italian=guillemets]{csquotes}

\usepackage[%
style=philosophy-modern,
scauthors=true,
%scauthorsbib=true,
%scauthorscite=true,
%publocformat=locpubyear,%loccolonpub,
%volnumformat=volnumstrings,% volnumparens,
%romanvol=true,
%inbeforejournal=true,
%origfieldsformat=brackets,%parens
%yearleft=true,
annotation=true,
library=true,
%square=true,
%natbib=true,
%maxnames=2,
%minnames=1,
backref,
hyperref,
]{biblatex}
\bibliography{philosophy-examples}
\bibliography{../philosophy-examples}
%\bibliography{examples} % il file examples.bib si trova nella cartella di biblatex

\defbibheading{esempio}{\section*{Bibliografia esemplificativa}} 
\defbibheading{primary}{\section*{Riferimenti bibliografici}}
\defbibheading{siti}{\section*{Sitografia}}


%% 			personalizzazioni
%%-------------------------------------------------------

%\setlength{\yeartitle}{0.8em}
%\setlength{\postnamesep}{0.5ex plus 2pt minus 1pt}

%\setlength{\bibnamesep}{1.5ex plus 2pt minus 1pt}
%\setlength{\bibitemsep}{\postautsep}
%\setlength{\bibhang}{4\parindent}

%\DeclareFieldFormat{backrefparens}{#1}
%\DeclareFieldFormat{backrefparens}{\mkbibbrackets{#1}}

%\renewcommand{\annotationfont}{\sffamily}
%\renewcommand{\libraryfont}{\itshape}
%\renewcommand*{\volnumpunct}{/}



\usepackage{hyperref}


\begin{document}
\nocite{*}

\section{Comando $\backslash$\texttt{sdcite}}
Per le opere senza data, permette di utilizzare lo schema autore--titolo: 

\sdcite{Metaphysica}

\sdcite{Ethica}

\section{Comando $\backslash$\texttt{footcitet}}
Riproduce in nota lo stile del comando \verb|\textcite|:\footcitet[12-13]{Corrocher:2009}



%% SIGLE
\printshorthands


% BIBLIOGRAFIE
Si noti che nelle bibliografie seguenti la lista degli autori/editori � riportata completamente, sebbene venga troncata nelle citazioni. Ci� si pu� ottenere attraverso le opzioni \texttt{maxnames=999} e \texttt{minnames=999}, passate direttamente al comando \verb|\printbibliography|.

% Note da stampare dopo il titolo ma prima della lista dei riferimenti
\defbibnote{notaesempio}{\small\sffamily Questa bibliografia contiene gli esempi, pi� o 
		meno fittizi, citati in questo articolo, esclusi i testi contenuti nei riferimenti 
		bibliografici. I dati relativi alle voci si trovano nel file \texttt{philosophy-examples.bib}.}

\defbibnote{notariferimenti}{\small\sffamily In questa bibliografia si noti come il campo 
\texttt{annotation} venga usato per produrre delle voci commentate. In questo modo possiamo fornire per ciascuna voce un breve sommario}

% Bibliografia esemplificativa
\phantomsection{}
\addcontentsline{toc}{section}{Bibliografia esemplificativa}
\printbibliography[maxnames=999,minnames=999,prenote=notaesempio,heading=esempio,keyword=Esempio]

% Riferimenti bibliografici
\phantomsection{}
\addcontentsline{toc}{section}{\refname}
\printbibliography[maxnames=999,minnames=999,prenote=notariferimenti,heading=primary,keyword=primary]


\end{document}
