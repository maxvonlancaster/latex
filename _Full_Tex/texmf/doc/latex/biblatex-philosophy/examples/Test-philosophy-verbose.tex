\documentclass[a4paper]{article}
\usepackage[latin1]{inputenc}
\usepackage[T1]{fontenc}
\usepackage[english,german,italian]{babel}
\usepackage[babel,italian=guillemets]{csquotes}

\usepackage[%
style=philosophy-verbose,
scauthors=true,
%scauthorsbib=true,
%scauthorscite=true,
%publocformat=locpubyear,%loccolonpub,
%volnumformat=volnumstrings,% volnumparens,
%romanvol=true,
%inbeforejournal=true,
%origfieldsformat=brackets,%parens
latinemph=true,
%commacit=true,
annotation=true,
%library=false,
backref,
hyperref,
%babel=other
]{biblatex}



\bibliography{philosophy-examples}
\bibliography{../philosophy-examples}
%\bibliography{biblatex-examples} % il file examples.bib si trova nella cartella di biblatex

\defbibheading{esempio}{\section*{Bibliografia esemplificativa}} 
\defbibheading{primary}{\section*{Riferimenti bibliografici}}
\defbibheading{siti}{\section*{Sitografia}}



%% 			personalizzazioni
%%-------------------------------------------------------

%\DeclareFieldFormat{backrefparens}{#1}
%\DeclareFieldFormat{backrefparens}{\mkbibbrackets{#1}}

%\renewcommand{\annotationfont}{}
%\renewcommand{\libraryfont}{\sffamily}
%\renewcommand*{\volnumpunct}{/}
%\renewcommand*{\volumfont}{}
%\renewcommand*{\volumfont}{\scshape}
%%-------------------------------------------------------


%%----------------------------------------------------------------------------------------------------------
\usepackage{hyperref}






\begin{document}
\nocite{*}
\section{File test per lo stile \texttt{philosophy-verbose}}

Qui abbiamo la prima citazione del \citetitle{Cartesio:2002} di Cartesio.\footcite[Cfr.][43]{Cartesio:2002}
Ecco poi un riferimento alla stessa opera, con l'indicazione della pagina\autocite[26]{Cartesio:2002}, e un altro al medesimo luogo della citazione precedente\autocite[26]{Cartesio:2002}. Ora citiamo un altro testo, che comparir� per esteso,\autocite[59--61]{Termini:2007} e poi nuovamente il testo di Cartesio, che comparir� nella forma abbreviata.\autocite[35]{Cartesio:2002} 

\section{Voci \texttt{@incollection}}
Qui abbiamo un articolo su collettanea\footnote{\label{nota:federspil}\cite{Federspil:2009}.}
e questo � un altro esempio.\footnote{\label{nota:corrocher}\cite{Corrocher:2009}.} Come si vede, le informazioni relative alla \texttt{collection} compaiono estesamente nella nota \ref{nota:federspil} e abbreviate nella nota \ref{nota:corrocher}. Nella bibliografia finale, invece, ogni voce \texttt{incollection} (o \texttt{inbook}) sar� completa di tutte le informazioni. In tutte le note di questo paragrafo � stato utilizzato il campo \texttt{crossref}, che nel caso di due o pi� citazioni dalla stessa \texttt{@collection} riporta quest'ultima in bibliografia, sebbene non direttamente citata.

\section{Comando $\backslash$\texttt{ccite}}

Ecco un esempio di nota composta utilizzando il comando \verb|\ccite|.\footnote{L'argomento � stato sviluppato da \cite{Pantieri:2008} e nel successivo 
\ccite{Pantieri:2009}.}


\newpage
%% SIGLE
\printshorthands


% BIBLIOGRAFIE
Si noti che nelle bibliografie seguenti la lista degli autori/editori � riportata completamente, sebbene venga troncata nelle citazioni. Ci� si pu� ottenere attraverso le opzioni \texttt{maxnames=999} e \texttt{minnames=999}, passate direttamente al comando \verb|\printbibliography|.

% Note da stampare dopo il titolo ma prima della lista dei riferimenti
\defbibnote{notaesempio}{\small\sffamily Questa bibliografia contiene gli esempi, pi� o 
		meno fittizi, citati in questo articolo, esclusi i testi contenuti nei riferimenti 
		bibliografici. I dati relativi alle voci si trovano nel file \texttt{philosophy-examples.bib}.}

\defbibnote{notariferimenti}{\small\sffamily In questa bibliografia si noti come il campo 
\texttt{annotation} venga usato per produrre delle voci commentate. In questo modo possiamo fornire per ciascuna voce un breve sommario}

%% Bibliografia esemplificativa
%\phantomsection{}
%\addcontentsline{toc}{section}{Bibliografia esemplificativa}
%\printbibliography[maxnames=999,minnames=999,prenote=notaesempio,heading=esempio,keyword=Esempio]
%
%% Riferimenti bibliografici
%\phantomsection{}
%\addcontentsline{toc}{section}{\refname}
%\printbibliography[maxnames=999,minnames=999,prenote=notariferimenti,heading=primary,keyword=primary]

\printbibliography
\end{document}
