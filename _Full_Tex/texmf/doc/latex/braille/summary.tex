\documentclass[draft]{article}
\usepackage{fullpage}
\usepackage[puttinydots]{braille}

\newcommand{\mytable}[1]{%
    \enskip\begin{tabular}[t]{r|l} 
    \hline #1 \hline
    \end{tabular}\enskip}

\newcommand{\mytablewith}[2]{%
    \enskip\begin{tabular}[t]{r|l} 
    \multicolumn{2}{c}{#1} \\
    \hline #2 \hline
    \end{tabular}\enskip}

\newcommand{\mytaglist}[1]{%
    \enskip\begin{tabular}[t]{@{}l@{}}
    #1
    \end{tabular}\enskip}


\begin{document}

\title{Summary of Grade 1 and 2 Braille}
\author{William Park}
\date{September 1998, April 1999}
\maketitle


\section{Braille Alphabets}

The following is list of \TeX\ macros and Braille symbols contained in
{\tt braille.sty}.  The user command \verb+\braille{}+ converts
sequence of tags to Braille symbols.  A tag can be one character which
appears as is, or multi characters which must be enclosed with \{\}.
For each tag, a predefined TeX macro is called to print out the
Braille symbols.  For example, \verb+\braille{a{the}b}+ prints out
Braille symbols for {\em a}, {\em the}, and {\em b}.

As usual, \TeX\ collapses multiple spaces to single space, and ignores
spaces at the end of line, and tabs and newlines everywhere.  This means
that for multi-line paragraph, there must be at least one space after
newline (\verb+\n+).

\begin{center}
\mytable{
\braille{a} & a 1 \\
\braille{b} & b 2 \\
\braille{c} & c 3 \\
\braille{d} & d 4 \\
\braille{e} & e 5 \\
\braille{f} & f 6 \\
\braille{g} & g 7 \\
\braille{h} & h 8 \\
\braille{i} & i 9 \\
\braille{j} & j 0 \\
\braille{k} & k \\
\braille{l} & l \\
\braille{m} & m \\
\braille{n} & n \\
\braille{o} & o \\
}
\mytable{
\braille{p} & p \\
\braille{q} & q \\
\braille{r} & r \\
\braille{s} & s \\
\braille{t} & t \\
\braille{u} & u \\
\braille{v} & v \\
\braille{w} & w \\
\braille{x} & x \\
\braille{y} & y \\
\braille{z} & z \\
\braille{{and}} & \{and\} \\
\braille{{ar}} & \{ar\} \\
\braille{{ble}} & \{ble\} \\
\braille{{ch}} & \{ch\} \\
}
\mytable{
\braille{{ed}} & \{ed\} \\
\braille{{en}} & \{en\} \\
\braille{{er}} & \{er\} \\
\braille{{for}} & \{for\} \\
\braille{{gh}} & \{gh\} \\
\braille{{in}} & \{in\} \\
\braille{{ing}} & \{ing\} \\
\braille{{into}} & \{into\} \\
\braille{{of}} & \{of\} \\
\braille{{ou}} & \{ou\} \\
\braille{{ow}} & \{ow\} \\
\braille{{sh}} & \{sh\} \\
\braille{{th}} & \{th\} \\
\braille{{the}} & \{the\} \\
\braille{{wh}} & \{wh\} \\
\braille{{with}} & \{with\} \\
}
\mytable{
\braille{,} & , \{ea\} \\
\braille{;} & ; \{bb\} \{be\} \\
\braille{:} & : \{cc\} \{con\} \\
\braille{.} & . \$ \{dd\} \{dis\} \\
\braille{!} & ! \{ff\} \{to\} \\
\braille{(} & ( ) \{gg\} \{were\} \\
\braille{{``}} & \{`\,`\} ? \{his\} \\
\braille{*} & *	\\
\braille{{''}} & \{'\,'\} \{by\} \{was\} \\
\braille{'} & ' \\
\braille{-} & - \{com\} \\
\braille{/} & / \{st\} \\
\braille{[} & [ \\
\braille{]} & ] \\
\braille{{.`}} & \{.`\} \\	% open single quote (`)
\braille{{'.}} & \{'.\} \\	% close single quote (')
\braille{{percent}} & \{percent\} \\
}
\end{center}

Double quotes are specified by the usual \TeX\ notation of 2
consecutive single quotes (\verb+{``}...{''}+).  Single quotes are
specified by 2 character notation (\verb+{.`}...{'.}+) similar to
actual Braille code.  This is because apostrophe (') and closing
single quote (') are identical in print and in ASCII text.  Also, \%
must be spelled out, because it is an escape character for comment in
\TeX.


\section{Prefix Indicator}

\begin{center}
\mytable{
\braille{{Capital}} & \{Capital\} \\
\braille{{Upper}} & \{Upper\} \\
\braille{{Italic}} & \{Italic\} \\
}
\mytable{
\braille{{Number}} & \{Number\} \\	
\braille{{Letter}} & \{Letter\} \\
}
\end{center}


\section{Contraction for Part of Word}

\begin{center}
\mytablewith{Anywhere}{
\{and\}&and \\
\{ar\}&ar \\
\{ch\}&ch \\
\{ed\}&ed \\
\{en\}&en \\
\{er\}&er \\
\{for\}&for \\
\{gh\}&gh \\
\{in\}&in \\
}
\mytablewith{Anywhere}{
\{of\}&of \\
\{ou\}&ou \\
\{ow\}&ow \\
\{sh\}&sh \\
\{st\}&st \\
\{th\}&th \\
\{the\}&the \\
\{wh\}&wh \\
\{with\}&with \\
}
\mytablewith{Beginning}{
\{be\}&be \\
\{com\}&com \\
\{con\}&con \\
\{dis\}&dis \\
}
\mytablewith{Middle}{
\{bb\}&bb \\
\{ble\}&ble \\
\{cc\}&cc \\
\{dd\}&dd \\
\{ea\}&ea \\
\{ff\}&ff \\
\{gg\}&gg \\
\{ing\}&ing \\
}
\mytablewith{End}{
\{ble\}&ble \\
\{ing\}&ing \\
}
\end{center}


\section{Final Letter Contraction for Middle or End of Word}

\begin{center}
\mytablewith{Prefix \braillebox{46}}{
d&ound \\
e&ance \\
n&sion \\
s&less \\
t&ount \\
}
\mytablewith{Prefix \braillebox{56}}{
e&ence \\
g&ong \\
l&ful \\
n&tion \\
s&ness \\
t&ment \\
y&ity \\
}
\mytablewith{Prefix \braillebox{6}}{
n&ation \\
y&ally \\
}
\end{center}


\section{Initial Letter Contraction for Whole or Part of Word}

\begin{center}
\mytablewith{Prefix \braillebox{45}}{
\{the\}&these \\
\{th\}&those \\
u&upon \\
\{wh\}&whose \\
w&word \\
}
\mytablewith{Prefix \braillebox{456}}{
c&cannot \\
h&had \\
m&many \\
s&spirit \\
\{the\}&their \\
w&world \\
}
\mytablewith{Prefix \braillebox{5}}{
\{ch\}&character \\
d&day \\
e&ever \\
f&father \\
h&here \\
k&know \\
l&lord \\
m&mother \\
n&name \\
o&one \\
\{ou\}&ought \\
}
\mytablewith{Prefix \braillebox{5}}{
p&part \\
q&question \\
r&right \\
s&some \\
\{the\}&there \\
\{th\}&through \\
t&time \\
u&under \\
\{wh\}&where \\
w&work \\
y&young \\
}
\end{center}


\section{Abbreviation for Whole Word}

\begin{center}
\mytable{
ab&about \\
abv&above \\
ac&according \\
acr&across \\
af&after \\
afn&afternoon \\
afw&afterward \\
ag&again \\
ag\{st\}&against \\
alm&almost \\
alr&already \\
al&also \\
al\{th\}&although \\
alt&altogether \\
alw&always \\
\{and\}&and \\
z&as \\
\{be\}&be \\
\{be\}c&because \\
\{be\}f&before \\
\{be\}h&behind \\
\{be\}l&below \\
\{be\}n&beneath \\
\{be\}s&beside \\
\{be\}t&between \\
\{be\}y&beyond \\
bl&blind \\
brl&braille \\
b&but \\
\{by\}&by \\
}
\mytable{
c&can \\
\{ch\}&child \\
\{ch\}n&children \\
\{con\}cv&conceive \\
\{con\}cvg&conceiving \\
cd&could \\
dcv&deceive \\
dcvg&deceiving \\
dcl&declare \\
dclg&declaring \\
d&do \\
ei&either \\
\{en\}&enough \\
e&every \\
f\{st\}&first \\
\{for\}&for \\
fr&friend \\
f&from \\
g&go \\
gd&good \\
grt&great \\
h&have \\
h\{er\}f&herself \\
hm&him \\
hmf&himself \\
\{his\}&his \\
imm&immediate \\
\{in\}&in \\
\{into\}&into \\
x&it \\
}
\mytable{
xs&its \\
xf&itself \\
j&just \\
k&knowledge \\
lr&letter \\
l&like \\
ll&little \\
m&more \\
m\{ch\}&much \\
m\{st\}&must \\
myf&myself \\
nec&necessary \\
nei&neither \\
n&not \\
o'c&o'clock \\
\{of\}&of \\
\{one\}f&oneself \\
\{ou\}rvs&ourselves \\
\{ou\}&out \\
pd&paid \\
p&people \\
p\{er\}cv&perceive \\
p\{er\}cvg&perceiving \\
p\{er\}h&perhaps \\
qk&quick \\
q&quite \\
r&rather \\
rcv&receive \\
rcvg&receiving \\
rjc&rejoice \\
}
\mytable{
rjcg&rejoicing \\
sd&said \\
\{sh\}&shall \\
\{sh\}d&should \\
s&so \\
\{st\}&still \\
s\{ch\}&such \\
t&that \\
\{the\}&the \\
\{the\}mvs&themselves \\
\{th\}&this \\
\{th\}yf&thyself \\
\{to\}&to \\
td&today \\
tgr&together \\
tm&tomorrow \\
tn&tonight \\
u&us \\
v&very \\
\{was\}&was \\
\{were\}&were \\
\{wh\}&which \\
w&will \\
\{with\}&with \\
wd&would \\
y&you \\
yr&your \\
yrf&yourself \\
yrvs&yourselves \\
}
\end{center}

Abbreviation \verb+{in}+ cannot touch any other word or punctuation.
And, abbreviations \verb+{be}+, \verb+{enough}+, \verb+{his}+
\verb+{was}+, and \verb+{were}+ cannot touch punctuations.  

Abbreviations \verb+{by}+, \verb+{into}+, and \verb+{to}+ must adjoin
next word without a space.  But, abbreviations \verb+{and}+,
\verb+{for}+, \verb+{of}+, \verb+{the}+, \verb+{with}+, and letter
\verb+{a}+ can join one another if they occur as consecutive words.


\section{List of All Tags}

\begin{center}
\mytaglist{
about \\
above \\
according \\
across \\
after \\
afternoon \\
afterward \\
again \\
against \\
ally \\
almost \\
already \\
also \\
although \\
altogether \\
always \\
ance \\
and \\
ar \\
as \\
ation \\
bb \\
be \\
because \\
}
\mytaglist{
before \\
behind \\
below \\
beneath \\
beside \\
between \\
beyond \\
ble \\
blind \\
braille \\
but \\
by \\
can \\
cannot \\
cc \\
ch \\
character \\
child \\
children \\
com \\
con \\
conceive \\
conceiving \\
could \\
}
\mytaglist{
day \\
dd \\
deceive \\
deceiving \\
declare \\
declaring \\
dis \\
do \\
ea \\
ed \\
either \\
en \\
ence \\
enough \\
er \\
ever \\
every \\
father \\
ff \\
first \\
for \\
friend \\
from \\
ful \\
}
\mytaglist{
gg \\
gh \\
go \\
good \\
great \\
had \\
have \\
here \\
herself \\
him \\
himself \\
his \\
immediate \\
in \\
ing \\
into \\
it \\
its \\
itself \\
ity \\
just \\
know \\
knowledge \\
less \\
}
\mytaglist{
letter \\
like \\
little \\
lord \\
many \\
ment \\
more \\
mother \\
much \\
must \\
myself \\
name \\
necessary \\
neither \\
ness \\
not \\
o'clock \\
of \\
one \\
oneself \\
ong \\
ou \\
ought \\
ound \\
}
\mytaglist{
ount \\
ourselves \\
out \\
ow \\
paid \\
part \\
people \\
perceive \\
perceiving \\
perhaps \\
question \\
quick \\
quite \\
rather \\
receive \\
receiving \\
rejoice \\
rejoicing \\
right \\
said \\
sh \\
shall \\
should \\
sion \\
}
\mytaglist{
so \\
some \\
spirit \\
st \\
still \\
such \\
th \\
that \\
the \\
their \\
themselves \\
there \\
these \\
this \\
those \\
through \\
thyself \\
time \\
tion \\
to \\
today \\
together \\
tomorrow \\
tonight \\
}
\mytaglist{
under \\
upon \\
us \\
very \\
was \\
were \\
wh \\
where \\
which \\
whose \\
will \\
with \\
word \\
work \\
world \\
would \\
you \\
young \\
your \\
yourself \\
yourselves \\
}
\end{center}


\section{Examples}

\sloppy
\begin{itemize}
\item I like computer 

    {\bf Grade 1:} I  like  computer \\
    \braille{I like computer}
    
    {\bf Grade 2:} I \{like\} \{com\}put\{er\} \\
    \braille{I {like} {com}put{er}}

\item This document was created with LaTeX running on Linux

    {\bf Grade 1:}
    This document was created with LaTeX running on Linux 
    \\
    \braille{This document was created with LaTeX running on Linux}

    {\bf Grade 2:}
    \{Capital\}\{this\} docu\{ment\} \{was\} cr\{ea\}t\{ed\} \{with\}
    LaTeX runn\{ing\} on L\{in\}ux 
    \\
    \braille{{Capital}{this} docu{ment} {was} cr{ea}t{ed} {with} LaTeX
    runn{ing} on L{in}ux}

\item Summary of grade 1 and 2 braille

    {\bf Grade 2:}
    Summ\{ar\}y \{of\} grade \{Number\}1 \{and\} \{Number\}2 \{braille\} 
    \\
    \braille{Summ{ar}y {of} grade {Number}1 {and} {Number}2 {braille}}

\item William Park

    {\bf Grade 1:} William Park \\
    \braille{William Park}

\item September 1998

    {\bf Grade 1:} September \{Number\}1998 \\
    \braille{September {Number}1998}

\item I do not have to know how to read braille in order to produce
    beautifully typeset braille documents.  All I have to do is parse
    ordinary text into tags for which TeX macros exist.  TeX will,
    then, print out predefined symbols associated with each tags.
    
    {\bf Grade 2:}
    I \{do\} \{not\} \{have\} \{to\}\{know\} h\{ow\} \{to\}r\{ea\}d
    \{braille\} \{in\} ord\{er\} \{to\}produce b\{ea\}uti\{ful\}ly
    typeset \{braille\} docu\{ment\}s.  All I \{have\} \{to\}\{do\} is
    p\{ar\}se ord\{in\}\{ar\}y text \{into\}tags \{for\} \{which\} TeX
    macros exist.  TeX \{will\}, \{the\}n, pr\{in\}t \{out\}
    pr\{ed\}ef\{in\}\{ed\} symbols associat\{ed\} \{with\} ea\{ch\}
    tags.
    \\
    \braille{I {do} {not} {have} {to}{know} h{ow} {to}r{ea}d {braille}
    {in} ord{er} {to}produce b{ea}uti{ful}ly typeset {braille}
    docu{ment}s.  All I {have} {to}{do} is p{ar}se ord{in}{ar}y text
    {into}tags {for} {which} TeX macros exi{st}.   TeX {will}, {the}n,
    pr{in}t {out} pr{ed}ef{in}{ed} symbols associat{ed} {with} ea{ch}
    tags.}

\end{itemize}

\end{document}
