\documentclass[a4paper]{article}
\usepackage{german}
\usepackage[T1]{fontenc}
\usepackage{cjhebrew}

\frenchspacing

\newcommand{\cjh}{\textsf{cjhebrew}}

\begin{document}

\section*{Ein kleines \cjh-Beispieldokument}

Das \cjh"=Paket erlaubt es, auf recht einfache Weise hebr"aischen
Text zu setzen, und zwar insbesondere innerhalb von "`normalen"'
Flie"stext. Wenn ich also nun einen beliebigen Vers aus dem
Jeremiabuch einfach so hier einmal hinschreiben wollte, dann
k"onnte ich das problemlos tun, etwa so: \<nim:.s:'U d:bArEykA
wA'ok:lem way:hiy d:bAr:kA liy l:,sA,sOn Ul:,sim:.hat l:bAbiy
k*iy--niq:rA' +sim:kA `Alay yhwh 'E:lohey .s:bA'Ot;> Hier zeigt
sich auch gleich, wie der hebr"aische Text in der richtigen
Schreibrichtung gesetzt und korrekt umbrochen wird.

Hier folgt nun ein l"angerer hebr"aischer Text in erh"ohter
Schriftgr"o"se, der auch zugleich die Verwendung von \verb+\cjLR+
demonstriert:

\def\vers#1{\cjLR{\rmfamily\normalsize#1}}

\Large

\begin{cjhebrew}

\vers{1}~b*:re'+siyt b*ArA' 'E:lohiym 'et ha+s*Amayim w:'et
hA'ArE.s; \vers{2}~w:hA'ArE.s hAy:tAh tohU wAbohU w:.ho+sEk:
`al--p*:ney t:hOm w:rU/a.h 'E:lohiym m:ra.hEpEt `al--p*:ney
ham*Ayim;

\vers{3}~way*o'mEr 'E:lohiym y:hiy 'Or way:hiy--'Or;
\vers{4}~way*ar:' 'E:lohiym 'Et--hA'Or k*iy--.tOb way*ab:d*el
'E:lohiym b*eyn hA'Or Ubeyn ha.ho+sEk:; \vers{5}~way*iq:rA'
'E:lohiym lA'Or yOm w:la.ho+sEk: qArA' lAy:lAh way:hiy--`ErEb
way:hiy--boqEr yOm 'E.hAd;

\end{cjhebrew}


\end{document}
