% This is the file xCJK.tex of the CJK package
%   for testing the xCJK style file.
%
% written by SUN Wenchang <sunwch@hotmail.com>
%
% Version 4.8.2 (29-Dec-2008)

% Read xCJK.txt for more details.

\documentclass[12pt]{article}
\usepackage[bookmarks=true,
            bookmarksopen=true,
            dvipdfm]{hyperref}

\usepackage[CJK, overlap]{ruby}
\usepackage[boldfont]{xCJK}
\usepackage{CJKulem}

\setmainfont{Times New Roman}


% You can specify a different default font for CJK characters.
\setCJKmainfont{Bitstream Cyberbit}

% You can also specify a font for a certain CJK family.
\setCJKfamilyfont{SimpChinese}[BoldFont={SimHei},
                               ItalicFont={FZKaiTi}]{SimSun}
\setCJKfamilyfont{TChinese}{FZMingTiB}
\setCJKfamilyfont{Japanese}{MS Gothic}
\setCJKfamilyfont{Korean}{Batang}

\renewcommand{\rubysep}{-0.3ex}


\begin{document}

\begin{CJK*}{UTF8}{song}


Samples for using the font specified with \verb|\setCJKmainfont|.

\section{本常问问答集}

本常问问答集~(FAQ list)~是从一些经常被问到的问题及其适当的解答中,以方
便的形式摘要而出的。\uline{跟上一版不同的是,其编排结构已彻底改变。}
\textbf{有关新结构的细节,可参考「如何阅读本问答集及了解其编排结构」该
  项中的说明。}


\section{本常問問答集}

本常問問答集~(FAQ list)~是從一些經常被問到的問題及其適當的解答中,以方
便的形式摘要而出的。\uline{跟上一版不同的是,其編排結構已徹底改變。}
\textbf{有關新結構的細節,可參考「如何閱讀本問答集及了解其編排結構」該
  項中的說明。}


\section{この}

この~FAQ~リストは、よくある質問とその答を集め、役に立つようにしたもので
す。\uline{この~FAQ~リストの構造は、以前のものと比べて大幅に変更されて
  います。}\textbf{\ruby{新}{あたら}しい構造に関しては、「この~FAQ~ の
  読み方とその構造」の項目を\ruby{参}{さん}\ruby{照}{しょう}して下さい。}


\CJKspace
\section{이}

이 FAQ 은 자주 반복되는 질문과 그에 대한 대답을 간단명료한 양식으로
모아 엮어졌습니다. \uline{이 FAQ 의 구조는 지난 판에 비하여 획기적으로
  변경되었습니다.}  \textbf{상세한 것은 “이 FAQ 을 어떻게 읽을
  것인가” 라는 대목을 참조하시기 바랍니다.}


\newpage


Samples for using fonts specified with \verb|\setCJKfamilyfont|.

\CJKnospace
\CJKfamily{SimpChinese}
\section{本常问问答集}

本常问问答集~(FAQ list)~是从一些经常被问到的问题及其适当的解答中,以方
便的形式摘要而出的。\uline{跟上一版不同的是,其编排结构已彻底改变。}
\textbf{有关新结构的细节,可参考「如何阅读本问答集及了解其编排结构」该
  项中的说明。}\textit{本常问问答集~(FAQ list)~是从一些经常被问到的问题
  及其适当的解答中,以方便的形式摘要而出的。}


\CJKfamily{TChinese}
\section{本常問問答集}

本常問問答集~(FAQ list)~是從一些經常被問到的問題及其適當的解答中,以方
便的形式摘要而出的。\uline{跟上一版不同的是,其編排結構已徹底改變。}
\textbf{有關新結構的細節,可參考「如何閱讀本問答集及了解其編排結構」該
  項中的說明。}


\CJKfamily{Japanese}
\section{この}

この~FAQ~リストは、よくある質問とその答を集め、役に立つようにしたもので
す。\uline{この~FAQ~リストの構造は、以前のものと比べて大幅に変更されて
  います。}\textbf{\ruby{新}{あたら}しい構造に関しては、「この~FAQ~ の
  読み方とその構造」の項目を\ruby{参}{さん}\ruby{照}{しょう}して下さい。}


\CJKspace
\CJKfamily{Korean}
\section{이}

이 FAQ 은 자주 반복되는 질문과 그에 대한 대답을 간단명료한 양식으로
모아 엮어졌습니다. \uline{이 FAQ 의 구조는 지난 판에 비하여 획기적으로
  변경되었습니다.}  \textbf{상세한 것은 “이 FAQ 을 어떻게 읽을
  것인가” 라는 대목을 참조하시기 바랍니다.}

\end{CJK*}

\end{document}


%%% Local Variables:
%%% coding: utf-8
%%% mode: latex
%%% TeX-master: t
%%% End:
