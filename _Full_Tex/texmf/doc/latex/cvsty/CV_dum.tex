\documentclass[a4paper]{article}

\usepackage{tabularx}

\usepackage{ae}
\usepackage[T1]{fontenc}
\usepackage{CV}

\begin{document}

\pagestyle{empty}

\noindent {\Large \textsc{PEREIRA Gilles}} \hfill 

\smallskip

\noindent \begin{flushright}
birth date: 14/06/1676 \hfill  Knuth Str., 6 \\
328 years \hfill  123453 LaTeX \\
Citizenship: European \hfill EUROPA\\
  Phone: +12.1234567890\\
  Lab. Phone: +12.0987654321\\
  Lab. Fax: +12.0987654320\\
  Email: gilles.pereira@tex.lx\\
\end{flushright}

\smallskip

\specialisation{LaTeX fan}
\section{Professional and training periods}
\begin{CV}[2]{-4.5ex}
%\smallskip
\item[1996] \textbf{Training period at universit� de Linux on RedHat.}
    \textit{Development of experimental devices, data treatments.} (1 month)
\item[1998] \textbf{Training period at the Institute for Irix on SGI.}
  \textit{Development of experimental devices, data treatment, modelization of
    the problem, bibliographic search, redaction of a report in english.} (5
    months)
\item[1999] \textbf{Training period at Institut de MacOS on Mac.}
    \textit{Realization of a device for digitalization of films, digital image
    treatment, data analysis, redaction of a
    report.} (3 months)
\item[1999-2002] \textbf{PhD. at Centre de Recherche GNU under the
    direction of Free Software and Open Source.} \textit{Development of a
    a dynamics model, of numerical differential equation
    solving methods, development of statistical
    corrections, test and validation of
    methods, analysis and programming in the source code of the molecular, use
    of the software, use of parallel computers.}
    (3 years)
\item[2003-2004] \textbf{Marie-Curie fellowship, at Universit� degli studi di
    TeX, Dipartimento di Processing, Pr. A. B. C. Oftex group. Development on
    LaTeX Scaffolds library, for cure of MSoffiss Disease.} \textit{Collaborations
    in European research network, dynamics,Monte
    Carlo simulations, free calculations, principal
    analysis, participation to development of a C++ code, data analysis,
    validation and test of methods, management and use of Linux-Irix computer
    and of an IBM cluster, responsibility of a graduated student, EU language
    learning.} (18 months)
\end{CV}
\section{Education}
\begin{CV}[2]{-4.5ex}
%\smallskip
\item[1993-1994] French secondary school diploma
\item[1994-1996] DEUG in Science of Materials
\item[1996-1997] Licence of Physics
\item[1997-1998] Maitrise (equivalent to a M. sc) of Physics
\item[1998-1999] DEA (one year degree required before doctoral studies) of
  Theoretical Physics
\item[1999-2002] PhD
\end{CV}

\begin{minipage}{0.29\linewidth}
\section{Language Knowledge}
\smallskip
\begin{tabular}{p{2.5cm}p{3cm}}
French  & native \\
English  & fluent \\
Italian & fluent\\
Russian & fair \\
\end{tabular}
\end{minipage}
\hspace{0.02\linewidth}
\begin{minipage}{0.29\linewidth}
\section{Computer skills}
\begin{itemize}
\item UNIX, HP-UX, Linux, Irix, Windows\\
\item Fortran77/90, C/C++, Pascal, Tcl/Tk, Python, Perl, awk/sed,...
\item LaTeX, xdvi, dvi2ps, ps2pdf,...
\end{itemize}
\end{minipage}
\hspace{0.02\linewidth}
\begin{minipage}{0.29\linewidth}
\section{Hobbies}
Traveling(Europe,USA,Russia), violin, cinema, sport (basket, karate), video
games
\end{minipage}
\pagebreak

\section{References}

\noindent These persons are familiar with my professional qualifications and
my personality: 

\begin{table}[h]
\begin{tabular}{@{}lll@{}}
\textbf{Dr. J. Alpha-Omega} \\
Thesis supervisor & Phone: & +33-333333333\\
ILPR & Fax: & +33-333333334\\
2 rue R. Pit & Email: & j.alfa_omega@ilpr.lx\\
33333 Fish\\
Europe \\
\end{tabular}
\end{table}

\begin{table}[h]
\begin{tabular}{@{}lll@{}}
\textbf{Pr. A. B. C. Oftex} \\
Marie-Curie Fellowship supervisor & Phone: & +39-0123456789\\
Universit\'a degli studi di TeX & Fax: & +39-0123456781\\
Dipartimento di Processing & Email: &
a.b.oftex@tex.lx\\
via Venizia 21 \\
13133 Bugs\\
Europe \\
\end{tabular}
\end{table}
\vspace{2\baselineskip}

\section{Thesis}

\noindent Great thesis topic that led to so many results that I decided to
retire after that to let other researcher have an occasion to find also something.

\section{Bibliography}
\begin{CV}[2]{-4.5ex}
%\smallskip
\item[ ]\textbf{Perfect paper}, J. Alfa-Omega and
G. Pereira, \textit{Best. J. of the World.}, 2001, 16, 61-68
\item[ ]\textbf{Another perfect paper}, J. Alfa-Omega and G. Pereira,
\textit{Kick. Ass. J.}, 2002, 67
\end{CV}

\vspace{1cm}

\noindent LaTeX, \today

\end{document}







