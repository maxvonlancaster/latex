%%% ====================================================================
%%%  @LaTeX-file{
%%%     filename        = "dia-driv.tex",
%%%     version         = "1.0h",
%%%     date            = "2-Dec-1994",
%%%     time            = "15:21:37 EST",
%%%     checksum        = "35247 246 1122 9059",
%%%     author          = "Michael Downes",
%%%     address         = "49 Weeks Street
%%%                        North Smithfield, RI 02986, USA",
%%%     email           = "mjd@math.ams.org (Internet)",
%%%     docstring       = "The checksum field above contains: CRC-16
%%%       checksum, number of lines, number of words, and number of
%%%       characters, as produced by Robert Solovay's checksum utility.",
%%%  }
%%% ====================================================================
%
%    Originally published in TUGboat with ltugboat documentstyle.
%    Switched to generic article/twocolumn afterward.
\documentstyle[doc,dialogl]{article}

\title{Interaction tools: {\tt dialog.sty} and {\tt menus.sty}}
\author{Michael Downes}
%\address{49 Weeks Street\\North Smithfield, RI 02895\\U.S.A}
%\netaddress{mjd@math.ams.org}
\date{November 3, 1994}

\hyphenation{pro-duces}

\renewcommand{\thepart}{\arabic{part}}
\renewcommand{\thesection}{\thepart.\arabic{section}}

%    Some customizations of doc.sty parameters.
\def\PrintMacroName#1{}
%
%\renewcommand{\PrintDescribeMacro}[1]{}
%\renewcommand{\PrintDescribeEnv}[1]{}

\setlength{\MacroIndent}{0pt}
\def\MacroFont{\verbatimfont}% doc.sty uses \MacroFont

%    No modules in the files to be processed; so don't bother checking.
\DontCheckModules

\multicoldefaultlayout
\begin{document}
\jobswitch % print only selected parts if \jobname so indicates; see
           % dialogl.sty
\maketitle
\thispagestyle{empty}

\begin{multicols}{2}
\section*{Introduction}

\ifall
This article describes
\fn{dialog.sty} and \fn{menus.sty}, which
provide functions for printing messages or menus on screen and reading
users' responses. The file \fn{dialog.sty} contains basic message and
input-reading functions; \fn{menus.sty} takes \fn{dialog.sty} for its
base and uses some of its functions in defining more complex menu
construction functions. These two files are set up in the form of
\LaTeX{} documentstyle option files, but in writing them I spent some
extra effort to try to make them usable with \plaintex/ or other
common macro packages that include \plaintex/ in their base, such as
\AmSTeX{} or \eplain/.

The appendix describes \fn{grabhedr.sty}, required by \fn{dialog.sty},
which provides two useful file-handling features: (1)~a command
\cw{inputfwh} that when substituted for \cw{input} makes it possible to
grab information such as file name, version, and date from standardized
file headers in the style promoted by Nelson Beebe\Dash and to grab it
in the process of first inputting the file, as opposed to inputting the
file twice, or \cw{read}ing the information separately (unreliable due
to system-dependent differences in the equivalence of \tex/'s \cw{input}
search path and \cw{openin} search path). And (2)~functions
\cw{localcatcodes} and \cw{restorecatcodes} that make it possible for
\fn{dialog.sty} (or any file) to manage internal catcode changes
properly regardless of the surrounding context.

These files and a few others are combined in
\else
This package is part of
\fi
a suite of files that goes
by the name of {\bf dialogl}, available on the Internet by anonymous ftp
from CTAN (Comprehensive \tex/ Archive Network), e.g., \fn{ftp.shsu.edu}
(USA), or \fn{ftp.uni-stuttgart.de} (Europe).
%
\ifall
The file \fn{listout.tex} is a utility for verbatim printing of plain
text files, with reasonably good handling of overlong lines, tab
characters, other nonprinting characters, etc. It uses \fn{menus.sty} to
present an elaborate menu system for changing options (like font size,
line spacing, or how many spaces should be printed for a tab character).
\else
It includes the packages \pkg{dialog}, \pkg{menus}, and \pkg{grabhedr}
and a utility file \fn{listout.tex} for verbatim printing of plain text
files.
\fi

\ifmenus
Here's an example from the menu system of \fn{listout.tex} to
demonstrate the use of some features from \fn{dialog.sty} and
\fn{menus.sty}. First, the menu that you would see if you wanted to
change the font or line spacing:
%%%%%%%%%%%%%%%%%%%%%%%%%%%%%%%%%%%%%%%%%%%%%%%%%%%%%%%%%%%%%%%%%%%%%%
%    This funny way of calling \samepage is (I think) the most
%    convenient way to avoid turning off page breaks for the preceding
%    paragraph.
%%%%%%%%%%%%%%%%%%%%%%%%%%%%%%%%%%%%%%%%%%%%%%%%%%%%%%%%%%%%%%%%%%%%%%
\def\next{\begin{verbatim}\samepage}\next
===============================================
  F  Change font
  S  Change font size
  L  Change line spacing

Current settings: typewriter 8 / 10.0pt.

    Q  Quit         X  Exit        ?  Help
===============================================

Your choice?
\end{verbatim}
Suppose you wanted to change line spacing to 9 points, so you entered
\qc{\l} (lowercase \qc{\L}) and then \verb'9pt', except that on your
first attempt you accidentally mistyped \verb'9pe' instead of
\verb'9pt'. Here's what you would see on screen:
\begin{verbatim}
Your choice? l
Desired line spacing [TeX units] ? 9pe
?---I don't understand "9pe".
Desired line spacing [TeX units] ? 9pt

* New line spacing: 9.0pt
\end{verbatim}
Both lowercase \verb'l' and capital \verb'L' are acceptable responses,
and the value given for line spacing is checked to make sure it's a
valid \tex/ dimension. Before continuing, the internalized version of
the user's value is echoed on screen to confirm that the entered value
was read correctly.

Now here's how the above menu is programmed in \fn{listout.tex}.
A function \cw{menuF} is constructed using \cw{fxmenu}:
\begin{verbatim}
\fxmenu\menuF{}{
F  Change font
S  Change font size
L  Change line spacing
}{
Current settings: &\mainfont &\mainfontsize / %
&\the&\normalbaselineskip.
}
%
\def\moptionF{\lettermenu F}
\end{verbatim}
In the definition of \cw{moptionF}, \cw{lettermenu} is a high-level
function from \fn{menus.sty} that prints \cw{menuF} on screen (given
the argument \verb'F'), reads a line of input from the user, extracts
the first character and forces it to uppercase, then branches to
the next menu as determined by that character. The response of
\verb'l' causes a branch to the function \cw{moptionFL}:
\begin{verbatim}
\def\moptionFL{%
  \promptmesj{%
    Desired line spacing [TeX units] ? }%
  \readline{Q}\reply
\end{verbatim}
If \verb'Q', \verb'X', or \verb'?' was entered, the test
\cw{xoptiontest} will return `true'; then we should skip the dimension
checking and go directly to \cw{optionexec}, which knows what to do
with those responses:
\begin{verbatim}
  \if\xoptiontest\reply
  \else
\end{verbatim}
Otherwise we check the given dimension to make sure it's usable. If so,
echo the new value as confirmation.
\begin{verbatim}
    \checkdimen\reply\dimen@
    \ifdim\dimen@>\z@
      \normalbaselineskip\dimen@\relax
      \normalbaselines
      \confirm{New line spacing:
        \the\normalbaselineskip}%
      \def\reply{Q}%
    \fi
\end{verbatim}
If \cw{reply} was changed to \verb'Q' during the above step,
\cw{optionexec} will pop back up to the previous menu level (normal
continuation); otherwise \cw{reply} retains its prior definition\Dash
e.g., \verb'9pe'\Dash to which \cw{optionexec} will simply say ``I
don't understand that'' and repeat the current prompt.
\begin{verbatim}
  \fi
  \optionexec\reply
}
\end{verbatim}

For maximum portability, \fn{listout.tex} uses in its menus only
lowest-common-denominator ordinary printable ASCII characters in the
range 32\dash 126. Fancier menus can be obtained at a cost of forgoing
system independence, for instance by using em\tex/'s \verb'/o' option
to output the box-drawing characters in the standard PC DOS character
set.
\fi

\subsection*{Notation}

Double-hat notation such as \verb'^^J' is used herein for
control characters, as in \TB{}, although occasionally the alternate
form `{\sc control}-J' is used when the emphasis is away from the
character's tokenized state inside \tex/. A couple of abbreviations
from \fn{grabhedr.sty} are used frequently in the macro code: \cw{xp@}
= \cw{expandafter}, and \cw{nx@} = \cw{noexpand}. Standard
abbreviations from \fn{plain.tex} such as \cw{z@} or \cw{toks@} are
used without special comment.

\ifall
\part{Basic dialog functions: \fn{dialog.sty}}
\fi
\ifdialog\hDocInput{dialog.dtx}\fi

\ifall
\part{Menu functions: \fn{menus.sty}}
\setcounter{section}{0}
\fi
\ifmenus\hDocInput{menus.dtx}\fi

\ifall
%    It's hard to make this part title come out right without assuming
%    \partname is defined.
\part*{Appendix\penalty-10000
Miscellaneous support functions:
\fn{grabhedr.sty}}
\setcounter{section}{0}
\renewcommand{\thepart}{A} % for use in \thesection
\fi
\ifgrabhedr\hDocInput{grabhedr.dtx}\fi

\end{multicols}

\ifdim\textwidth>\textheight
\typeout{^^JWarning: remember to print using LANDSCAPE orientation^^J}
\fi
\end{document}
