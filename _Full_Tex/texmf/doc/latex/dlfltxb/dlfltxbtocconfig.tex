\RequirePackage{etex}
\documentclass[11pt,oneside,a4paper,oldfontcommands,danish,english,article]{memoir}
\usepackage[latin1]{inputenc}
\usepackage{babel}
\usepackage[T1]{fontenc}
\usepackage[widespace]{fourier}

\setlxvchars[\normalfont]

\settypeblocksize{*}{1.3\lxvchars}{1.618}

\setlrmargins{*}{*}{0.7}
\setulmargins{*}{*}{1}
\setlength\marginparwidth{4cm}

\checkandfixthelayout


\hfuzz=30pt

\setfootnoterule[\vfill]{3pt}{0.4\columnwidth}{\normalrulethickness}

\usepackage{color}
\usepackage[colorlinks,breaklinks]{hyperref}
\definecolor{linkcolour}{rgb}{0,0.2,0.6}
\definecolor{citecolour}{rgb}{0,0.6,0.2}
\definecolor{urlcolour} {rgb}{0.8,0,0.8}
  
\hypersetup{
  pdftitle={The dlfltxbtocconfig package},
  pdfauthor={Copyright \textcopyright\ \number\year\ Lars Madsen},
  linkcolor=linkcolour,citecolor=citecolour,
  filecolor=urlcolour,urlcolor=urlcolour,
  plainpages=false,
}
  
\ifpdf\else\usepackage{breakurl}\fi
\usepackage{memhfixc}

\clubpenalty=300
\widowpenalty=300

\usepackage{microtype}

\usepackage{amsmath,amssymb}
\usepackage[amsmath,thmmarks,framed]{ntheorem}
\usepackage[round]{natbib}

\usepackage{dlfltxbcodetips}
\usepackage[loadsampleconfig]{dlfltxbmarkup}
\renewcommand\felineMarginAdjustment{\RaggedLeft}



\usepackage{dlfltxbmisc}
\usepackage{graphicx}

\chapterstyle{article}

\setsecheadstyle{\normalfont\large\bfseries\raggedright}


\reversemarginpar
\reversesidepartrue

\definecolor{shadecolor}{gray}{0.8}
\theorembodyfont{\normalfont}
\theoremseparator{.}
\def\theoremframecommand{\colorbox{shadecolor}}

\theoremstyle{nonumberplain}

\NewShadedTheorem{caveat}{Caveat}

\NewShadedTheorem{remark}{Remark}

% until dlfltxbsd is released we will have to use the following
% construction

\usepackage{fancyvrb}
\fvset{fontsize=\footnotesize}

\newcommand\verbfile{\jobname.vrb}
\newenvironment{sourcedisplay}{%
\par\vskip0.5\baselineskip\noindent
\VerbatimOut{\verbfile}}{%
\endVerbatimOut%
\noindent
\begin{minipage}{0.48\linewidth}
\VerbatimInput{\verbfile}
\end{minipage}
\hfill
\begingroup%
\begin{minipage}{0.48\linewidth}
\begin{framed}
\input{\verbfile}
\end{framed}
\end{minipage}
\endgroup\vskip0.5\baselineskip}


\newenvironment{Sourcedisplay}{%
\par\vskip0.5\baselineskip\noindent
\VerbatimOut{\verbfile}}{%
\endVerbatimOut%
\noindent
\begin{minipage}{\linewidth}
\VerbatimInput{\verbfile}
\end{minipage}
\par\bigskip\noindent
\begin{minipage}{\linewidth}
\begin{framed}
\input{\verbfile}
\end{framed}
\end{minipage}
\vskip0.5\baselineskip}


\pagestyle{plain}

\begin{document}

\title{The \textsf{\jobname} package}
\author{Lars Madsen\thanks{Email: \url{daleif@imf.au.dk}}}
\maketitle


\setsecnumdepth{part}


\chapter{Contents of this package}

This package is not really meant for public use. It is used in the
preparation of
\emph{\href{http://www.imf.au.dk/system/latex/bog}{Introduktion til
    \LaTeX}} my (Danish) \LaTeX\ user 
manual. 

The package takes care of configuring the two main table of contents
used in the book. People should use the code for inspiration (be
aware, it is quite messy) for their own ToC configurations.

One interesting feature used in this version of the package: In the
latest memoir (as of February 2010) there are some added hooks in the
ToC macros that aren't documented in the memoir user manual. In this
package we use these hooks to measure the widest section numbers and
such in the ToC, save it to the main aux file, and then on the next
run, use it to auto adjust the width of the section number boxes.

A new release of the \emph{memoir experimental} package will explain
more about how this works. 





\end{document}

%%% Local Variables: 
%%% mode: latex
%%% TeX-master: t
%%% TeX-source-specials-mode: t
%%% TeX-PDF-mode: nil
%%% End: 

