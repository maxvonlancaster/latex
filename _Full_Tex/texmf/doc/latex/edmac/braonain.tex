\input edmac
\firstlinenum=4
\linenumincrement=4
\vsize 40pc
\hsize 24pc
\bigskipamount=12pt plus 6pt minus 6pt
\expandafter\let\expandafter\numlabfont\the\scriptfont0

\footparagraph{A}
\footparagraph{B}
\footparagraph{C}


\font\nfont = cmr7
\let\notenumfont\nfont
\def\notetextfont{\eightpoint\rm}
\let\Afootnoterule=\relax
\count\Afootins=800
\count\Bfootins=800
\count\Cfootins=800
\makeatletter
\def\xparafootfmt#1#2#3{%
  \normal@pars
  \parindent=\z@ \parfillskip=\z@ plus1fil
  {\notenumfont\printlines#1|}\enspace
%  {\select@lemmafont#1|#2}\rbracket\enskip
       \notetextfont #3\penalty-10 }
\def\yparafootfmt#1#2#3{%
  \normal@pars
  \parindent=\z@ \parfillskip=\z@ plus1fil
%  {\notenumfont\printlines#1|}\enspace
%  {\select@lemmafont#1|#2}\rbracket\enskip
       \notetextfont #3\penalty-10 }
\let\Afootfmt=\xparafootfmt
\let\Bfootfmt=\xparafootfmt
\let\Cfootfmt=\xparafootfmt
\skip\Cfootins=\bigskipamount
\makeatother
\input edstanza.doc
\stanzaindentbase=20pt % This is the default, but we include it
                       % to illustrate the way it should be set.
\setstanzavalues{sza}{4,1,2,1,2,3,3,1,2} % indent multiples; first=hangindent.
  % The above must be nonnegative whole numbers.
\setstanzavalues{szp}{1,5000,10500,5000,10500,5000,5000,5000,0}% line penalties
  % The above must be nonnegative whole numbers.
  % If the first entry is 0, no penalties are passed on to TeX.
\let\endstanzaextra=\bigbreak % ==> \bigskip \penalty -200

\mathchardef\IMM=9999
\def\lbreak{\hfil\penalty-\IMM} % (almost) force line break in foot paragraph


\beginnumbering

\pstart  \centerline{\bf 22}  \pend

\bigskip
\firstlinenum=1000  % do not print line number beside heading

\let\cfootfmt=\yparafootfmt % poem heading use diffferent format
\pstart
\centerline{[Se\'an \'O Braon\'ain cct] chuim Tom\'ais U\'{\i}
\text{Dh\'unlaing}\Cfootnote{{\bf 22} {\it Teideal\/}: Dhuinnluinng T,
Se\'aghan Mac Domhnaill cct B\lbreak}/}
\pend

\pstart
\centerline{[Fonn: M\'airse\'ail U\'{'i} Sh\'uilleabh\'ain (P\'ainseach
             na nUbh]}
\pend

\let\cfootfmt=\xparafootfmt % standard format

\bigskip


\setline{0}     % reset line number
\firstlinenum=4


% each verse starts with \stanza . Lines end with & ; the last line with \& .
\stanza
A \text{dhuine}\Cfootnote{dhuinne T}/ gan ch\'eill do
\text{mhaisligh}\Cfootnote{mhaslaidh T, mhaslaig B}/ an chl\'eir&
is tharcaisnigh naomhscruipt na bhf\'aige,&
na haitheanta \text{r\'eab}\Cfootnote{raob T}/ 's an t-aifreann thr\'eig&
\text{re}\Cfootnote{le B}/ taithneamh do chlaonchreideamh Mh\'artain,&
c\'a rachair \text{'od}\Cfootnote{dod B}/ dh\'{\i}on ar \'Iosa Nasardha&
nuair \text{chaithfimid}\Cfootnote{chaithfam\'{\i}d T}/ cruinn
bheith ar \text{mhaoileann}\Cfootnote{maoilinn B}/ Josepha?&
N\'{\i} caraid Mac Crae chuim t'anama ' \text{phl\'e}\Cfootnote{phleidh T}/&
n\'a Calvin \text{bhiais}\Cfootnote{bh\'{\i}os B}/ taobh
\text{ris}\Cfootnote{leis B}/ an l\'a sin.\&
\stanza
N\'ach damanta an sc\'eal don chreachaire \text{chlaon}\Cfootnote{claon B}/&
ghlac baiste na cl\'eire 'na ph\'aiste&
's do \text{glanadh}\Cfootnote{glannuig T}/ mar ghr\'ein \'on bpeaca r\'o-dhaor&
tr\'{\i} \text{ainibhfios}\Cfootnote{ainnibhfios T, ainnbhfios B}/
\text{\'Eva}\Cfootnote{\'Eabha B}/ rinn \'Adam,&
tuitim ar\'{\i}s f\'e chuing na haicme sin&
tug atharrach br\'{\i} don scr\'{\i}bhinn bheannaithe,&
d'aistrigh b\'easa \text{agus}\Cfootnote{is B}/ reachta na cl\'eire&
's n\'ach \text{tugann}\Cfootnote{tuigionn T}/ aon gh\'eilleadh don Ph\'apa?\&


\stanza
Gach \text{scolaire}\Cfootnote{sgollaire T}/ baoth, n\'{\i}
\text{mholaim}\Cfootnote{mholluim T}/ a cheird&
\text{'t\'a ag obair}\Cfootnote{'t\'ag ccobar T}/ \text{le}\Cfootnote{re B}/
g\'eilleadh d\'a th\'aille&
don \text{doirbhchoin chlaon}\Cfootnote{dorbhchon daor B}/
d\'a ngorthar Mac Crae,&
deisceabal \text{straeigh}\Cfootnote{straodhaig T}/ as an gcoll\'aiste.&
T\'a \text{\text{adaithe}\Cfootnote{fadaighthe B}/
th\'{\i}os}\Cfootnote{fhadoghthe ts\'{\i}os T}/ in \'{\i}ochtar ifrinn,&
gan \text{solas}\Cfootnote{sollus T}/ gan soilse i dt\'{\i}orthaibh dorcha,&
tuigsint an l\'einn, gach \text{cuirpeacht}\Cfootnote{cuirripeacht T}/ d\'ein&
is \text{Lucifer}\Cfootnote{Luicifer T, L\'ucifer B}/ aosta
'na \text{mh\'aistir}\Cfootnote{mhaighistir T}/.\&

\stanza
'S \'e \text{Lucifer}\Cfootnote{Luicifir T}/ f\'ein go follas don
\text{tsaol}\Cfootnote{ts\'aoghuill T}/&
ar \text{aingealaibh}\Cfootnote{aingeallaibh T}/ D\'e do thug \'arghoin&
is J\'udas 'na \text{dh\'eidh}\Cfootnote{dh\'eig T}/ do reic go soil\'eir&
Leanbh r\'o-naomhtha na P\'aise;&
\text{L\'uitir}\Cfootnote{L\'uter B}/ ar\'{\i}s an tr\'{\i}mhadh pearsa aco&
fuair eochair \text{na r\'{\i}ochta bh\'{\i} l\'{\i}onta
d'ainibhfios}\Cfootnote{na .~.~.\ d'ainibhfios] an .~.~.\ l\'{\i}onta
mhallaightheacht B}/;&
a ndeisceabal Crae is gach \text{nduine}\Cfootnote{nduinne T}/
\text{d\'a chl\'eir}\Cfootnote{dh\'a cheird B}/&
ar an obair go l\'eir do chuir pl\'ana.\&

\stanza
Mo theagasc don t\'e 't\'a ag imtheacht ar \text{strae}\Cfootnote{straodh T}/&
a' taisteal an tsaoghail \text{re}\Cfootnote{le B}/ d\'asacht,&
a' dalladh gach n-aon re dlithe Mhic \text{Crae}\Cfootnote{Craodh T}/&
agus \text{B\'{\i}obla\'{\i}}\Cfootnote{b\'{\i}oblaimhe T}/
br\'eige \text{'na}\Cfootnote{iona B}/ \text{mh\'ala}\Cfootnote{mh\'aladh T}/.&
F\'ogair don \text{diabhal}\Cfootnote{diabhuil T}/ a gcliar 's a gcaradas&
is f\'oirfidh ort Dia \text{'t\'a dian a' tathaint}\Cfootnote{at\'a dian
atathant B}/ ort;&
maithfidh Mac D\'e dhuit ar imp\'{\i} na cl\'eire&
agus \text{ciontaigh tu \text{f\'einig}\Cfootnote{feinaidh T}/ 'na
l\'athair}\Cfootnote{umhlaig tu fein iona l\'athair B}/.\&
\endnumbering

\bye
