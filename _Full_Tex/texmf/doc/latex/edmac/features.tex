\input edmac.doc
\hsize 28pc
\vsize 35pc
\cropsetup{45pc}{35pc}{4pc}{4pc}
\headline{Headline \hfil \folio}
\footline{Footline \hfil --\folio--\hfil footline}

\makeatletter
% I'd like a spaced out colon after the lemma:
\def\spacedcolon{{\rm\thinspace:\thinspace}}
\def\normalfootfmt#1#2#3{%
  \normal@pars
  \parindent=0pt \parfillskip=0pt plus 1fil
  {\notenumfont\printlines#1|}\strut\enspace
  {\select@lemmafont#1|#2}\spacedcolon\enskip#3\strut\par}

% And I'd like the 3-col notes printed with a hanging indent:
\def\threecolfootfmt#1#2#3{%
  \normal@pars
  \hsize .3\hsize
  \parindent=0pt
  \tolerance=5000       % high, but not infinite
  \raggedright
  \hangindent1.5em \hangafter1
  \leavevmode
  \strut\hbox to 1.5em{\notenumfont\printlines#1|\hfil}\ignorespaces
  {\select@lemmafont#1|#2}\rbracket\enskip
  #3\strut\par\allowbreak}

% And I'd like the 2-col notes printed with a double colon:
\def\doublecolon{{\rm\thinspace::\thinspace}}
\def\twocolfootfmt#1#2#3{%
  \normal@pars
  \hsize .45\hsize
  \parindent=0pt
  \tolerance=5000
  \raggedright
  \leavevmode
  \strut{\notenumfont\printlines#1|}\enspace
  {\select@lemmafont#1|#2}\doublecolon\enskip
  #3\strut\par\allowbreak}

% And in the paragraphed footnotes, I'd like a colon too:
\def\parafootfmt#1#2#3{%
  \normal@pars
  \parindent=0pt \parfillskip=0pt plus 1fil
  {\notenumfont\printlines#1|}\enspace
  {\select@lemmafont#1|#2}\spacedcolon\enskip
  #3\penalty-10 }
\makeatother

% I'd like the line numbers picked out in bold.
\let\notenumfont=\eightbf
\lineation{page}
\linenummargin{inner}
\firstlinenum=3       % just because I can
\linenumincrement=1
\foottwocol{A}
\footthreecol{B}
\footparagraph{E}
% I've changed \normalfootfmt, so invoke it again for C and D notes.
\footnormal{C}
\footnormal{D}

\beginnumbering

\pstart
This is an \text{example}
  \Afootnote{eximemple C, D.}/
of some text with \text{variant}
  \Afootnote{alternative, A, B.}/
readings recorded as `A' footnotes.  From here on, \text{though}
  \Afootnote{however $\alpha$, $\beta$}/,
we shall have \text{`C'}
  \Bfootnote{B, {\it pace\/} the text}/.
\text{For spice, let us mark a longer passage, but give a different
  lemma for it, so that we don't get a \text{huge}
    \Dfootnote{vast E, F; note that this is
    a `D' note to section of text within a longer lemma}/
  amount of text in a note}\lemma{For spice \dots\ note}
  \Cfootnote{The note here is type `C'}/.
\text{Finally}
  \Efootnote{in the end X, Y}/,
\text{we}
  \Efootnote{us K}/
\text{shouldn't}
  \Efootnote{ought not to L, M}/
\text{forget the}
  \Efootnote{omit to mention the \S, \P}/
\text{paragraphed}
  \Efootnote{blocked M, N}/
\text{notes}
  \Efootnote{variants HH, KK}/,
which are so \text{useful}
  \Efootnote{truly useful L, P}/
when there are \text{a great number of}
  \Efootnote{many, many (preferably)}/
short notes to be \text{recorded}
  \Efootnote{noted: repetition}/.
\pend

\pstart
This is a second paragraph, giving more {\it \text{examples}
  \Afootnote{eximples L, M.}/}
of text with \text{variant}
  \Afootnote{alternative, A, B.}/
readings recorded as `A' footnotes.  From here on, \text{though}
  \Bfootnote{however $\alpha$, $\beta$}/,
we shall have \text{`B'}
  \Bfootnote{B, as correctly stated in the text}/ notes in the text.
\text{For spice, let us mark a longer passage, but give a different
  lemma for it, so that we don't get a {\it \text{huge}
    \Dfootnote{vast E, F; note that this is
    a `D' note to text within a longer lemma.}/}
  amount of text in a note}\lemma{For spice, \dots\ note}
  \Cfootnote{This is a rogue note of type `C'.}/.
\text{Finally}
  \Bfootnote{In the end X, Y}/,
\text{we}
  \Bfootnote{we here K}/
\text{shouldn't}
  \Bfootnote{ought not to L, M}/
\text{forget the}
  \Bfootnote{omit to mention the \S, \P}/
\text{column}
  \Bfootnote{blocked M, N}/
\text{notes}
  \Bfootnote{variants H}/,
which are so \text{useful}
  \Bfootnote{very, very useful L, P}/
when there are \text{many}
  \Bfootnote{lots of Z}/
short notes to be \text{recorded}
  \Bfootnote{recorded and put down: M (repetition)}/.
\pend

\endnumbering
\bye
