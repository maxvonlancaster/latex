\documentclass[oldtoc,a4paper,10pt]{artikel3}
%\usepackage{a4wide}
\begin{document}
\author{J.B.~Rhebergen}
\date{\today}
\title{{\tt fax.cls} Documentation}
\maketitle

\section{Introduction}
This fax class is written to work with the following other classes or
packages:

\begin{tabular}{ll}
{\tt artikel3.cls} &One of the dutch article styles.\\
{\tt lastpage.sty} &A small package that lets you reference the last
page.\\
\end{tabular}

It can however still work without these files. If the {\tt artikel3.cls}
cannot be found the {\tt article.cls} will be loaded instead. If available
{\tt lastpage.sty} will also be loaded. If you do not have {\tt
lastpage.sty} you can set the fax page numbers by hand by using the
\verb|\setcounter{faxpages}{..}| command.

\section{Other files}
The following files should be prepared by the user:

\begin{tabular}{ll}
{\tt from.fax} &containing your name etc.\\
{\tt note.fax} &containing a note which you would like to add.\\
\end{tabular}

If you are using the {\tt dutch} class option you need:

\begin{tabular}{ll}
{\tt van.fax} &containing your name etc. (in dutch)\\
{\tt nb.fax}  &containing a note which you would like to add. (in dutch)\\
\end{tabular}

Of course this can easily be changed or added for any other language. Feel
free to do so.

\section{Options}
The following class options are at your disposal:
\begin{description}
\item[dutch] Switches to dutch language and reads the files {\tt van.fax}
  and {\tt nb.fax} instead of the default {\tt from.fax} and {\tt note.fax}
\item[note] Adds a note to your fax header. This note should be specified
  in the file {\tt note.fax} or if you have used the {\tt dutch} option
  {\tt nb.fax}
\item[nosep] Omits the {\tt faxtopset} length variable which is currently
  set at {\tt 3ex} This white space separates your fax header from the rest
  of the text.
\end{description}

The fax is typeset in \verb|\sf| (sans serif) by default, because I felt
that this would give the most legible results on a fax machine.

\section{Commands}
In the preamble of the fax document the following commands are available:
\begin{description}
\item[$\backslash$to] Specifies what you want in the `to' box. This should
  preferably be in the form of a {\tt tabular} environment. If omitted the
  box will stay empty.
\begin{verbatim}
\to{\begin{tabular}[t]{rl}
  to:      &You\\
  address: &Your place\\
  tel:     &Your number\\
  fax:     &Whatever\\
  \end{tabular}}
\end{verbatim}
\item[$\backslash$from] Specifies what you want in the `from' box. This should
  preferably be in the form of a {\tt tabular} environment. If omitted the
  {\tt from.fax} file will be used.
\begin{verbatim}
\from{\begin{tabular}[t]{rl}
  from:    &Me\\
  address: &My place\\
  tel:     &My number\\
  fax:     &Whatever\\
  \end{tabular}}
\end{verbatim}
  If you want an empty `from' box use \verb|\from{}|.
\item[$\backslash$mymsg] Normally the fax header begins with the message:
  {\sc Tele-Fax Message}. If you have specified the {\tt dutch} option the
  fax header will begin with: {\sc Fax Boodschap} If you want to start with
  your own header (or logo even) you can use the \verb|\mymsg| command like
  this:
  \begin{center}
  \vspace{1.5ex}
  \verb|\mymsg{{\Large\bsf Watch out! here comes a FAX!}}|
  \vspace{1.5ex}
  \end{center}
  If you want absolutely no fax top message just specify \verb|\mymsg{}|
  i.e. an empty command.
\item[$\backslash$bsf] Switches to bold sans serif. i.e. \verb|{\bsf this is bold}|
\item[$\backslash$slsf] Switches to slanted sans serif. i.e. \verb|{\slsf this is slant}|
\end{description}

\section{Suggestions}
To make your faxes even more `beautiful' you could use the {\tt wasysym}
package which provides a \verb|\phone| command. This can be used to get a
cute little telephone icon in your `to' or `from' box. Like this:
\begin{verbatim}
\from{\begin{tabular}[t]{rl}
  from:    &Me\\
  address: &My place\\
  \phone:  &My number\\
  fax:     &Whatever\\
  \end{tabular}}
\end{verbatim}
Don't forget to put \verb|\usepackage{wasysym}| in your preamble though.

For my own convenience I have made a signature package. This provides a
signature for my fax documents. Of course you need to get your signature
scanned first and make the signature available in the form of a {\tt .pcx}
file or {\tt .msp} file. If you want to know more about this compile and
read the {\tt sigdoc.tex} file.

\section{Conversion}
Once you've successfully used the {\tt fax.cls} to make a fax document you
have a choice.

\begin{itemize}
\item Print it, feed it to the fax machine and send it. (trivial)
\item Convert it to a fax-modem compatible file and send it.
\end{itemize}

These days lots of people have a fax-modem. To convert the {\tt .dvi} file
you generated, to some kind of fax-modem compatible file you need a
conversion program. Let me describe how I go about sending a fax.

I am using em\TeX{} for {\tt DOS} to compile my {\tt .tex} documents. The
latest em\TeX{} release, version [4a], has a {\tt fax.cnf} file and
\verb|fax_base.fli| file which facilitate generating {\tt .dvi} files in fax
resolution.

I subsequently convert this fax {\tt .dvi} file to a {\tt .pcx} file, using
the following batch file.

\begin{verbatim}
@echo off
dvidrv dvidot pcx @fax.cnf /fl=-1 %1 %1.p?? %2 %3 %4 %5 %6 %7 %8 %9
\end{verbatim}

This will generate a number of {\tt .pcx} files one for each page of the
fax document. The {\tt .pcx} files are numbered {\tt myfax.p01, myfax.p02,
myfax.p03} if your original {\tt myfax.dvi} has three pages.

The last step is to convert these {\tt .p??} files to one single {\tt .fax}
file. I have found the {\tt 2FAX} program by Hans Harder very useful.
However there was a bug in this program. Herre de Jonge the co\-author of
this {\tt fax.cls} and I have debugged the program and added some
functionality. Actually, Herre did most of the work on debugging the {\tt
2FAX} program while I did most of the work on the {\tt fax.cls}. We've
modified version 1.62 of {\tt 2FAX} and called it version 1.62a. I will
make this debugged version available for ftp as soon as possible. We've
also added some functionality. You now can use wildcards at the {\tt 2FAX}
command line. Like this: \verb|2fax indoc.p?? outdoc.fax /hr|

At this stage you have a file which can be sent by the fax-modem. In the
case of {\tt 2FAX} this file is in the {\tt ZFAX} format. I am using an
unregistered version of {\tt BGFAX} to send (or receive) faxes. {\tt BGFAX}
is written by B.J.Guillot and is a simple but solid command line driven fax
program. It's primarily intended for use with people who operate a bulletin
board but who also want to send or receive faxes. There are various other
programs for {\tt DOS} that will also send {\tt ZFAX} files. I will
probably send the debugged version of {\tt 2FAX} to B.J. Guillot's bulletin
board soon so people can download or ftp it from there. At the moment I
still have to write a readme file and make some minor changes. If you're in
a hurry to get the debugged version of {\tt 2FAX}, mail me and I'll send it
to you.

{\tt jbrhebergen@et.tudelft.nl}\\
{\tt rheberg@morra.et.tudelft.nl}

\end{document}

