%%
%% This is file `formpel.tex',
%% generated with the docstrip utility.
%%
%% The original source files were:
%%
%% formula.dtx  (with options: `exampl')
%% 
%% Copyleft 1997  Andreas Tille
%% 
%% Usage without any waranty
%% 

\documentclass{article}
\usepackage[T1]{fontenc}
\usepackage[nopredefinition]{formula}

\pagestyle{empty}
\textheight 55\baselineskip
\advance\oddsidemargin by -10mm
\textwidth 150mm

\formuladef IQ  {\text{IQ}}  {Intelligenzquotient}  {der}
\formuladef Kl  {K}          {Klugheit}             {die}
\formuladef Du  {D}          {Dummheit}             {die}
\begin{document}
\section*{Wie dumm ist der Zufall}

Herr Rainer Zufall hat einen \IQ von \texttt{???} {(\em Zensiert,
Frau Emma N. Z. Daten-Schutz)}. Dabei ist der \IQ definiert als
\beq
   \IQ = \frac{\Kl}{\Du} \label{IQ}
\eeq
wobei \Kl \Klart \Kltxt des Herrn Zufall und \Du \Duart \Dutxt
dieses werten Herrn ist. \par
\formulaarg {Kli} {K_} {Klugheit bei der T\"atigkeit $X$} {die} {X}
\formulaarg {Dui} {D_} {Dummheit bei der T\"atigkeit $X$} {die} {X}

Angenommen, Herr Zufall stellt sich bei verschiedenen T\"atigkeiten
unterschiedlich dumm an, so l\"a\ss{}t sich (\ref{IQ}) folgenderma\ss{}en
pr\"azisieren:
\beq \label{IQi}
   \IQ = \frac{\sum\limits_{i=1}^M\Kli{i}}{\sum\limits_{i=1}^N\Dui{i}}
\eeq\par
\formulamit {KlT} {K_} {^T} {Klugheit am Tag} {die} {X}
\formulamit {KlN} {K_} {^N} {Klugheit in der Nacht} {die} {X}
\formulamit {DuT} {D_} {^T} {Dummheit am Tag} {die} {X}
\formulamit {DuN} {D_} {^N} {Dummheit in der Nacht} {die} {X}

Wird nun noch in Betracht gezogen, da\ss{} der Zufall im Traum eine ganz
andere Rolle spielt, als im realen Leben, so kann zwischen den Klugheiten
bei Tag (\KlT{X}) und Nacht (\KlN{X}), sowie den Dummheiten zu diesen
Tageszeiten (\DuT{X} bzw. \DuN{X}) unterschieden werden. Bei Reiner
Zufall ist die Klugheit tags\"uber vernachl\"assigbar klein gegen\"uber
der beim Tr\"aumen. Die Dummheit verh\"alt sich genau umgekehrt.

Also ergibt sich aus (\ref{IQi})
\beq
   \IQ \simeq \frac{\sum\limits_{i=1}^M\KlN{i}}{\sum\limits_{i=1}^N\DuT{i}}.
\eeq

\formuladiff Klt {d} {\Kl} {t} {Klugheitszuwachs n-ten Grades} {der}

Da Herr Zufall ein eifriger Leser der Boulevardpresse ist, gilt:
\bea
  \Klt{}  & < & 0 \qquad\forall\quad t > t_0 \\
  \Klt{2} & > & 0 \qquad\forall\quad t > t_0
\eea

\bigskip\bigskip\noindent
Und hier noch einmal alle definierten Gr\"o\ss{}en auf einen Blick:

\medskip
\begin{tabular}{@{}ll}
\IQdoc
\Kldoc
\Dudoc
\Klidoc
\Duidoc
\KlTdoc
\KlNdoc
\DuTdoc
\DuNdoc
\Kltdoc
\end{tabular}
\end{document}
\endinput
%%
%% End of file `formpel.tex'.
