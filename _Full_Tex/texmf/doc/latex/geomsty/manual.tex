% $Header: /usr/home/levy/texts/geomsty/RCS/manual.tex,v 1.12 95/07/03 15:57:22 levy Exp Locker: levy $

\documentstyle[11pt,geom]{article}

\def\leftfoot{\RCSstring$Revision: 1.12 $}
\def\rightfoot{\RCSstring$Date: 95/07/03 15:57:22 $}
\def\RCSstring$#1 ${#1}

\addtolength{\textwidth}{.5in}
\addtolength{\oddsidemargin}{-.25in}
\addtolength{\evensidemargin}{-.25in}
\def\AmS{{\the\textfont2 A\kern-.1667em\lower.5ex\hbox{M}\kern-.125emS}}
\def\amstex/{\AmS-\TeX}         
\def\amslatex/{\AmS-\LaTeX{}}
\def\latex/{\LaTeX{}}

\makeatletter
\def\@verbatim{\par\small\trivlist \item[]\if@minipage\else\vskip\parskip\fi
\leftskip\@totalleftmargin\advance\leftskip 1em \rightskip\z@
\parindent\z@\parfillskip\@flushglue\parskip\z@
\@tempswafalse \def\par{\if@tempswa\hbox{}\fi\@tempswatrue\@@par}
\obeylines \tt \catcode``=13 \@noligs \let\do\@makeother \dospecials}
\makeatother

\newenvironment{wizard}{\medbreak\begin{small}
    \leavevmode\llap{\hbox to \parindent{$*$\hfil}}}{\par\end{small}\medbreak}
\def\preref#1{{\normalshape#1}}
\newtheorem[{\par\large}{\it}]{theorem}{Theorem}[section]
\equationwith{theorem}
\newtheorem[{\par\large}{}]{warning}{Warning}
\def\thewarning{\unskip}
\def\picpath{}

\makeatletter
\ifx\fmtname\format@e
\DeclareSymbolFont{AMSa}{U}{msa}{m}{n}
\DeclareMathSymbol\square           {\mathord}{AMSa}{"03}
\fi
\makeatother

\makeindex

\begin{document}

\title{The {\tt geom} style for \latex/}
\author{Silvio Levy\\Geometry Center\\
University of Minnesota}
\date{Revised for \latex/2$_\epsilon$, July 1995}

\maketitle

\tableofcontents
%\listoffigures

\vfil

\copyright 1992 Silvio Levy. The software described in this document
is available free of charge through the Geometry Center at the
University of Minnesota.  It is distributed without any warranty,
express or implied.

Bug reports and comments about the software and documentation
can be sent to {\tt levy@geom.umn.edu}.  I will take them into
consideration in future releases, but there is no
guarantee that particular bugs will be fixed.

\vfil\eject

\section{Introduction} 

This document describes a \latex/ style called \ix{{\tt geom}},
which is loosely based on Lamport's \verb+book+ 
and \verb+article+ styles \cite{Lamport},
but provides a number of additional features, such as:
\begin{itemize}
\item 
inclusion of \ix{PostScript} figures in the document,
and of \TeX\ text within figures;
\item
automatic creation of index entries and cross-reference
labels where appropriate;
\item
no restriction on the use of macros and special characters in
titles, cross-references, captions, etc.;
\item
greater versatility in defining theorem-like environments;
\item
upright digits in an italics environment (optionally); and
\item
proofing aids such as version numbers and a running index.
\end{itemize}
There are also purely esthetic differences,
many of which can be controlled by the user.

This style had its inception in a set of \latex/ macros
\ixinv{historical notes}%
\ixinv{acknowledgements}%
written for the formatting of \ix{Bill Thurston}'s \ix{{\it
Three-Dimensional Geometry and Topology}}, first by \ix{Yair Minsky}
and subsequently by me.  After three years of gestation, the macros
started being used intensively by some other people---notably Thurston,
\ix{David Epstein} and a number of students at \ix{Warwick
University}---and more modifications, including a complete rewrite,
were made.  The first book published using the \verb+geom+ style
\cite{Epstein:WordProcessing} came out in February 1992.

The Appendix to this document describes \verb+mathfig+, a system for
typesetting \TeX\ labels in Mathematica figures. (Mathematica is
described in \cite{wolfram}).  This system can also be used
independently of \verb+geom+.

This release owes much to the patience of these first users and the
feedback they provided, for which I am very grateful.  Naturally, my
debt to the people on whose work I've built---\ix{Don Knuth},
\ix{Leslie Lamport}, \ix{Rainer Sch\"opf}, \ix{Frank Mittelbach},
\ix{Trevor Darrell} and others---is impossible to quantify. I wish to
take this opportunity to thank them, and also John Rawnsley, who
provided the port to \latex/2$_\epsilon$.  I also thank Al Marden, the
Founding Director of the Geometry Center, for creating the conditions
needed for this work to happen.

\section{Overall Organization}

In order to use this style, you need to be running a version of
\latex/ that supports the so-called \ix{new font selection 
scheme} (NFSS), due to Frank Mittelbach and Rainer Sch\"opf.
If your version of \latex/ is \latex/2$_\epsilon$, you are
automatically using NFSS.  If your version is \latex/2.09, you must
check whether it supports NFSS. Start \latex/ and type
\verb+\show\selectfont+.  If the response includes
%
\begin{verbatim}
> \selectfont=undefined.
\end{verbatim}
%
this is the wrong \latex/.  Don't despair yet; 
the right \latex/ may be available at your
site, but under a different name---try \verb+amslatex+,
or ask your local guru. (Strictly speaking, \amslatex/
\cite{amslatex} is a set of \latex/ style files that emulate \amstex/.
Since their use requires NFSS and the
necessary modifications used to be supplied with the \amslatex/ style files,
the name is sometimes used for \latex/+NFSS.)  If you can't find
\latex/+NFSS at all, see further instructions in \Fullref{Installation}.

Unlike most other \latex/ styles, \verb+geom+ comes in several
files, devoted to different capabilities such as cross-referencing,
figures, indexing and so on.  
The files currently in the distribution are:
$$\hbox to \hsize{\small
\valign{&\hbox{\strut #}\cr
{\tt dvips.chg}&
{\tt geom.sty}&
{\tt geomar10.sty}&
{\tt geomar11.sty}&
{\tt geomar12.sty}&
{\tt geomart.sty}&
{\tt geombk10.sty}&
{\tt geombk11.sty}&
{\tt geombk12.sty}\cr
\noalign{\hfil}
%
driver dependencies&
master file&
changes to \verb+art10.sty+&
changes to \verb+art11.sty+&
changes to \verb+art12.sty+&
changes to \verb+article.sty+&
changes to \verb+book10.sty+&
changes to \verb+book11.sty+&
changes to \verb+book12.sty+\cr
\noalign{\hfil}
%
{\tt geombook.sty}&
{\tt geomenv.sty}&
{\tt geomeqns.sty}&
{\tt geomfig.sty}&
{\tt geomfnt.sty}&
{\tt geomindx.sty}&
{\tt geompsfi.sty}&
{\tt multicol.doc}&
{\tt multicol.sty}\cr
\noalign{\hfil}
%
changes to \verb+book.sty+&
environments&
option file&
high-level figure support&
fonts and special symbols&
index and glossary&
PostScript figure support&
source for \verb+multicol.sty+&
two-column formatting\cr
}}$$

The file \verb+geom.sty+ is the master, and it is read in when
\verb+geom+ is called as an option to a main document style, which can
be either \verb+book+ or \verb+article+.
That is, to use the stuff described in this document,
\ixinv{calling {\tt geom}}%
your text file should start with a line such as
\begin{verbatim}
\documentstyle[12pt,geom,...]{book}
\end{verbatim}
When \verb+geom.sty+ is loaded in, it guesses what your main style is,
and reads in some changes to that style, contained in the files
\verb+geombook.sty+, \verb+geombk12.sty+, and so on.

%Each of the following sections is devoted to one major feature of
%\verb+geom+.  Where appropriate, I discuss the file(s) where the feature
%is implemented, to allow you to make your own modifications.

The file \verb+multicol.sty+ is part of the package files developed by
Frank Mittelbach and Rainer Sch\"opf at the University of Mainz.
It is accompanied by \verb+multicol.doc+, which is a ``source'' file
with documentation.  To obtain more of the excellent software written
by Mittelbach and Sch\"opf, check the CTAN archives (see the
Installation section below).

The rest of this document describes the installation of \verb+geom+
(skip if it's already installed), its major features,
and how you can take advantage of them and (sometimes) modify them.
Some material that is either very technical or unlikely to be needed
except under unusual circumstances is printer in smaller type and
introduced by an asterisk $*$.

\section{Installation}

You only need to read this section if you are in charge of installing
\verb+geom+ at your site.  Installation is very simple if your \latex/
supports NFSS and your \TeX-to-PostScript driver is Tom Rokicki's
\verb+dvips+ \cite{Rokicki}.  

To check whether you have \latex/+NFSS, see the
beginning of \Fullref{Overall Organization}. If you don't,
it's time to switch over to \latex/2$_\epsilon$.
Get it by ftp from one of the worldwide CTAN archives
(currently \verb+pip.shsu.edu+,
\verb+ftp.uni-stuttgart.de+, and \verb+tex.ac.uk+).

If your driver is not
\verb+dvips+, you have two choices: either install \verb+dvips+, which
is almost certainly a good idea (get it by ftp from a CTAN archive); or make a
copy of the file \verb+dvips.chg+ in this distribution, and modify it
so it writes the \verb+\special+ commands required by your driver.  If
you adapt \verb+dvips.chg+ for your driver, give the new file a
different name, and replace with this name all occurrences of
\verb+dvips.chg+ in the rest of this section.  Also, please send me a
copy of the file, so I can include it in future distributions.

As I was saying before I got sidetracked, installation is simple.
If you're on a UNIX system, edit the \verb+Makefile+ to make sure the 
pathnames are correct for your site, and run \verb+make install+.  If you're
not on a UNIX system, here are the step-by-step instructions:

Place the files listed in the previous section (all the distribution
files that end in \verb+.sty+, \verb+.chg+ or \verb+.doc+) in the
system-wide directory for \TeX\ input files.  Create a symbolic link
\verb+driver.chg+ in this directory, pointing to \verb+dvips.chg+.
(If your system does not support symbolic links, copy \verb+dvips.chg+
to \verb+driver.chg+.)

Copy the file \verb+mathlabels.orig+ to \verb+mathlabels+ and edit its
first line to be the correct pathname of the perl executable on your
system.  (If you don't have \verb|perl| at all, you don't know what
you're missing---get it as soon as possible.)  Decide on a directory
where you want to install the auxiliary scripts \verb+mathlabels+ and
\verb+geomfix+ (which are used by \verb+mathfig+), and put copies of
them there.  Copy the file \verb+mathfig.orig+ to \verb+mathfig+ and
edit the lines at the top to specify the location of the
\verb+mathlabels+ and \verb+geomfix+ scripts, then copy
\verb+mathlabels+ to a system-wide directory for executables.  
%On a
%UNIX system, copy \verb+mathfig.1.orig+ to
%\verb+mathfig.1+ and edit it to correctly mention the loations of the
%\verb+mathlabels+ and \verb+geomfix+ scripts, then
%install the manual page \verb+mathfig.1+ in the appropriate place.

Finally, copy \verb+math.pro+ and \verb+illustrator.pro+ into the
directory where your \TeX-to-PostScript driver looks for prolog files.
(This assumes your driver supports the inclusion of prolog files;
switch to \verb+dvips+ if it doesn't.)

\section{No More Fragile Commands}

In standard \latex/, commands such as \verb+\section+ have {\it moving
arguments}, that is, arguments that migrate to an auxiliary file {\it
jobname}\verb+.aux+ to be used later in typesetting the table of
contents and so on \cite[33--34]{Lamport}.
In this process, macros in the body of the argument
are expanded, so that when you type, say,
\begin{verbatim}
\section{Annoying \TeX nicalities}
\end{verbatim}
what goes into the auxiliary file is something like
\begin{verbatim}
\@writefile{toc}{\string\contentsline\space {chapter}{\string\numberline\space 
{1}Annoying T\kern -.1667em\lower .5ex\hbox {E}\kern -.125emXnicalities}{1}}
\end{verbatim}
where the second line contains the expansion of the macro \verb+\TeX+.

There are several problems with this. First, the expansion, and
consequently the line where it occurs in the auxiliary file, can be
very long, which in some implementations means that \TeX\ cannot
read the line back in.  Secondly, the expansion might not
make sense, or might do the wrong thing, at the time when it is read in again.
Even worse, the expansion 
of \verb+\section{\small foo}+, for example, actually
causes an infinite loop.

\begin{wizard}
Experienced users of \TeX\ who
want to know why this happens can try typing the following lines to
plain \TeX:
\begin{verbatim}
\def\foo{\let\bar\foo}
\foo                    %Nothing much happens
\immediate\write0{\foo} %Disaster strikes
\end{verbatim}
The problem is that at the time of the \verb+\write+ no assignment is
taking place, so the \verb+\foo+ at the end of the expansion is not
handled as a literal.
\end{wizard}

Standard \latex/'s solution to these problems is for the user to
precede ``fragile'' macros with the \verb+\protect+ command
\cite[33--34]{Lamport}.  By contrast, in the \verb+geom+ style,
you don't have to worry about ``fragile'' commands
in the following situations:
\begin{itemize} 
\item
in the arguments of the sectioning commands
\verb+\chapter+, \verb+\section+, and so on;
\item
in the optional argument to the \verb+theorem+
environment and its relatives (\Fullref{Theorems and Their Friends});
\item
in the arguments to \verb+\caption+, \verb+\fig+ and its relatives
(\Fullref{PostScript Figures}); 
\item
in the arguments to \verb+\markright+ and \verb+\markboth+;
\item
in the argument to \verb+\cite+.
\end{itemize} 
This covers almost all important situations where fragile
commands need to be \verb+\protect+ed \cite[151]{Lamport}.  In future
versions I hope to extend this permissiveness to the remaining
situations.

Notice, however, that the \verb+\verb|...|+ construction continues to
be forbidden inside arguments to most commands.  This is due to
reasons deeply embedded in \TeX\ itself, and is unlikely ever to 
change.

\begin{wizard}
In some circumstances you might want the old \latex/ behavior of
macro expansion inside the argument of \verb+\chapter+, say.
For instance, you might have a locally defined macro whose definition
will be unknown at the time the table of contents is read.
In this case you can use the following construction:
\begin{verbatim}
\edef\mystring{...} % where ... stands for the argument to be expanded
\expandafter\chapter\expandafter{\mystring}
\end{verbatim}
\end{wizard}

\section{Labels and Cross-References}

In standard \latex/, the argument to the \verb+\label+ command cannot
contain anything but characters (after expansion).  In the \verb+geom+
style, this restriction is lifted, so labels contain just about
anything.  For example, \verb+\label{\TeX}+ and
\verb+\label{Isometries of $\bold R^2$}+ are acceptable.  Macros are
no longer expanded.

One advantage of this flexibility is that subdivisions can be
automatically labeled by their titles.  In the \verb+geom+ format, by
default, every sectioning command such as \verb+\section+ generates a
label with its title. If this feature causes you to run out of
memory, you can turn it off by saying \verb+\autolabelfalse+ in your
document's preamble.

To refer to the place where a label occurs, you can still use the
\verb+\ref+ command, as in standard \latex/.  This gives a bald
number---for example, \verb+\ref{+{\tt Isometries of }\verb+$\bold R^2$}+ will
print ``8.3'' if Section~8.3 starts with 
\verb+\section{+{\tt Isometries of }\verb+$\bold R^2$}+ 
while automatic labeling is in effect.

A new feature is that you can
replace \verb+\ref+ by \verb+\fullref+ or \verb+\Fullref+, to get
``section 8.3'' or ``Section 8.3'', respectively.
The way \verb+\fullref+ works is by storing in a certain register the
type of environment in which a label was created.  Thus, when
a new section starts, this register gets the value ``section''.  When
a \verb+\begin+ command introducing
one of the environments described in \Fullref{Theorems and Their Friends} is encountered
inside this section, the register gets updated to the type of that
environment, say ``theorem''.  It reverts to ``section'' when
the matching \verb+\end+ command is read.

Even if you have nested
environments, labels placed in each of them are assigned the correct
type. Consider the input 
%
\begin{verbatim}
\newtheorem[{\par\large}{\it}]{theorem}{Theorem}[section]
\equationwith{theorem}                  % These two lines are explained
...                                     % in later sections
\begin{theorem}[Erd\"os]
A theorem both deep and profound / Is that a circle is round:
\begin{equation}
\mathrm{curvature}\,S^1=\mathrm {const}.
\label{circle}
\end{equation}
In a paper by Erd\"os / Written in Kurdish / A counterexample is found.
\label{Kurdish}
\end{theorem}
\end{verbatim}
%
which gives
%
\begin{theorem}[Erd\"os]
A theorem both deep and profound / Is that a circle is round:
\begin{equation}
\mathrm{curvature}\,S^1=\mathrm {const}.
\label{circle}
\end{equation}
In a paper by Erd\"os / Written in Kurdish / A counterexample is found.
\label{Kurdish}
\end{theorem}
Now \verb+\Fullref{Erd\"os}+ gives ``\Fullref{Erd\"os}'',
\verb+\Fullref{circle}+ gives ``\Fullref{circle}'', and
\verb+\Fullref{Kurdish}+ gives ``\Fullref{Kurdish}''.  Furthermore,
\verb+\ref*{Erd\"os}+ gives ``\ref*{Erd\"os}'', that is, it prints the
(optional) tag associated with theorems and the like.  Of course you
can also use \verb+\fullref*+ and \verb+\Fullref*+.

\begin{wizard}
If you disagree with \latex/'s idea of what the label type should be
at a particular point, you can override is using \verb+\setlabeltype+.
For example, if we had \verb+\setlabeltype{formula}+ right before the
\verb+\label+ command in the previous example, the result of
\verb+\Fullref{circle}+ would be ``Formula~\ref{circle}'' instead of
``\Fullref{circle}''.
\end{wizard}

Before \verb+\ref+ actually prints the number of a cross-reference, it
applies the macro \verb+\preref+ to it.  By default, this macro does
nothing; but you can redefine it to do whatever you want with the
number.  If you want the number to appear upright even if the
surrounding text is in italics, you can say
\begin{verbatim}
\def\preref#1{{\normalshape#1}}
\end{verbatim}
The output of \verb+\pageref+ and of \verb+\cite+ is also passed to
\verb+\pageref+ before being printed.

\section{Theorems and Their Friends}

Standard \latex/ provides the \verb+\newtheorem+ command to create
environments for statements like theorems, lemmas, and so on.
The basic idea can be adapted for other types of text that should
stand out, such an exercises, conventions, and so on.
Not all of these should be treated typographically the same: for
example, you might want exercises in smaller type.

The \verb+geom+ style extends \latex/'s \verb+\newtheorem+ command
to allow you to do this easily.  Recall from \cite[58]{Lamport} that
\verb+\newtheorem+ has two required arguments: the name of the new
environment (say ``theorem''), and the text used to introduce it (say
``Theorem''). In standard \latex/ it also takes an optional argument,
whose meaning depends on whether it comes before or after the second
required argument.

In the \verb+geom+ style, you can use yet another optional argument,
right after \verb+\newtheorem+, to tell \latex/ how this environment
differs from a vanilla theorem-like environment.  This optional
argument should consist of two sequences of commands, each surrounded
by braces, like this:
%
\begin{verbatim}
\newtheorem[{\par\large}{\it}]{theorem}{Theorem}[section]
\end{verbatim}
%
(This is the same definition used for the
example in \Fullref{Labels and Cross-References}.)

The first sequence of commands in braces, \verb+\par\large+, will be
executed as soon as \latex/ sees a \verb+\begin{theorem}+.  This means
the theorem will come out in larger type (see the output in
\Fullref{Labels and Cross-References}).  \latex/ implicitly starts a
new level of braces when it encounters a \verb+\begin+
command---changes inside the environment are local, so the
\verb+\large+ will cease to have effect at the end of the environment.

The second sequence of commands is inserted right after the
introductory text {\bf{\def\preref#1{#1} \Fullref{Erd\"os}} (Erd\"os)}.  It affects
everything from that point till the end of the environment.  Here
we've used \verb+\it+, so the statement of the theorem is printed in
italics.  Note that, unlike the standard \verb+article+ and
\verb+book+ styles, \verb+geom+ does not italicize theorems by
default.  If we didn't have the \verb+\it+ in the \verb+\newtheorem+
command, the text of the theorem would be upright.

\begin{wizard}
You can also control the appearance of the introductory text itself 
by changing the definition of \verb+\theoremintro+, which by default is
%
\begin{verbatim}
\def\theoremintro#1{\normalshape\bf#1}
\end{verbatim}
%
The argument \verb+#1+ stands for the introductory text.
\end{wizard}

You need two pairs of braces in the new optional argument to
\verb+\newtheorem+, even if one is empty.  If both are empty,
the behavior is the same as if the optional argument weren't there.

Another enhancement to \verb+\newtheorem+ in 
the \verb+geom+ style is automatic labeling.  Suppose you say
%
\begin{verbatim}
\newtheorem{definition}[theorem]{Definition}
\end{verbatim}
\newtheorem[{\setlabeltype{def.}}{}]{definition}[theorem]{Definition}
%
(recall from \cite[59]{Lamport} that the use of \verb+[theorem]+ causes
definitions to be numbered in the same sequence as theorems,
assuming the \verb+theorem+ environment has been previously defined).
%
Now saying
%
\begin{verbatim}
\begin{definition}[universe]
The universe is a Hubble bubble.
\end{definition}
\end{verbatim}
prints
\begin{definition}[universe]
The universe is a Hubble bubble.
\end{definition}
%
and implicitly calls \verb+\label{universe}+.  (As
mentioned in \Fullref{Labels and Cross-References}, you can turn this
feature off by saying \verb+\autolabelfalse+.)  The environment knows
its type at the time the label is made, so saying
\verb+\Fullref{universe}+ will print ``Definition~\ref{universe}''.

\begin{wizard}
The type is usually just the environment name, but it can be
overridden by including \verb+\setlabeltype+ (\Fullref{Labels and
Cross-References}) in the first optional argument to
\verb+\newtheorem+.  Thus if the \verb+definition+ environment had
been created with
\begin{verbatim}
\newtheorem[{\setlabeltype{def.}}{}]{definition}[theorem]{Definition}
\end{verbatim}
\verb+\Fullref{universe}+ would appear as ``\Fullref{universe}''.
\end{wizard}

The preferred style of some publishing houses calls for digits and
punctuation to be typeset upright in theorems and such, amidst
surrounding text in italics.  The control sequence
\verb+\specialdigits+ achieves this.  It causes each digit in text, as
well as parentheses, commas, colon, semicolons and periods, to
effectively be preceded by \verb+\normalshape+.  You can use
\verb+\specialdigits+ after \verb+\it+ in the definition of a new
environment with \verb+\newtheorem+.

\begin{wizard}
The way \verb+\specialdigits+ works is by making digits and
punctuation into active characters (changing their \verb+\catcode+s to
13). This has two consequences.  First, characters that arise from
macro expansion don't get the special treatment, because their
\verb+\catcode+s are frozen beforehand; this is true, in particular,
of the expansion of \verb+\ref+ commands (but see the end of
\Fullref{Labels and Cross-References}).

The second consequence is that the digits and decimal point no longer
work in specifying dimensions: if you write \verb+\vskip 1in+ while
\verb+\specialdigits+ is in effect, you get an error message
complaining that a number was expected.  To get around this, you can
temporarily cancel the specialness with \verb+\regulardigits+.  If you
run into this problem more than once or twice, it's best to create a
macro, outside the scope of \verb+\specialdigits+, that expands to the
desired command (\verb+\def\inch{\vskip 1in}+).
\end{wizard}

\section{The Proof Environment}

Theorems and proofs go hand in hand.  
The \verb+proof+ environment in \verb+geom+, in its most basic form,
prints {\it Proof.} at the beginning of the proof, and an end-of-proof
symbol $\square$ at the end.  With an optional argument it behaves
like this: 
%
\begin{verbatim}
\begin{proof}[Poincar\'e conjecture]
Notice first that ...
... as we wished to show.
\end{proof}
\end{verbatim}
%
\begin{proof}[Poincar\'e conjecture]
Notice first that \dots\ as we wished to show.
\end{proof}

\begin{wizard}
The formatting of the proof's introductory text is controlled by
the macro \verb+\proofintro+, whose single argument is the optional
argument to \verb+\begin{proof}+---in this case, ``Poincar\'e
conjecture''.  If the optional argument is not used,
\verb+\proofintro+ is passed an empty argument.
The default setting of \verb+\proofintro+ is
%
\begin{verbatim}
\def\proofintro#1{\def\@tempa{#1}%
  {\it Proof\ifx\@tempa\empty\else\ of #1\fi. }\ignorespaces}
\end{verbatim}
%
You can probably figure out how it decides whether to write 
{\it Proof} or {\it Proof of\dots}. You can change its definition to
obtain other effects.  

Likewise, the formatting of the box at the end of the proof is
controlled by the macro \verb+\provedboxcontents+, which is entirely
analogous.  Its standard setting is
\begin{verbatim}
\def\provedboxcontents#1{\def\@tempa{#1}%
  \ifx\@tempa\empty$\square$\else\fbox{\small#1}\fi}
\end{verbatim}
\end{wizard}

To identify your proofs with the number of their corresponding
theorems, lemmas, corollaries, etc., use \verb+\setprooftag+ in the
definition of these environments.  For instance, after you say
%
\begin{verbatim}
\newtheorem[{\setprooftag}{\it}]{theorem}{Theorem}[section]
\newtheorem[{\setprooftag}{\it}]{corollary}[theorem]{Corollary}
\end{verbatim}
%
\verb+\begin{proof}+ will behave as if you were passing it as
its optional argument the number of the most recent theorem or corollary.
Nested proofs keep track of their arguments correctly, so the box at
the end of each gets the same tag as the {\it Proof of\dots} at the
beginning. 

However, certain situations are too confusing for \verb+\setprooftag+.
For example, if you state a corollary between a theorem and its proof,
the proof will incorrectly get the corollary's number, which is more
recent.  In this situation you have to use the optional argument.  Of
course, it's best to use \verb+\ref{...}+ as the argument, rather than
an explicit number, which might change later.

If a proof
ends with a displayed formula, it is recommented that the proved box
be placed next to the formula (as if it were a tag).  This is done
by inserting \verb+\proved+ before the closing \verb+$$+.  
Likewise, if the last line of the proof is part of an \verb+\item+,
you can use \verb+\proved+ before the closing \verb+\end{enumerate}+
(or whatever) so the proved box won't be set on a line by itself.
In even more difficult
cases---for instance, a formula that already has a tag, or is part of a 
multi-line display---plop \verb+\box\provedbox+ where you want the box to be.

\section{Equations}

The \verb+geom+ style reinstates the plain \TeX\ macro
\verb+\eqalign+, which is not part of standard \latex/ but is
nonetheless very useful.  Unlike the \verb+eqnarray+ environment,
\verb+\eqalign+ creates math alignments that are not treated as whole
lines, and so can be combined with other elements.

The example in \Fullref{Labels and Cross-References} has
\verb+\equationwith{theorem}+ after the \verb+theorem+
environment is created.  
This causes equations made with the \verb+equation+ and
the \verb+eqnarray+ environments to be numbered in the same sequence
as theorems.  You can replace \verb+theorem+ with any environment that
was defined with \verb+\newtheorem+ and that has a counter with its name
(the environment has a counter with its name if 
there was no optional argument between
the first and second required arguments to \verb+\newtheorem+).

If you load the optional style \verb+geomeqns+---that is,
if your document starts with
\begin{verbatim}
\documentstyle[geom,geomeqns,...
\end{verbatim}
the tags in the \verb+equation+ and the \verb+eqnarray+ environments
are printed on the left, in boldface.  This makes them visually
similar to the tags that introduce theorems, so it is generally
appropriate to use this style if you're numbering equations and
theorems in the same sequence.

\section{The {\mediumseries \tt Figure} and {\mediumseries \tt Table} environments}

The \verb+geom+ style supports the \verb+figure+ and \verb+table+
environments of standard \latex/, and, as mentioned above, if you use
the \verb+\caption+ command you don't have to worry about fragile
macros inside its argument.  However, I recommend that you use
instead the new \verb+Figure+ and \verb+Table+ environments, which
provide some bells and whistles.

These environments take three arguments: a cross-reference label, a title and
a legend (or explanatory caption). The command
%
\begin{verbatim}
\begin{Figure}{torus}{The square torus}
{A torus can be obtained, topologically,
by gluing together parallel sides of a square.}
$$
\begin{picture}(1,1)
...     % lots of \put commands
\end{picture}
$$
\end{Figure}
\end{verbatim}
%
prints
%
\begin{Figure}{torus}{The square torus}
{A torus can be obtained, topologically,
by gluing together parallel sides of a square.}
$$
\unitlength=1in
\begin{picture}(1,1)
\put(0,0){\vector(1,0){1}}
\put(0,0){\vector(0,1){1}}
\put(1,0){\vector(0,1){1}}
\put(0,1){\vector(1,0){1}}
\end{picture}
$$
\end{Figure}
%

\noindent
and in addition it creates a cross-reference label \verb+torus+ and an
entry in the table of contents with the text ``The square torus''.
As you can see,
the figure number and the title appear in
boldface, the legend in a lighter face. In proofing mode
(\Fullref{Proofing Aids}), the file
name also appears under the figure, in tiny letters.  If you don't
want a legend, just write \verb+{}+ after the first two arguments.
(In other words, the legend is {\it not} an optional argument---it is
simply an argument that sometimes happens to be empty.)
There is an optional argument to the \verb+\Figure+ environment, consisting
of a subset of the letters \verb+htbp+ and indicating a location 
where the figure may be placed \cite[176]{Lamport}; it should come
before all other arguments.
All of this applies to the \verb+Table+ environment as well.

\section{PostScript Figures}

Most often your picture won't be made with \latex/ commands; it will
exist in a separate PostScript file.  Inclusion of PostScript figures
is still a bit of a bugaboo for many \TeX\ users, in spite of the
existence for many years now of Trevor Darrell's excellent macro
package \ix{Psfig/\TeX} \cite{psfig} and others.

The \verb+geom+ style automatically loads the file
\verb+geompsfi.sty+, a modified version of \ix{Psfig/\TeX}
(incorporating some improvements by \ix{Stephan Bechtols\-heim} and
others of my own).  It allows you to include PostScript figures very
simply: 
%
\begin{verbatim}
\fig{PStorus}{The square torus}{A torus can be obtained, topologically,
by gluing together parallel sides of a square.}
\end{verbatim}
%
\fig{PStorus}{The square torus}{A torus can be obtained, topologically,
by gluing together parallel sides of a square.}

The \verb+\fig+ macro takes the same three arguments (plus an optional
one) as the \verb+Figure+ environment.  But the first argument, in
addition to being used to make a label, is also the name of the file
where the PostScript figure is stored (minus the \verb+.ps+ suffix).
Thus \Fullref{PStorus} is stored in the file \verb+PStorus.ps+.

To get two or three figures side by side, sharing the same caption,
use \verb+\doublefig+ or \verb+\triplefig+.  The PostScript file names
for \verb+\doublefig{foo}...+ would be in this case would be
\verb+foo-1.ps+ and \verb+foo-2.ps+, and likewise for
\verb+\triplefig+.  There is no similar mechanism for more than three
figures.

The directory where the figures are to be found is governed by the
macro \verb+\picpath+.  For example, if the directory is
called \verb+/u/levy/pictures+, you should say 
\begin{verbatim}
\def\picpath{/u/levy/pictures/}
\end{verbatim}
at the beginning of your document.  (You need the \verb+/+ at the end.)

\begin{wizard}
You can also use a relative pathname, such as \verb+pictures/+, or even
the empty string (or \verb+./+) for the current directory.  
However, the program that sends
your \verb+.dvi+ file to the printer may not find the pictures since
it is likely to be invoked from a different directory.

If a pathname includes characters that are special to \TeX, you need
to make then unspecial in order to define \verb+\picpath+.  Here is an
example: 
\begin{verbatim}
\begingroup     % start a group so the change in _ is local
\catcode`\_=12  % make _ behave like an ordinary character
\gdef\picpath{/home/tampa_a/fac/wpt/pictures/} % this def is global
\endgroup       % now _ reverts to its usual meaning
\end{verbatim}
\end{wizard}

If you have a long document with lots of pictures, it is better to
organize them into sections and chapters.  In that case you should
leave \verb+\picpath+ unset, and set the macro \verb+\masterpicpath+
to the master picture directory.  If I were using this scheme in this
document, I might say
%
\begin{verbatim}
\def\masterpicpath{/home/levy/pictures/}
\end{verbatim}
%
and put \Fullref{torus} in file
\verb+/home/levy/pictures/+{\tt\ref{PostScript Figures}}\verb+/PStorus.ps+.
(With the \verb+book+ style, which has chapters, I would need
\verb+chap3/+{\tt\ref{PostScript Figures}} 
instead of just {\tt\ref{PostScript Figures}}.)

The way \verb+\fig+ works is by calling
the command \verb+\psfig+ \cite{psfig}, which creates a \TeX\ \verb+\vbox+
of the right size and containing instructions for the figure to be
included at the time of printing.  You normally won't use 
\verb+\psfig+ directly, since \verb+\fig+ calls it automaticaly.

\begin{wizard}
However, occasionally you many want to use \verb|\psfig| directly to
take advantage of this macro's many options, such as resizing.  See
the Psfig/\TeX\ manual \cite{psfig} for details; but some of the newer
options to Psfig/\TeX\ are not supported here.
\end{wizard}

In order for the \verb+\psfig+ command to know what size box to make,
it needs to find your PostScript figure file, and
the file must have near the top a line like this:
\begin{verbatim}
%%BoundingBox: 0 30 98 128
\end{verbatim}
The first two numbers represent the low $x$- and $y$-coordinates of
the figure's bounding box, in points, and the last two represent
the high $x$- and $y$-coordinates.  In the example just given, which
corresponds to \Fullref{PStorus}, the bounding box goes from $0$ to
$98$~pt in the $x$-direction and from $30$ to $128$~pt in the
$y$-direction, so its height and width are both $98$~pt, or
approximately 1.35~inches.  The corners of the bounding box appear on
the page in proof mode (\Fullref{Proofing Aids}). 

Note that \verb+\psfig+ only cares about the height and the width of
the bounding box, not about the actual values of the high and low
coordinates.  The latter are relevant only in reference to the
coordinates of the objects inside the picture, about which
\verb+\psfig+ knows nothing.  (Thus the bottom left corner of the
square in \Fullref{PStorus} has coordinates $(4,34)$, which is a
little bit above and to the right of the bounding box's bottom left
corner $(0,30)$.)  If one added the same number to the $x$-coordinates
of the bounding box and of every object in the picture, the result
would be exactly the same.

Normally, the \verb+%%BoundingBox+ line should have been placed in the
PostScript file automatically, by whatever program you used to create
the figure.  If the program did not write a line like that, you have
to do it by hand, and you should consider switching to another program.

Notice that the labels $a$ and $b$ in \Fullref{PStorus} appear in
the same font as they would in the text.  In fact the labels were
typeset by \TeX\ itself, and do not come from the PostScript file,
where it would be hard to specify them satisfactorily---especially
details such as the placement of the primes and other subscripts and
superscripts.  The information for these labels is contained in a file
with the suffix \verb+.lab+, which normally is generated by the same
program that creates the PostScript file, or by a postprocessor.
If you want to include {\it Mathematica} graphics, for example, you
should use the \verb+mathfig+ utility, described in the Appendix.

\begin{wizard}
The average user has no need to know what the \verb~.lab~ file looks
like, but if you want to write a new postprocessor in order to
implement this labeling scheme in conjunction with your favorite
drawing program, here are the details.  Each line of the file
yields one label, and contains the \verb+\setlabel+ command
followed by five arguments: the text of the label, to be typeset in
math mode; the $x$- and $y$-coordinates of the label's {\it reference
point}, in points; and the {\it relative} $x$- and $y$-coordinates of
the reference point.  A value of 1 for the last two arguments means
the reference point of the label it at the top or right of the label's
bounding box; a value of $-1$ means bottom or left, a value of $0$
means middle, and so on.

For example, here is the file \verb+PStorus.lab+ corresponding to
\Fullref{PStorus}:
\begin{verbatim}
\setlabel{a}{48.600000}{34.162104}{0}{1}
\setlabel{b}{4.162104}{78.600000}{1}{0}
\setlabel{a'}{48.600000}{123.037896}{0}{-1}
\setlabel{b'}{93.037896}{78.600000}{-1}{0}
\end{verbatim}
For the label $a$, the reference point is at the top middle of the
label's bounding box, so the label itself will be centered under the
point with coordinates $(48.6,34.1)$.

If you want to typeset a label outside math mode, use \latex/'s
\verb+\mbox+ command (or \TeX's \verb+\hbox+).  The font used by
default in this case is the same as for captions.  Thus, to get the
label {\sf foo} you'll probably want to type the string
\verb+\hbox{foo}+ into your figure-making program.

Sometimes the fact that the source \verb+\hbox{foo}+ has many more
characters than the output causes problems---for example, the
figure-making program might leave too much space for the label.  The
macro \verb+\hyperactivelabels+ provides a way around this.  This
macro is called every time a label is about to be typeset, and you can
define it at will, to allow abbreviations, say.  Thus, after the
commands
\begin{verbatim}
\def\hyperactivelabels{\mathcode`:="8000}
\def\activecolon#1:{\hbox{#1}}
{\catcode`\:=\active\global\let:=\activecolon}
\end{verbatim}
the construction \verb+:foo:+ has exactly the same effect as
\verb+\hbox{foo}+---but only inside labels.  (The command
\verb+\mathcode`:="8000+ causes the colon to be treated as an active
character in math mode; the next two lines say, in a roundabout way,
what this active character should do.)
\end{wizard}

Utilities (programs) that generate PostScript graphics files generally
start the file with a ``signature'' (a comment line stating the
utility's name and version) and a chunk of PostScript code---call it a
header---establishing abbreviations, subroutines, fonts and the like.
The header is always the same, so if you have dozens of figures
generated by the same utility it's good to factor out this
commonality, if possible.  With certain utilities, you can save your
files without the header.  Then they won't print on their own; but
when \verb+\psfig+ looks into them, if it can recognize the utility's
signature, it puts into the dvi file a command that causes the
inclusion of the header at printing time.  This way the header is
included only once for all figures, saving memory and time.  The
utilities that \verb+\psfig+ currently knows about are Mathematica
\cite{wolfram} and Adobe Illustrator \cite{illustrator}.  With
Illustrator, saving without the header is accomplished by clicking on
the appropriate entry in the ``Save As'' dialog box.  For Mathematica,
see the Appendix.

\begin{wizard}
For the signature to be seen by \verb+\psfig+, it must appear near the
top of the file; the actual number of lines that \verb+\psfig+ scans
is controlled by the counter \verb+\maxheaderlines+.  The
\verb+%%BoundingBox+ line must also lie in that same initial stretch.
The default value of \verb+\maxheaderlines+ is 100, which is almost
always adequate.
\end{wizard}

\section{Fonts}
As mentioned in \Fullref{Overall Organization}, the \verb+geom+ style
requires NFSS,
a rational font management system that makes the use of new
fonts easier than under old \latex/.

Under this scheme, unlike the old, if you say \verb+\bf... \small...+,
the \verb+\small+ has no effect on the boldness of the current
font---size, boldness, style and so on are independent attributes.
This is generally good, but it does create backward incompatibilities.
For example, if you have an old \latex/ document where you've relied
on \verb+\small+ to reset the weight of the font, you're in for a
surprise.  Also, \verb+\rm+ should generally be replaced by
\verb+\normalshape+.  For a list of all such differences, see
\cite{amslatex}.

Another case of conflicting ideas on what font-change commands should
do is inside math mode.  Under NFSS,
\verb+\bf+ has no effect inside math mode---you need to say
\verb+$\mathbf{xyz}$+, for example.  The \verb+geom+ style will
print a warning on your terminal if it sees \verb|\bf| in math mode.  

In addition to the math-mode commands \verb+\mathbf+, \verb+\mathrm+ and
\verb+\mathit+, the \verb+geom+ style defines \verb+\mathss+ and
\verb+\mathtt+ as the math mode counterparts of \verb+\sf+ (sans-serif)
and \verb+\tt+ (typewriter type).

Support for Fraktur (``gothic'') and blackboard-bold fonts is
provided, as well as for the fonts \verb+msam+ and \verb+msbm+, which
contain a plethora of mathematical symbols. This is all thanks to the
AMS, and you should turn to the \amslatex/ distribution
\cite{amslatex} if your site lacks the fonts themselves or the
prerequisite auxiliary files \verb+amsfonts.sty+ and
\verb+amssymb.sty+.

\section{The Index, the Table of Contents and the Glossary}

The basic command for putting something into the index is
\verb+\index+, as in standard \latex/.  But \verb+\index+ does not
print its argument.  To index a word that actually appears in the text
say \verb+\ix{...}+.  To index mathematical notation say
\verb+\ixn$...$+ or \verb+\ixninv$...$+, depending on whether the
contents should appear or not.  And, just for symmetry, you can also
use \verb+\ixinv+, which equals \verb+\index+:
$$
\vbox{\halign to .7\hsize{\tabskip=0pt plus 1fil
#\hfil &&{\tt \char`\\ #}\hfil\cr
&\omit Text&\omit Notation\cr
\noalign{\medskip}
Visible&ix&ixn\cr
Invisible&ixinv&ixninv\cr}}
$$

An index entry is made automatically for any \verb+\label+ or
\verb+\ref+ command.  In the second case the entry goes into the index
file preceded by a \verb+%+. 

Remember that, unless you have issued the
command \verb|\autolabelfalse|, there are lots of situations in which
a label, and consequently an index entry, is generated automatically:
in particular, this happens for each sectioning command
(\Fullref{Labels and Cross-References}) and for each tagged
theorem, lemma, and so on (\Fullref{Theorems and Their Friends}).  
This provides a
wealth of index entries, which later you can winnow out
semi-automatically, if necessary, since entries of this sort appear in
quotation marks, preceded by their type.  You can see all the entries
that a given page contributes to the index by looking at the small
print on the top left of the page when proofing mode is on 
(\Fullref{Proofing Aids}).

\begin{wizard}
The \verb+geom+ style modifies the \verb+\theindex+ format of standard
\latex/ so that the index is printed in smaller type.  Moreover, it
achieves a double-column format not by using standard \latex/'s
\verb+\twocolumn+ command, which starts a new page, but rather
\ix{Frank Mittelbach}'s \verb+multicols+ environment \cite{Mittelbach},
which is implemented in the file \verb+multicol.sty+.  Using this
environment you can have single-column and double-column text on the
same page.  If you want to add an explanation before the index, for
example, you might say something like this:
\begin{verbatim}
\begin{theindex}
\null            % this avoids what seems to be a bug in multicol.sty
\end{multicols}  % end double column format started by \begin{theindex}

Numbers in parentheses refer to pages where the entry should be
discussed but isn't, due to laziness on the author's part.

\begin{multicols}2 % restart double column format
\item aardvark...
\end{theindex}
\end{verbatim}
\end{wizard}

Support for a glossary is somewhat more extensive than in standard
\latex/. You can mark a word as a glossary entry by saying
\verb+\glo{...}+; this causes a \glo{dagger} to be placed before the
word, and an index entry for the word to be generated in the file {\it
jobname}\verb+.idx+, preceded by \verb+\dag+.  

\begin{wizard}
Note that the
\verb+geom+ style does not use a separate {\it jobname}\verb+.glo+
file; although standard \latex/'s \verb+\glossary+ command is still
available if you insist in writing entries to that file, this command
has not been made robust in the way that \verb+\index+ and \verb+\label+
have. (See \Fullref{No More Fragile Commands}.)
\end{wizard}

The glossary itself is
delimited by \verb+\begin{theglossary}+ and \verb+\end{theglossary}+.
Glossary entries start like this:
%
\begin{verbatim}
\entry{doohickey} ...
\entry[\cite{...}]{thingamajig} ...
\end{verbatim}
%
The entry name is printed in boldface, followed by a period.
The contents of the optional argument are printed immediately after
the entry name, before the period.  The comments made above about
double-column formatting apply to the glossary as well.

The table of contents is treated in much the same way as in standard
\latex/, but there is one important difference: starred sectioning
commands, like \verb+\section*+, do show up in the table of contents,
unless you've said \verb+\starredcontentsfalse+.  
Since one generally doesn't want the table of contents
to include an entry for the table of contents itself,
the definition of \verb+\tableofcontents+ in 
\verb+geomart.sty+ starts like this:
\begin{verbatim}
\def\tableofcontents{{\starredcontentsfalse\section*{Contents}...}}
\end{verbatim}
The extra pair of braces limits the action of
\verb+\starredcontentsfalse+ to this entry only.

\section{Proofing Aids}

The \verb+geom+ style provides a number of proofing aids that you can
turn on and off by saying \verb+\proofingtrue+ and
\verb+\proofingfalse+.  The default is for proofing to be on.  These
facilities are independent of standard \latex/'s \verb+draft+ option
to the \verb+book+ and \verb+article+ style, whose only effect is to
make overfull box markers visible.

If you use \latex/'s \verb+\includeonly+ mechanism \cite[76]{Lamport},
you can print your final version by passing the macro
\verb+\noproofing+ as the argument to \verb+\includeonly+.  This not
only turns off all proofing aids, but also prevents \latex/ from
starting a new page for each input file.  (Naturally, the page breaks
will likely be different because of that, so you'll need to run the
job twice to get cross-references right.)

\begin{wizard}
Normally you have two lines like this in the main file for your
document:
\begin{verbatim}
\typein[\sectionstoinclude]{Enter the sections that should be processed: }
\includeonly{\sectionstoinclude}
\end{verbatim}
so you can type in the sections to be included from the terminal,
without having to modify the source file.  The macro
\verb+\noproofing+ works in this mode, too.
\end{wizard}

By the way, if you are using the \verb+book+ style, the construction
\verb+\chap1{345}+ expands to 
\begin{verbatim}
chap1/1.3,chap1/1.4,chap1/1.5
\end{verbatim}
This has nothing to do with proofing, but it provides a useful shorthand
for use with \verb+\includeonly+.
Braces should be used around a section or chapter number having more
than one digit.

In proofing mode you get, to the left of the main text, a column in
tiny print with all the index entries contributed by that page.  This
includes automatic entries with their respective prefixes (see
\Fullref{The Index, the Table of Contents and the Glossary}).

If you are in proof mode and the macros \verb+\leftfoot+ and
\verb+\rightfoot+ are defined, they will be typeset flush left and
flush right on the footline.  These macros are often defined to
contain information about draft version, date of last modification,
etc., which can be updated automatically.

\begin{wizard}
Near the top of the 
source file for this document, I wrote the lines
\begin{verbatim}
\def\leftfoot{\RCSstring$Revision: 1.12 $}
\def\rightfoot{\RCSstring$Date: 95/07/03 15:57:22 $}
\def\RCSstring$#1 ${#1}
\end{verbatim}
The first two lines are automatically updated by the UNIX utility
\ix{RCS} (Revision Control System) whenever I check the file in or
out.  RCS requires that the keywords \verb+Revision+, \verb+Date+ and
so on be found within dollar signs, and it tacks on a blank before the
closing \verb+$+ for good measure.  The \verb+\RCSstring+ macro
eliminates this extraneous stuff.
\end{wizard}

\section*{Appendix: {\mediumseries \tt mathfig}---A System for Doing Typesetting in
Mathematica} 

\begin{warning}
The software described in this Appendix depends on Mathematica
\cite{wolfram}.  I make no representation whatsoever concerning its
performance.  Although I have used \verb+mathfig+ under Mathematica
versions 1.2, 2.0 and 2.1, I emphasize that \verb+mathfig+ may fail to
work as described here and that I take no responsibility for such
failure.
\end{warning}

The \verb+mathfig+ program, for use on UNIX systems, takes a
Mathematica source file \verb+foo.m+ and calls Mathematica to create
a PostScript file \verb+foo.ps+.  
Then \verb+mathfig+ postprocesses \verb+foo.ps+ to
take out the PostScript labels putting the same information in a file
\verb+foo.lab+ that can be read later by \latex/, as explained in
\Fullref{PostScript Figures}.  The usage of this program is the following:
$$
\hbox to \hsize{\small\hskip 1em
{\tt mathfig} [ {\tt -d} {\it directory} ] [ {\tt -i} {\it initfile} ]
{\it files}\hfil}
$$
The \verb+-d+ option makes all file names relative to the specified
directory.  The \verb+-i+ option tells Mathematica to read an
initialization file.  Any number of files can be passed as subsequent
argument; \verb+\mathfig+ processes one at a time, starting a new
Mathematica session each time.

The steps performed by \verb+mathfig+ when processing the file
\verb+foo.m+ are the following.  It calls \verb+math+ and gives it the
commands \verb+Get["+{\it initfile}\verb+"]+ (if an initialization
file was specified) and \verb+Get["foo.m"]+.  The result of
\verb+Get["foo.m"]+ should be an expression with head \verb+Graphics+
or \verb+Graphics3D+.  This expression is passed to \verb+Display+,
with output sent to the filter \verb+geomfix+; this filter is
analogous to the \verb+psfix+ program supplied with Mathematica.  The
output of \verb+geomfix+ is further processed to create the files
\verb+foo.ps+ and \verb+foo.lab+.  This last step is performed by
\verb+mathlabels+, a \verb+perl+ program.  (\verb+perl+ is a widely
used UNIX utility, and it should be available at your site. If
\verb+mathfig+ claims that \verb+mathlabels+ cannot be found, it may
be because \verb+perl+ is not where \verb+mathfig+ expects to find it;
see the end of \Fullref{Installation}.)

By default, \verb+\mathfig+ creates a figure two inches high and five
inches wide.  A different height can be specified by including the
assignment \verb+$height=+{\it height} in the Mathematica source file,
where {\it height} is a number specifying the height in inches.  You
can also say \verb+$width=+{\it width}.  The reason to put these numbers
in the source file, rather than choosing them at run time, is
archival: in this way the output is entirely determined by the
contents of the source file.

\begin{wizard}
Furthermore, if you assign values to the variables \verb+$xlo+,
\verb+$xhi+, \verb+$ylo+ and \verb+$yhi+, the figure produced by
Mathematica will be clipped to a window with the given boundaries, and
rescaled so as to fit in a box of the given height and width.  For
example, setting \verb+$xlo=$ylo=.1+ and \verb+$xhi=$yhi=.9+ will trim
10\% off the size of the box around each edge, and the remaining
graphics will be rescaled by a factor of $1/(.9-.1)=1.25$.  Under
Mathematica 1.2, this is the only way I know to make a 3D-graphics
object occupy the whole bounding box.  Unfortunately the values of
\verb+$xlo+ and so on have to be determined by trial and error.

The symbols \verb+$height+, \verb+$width+, \verb+$xlo+, \verb=$ylo=,
\verb+$xhi+, \verb=$yhi= and \verb=$Def= should not be used in your
Mathematica source file except with the meanings described above.
(\verb=$Def= is used internally.)

When Mathematica generates a picture with labels, the scaling
computations are implicit in the resulting PostScript code, and they
take into account the dimensions of the labels present in the code.
Thus the graph generated by
%
\begin{verbatim}
Plot[x,{x,0,1}]
\end{verbatim}
%
looks a bit smaller on the screen than the one generated by
%
\begin{verbatim}
Plot[x,{x,0,1},Ticks->False]
\end{verbatim}
%
because the first has to accomodate labels along the axes, in the same
size window.  However, when you run \verb+mathfig+, the labels are
taken out of the PostScript file, and therefore don't have any effect
on the scaling of the picture.  The \verb+mathfig+ output for the two
commands above would be scaled by exactly the same amount, and the
\TeX\ labels generated for the first command would stick out of the
bounding box a little bit, just as they do in \Fullref{PStorus}.
\end{wizard}

\bibliography{manual}
\bibliographystyle{alpha}
\vfil\eject

\begin{theindex}
\end{theindex}

\end{document}

% $Log:	manual.tex,v $
%Revision 1.12  95/07/03  15:57:22  levy
%update for latex2e
%
%Revision 1.11  1993/09/16  22:17:03  levy
%\tt -> \mediumseries\tt in section heads
%
%Revision 1.10  92/08/25  19:55:41  levy
%typo
%
%Revision 1.9  92/08/03  13:47:49  levy
%*** empty log message ***
%
%Revision 1.8  92/07/22  13:32:23  levy
%post - TUG meeting
%
%Revision 1.7  92/07/17  13:32:29  levy
%typos
%
%Revision 1.6  92/07/17  13:27:50  levy
%*** empty log message ***
%
%Revision 1.5  92/07/11  21:26:42  levy
%*** empty log message ***
%
%Revision 1.4  92/04/02  22:45:56  levy
%*** empty log message ***
%
%Revision 1.3  92/04/02  09:58:12  levy
%*** empty log message ***
%
%Revision 1.2  92/04/01  18:57:39  levy
%*** empty log message ***
%
%Revision 1.1  1992/03/19  18:08:13  levy
%Initial revision
%
