\documentclass[11pt]{article}
\usepackage{hhflxbox}               % for presentation
\usepackage{amssymb,epic,curves}    % for illustrations
\usepackage{verbatim}               % for verbatim displaying of examples
\usepackage{xspace}                 % for ease of typing

\makeatletter

\setlength\parindent\z@
\setlength\parskip{.5\baselineskip}

\setcounter{errorcontextlines}{10}     % For ease of debugging.
\def\0#1.{\oldstylenums{#1}}           % For ease of typing.
\def\packagename#1{{\sffamily #1}}     % For consistent displaying of
                                       % package names. To be redefined
                                       % by the editor if desired.
\chardef\@ttbs="5C                     % This the only way I could figure
\def\macroname#1{{\ttfamily\@ttbs#1}}  % out to get the right backslashes
                                       % when displaying macro names
                                       % (math \backspace is too thin).
\def\envirname#1{{\ttfamily #1}}       % For consistent etc.
\def\scheiding{\par                    % Because I cannot help to show my
                                       % `stamp' in and out of season.
                                       % Remove the stamps it you cannot
                                       % stand them.
  \nobreak\addvspace{18pt plus 6pt minus 6pt}%
  \nobreak\centerline{{\unitlength1pt\begin{picture}(0,0)%
       \thicklines
       \put(-10,0){\line(1,-1){10}}\put(-10,0){\line(1,1){10}}%
       \put(10,0){\line(-1,-1){10}}\put(10,0){\line(-1,1){10}}%
       \put(-5,5){\line(0,-1){10}}\put(5,5){\line(0,-1){10}}%
       \put(-5,-2.5){\line(2,1){10}}%
     \end{picture}}}%
  \addvspace{18pt plus 6pt minus 6pt}}

% The following are document specific macros defined for ease of typing:

\def\hhflxbox{\packagename{hhflxbox}\xspace}
% If incorrect fl ligature shows, then replace following line by:
% \def\hhfLxbox{\packagename{\hbox{hhf}\hbox{lxbox}}\xspace}
\def\hhfLxbox{\packagename{hhf\hbox{}lxbox}\xspace}
\def\={\verb=}
\def\<#1>{\macroname{#1}}
\def\:{\linebreak[1]}

% The following input definitions used for examples:

\newcommand\sillyshape{%
  \begin{picture}(2000,2000)
    \thicklines
    \put(1000,1000){\arc(0,1000){360}}
    \put(1000,-498){\arc(662,749){83}}
    \put(1000,2498){\arc(-662,-749){83}}
  \end{picture}}
\newcommand\jarshape{%
  \begin{picture}(1800,1500)
    \thicklines
    \put(360,0){\line(-1,3){360}}
    \put(0,1080){\line(3,1){1260}}
    \put(540,1500){\line(1,0){720}}
    \put(1440,0){\line(1,3){360}}
    \put(1800,1080){\line(-3,1){1260}}
    \put(360,0){\line(1,0){1080}}
  \end{picture}}
\newcommand\jarframe[1]{%
 \iframe\jarshape(300,180){1200}{10pt}%
  \ifrch\ifrb:{\separbox{1pt}{#1}}}
\newcommand\templebox[1]{\sframe
  [1]\fancycolumn [2]\fancytympan
  [1]\fancycolumn [-]\-%
  {\separbox{3pt}{#1}}}

\def\dashbox(#1,#2)(#3,#4){
  \dashline{40}(#1,#2)(#1,#4)\dashline{40}(#1,#4)(#3,#4)
  \dashline{40}(#1,#2)(#3,#2)\dashline{40}(#3,#2)(#3,#4)}
\def\hmeasure(#1,#2)(#3,#4){
  \put(#1,#2){\makebox(0,0)[l]{$\blacktriangleleft$}}
  \put(#3,#4){\makebox(0,0)[r]{$\blacktriangleright$}}
  \drawline(#1,#2)(#3,#4)}
\def\vmeasure(#1,#2)(#3,#4){
  \put(#1,#2){\makebox(0,0)[b]{$\blacktriangledown$}}
  \put(#3,#4){\makebox(0,0)[t]{$\blacktriangle$}}
  \drawline(#1,#2)(#3,#4)}

\def\fancycolumn#1#2#3#4#5#6#7#8{%
  \sbox#1{%
    \sfrcalclength{#2}{#3}{#4}%
    \def\\##1;##2;{\vrule height\@tempdimq width ##1pt\kern ##2pt}%
    \\0.05;0.15;\\0.60;0.44;\\1.11;0.65;\\1.45;0.77;\\1.57;0.77;%
    \\1.45;0.65;\\1.11;0.44;\\0.60;0.15;\\0.05;0;}%
  \sfrsetoffsets{15pt}{#3}{#4}#5#6#7#8}%
\def\fancytympan#1#2#3#4#5#6#7#8{%
  \sbox#1{%
    \sfrcalclength{#2}{#3}{#4}%
    \unitlength\@tempdimq
    \vbox{\hsize\@tempdimq\offinterlineskip
      \hbox{\begin{picture}(1,0.1)
        \thicklines
        \put(0,0){\line(1,0){1}}
        \put(0,0){\line(5,1){0.5}}
        \put(0.5,0.1){\line(5,-1){0.5}}
        \thinlines
        \put(0.5,0.05){\circle*{0.072}}
      \end{picture}}
      \hbox{\vrule width\@tempdimq height 2pt}}}%
  \sfrsetoffsets{\ht#1}{#3}{#4}#5#6#7#8}%

% THE FOLLOWING DEFINITION IS ALSO INCLUDED VERBATIM AS AN EXAMPLE!
\newenvironment{templeboxed}{%
  \begin{sframed}%
    [1]\fancycolumn[2]\fancytympan[1]\fancycolumn[-]\-
    \begin{separboxed}{3pt}
      \begin{broadboxed}{30pt}
}{%
      \end{broadboxed}%
    \end{separboxed}%
  \end{sframed}%
}

\makeatother

\title{\hhflxbox\ --- Manual}
\author{Herman Haverkort\\\normalsize\normalfont\texttt{hermanh@cistron.nl}}
\date{May 1995 -- Addresses corrected March 1999}

\begin{document}

\maketitle

\section{Introduction}

\hhflxbox contains a number of boxing macros. The kernel consists of
\<iframe>, which boxes things and sets self-scaling frames around, and
\<sframe>, which sets more complex self-scaling and -stretching frames.
Besides \hhflxbox provides the encircling macros \<ringbox>, \<bellybox>
and \<outringbox> (which use \<iframe>), the macros \<sepbox> and
\<separbox>, which set empty space around boxes, and \<broadbox>, which
boxes its argument in a \<vbox> of which the width is the line width
minus some specified value. Thus \<broadbox> facilitates the construction
of frames which stretch to line width.

Furthermore \hhflxbox defines some tools for the proper positioning
of boxes: \<clap> for some kind of horizontal centering
and \<lcenter> (line centering) for vertical centering on a line of
text. \<boxhigh>, \<boxlow>, \<topsqueezeout>, \<topsqueezein>,
\<botsqueezeout> and \<botsqueezein> help getting the spacing in and
around framed passages right.

To be able to use \hhflxbox one should have
\texttt{hhflxbox.sty}, \texttt{hhunits.sty}, \texttt{hhqueue.sty}
and \texttt{hhutils0.sty} available, all of which can be obtained
from the library of my website at
http://come.to/hh and from \textsc{ctan}.

\section{\<sepbox> and \<separbox>}
For introduction of \<sepbox> and \<separbox> it is convenient to
introduce \<bellybox> first. \<bellybox> is one of the
\hhflxbox macros which can be used to encircle things, for example
\bellybox:{3}, which is set with: \=\bellybox=\:\=:{3}=.

You probably notice that the circle around the digit is somewhat
tight. This problem can be solved by putting a \<separbox> around the
digit, like in \=\bellybox=\:\=:{\separbox=\:\={1pt}{3}}=, which
yields: \bellybox:{\separbox{1pt}{3}}.
Actually \=\separbox=\:\={={\it dimension\/}\=}{={\it stuff\/}\=}= puts
{\it dimension\/} wide empty space around {\it stuff\/} on all sides.

A more general form is:\\
\=\sepbox=\:\=(={\it leftspace\/}\=,={\it topspace\/}%
\=,={\it rightspace\/}\=,={\it bottomspace\/}\=){={\it stuff\/}\=}=\\
which adds empty spaces of the specified widths to the sides of the box
containing {\it stuff\/}.

\section{\<iframe>: Isomorphous Frames}
\<iframe> is only a frame drawing {\em tool}: it does not draw frames
itself but it can take care of the proper positioning and scaling of frames
drawn by other macros. To explain the functioning of \<iframe> it is
probably best to give an example of the development of a framing macro
using \<iframe>.

Suppose we want to set self-scaling frames which have the following shape:
\begin{center}\unitlength.01pt\sillyshape\end{center}
then we could imagine a box-shaped area in the frame which will contain
the frame's contents (the inner dashed box in the figure below). Also
we could imagine a box surrounding the frame (the outer dashed box in
the figure).
\begin{center}\unitlength.05pt\noindent
  \hbox{%
    \rlap{\sillyshape}%
    \begin{picture}(2000,2000)
      \thinlines
      \dashbox(0,0)(2000,2000)\dashbox(134,500)(1866,1500)
      \hmeasure(134,500)(0,500)\put(67,450){\makebox(0,0)[t]{$x$}}
      \vmeasure(134,500)(134,0)\put(184,150){\makebox(0,0)[l]{$y$}}
      \hmeasure(134,1250)(1866,1250)\put(1000,1300){\makebox(0,0)[b]{$w$}}
      \vmeasure(584,500)(584,1500)\put(634,800){\makebox(0,0)[l]{$h=1000$}}
    \end{picture}}%
\end{center}

\<iframe> expects the inner box height to be 1000 times the
\<unitlength>, so all dimensions have to be chosen so that the inner box
height equals 1000 indeed. Then \<iframe> can scale the frame by
setting the \<unitlength>. Furthermore \<iframe> expects the lower left
corner of the outer box to have coordinates $(0,0)$. Taking these
expectations in account we can design a macro which draws the frame:

\begin{verbatim}
\newcommand\sillyshape{%
  \begin{picture}(2000,2000)
    \thicklines
    \put(1000,1000){\arc(0,1000){360}}
    \put(1000,-498){\arc(662,749){83}}
    \put(1000,2498){\arc(-662,-749){83}}
  \end{picture}}
\end{verbatim}

(\<arc> is defined in the \packagename{curves} package by I.L. Maclaine-cross.)
Now we can define a silly shape framing macro by defining \<sillyframe>
as: \=\iframe=\:\=\sillyshape=\:\=(=$x$\=,=$y$\=){=$w$\=}{0pt}\ifrch=\:%
\=\ifrcv=\:\=:{#1}=. Actually we will set \=#1= in a
\=\sepbox=\:\=(0pt,2pt,0pt,2pt){#1}= to prevent the frame from touching
its contents. In the above example we have $x=134$, $y=500$ and $w=1732$,
so we write:

\begin{verbatim}
\newcommand\sillyframe[1]{%
 \iframe\sillyshape(134,500){1732}{0pt}%
  \ifrch\ifrcv:{\sepbox(0pt,2pt,0pt,2pt){#1}}}

Now we can put \sillyframe{all}
\sillyframe{sorts} \sillyframe{of}
\sillyframe{things} in silly frames.
\end{verbatim}

\newcommand\sillyframe[1]{%
 \iframe\sillyshape(134,500){1732}{0pt}%
  \ifrch\ifrcv:{\sepbox(0pt,2pt,0pt,2pt){#1}}}

Now we can put \sillyframe{all}
\sillyframe{sorts} \sillyframe{of}
\sillyframe{things} in silly frames.

Note that I do not claim this kind of silly frame to be good-looking:
it is just an example.

The dimension \=0pt= in the example above determines the minimal height
of the silly frame's inner box. Sometimes it is necessary to define it
because \LaTeX's picture environment suppresses small line segments.

The macro \<ifrch> determines what should be done if the frame's
contents width/height ratio is too small. By specifying \<ifrch>
we instruct \<iframe> to center the contents. Instead of
\<ifrch> we could have specified \<ifrl> or \<ifrr> to
have the contents flush left or right.

The macro \<ifrcv> determines what should be done if the frame's
contents height/width ratio is too small. \<ifrcv> yields
vertical centering, while \<ifrt> and \<ifrb> yield top and bottom
flushing.

\leavevmode\jarframe{As} another example of isomorphous frames, consider
the following framing macro.
Note that the inner box height is 1000 again, as expected
by \<iframe>.
\begin{center}\unitlength.05pt\noindent
  \hbox{%
    \rlap{\jarshape}%
    \begin{picture}(1800,1500)
      \thinlines
      \dashbox(0,0)(1800,1500)\dashbox(300,180)(1500,1180)
      \hmeasure(300,180)(0,180)\put(150,280){\makebox(0,0)[br]{$x=300$}}
      \vmeasure(300,180)(300,0)\put(400,60){\makebox(0,0)[l]{$y=180$}}
      \hmeasure(300,930)(1500,930)\put(1050,980){\makebox(0,0)[b]{$w=1200$}}
      \vmeasure(550,180)(550,1180)\put(600,500){\makebox(0,0)[l]{$h=1000$}}
    \end{picture}}%
\end{center}

\begin{verbatim}
\newcommand\jarshape{%
  \begin{picture}(1800,1500)
    \thicklines
    \put(360,0){\line(-1,3){360}}
    \put(0,1080){\line(3,1){1260}}
    \put(540,1500){\line(1,0){720}}
    \put(1440,0){\line(1,3){360}}
    \put(1800,1080){\line(-3,1){1260}}
    \put(360,0){\line(1,0){1080}}
  \end{picture}}

\newcommand\jarframe[1]{%
 \iframe\jarshape(300,180){1200}{10pt}%
  \ifrch\ifrb:{\separbox{1pt}{#1}}}
\end{verbatim}

\section{Using the Units System with \<iframe>}

If we put frames around for example page numbers, then
the self-scaling properties of isomorphous frames may have an unpleasant
result: numbers of the same type, like
page number \sillyframe{\oldstylenums{21}} and
page number \sillyframe{\oldstylenums{25}}, might get differently sized
frames because of their different natural sizes. This can be solved by
redefining \<sillyframe> to specify a {\em unit name}, since all things
typeset with the same unit name get equally sized frames. The unit name,
for example \=pagenr=, should be placed between the vertical alignment
specification and the colon, like in:

\begin{verbatim}
\newcommand\pagenrframe[1]{%
  \iframe\sillyshape(134,500){1732}%
    {0pt}\ifrch\ifrcv pagenr:{%
      \sepbox(0pt,2pt,0pt,2pt){#1}}}

framed numbers like \pagenrframe{\oldstylenums
{21}} and \pagenrframe{\oldstylenums{25}}.
\end{verbatim}

which yields (after compiling our document twice):

\newcommand\pagenrframe[1]{%
  \iframe\sillyshape(134,500){1732}%
    {0pt}\ifrch\ifrcv pagenr:{%
      \sepbox(0pt,2pt,0pt,2pt){#1}}}

framed numbers like \pagenrframe{\oldstylenums
{21}} and \pagenrframe{\oldstylenums{25}}.

\section{\<lcenter>: Centering on a Line}

The result shown above is not fully satisfactory yet: now the frames are
equally sized but the first frame is positioned higher than the second.
This is no bug, it is a feature. No really, it is! It is, however, a
sometimes unwanted feature. The solution is using \<lcenter> to center
the frames on their line, like in:

\begin{verbatim}
pages \lcenter{\pagenrframe{\oldstylenums{21}}}
and \lcenter{\pagenrframe{\oldstylenums{25}}}
\end{verbatim}

resulting in:

pages \lcenter{\pagenrframe{\oldstylenums{21}}}
and \lcenter{\pagenrframe{\oldstylenums{25}}}

\section{\<ringbox>, \<bellybox> and \<outringbox>: Encircling}

\=\ringbox{={\it optional unit name\/}\=}:{={\it stuff\/}\=}= sets
a circle around {\it stuff\/}. The specification of a unit name is optional;
its use is explained above.

\<outringbox> is very much like \<ringbox>,
but the following example demonstrates their difference:

\begin{verbatim}
1\ringbox:{2}3 and 1\outringbox:{2}3
\end{verbatim}

yields:

1\ringbox:{2}3 and 1\outringbox:{2}3

If \<ringbox> is used, the circle contributes to the width, height and
depth of the result. If \<outringbox> is used, the circle does not
contribute any width, height or depth, so that the text is typeset as
if the circle were not present and the circle were added after typesetting
the text.

The result of \<bellybox> is a circle which contributes a bit to the
dimensions of the encircled result but also sticks out a bit
(by 10 percent of its radius to be sort of exact). So \<bellybox> is
an intermediate form of \<ringbox> and \<outringbox>.

\section{\<sframe>: Stretchable Frames}

Putting the current \<sframe> to good use is a rather complex task,
so I still look for ways to simplify its `user' interface.
Therefore there is a chance that \<sframe> will
be reorganized quite thoroughly in future. However there is no need
to be afraid that macros using \<sframe> run the risk of incompatibility
with future releases of \hhflxbox: as soon as I find out that anyone
is actually using \<sframe>, I will take care that future releases
of \hhflxbox will always support the old \<sframe>, possibly along
with a new \<sframe>-like macro.

In the remainder of this section I will first explain how \<sframe>
constructs a frame. After that I will give an example of the development
of a simple frame, using concepts introduced in the preceding explanation.
Reading the explanation and the example is necessary if you want to
take full advantage of \<sframe>. Maybe it does not make clear enough
how to construct the most complex frames, but at least it will give you
an idea of what to experiment with. By the way: I will very
gratefully welcome any alternative or supplementary explanations and
suggestions for improving the explanation or the examples.\footnote
{Please e-mail me at: \texttt{hermanh@cistron.nl}}
After this thorough tutorial some more examples of frames constructed
with \<sframe> are given, with full \TeX\ code. If you do not feel like
reading the entire manual, then just studying the examples may
teach you enough to learn how to construct moderately complex frames
yourself, by imitating the examples and experimenting with small
modifications.

\<sframe> constructs a frame out of four frame components: a left side,
a top ledger, a right side and a bottom ledger.
The frame components actually are macros which set the values of a
box and several dimension registers.
To understand how these values should be determined by the component
macros, one needs to understand the internal representation of a
box which is being framed.

Such a box is represented by four
junctures $A$, $B$, $C$ and $D$, to which the frame components will
eventually be attached. Each juncture $J$ is related to
a corner $R_J$ of a surrounding box $R = R_AR_BR_CR_D$. The location
of each juncture $J$ is given by a vector $(x_J,y_J)$, which has its
origin in the related corner $R_J$ while it is oriented \emph{into}
the box.
Thus each juncture $J$ which is \emph{inside} the box $R$ will have positive
coordinates $x_J$ and $y_J$, while a juncture $J$ outside the box
will have one or both of its coordinates negative.

The following figure might clarify the above definitions. In this figure
all of the coordinates $x$ and $y$ are positive.

\begin{center}\unitlength.05pt\noindent
  \hbox{%
    \begin{picture}(2600,2100)
      \thinlines
      \dashbox(0,0)(2600,2100)
      \put(-100,2100){\makebox(0,0)[tr]{$R_A$}}
      \put(2700,2100){\makebox(0,0)[tl]{$R_B$}}
      \put(2700,0){\makebox(0,0)[bl]{$R_C$}}
      \put(-100,0){\makebox(0,0)[br]{$R_D$}}
      \put(700,1600){\circle*{100}}\put(800,1400){\makebox(0,0)[tl]{$A$}}
      \hmeasure(0,1600)(700,1600)\put(350,1500){\makebox(0,0)[t]{$x_A$}}
      \vmeasure(700,1600)(700,2100)\put(800,1850){\makebox(0,0)[l]{$y_A$}}
      \put(2000,1600){\circle*{100}}\put(1800,1400){\makebox(0,0)[tr]{$B$}}
      \hmeasure(2000,1600)(2600,1600)\put(2300,1500){\makebox(0,0)[t]{$x_B$}}
      \vmeasure(2000,1600)(2000,2100)\put(1900,1850){\makebox(0,0)[r]{$y_B$}}
      \put(2200,400){\circle*{100}}\put(2000,600){\makebox(0,0)[br]{$C$}}
      \hmeasure(2200,400)(2600,400)\put(2400,500){\makebox(0,0)[b]{$x_C$}}
      \vmeasure(2200,0)(2200,400)\put(2100,200){\makebox(0,0)[r]{$y_C$}}
      \put(400,700){\circle*{100}}\put(600,800){\makebox(0,0)[bl]{$D$}}
      \hmeasure(0,700)(400,700)\put(200,800){\makebox(0,0)[b]{$x_D$}}
      \vmeasure(400,0)(400,700)\put(500,350){\makebox(0,0)[l]{$y_D$}}
      \put(700,1600){\line(1,0){1300}}
      \put(2000,1600){\line(1,-6){200}}
      \put(400,700){\line(1,3){300}}
      \put(400,700){\line(6,-1){1800}}
    \end{picture}}%
\end{center}

In practice the surrounding box is most often the smallest rectangle which
includes all junctures, like in the following figure:

\begin{center}\unitlength.05pt\noindent
  \hbox{%
    \begin{picture}(2600,2100)
      \thinlines
      \dashbox(400,400)(2200,1600)
      \put(300,1600){\makebox(0,0)[tr]{$R_A$}}
      \put(2300,1600){\makebox(0,0)[tl]{$R_B$}}
      \put(2300,300){\makebox(0,0)[bl]{$R_C$}}
      \put(400,300){\makebox(0,0)[tl]{$R_D$}}
      \put(700,1600){\circle*{100}}\put(800,1400){\makebox(0,0)[tl]{$A$}}
      \hmeasure(700,1700)(400,1700)\put(550,1800){\makebox(0,0)[b]{$x_A$}}
      \put(2000,1600){\circle*{100}}\put(1800,1400){\makebox(0,0)[tr]{$B$}}
      \hmeasure(2200,1700)(2000,1700)\put(2100,1800){\makebox(0,0)[b]{$x_B$}}
      \put(2200,400){\circle*{100}}\put(2000,600){\makebox(0,0)[br]{$C$}}
      \put(400,700){\circle*{100}}\put(600,800){\makebox(0,0)[bl]{$D$}}
      \vmeasure(300,700)(300,400)\put(200,550){\makebox(0,0)[r]{$y_D$}}
      \put(700,1600){\line(1,0){1300}}
      \put(2000,1600){\line(1,-6){200}}
      \put(400,700){\line(1,3){300}}
      \put(400,700){\line(6,-1){1800}}
    \end{picture}}%
\end{center}

where $y_A = y_B = y_C = 0$ and $x_C = x_D = 0$. The reason to use
such a minimal surrounding rectangle for $R$ is that frame components
can only be attached to a pair of junctures which lies on a border
of $R$ (or on a line overlapping a border).

In the figure above $(A,B)$ is such a pair. Since the corresponding
corners $(R_A,R_B)$ represent the \emph{top} edge of the box, the
\emph{top} ledger component can be attached to $A$ and $B$. In order
to do so the top ledger component macro is called to construct the ledger,
so that \<sframe> can subsequently attach the ledger to the junctures.

The top ledger component macro gets eight arguments. The first identifies
the box register in which the ledger is to be set. The second, third
and fourth give the values of $|R_AR_B|$ (the length of $R$'s top edge),
$x_A$ and $x_B$. These values can be used to determine the appropriate
length for the ledger. Usually this length is $|AB|$, which equals
$|R_AR_B| - x_A - x_B$.
To ease the development of component macros \hhflxbox defines
a macro \<sfrcalclength>, which takes these three terms as its
three arguments and stores the result of the above calculation in
\<@tempdimq>. Note that \<sfrcalclength> should only be used inside
a group inside a component macro, to avoid interference with other
macros using \<@tempdimq> (like \<sframe> itself).

Now let us construct our first prototype of a top ledger part. We simply
ignore the last four arguments for the ledger component macro: we will
add them later. We will construct a ledger which consists of an
hollow beam spanning its junctures, adorned with two differently
sized squares at the ends.

\begin{verbatim}
\newcommand\squaredbeam[4]{%
  \sbox#1{%    #1 is the box register to contain the ledger;
          %    set left square, raised 10pt, centered, and
          %    not contributing any width:
    \raise 10pt\clap{\vrule height 20pt width 20pt}%
          %    set left end of beam, not contrib. any width:
    \rlap{\vrule height 20pt width 1.2pt}%
          %    calculate distance between the junctures and
          %    store result in \@tempdimq:
    \sfrcalclength{#2}{#3}{#4}%
          %    set hollow beam (without the ends) of
          %    appropriate width:
    \vbox{\hsize\@tempdimq\offinterlineskip
      \hrule height 2pt width \hsize
      \kern 17.6pt
      \hrule height 2pt width \hsize}%
          %    set right end of beam, not contrib. any width:
    \llap{\vrule height 20pt width 1.2pt}%
          %    set right square, raised 10pt, centered etc.:
    \raise 10pt\clap{\vrule height 15pt width 15pt}}}
\end{verbatim}
\newcommand\squaredbeam[1]{\hbox{%
    \raise 10pt\clap{\vrule height 20pt width 20pt}%
    \rlap{\vrule height 20pt width 1.2pt}%
    \vbox{\hsize #1\offinterlineskip
      \hrule height 1.2pt width \hsize
      \kern 17.6pt
      \hrule height 1.2pt width \hsize}%
    \llap{\vrule height 20pt width 1.2pt}%
    \raise 10pt\clap{\vrule height 15pt width 15pt}}}

\<squaredbeam> will produce beams like shown in the following figure.
The dashed box denotes the size and shape of the \<hbox> in which
the beam is set. The $P$ spot indicates its reference point.

\begin{center}\unitlength.05pt\noindent
  \hbox{%
    \begin{picture}(2600,700)
      \thinlines
      \dashbox(700,100)(2000,700)
      \put(700,100){\makebox(0,0)[bl]{\squaredbeam{1300\unitlength}}}
      \put(700,100){\circle*{100}}\put(600,0){\makebox(0,0)[tr]{$P$}}
    \end{picture}}%
\end{center}

It is important to be aware of the position of the reference point.
When the ledger is attached, the reference point is aligned with the
left juncture, as in the following figure:

\begin{center}\unitlength.05pt\noindent
  \hbox{%
    \begin{picture}(2600,2200)
      \thinlines
      \dashbox(400,400)(2200,1600)
      \put(300,1600){\makebox(0,0)[tr]{$R_A$}}
      \put(2300,1600){\makebox(0,0)[tl]{$R_B$}}
      \put(2300,300){\makebox(0,0)[bl]{$R_C$}}
      \put(400,300){\makebox(0,0)[tl]{$R_D$}}
      \put(700,1600){\circle*{100}}\put(800,1400){\makebox(0,0)[tl]{$A$}}
      \put(2000,1600){\circle*{100}}\put(1800,1400){\makebox(0,0)[tr]{$B$}}
      \put(2200,400){\circle*{100}}\put(2000,600){\makebox(0,0)[br]{$C$}}
      \put(400,700){\circle*{100}}\put(600,800){\makebox(0,0)[bl]{$D$}}
      \put(2000,1600){\line(1,-6){200}}
      \put(400,700){\line(1,3){300}}
      \put(400,700){\line(6,-1){1800}}
      \put(700,1600){\makebox(0,0)[bl]{\squaredbeam{1300\unitlength}}}
      \put(700,1600){\circle*{100}}\put(600,1500){\makebox(0,0)[tr]{$P$}}
    \end{picture}}%
\end{center}

Now that the ledger has been attached to the junctures $A$ and $B$, those
junctures cannot be used anymore. Therefore the ledger component macro
should define new positions for the junctures $A$ and $B$, and that is
what the fifth to eighth parameters of the component macro are for.
The fifth argument specifies the dimension register in which the new
value of $x_A$ should be set. We denote that value by $x_{A'}$ and its
register by $[x_{A'}]$. Similarly the sixth, seventh and eighth argument
provide $[x_{B'}]$, $[y_{A'}]$ and $[y_{B'}]$. Now to what values should those
registers be set? The following figure shows the new positions $A'$ and
$B'$ of the junctures:

\begin{center}\unitlength.05pt\noindent
  \hbox{%
    \begin{picture}(2600,2500)
      \thinlines
      \dashbox(400,400)(2200,1600)
      \put(300,1600){\makebox(0,0)[tr]{$R_A$}}
      \put(2300,1600){\makebox(0,0)[tl]{$R_B$}}
      \put(2300,300){\makebox(0,0)[bl]{$R_C$}}
      \put(300,300){\makebox(0,0)[br]{$R_D$}}
      \hmeasure(500,2200)(400,2200)\put(450,2400){\makebox(0,0)[b]{$x_{A'}$}}
      \vmeasure(400,1600)(400,2200)\put(300,1900){\makebox(0,0)[r]{$y_{A'}$}}
      \hmeasure(2200,2100)(2150,2100)\put(2175,2300){\makebox(0,0)[b]{$x_{B'}$}}
      \vmeasure(2200,1600)(2200,2100)\put(2300,1850){\makebox(0,0)[l]{$y_{B'}$}}
      \put(2000,1600){\line(1,-6){200}}
      \put(400,700){\line(1,3){300}}
      \put(400,700){\line(6,-1){1800}}
      \put(700,1600){\makebox(0,0)[bl]{\squaredbeam{1300\unitlength}}}
    \end{picture}}%
\end{center}

Now the values to store in $[x_{A'}]$, $[x_{B'}]$, $[y_{A'}]$ and $[y_{B'}]$
can be measured from the figure above. Note that the positive direction is
always pointing into the surrounding box $R$ (which is too small to really
surround now; we will discuss that later). In the example above this means
that $y_{A'}$ and $y_{B'}$ are negative. Now let us incorporate the setting
of the $A'$ and $B'$ positions in our ledger component macro:

\begin{verbatim}
\newcommand\squaredbeam[8]{%
  \sbox#1{%    #1 is the box register to contain the ledger;
          %    set left square, raised 10pt, centered, and
          %    not contributing any width:
    \raise 10pt\clap{\vrule height 20pt width 20pt}%
          %    set left end of beam, not contrib. any width:
    \rlap{\vrule height 20pt width 1.2pt}%
          %    calculate distance between the junctures and
          %    store result in \@tempdimq:
    \sfrcalclength{#2}{#3}{#4}%
          %    set hollow beam (without the ends) of
          %    appropriate width:
    \vbox{\hsize\@tempdimq\offinterlineskip
      \hrule height 2pt width \hsize
      \kern 17.6pt
      \hrule height 2pt width \hsize}%
          %    set right end of beam, not contrib. any width:
    \llap{\vrule height 20pt width 1.2pt}%
          %    set right square, raised 10pt, centered etc.:
    \raise 10pt\clap{\vrule height 15pt width 15pt}}%
  \setlength#5{#3}%       initialize xA' to xA
  \addtolength#5{-10pt}%  subtract half width of left square
  \setlength#7{-30pt}%    set yA' to -30pt
  \setlength#6{#4}%       initialize xB' to xB
  \addtolength#6{7.5pt}%  add half the width of right square
  \setlength#8{-25pt}%    set yB' to -25pt
}
\end{verbatim}

Now our top ledger component macro has been completed.
Component macros for the bottom ledger and the sides have similar
structure. Their characteristics are summarized in the following table:

\begin{tabular}{|r|c|c|c|c|}\hline\label{sframetable}%
           & left       & top        & right      & bottom     \\\hline\hline
junctures  & $A,D$      & $A,B$      & $B,C$      & $C,D$      \\\hline
arg. \#2   & $|R_AR_D|$ & $|R_AR_B|$ & $|R_BR_C|$ & $|R_CR_D|$ \\\hline
arg. \#3   & $y_A$      & $x_A$      & $y_B$      & $x_D$      \\\hline
arg. \#4   & $y_D$      & $x_B$      & $y_C$      & $x_C$      \\\hline
arg. \#5   & $[y_{A'}]$ & $[x_{A'}]$ & $[y_{B'}]$ & $[x_{D'}]$ \\\hline
arg. \#6   & $[y_{D'}]$ & $[x_{B'}]$ & $[y_{C'}]$ & $[x_{C'}]$ \\\hline
arg. \#7   & $[x_{A'}]$ & $[y_{A'}]$ & $[x_{B'}]$ & $[y_{D'}]$ \\\hline
arg. \#8   & $[x_{D'}]$ & $[y_{B'}]$ & $[x_{C'}]$ & $[y_{C'}]$ \\\hline
\parbox{80pt}{\raggedleft\strut reference point aligned with\strut} &
             $y_D$      & $x_A$      & $y_C$      & $x_D$      \\\hline
\end{tabular}

Now that we know the structure of frame component macros, let us construct
a simple frame. In fact we will construct a \emph{very} simple frame:
we will simulate an \<fbox> with zero \<fboxsep>. The result is not useful
(\<fbox> itself is much more efficient than an \<sframe> simulation), but
the construction process is useful as an example.

We can consider fboxing to be a four-phase process. First the box to
be framed is constructed and it is given four junctures at the corners.
Then the left and right sides of the frame are added, causing the junctures
to be moved out horizontally. Third a new surrounding box is assumed,
which surrounds both the original box and the already added frame components,
thus causing the moved out junctures to lie on the borders of the new
surrounding box. Finally the top and bottom ledgers of the frame are added
(and the junctures are moved again, although that is not really necessary
anymore). The first three phases are shown in the following figure:

\begin{center}\unitlength.05pt\noindent
  \hbox{%
    \begin{picture}(2000,2000)
      \dashbox(100,100)(1900,1900)
      \put(100,100){\circle*{100}}
      \put(1900,100){\circle*{100}}
      \put(1900,1900){\circle*{100}}
      \put(100,1900){\circle*{100}}
      \put(0,0){\makebox(0,0)[tl]{$D=R_D$}}
      \put(2000,0){\makebox(0,0)[tr]{$C=R_C$}}
      \put(2000,2000){\makebox(0,0)[br]{$B=R_B$}}
      \put(0,2000){\makebox(0,0)[bl]{$A=R_A$}}
    \end{picture}%
    \hspace{1cm}%
    \begin{picture}(2400,2000)
      \dashbox(300,100)(2100,1900)
      \put(100,100){\circle*{100}}
      \put(2300,100){\circle*{100}}
      \put(2300,1900){\circle*{100}}
      \put(100,1900){\circle*{100}}
      \put(300,100){\makebox(0,0)[br]%
        {\vrule width 200\unitlength height 1800\unitlength}}
      \put(2100,100){\makebox(0,0)[bl]%
        {\vrule width 200\unitlength height 1800\unitlength}}
      \put(300,0){\makebox(0,0)[tl]{$R_D$}}    \put(0,0){\makebox(0,0)[tl]{$D$}}
      \put(2100,0){\makebox(0,0)[tr]{$R_C$}}   \put(2400,0){\makebox(0,0)[tr]{$C$}}
      \put(2100,2000){\makebox(0,0)[br]{$R_B$}}\put(2400,2000){\makebox(0,0)[br]{$B$}}
      \put(300,2000){\makebox(0,0)[bl]{$R_A$}} \put(0,2000){\makebox(0,0)[bl]{$A$}}
    \end{picture}%
    \hspace{1cm}
    \begin{picture}(2400,2000)
      \dashbox(100,100)(2300,1900)
      \put(100,100){\circle*{100}}
      \put(2300,100){\circle*{100}}
      \put(2300,1900){\circle*{100}}
      \put(100,1900){\circle*{100}}
      \put(300,100){\makebox(0,0)[br]%
        {\vrule width 200\unitlength height 1800\unitlength}}
      \put(2100,100){\makebox(0,0)[bl]%
        {\vrule width 200\unitlength height 1800\unitlength}}
      \put(100,0){\makebox(0,0)[t]{$D=R_D$}}
      \put(2300,0){\makebox(0,0)[t]{$C=R_C$}}
      \put(2300,2000){\makebox(0,0)[b]{$B=R_B$}}
      \put(100,2000){\makebox(0,0)[b]{$A=R_A$}}
    \end{picture}}%
\end{center}

The initialisation of the junctures and the (re)selection of surrounding
boxes is all taken care of by \<sframe>. Our job is to define proper
frame component macros. Let us start with the left side part:

\begin{verbatim}
\newcommand\leftside[8]{%
  \sbox#1{%   as usual, #1 is the box register to contain
          %   the side part;
          %   as usual, calculate the distance between the
          %   junctures involved; store result in \@tempdimq:
    \sfrcalclength{#2}{#3}{#4}%
          %   set solid side part:
    \vrule height \@tempdimq width \fboxrule}%
  \setlength#5{#3}%         set yA' to yA
  \setlength#7{-\fboxrule}% set xA' to negative width of
                          % frame component
  \setlength#6{#4}%         set yD' to yD
  \setlength#8{-\fboxrule}% set xD' to negative width of
                          % frame component
}
\end{verbatim}

The meaning of the various parameters was found in the table
on page \pageref{sframetable}. Now let us write the right side
part:

\begin{verbatim}
\newcommand\rightside[8]{%
  \sbox#1{%   as usual, #1 is the box register to contain
          %   the side part;
          %   as usual, calculate the distance between the
          %   junctures involved; store result in \@tempdimq:
    \sfrcalclength{#2}{#3}{#4}%
          %   set solid side part:
    \vrule height \@tempdimq width \fboxrule}%
  \setlength#5{#3}%         set yB' to yB
  \setlength#7{-\fboxrule}% set xB' to negative width of
                          % frame component
  \setlength#6{#4}%         set yC' to yC
  \setlength#8{-\fboxrule}% set xC' to negative width of
                          % frame component
}
\end{verbatim}

Now take a close look at the left and right side macros: they
consist of exactly the same \TeX\ code! This is not purely a coincedence.
The meanings of the frame component macro parameters have been
chosen so that the same macro can be used for opposite frame
components in many cases. Let us turn to the top ledger macro:

\begin{verbatim}
\newcommand\topledger[8]{%
  \sbox#1{%   as usual, #1 is the box register to contain
          %   the side part;
          %   as usual, calculate the distance between the
          %   junctures involved; store result in \@tempdimq:
    \sfrcalclength{#2}{#3}{#4}%
          %   set solid side part:
    \vrule width \@tempdimq height \fboxrule}%
  \setlength#5{#3}%         set xA' to xA
  \setlength#7{-\fboxrule}% set yA' to negative height of
                          % frame component
  \setlength#6{#4}%         set xB' to xB
  \setlength#8{-\fboxrule}% set yB' to negative height of
                          % frame component
}
\end{verbatim}

Compare the top ledger macro to the side macros. You will notice that
there are many similarities. Moreover, all the \<setlength> lines
are exactly the same (ignoring the explanatory comments). In fact
the \<setlength> lines in the above macro represent a common case,
in which the only variable is the thickness (width for side parts,
height for ledger parts) of the frame component. To ease the development
of frame component macros a macro \<sfrsetoffsets> has been included
in \hhflxbox which handles the common case. Using this macro, one can
replace the macro definitions above by:

\begin{verbatim}
\newcommand\side[8]{%
  \sbox#1{%
    \sfrcalclength{#2}{#3}{#4}%
    \vrule height \@tempdimq width \fboxrule}%
  \sfrsetoffsets{\fboxrule}{#3}{#4}#5#6#7#8}

\newcommand\ledger[8]{%
  \sbox#1{%
    \sfrcalclength{#2}{#3}{#4}%
    \vrule width \@tempdimq height \fboxrule}%
  \sfrsetoffsets{\fboxrule}{#3}{#4}#5#6#7#8}
\end{verbatim}

Yes indeed, the ledger macro is suited for use as both a top and a bottom
ledger. Now that our frame components are ready, we only have to let them
be combined by \<sframe>. This is done by defining:

\begin{verbatim}
\newcommand\simulatedfbox[1]{%
  \sframe [1]\side [2]\ledger [1]\side [2]\ledger {#1}}
\end{verbatim}

In the example above the left side, top ledger, right side and bottom
ledger are succesively specified. Each of the frame components is
preceded by a bracketed number which determines its phase. The choice
of phases above guarantees that the sides are added before the ledgers,
as planned in our four-phase fboxing model. In general phase numbers
1 to 4 can be used; if omitted, phase 4 is assumed.

The parameter \=#1= in the example above will contain the contents of
the framed box.

\makeatletter
\newcommand\side[8]{%
  \sbox#1{%
    \sfrcalclength{#2}{#3}{#4}%
    \vrule height \@tempdimq width \fboxrule}%
  \sfrsetoffsets{\fboxrule}{#3}{#4}#5#6#7#8}
\newcommand\ledger[8]{%
  \sbox#1{%
    \sfrcalclength{#2}{#3}{#4}%
    \vrule width \@tempdimq height \fboxrule}%
  \sfrsetoffsets{\fboxrule}{#3}{#4}#5#6#7#8}
\newcommand\simulatedfbox[1]{%
  \sframe [1]\side [2]\ledger [1]\side [2]\ledger {#1}}
\makeatother

And finally here is an example, the result of \=\simulatedfbox=\:\={hello}=:
\simulatedfbox{hello}.

\section{An example of \<sframe>}

\begin{verbatim}
\def\fancycolumn#1#2#3#4#5#6#7#8{%
  \sbox#1{%
    \sfrcalclength{#2}{#3}{#4}%
    \def\\##1;##2;{\vrule height\@tempdimq width ##1pt\kern ##2pt}%
    \\0.05;0.15;\\0.60;0.44;\\1.11;0.65;\\1.45;0.77;\\1.57;0.77;%
    \\1.45;0.65;\\1.11;0.44;\\0.60;0.15;\\0.05;0;}%
  \sfrsetoffsets{15pt}{#3}{#4}#5#6#7#8}%

\def\fancytympan#1#2#3#4#5#6#7#8{%
  \sbox#1{%
    \sfrcalclength{#2}{#3}{#4}%
    \unitlength\@tempdimq
    \vbox{\hsize\@tempdimq\offinterlineskip
      \hbox{\begin{picture}(1,0.1)
        \thicklines
        \put(0,0){\line(1,0){1}}
        \put(0,0){\line(5,1){0.5}}
        \put(0.5,0.1){\line(5,-1){0.5}}
        \thinlines
        \put(0.5,0.05){\circle*{0.072}}
      \end{picture}}
      \hbox{\vrule width\@tempdimq height 2pt}}}%
  \sfrsetoffsets{\ht#1}{#3}{#4}#5#6#7#8}%
\end{verbatim}

Now we can define:

\begin{verbatim}
\newcommand\templebox[1]{\sframe
  [1]\fancycolumn [2]\fancytympan
  [1]\fancycolumn [-]\-%
  {\separbox{3pt}{#1}}}
\end{verbatim}

which should be read as: after 3pt wide empty space is set around \=#1=,
first add columns to the left and the right, second put a tympan on top
of the result, and never put anything at the foot. Then typing this:

\begin{verbatim}
\templebox{hello there!} and \templebox{%
  \vbox{\hbox{b}\hbox{y}\hbox{e}}}
\end{verbatim}

will yield:

\leavevmode
\templebox{hello there!} and \templebox{%
  \vbox{\hbox{b}\hbox{y}\hbox{e}}}

More examples will be added later.

\section{Alignment and Units System with \<sframe>}

This section is yet to be written.

\section{\<broadbox> for Setting Line Wide Frames}

\noindent
\begin{templeboxed}
\<broadbox> can be useful to set frames that fill the line. Its use is best
explained through an example. Suppose we want to set a paragraph of text
in a line wide temple box. Then the lines will be filled by (from left to
right): a column, empty space added by \<separbox>, text, empty space and
a column. In other words: the whole line is available for setting text,
except for the space needed by the columns and the empty space set by
\<separbox>. The columns are 12pt each while \<separbox> adds 3pt wide empty
space to the left and the right: that makes a total of 30pt. So we write:
\=\templebox=\:\={\broadbox=\:\={30pt}{\<broadbox>= \=can= \=be= \ldots
\=\textit=\:\={dimension}.}}=, yielding a paragraph typeset like this.
So \=\broadbox=\:\={={\it dimension\/}\=}{={\it stuff\/}\=}= sets {\it stuff\/}
in a \<vbox> which has width line width minus \textit{dimension}.
\end{templeboxed}

\section{Environment Versions}

Some of the macros defined in \hhflxbox are available as \LaTeX\
environments. For example: instead of \=\broadbox=\:\={30pt}{={\it
text to be boxed\/}\=}= one could also use \=\begin=\:\={broadboxed}{30pt}=
{\it text to be boxed\/}\=\end=\:\={broadboxed}=. Similarly one
could use the environments \envirname{sepboxed}, \envirname{separboxed}
and \envirname{sframed} instead of the macros \<sepbox>, \<separbox>
and \<sframe>.

Actually I have to confess something: I lied to you about the
typesetting of the section about \<broadbox>. I did it with:

\begin{verbatim}
\newenvironment{templeboxed}{%
  \begin{sframed}%
    [1]\fancycolumn [2]\fancytympan
    [1]\fancycolumn [-]\-
    \begin{separboxed}{3pt}
      \begin{broadboxed}{30pt}
}{%
      \end{broadboxed}%
    \end{separboxed}%
  \end{sframed}%
}

\begin{templeboxed}%
  \<broadbox> can be useful to set frames that
       :        :        :        :        :
  width line width minus \textit{dimension}.
\end{templeboxed}
\end{verbatim}

I hope you will forgive me my cheating. I mean\ldots\ without using
the environments typesetting verbatim stuff is so troublesome\ldots

\section{Squeezing out white space}

When framing passages, leading white space should often be set above
the frame, not in the frame. Suppose a macro has been defined for
type-setting 1 cm of white space, some sentence, and another cm of
white space:

\begin{verbatim}
\newcommand\sentence[1]{\addvspace{1cm}#1\par\addvspace{1cm}}
\end{verbatim}
\newcommand\sentence[1]{\addvspace{1cm}#1\par\addvspace{1cm}}

Now suppose you want to templebox the sentence occasionally, without
modifying the argument of \<sentence>. (Keeping the argument of \<sentence>
free of templeboxing and other macros may be needed if the argument may
be reused for entries in the table of contents or the like.)
This may seem to be a solution:

\begin{verbatim}
\templebox{\vbox{\hsize 4cm\sentence{lots of bla}}}
\end{verbatim}

but it has an unwanted side-effect: the templebox now includes
the white space above and below the sentence:

\templebox{\vbox{\hsize 4cm\sentence{lots of bla}}}

The white space can be squeezed out of the templebox as follows
(rules have been added to clearify the amounts of white space added):

\begin{verbatim}
\hrule height 1pt width 100pt
\sbox0{\templebox{\vbox{\hsize 4cm
      \topsqueezeout
      \sentence{lots of bla}
      \botsqueezeout}}}
\topsqueezein
\box0%
\botsqueezein
\hrule height 1pt width 100pt
\end{verbatim}

which produces:

\hrule height 1pt width 100pt
\sbox0{\templebox{\vbox{\hsize 4cm
      \topsqueezeout
      \sentence{lots of bla}
      \botsqueezeout}}}
\topsqueezein
\box0%
\botsqueezein
\hrule height 1pt width 100pt

The macro \<topsqueezeout> stores and suppresses the leading white space, while
\<topsqueezein> typesets the suppressed leading space.
Similarly the macros \<botsqueezeout> and \<botsqueezein> take care of white space
at the bottom.

\end{document}


