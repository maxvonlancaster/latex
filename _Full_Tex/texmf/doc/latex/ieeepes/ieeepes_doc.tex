%%----------------------------------------------------------------------
%% ieeepes_doc.tex
%%
%% Volker Kuhlmann
%% c/o EEE Dept
%% University of Canterbury
%% Private Bag 4800
%% Christchurch, New Zealand
%% Email: KUHLMAV@ELEC.CANTERBURY.AC.NZ
%%
%% 4.0 20Apr99	Environments Table, Figure. Reserved name nophoto.
%% 1.7 03Apr96	Option noieeebox.
%% 1.6 20Nov95	Corrections.
%% 1.5 16Nov95	Added summary.
%% 1.4 14Nov95	Corrections.
%% 1.3 12Nov95	Finished first version.
%% 1.2 11Nov95	Extended.
%% 1.1 09Nov95	Extended.
%% Version 1.0, 24 Sep 1995
%%----------------------------------------------------------------------
 
\def\Filename{ieeepes\_doc.tex}
\def\Fileversion{V4.0}
\def\Filedate{20 April 1999}
\def\Packageversion{package ieeepes version 4.0, 1999/04/13}


\documentclass[twoside,twocolumn,a4paper]{article}


\usepackage{doc}
\MakeShortVerb |

\parindent	0em
\parskip	1.5ex plus.5ex minus.5ex
\columnsep	7mm

%\usepackage{ieeepes}

% Load PostScript Times font, if desired
%\usepackage{times}



\title{A \LaTeX-Package for IEEE PES Transactions}

\author{
	Volker Kuhlmann\\
	Dept of Electrical and Electronic Engineering\\
	Christchurch, New Zealand
}



\abovedisplayskip	0\abovedisplayskip
\belowdisplayskip	0\belowdisplayskip
\abovedisplayshortskip	0\abovedisplayshortskip
\belowdisplayshortskip	0\belowdisplayshortskip



\begin{filecontents}{ieeepes_doc.bbl}
\begin{thebibliography}{1}

\bibitem{IEEEPES_PaperPreparation}
J.~W. Hagge and L.~L. Grigsby,
\newblock ``Preparation of papers in a two-column format for ieee transactions
  on energy conversion ieee transactions on power delivery ieee transactions on
  power systems'',
\newblock in {\em IEEE Power Engineering Society Publication Guide}. IEEE Power
  Engineering Society, January 1995.

\end{thebibliography}
\end{filecontents}



\begin{document}

\def\today{\Filedate}
\newcommand\file{\textsf}
\newcommand\ag{\textsc}



\maketitle

{
\parskip	0ex plus.2ex
\tableofcontents
\par
}


%\newpage
\begin{abstract}
\bfseries
This \LaTeXe\ package implements the layout requirements for
Transactions of the IEEE Power Engineering Society (PES). This covers
the Transactions on Energy Conversion (T-EC), Transactions on Power
Delivery (T-PWRD), Transactions on Power Systems (T-PWRS), and Special
Publications. Discussions and closures can also be generated in the
required form.

This document is version \Fileversion, \Filedate, and describes
\Packageversion.

Thanks are due to John Crequer for proof-reading an early version of the
documentation.
\end{abstract}



\section{Introduction}

This document comprises the documentation for the \LaTeXe\ package ieeepes,
which implements the layout for publications of the Power Engineering Society
(a branch of IEEE). It is assumed that the reader is familiar with a standard
\LaTeX\ setup. Only new commands implemented by ieeepes are described in this
document. This document by itself is by now means sufficient in describing the
requirements to papers for submission to the IEEE PES. The
specifications~\cite{IEEEPES_PaperPreparation} must still be consulted.
Wherever possible, ieeepes enforces any requirements, but there are limits to
what can be done. Refer to section~\ref{s:cantdo} for a list of limitations of
ieeepes. Every author should be particularly careful with these.

Provided with ieeepes are the files \file{ieeepes\_skel.tex}, a skeleton for
new papers which might be useful, and \file{ieeepes\_check.tex}, a document
exercising the various features of ieeepes and intended as a test. It is also
useful as example.

This documentation can be compiled with standard \LaTeX, but the check file
needs ieeepes to be installed, and \file{ieeepes\_check.bib} to be available.

ieeepes requires \LaTeXe\ version 1998/06/01. It will probably work
with older versions of \LaTeXe, however this has not been tested. It
will not work with \LaTeX\ 2.09.

Please report any problems to Volker Kuhlmann\footnote
	{\textbf{v.kuhlmann@elec.canterbury.ac.nz}}%
, and I will do my best to fix them.



\section{Installation}
\label{s:installation}

The file \file{ieeepes.sty} must be copied into a directory where \TeX\ looks
for input files. The file \file{ieeepes.bst} must be copied into a directory
where \BibTeX\ looks for \BibTeX\ styles. The exact location of these
directories is dependant on the particular platform used and can not be
discussed here. Refer to the documentation of your \LaTeX\ software.

Package ieeepes requires package vmargin. Refer to section~\ref{s:options} for
other requirements. All the software mentioned can be downloaded from any
CTAN\footnote
	{Comprehensive TeX Archive Network.
	Try \textbf{ftp://ftp.dante.de/} or \textbf{http://www.dante.de/}.}
host. A copy of package vmargin is included for your convenience,
\file{vmargin.sty} should be installed in the same place as \file{ieeepes.sty}.



\section{Changes from older Versions}
\label{s:changes}

It is now necessary to use |\maketitle|. The \LaTeXe\ user interface changed,
and ieeepes can no longer use the |\AtBeginDocument| hook to insert the page
title. A warning is displayed if |\maketitle| is not used.

Using a reserved filename for the image file in the |biography| environment
suppresses the author image for this instance only. See
section~\ref{s:biographies}.



\section{Options to the Package}
\label{s:options}

The following options will be recognised by the ieeepes package:

\begin{description}
\item{\textbf{draft}}:
	Print page numbers. This violates requirements, but is very useful
	while writing the paper. This also enables markers which can be used to
	determine a useful width for a |minipage| (section~\ref{s:figtab}). Do
	not use this for the final version.

\item{\textbf{psphotos}}:
	A photographic image of the author can be printed into the
	space which is reserved for this. See
	section~\ref{s:biographies} for further details.

	This option requires the graphics bundle to be installed, as the
	graphicx package is loaded. The graphics bundle can be obtained from
	any CTAN host (see section~\ref{s:installation}).

\item{\textbf{photofit}}:
	This option will scale the photographic image of the author in
	the biography in both directions so that the image fills up the
	space provided. If the image had the required aspect ratio,
	this scaling will have no effect. If the aspect ratio was not
	as required, the image will be slightly distorted. However,
	this distortion might be less visible than an image which does
	not ``fit'' the space. Also see section~\ref{s:biographies}.
	
\item{\textbf{PStimes}}:
	Use font PostScript Times for the main document font.
	Typesetting mathematics is shifted over to PostScript fonts as
	much as possible without using commercial fonts.

	This option requires the psnfss bundle to be installed. The psnfss
	bundle can be obtained from any CTAN host (see
	section~\ref{s:installation}). Packages times and mathptm are loaded.

\item{\textbf{noieeebox}}:
	This option suppresses the empty box at the bottom of the left
	column of the title page. I added this option because someone
	had a purpose for it.

	Do not use this option for papers submitted to the IEEE PES!

\item{\textbf{puttoc}}:
	Put a table of contents into the paper, which is useful while writing
	the paper, but do not use this for the final version! This option has
	no effect unless option draft is also used.

\end{description}



\section{Document Structure}

The main structure of an IEEE PES document is as follows:

\begin{verbatim}
\documentclass[10pt,...]{article}
\usepackage[...]{ieeepes}
\title{...}
\author{... \and ... \and ...}
\begin{document}
\maketitle
\begin{abstract}
...
\end{abstract}
...
\end{document}
\end{verbatim}

The point size must be 10pt (which is the default). Do not use any of
the paper size options for the class, because the paper size is set up
by the ieeepes package.

The syntax for |\title| and |\author| is as for standard \LaTeX. There
can be any number of authors (separated by |\and|), but they all have
to fit next to each other on the width of the paper. No overfull
warning is generated if the author names overlap, or extend into the
margin. Within the argument to |\author| lines can be separated by
|\\|.

If the space available does not fit all the authors, other solutions
must be found. The standard \LaTeX\ command |\parbox| and environments
minipage and tabular might be useful, but |\and| probably is not. The
argument to |\author| is inside a tabular environment.

The standard \LaTeX\ commands |\tableofcontents| is not necessary and has been
disabled. It is now necessary to use |\maketitle| at the beginning of the
document. The use of \LaTeX's |\appendix| command might lead to papers not
meeting the requirements.

The text of the abstract is enclosed in the abstract environment, which
follows the |\begin{document}| and the |\maketitle|.

The sectioning commands |\section|, |\subsection|, and |\subsubsection|
are available, but |\paragraph| and |\subparagraph| can not be used in
IEEE PES papers.

Strictly speaking, the title text for |\section| should be all upper
case, but this can not always be accomplished easily. Currently it is
set in small caps. If this is not desired, entering the text in capitals
will have the desired result.

When the text of the paper is finished, the two columns on the last page must
be justified manually by inserting a |\columnbreak| at the correct position.
This should put the text on the last page equally into the two column.
Automation of this is tricky and left for a future version (if not left out).



\section{Paper Sizes}

Printing can be done on either A4 or USletter paper, there is no
difference for the resulting camera ready copy. Refer
to~\cite{IEEEPES_PaperPreparation} when using A4 paper, for cutting the
paper after printing.

When using dvips for generating PostScript code for printing, the
default paper size for which dvips generates code can be overridden with
the \textbf{-t} option: \textbf{-t letter} for USletter paper, and
\textbf{-t a4} for A4 paper. This might help to keep the printer happy.



\section{PostScript Fonts}

Package option PStimes switches the text font and as much math as
possible to PostScript Times (see section~\ref{s:options}). There are
no complete mathematical fonts in the public domain, if these are
desired then they must be purchased. Do not use option PStimes for
selecting purchased fonts.



\section{Figures and Tables}
\label{s:figtab}

Figures and tables are used exactly as before, except that their
contents is now centred by default. Care must be taken with table
captions, which have to be inserted \emph{before} the table.
Example:

\begin{verbatim}
\begin{table}
\caption{Table caption text.}
\label{label name}
   The table matter goes here.
\end{table}
\end{verbatim}

As always with \LaTeX, the |\label| must be after the |\caption|, and
inside the figure or table environments.

The new environments |Table| and |Figure| have been introduced to make figures
and tables easier to handle. Use of these environments is recommended because
they take care of a few things which otherwise would have to be done manually
(e.g.\ the caption position). Their syntax is:

\begin{verbatim}
\begin{Table}[FLOATPLACE]{LABEL}%
                 [TOC CAPTION]{CAPTION}
   The table matter goes here.
\end{Table}
\end{verbatim}

Arguments in square brackets are optional and can be left out, those in braces
are required. \ag{floatplace} is the float placement parameter, and
\ag{toc caption} is the caption for the table of contents if they have been
enabled with the package options draft and puttoc. \ag{toc caption}
defaults to \ag{caption}.

|Figure| has the same syntax as |Table|.

The width of the caption is held in |\capwidth|, and is initialised to
|0.8\columnwidth|.

Reference figures with |\figref|, and tables with |\tabref|. Their
syntax is equal to |\ref|. Use these two new reference commands within
as well as at the beginning of a sentence, and do not write out
``figure'', ``table'', or something to this effect. Do not use |\ref|
for figures or tables. Example:

\begin{verbatim}
  is shown in \tabref{table1} 
  and \figref{figure2}.
\end{verbatim}

Footnotes can be used within tables. For this the table must be put
inside a minipage environment.  The problem with this is that the width
of the minipage must be specified before the width of the contents can
be known. When writing the paper, specify the width as |1\columnwidth|,
and when finished, step by step reduce the width of the minipage (by
reducing the |1|) to the width of the table produced by tabular. 

To aid with this, markers can be printed which show the extent of the
tabular and the minipage. Ideally, the two arrows facing the same
direction are horizontally aligned. The markers are generated by
|\Lhighlight| and |\Rhighlight|, they do not take up any space and are
only displayed when option draft is in effect. |\Lhighlight| and
|\Rhighlight| are equivalent to |\mbox{}|.

\begin{verbatim}
\begin{table}
\caption{...}
\label{...}
\Lhighlight
\begin{minipage}{1\columnwidth}
   \centering
   \Lhighlight
   \begin{tabular}{...}
   Here can be footnotes.
   \end{tabular}
   \Rhighlight
\end{minipage}% <--!!
\Rhighlight
\end{table}
\end{verbatim}

Note the \% sign after |\end{minipage}|, which ensures that there is no white
space between the minipage and the arrow produced by the following
|\Rhighlight|. Refer to file \file{ieeepes\_check.tex} for an example.



\section{Equations}
\label{s:equations}

Equations are used in the same way as described in the \LaTeX\ manual.

For referencing equations, use |\equref| within a sentence, and
|\Equref| at the beginning of a sentence. The syntax is the same as for
|\ref|. Do not spell out ``Eq.'', ``equation'', or anything similar.
Using these two commands will keep your paper in line with requirements.



\section{Footnotes}
\label{s:footnotes}

There are no changes to the standard \LaTeX\ use of footnotes.



\section{Referencing}

For figure and table references, see section~\ref{s:figtab}. For
equation references, see section~\ref{s:equations}.



\section{Citations}
\label{s:citations}

There are no changes in the use of the |\cite| command. Do not spell
out ``ref.'', ``reference'', or similar.

For conference citations (\BibTeX\ entry type InProceedings) the
publication number (e.g. ``91CH3070-0'') is entered into the \BibTeX\
note entry field. This will result in the number being printed after
the page number instead of before, as shown in the example
paper~\cite{IEEEPES_PaperPreparation}. Correction of this is left for a
future version of the ieeepes package.

%The example paper~\cite{IEEEPES_PaperPreparation}

The \BibTeX\ style ieeepes.bst was derived from ieee.bst found on CTAN.
The only changes made are the spelling out of the month names (as in
plain.bst), and the definition of the additional strings:
\textbf{ieeepes},
\textbf{ieeetec},
\textbf{ieeetpd},
\textbf{ieeetps},
yielding the respective texts ``IEEE Power Engineering Society'', and
the titles of the three transactions.

A call to |\bibliographystyle| is performed by the ieeepes package, and
it is not necessary to use this command again.

The bibliography supplied with the ieeepes package should be consulted
for an example of how to enter bibliographic data.



\section{Biographies}
\label{s:biographies}

A biography for each author of the paper can be typeset with the |biography|
environment. Space is reserved for the image, which is inserted by the
publisher photographically. Optionally, an encapsulated postscript image of the
author can be printed with the text. The general syntax is:
%
\begin{verbatim}
\begin{biography}{AUTHOR NAME}%
                   [UP SHIFT]{FILENAME}
\end{verbatim}
%
Substitute the name of the author for \ag{author name}.

\ag{filename} is the name of the file containing the image of the author.
This argument can be empty and no image is printed. \ag{up shift} is
optional and is the amount by which the image is shifted up or down,
the default is 0\,mm. This parameter might be useful for certain aspect
ratios of the author's image. Package option psphotos is required for
printing author images, see section~\ref{s:options}.

Using the reserved \ag{filename} |nophoto| will suppress the insertion of an
image altogether and will not reserve space for one, for this instance only.
This is useful if not all authors of a paper want to have an image appear in
the paper.

Because of the use of the \TeX\ paragraph parameters |\hangindent| and
|\hangafter| to leave enough space for the photograph, it is vitally
important to have enough text material in the first paragraph of the
|biography| environment. Care should also be taken when the biography
starts close to the bottom of the column; if the photo does not fit
into the space left it will extend into the bottom margin.

As many |biography| environments as needed can be used.



\section{Summary}
\label{s:summary}

A summary is started with the |\summary| command, which is used much in
the same way as |\appendix| from \LaTeX. A summary can be put into a
separate document:
%
\begin{verbatim}
\begin{document}
\summary
  ...
\end{document}
\end{verbatim}
%
or appear at the end of the paper, before the |\end{document}|.



\section{Discussions}
\label{s:discussions}

The discussion environment is provided for typesetting discussions. The
syntax is:
%
\begin{verbatim}
\begin{discussion}{PAPER NUMBER}%
        {PAPER TITLE}%
        {AUTHOR NAMES}%
        {DISCUSSER NAME}%
        {AFFILIATION INCL ADDRESS}%
        {SHORT AFFILIATION}
\end{verbatim}
%
enter the respective data. \ag{author names} are the authors of the paper,
\ag{discusser name} is the author of the discussion about the paper. The
next argument is the affiliation including a complete mailing address,
while the last argument is of the form ``\emph{University of \ldots,
town, country}''.

The general document structure for a discussion is:
%
\begin{verbatim}
...
\begin{document}
\begin{discussion}{.....
.. text of discussion ..
\end{discussion}
\end{document}
\end{verbatim}
%
There can be multiple discussion environments, though this is not
of much use.



\section{Closures}
\label{s:closures}

Closures are written using the closure environment:
%
\begin{verbatim}
...
\begin{document}
\begin{closure}{AUTHOR NAME}
.. text of closure ..
\end{closure}
\end{document}
\end{verbatim}

There can be multiple closure environments in one document, but again
this is not of much use. It is however possible to have a closure
environment at the end of a paper, following the biographies, or the
summary.  This feature relies on an internal \LaTeX\ mechanism behaving
in a certain way, so caution is required. It works with the
example---but please report any problems.



\section{What This Package Can Not Do}
\label{s:cantdo}

There are a few things with which authors must take care themselves,
because they can not be enforced by \LaTeX.
Consult~\cite{IEEEPES_PaperPreparation} for details. Some are:

Table captions must be inserted \emph{before} the table. See
section~\ref{s:figtab} for details. Better, use the |Table| environment.

Commands provided for referencing figures, tables, and equations should
be used, and no additional words should be spelled out in the
sentence.

Punctuation marks follow the |\cite| command.

The main document point size must be 10pt.

Use initials for the Christian names of authors in the |\author|
command.

Ensure that there is enough material in the first paragraph of a biography
environment, and that the biography does not appear at the bottom of a page.

Ensure the two columns on the last page are balanced.



%\bibliographystyle{ieeepes}
%\bibliography{ieeepes_check}
% To remove the dependency on ieeepes.bst and ieeepes_check.bib for this
% document, the bibliography is copied into this file (see preamble,
% filecontents environment).
\bibliography{ieeepes_doc}



\end{document}

%%
%% EOF ieeepes_doc.tex
%%----------------------------------------------------------------------
