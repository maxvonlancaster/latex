%%%%%%%%%%%%%%%%%%%%%%%%%%%%%%%%%%%%%%%%%%%%%%%%%%%%%%%%%%%%%%%%%%%%%%%%%
%
%  INDEX LATEX FILE EXAMPLE
%
%%%%%%%%%%%%%%%%%%%%%%%%%%%%%%%%%%%%%%%%%%%%%%%%%%%%%%%%%%%%%%%%%%%%%%%%%
\documentclass[dvips]{article}
\usepackage{interactiveworkbook} % put in style directory; cannot appear in any other directory

\begin{document}

\exerciseintroduction{Index. $\;$ This is an index of, at most, twenty questions:
a multiple check box question, a multiple popup question, a multiple text
field question and a radio button question.
}

\vspace{.2in}

% locations of twenty question files for this exercise
% "exerquessetupone", ..., "exerquessetuptwenty" commands defined in buttonappearance.eps
\exerquessetupone{check.pdf}
\exerquessetuptwo{popup.pdf}
\exerquessetupthree{field.pdf}
\exerquessetupfour{radio.pdf}
\exerquessetupfive{check.pdf}
\exerquessetupsix{popup.pdf}
\exerquessetupseven{field.pdf}
\exerquessetupeight{radio.pdf}
\exerquessetupnine{check.pdf}
\exerquessetupten{popup.pdf}
\exerquessetupeleven{field.pdf}
\exerquessetuptwelve{radio.pdf}
\exerquessetupthirteen{check.pdf}
\exerquessetupfourteen{popup.pdf}
\exerquessetupfifteen{field.pdf}
\exerquessetupsixteen{radio.pdf}
\exerquessetupseventeen{check.pdf}
\exerquessetupeighteen{popup.pdf}
\exerquessetupnineteen{field.pdf}
\exerquessetuptwenty{radio.pdf}

% twenty (blank) question buttons
% "exerquesone", ..., "exerquestwenty" commands defined in buttonappearance.eps
\begin{center}
\begin{tabular}{||lc|lc||} \hline
    Q1. Check & \exerquesone   & Q11. Field & \exerqueseleven \\ \hline
    Q2. Popup & \exerquestwo   & Q12. Radio & \exerquestwelve\\ \hline
    Q3. Field & \exerquesthree & Q13. Check & \exerquesthirteen\\ \hline
    Q4. Radio & \exerquesfour  & Q14. Popup & \exerquesfourteen\\ \hline
    Q5. Check & \exerquesfive   & Q15. Field & \exerquesfifteen \\ \hline
    Q6. Popup & \exerquessix   & Q16. Radio & \exerquessixteen\\ \hline
    Q7. Field & \exerquesseven & Q17. Check & \exerquesseventeen\\ \hline
    Q8. Radio & \exerqueseight  & Q18. Popup & \exerqueseighteen\\ \hline
    Q9. Check & \exerquesnine & Q19. Field & \exerquesnineteen\\ \hline
    Q10. Popup & \exerquesten  & Q20. Radio & \exerquestwenty\\ \hline
\end{tabular}
\end{center}

\end{document}
