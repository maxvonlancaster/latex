\documentclass[jasheads]{jasthesis}

\begin{document}

\thesistitle{The \texttt{jasthesis} \LaTeX\ class}

\thesissubtitle{(How to typeset a thesis)}

\thesisvolume{Volume 1 of 1}

\thesisauthor{James A. Shepherd}

\thesisdegree{A dissertation submitted to the University of Life in accordance
with the requirements of\ldots}

\thesiswordcount{Word Count: \texttt{many}}

\thesismaketitle

\thesisabstract{A \LaTeX\ class that conforms to the requirements of a
University Thesis. It is approximately BS4821:1990.

Briefly these requirements are:

\begin{itemize}
\item A4 papersize
\item Consecutive numbering at the bottom center of each page.
\item Text in 1.5 line spacing (Titles and contents pages are still singlespaced)
\item 12pt font size.
\item Doublesided printing.
\item Top, bottom and outer margin at least 15mm. Inner margin 40mm.
\item Preliminary pages of:
\begin{itemize}
\item Title Page (listing chapters and sections)
\item Abstract
\item Dedication and acknowledgements
\item Author's Declaration
\item Table of Contents, List of Figures, List of Tables
\end{itemize}
\item Top left heading is the current chapter, top right current section
\end{itemize}

Note that when using the option \texttt{jasheads} the $*$-form of the {\ttfamily section}
command is not supported properly.

Also, the requirement that headings be in no more than 14pt is ignored.
This would look silly, given the large gaps in the text due to displayed
mathematics. Though if you really want it, the class file would not take much
editing to do it.
}

\thesisdedication{This document class is dedicated to my girlfriend, Codina Cotar.}

\thesisdeclaration{I declare that the work in this dissertation was carried out
in accordance with the Regulations of the\ldots}

\tableofcontents

%-------------------------------------------------------------------------------

\chapter{Using the Class}

Place a copy of the \texttt{jasthesis.cls} file in the same directory/folder
as the \texttt{.tex} file you are editing. Alternatively, if you are able to,
install the class in your \LaTeX\ source tree.

At the beginning of you \LaTeX\ use the command:

{\ttfamily\verb=\documentclass{jasthesis}=}

\noindent instead of your usual {\ttfamily\verb=\documentclass{\ldots}=}.

\section{Options}

The option {\ttfamily\verb=jasheads=} gives more modern headings on the top left
and right. So, you would invoke this with:

{\ttfamily\verb=\documentclass{jasthesis}=}

\noindent instead of the above command.

All the remaining command will usually be used immediately after the above.

\chapter{Title and Contents Pages}

\section{Title Page}

Use the following commands to enter the details for the title page:

{\ttfamily\verb=\thesistitle{\ldots}=}

{\ttfamily\verb=\thesissubtitle{\ldots}=}

{\ttfamily\verb=\thesisvolume{\ldots}=}

{\ttfamily\verb=\thesisauthor{\ldots}=}

{\ttfamily\verb=\thesisdegree{\ldots}=}

{\ttfamily\verb=\thesiswordcount{\ldots}=}

Then use the command {\ttfamily\verb=\thesismaketitle=} to actually produce the page.

\section{Contents Pages}

Using the standard \LaTeX\ command {\ttfamily\verb=\tableofcontents=}
will produce the table of contents with chapters and sections. It will
be only singlespaced.

If you require a single sided contents page, then this will have to be done manually.
Where you want the page break on the contents page, go to that section in your
\texttt{.tex} file and after the section command put the command:

{\ttfamily\verb=\addtocontents{toc}{\cleardoublepage}=}

The standard \LaTeX\ commands {\ttfamily\verb=\listoffigures=} and
{\ttfamily\verb=\listoftables=} may be used as usual.


\chapter{Abstract, Dedication, and Declaration}

These details are entered using the commands:

{\ttfamily\verb=\thesisabstract{\ldots}=}

{\ttfamily\verb=\thesisdedication{\ldots}=}

{\ttfamily\verb=\thesisdeclaration{\ldots}=}

These commands also produce the pages. The second one typesets the text in italics.
The last one produces the signed and date lines.

\chapter{Other}

The rest of the text will be 1.5 linespaced.

I guess that's all.

~

{\ttfamily\verb=http://www.jamesAshepherd.com/=}
\end{document}
