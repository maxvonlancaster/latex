\documentclass{nlctdoc}

\usepackage{amsmath}
\usepackage[utf8]{inputenc}
\usepackage[T1]{fontenc}
\usepackage{ifthen}
\usepackage[colorlinks,
            bookmarks,
            hyperindex=false,
            pdfauthor={Nicola L.C. Talbot},
            pdftitle={jmlr: LaTeX2e Classes for the Journal of Machine Learning Research},
            pdfkeywords={LaTeX,jmlr}]{hyperref}

\CheckSum{3348}
\OnlyDescription

\begin{document}
\MakeShortVerb{"}
\DeleteShortVerb{\|}

 \title{\LaTeXe\ Classes for the Journal of Machine
Learning Research}
 \author{Nicola L. C. Talbot\\[10pt]
\url{http://theoval.cmp.uea.ac.uk/~nlct/}}

 \date{2010-07-27 (version 1.08)}
 \maketitle
\tableofcontents

\section{Introduction}

The \clsfmt{jmlr} class is for articles that need to be formatted
according to the Journal of Machine Learning Research style. This
class is based on the \sty{jmlr2e} and \sty{jmlrwcp2e} packages
but has been adapted to enable it to work better with the
\cls{combine} class to collate the articles into a book.
\sectionref{sec:jmlr} describes how to use the \clsfmt{jmlr} class.

The \clsfmt{jmlrbook} class is for combining JMLR articles into a book.
This class uses \cls{combine} and \sty{hyperref}, which are
troublesome enough on their own but together are quite fragile.  The
\clsfmt{jmlrbook} class redefines some internals to get \clsfmt{combine}
and \clsfmt{hyperref} to work together but some packages (e.g.\
\sty{subfig} and \sty{pdfpages}) are likely to mess everything up
and cause errors.  This is why the guidelines to authors are fairly
stringent and why \clsfmt{jmlr} will give an error message if certain
packages are loaded.\footnote{Currently \clsfmt{jmlr} will check if
\sty{subfig}, \sty{pdfpages}, \sty{geometry}, \sty{psfig},
\sty{epsfig} and \sty{theorem} are loaded and will throw an error.
If other packages are found to be a problem, they will be added to
the list.} The \clsfmt{jmlrbook} class works best with PDF\LaTeX\ so
authors should ensure that their articles can compile with
PDF\LaTeX. \sectionref{sec:jmlrbook} describes how to use the 
\clsfmt{jmlrbook} class.

The \app{makejmlrbook} Perl script can be used to make a book that
uses the \clsfmt{jmlrbook} class. In addition to creating the print
and online versions of the book, it will compile the individual
articles, running Bib\TeX\ where necessary, and create a set of
HTML files containing a list of all the articles imported into the
book along with links to the abstracts and PDFs of the individual
articles. \sectionref{sec:makejmlrbook} describes how 
to use the \app{makejmlrbook} application.

\section{Required Packages}

The \clsfmt{jmlr} class is based on the \cls{scrartcl} class and loads
the following packages: \sty{amsmath}, \sty{amssymb},
\sty{natbib}, \sty{url}, \sty{graphicx} and \sty{algorithm2e}.
Note that unlike the \sty{jmlr2e} and \sty{jmlrwcp2e} packages,
this class file does not load the obsolete \sty{epsfig} package.

The \clsfmt{jmlrbook} class additionally loads the \cls{combine} class
and the following packages: \sty{hyperref}, \sty{xkeyval}, 
\sty{combnat} and \sty{setspace}.

The \app{makejmlrbook} script requires Perl, \TeX\ and \TeX4ht.

\section{Guidelines for Article Authors}
\label{sec:jmlr}

Article authors should use the \clsfmt{jmlr} class. This class
comes with example files \texttt{jmlr-sample.tex} and
\texttt{jmlrwcp-sample.tex}, which can be used as templates.

The following class options are available:
\begin{description}
\item[\clsopt{nowcp}]The article is for the Journal of Machine
Learning Research (default).
\item[\clsopt{wcp}] The article is for JMLR Workshop and Conference 
Proceedings.

\item[\clsopt{twocolumn}] Use two-column style.

\item[\clsopt{onecolumn}] Use one-column style (default).

\item[\clsopt{color}] Color version (see \sectionref{sec:color}).

\item[\clsopt{gray}] Grayscale version (see \sectionref{sec:color}).

\item[\clsopt{tablecaptiontop}] in a \env{table} environment,
\ics{floatconts} puts the caption at the top.

\item[\clsopt{tablecaptionbottom}] in a \env{table} environment,
\ics{floatconts} puts the caption at the bottom.
\end{description}

\subsection{Title Information}

The \clsfmt{jmlr} class uses different syntax from \sty{jmlr2e} and
\sty{jmlrwcp2e} to specify the title information. In particular, it
doesn't define \cs{jmlrheading} and \cs{ShortHeading}. Instead, the
following commands should be used:

\begin{definition}[\DescribeMacro{\jmlrvolume}]
\cs{jmlrvolume}\marg{number}
\end{definition}
This specifies the volume number. For example:
\begin{verbatim}
\jmlrvolume{2}
\end{verbatim}

\begin{definition}[\DescribeMacro{\jmlryear}]
\cs{jmlryear}\marg{year}
\end{definition}
This specifies the year. For example:
\begin{verbatim}
\jmlryear{2010}
\end{verbatim}

\begin{definition}[\DescribeMacro{\jmlrsubmitted}]
\cs{jmlrsubmitted}\marg{date}
\end{definition}
This specifies the submission date.

\begin{definition}[\DescribeMacro{\jmlrpublished}]
\cs{jmlrpublished}\marg{date}
\end{definition}
This specifies the publication date.

\begin{definition}[\DescribeMacro{\jmlrworkshop}]
\cs{jmlrworkshop}\marg{title}
\end{definition}
This specifies the workshop title (for use with the \clsopt{wcp}
class option).

The title information is specified using the commands described
below. These commands should typically go in the preamble. As
with most class files, The title itself is produced using
\begin{definition}[\DescribeMacro{\maketitle}]
\cs{maketitle}
\end{definition}
This command should go after \verb|\begin{document}|. For example:
\begin{verbatim}
\begin{document}
\maketitle
\end{verbatim}
Before \cs{maketitle}, you must specify the title information
using the following commands:

\begin{definition}[\DescribeMacro{\title}]
\cs{title}\oarg{short title}\marg{title}
\end{definition}
This specifies the article's title. A short title for the page
header can be supplied via the optional argument \meta{short title}.

\begin{definition}[\DescribeMacro{\editor}]
\cs{editor}\marg{name}
\end{definition}
This specifies the editor's name. If there is more than one 
editor, use:
\begin{definition}[\DescribeMacro{\editors}]
\cs{editors}\marg{names}
\end{definition}

\begin{definition}[\DescribeMacro{\author}]
\cs{author}\marg{author specs}
\end{definition}
This specifies the author. The specifications \meta{author specs}
are a bit different to \sty{jmlr2e} and \sty{jmlrwcp2e}. Use
\begin{definition}[\DescribeMacro{\Name}]
\cs{Name}\marg{author's name}
\end{definition}
to specify the author's name. Note that if the surname contains a
space it must be grouped (enclosed in braces \{\}). Similarly if
the initial letter of each forename is a diacritic it must be
grouped. (See below for examples.)
\begin{definition}[\DescribeMacro{\Email}]
\cs{Email}\marg{author's email}
\end{definition}
This specifies the author's email address. It should only be used
within the argument to \cs{author}.

\begin{definition}[\DescribeMacro{\and}]
\cs{and}
\end{definition}
This should be used to separate two authors with the same address.

\begin{definition}[\DescribeMacro{\AND}]
\cs{AND}
\end{definition}
This should be used to separate authors with different addresses.

\begin{definition}[\DescribeMacro{\\}]
\verb|\\|
\end{definition}
This should be used before an author's address or between authors
with the same address where there are more that two authors.

\begin{definition}[\DescribeMacro{\addr}]
\cs{addr}
\end{definition}
This should be used at the start of the address.

\begin{description}
\item[Example 1] Two authors with the same address:
\begin{verbatim}
\author{\Name{Jane Doe} \Email{abc@sample.com}\and
   \Name{John {Basey Fisher}} \Email{xyz@sample.com}\\
   \addr Address}
\end{verbatim}
In this example, the second author has a space in his surname
so the surname needs to be grouped.

\item[Example 2] Three authors with the same address:
\begin{verbatim}
\author{\Name{Fred Arnold {de la Cour}} \Email{an1@sample.com}\\
   \Name{Jack Jones} \Email{an3@sample.com}\\
   \Name{{\'E}louise {\'E}abhla Finchley} \Email{an2@sample.com}\\
   \addr Address}
\end{verbatim}
In this example, the third author has an accent on her
forename initials so grouping is required. 

\item[Example 3] Authors with a different address:
\begin{verbatim}
\author{\Name{John Smith} \Email{abc@sample.com}\\
  \addr Address 1
  \AND
  \Name{May Brown} \Email{xyz@sample.com}\\
  \addr Address 2
 }
\end{verbatim}
\end{description}

\subsection{Font Changing Commands}

Use the \LaTeXe\ font changing commands, such as \cs{bfseries} or
\cs{textbf}\marg{text}, rather than the obsolete \LaTeX2.09
commands, such as \cs{bf}.

\begin{definition}[\DescribeMacro{\url}]
\cs{url}\marg{address}
\end{definition}
This will typeset \meta{address} in a typewriter font. Special
characters, such as \verb|~|, are correctly displayed. Example:
\begin{verbatim}
\url{http://theoval.cmp.uea.ac.uk/~nlct/}
\end{verbatim}

\begin{definition}[\DescribeMacro{\mailto}]
\cs{mailto}\marg{email address}
\end{definition}
This will typeset the given email address in a typewriter font.
Note that this is not the same as \cs{Email}, which should only be
used in the argument of \cs{author}.

\subsection{Structure}

\begin{definition}[\DescribeEnv{abstract}]
\cs{begin}\{abstract\}\\
\meta{text}\\
\cs{end}\{abstract\}
\end{definition}
The abstract text should be displayed using the \envfmt{abstract}
environment.

\begin{definition}[\DescribeEnv{keywords}]
\cs{begin}\{keywords\}\meta{keyword list}\cs{end}\{keywords\}
\end{definition}
The keywords should be displayed using the \envfmt{keywords}
environment.

\begin{definition}[\DescribeMacro{\acks}]
\cs{acks}\marg{text}
\end{definition}
This displays the acknowledgements.

\begin{definition}[\DescribeMacro{\section}]
\cs{section}\marg{title}
\end{definition}
Section titles are created using \cs{section}. The heading is
automatically numbered and can be cross-referenced using
\cs{label} and \cs{ref}. Unnumbered sections can be produced
using:
\begin{definition}[\DescribeMacro{\section*}]
\cs{section*}\marg{title}
\end{definition}

\begin{definition}[\DescribeMacro{\subsection}]
\cs{subsection}\marg{title}
\end{definition}
Sub-section titles are created using \cs{subsection}. Unnumbered 
sub-sections can be produced using:
\begin{definition}[\DescribeMacro{\subsection*}]
\cs{subsection*}\marg{title}
\end{definition}

\begin{definition}[\DescribeMacro{\subsubsection}]
\cs{subsubsection}\marg{title}
\end{definition}
Sub-sub-section titles are created using \cs{subsubsection}.
Unnumbered sub-sub-sections can be produced using:
\begin{definition}[\DescribeMacro{\subsubsection*}]
\cs{subsubsection*}\marg{title}
\end{definition}

Further sectioning levels can be obtained using \cs{paragraph}
and \cs{subparagraph}, but these are unnumbered with running heads.

\begin{definition}[\DescribeMacro{\appendix}]
\cs{appendix}
\end{definition}
Use \cs{appendix} to switch to the appendices. This changes
\cs{section} to produce an appendix. Example:
\begin{verbatim}
\appendix
\section{Proof of Theorems}
\end{verbatim}

\subsection{Citations and Bibliography}

The \clsfmt{jmlr} class automatically loads \sty{natbib} and sets
the bibliography style to \texttt{plainnat}. References should
be stored in a \texttt{.bib} file.

\begin{definition}[\DescribeMacro{\bibliography}]
\cs{bibliography}\marg{bib file}
\end{definition}
This displays the bibliography.

\begin{definition}[\DescribeMacro{\citep}]
\cs{citep}\oarg{pre note}\oarg{post note}\marg{label}
\end{definition}
Use \cs{citep} for a parenthetical citation.

\begin{definition}[\DescribeMacro{\citet}]
\cs{citet}\oarg{note}\marg{label}
\end{definition}
Use \cs{citet} for a textual citation.

See the \ctandoc{natbib} for further details.

\subsection{Figures and Tables}

Floats, such as figures, tables and algorithms, are moving objects
and are supposed to float to the nearest convenient location.
Please don't force them to go in a particular place. In general
it's best to use the \texttt{htbp} specifier and don't put
the float in the middle of a paragraph (that is, make sure there's
a paragraph break above and below the float). Floats are supposed
to have a little extra space above and below them to make them 
stand out from the rest of the text. This extra space is put in
automatically and shouldn't need modifying.

To ensure consistency, please \emph{don't} try changing the
format of the caption by doing something like:
\begin{verbatim}
\caption{\textit{A Sample Caption.}}
\end{verbatim}
or
\begin{verbatim}
\caption{\em A Sample Caption.}
\end{verbatim}
You can, of course, change the font for individual words or
phrases. For example:
\begin{verbatim}
\caption{A Sample Caption With Some \emph{Emphasized Words}.}
\end{verbatim}

The \clsfmt{jmlr} class provides the following command for displaying
the contents of a figure or table:
\begin{definition}[\DescribeMacro{\floatconts}]
\cs{floatconts}\marg{label}\marg{caption command}\marg{contents}
\end{definition}
This ensures that the caption is correctly positioned and that
the contents are centered. For example:
\begin{verbatim}
\begin{table}[htbp]
\floatconts
  {tab:example}% label
  {\caption{An Example Table}}% caption command
  {%
    \begin{tabular}{ll}
    \bfseries Dataset & \bfseries Result\\
    Data1 & 0.123456
    \end{tabular}
  }
\end{table}
\end{verbatim}

The \clsfmt{jmlr} class automatically loads \sty{graphicx} which
defines:
\begin{definition}[\DescribeMacro{\includegraphics}]
\cs{includegraphics}\oarg{options}\marg{file name}
\end{definition}
where \meta{options} is a comma-separated list of options.

For example, suppose you have an image called 
\texttt{mypic.png} in a subdirectory called \texttt{images}:
\begin{verbatim}
\begin{figure}[htbp]
\floatconts
  {fig:example}% label
  {\caption{An Example Figure}}% caption command
  {\includegraphics[width=0.5\textwidth]{images/mypic}}
\end{figure}
\end{verbatim}

Note that you shouldn't specify the file extension when including
the image. It's helpful if you can also provide a grayscale
version of color images. This should be labeled as the color
image but with \texttt{-gray} immediately before the extension.
(The extension need not be the same as that of the color image.)
For example, if you have an image called \texttt{mypic.pdf}, the
grayscale can be called \texttt{mypic-gray.pdf}, 
\texttt{mypic-gray.png} or \texttt{mypic-gray.jpg}.
See \sectionref{sec:color} for further details.

\begin{definition}[\DescribeMacro{\includeteximage}]
\cs{includeteximage}\oarg{options}\marg{file name}
\end{definition}
If your image file is made up of \LaTeX\ code (e.g.\ \sty{tikz}
commands) the file can be included using \cs{includeteximage}.
The optional argument is a key=value comma-separated list
where the keys are a subset of those provided by 
\cs{includegraphics}. The main keys are: \texttt{width},
\texttt{height}, \texttt{scale} and \texttt{angle}.

\subsubsection{Sub-Figures and Sub-Tables}

The \sty{subfig} package causes a problem for \clsfmt{jmlrbook} so
the \clsfmt{jmlr} class will give an error if it is used. Therefore
the \clsfmt{jmlr} class provides its own commands for including
sub-figures and sub-tables.

\begin{definition}[\DescribeMacro{\subfigure}]
\cs{subfigure}\oarg{title}\oarg{valign}\marg{contents}
\end{definition}
This makes a sub-figure where \meta{contents} denotes the contents
of the sub-figure. This should also include the \cs{label}.
The first optional argument \meta{title} indicates a caption for
the sub-figure. By default, the sub-figures are aligned at the
base. This can be changed with the second optional argument
\meta{valign}, which may be one of: \texttt{t} (top), \texttt{c}
(centred) or \texttt{b} (base).

For example, suppose there are two images files, \texttt{mypic1.png}
and \texttt{mypic2.png}, in the subdirectory \texttt{images}.
Then they can be included as sub-figures as follows:
\begin{verbatim}
\begin{figure}[htbp]
\floatconts
  {fig:example2}% label for whole figure
  {\caption{An Example Figure.}}% caption for whole figure
  {%
    \subfigure{%
      \label{fig:pic1}% label for this sub-figure
      \includegraphics{images/mypic1}
    }\qquad % space out the images a bit
    \subfigure{%
      \label{fig:pic2}% label for this sub-figure
      \includegraphics{images/mypic2}
    }
  }
\end{figure}
\end{verbatim}

\begin{definition}[\DescribeMacro{\subtable}]
\cs{subtable}\oarg{title}\oarg{valign}\marg{contents}
\end{definition}
This is an analogous command for sub-tables. The default value
for \meta{valign} is \texttt{t}.

\subsection{Algorithms}

\begin{definition}[\DescribeEnv{algorithm}]
\cs{begin}\{algorithm\}\\
\meta{contents}\\
\cs{end}\{algorithm\}
\end{definition}
Enumerated textual algorithms can be displayed using the 
\envfmt{algorithm} environment. Within this environment, use
\ics{caption} to set the caption (and \ics{label} to cross-reference
it). Within the body of the environment you can use the
\env{enumerate} environment.

\begin{definition}[\DescribeEnv{enumerate*}]
\cs{begin}\{enumerate*\}\\
\cs{item} \meta{text}\\
\ldots\\
\cs{end}\{enumerate*\}
\end{definition}
If you want to have nested \env{enumerate} environments but you want
to keep the same numbering throughout the algorithm, you can use the
\envfmt{enumerate*} environment, provided by the \clsfmt{jmlr}
class. For example:
\begin{verbatim}
\begin{enumerate*}
  \item Set the label of vertex $s$ to 0
  \item Set $i=0$
  \begin{enumerate*}
    \item \label{step:locate}Locate all unlabelled vertices 
          adjacent to a vertex labelled $i$ and label them $i+1$
    \item If vertex $t$ has been labelled,
    \begin{enumerate*}
      \item[] the shortest path can be found by backtracking, and 
      the length is given by the label of $t$.
    \end{enumerate*}
    otherwise
    \begin{enumerate*}
      \item[] increment $i$ and return to step~\ref{step:locate}
    \end{enumerate*}
  \end{enumerate*}
\end{enumerate*}
\end{algorithm}
\end{verbatim}


\begin{definition}[\DescribeEnv{algorithm2e}]
\cs{begin}\{algorithm2e\}\\
\meta{contents}\\
\cs{end}\{algorithm2e\}
\end{definition}
Pseudo code can be displayed using the \envfmt{algorithm2e} environment,
provided by the \sty{algorithm2e} package, which is automatically
loaded. For example:
\begin{verbatim}
\begin{algorithm2e}
\caption{Computing Net Activation}
\label{alg:net}
\dontprintsemicolon
\linesnumbered
\KwIn{$x_1, \ldots, x_n, w_1, \ldots, w_n$}
\KwOut{$y$, the net activation}
$y\leftarrow 0$\;
\For{$i\leftarrow 1$ \KwTo $n$}{
  $y \leftarrow y + w_i*x_i$\;
}
\end{algorithm2e}
\end{verbatim}

See the \ctandoc{algorithm2e} for more details.

\subsection{Description Lists}

\begin{definition}[\DescribeEnv{altdescription}]
\cs{begin}\{altdescription\}\marg{widest label}\\
\cs{item}\oarg{label} \meta{item text}\\
\cs{end}\{altdescription\}
\end{definition}
In addition to the standard \env{description} environment, the
\clsfmt{jmlr} class also provides the \envfmt{altdescription} environment.
This has an argument that should be the widest label used in the
list. For example:
\begin{verbatim}
\begin{altdescription}{differentiate}
\item[add] A method that adds two variables.
\item[differentiate] A method that differentiates a function.
\end{altdescription}
\end{verbatim}

\subsection{Theorems, Lemmas etc}

The \clsfmt{jmlr} class provides the following theorem-like
environments: \env{theorem}, \env{example}, \env{lemma},
\env{proposition}, \env{remark}, \env{corollary}, \env{definition},
\env{conjecture} and \env{axiom}. Within the body of those
environments, you can use the \env{proof} environment to display the
proof if need be. The theorem-like environments all take an
optional argument, which gives the environment a title. For example:

\begin{verbatim}
\begin{theorem}[An Example Theorem]
\label{thm:example}
This is the theorem.
\begin{proof}
This is the proof.
\end{proof}
\end{theorem}
\end{verbatim}

\subsection{Cross-Referencing}
\label{sec:crossref}

Always use \ics{label} when cross-referencing, rather than writing
the number explicitly. The \clsfmt{jmlr} class provides some 
convenience commands to assist referencing. These commands,
described below, can all take a comma-separated list of labels.

\begin{definition}[\DescribeMacro{\sectionref}]
\cs{sectionref}\marg{label list}
\end{definition}
Used to refer to a section or sections. For example, if you defined 
a section as follows:
\begin{verbatim}
\section{Results}\label{sec:results}
\end{verbatim}
you can refer to it as follows:
\begin{verbatim}
The results are detailed in \sectionref{sec:results}.
\end{verbatim}
This command may also be used for sub-sections and sub-sub-sections.

\begin{definition}[\DescribeMacro{\appendixref}]
\cs{appendixref}\marg{label list}
\end{definition}
Used to refer to an appendix or multiple appendices.

\begin{definition}[\DescribeMacro{\equationref}]
\cs{equationref}\marg{label list}
\end{definition}
Used to refer to an equation or multiple equations.

\begin{definition}[\DescribeMacro{\tableref}]
\cs{tableref}\marg{label list}
\end{definition}
Used to refer to a table or multiple tables. This can also be
used for sub-tables where the main table number is also required.

\begin{definition}[\DescribeMacro{\subtabref}]
\cs{subtabref}\marg{label list}
\end{definition}
Used to refer to sub-tables without the main table number, e.g.
(\emph{a}) or (\emph{b}).

\begin{definition}[\DescribeMacro{\figureref}]
\cs{figureref}\marg{label list}
\end{definition}
Used to refer to a figure or multiple figures. This can also
be used for sub-figures where the main figure number is also
required, e.g.\ 2(\emph{a}) or 4(\emph{b}).

\begin{definition}[\DescribeMacro{\subfigref}]
\cs{subfigref}\marg{label list}
\end{definition}
Used to refer to sub-figures without the main figure number, e.g.
(\emph{a}) or (\emph{b}).

\begin{definition}[\DescribeMacro{\algorithmref}]
\cs{algorithmref}\marg{label list}
\end{definition}
Used to refer to an algorithm or multiple algorithms.

\begin{definition}[\DescribeMacro{\theoremref}]
\cs{theoremref}\marg{label list}
\end{definition}
Used to refer to a theorem or multiple theorems.

\begin{definition}[\DescribeMacro{\lemmaref}]
\cs{lemmaref}\marg{label list}
\end{definition}
Used to refer to a lemma or multiple lemmas.

\begin{definition}[\DescribeMacro{\remarkref}]
\cs{remarkref}\marg{label list}
\end{definition}
Used to refer to a remark or multiple remarks.

\begin{definition}[\DescribeMacro{\corollaryref}]
\cs{corollaryref}\marg{label list}
\end{definition}
Used to refer to a corollary or multiple corollaries.

\begin{definition}[\DescribeMacro{\definitionref}]
\cs{definitionref}\marg{label list}
\end{definition}
Used to refer to a definition or multiple definitions.

\begin{definition}[\DescribeMacro{\conjectureref}]
\cs{conjectureref}\marg{label list}
\end{definition}
Used to refer to a conjecture or multiple conjectures.

\begin{definition}[\DescribeMacro{\axiomref}]
\cs{axiomref}\marg{label list}
\end{definition}
Used to refer to an axiom or multiple axioms.

\begin{definition}[\DescribeMacro{\exampleref}]
\cs{exampleref}\marg{label list}
\end{definition}
Used to refer to an example or multiple examples.

\subsection{Mathematics}

The \clsfmt{jmlr} class loads the \sty{amsmath} package so you can use
any of the commands and environments defined in that package.  A
brief summary of some of the more common commands and environments
is provided here.  See the \ctandoc{amsmath} for further details.

\begin{definition}[\DescribeMacro{\set}]
\cs{set}\marg{text}
\end{definition}
In addition to the commands provided by \sty{amsmath}, the
\clsfmt{jmlr} class also provides the \cs{set} command which can
be used to typeset a set. For example:
\begin{verbatim}
The universal set is denoted $\set{U}$
\end{verbatim}

Unnumbered single-line equations should be displayed using
\cs{[} and \cs{]}. For example:
\begin{verbatim}
\[E = m c^2\]
\end{verbatim}
Numbered single-line equations should be displayed using the
\env{equation} environment. For example:
\begin{verbatim}
\begin{equation}\label{eq:trigrule}
\cos^2\theta + \sin^2\theta \equiv 1
\end{equation}
\end{verbatim}
Multi-lined numbered equations should be displayed using the
\env{align} environment. For example:
\begin{verbatim}
\begin{align}
f(x) &= x^2 + x\label{eq:f}\\
f'(x) &= 2x + 1\label{eq:df}
\end{align}
\end{verbatim}
Unnumbered multi-lined equations should be displayed using the
\env{align*} environment. For example:
\begin{verbatim}
\begin{align*}
f(x) &= (x+1)(x-1)\\
&= x^2 - 1
\end{align*}
\end{verbatim}
If you want to mix numbered with unnumbered lines use the
\env{align} environment and suppress unwanted line numbers with
\cs{nonumber}. For example:
\begin{verbatim}
\begin{align}
y &= x^2 + 3x - 2x + 1\nonumber\\
&= x^2 + x + 1\label{eq:y}
\end{align}
\end{verbatim}
An equation that is too long to fit on a single line can be
displayed using the \env{split} environment. 

Text can be embedded in an equation using \ics{text}\marg{text} or 
you can use \ics{intertext}\marg{text} to interupt a multi-line
environment such as \env{align}.

Predefined operator names are listed in \tableref{tab:operatornames}. 
For additional operators, either use 
\begin{definition}[\DescribeMacro{\operatorname}]
\cs{operatorname}\marg{name}
\end{definition}
for example
\begin{verbatim}
If $X$ and $Y$ are independent,
$\operatorname{var}(X+Y) = 
\operatorname{var}(X) + \operatorname{var}(Y)$
\end{verbatim}
or declare it with
\begin{definition}[\DescribeMacro{\DeclareMathOperator}]
\cs{DeclareMathOperator}\marg{command}\marg{name}
\end{definition}
for example
\begin{verbatim}
\DeclareMathOperator{\var}{var}
\end{verbatim}
and then use this new command:
\begin{verbatim}
If $X$ and $Y$ are independent,
$\var(X+Y) = \var(X)+\var(Y)$
\end{verbatim}

If you want limits that go above and
below the operator (like \ics{sum}) use the starred versions
(\ics{operatorname*} or \ics{DeclareMathOperator*}).

\begin{table}[htbp]
\caption{Predefined Operator Names (taken from 
   \sty{amsmath} documentation)}
\label{tab:operatornames}%
\vskip\baselineskip
\centering
\begin{tabular}{rlrlrlrl}
\cs{arccos} & $\arccos$ &  \cs{deg} & $\deg$ &  \cs{lg} & $\lg$ &  \cs{projlim} & $\projlim$ \\
\cs{arcsin} & $\arcsin$ &  \cs{det} & $\det$ &  \cs{lim} & $\lim$ &  \cs{sec} & $\sec$ \\
\cs{arctan} & $\arctan$ &  \cs{dim} & $\dim$ &  \cs{liminf} & $\liminf$ &  \cs{sin} & $\sin$ \\
\cs{arg} & $\arg$ &  \cs{exp} & $\exp$ &  \cs{limsup} & $\limsup$ &  \cs{sinh} & $\sinh$ \\
\cs{cos} & $\cos$ &  \cs{gcd} & $\gcd$ &  \cs{ln} & $\ln$ &  \cs{sup} & $\sup$ \\
\cs{cosh} & $\cosh$ &  \cs{hom} & $\hom$ &  \cs{log} & $\log$ &  \cs{tan} & $\tan$ \\
\cs{cot} & $\cot$ &  \cs{inf} & $\inf$ &  \cs{max} & $\max$ &  \cs{tanh} & $\tanh$ \\
\cs{coth} & $\coth$ &  \cs{injlim} & $\injlim$ &  \cs{min} & $\min$ \\
\cs{csc} & $\csc$ &  \cs{ker} & $\ker$ &  \cs{Pr} & $\Pr$
\end{tabular}\par
\begin{tabular}{rlrl}
\cs{varlimsup} & $\varlimsup$ 
& \cs{varinjlim} & $\varinjlim$\\
\cs{varliminf} & $\varliminf$ 
& \cs{varprojlim} & $\varprojlim$
\end{tabular}

\end{table}

\subsection{Color vs Grayscale}
\label{sec:color}

It's helpful if authors supply grayscale versions of their
articles in the event that the article is to be incorporated into
a black and white printed book. With external PDF, PNG or JPG
graphic files, you just need to supply a grayscale version of the
file. For example, if the file is called \texttt{myimage.png},
then the gray version should be \texttt{myimage-gray.png} or
\texttt{myimage-gray.pdf} or \texttt{myimage-gray.jpg}. You don't
need to modify your code. The \clsfmt{jmlr} class checks for
the existence of the grayscale version if it is print mode 
(provided you have used \ics{includegraphics} and haven't
specified the file extension).

\begin{definition}[\DescribeMacro{\ifprint}]
\cs{ifprint}\marg{true part}\marg{false part}
\end{definition}
You can use \cs{ifprint} to determine which mode you are in.
For example:
\begin{verbatim}
in \figureref{fig:nodes}, the
\ifprint{dark gray}{purple}
ellipse represents an input and the
\ifprint{light gray}{yellow} ellipse
represents an output.
\end{verbatim}
Another example:
\begin{verbatim}
{\ifprint{\bfseries}{\color{red}}important text!}
\end{verbatim}

You can use the class option \clsopt{gray} to see how the
document will appear in gray scale mode.

The \sty{xcolor} class is loaded with the \pkgoptfmt{x11names}
option, so you can use any of the x11 predefined colors (listed
in the \ctandoc{xcolor}).

\subsection{Where To Go For Help}

If you have a \LaTeX\ query, the first place to go to is the 
\urlfootref{http://www.tex.ac.uk/faq}{UK TUG FAQ}.

If you are unfamiliar or just getting started with \LaTeX, there's
a list of on-line introductions to \LaTeX\ at:
\url{http://www.tex.ac.uk/cgi-bin/texfaq2html?label=man-latex}

There are also forums, mailing lists and newsgroups. For example,
the \LaTeX\ Community (\url{http://www.latex-community.org/}),
the \texttt{texhax} mailing list 
(\url{http://tug.org/mailman/listinfo/texhax}) and
\texttt{comp.text.tex} (archives available at
\url{http://groups.google.com/group/comp.text.tex/}).

Documentation for packages or classes can be found using the
\texttt{texdoc} application. For example:
\begin{verbatim}
texdoc natbib
\end{verbatim}
Alternatively, you can go to 
\texttt{http://www.ctan.org/pkg/}\meta{name} where
\meta{name} is the name of the package. For example:
\url{http://www.ctan.org/pkg/natbib}

For a general guide to preparing papers (regardless of whether you
are using \LaTeX\ or a word processor), see Kate L.~Turabian, \qt{A
manual for writers of term papers, theses, and dissertations}, The
University of Chicago Press, 1996.

\section{Guidelines for Production Editors}
\label{sec:jmlrbook}

The \clsfmt{jmlrbook} class can be used to combine articles that
use the \clsfmt{jmlr} document class into a book. The following sample
files are provided: \texttt{paper1/paper1.tex},
\texttt{paper2/paper2.tex}, \texttt{paper3/paper3.tex},
\texttt{jmlr-sample.tex}, \texttt{jmlrwcp-sample.tex} and
\texttt{jmlrbook-sample.tex}.  All but the last named one are
articles using the \clsfmt{jmlr} class. The last one
(\texttt{jmlrbook-sample.tex}) uses the \clsfmt{jmlrbook} class file to
combine the articles into a book.  Note that no modifications are
needed to the files using the \clsfmt{jmlr} class when they are
imported into the book. They can either be compiled as stand-alone
articles or with the entire book.

Before you compile the book, make sure that all the articles 
compile as stand-alone documents (and run Bib\TeX\ where
necessary). You can use the \app{makejmlrbook} Perl script to compile
the book and create associated HTML files. See 
\sectionref{sec:makejmlrbook} for details.

\subsection{\clsfmt{jmlrbook} Class Options}
\begin{description}
\item[\clsopt{nowcp}]The imported pre-published articles were 
published in the Journal of Machine Learning Research (default).
\item[\clsopt{wcp}] The imported pre-published articles were
published in the JMLR Workshop and Conference Proceedings.

If the book has a mixture of JMLR and JMLR WCP articles, you
can switch between them using
\begin{definition}[\DescribeMacro{\jmlrwcp}]
\cs{jmlrwcp}
\end{definition}
and
\begin{definition}[\DescribeMacro{\jmlrnowcp}]
\cs{jmlrnowcp}
\end{definition}
Alternatively, you can set the name of the journal or conference
proceedings using:
\begin{definition}[\DescribeMacro{\jmlrproceedings}]
\cs{jmlrproceedings}\marg{short title}\marg{long title}
\end{definition}

\item[\clsopt{color}] Color version (see \sectionref{sec:color}). 
Use this option for the on-line version with hyperlinks enabled
(default).

\item[\clsopt{gray}] Grayscale version (see \sectionref{sec:color}). 
Use this option for the print version without hyperlinks.

\item[\clsopt{tablecaptiontop}] in a \env{table} environment,
\ics{floatconts} puts the caption at the top.

\item[\clsopt{tablecaptionbottom}] in a \env{table} environment,
\ics{floatconts} puts the caption at the bottom.

\item[\clsopt{letterpaper}] Set the paper size to letter (default).

\item[\clsopt{7x10}] Set the paper size to $7\times10$ inches.

\item[\clsopt{prehyperref}] The \clsfmt{jmlrbook} automatically loads
the \sty{hyperref} package, but some packages need to be loaded
before \sty{hyperref}. This information can be specified using
the \clsopt{prehyperref} option. This is a key=value option. For
example, to load the packages \styfmt{foo} and \styfmt{bar} before 
\sty{hyperref}, you can do:
\begin{verbatim}
\documentclass[prehyperref={\usepackage{foo,bar}}]{jmlrbook}
\end{verbatim}
or:
\begin{verbatim}
\documentclass[prehyperref={\usepackage{foo}\usepackage{bar}}]{jmlrbook}
\end{verbatim}

\item[\clsopt{10pt}] Use 10pt as the normal text size.
\item[\clsopt{11pt}] Use 11pt as the normal text size (default).
\item[\clsopt{12pt}] Use 12pt as the normal text size.

\end{description}

\subsection{The Preamble}

Any packages that the imported articles load (which aren't 
automatically loaded by \clsfmt{jmlr}) must be loaded in the book's
preamble. For example, if one or more of the articles load the
\sty{siunitx} package, this package must be loaded in the book.

Commands that are defined in the imported articles will be local
to that article unless they have been globally defined using
\ics{gdef} or \ics{global}. Since most authors use \ics{newcommand}
and \ics{newenvironment} (or \ics{renewcommand} and 
\ics{renewenvironment}) this shouldn't cause a conflict if more
that one article has defined the same command or environment.
For example, in the sample files supplied, both 
\texttt{paper1/paper1.tex} and \texttt{paper2/paper2.tex} have 
defined the command \cs{samplecommand} using \cs{newcommand}. As
long as this command isn't also defined in the book, there won't
be a conflict.

\begin{definition}[\DescribeMacro{\title}]
\cs{title}\oarg{PDF title}\marg{book title}
\end{definition}
In the book preamble, \cs{title} sets the book title and the optional
argument is used for the PDF title, which will be displayed
when the reader views the PDF file's properties in their PDF viewer.
(Note that in the imported articles, \cs{title} sets the article's
title and the optional argument sets the short title for the
page header and table of contents.)

\begin{definition}[\DescribeMacro{\author}]
\cs{author}\oarg{PDF author(s)}\marg{book author(s)}
\end{definition}
In the book preamble, \cs{author} sets the book's author (or editor)
and the optional argument is used for the PDF author, which will be
displayed when the reader views the PDF file's properties in their
PDF viewer.  (Note that in the imported articles, \cs{author} sets
the article's author and the optional argument sets the short author
list for the page header.)

\begin{definition}[\DescribeMacro{\volume}]
\cs{volume}\marg{number}
\end{definition}
This command sets the book's volume number. Omit if the book has no
volume number.

\begin{definition}[\DescribeMacro{\subtitle}]
\cs{subtitle}\marg{sub-title}
\end{definition}
This command sets the book's subtitle. Omit if the book has no 
sub-title.

\begin{definition}[\DescribeMacro{\logo}]
\cs{logo}\marg{image command}
\end{definition}
This sets the book's title image. Use \ics{includegraphics} and
omit the file extension. If you provide a grayscale version as
well as a color version, the grayscale version will be used for
the print version of the book. (See \sectionref{sec:color} 
for further details.)

\begin{definition}[\DescribeMacro{\team}]
\cs{team}\marg{team title}
\end{definition}
This can be used to set the name of the editorial team. This 
command may be omitted if not required.

\begin{definition}[\DescribeMacro{\productioneditor}]
\cs{productioneditor}\marg{name}
\end{definition}
This command may be used to name the production editor. The command
may be omitted if not required.

See \sectionref{sec:modifytitle} for details on how to modify the
layout of the title page.

\subsection{Main Book Commands}

All commands that are provided by the \clsfmt{jmlr} class are 
also available with the \clsfmt{jmlrbook} class, but some commands
might behave differently depending on whether they are in the
main part of the book or within the imported articles.

In the main part of the book you can use the following commands:
\begin{definition}[\DescribeMacro{\maketitle}]
\cs{maketitle}
\end{definition}
This displays the book's title page. Note that \cs{maketitle} has
a different effect when used in imported articles.

\begin{definition}[\DescribeMacro{\frontmatter}]
\cs{frontmatter}
\end{definition}
Use this command at the start of the front matter (e.g.\ before the
foreword or preface). This will make chapters unnumbered even if you
use \cs{chapter} instead of \cs{chapter*}. It also sets the page
style and sets the page numbering to lower case Roman numerals.

\begin{definition}[\DescribeEnv{authorsignoff}]
\cs{begin}\{authorsignoff\}\\
\meta{author list}\\
\cs{end}\{authorsignoff\}
\end{definition}
This environment may be used by the author signing off at the end of a chapter such as the
foreword. Within the environment use:
\begin{definition}[\DescribeMacro{\Author}]
\cs{author}\marg{details}
\end{definition}
for the author's details. More than one \cs{Author} should be used
if there is more than one author. Example:
\begin{verbatim}
\begin{authorsignoff}
\Author{Nicola Talbot\\
University of East Anglia}
\Author{Anne Author\\
University of No Where}
\end{authorsignoff}
\end{verbatim}

\begin{definition}[\DescribeEnv{signoff}]
\cs{begin}\{signoff\}\oarg{team name}\marg{date}\\
\meta{editor list}\\
\cs{end}\{signoff\}
\end{definition}
This environment may be used by the editorial team when signing off
a chapter such as the preface. If the optional argument is omitted,
\qt{The Editorial Team} is used.

Within the environment use:
\begin{definition}[\DescribeMacro{\Editor}]
\cs{Editor}\marg{details}
\end{definition}
for each editor. Example:
\begin{verbatim}
\begin{signoff}{March 2010}
% First editor:
\Editor{Nicola Talbot\\
University of East Anglia\\
\mailto{N.Talbot@uea.ac.uk}}
% Second editor:
\Editor{Anne Editor\\
University of Nowhere\\
\mailto{ae@sample.com}}
\end{signoff}
\end{verbatim}

\begin{definition}[\DescribeMacro{\tableofcontents}]
\cs{tableofcontents}
\end{definition}
This command displays the book's table of contents. Note that it
has a different effect if used in an imported article.

\begin{definition}[\DescribeMacro{\mainmatter}]
\cs{mainmatter}
\end{definition}
Use this command to switch to the book's main matter. This will 
switch the chapter numbering back on, reset the page numbering to
Arabic and set up the main page style.

\begin{definition}[\DescribeMacro{\part}]
\cs{part}\oarg{short title}\marg{title}
\end{definition}
If used in the main part of the book, this command will start a
new part and issue a clear double page. Note that this command
has a different effect if used in an imported article.

\begin{definition}[\DescribeMacro{\addtocpart}]
\cs{addtocpart}\marg{title}
\end{definition}
This adds \meta{title} to the table of contents, issues a clear
double page, but doesn't display any text or affect the part
numbering.

\begin{definition}[\DescribeMacro{\chapter}]
\cs{chapter}\oarg{short title}\marg{title}
\end{definition}
This command may be used in the main body of the book but will 
cause an error if used within an imported article.

\begin{definition}[\DescribeMacro{\section}]
\cs{section}\oarg{short title}\marg{title}
\end{definition}
\begin{definition}[\DescribeMacro{\subsection}]
\cs{subsection}\oarg{short title}\marg{title}
\end{definition}
\begin{definition}[\DescribeMacro{\subsubsection}]
\cs{subsubsection}\oarg{short title}\marg{title}
\end{definition}
\begin{definition}[\DescribeMacro{\paragraph}]
\cs{paragraph}\oarg{short title}\marg{title}
\end{definition}
\begin{definition}[\DescribeMacro{\subparagraph}]
\cs{subparagraph}\oarg{short title}\marg{title}
\end{definition}
These commands may be used in the main body of the book or within
imported articles. In the main body of the book they need to be
within a chapter and will be numbered according to the chapter.

\begin{definition}[\DescribeMacro{\appendix}]
\cs{appendix}
\end{definition}
If used in the main body of the book, this will switch to the
book appendices. Subsequent \cs{chapter} commands will produce the
appendices. If used within an imported article, it will switch
to the article appendices and won't affect the main part of
the book.

\begin{definition}[\DescribeEnv{jmlrpapers}]
\cs{begin}\{jmlrpapers\}\\
\meta{imported papers}\\
\cs{end}\{jmlrpapers\}
\end{definition}
This environment must be used when importing articles. Within this
environment, use the following commands to import articles:
\begin{definition}[\DescribeMacro{\importpubpaper}]
\cs{importpubpaper}\oarg{label}\marg{directory}\marg{file}\marg{pages}
\end{definition}
This imports an article that has already been published elsewhere.
The \meta{pages} argument should be the page range from the
\emph{previously published} version of this article. This may not
necessarily be the same as the page range of the article in the book.
The directory the imported file is contained in is is given by
\meta{directory}. If the file is in the same directory as the 
book, use a dot. The file name is given by \meta{file}. The article
is also given a label, specified by the optional argument. This
is \meta{directory}/\meta{file} by default. The label is used
as a prefix to labels in the imported articles which ensures that
cross-references are unique. You can also use this label to reference
the article elsewhere in the book (see \sectionref{sec:bkcrossref}).

\begin{definition}[\DescribeMacro{\importpaper}]
\cs{importpaper}\oarg{label}\marg{directory}\marg{file}
\end{definition}
Imports an article that is being published in the book. The 
arguments are the same as above except that there is no page
range (the page range is computed automatically).

\begin{definition}[\DescribeMacro{\importarticle}]
\cs{importarticle}\oarg{label}\marg{directory}\marg{file}
\end{definition}
This imports an article that hasn't been published elsewhere. There
is no page range, but the other arguments are the same as
those describe above for \cs{importpubpaper}.

Example: to import a previously published paper 
\texttt{paper1/paper1.tex} and an unpublished paper
\texttt{paper2/paper2.tex}:
\begin{verbatim}
\begin{jmlrpapers}
\importpubpaper{paper1}{paper1}{23--45}
\importarticle{paper2}{paper2}
\end{jmlrpapers}
\end{verbatim}

\subsubsection{Two Column Articles in a One Column Book}

The \clsfmt{jmlrbook} class column style will override the column style
of the imported articles. You can use the \clsopt{twocolumn} class
option to \clsfmt{jmlrbook}, but this will make the whole book with
two columns. If you only want the imported articles to be in two
columns, then put \ics{twocolumn} in the \env{jmlrpapers}
environment to switch on two column formatting. The effect will be
localised to the end of the environment.

\subsubsection{Cross-Referencing}
\label{sec:bkcrossref}

You can cross-reference other parts of the book using the
standard \cs{label}/\cs{ref} mechanism, but if you want to
reference something within an imported article, you must prefix
the label with the label given when importing the article (that
is, the optional argument to \ics{importpubpaper},
\ics{importpaper} or \cs{importarticle}).  For example, if you
want to reference a section labeled \texttt{sec:results} in the
imported paper \texttt{paper1/paper1.tex}, you would need to do:
\begin{verbatim}
see Section~\ref{paper1/paper1sec:results}
\end{verbatim}
or
\begin{verbatim}
see \sectionref{paper1/paper1sec:results}
\end{verbatim}

In addition to the commands described in \sectionref{sec:crossref},
the \clsfmt{jmlrbook} class also provides the following
cross-referencing commands:

\begin{definition}[\DescribeMacro{\chapterref}]
\cs{chapterref}\marg{label list}
\end{definition}
Reference a chapter or chapters. The argument is a comma-separated
list of labels.

\begin{definition}[\DescribeMacro{\articlepageref}]
\cs{articlepageref}\marg{label}
\end{definition}
This displays the starting page number of the article whose label
is given by \meta{label}. Note that this must a single label, not
a list. For example:
\begin{verbatim}
An interesting article starts on page~\articlepageref{paper1/paper1}
\end{verbatim}

\begin{definition}[\DescribeMacro{\articlepagesref}]
\cs{articlepagesref}\marg{label}
\end{definition}
This displays the page range of the article whose label is
given by \meta{label}. Again, this must be a single label, not a
list. This page range is unrelated to the \meta{pages} argument of
\ics{importpubarticle}.

\begin{definition}[\DescribeMacro{\articletitleref}]
\cs{articletitleref}\marg{label}
\end{definition}
This displays the short title for the article whose label is
given by \meta{label}. Again, this must be a single label, not a 
list.

\begin{definition}[\DescribeMacro{\articleauthorref}]
\cs{articleauthorref}\marg{label}
\end{definition}
This displays the author list for the article whose label is
given by \meta{label}. Again, this must be a single label, not a 
list.

\subsection{Altering the Layout of the Main Title Page}
\label{sec:modifytitle}

\begin{definition}[\DescribeMacro{\titlebody}]
\cs{titlebody}
\end{definition}
The main body of the book's title page is given by the command
\cs{titlebody}. Within the definition of this command, you can
use:
\begin{definition}[\DescribeMacro{\SetTitleElement}]
\cs{SetTitleElement}\marg{element}\marg{pre}\marg{post}
\end{definition}
where \meta{element} can be: \texttt{title}, \texttt{volume},
\texttt{issue}\footnote{The default title page layout doesn't use
\texttt{issue}, but if required it can be set with \ics{issue}\marg{number}},
\texttt{subtitle}, \texttt{logo}, \texttt{team}, \texttt{author},
\texttt{date}, \texttt{productioneditor}. The \meta{pre} and
\meta{post} arguments specify what to do before and after the
element.  Note that \cs{SetTitleElement} does nothing if that
element hasn't been set. For example, if \cs{volume} has been
omitted or \verb|\volume{}| is used, then
\begin{verbatim}
\SetTitleElement{volume}{\mainvolumefont}{\postmainvolume}
\end{verbatim}
will do nothing (so you don't end up with \textbf{Volume :}).

\begin{definition}[\DescribeMacro{\IfTitleElement}]
\cs{IfTitleElement}\marg{element}\marg{true part}\marg{false part}
\end{definition}
This does \meta{true part} if \meta{element} has been set
otherwise it does \meta{false part}. For example, 
\cs{postmainvolume} is defined as:
\begin{verbatim}
\newcommand{\postmainvolume}{%
  \IfTitleElement{subtitle}{}{:}\par\relax
}
\end{verbatim}
This means that it will only print a colon after the volume
number if the subtitle has been set.

The default definition of \cs{titlebody} is:
\begin{verbatim}
\newcommand{\titlebody}{%
  \SetTitleElement{title}{\maintitlefont}{\postmaintitle}%
  \SetTitleElement{volume}{\mainvolumefont}{\postmainvolume}%
  \SetTitleElement{subtitle}{\mainsubtitlefont}{\postmainsubtitle}%
  \SetTitleElement{logo}{\mainlogofont}{\postmainlogo}%
  \SetTitleElement{team}{\mainteamfont}{\postmainteam}%
  \SetTitleElement{author}{\mainauthorfont}{\postmainauthor}%
  \SetTitleElement{productioneditor}{\mainproductioneditorfont}%
    {\postmainproductioneditor}%
}
\end{verbatim}

\subsection{Potential Pitfalls}

The \cls{combine} class and \sty{hyperref} package are
individually both easily broken by packages that change certain
internals and they don't ordinarily work together. The
\clsfmt{jmlrbook} class applies patches to the internal referencing
mechanism to make them work together, but it's a fairly fragile
alliance. Some packages are known to break it, for example
\sty{subfig}, \sty{pdfpages} and \sty{geometry}. This is why the
\clsfmt{jmlr} class checks for known problem packages and generates an
error message to dissuade authors from using them. It's likely that
there are other packages that may cause a problem and, as they are
found, they will be added to the check list. Also, it's possible for
an author to disable the package checking mechanism if they are
determined to use a particular package.

In the event that an article has loaded a problem package, the
editors will have to decide whether to ask the author to change
the article so that it doesn't cause a problem or to make the changes
themselves or to find a way of fudging things to get it to work. It
depends on the level of \LaTeX\ expertise amongst the editors and
the time available.

Another problem that can arise is when different articles use
packages that conflict. For example, one article uses package
\styfmt{foo} and another uses package \styfmt{bar}. Each article compiles
okay as a stand-alone article, but when combined \styfmt{foo} and
\styfmt{bar} conflict. Another problem may occur when articles load the
same package but with conflicting package options.  To reduce the
chance of this occurring, the \clsfmt{jmlr} class loads some commonly
used packages. For example, it loads the \sty{algorithm2e}
package with the \pkgoptfmt{algo2e} and \pkgoptfmt{ruled} options and
provides the \env{algorithm} environment in addition to 
\sty{algorithm2e}'s \env{algorithm2e} environment.

Articles that use different input encodings can also cause a
problem. For example, if one article uses \texttt{utf8} and another
uses \texttt{latin1}. If the authors have directly entered a
diacritic or ligature, such as \'e or \ae, instead of using a
\LaTeX\ command, such as \cs{'e} or \cs{ae}, then
this will cause an error on compiling the book.\footnote{and may also
cause a problem for the editor's text editor.} The choice then is to
either change all non-keyboard characters with the appropriate
\LaTeX\ commands or to use the \cs{inputencoding} command, supplied
by the \sty{inputenc} package, to switch the encoding at the start
of each article.

Authors who use \cs{nonumber} within an \env{equation} environment
can mess up the hyperlinks. Remove \cs{nonumber} and change the
equation environment to \cs{[} \ldots\ \cs{]} (or just make it a
numbered equation).

If the article changes the graphics path using \cs{graphicspath},
\clsfmt{jmlrbook} won't find the graphics if the imported articles
aren't in the same directory as the book.

\subsection{Creating the Book Using \appfmt{makejmlrbook}}
\label{sec:makejmlrbook}

The \app{makejmlrbook} Perl script is designed to make it
easier to produce the print and online versions of the book, as
well as producing an HTML index of all the imported articles with
links to the abstracts and PDFs of individual articles. Note that
for it to work properly, the articles must be imported using
\cs{importarticle}, \cs{importpaper} or \cs{importpubpaper}, and
the imported articles must use the \clsfmt{jmlr} class.

On UNIX style systems, the script can be invoked from a terminal
using:
\begin{prompt}
makejmlrbook \oarg{options} \meta{filename}
\end{prompt}
If that doesn't work, or you aren't using a UNIX style operating
system, the script can be invoked from a terminal or command
prompt using:
\begin{prompt}
perl makejmlrbook \oarg{options} \meta{filename}
\end{prompt}
The mandatory argument \meta{filename} is the name of the master
\TeX\ file containing the book. It must use the \clsfmt{jmlrbook}
class. You may omit the \texttt{.tex} extension. For example, if
the file is called \texttt{proceedings.tex}, you can call 
\app{makejmlrbook} as follows:
\begin{verbatim}
perl makejmlrbook proceedings
\end{verbatim}
This will create the files \texttt{proceedings-print.pdf} (the
print version) and \texttt{proceedings-online.pdf} (the online
version). It will also create a directory (folder) called 
\texttt{html} in which the HTML files and individual article PDFs
will be placed.

The options to \app{makejmlrbook} are as follows:
\begin{description}
\item[--online] Generate the color on-line version (default).
\item[--noonline] Don't generate the color on-line version.
\item[--print] Generate the grayscale print version (default).
\item[--noprint] Don't generate the grayscale print version.
\item[--html] Generate the HTML files and the individual article
PDFs (default).
\item[--nohtml] Don't generate the HTML files and the individual
article PDFs.
\item[--logourl \meta{url}] Make the logo on the HTML index page link
to \meta{url}.
\item[--batchtex] Run \TeX\ in batch mode.
\item[--nobatchtex] Don't run \TeX\ in batch mode (default).
\item[--quieter] Reduce chatter to STDOUT (doesn't eliminate
all messages). This also runs \TeX\ in batch mode.
\item[--noquieter] Don't reduce messages to STDOUT (default).
\item[--version] Display the version number and exit.
\item[--help] List all available options.
\end{description}

There are also some more advanced options, but these haven't been
fully tested:
\begin{description}
\item[--latexapp \meta{name}] Application used to call \LaTeX.
Defaults to \qt{pdflatex}.
\item[--latexopt \meta{string}] Options to pass to \LaTeX.
\item[--format \meta{string}] Output format (defaults to \qt{pdf}).
This may need to be changed if you change the \LaTeX\ application.
\item[--bibtexapp \meta{name}] Application use to process the
bibliography. Defaults to \qt{bibtex}.
\item[--bibtexopt \meta{string}] Options to pass to Bib\TeX.
\end{description}

\StopEventually{\clearpage\phantomsection
  \addcontentsline{toc}{section}{Index}\PrintIndex
}

\end{document}
