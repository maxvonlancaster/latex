%%
%% This is file `djd17nov.tex',
%% generated with the docstrip utility.
%%
%% The original source files were:
%%
%% ledpar.dtx  (with options: `djd17nov')
%% 
%%   Author: Peter Wilson (Herries Press) herries dot press at earthlink dot net
%%   Copyright 2004, 2005 Peter R. Wilson
%% 
%%   This work may be distributed and/or modified under the
%%   conditions of the LaTeX Project Public License, either
%%   version 1.3 of this license or (at your option) any
%%   later version.
%%   The latest version of the license is in
%%      http://www.latex-project.org/lppl.txt
%%   and version 1.3 or later is part of all distributions of
%%   LaTeX version 2003/06/01 or later.
%% 
%%   This work has the LPPL maintenance status "unmaintained".
%% 
%%   This work consists of the files listed in the README file.
%% 

%%% This is djd17nov.tex, a sample critical text edition
%%% written in LaTeX2e with the ledmac and ledpar packages.
%%% (c) 2003--2004 by Dr. Dirk-Jan Dekker,
%%% Radboud University, Nijmegen (The Netherlands)
%%% (PRW) Modified slightly by PRW to fit the ledpar manual

\documentclass[10pt, letterpaper, twoside]{article}
\usepackage[latin,english]{babel}
\usepackage{makeidx}
\usepackage{ledmac,ledpar}
\lineation{section}
\linenummargin{inner}
\sidenotemargin{outer}

\makeindex

\renewcommand{\notenumfont}{\footnotesize}
\newcommand{\notetextfont}{\footnotesize}

\let\Bfootnoterule=\relax
\let\Cfootnoterule=\relax

\addtolength{\skip\Afootins}{1.5mm}

\makeatletter

\renewcommand*{\para@vfootnote}[2]{%
  \insert\csname #1footins\endcsname
  \bgroup
    \notefontsetup
    \interlinepenalty=\interfootnotelinepenalty
    \floatingpenalty=\@MM
    \splittopskip=\ht\strutbox \splitmaxdepth=\dp\strutbox
    \leftskip=\z@skip \rightskip=\z@skip
    \l@dparsefootspec #2\ledplinenumtrue%                  new from here
    \ifnum\@nameuse{previous@#1@number}=\l@dparsedstartline\relax
      \ledplinenumfalse
     \fi
     \ifnum\previous@page=\l@dparsedstartpage\relax
     \else \ledplinenumtrue \fi
     \ifnum\l@dparsedstartline=\l@dparsedendline\relax
     \else \ledplinenumtrue \fi
     \expandafter\xdef\csname previous@#1@number\endcsname{\l@dparsedstartline}%
     \xdef\previous@page{\l@dparsedstartpage}%            to here
     \setbox0=\vbox{\hsize=\maxdimen
       \noindent\csname #1footfmt\endcsname#2}%
      \setbox0=\hbox{\unvxh0}%
      \dp0=0pt
      \ht0=\csname #1footfudgefactor\endcsname\wd0
      \box0
      \penalty0
  \egroup
}

\newcommand*{\previous@A@number}{-1}
\newcommand*{\previous@B@number}{-1}
\newcommand*{\previous@C@number}{-1}
\newcommand*{\previous@page}{-1}

\newcommand{\abb}[1]{#1%
        \let\rbracket\nobrak\relax}
\newcommand{\nobrak}{\textnormal{}}
\newcommand{\morenoexpands}{%
        \let\abb=0%
}

\newcommand{\Aparafootfmt}[3]{%
  \ledsetnormalparstuff
  \scriptsize
  \notenumfont\printlines#1|\enspace
  \notetextfont
  #3\penalty-10\hskip 1em plus 4em minus.4em\relax}

\newcommand{\Bparafootfmt}[3]{%
  \ledsetnormalparstuff
  \scriptsize
  \notenumfont\printlines#1|%
  \ifledplinenum
   \enspace
  \else
   {\hskip 0em plus 0em minus .3em}%
  \fi
  \select@lemmafont#1|#2\rbracket\enskip
  \notetextfont
  #3\penalty-10\hskip 1em plus 4em minus.4em\relax }

\newcommand{\Cparafootfmt}[3]{%
  \ledsetnormalparstuff
  \scriptsize
  \notenumfont\printlines#1|\enspace
  \notetextfont
  #3\penalty-10\hskip 1em plus 4em minus.4em\relax}

\makeatother

\footparagraph{A}
\footparagraph{B}
\footparagraph{C}

\let\Afootfmt=\Aparafootfmt
\let\Bfootfmt=\Bparafootfmt
\let\Cfootfmt=\Cparafootfmt

\renewcommand*{\Rlineflag}{}

\emergencystretch40pt

\author{Guillelmus de Berchen}
\title{Chronicon Geldriae}
\date{}
\hyphenation{archi-epi-sco-po Huns-dis-brug li-be-ra No-vi-ma-gen-si}
\begin{document}
\begin{pages}
\begin{Leftside}
\beginnumbering\pstart
\selectlanguage{latin}
\section{De ecclesia S. Stephani Novimagensi}

\noindent\setline{1}
Nobilis itaque comes Otto\protect\edindex{Otto II of Guelders}
imperio et dominio Novimagensi sibi, ut praefertur, impignoratis
et commissis
\edtext{proinde}{\Bfootnote{primum D}} praeesse cupiens, anno
\textsc{liiii} superius descripto, mense
Iu\edtext{}{\Afootnote{p.\ 227~R}}nio, una cum iudice, scabinis ceterisque
civibus civitatis Novimagensis, pro ipsius et inhabitantium in ea
necessitate,\edtext{}{\Afootnote{p.\ 97~N}} commodo et utilitate,
ut \edtext{ecclesia eius}{\Bfootnote{ecclesia D: eius eius H}} parochialis
\edtext{\abb{extra civitatem}}{\Bfootnote{\textit{om.}~H}} sita
destrueretur et \edtext{infra}{\Bfootnote{intra D}} muros
\edtext{transfer\edtext{}{\Afootnote{p.\ 129~D}}retur}%
{\Bfootnote{transferreretur NH}}
ac de novo construeretur,
\edtext{a reverendo patre domino
Conrado\protect\edindex{Conrad of Hochstaden} de
\edtext{Hofsteden}{\Bfootnote{Hoffstede D: Hoffsteden H}}, archiepiscopo
\edtext{Coloniensi}{\Bfootnote{Colononiensi H}}, licentiam}%
{\Cfootnote{William is confusing two charters that are five years
apart. Permission from St.\ Apostles' Church in Cologne had been
obtained as early as 1249. Cf.\
Sloet\protect\index{Sloet van de Beele, L.A.J.W.},
\textit{Oorkondenboek} nr.\ 707 (14 November 1249):
``\ldots{}nos devotionis tue precibus annuentes, ut ipsam ecclesiam
faciens demoliri transferas in locum alium competentem, tibi
auctoritate presentium indulgemus\ldots''}}, et a venerabilibus
\edtext{dominis}{\Bfootnote{viris H}} decano et capitulo sanctorum
Apostolorum\protect\edindex{St. Apostles' (Cologne)}
\edtext{Coloniensi}{\Bfootnote{Coloniae H}}, ipsius ecclesiae ab
antiquo veris et pacificis patronis, consensum, citra tamen
praeiudicium, damnum aut gravamen \edtext{iurium}{\Bfootnote{virium D}}
et bonorum eorundem, impetravit.
\pend

\pstart
\edtext{Et exinde \edtext{liberum}{\Bfootnote{librum H}}
locum eiusdem civitatis
\edtext{qui}{\Bfootnote{quae D}} dicitur
\edtext{Hundisburg}{\Bfootnote{Hundisburch D: Hundisbrug HMN:
Hunsdisbrug R}}\protect\edindex{Hundisburg},
de praelibati Wilhelmi\protect\edindex{William II of Holland} Romanorum
\edtext{regis}{\Bfootnote{imperatoris D}}, ipsius fundi
do\edtext{}{\Afootnote{f.\ 72v~M}}mini, consensu, ad aedificandum
\edtext{\abb{et consecrandum}}{\Bfootnote{\textit{om.}\ H}}
ecclesi\edtext{}{\Afootnote{p.\ 228~R}}am et coemeterium,
\edtext{eisdem}{\Bfootnote{eiusdem D}} decano et capitulo de expresso
eiusdem civitatis assensu libera contradiderunt voluntate, obligantes
se ipsi \edtext{comes}{\Bfootnote{comites D}} et civitas
\edtext{\abb{dictis}}{\Bfootnote{\textit{om.}\ H}} decano et capitulo,
quod in recompensationem illius areae infra castrum et portam, quae
fuit dos ecclesiae, in qua plebanus habitare solebat---quae
\edtext{tunc}{\Bfootnote{nunc H}} per novum fossatum civitatis est
destructa---aliam aream competentem et ecclesiae novae,
\edtext{ut praefertur, aedificandae}{%
\lemma{\abb{ut\ldots aedificandae}}\Bfootnote{\textit{om.}\ H}} satis
\edtext{contiguam}{\Bfootnote{contiguum M}}, ipsi plebano darent et
assignarent.}{\Cfootnote{Cf.\ Sloet, \textit{Oorkondenboek} nr.\ 762
(June 1254)}} Et desuper
\edtext{\abb{apud}}{\Bfootnote{\textit{om.}\ H}} dictam ecclesiam
sanctorum Apostolorum \edtext{est}{\Bfootnote{et H}}
\edtext{littera}{\Bfootnote{litteram H}} sigillis ipsorum
Ottonis\edtext{}{\Afootnote{p.\ 130~D}} comitis et civitatis
\edtext{Novimagensis}{\Bfootnote{Novimagii D}}
\edtext{sigillata}{\Bfootnote{sigillis communita H}}.
\pend

\pstart
 // One additional line to show synchronization. //
\pend
\endnumbering
\end{Leftside}

\begin{Rightside}
\sidenotemargin{right}\selectlanguage{english}
\beginnumbering
\pstart
\addtocounter{section}{-1}%
\leavevmode\section{St.\ Stephen's Church in Nijmegen}

\noindent\setline{1}%
After the noble count Otto had taken in pledge the power over
Nijmegen,\footnote{In 1247 William II\protect\index{William II of Holland}
(1227--1256) count of Holland needed money to fight his way to
Aachen\protect\index{Aachen} to be crowned King of the Holy Roman
Empire. He gave the town of Nijmegen in pledge to Otto
II\protect\index{Otto II of Guelders} (1229--1271) count of Guelders.}
like I have written above, he wanted to protect the town. So in June
1254\ledsidenote{1254} he and the judge, the sheriffs and other
citizens of Nijmegen obtained permission to demolish the parish
church that lay outside the town walls,\footnote{Since the early
seventh century old St.\ Stephen's church had been located close
to the castle, at today's
Kelfkensbos\protect\index{Kelfkensbos (Nijmegen)} square.
Traces of the church and the presbytery were found during excavations
in 1998--1999.} to move it inside the walls and to rebuild it new.
This operation was necessary and useful both for Otto himself and
for the inhabitants of the town. The reverend father Conrad of
Hochstaden, archbishop of
Cologne,\footnote{Conrad of Hochstaden ({\textdagger} 1261) was
archbishop of Cologne in 1238--1261. Nijmegen belonged to the
archdiocese of Cologne until 1559.} gave his permission. So did the
reverend dean and canons of the chapter of St.\
Apostles'\protect\index{St. Apostles' (Cologne)} in Cologne, who had
long\footnote{They probably became the patrons when the chapter was
established in the early eleventh century. About the church and the
chapter, see Gottfried Stracke\protect\index{Stracke, G.},
\textit{K\"{o}ln:\ St.\ Aposteln}, Stadtspuren -- Denkm\"{a}ler in
K\"{o}ln, vol.\ 19, K\"{o}ln: J.\,P.\ Bachem, 1992.} been the true
and benevolent patrons of the church---but they did not allow Otto
to do anything without their knowledge, nor to infringe their rights,
nor to damage their property.
\pend

\pstart
And so the count and the town voluntarily gave an open space in town
called Hundisburg, which was owned by the aforementioned king William,
to the dean and chapter of St.\ Apostles' in order to build and
consecrate a church and graveyard. King William approved and the
town of Nijmegen explicitly expressed its assent. A new ditch was dug
on property of the church near the castle and the
harbour,\footnote{Nowadays, the exact location of the medieval
ditch---and of two Roman ones---can be seen in the pavement of
Kelfkensbos\protect\index{Kelfkensbos (Nijmegen)} square.} causing
the demolition of the presbytery. In compensation, the count and
citizens committed themselves to giving the parish priest another
suitable space close enough to the new church that was about to be
built. A letter about these transactions, with the seals of count
Otto and the town of Nijmegen, is kept at St.\ Apostles'
church.\footnote{The original letter is lost. A 15th century
transcription of it is kept at the Historisches Archiv der
Stadt K\"{o}ln (HAStK).}
\pend

\pstart
// One additional line to show synchronization. //
\pend
\endnumbering
\end{Rightside}
\Pages
\end{pages}

%%%%%%%%%%%%%%%%%%%%%%%%%%%
\printindex
\end{document}
%%%%%%%%%%%%%%%%%%

\endinput
%%
%% End of file `djd17nov.tex'.
