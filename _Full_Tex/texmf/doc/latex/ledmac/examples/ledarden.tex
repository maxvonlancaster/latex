%%
%% This is file `ledarden.tex',
%% generated with the docstrip utility.
%%
%% The original source files were:
%%
%% ledmac.dtx  (with options: `arden')
%% 
%%   Author: Peter Wilson (Herries Press) herries dot press at earthlink dot net
%%   Copyright 2003 -- 2005 Peter R. Wilson
%% 
%%   This work may be distributed and/or modified under the
%%   conditions of the LaTeX Project Public License, either
%%   version 1.3 of this license or (at your option) any
%%   later version.
%%   The latest version of the license is in
%%      http://www.latex-project.org/lppl.txt
%%   and version 1.3 or later is part of all distributions of
%%   LaTeX version 2003/06/01 or later.
%% 
%%   This work has the LPPL maintenance status "unmaintained".
%% 
%%   This work consists of the files listed in the README file.
%% 
%% ledarden.tex
\documentclass{article}
\usepackage{ledmac}

\makeatletter
 \newcommand{\stage}[1]{\rlap{\hbox to \the\linenumsep{%
                        \hfil\llap{[\textit{#1}]}}}}

 \newcommand{\speaker}[1]{\pstart\hangindent2em\hangafter1
   \leavevmode\textit{#1}\enspace\ignorespaces}

 \newcommand{\exit}[1]{\hfill\stage{#1}}

 % LEDMAC customizations:
 \noendnotes
 \setlength{\parindent}{0pt}
 \setlength{\linenumsep}{.4in}
 \rightskip\linenumsep

 \renewcommand{\interparanoteglue}{1em plus.5em minus.1em}

 \newcommand{\scf}{\tiny}
 \let\Afootnoterule=\relax \let\Bfootnoterule=\relax

 \renewcommand{\rightlinenum}{\numlabfont\llap{\the\line@num}}
 \frenchspacing

 % Footnote formats:
 % \nonumparafootfmt is a footnote format without line numbers.
 \newcommand{\nonumparafootfmt}[3]{%
   \ledsetnormalparstuff
   \rightskip=0pt
   \select@lemmafont#1|#2\rbracket\enskip
   \itshape #3\penalty-10 }

 \newcommand{\newparafootfmt}[3]{%
   \ledsetnormalparstuff
   {\notenumfont\printlines#1|}\fullstop\enspace
   {\select@lemmafont#1|#2}\rbracket\enskip
   \itshape #3\penalty-10 }

 \newcommand{\newtwocolfootfmt}[3]{%
   \normal@pars
   \hsize .48\hsize
   \tolerance=5000
   \rightskip=0pt \leftskip=0pt \parindent=5pt
   \strut\notenumfont\printlines#1|\fullstop\enspace
   \itshape #2\/\rbracket\penalty100\hskip .5em plus .5em
   \normalfont #3\strut\goodbreak}

 % Footnote style selections etc. (done last):
 \footparagraph{A}
 \foottwocol{B}
 \let\Afootfmt=\newparafootfmt
 \let\Bfootfmt=\newtwocolfootfmt
 \let\collation=\Afootnote
 \let\note=\Bfootnote
 \lineation{section}
 \linenummargin{right}
 \makeatother

%%%%%%%%%%%%%%%%%%%%%%%%%%%%%%%%

\begin{document}
 \pagestyle{empty}

 % Initially, we don't want line numbers.
 \let\Afootfmt=\nonumparafootfmt

 \beginnumbering
 \pstart
 \centerline{[\edtext{SCENE III}{
   \lemma{Scene III}
   \collation{Capell; om. Q, F; \textnormal{Scene IV} Pope.}}.---%
   \edtext{\textit{Venice}}{
   \collation{om. Q, F; Shylock's house Theobald; The same.
   A Room in Shylock's House Capell.}}.]}
 \pend
 \bigskip

 \pstart
 \centerline{\textit{Enter} JESSICA \textit{and}
   [\edtext{LAUNCELOT}{
   \lemma{Launcelot}
   \collation{Rowe; om. Q, F.}}] \textit{the clown.}} \pend \bigskip

 \let\Afootfmt=\newparafootfmt % we do want line numbers from now

  \setline{0}%

 \speaker{Jes.}\edtext{I am}{
   \collation{Q, F; \textnormal{I'm} Pope.}}
                       sorry thou wilt leave my father so,\\
 Our house is hell, and thou (a merry devil)\\
 Didst rob it of some taste of tediousness,---\\
 But fare thee well, there is a ducat for thee,\\
 And Launcelot, \edtext{soon}{
   \note{early.}}
                        at supper shalt thou see\\
 Lorenzo, who is thy new master's guest,\\
 Give him this letter,---do it secretly,---\\
 And so farewell: I would not have my father\\
 See me \edtext{in}{
   \collation{Q; om. F.}}
              talk with thee.
 \pend

 \speaker{Laun.}
   \edtext{}{\lemma{\textit{Laun.}}\collation{Q2; Clowne. Q, F.}}%
 \edtext{Adieu!}{
   \collation{\textnormal{Adiew}, Q, F.}}
 tears \edtext{exhibit}{
   \note{Eccles paraphrased ``My tears serve to express what my
   tongue should, if sorrow would permit it,'' but probably it is
   Launce\-lot's blunder for prohibit (Halliwell) or inhibit
   (Clarendon).}}
 my tongue, most beautiful \edtext{pagan}{
   \note{This may have a scurrilous undertone: cf. \textit{2 H 4,}
   {\scf II.} \textrm{ii. 168.}}}%
 , most sweet \edtext{Jew!}{
   \collation{\textnormal{Iewe}, Q, F. \quad \textnormal{do]} Q, F;
              \textnormal{did} F2.}}%
 ---if a Christian \edtext{do}{
   \note{Malone upheld the reading of Qq and F by comparing {\scf II.}
    vi. 23: ``When you shall please to play the thieves for
   wives''; Launcelot seems fond of hinting at what is going to
   happen (cf. {\scf II.} v. 22--3). If F2's ``did'' is accepted,
   \textit{get} is used for beget, as in {\scf III.} v. 9.}}
 not play the knave and get thee, I am much deceived; but \edtext{adieu!}{
   \collation{\textnormal{adiew}, Q, F.}}
 these \edtext{foolish drops do \edtext{something}{
   \collation{Q; \textnormal{somewhat} F.}}
 drown my manly spirit}{
   \lemma{foolish\textnormal{\dots}spirit}
   \note{``tears do not become a man'' (\textit{AYL.}, {\scf III.}
   iv. 3); cf. also \textit{H 5,} {\scf IV.} vi. 28--32.}}%
 : \edtext{adieu!}{
   \collation{\textnormal{adiew}. Q, F. \quad \textnormal{S. D.]} Q2, F; om. Q;
   after l. 15 Capell.}}
 \exit{Exit.}
 \pend

 \speaker{Jes.}
 Farewell good Launcelot.\\
 Alack, what heinous sin is it in me\\
 To be ashamed to be my father's \edtext{child!}{
   \collation{\textnormal{child}, Q, F; \textnormal{Child?} Rowe.}}
 \pend
 \endnumbering

\end{document}

\endinput
%%
%% End of file `ledarden.tex'.
