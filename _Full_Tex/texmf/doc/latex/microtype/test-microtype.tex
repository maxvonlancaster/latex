%%
%% This is file `test-microtype.tex',
%% generated with the docstrip utility.
%%
%% The original source files were:
%%
%% microtype.dtx  (with options: `test')
%% 
%% ------------------------------------------------------------------------
%% 
%%                       The `microtype' package
%%        An interface to the micro-typographic extensions of pdfTeX
%%           Copyright (c) 2004--2010  R Schlicht <w.m.l@gmx.net>
%% 
%% This work may be distributed and/or modified under the conditions of the
%% LaTeX Project Public License, either version 1.3c of this license or (at
%% your option) any later version. The latest version of this license is in:
%% http://www.latex-project.org/lppl.txt, and version 1.3c or later is part
%% of all distributions of LaTeX version 2005/12/01 or later.
%% 
%% This work has the LPPL maintenance status `author-maintained'.
%% 
%% This work consists of the files microtype.dtx and microtype.ins and the
%% derived files microtype.sty, microtype.lua and letterspace.sty.
%% 
%% ------------------------------------------------------------------------
%%   This file might be useful to test protrusion settings for a font.
%%   You can do whatever you want with it.
%% ------------------------------------------------------------------------ 
%%
\documentclass{article}

%% Here you can specify the font you want to test, using
%% the commands \fontfamily, \fontseries and \fontshape.
%% Make sure to end all lines with a comment character!
\newcommand*\TestFont{%
  \fontfamily{ppl}%
%%  \fontseries{b}%
%%  \fontshape{it}% sc, sl
}

\usepackage{ifthen}
\usepackage[T1]{fontenc}
\usepackage[latin1]{inputenc}
\usepackage[verbose,expansion=alltext,stretch=50]{microtype}

\pagestyle{empty}
\setlength{\parindent}{0pt}
\newcommand*\crulefill{\cleaders\hbox{$\mkern-2mu\smash-\mkern-2mu$}\hfill}
\newcommand*\testprotrusion[2][]{%
  \ifthenelse{\equal{#1}{r}}{}{#2}%
  lorem ipsum dolor sit amet,
    \ifthenelse{\equal{#1}{r}}{\crulefill}{\leftarrowfill} #2
    \ifthenelse{\equal{#1}{l}}{\crulefill}{\rightarrowfill}
  you know the rest%
  \ifthenelse{\equal{#1}{l}}{}{#2}%
  \linebreak
  {\fontencoding{\encodingdefault}%
  \fontseries{\seriesdefault}%
  \fontshape{\shapedefault}%
  \selectfont
  Here is the beginning of a line, \dotfill and here is its end}\linebreak
}
\newcommand*\showTestFont{\expandafter\stripprefix\meaning\TestFont}
\def\stripprefix#1>{}
\newcount\charcount
\begin{document}

\microtypesetup{expansion=false}

{\centering The font in this document is called by:\\
 \texttt{\showTestFont}\par}\bigskip

\TestFont\selectfont
 This line intentionally left empty\linebreak
%% A -- Z
\charcount=65
\loop
  \testprotrusion{\char\charcount}
  \advance\charcount 1
  \ifnum\charcount < 91 \repeat
%% a -- z
\charcount=97
\loop
  \testprotrusion{\char\charcount}
  \advance\charcount 1
  \ifnum\charcount < 123 \repeat
%% 0 -- 9
\charcount=48
\loop
  \testprotrusion{\char\charcount}
  \advance\charcount 1
  \ifnum\charcount < 58 \repeat
%%
 \testprotrusion[r]{,}
 \testprotrusion[r]{.}
 \testprotrusion[r]{;}
 \testprotrusion[r]{:}
 \testprotrusion[r]{?}
 \testprotrusion[r]{!}
 \testprotrusion[l]{\textexclamdown}
 \testprotrusion[l]{\textquestiondown}
 \testprotrusion[r]{)}
 \testprotrusion[l]{(}
 \testprotrusion{/}
 \testprotrusion{\char`\\}
 \testprotrusion{-}
 \testprotrusion{\textendash}
 \testprotrusion{\textemdash}
 \testprotrusion{\textquoteleft}
 \testprotrusion{\textquoteright}
 \testprotrusion{\textquotedblleft}
 \testprotrusion{\textquotedblright}
 \testprotrusion{\quotesinglbase}
 \testprotrusion{\quotedblbase}
 \testprotrusion{\guilsinglleft}
 \testprotrusion{\guilsinglright}
 \testprotrusion{\guillemotleft}
 \testprotrusion{\guillemotright}

\newpage
The following displays the current font stretched by 5\%,
normal, and shrunk by 5\%:

\bigskip
\newlength{\MTln}
\newcommand*\teststring
  {ABCDEFGHIJKLMNOPQRSTUVWXYZabcdefghijklmnopqrstuvwxyz0123456789}
\settowidth{\MTln}{\teststring}
\microtypesetup{expansion=true}

\parbox{1.05\MTln}{\teststring\linebreak\\
                   \teststring}\par\bigskip
\parbox{0.95\MTln}{\teststring}

\end{document}
\endinput
%%
%% End of file `test-microtype.tex'.
