% This is a demonstration for a full Mongolian
% document. You can either elatex or pdfelatex it.
%
\documentclass[a4paper,10pt]{article}
\usepackage[bicig]{mls}
\begin{document}
\title{uindur gegen zanabazar}
\date{}
\maketitle

17||18 d'ugar zagun-u munggul-un neiigem, ulus tuiru, shasin-u
uiiles-tu, ilangguy=a uralig-un kuikzil-du uncugui ekurge
kuiicedgeksen uindur gegen zanabazar, cingkis xagan-u aldan
urug-un izagur surbulzidan abadai saiin nuyan xan-u kuiu
tuisiyedu xan gumbudurzi-yin ger-tu 1635 un-du tuiruksen.
badum muingke dayan xagan-u 6-d'aki uiy=e-yin kuimun. gurban
nasudai-d'agan num ungsizu enedkek gazar tuibed kele-yi xar=a
ayandagan surcu, keuked axui cag-aca erdem num-un duiri-tei
bulugsan zanabazar 15 nasu-tai-dagan baragun zuu (lhasa)
uruzu tabudugar dalai lam=a-d'u shabilan saguzu, ulamar
zebCundamba-yin xubilgan tudurazei. uran barimalci, zirugaci,
kele sinzigeci, uran barilgaci, kuin uxagandan zanabazar ulan
zagun zil-un daiin tululdugan-d'u nerbekden suliduzu, zugsunggi
baiidal-d'u urugsan dumdadu zagun-u munggul-un suyul uralig-i
serkun manduxu-d'u yeke xubi nemekuri urugulugsan yum. tekun-u
abiyas bilig nuiri yeke kuidelmuri-ber munggul-un uralig nigen
uiy=e tanigdasi uigei uindurlik-tu kuiruksen azei. xarin 1654
un-d'u neiislel kuiriyen-u tulg=a-yin cilagu-yi tabilcagsan
zanabazar-un uran barilg=a-yin buidugel-ece uinudur-i uizeksen
zuiil barug uigei ni xaramsaldai. zanabazar uindesun-u bicig
uisuk-i kuikzikulku-d'u beyecilen urulcazu, suyungbu uisuk-i
zukiyazu ene uiy=e suyungbu ni man-u tusagar tugdanil-un belge
temdek bulugsagar baiin=a.  tere-ber <<cag-i tukinagulugci>>
gedek silukleksen zukiyal-d'agan arad tuimen-u-ben engke
amugulang, saiin saiixan-i imagda kuisen muirugedezu yabudag
sedkil-un-iien uige-i ilerkeiileksen baiidag. uindur gegen
duirsuleku uralig-un xubi-d'u uirun=e-yin sunggudag-ud-tai
eng zergeceku buidugel-tei kuimun abacu basa xari ulus-un
buzar bacir arg=a-d'u abdagdan yabugsan nigen.

munggul-d'u urcigulxu uxagan yeke delgerezu baiigsan ni man-u
erden ba dumdadu uiy=e-yin suyul-un nigen uncalig azei. erden-u
enedkek-un kuin uxagan-u iragu naiirag, kele bicik-un sudulul,
anagaxu uxagan, uralaxu uxagan zerge tabun uxagan-u zukiyal-i
bagdagagsan buikude 334 budi <<ganzuur>>, <<danzuur>>-i num-un
mergen bagsi kuinggaudsar terikudei 64 erdemden lama urcigulun
neiideleksen baiin=a. 400 zil-un terdege urcigulg=a-yin iimu
eke kuiriyeleng munggul-d'u azillazu baiigsan-i tuisugelen
buduxu-d'u baxadai. munggulcud erden-ece  inagsi daguu
xugur-tai buizik nagadum-tai xurdun kuiluk murid-tai. er=e-yin
gurban nagadum-i erkimelen kuikzilduzu, ide xabu-ban
bulgazu ireksen baiin=a.  munggul-un zirgalang ni buizik,
xurim bile. xudala-i  xagan-d'u erkumzileged xurxunag-un
sagalagar mudun-u duur=a xabirg=a gazar-i xalcaradal=a, ebuduk
gazar-i uilduredel=e debkecen buiziklezu xurimlaba gesen uige
<<niguca tubciyan>>-d'u bui. munggul arad-un medelge uxagan
erde-ece inagsi mal azu axui, udun urun, gazar zuii, anagaxu
uxagan, baiigali, neiigem-un ulan salburi-bar kuikzizu irebe.
<<aldan tubci>>, erdeni-yin tubci>>, <<bulur tuli>>, <<subud
erike>> medu teuke-yin ulan arban zukiyal gargazei.

manzu nar <<munggul uyun>>-i muikugeku-yi kedui-ber uruldubacu
uyun bilikdu, cecen celmek, erdem uxagandan tuduran garugsagar
baiiba.  19-d'uger zagun bul iragu naiiragci dangzirabzai. yeke
zukiyalci inzinasi dangzigvangzil nar-un amidurazu, buidugezu
baiigsan uiy=e bile.
\end{document}
