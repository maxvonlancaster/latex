%%
%% This is file `morefloats-example.tex',
%% generated with the docstrip utility.
%%
%% The original source files were:
%%
%% morefloats.dtx  (with options: `example')
%% 
%% This is a generated file.
%% 
%% IMPORTANT NOTICE:
%% The usual disclaimers apply:
%% If it doesn't work right that's your problem.
%% (Nevertheless, send an e-mail to the maintainer
%%  when you find an error in this package.)
%% 
%% This work may be distributed and/or modified under the
%% conditions of the LaTeX Project Public License, either
%% version 1.3c of this license or (at your option) any later
%% version. This version of this license is in
%%    http://www.latex-project.org/lppl/lppl-1-3c.txt
%% and the latest version of this license is in
%%    http://www.latex-project.org/lppl.txt
%% and version 1.3c or later is part of all distributions of
%% LaTeX version 2005/12/01 or later.
%% 
%% This work has the LPPL maintenance status "maintained".
%% 
%% The Current Maintainer of this work is H.-Martin Muench
%% (Martin dot Muench at Uni-Bonn dot de).
%% 
%% The main code of this package was invented by
%% Don Hosek, Quixote 1990/07/27 (Thanks!).
%% maintenance has been taken over in September 2010 by H.-Martin M{\accent "7F u}nch.
%% 
%% This work consists of the main source file morefloats.dtx
%% and the derived files
%%    morefloats.sty, morefloats.ins, morefloats.drv,
%%    morefloats-example.tex, morefloats.pdf.
%% 

\documentclass[british]{article}
%%%%%%%%%%%%%%%%%%%%%%%%%%%%%%%%%%%%%%%%%%%%%%%%%%%%%%%%%%%%%%%%%%%%%
\usepackage[maxfloats=19]{morefloats}
\gdef\unit#1{\mathord{\thinspace\mathrm{#1}}}%
\listfiles
\begin{document}

\section*{Example for morefloats}
\markboth{Example for morefloats}{Example for morefloats}

This example demonstrates the use of package\newline
\textsf{morefloats}, v1.0c as of 2010/09/20 (HMM; DH).\newline
The package takes options (here: maxfloats=19 is used).\newline
For more details please see the documentation!\newline

To reproduce the\newline
\texttt{\LaTeX\ Error: Too many unprocessed floats},\newline
comment out the \texttt{\textbackslash usepackage...} in the preamble (line~3)
(by placing a \% before it).\newline

\bigskip

Save per page about $200\unit{ml}$~water, $2\unit{g}$~CO$_{2}$
and $2\unit{g}$~wood:\newline
Therefore please print only if this is really necessary.\newline
I do NOT think, that it is necessary to print THIS file, really!

\pagebreak

Here are a lot of floating tables:\newline

\begin{table}[t] \centering%
\begin{tabular}{|l|}
\hline
A table, which will keep floating.\\ \hline
\end{tabular}%
\caption{The first Table}%
\end{table}%

\begin{table}[t] \centering%
\begin{tabular}{|l|}
\hline
A table, which will keep floating.\\ \hline
\end{tabular}%
\caption{The second Table}%
\end{table}%

\begin{table}[t] \centering%
\begin{tabular}{|l|}
\hline
A table, which will keep floating.\\ \hline
\end{tabular}%
\caption{The third Table}%
\end{table}%

\begin{table}[t] \centering%
\begin{tabular}{|l|}
\hline
A table, which will keep floating.\\ \hline
\end{tabular}%
\caption{The fourth Table}%
\end{table}%

\begin{table}[t] \centering%
\begin{tabular}{|l|}
\hline
A table, which will keep floating.\\ \hline
\end{tabular}%
\caption{The fifth Table}%
\end{table}%

\begin{table}[t] \centering%
\begin{tabular}{|l|}
\hline
A table, which will keep floating.\\ \hline
\end{tabular}%
\caption{The sixth Table}%
\end{table}%

\begin{table}[t] \centering%
\begin{tabular}{|l|}
\hline
A table, which will keep floating.\\ \hline
\end{tabular}%
\caption{The seventh Table}%
\end{table}%

\begin{table}[t] \centering%
\begin{tabular}{|l|}
\hline
A table, which will keep floating.\\ \hline
\end{tabular}%
\caption{The eighth Table}%
\end{table}%

\begin{table}[t] \centering%
\begin{tabular}{|l|}
\hline
A table, which will keep floating.\\ \hline
\end{tabular}%
\caption{The ninth Table}%
\end{table}%

\begin{table}[t] \centering%
\begin{tabular}{|l|}
\hline
A table, which will keep floating.\\ \hline
\end{tabular}%
\caption{The tenth Table}%
\end{table}%

\begin{table}[t] \centering%
\begin{tabular}{|l|}
\hline
A table, which will keep floating.\\ \hline
\end{tabular}%
\caption{The eleventh Table}%
\end{table}%

\begin{table}[t] \centering%
\begin{tabular}{|l|}
\hline
A table, which will keep floating.\\ \hline
\end{tabular}%
\caption{The twelfth Table}%
\end{table}%

\begin{table}[t] \centering%
\begin{tabular}{|l|}
\hline
A table, which will keep floating.\\ \hline
\end{tabular}%
\caption{The thirteenth Table}%
\end{table}%

\begin{table}[t] \centering%
\begin{tabular}{|l|}
\hline
A table, which will keep floating.\\ \hline
\end{tabular}%
\caption{The forteenth Table}%
\end{table}%

\begin{table}[t] \centering%
\begin{tabular}{|l|}
\hline
A table, which will keep floating.\\ \hline
\end{tabular}%
\caption{The fifteenth Table}%
\end{table}%

\begin{table}[t] \centering%
\begin{tabular}{|l|}
\hline
A table, which will keep floating.\\ \hline
\end{tabular}%
\caption{The sixteenth Table}%
\end{table}%

\begin{table}[t] \centering%
\begin{tabular}{|l|}
\hline
A table, which will keep floating.\\ \hline
\end{tabular}%
\caption{The seventeenth Table}%
\end{table}%

\begin{table}[t] \centering%
\begin{tabular}{|l|}
\hline
A table, which will keep floating.\\ \hline
\end{tabular}%
\caption{The eighteenth Table}%
\end{table}%

\begin{table}[t] \centering%
\begin{tabular}{|l|}
\hline
One floating table too much
(without \textsf{morefloats} and appropriate option(s)).\\ \hline
\end{tabular}%
\caption{The nineteenth Table}%
\end{table}%

\end{document}
\endinput
%%
%% End of file `morefloats-example.tex'.
