\typeout{niceverb.tex 2010/04/05 documenting niceverb.sty}
\title{\textsf{niceverb.sty}\\---\\Minimizing 
  Markup\\for Documenting \LaTeX\ packages%%% \thanks{This 
%     manual describes package version
%     version 0.2 as of April 09, 2009%%%\fileversion\ as of \filedate\ 
%     .}}%%%of the package.}%
}
% \listfiles 2010/03/19
{ \RequirePackage{makedoc} \ProcessLineMessage{} %% 2010/03/11
  \MakeJobDoc{19}{\SectionLevelThreeParseInput}  }
\documentclass{article}%% TODO paper dimensions!?
\typeout{makedoc.tex 2010/03/30 documentation for `makedoc.sty'}
\listfiles
\RequirePackage{makedoc}
\documentclass{article}
\typeout{makedoc.tex 2010/03/30 documentation for `makedoc.sty'}
\listfiles
\RequirePackage{makedoc}
\documentclass{article}
\typeout{makedoc.tex 2010/03/30 documentation for `makedoc.sty'}
\listfiles
\RequirePackage{makedoc}
\documentclass{article}
\input{makedoc.cfg} %% with pdf stuff and 'niceverb'
%% removed for niceverb v0.31 TODO!? 2010/03/20:
\sfcode`/=1001 %% TODO makedoc.cfg!? 2010/03/21
% \makeatletter %% TEST for hyperref compatibility 2010/03/11
%   \def\@testdef #1#2#3{%
%     \def\reserved@a{#3}%
%     \expandafter \ifx \csname #1@#2\endcsname
%    \reserved@a  \else \@tempswatrue \fi
%     \if@tempswa
%       \typeout{^^J*** Type `r' <input> to get around 
%                       \string\label\space issues! ***^^J}
%       \errorcontextlines=0
%       \show\reserved@a
%       \expandafter \show \csname #1@#2\endcsname
%     \fi
%    }
% \makeatother
\begin{document}
\title{'makedoc'---Preprocessing documentation by \TeX}
  %% 2009/04/10: \\---\\\ breaks TOC
\maketitle
\begin{abstract}\noindent
'makedoc' provides commands for generating \LaTeX\ input from a 
package file in order to typeset the latter's documentation 
(somewhat similar and opposite to 'docstrip')---with 
v0.3 \emph{a single one usually suffices}. 
Certain comment marks are removed, listing commands are inserted, 
and some (configurable) typographical `txt'$\to$\TeX\ corrections 
are applied.---This 
continues the policy of 'niceverb' to minimize documentation markup in 
package files. 'makedoc' extends and exemplifies the parsing package 
'fifinddo'. After an edit (and test) of your package, you get the new 
documentation in one run (or the usual number of runs) of the 
documentation driver file.---The present approach is meant to be an 
\emph{alternative} to the standard 'doc' package and its `\DocInput'. 
It provides \emph{less} than 'doc' does, rather deliberately. It may 
be helpful at least for the development of small packages, or at least 
at early stages.
\end{abstract}
\tableofcontents
\section{Introduction}
\emph{The abstract will not be repeated in this section.} Let me add 
instead that I was in dire need of such a package, I got stuck with my 
packages because I lost orientation in them, and I was unhappy with 
the forms of documentations of my other packages, and documenting them 
with the standard \LaTeX\ 'doc' system was not attractive for me 
(neither considered helpful). %% clarified 2010/03/13
I also worked on \emph{Windows} until September 2008, and I 
find a system like the present one still more attractive then using 
(learning!\@) other filtering utilities (see below on 'awk'). And I 
may work on \emph{Windows} once again and don't want to depend on 
installing some $\dots$ there---\emph{I really would like to have 
powerful tools for everything depending on nothing but \TeX\slash 
\LaTeX!}

\section{Prior work and what is new}
It is, of course, not a new idea to get around comment marks `%' to 
typeset the documentation. 'doc''s `\DocInput' does this by making `%' 
an ``ignored" character. This way you cannot use `%' for commenting 
comments (so 'doc' offers a ``new comment mark" 
`^'`^'`A'). %% TODO `^^A' suddenly failed 2010/03/15 -- "ligature"!?
You also cannot use `%' for commenting out code (that you are 
pondering---or using for debugging---only). %% clarified 2010/03/13

Moreover, 'doc' requires enclosing package code explicitly by 
environment commands (behind comment marks). Stephan I. B\"ottcher 
with his '\href{http://ctan.org/pkg/lineno}{lineno.sty}' 
and Grzegorz Murzynowski in \ctanpkgref{gmdoc}
aimed at doing away with this requirement. 
'lineno.sty' contains 'awk' scripts 
to remove starting comment marks and to insert listing commands. 
A file 'lineno.tex' is generated that typesets the documentation. 
By the way, 'lineno.sty' is full of discussions, but it is not 
'docstrip'ped---the maintainers never have received a complaint 
that inputting 'lineno.sty' were too slow. 

'gmdoc' seems to get around comment marks and insert listing commands 
\emph{while typesetting} by a refined version of `\DocInput', 
through some careful detecting and analysing comment marks, 
the approach resembles detection of lists in 'wiki.sty'.\footnote{See 
  'gmdoc.pdf' on &\DocInput. You can learn a lot from this 220 pages 
  document! I also find 
  \ctanpkgref{pauldoc} and \ctanpkgref{xdoc} inspiring.}
And this is a matter of principles---comparing the approaches of 
\emph{preprocessing} ('lineno.sty') and \emph{``smart typesetting"} 
('gmdoc', 'wiki'). Sometimes preprocessing seems to be simpler, 
sometimes detecting while typesetting. 
(Another example is the preprocessor 
\ctanpkgref{easylatex}
of which 'wiki.sty' is a much reduced ``while typesetting" variant.) 
``While typesetting" may be easier when single characters or 
sequences of two or three encode markup
information---but such detection can badly interfere with other 
packages etc. ``Preprocessing" may be easier when entire ``strings" 
of characters decide, which may be anywhere in a file line. 

'makedoc' chooses \emph{preprocessing}, as 'lineno.sty', but by 
\emph{\TeX}. There is a general discussion of this choice in the 
documentation of 'fifinddo'. Preprocessing here can be done in the 
same \LaTeX\ run as typesetting, though you can avoid 
incompatabilities with packages needed for typesetting 
(by inputting them only \emph{after} preprocessing). 

'lineno.sty' exemplifies why preprocessing with \emph{\TeX} may be 
preferable to preprocessing with other utilities: 
When I took over maintenance of 'lineno.sty', 
I needed hard work to get the 'awk' script running. 
The \emph{Munich} 'awk' seemed \emph{not} to behave as the \emph{Kiel} 
'awk' (I chose a Munich 'nawk' and reworked the script a little). 
\TeX\ seems to have better fixed functionality than other utilities! 

A different alternative to \LaTeX's 'doc' system is 
Paul Isambert's '\href{http://ctan.org/pkg/codedoc}{CodeDoc}' 
where the code environments extract package code in typesetting the 
documentation. %% added 2010/03/10

\section{Styles supported (parsers provided)}\label{sec:styles}
% \section{Styles of commenting '.sty's}
We find different styles of documenting \LaTeX\ packages. 
As the main aspects I consider 
(i)~\emph{telling code from comments} 
and (ii)~\emph{markup in comments}. 
(You may find more details on the next matters in the 
 ``implementation" section.)

\subsection{Telling code from comments}
\emph{Comment marks} (usually \lq`%'\rq\ in the case of \TeX) 
probably were named so to mark \emph{``comments"} as opposed 
to code $\dots$ great, but actually, in ``daily practice," 
they are so handy---and used---for ``commenting out" \emph{code}, 
i.e., \emph{managing code versions} in a simple way: 
one does not actually want to \emph{delete} code, 
one might want to use it another time, maybe for debugging
$\dots$ or to remind of earlier attempts that should not be tried 
again $\dots$

This is a problem for \emph{high-quality typesetting} of 
documentation. \emph{Code} should be typeset about as you see it on 
the \emph{screen}---\emph{monospaced}, this allows structuring by 
indenting, it is common practice to use a typewriter typeface for 
this. Real \emph{comments} should be typeset in \emph{high quality} as 
usual with \LaTeX. Little dilemmas therefore occur with \emph{``hidden 
code"} (``commented-out"). A comment mark starts the line, but 
obviously it is not really a comment and rather should be typeset 
like code (and otherwise they may break).           %% 2010/03/22
Another problem are comments at the \emph{end} of a 
\emph{code} line. Sometimes they are ``real comments" ('gmdoc' 
supports this style). But sometimes 
this is only another version of ``version management," code 
``commented-out."

I like the style of writing packages described before and use it all 
the time. I mark ``real comments" with \emph{two} adjacent comment 
marks and an ensuing space to distinguish them clearly from code 
commented out.
%% Adapted to v0.4 2010/03/29:
\emph{This style is presently the one supported by \textup{'makedoc'} 
      as default.}
This way only a line starting with 
|%% | is considered a ``real" comment line. The first three 
characters are removed, and the rest is typeset in high quality. 
Any other lines are typeset verbatim. The 'makedoc' \emph{parser} 
doing this has an ``identifier" |PPScomment| (``percent, percent, 
space"). Another identifier |comment| is a placeholder for 
the comment parser to be used, by default it is an alias for 
`PPScomment'. Lines just containing |%%| (without the space) may be 
used to suppress empty code lines preceding section titles and for 
keeping some visual, relieving space between code and comment lines.

The style I described previously may be considered ``unprofessional." 
The many \LaTeX\ packages documented using the 'doc'\slash'.dtx' 
system don't use comment marks for \emph{``commenting-out"}. 
Or one may mark code commented out by putting no space between the 
percent mark and the code. 
With v0.4 of 'makedoc', this style is supported as |PScomment|. 
You can directly call this as <main-parser> as described below, 
or you can switch to it by 
\[`\CopyFDconditionFromTo{PScomment}{comment}'\]

\subsection{Markup in comments}
Packages using the 'doc'\slash '.dtx' system as well as alternative 
highly developed systems mentioned above use (enhanced) usual 
\emph{\LaTeX} syntax for markup of comments. Other packages just use 
an \emph{ASCII} style \emph{without} any markup. My idea was to 
support the latter style by some `txt'$\to$\LaTeX\ functionality. 
'makedoc' does this using a file 'mdoccorr.cfg' which is very small 
right now.

I also thought of introducing another sort of ``decent" markup not 
needing much more space than the ``ASCII kernel" of the comments. 
This is to some extent implemented in 'niceverb.sty'. I thought of the 
syntax of editing \textit{Wikipedia} pages; this is partially 
implemented in 'wiki.sty' which unfortunately is not yet compatible 
with 'niceverb'.

But 'makedoc' implements one \textit{Wikipedia} feature in a different 
way than 'wiki.sty' (cf.~'wikicheat.pdf') that looks about as follows:
\begin{eqnarray*}
  \endcell\endcell`%% == Section =='\\
  \endcell\endcell`%% === Subsection ==='\\
  \endcell\endcell`%% ==== Subsubsection ===='
\end{eqnarray*}
i.e., you type `== <title> ==' in place of `\section{<title>}' etc.
The parser must replace `====<title3>===' before `===<title2>===' and 
the latter before `==<title1>=='. In fact, 'makedoc' provides three 
parsers for these situations:
\begin{description}
\cmdboxitem|\SectionLevelThreeParseInput| is the most general parser 
    offered. If it does not find two strings \lq`===='\rq\ enclosing 
    \emph{something}, it passes to
\cmdboxitem|\SectionLevelTwoParseInput| which unless finding 
    two strings `===' enclosing something passes to
\cmdboxitem|\SectionLevelOneParseInput| $\dots$ passes to the comment 
    detector |comment|. 
\end{description}


\section{Requirements}
'makedoc' requires \LaTeXe\ (supporting star forms of `\newcommand' 
etc.)\ as \TeX-format, the package 'fifinddo.sty' from the same 
directory (on CTAN etc.)\ as where 'makedoc.sty' is, and the 
\LaTeX-package 'moreverb' by Robin Fairbairns (after others)---it 
should be installed anyway, or you can get its latest version 
(v2.3, 2008/06/03?) from CTAN. 

'makedoc''s `.txt'$\to$\TeX\ functionality moreover needs a file 
'mdoccorr.cfg' that should have come along with 'makedoc.sty' and 
'fifinddo.sty'. You may need to have a modified copy of it in the 
directory of your main `.tex' file `<jobname>.tex' fitting special 
needs of your project. 

\section{Using 'makedoc' the simplest way}
In the most simple case, you are preparing documentation for a package 
file `<jobname>.sty' only, and you prepare a file `<jobname>.tex' 
containing 
\[`\title{\textsf{<jobname>}---a \LaTeX\ Package for <whatever>}'\]
and `\maketitle' etc.\ about your package.\footnote{With 'niceverb' 
    and &\title\ after &\begin{document}, you may replace 
    \lq&\textsf{<jobname>}\rq\ by \lq&'<jobname>&'\rq.}
The documentation will be produced by running `<jobname>.tex' with 
\LaTeX\ (e.g., \texttt{latex <jobname>.tex}).

First, `<jobname>.tex' must have |\usepackage{makedoc}| in its preamble. 
There are no package options. 

Second, to typeset the commented implementation from `<jobname>.sty', 
include in <jobname>.tex's `document' environment a line 
\[|\MakeInputJobDoc{<header-lines>}{\SectionLevelThreeParseInput}|\]
<header-lines> refers to a non-negative integer as follows: 
We think the most simple and useful way of typesetting the first lines 
of a package file including license and copyrights is ``depicting them 
as image," i.e., \textit{verbatim}. We could try to determine the 
number of these lines by parsing, but we won't do so soon. Please just 
count them and enter the number as <header-lines>---and change it 
until you can accept the outcome.

\section{Steps of advanced usage}
\subsection{Different main parsers (second mandatory argument)}

`\MakeInputJobDoc''s mandatory syntax actually is 
\[|\MakeInputJobDoc{<header-lines>}{<main-parser>}|\]
<main-parser> refers to the parsing macro that is applied to each 
input line whose number is greater than <header-lines>. 
Examples for <main-parser> are named in section~\ref{sec:styles} above. 
  %% TODO above/below macro 2010/03/15
`\SectionLevelThreeParseInput' is just the most general one. 
For \emph{efficiency} (!? or also to avoid problems?) you may 
replace `Three' by `Two' or by `One', if the `====' or the `===' 
feature is not used in `<jobname>.sty'. If the ``\textit{Wikipedia} 
sectioning" feature is not used at all, use 
\[|\MakeInputJobDoc{<header-lines>}{\ProcessInputWith{comment}}|\]
---provided you want to adopt the \lq`%% '\rq\ style of marking 
comments, cf.~section~\ref{sec:styles}. For the \lq`% '\rq style 
instead, use 
\[|\MakeInputJobDoc{<header-lines>}{\ProcessInputWith{PScomment}}|\]

\subsection{Different extensions (optional arguments)}
If your package to be documented is a \emph{class} `<jobname>.cls', 
a local configuration file `<jobname>.cfg' or something 
else---<jobname>.<ext-in>, e.g., <ext-in>=`cls' or <ext-in>=`cfg', 
use 
\[|\MakeInputJobDoc[<ext-in>]{<header>}{<parser>}|\]
Moreover, `\MakeInputJobDoc' writes an intermediate file 
`<jobname>.doc' and then `\input's it. If you do not like `doc' 
as extension for the written file name (maybe you use 
`<jobname>.doc' for something different already), preferring extension 
<ext-out>, use
\[|\MakeInputJobDoc[<ext-in>][<ext-out>]{<header>}{<parser>}|\]
Yes, you must state <ext-in> then as well, I can't help $\dots$

If even <jobname> is wrong in your view, see next step $\dots$

\subsection{Commands modifying &\MakeInputJobDoc's behaviour}
\label{sec:modimake}
Already <jobname> may not be what you want. E.g., you may want 
to collect documentations of some other files <job-1>, <job-2>, 
$\dots$ in a single <jobname>. Then precede `\MakeInputJobDoc'
with 
\[`\renewcommand*{\mdJobName}{<job-1>}'\]
etc.\ (please reason yourself about additional requirements \dots)
As a matter of fact, `\MakeInputJobDoc' reads 
\[`\mdJobName.<ext-in>' \mbox{\quad and writes\quad} 
  `\mdJobName.<ext-out>'\]
Stated another way, <jobname> above referred to |\mdJobName|.

`\MakeInputJobInput' moreover (by default) produces one dot 
per input line processed on screen to show progress. 
The reason is that `makedoc' issues the command 
|\ProcessLineMessage{\message{.}}|.
Already this trivial thing seems to slow down processing considerably 
(nowadays). `\MakeInputJobInput' will run faster if preceded by 
\[|\ProcessLineMessage{}|\]
which will suppress any message about processing.
However, the message may be helpful in trouble-shooting.

\subsection{Separating preprocessing from typesetting}
  %% extended 2010/03/16
To some surprise, I observe that `\MakeInputJobDoc' \emph{works.} 
This is quite a new discovery of mine (2010/03/13); 
before I thought that, for safety, preprocessing should happen 
inside a local group \emph{preceding} `\documentclass'.
|\MakeJobDoc| works like `\MakeInputJobDoc' described above, 
yet it just \emph{preprocesses} the package to be documented, 
waiting for an 
\[`\input{<jobname>.<ext-out>}'\] 
in the `document' environment to \emph{typeset} the documentation. 
So 'makedoc.tex' once had in its preamble
% \[`{\RequirePackage{makedoc} \MakeJobDoc{<header>}{<parser>}}'\]
% at the top of `<jobname>.tex' and `\input{<jobname>.<out-ext>}' later. 
\begin{eqnarray*}\endcell\endcell
  `{\RequirePackage{makedoc}'\cr    \endcell\endcell
  ` \ProcessLineMessage{}'\cr       \endcell\endcell
  ` \MakeJobDoc{22}{\ProcessInputWith{comment}}}'\cr
                                    \endcell\endcell
  `\documentclass{article}'
\end{eqnarray*} 
I did experience some truth in my earlier safety policy: 
With 'niceverb' ``running," `\MakeJobDoc' cannot (easily) be used 
in the `document' environment. `\MakeInputJobDoc' in fact does some 
'niceverb' switching (provided 'niceverb' has been loaded) 
when making use of `\MakeJobDoc'.
  %% <- verbose to improve line breaks 2010/03/16

Thinking of this ``safety" approach, especially grouping (`{\code}'), 
I had not much cared for compatibility with other packages 
in choosing 'makedoc' macro names. 

\subsection{Other 'makedoc' (and 'fifindo') script commands}
The next script commands may be considered of a lower level than 
`\MakeJobDoc' and `\MakeInputJobDoc', they underlie the latter 
commands. We also list commands from 'fifinddo.sty' that may be useful 
or, indeed, are needed for preparing package documentations. 
This may result in ideas on how do use the script commands for 
different purposes than for preparing package documentations---e.g., 
apply `txt'$\to$\TeX\ preprocessing to arbitrary text files. 

\subsubsection{Choosing parameter values for next preprocessing run}

This actually continues section~\ref{sec:modimake}.

\begin{description}
\cmdboxitem|\ResultFile{<output>}| (from 'fifinddo') 
    determines (and opens through the \TeX\ primitive `\openout') 
    the file <output> which will contain the result of 
    preprocessing the package file.
\cmdboxitem|\LaTeXresultFile{<output>}|---see next section.
\cmdboxitem|\Headerlines{<number>}| determines the <number> of lines 
    starting the input file to be copied \emph{verbatim} 
    (the first mandatory argument of `\MakeJobDoc'). 
\cmdboxitem|\MainDocParser{<parser>}| determines the <parser> 
    as in the \emph{second} mandatory argument of `\MakeJobDoc'.
\end{description}

We are now describing some parameters which rather must be switched 
``manually" instead of being modifiable by comfortable 'makedoc' 
script commands.

With the \emph{``Wikipedia sectioning"} feature, you may change the outcome 
regarding levels. Assume, e.g., the package file has titles along the 
scheme `== <title> ==' only, but these should become 
\emph{subsections} of the ``implementation" section of the 
corresponding `.tex' file. Then 
\[`\renewcommand*{\mdSectionLevelOne}{\string\subsection}'\]
-- see the implementation of the sectioning feature for details. 

There is a command 
\[|\NoEmptyInputLines| \mbox{\quad and a parameter macro\quad}
  |\OnEmptyInputLine|\] 
which is modified by the former. However, I cannot say much about them 
right now, I think they just were a difficult matter that I soon 
decided no longer to think about for a while (cf.\ the 
implementation). About the same holds for the hook |\EveryComment|.

The `txt'$\to$\TeX\ functionality is implemented through a hook 
\[|\MakeDocCorrectHook{<characters>}|\] 
'makedoc' initializes it as an alias of \LaTeX's `\@firstofone', i.e., 
it won't change <characters>. 'mdoccorr.cfg' should redefine it so it 
really ``corrects" <characters>. You might try other definitions of 
`\MakeDocCorrectHook' for different ``correcting" functions.
It should be \emph{noted} that (currently) 
`\MakeDocCorrectHook' must be \emph{expandable}, 'fifinddo.sty' 
provides setup for (expandable) chains of expandable replacements. 
The ``Wikipedia" sectioning feature moreover uses expandable 
trimming (single) surrounding spaces, this might be provided in a more 
general way.\footnote{%% TODO 2010/03/16
    The \ctanpkgref{trimspaces} package 
    has been a \emph{model} for this feature here. It cannot be used 
    directly here because it reads blank spaces as \TeX\ characters with 
    category code 10 while 'makedoc' reads blank spaces as ``other" 
    characters (category code 12) in order to \emph{keep} all blank spaces.}

\subsubsection{``Manual" insertions to the output file}
\begin{description}
\cmdboxitem|\WriteResult{<balanced>}| (from 'fifinddo') writes 
    <balanced> to <output> according to the earlier command 
    `\ResultFile{<output>}'.
\cmdboxitem|\WriteProvides| (from 'fifindo') writes a 
    `\ProvidesFile' line into <output> that declares the file 
    to be generated by 'fifindo'.
\cmdboxitem|\LaTeXresultFile{<output>}| issues 
    `\ResultFile{<output>}' and then writes a 
    `\ProvidesFile' line into <output> that declares the file 
    to be generated by 'makedoc'.
\end{description}

\subsubsection{Processing input and closing output}
\begin{description}
\cmdboxitem|\MakeDoc{<input>}|\hskip 0pt plus 4em
    reads 'mdoccorr.cfg' 
    (for `\MakeDocCorrectHook', see above),
    %% removed \LaTeXresult... 2010/03/17
    copies <number> according to `\HeaderLines' (see above) 
    from <input> into <output> (according to `\ResultFile'), 
    then processes the remaining lines of <input> according 
    to `\MainDocParser' (writing several things to <output>). 
    `\MakeDoc' invokes{\sloppy\par}
\cmdboxitem|\ProcessFileWith{<input>}{<loop-body>}| 
    (from 'fifindo') reads <input> line by line---each one stored as 
    macro |\fdInputLine| and applies <loop-body> to it. 
    \TeX's ``special" character codes (of characters listed in 
    macro `\dospecials') are replaced by 12 (``other") by 
    default---see the 'fifinddo' documentation.
\cmdboxitem|\CloseResultFile| (from 'fifinddo') 
    \hskip 0pt plus 1em
    \emph{closes} 
    <output> (using \TeX's primitive `\closeout'). 
\cmdboxitem|\MakeCloseDoc{<input>}| issues 
    `\MakeDoc{<input>}\CloseResultFile'.
\end{description}
%
Using `\MakeDoc' \emph{instead} of `\MakeCloseDoc' allows processing 
additional <input> files writing into the same <output>. Or maybe you 
want to add some additional lines manually to <output> using 
`\WriteResult'.

%% removed 2010/03/09:
% At least in the long run, using 'makedoc' should not imply commitment 
% to a certain design or to certain \LaTeX\ packages for typesetting 
% listings and documentations. Therefore, 'makedoc.cfg' (currently) 
% contains \emph{local} or \emph{personal choices}, but also 
% \emph{experiments} with future features of 'niceverb'. 
% Especially, (at present) the `packagecode' 
% environment that 'makedoc' `\write's must be chosen. 
% Currently this is the `listing' environment from 'moreverb' 
% with some modifications or extra settings. 
% It may be vital to `\MakeOther' the active characters from 'niceverb' 
% in the setup of `packagecode'. See the \emph{example} in 
% section~\ref{sec:fifinddo}.
% 
% Finally, 
% Each package file to be typeset needs its own little
% \emph{script} of 'makedoc' commands. 
% With v0.3, one or two of these should suffice. 

% It should fit into the preamble 
% of the main file for documenting the package (currently %% 2009/04/09
%   just 5 commands seem to suffice, see the \emph{example} in 
%   section~\ref{sec:fifinddo}). 
% The script commands are described 
% in the \dqtd{File handling} section of 'fifinddo.pdf' and in the 
% present section~\ref{sec:script} (and \ref{sec:emptylines}).
% As an alternative, you may prefer to have ``content only" (as much as 
% possible) in the main typesetting file and in its preamble only 
% `\input' a separate script file.
%% removed 2010/03/10:
% Yes, the idea of documenting a package \emph{here} is to have a 
% separate ``driver" file for typsetting the documentation. 
% It may contain an introduction and a guide for users. 
% The documentation of the package code that has been prepared by the 
% 'makedoc' script will be `\input'. Alternatively, the ``driver file" 
% could have title etc.\ only, or preamble and a minimal `document' 
% environment only. 
% 
% So there may be many files, which may look confusing, especially as 
% compared with the 'doc' procedure. However, 
% \begin{enumerate}
% \item ``One file distribution" still is possible thanks to the 
%       `filecontents' environment. 
% \item The 'makedoc' script can create a batch file (fitting the 
%       system, maybe using Will Robertson's 'ifplatform', or 
%       'texsys.cfg', or \dots) 
%       that removes certain auxiliary files or moves them to a 
%       certain directory. 
% \item I find it helpful to have rather little ``contentual" text 
%       in the package file. 
% \item The procedure now runs very smoothly, once the stumbling blocks 
%       have been overcome.\footnote{\hspace{1sp}%% TODO help in 'niceverb'!
%         'niceverb' v0.1 was too sloppy with 
%         some things, and self-documentation of 'makedoc.sty' was 
%         difficult---its parsing and that from 'verbatim' cannot 
%         distinguish between markup code and typeset code.}
% \end{enumerate}

\section{Examples}%%% (documentation of 'mdoccorr.cfg')}
%% moved here 2010/03/23
\subsection{'nicetext', especially 'mdoccorr.cfg'}
The documentations of 'fifinddo', 'makedoc', and 'niceverb' 
themselves are typeset using 'makedoc'.
'fifinddo.pdf' documents 'fifinddo.sty', typeset 
from 'fifinddo.tex', likewise 'makedoc.pdf' and 'niceverb.pdf'. 
% Section~\ref{sec:fifinddo} contains a listing of 
% 'makedoc.cfg' and 
% the 'makedoc' script file 'mkfddoc.tex' 
% especially made for 'fifinddo.pdf'. 
% 'fifinddo.doc', 'makedoc.doc', and 'niceverb.doc' are the \TeX\ input 
% files that were made with 'makedoc.sty'---I have only looked at them 
% when something was wrong (often syntax mistakes in typing). 
The Wikipedia syntax feature 
\begin{quote}
  `%% === subsection ===' 
\end{quote}
is used in 'fifinddo.sty' and 'niceverb.sty' only.

Along with 'makedoc' should come files `makedoc.tpl'---a template 
'makedoc' script, and a file `fdtxttex.tex' that should start a dialogue 
on trying `\MakeDocCorrectHook' if you can manage to run it ('WinShell'?). 
With other definitions of `\MakeDocCorrectHook'---see below---you can 
use this dialogue for arbitrary replacement jobs (as an application of 
'fifinddo' rather than 'makedoc').{\sloppy\par}

'fifinddo.pdf', 'makedoc.pdf', and 'niceverb.pdf' were typeset with the following 
typographical corrections in 'mdoccorr.cfg' that defines 
`\MakeDocCorrectHook':
\strut\hrule
\begingroup 
  \hfuzz=\textwidth \advance \hfuzz by 28pt
  \MakeOther\|\MakeOther\`\MakeOther\'\MakeOther\<
  \listinginput[5]{1}{mdoccorr.cfg}
\endgroup
\hrule\noindent\strut
This code also exemplifies the syntax 'niceverb' provides for writing 
about \LaTeX\ macros. It is typeset here with 'makedoc.sty' 
and then looks thus:
%  \sloppy %% 2010/03/29
\strut\hrule
\renewcommand*{\mdJobName}{mdoccorr}
\MakeInputJobDoc[cfg]{0}{\ProcessInputWith{comment}}
\hrule \noindent\strut
And this is the content of the intermediate generated file:
\hrule
\begingroup 
  \hfuzz=\textwidth \advance \hfuzz by 28pt
  \MakeOther\|\MakeOther\`\MakeOther\'\MakeOther\<
  \listinginput[5]{1}{mdoccorr.doc}
\endgroup
\hrule 
%  \fussy %% 2010/03/29

\subsection{Packages from other authors}
`substr.tex' should typeset a nicely formatted documentation 
of Harald Har\-der's 'substr.sty', see my own result `substr.pdf'. 
'substr.sty' is a rare case of the \lq`%% '\rq\ commenting style 
that 'nicetext' has used itself.

`arseneau.tex' should typeset nicely formatted documentations 
of a few packages by Donald Arseneau, see my own result `arseneau.pdf'. 
This demonstrates the usual \lq`% '\rq\ commenting style 
that 'makedoc' supports with v0.4.

\pagebreak              %% 2010/03/29
\ResetCodeLineNumbers   %% 2010/03/29
\section{Implementation}
\subsection{Preliminaries} 
Head of file (Legalese):
\sloppy
\renewcommand*{\mdJobName}{makedoc}
\ProcessLineMessage{}
\MakeInputJobDoc{22}{\ProcessInputWith{comment}}
The previous empty code line is the one \TeX\ insists to add at every 
end of a file it writes. %% todo TeXbook where? 2009/04/08

%% removed (TODO) 2010/03/15:
% \section{Examples: documentation of 'fifinddo'}
% \label{sec:fifinddo} 
%% removed 2010/03/10:
% \subsection{'makedoc.cfg'} 
% 'fifinddo.pdf' and 'makedoc.pdf' were typeset with the following 
% configuration file 'makedoc.cfg':
% \begingroup \MakeOther\|\MakeOther\`\MakeOther\'\MakeOther\<
%   %% <- TODO should be 'niceverb' command 2009/04/08
%   \listinginput[5]{1}{makedoc.cfg}
% \endgroup
%
%% TODO 'niceverb' example---to 'niceverb.tex'!? 2010/03/15
% \subsection{'mkfddoc.tex'}
% 'fifinddo.pdf' was typeset with the following 'makedoc' script file 
% 'mkfddoc.tex':
% \begingroup 
%   \MakeOther\|\MakeOther\`\MakeOther\'\MakeOther\<
%   \listinginput[5]{1}{mkfddoc.tex}
% \endgroup
% 
%
\end{document}

2009/04/12  more on examples
2009/04/15  exemplification of niceverb.sty by mdcorr.cfg
2009/04/21  === subsubsection -> === subsection
2010/03/08  moved `only' for better line break
2010/03/09  removed something from "Basics"
2010/03/10  more changes in "Basics", pdf stuff to makedoc.cfg, 
            makedoc.cfg no longer example; CodeDoc
2010/03/11  use \MakeCloseDoc; \hfuzz with \listinginput;
            tracing spurious `Label(s) may have changed'
2010/03/12  tests for hyperref compatibility completed
2010/03/13  use \MakeInputJobDoc; clarified ...; ctan.org/pkg
2010/03/14  updated ``Examples" and abstract; \href...easylatex
2010/03/15  ``styles supported"; abstract: txt->TeX; usage
2010/03/16  more on usage; mdcorr -> mdoccorr
2010/03/17  corr. mistake with \MakeDoc
2010/03/19  '' -> " 
2010/03/20  for niceverb v0.31
2010/03/21  for niceverb v0.32
2010/03/22  "may break"
2010/03/23  \noindent in example, moved, added mdoccorr.doc, 
            makedoc.tpl
2010/03/29  \ResetCodeLineNumbers, 
            examples and explanations for v0.4
2010/03/30  \listfiles test
 %% with pdf stuff and 'niceverb'
%% removed for niceverb v0.31 TODO!? 2010/03/20:
\sfcode`/=1001 %% TODO makedoc.cfg!? 2010/03/21
% \makeatletter %% TEST for hyperref compatibility 2010/03/11
%   \def\@testdef #1#2#3{%
%     \def\reserved@a{#3}%
%     \expandafter \ifx \csname #1@#2\endcsname
%    \reserved@a  \else \@tempswatrue \fi
%     \if@tempswa
%       \typeout{^^J*** Type `r' <input> to get around 
%                       \string\label\space issues! ***^^J}
%       \errorcontextlines=0
%       \show\reserved@a
%       \expandafter \show \csname #1@#2\endcsname
%     \fi
%    }
% \makeatother
\begin{document}
\title{'makedoc'---Preprocessing documentation by \TeX}
  %% 2009/04/10: \\---\\\ breaks TOC
\maketitle
\begin{abstract}\noindent
'makedoc' provides commands for generating \LaTeX\ input from a 
package file in order to typeset the latter's documentation 
(somewhat similar and opposite to 'docstrip')---with 
v0.3 \emph{a single one usually suffices}. 
Certain comment marks are removed, listing commands are inserted, 
and some (configurable) typographical `txt'$\to$\TeX\ corrections 
are applied.---This 
continues the policy of 'niceverb' to minimize documentation markup in 
package files. 'makedoc' extends and exemplifies the parsing package 
'fifinddo'. After an edit (and test) of your package, you get the new 
documentation in one run (or the usual number of runs) of the 
documentation driver file.---The present approach is meant to be an 
\emph{alternative} to the standard 'doc' package and its `\DocInput'. 
It provides \emph{less} than 'doc' does, rather deliberately. It may 
be helpful at least for the development of small packages, or at least 
at early stages.
\end{abstract}
\tableofcontents
\section{Introduction}
\emph{The abstract will not be repeated in this section.} Let me add 
instead that I was in dire need of such a package, I got stuck with my 
packages because I lost orientation in them, and I was unhappy with 
the forms of documentations of my other packages, and documenting them 
with the standard \LaTeX\ 'doc' system was not attractive for me 
(neither considered helpful). %% clarified 2010/03/13
I also worked on \emph{Windows} until September 2008, and I 
find a system like the present one still more attractive then using 
(learning!\@) other filtering utilities (see below on 'awk'). And I 
may work on \emph{Windows} once again and don't want to depend on 
installing some $\dots$ there---\emph{I really would like to have 
powerful tools for everything depending on nothing but \TeX\slash 
\LaTeX!}

\section{Prior work and what is new}
It is, of course, not a new idea to get around comment marks `%' to 
typeset the documentation. 'doc''s `\DocInput' does this by making `%' 
an ``ignored" character. This way you cannot use `%' for commenting 
comments (so 'doc' offers a ``new comment mark" 
`^'`^'`A'). %% TODO `^^A' suddenly failed 2010/03/15 -- "ligature"!?
You also cannot use `%' for commenting out code (that you are 
pondering---or using for debugging---only). %% clarified 2010/03/13

Moreover, 'doc' requires enclosing package code explicitly by 
environment commands (behind comment marks). Stephan I. B\"ottcher 
with his '\href{http://ctan.org/pkg/lineno}{lineno.sty}' 
and Grzegorz Murzynowski in \ctanpkgref{gmdoc}
aimed at doing away with this requirement. 
'lineno.sty' contains 'awk' scripts 
to remove starting comment marks and to insert listing commands. 
A file 'lineno.tex' is generated that typesets the documentation. 
By the way, 'lineno.sty' is full of discussions, but it is not 
'docstrip'ped---the maintainers never have received a complaint 
that inputting 'lineno.sty' were too slow. 

'gmdoc' seems to get around comment marks and insert listing commands 
\emph{while typesetting} by a refined version of `\DocInput', 
through some careful detecting and analysing comment marks, 
the approach resembles detection of lists in 'wiki.sty'.\footnote{See 
  'gmdoc.pdf' on &\DocInput. You can learn a lot from this 220 pages 
  document! I also find 
  \ctanpkgref{pauldoc} and \ctanpkgref{xdoc} inspiring.}
And this is a matter of principles---comparing the approaches of 
\emph{preprocessing} ('lineno.sty') and \emph{``smart typesetting"} 
('gmdoc', 'wiki'). Sometimes preprocessing seems to be simpler, 
sometimes detecting while typesetting. 
(Another example is the preprocessor 
\ctanpkgref{easylatex}
of which 'wiki.sty' is a much reduced ``while typesetting" variant.) 
``While typesetting" may be easier when single characters or 
sequences of two or three encode markup
information---but such detection can badly interfere with other 
packages etc. ``Preprocessing" may be easier when entire ``strings" 
of characters decide, which may be anywhere in a file line. 

'makedoc' chooses \emph{preprocessing}, as 'lineno.sty', but by 
\emph{\TeX}. There is a general discussion of this choice in the 
documentation of 'fifinddo'. Preprocessing here can be done in the 
same \LaTeX\ run as typesetting, though you can avoid 
incompatabilities with packages needed for typesetting 
(by inputting them only \emph{after} preprocessing). 

'lineno.sty' exemplifies why preprocessing with \emph{\TeX} may be 
preferable to preprocessing with other utilities: 
When I took over maintenance of 'lineno.sty', 
I needed hard work to get the 'awk' script running. 
The \emph{Munich} 'awk' seemed \emph{not} to behave as the \emph{Kiel} 
'awk' (I chose a Munich 'nawk' and reworked the script a little). 
\TeX\ seems to have better fixed functionality than other utilities! 

A different alternative to \LaTeX's 'doc' system is 
Paul Isambert's '\href{http://ctan.org/pkg/codedoc}{CodeDoc}' 
where the code environments extract package code in typesetting the 
documentation. %% added 2010/03/10

\section{Styles supported (parsers provided)}\label{sec:styles}
% \section{Styles of commenting '.sty's}
We find different styles of documenting \LaTeX\ packages. 
As the main aspects I consider 
(i)~\emph{telling code from comments} 
and (ii)~\emph{markup in comments}. 
(You may find more details on the next matters in the 
 ``implementation" section.)

\subsection{Telling code from comments}
\emph{Comment marks} (usually \lq`%'\rq\ in the case of \TeX) 
probably were named so to mark \emph{``comments"} as opposed 
to code $\dots$ great, but actually, in ``daily practice," 
they are so handy---and used---for ``commenting out" \emph{code}, 
i.e., \emph{managing code versions} in a simple way: 
one does not actually want to \emph{delete} code, 
one might want to use it another time, maybe for debugging
$\dots$ or to remind of earlier attempts that should not be tried 
again $\dots$

This is a problem for \emph{high-quality typesetting} of 
documentation. \emph{Code} should be typeset about as you see it on 
the \emph{screen}---\emph{monospaced}, this allows structuring by 
indenting, it is common practice to use a typewriter typeface for 
this. Real \emph{comments} should be typeset in \emph{high quality} as 
usual with \LaTeX. Little dilemmas therefore occur with \emph{``hidden 
code"} (``commented-out"). A comment mark starts the line, but 
obviously it is not really a comment and rather should be typeset 
like code (and otherwise they may break).           %% 2010/03/22
Another problem are comments at the \emph{end} of a 
\emph{code} line. Sometimes they are ``real comments" ('gmdoc' 
supports this style). But sometimes 
this is only another version of ``version management," code 
``commented-out."

I like the style of writing packages described before and use it all 
the time. I mark ``real comments" with \emph{two} adjacent comment 
marks and an ensuing space to distinguish them clearly from code 
commented out.
%% Adapted to v0.4 2010/03/29:
\emph{This style is presently the one supported by \textup{'makedoc'} 
      as default.}
This way only a line starting with 
|%% | is considered a ``real" comment line. The first three 
characters are removed, and the rest is typeset in high quality. 
Any other lines are typeset verbatim. The 'makedoc' \emph{parser} 
doing this has an ``identifier" |PPScomment| (``percent, percent, 
space"). Another identifier |comment| is a placeholder for 
the comment parser to be used, by default it is an alias for 
`PPScomment'. Lines just containing |%%| (without the space) may be 
used to suppress empty code lines preceding section titles and for 
keeping some visual, relieving space between code and comment lines.

The style I described previously may be considered ``unprofessional." 
The many \LaTeX\ packages documented using the 'doc'\slash'.dtx' 
system don't use comment marks for \emph{``commenting-out"}. 
Or one may mark code commented out by putting no space between the 
percent mark and the code. 
With v0.4 of 'makedoc', this style is supported as |PScomment|. 
You can directly call this as <main-parser> as described below, 
or you can switch to it by 
\[`\CopyFDconditionFromTo{PScomment}{comment}'\]

\subsection{Markup in comments}
Packages using the 'doc'\slash '.dtx' system as well as alternative 
highly developed systems mentioned above use (enhanced) usual 
\emph{\LaTeX} syntax for markup of comments. Other packages just use 
an \emph{ASCII} style \emph{without} any markup. My idea was to 
support the latter style by some `txt'$\to$\LaTeX\ functionality. 
'makedoc' does this using a file 'mdoccorr.cfg' which is very small 
right now.

I also thought of introducing another sort of ``decent" markup not 
needing much more space than the ``ASCII kernel" of the comments. 
This is to some extent implemented in 'niceverb.sty'. I thought of the 
syntax of editing \textit{Wikipedia} pages; this is partially 
implemented in 'wiki.sty' which unfortunately is not yet compatible 
with 'niceverb'.

But 'makedoc' implements one \textit{Wikipedia} feature in a different 
way than 'wiki.sty' (cf.~'wikicheat.pdf') that looks about as follows:
\begin{eqnarray*}
  \endcell\endcell`%% == Section =='\\
  \endcell\endcell`%% === Subsection ==='\\
  \endcell\endcell`%% ==== Subsubsection ===='
\end{eqnarray*}
i.e., you type `== <title> ==' in place of `\section{<title>}' etc.
The parser must replace `====<title3>===' before `===<title2>===' and 
the latter before `==<title1>=='. In fact, 'makedoc' provides three 
parsers for these situations:
\begin{description}
\cmdboxitem|\SectionLevelThreeParseInput| is the most general parser 
    offered. If it does not find two strings \lq`===='\rq\ enclosing 
    \emph{something}, it passes to
\cmdboxitem|\SectionLevelTwoParseInput| which unless finding 
    two strings `===' enclosing something passes to
\cmdboxitem|\SectionLevelOneParseInput| $\dots$ passes to the comment 
    detector |comment|. 
\end{description}


\section{Requirements}
'makedoc' requires \LaTeXe\ (supporting star forms of `\newcommand' 
etc.)\ as \TeX-format, the package 'fifinddo.sty' from the same 
directory (on CTAN etc.)\ as where 'makedoc.sty' is, and the 
\LaTeX-package 'moreverb' by Robin Fairbairns (after others)---it 
should be installed anyway, or you can get its latest version 
(v2.3, 2008/06/03?) from CTAN. 

'makedoc''s `.txt'$\to$\TeX\ functionality moreover needs a file 
'mdoccorr.cfg' that should have come along with 'makedoc.sty' and 
'fifinddo.sty'. You may need to have a modified copy of it in the 
directory of your main `.tex' file `<jobname>.tex' fitting special 
needs of your project. 

\section{Using 'makedoc' the simplest way}
In the most simple case, you are preparing documentation for a package 
file `<jobname>.sty' only, and you prepare a file `<jobname>.tex' 
containing 
\[`\title{\textsf{<jobname>}---a \LaTeX\ Package for <whatever>}'\]
and `\maketitle' etc.\ about your package.\footnote{With 'niceverb' 
    and &\title\ after &\begin{document}, you may replace 
    \lq&\textsf{<jobname>}\rq\ by \lq&'<jobname>&'\rq.}
The documentation will be produced by running `<jobname>.tex' with 
\LaTeX\ (e.g., \texttt{latex <jobname>.tex}).

First, `<jobname>.tex' must have |\usepackage{makedoc}| in its preamble. 
There are no package options. 

Second, to typeset the commented implementation from `<jobname>.sty', 
include in <jobname>.tex's `document' environment a line 
\[|\MakeInputJobDoc{<header-lines>}{\SectionLevelThreeParseInput}|\]
<header-lines> refers to a non-negative integer as follows: 
We think the most simple and useful way of typesetting the first lines 
of a package file including license and copyrights is ``depicting them 
as image," i.e., \textit{verbatim}. We could try to determine the 
number of these lines by parsing, but we won't do so soon. Please just 
count them and enter the number as <header-lines>---and change it 
until you can accept the outcome.

\section{Steps of advanced usage}
\subsection{Different main parsers (second mandatory argument)}

`\MakeInputJobDoc''s mandatory syntax actually is 
\[|\MakeInputJobDoc{<header-lines>}{<main-parser>}|\]
<main-parser> refers to the parsing macro that is applied to each 
input line whose number is greater than <header-lines>. 
Examples for <main-parser> are named in section~\ref{sec:styles} above. 
  %% TODO above/below macro 2010/03/15
`\SectionLevelThreeParseInput' is just the most general one. 
For \emph{efficiency} (!? or also to avoid problems?) you may 
replace `Three' by `Two' or by `One', if the `====' or the `===' 
feature is not used in `<jobname>.sty'. If the ``\textit{Wikipedia} 
sectioning" feature is not used at all, use 
\[|\MakeInputJobDoc{<header-lines>}{\ProcessInputWith{comment}}|\]
---provided you want to adopt the \lq`%% '\rq\ style of marking 
comments, cf.~section~\ref{sec:styles}. For the \lq`% '\rq style 
instead, use 
\[|\MakeInputJobDoc{<header-lines>}{\ProcessInputWith{PScomment}}|\]

\subsection{Different extensions (optional arguments)}
If your package to be documented is a \emph{class} `<jobname>.cls', 
a local configuration file `<jobname>.cfg' or something 
else---<jobname>.<ext-in>, e.g., <ext-in>=`cls' or <ext-in>=`cfg', 
use 
\[|\MakeInputJobDoc[<ext-in>]{<header>}{<parser>}|\]
Moreover, `\MakeInputJobDoc' writes an intermediate file 
`<jobname>.doc' and then `\input's it. If you do not like `doc' 
as extension for the written file name (maybe you use 
`<jobname>.doc' for something different already), preferring extension 
<ext-out>, use
\[|\MakeInputJobDoc[<ext-in>][<ext-out>]{<header>}{<parser>}|\]
Yes, you must state <ext-in> then as well, I can't help $\dots$

If even <jobname> is wrong in your view, see next step $\dots$

\subsection{Commands modifying &\MakeInputJobDoc's behaviour}
\label{sec:modimake}
Already <jobname> may not be what you want. E.g., you may want 
to collect documentations of some other files <job-1>, <job-2>, 
$\dots$ in a single <jobname>. Then precede `\MakeInputJobDoc'
with 
\[`\renewcommand*{\mdJobName}{<job-1>}'\]
etc.\ (please reason yourself about additional requirements \dots)
As a matter of fact, `\MakeInputJobDoc' reads 
\[`\mdJobName.<ext-in>' \mbox{\quad and writes\quad} 
  `\mdJobName.<ext-out>'\]
Stated another way, <jobname> above referred to |\mdJobName|.

`\MakeInputJobInput' moreover (by default) produces one dot 
per input line processed on screen to show progress. 
The reason is that `makedoc' issues the command 
|\ProcessLineMessage{\message{.}}|.
Already this trivial thing seems to slow down processing considerably 
(nowadays). `\MakeInputJobInput' will run faster if preceded by 
\[|\ProcessLineMessage{}|\]
which will suppress any message about processing.
However, the message may be helpful in trouble-shooting.

\subsection{Separating preprocessing from typesetting}
  %% extended 2010/03/16
To some surprise, I observe that `\MakeInputJobDoc' \emph{works.} 
This is quite a new discovery of mine (2010/03/13); 
before I thought that, for safety, preprocessing should happen 
inside a local group \emph{preceding} `\documentclass'.
|\MakeJobDoc| works like `\MakeInputJobDoc' described above, 
yet it just \emph{preprocesses} the package to be documented, 
waiting for an 
\[`\input{<jobname>.<ext-out>}'\] 
in the `document' environment to \emph{typeset} the documentation. 
So 'makedoc.tex' once had in its preamble
% \[`{\RequirePackage{makedoc} \MakeJobDoc{<header>}{<parser>}}'\]
% at the top of `<jobname>.tex' and `\input{<jobname>.<out-ext>}' later. 
\begin{eqnarray*}\endcell\endcell
  `{\RequirePackage{makedoc}'\cr    \endcell\endcell
  ` \ProcessLineMessage{}'\cr       \endcell\endcell
  ` \MakeJobDoc{22}{\ProcessInputWith{comment}}}'\cr
                                    \endcell\endcell
  `\documentclass{article}'
\end{eqnarray*} 
I did experience some truth in my earlier safety policy: 
With 'niceverb' ``running," `\MakeJobDoc' cannot (easily) be used 
in the `document' environment. `\MakeInputJobDoc' in fact does some 
'niceverb' switching (provided 'niceverb' has been loaded) 
when making use of `\MakeJobDoc'.
  %% <- verbose to improve line breaks 2010/03/16

Thinking of this ``safety" approach, especially grouping (`{\code}'), 
I had not much cared for compatibility with other packages 
in choosing 'makedoc' macro names. 

\subsection{Other 'makedoc' (and 'fifindo') script commands}
The next script commands may be considered of a lower level than 
`\MakeJobDoc' and `\MakeInputJobDoc', they underlie the latter 
commands. We also list commands from 'fifinddo.sty' that may be useful 
or, indeed, are needed for preparing package documentations. 
This may result in ideas on how do use the script commands for 
different purposes than for preparing package documentations---e.g., 
apply `txt'$\to$\TeX\ preprocessing to arbitrary text files. 

\subsubsection{Choosing parameter values for next preprocessing run}

This actually continues section~\ref{sec:modimake}.

\begin{description}
\cmdboxitem|\ResultFile{<output>}| (from 'fifinddo') 
    determines (and opens through the \TeX\ primitive `\openout') 
    the file <output> which will contain the result of 
    preprocessing the package file.
\cmdboxitem|\LaTeXresultFile{<output>}|---see next section.
\cmdboxitem|\Headerlines{<number>}| determines the <number> of lines 
    starting the input file to be copied \emph{verbatim} 
    (the first mandatory argument of `\MakeJobDoc'). 
\cmdboxitem|\MainDocParser{<parser>}| determines the <parser> 
    as in the \emph{second} mandatory argument of `\MakeJobDoc'.
\end{description}

We are now describing some parameters which rather must be switched 
``manually" instead of being modifiable by comfortable 'makedoc' 
script commands.

With the \emph{``Wikipedia sectioning"} feature, you may change the outcome 
regarding levels. Assume, e.g., the package file has titles along the 
scheme `== <title> ==' only, but these should become 
\emph{subsections} of the ``implementation" section of the 
corresponding `.tex' file. Then 
\[`\renewcommand*{\mdSectionLevelOne}{\string\subsection}'\]
-- see the implementation of the sectioning feature for details. 

There is a command 
\[|\NoEmptyInputLines| \mbox{\quad and a parameter macro\quad}
  |\OnEmptyInputLine|\] 
which is modified by the former. However, I cannot say much about them 
right now, I think they just were a difficult matter that I soon 
decided no longer to think about for a while (cf.\ the 
implementation). About the same holds for the hook |\EveryComment|.

The `txt'$\to$\TeX\ functionality is implemented through a hook 
\[|\MakeDocCorrectHook{<characters>}|\] 
'makedoc' initializes it as an alias of \LaTeX's `\@firstofone', i.e., 
it won't change <characters>. 'mdoccorr.cfg' should redefine it so it 
really ``corrects" <characters>. You might try other definitions of 
`\MakeDocCorrectHook' for different ``correcting" functions.
It should be \emph{noted} that (currently) 
`\MakeDocCorrectHook' must be \emph{expandable}, 'fifinddo.sty' 
provides setup for (expandable) chains of expandable replacements. 
The ``Wikipedia" sectioning feature moreover uses expandable 
trimming (single) surrounding spaces, this might be provided in a more 
general way.\footnote{%% TODO 2010/03/16
    The \ctanpkgref{trimspaces} package 
    has been a \emph{model} for this feature here. It cannot be used 
    directly here because it reads blank spaces as \TeX\ characters with 
    category code 10 while 'makedoc' reads blank spaces as ``other" 
    characters (category code 12) in order to \emph{keep} all blank spaces.}

\subsubsection{``Manual" insertions to the output file}
\begin{description}
\cmdboxitem|\WriteResult{<balanced>}| (from 'fifinddo') writes 
    <balanced> to <output> according to the earlier command 
    `\ResultFile{<output>}'.
\cmdboxitem|\WriteProvides| (from 'fifindo') writes a 
    `\ProvidesFile' line into <output> that declares the file 
    to be generated by 'fifindo'.
\cmdboxitem|\LaTeXresultFile{<output>}| issues 
    `\ResultFile{<output>}' and then writes a 
    `\ProvidesFile' line into <output> that declares the file 
    to be generated by 'makedoc'.
\end{description}

\subsubsection{Processing input and closing output}
\begin{description}
\cmdboxitem|\MakeDoc{<input>}|\hskip 0pt plus 4em
    reads 'mdoccorr.cfg' 
    (for `\MakeDocCorrectHook', see above),
    %% removed \LaTeXresult... 2010/03/17
    copies <number> according to `\HeaderLines' (see above) 
    from <input> into <output> (according to `\ResultFile'), 
    then processes the remaining lines of <input> according 
    to `\MainDocParser' (writing several things to <output>). 
    `\MakeDoc' invokes{\sloppy\par}
\cmdboxitem|\ProcessFileWith{<input>}{<loop-body>}| 
    (from 'fifindo') reads <input> line by line---each one stored as 
    macro |\fdInputLine| and applies <loop-body> to it. 
    \TeX's ``special" character codes (of characters listed in 
    macro `\dospecials') are replaced by 12 (``other") by 
    default---see the 'fifinddo' documentation.
\cmdboxitem|\CloseResultFile| (from 'fifinddo') 
    \hskip 0pt plus 1em
    \emph{closes} 
    <output> (using \TeX's primitive `\closeout'). 
\cmdboxitem|\MakeCloseDoc{<input>}| issues 
    `\MakeDoc{<input>}\CloseResultFile'.
\end{description}
%
Using `\MakeDoc' \emph{instead} of `\MakeCloseDoc' allows processing 
additional <input> files writing into the same <output>. Or maybe you 
want to add some additional lines manually to <output> using 
`\WriteResult'.

%% removed 2010/03/09:
% At least in the long run, using 'makedoc' should not imply commitment 
% to a certain design or to certain \LaTeX\ packages for typesetting 
% listings and documentations. Therefore, 'makedoc.cfg' (currently) 
% contains \emph{local} or \emph{personal choices}, but also 
% \emph{experiments} with future features of 'niceverb'. 
% Especially, (at present) the `packagecode' 
% environment that 'makedoc' `\write's must be chosen. 
% Currently this is the `listing' environment from 'moreverb' 
% with some modifications or extra settings. 
% It may be vital to `\MakeOther' the active characters from 'niceverb' 
% in the setup of `packagecode'. See the \emph{example} in 
% section~\ref{sec:fifinddo}.
% 
% Finally, 
% Each package file to be typeset needs its own little
% \emph{script} of 'makedoc' commands. 
% With v0.3, one or two of these should suffice. 

% It should fit into the preamble 
% of the main file for documenting the package (currently %% 2009/04/09
%   just 5 commands seem to suffice, see the \emph{example} in 
%   section~\ref{sec:fifinddo}). 
% The script commands are described 
% in the \dqtd{File handling} section of 'fifinddo.pdf' and in the 
% present section~\ref{sec:script} (and \ref{sec:emptylines}).
% As an alternative, you may prefer to have ``content only" (as much as 
% possible) in the main typesetting file and in its preamble only 
% `\input' a separate script file.
%% removed 2010/03/10:
% Yes, the idea of documenting a package \emph{here} is to have a 
% separate ``driver" file for typsetting the documentation. 
% It may contain an introduction and a guide for users. 
% The documentation of the package code that has been prepared by the 
% 'makedoc' script will be `\input'. Alternatively, the ``driver file" 
% could have title etc.\ only, or preamble and a minimal `document' 
% environment only. 
% 
% So there may be many files, which may look confusing, especially as 
% compared with the 'doc' procedure. However, 
% \begin{enumerate}
% \item ``One file distribution" still is possible thanks to the 
%       `filecontents' environment. 
% \item The 'makedoc' script can create a batch file (fitting the 
%       system, maybe using Will Robertson's 'ifplatform', or 
%       'texsys.cfg', or \dots) 
%       that removes certain auxiliary files or moves them to a 
%       certain directory. 
% \item I find it helpful to have rather little ``contentual" text 
%       in the package file. 
% \item The procedure now runs very smoothly, once the stumbling blocks 
%       have been overcome.\footnote{\hspace{1sp}%% TODO help in 'niceverb'!
%         'niceverb' v0.1 was too sloppy with 
%         some things, and self-documentation of 'makedoc.sty' was 
%         difficult---its parsing and that from 'verbatim' cannot 
%         distinguish between markup code and typeset code.}
% \end{enumerate}

\section{Examples}%%% (documentation of 'mdoccorr.cfg')}
%% moved here 2010/03/23
\subsection{'nicetext', especially 'mdoccorr.cfg'}
The documentations of 'fifinddo', 'makedoc', and 'niceverb' 
themselves are typeset using 'makedoc'.
'fifinddo.pdf' documents 'fifinddo.sty', typeset 
from 'fifinddo.tex', likewise 'makedoc.pdf' and 'niceverb.pdf'. 
% Section~\ref{sec:fifinddo} contains a listing of 
% 'makedoc.cfg' and 
% the 'makedoc' script file 'mkfddoc.tex' 
% especially made for 'fifinddo.pdf'. 
% 'fifinddo.doc', 'makedoc.doc', and 'niceverb.doc' are the \TeX\ input 
% files that were made with 'makedoc.sty'---I have only looked at them 
% when something was wrong (often syntax mistakes in typing). 
The Wikipedia syntax feature 
\begin{quote}
  `%% === subsection ===' 
\end{quote}
is used in 'fifinddo.sty' and 'niceverb.sty' only.

Along with 'makedoc' should come files `makedoc.tpl'---a template 
'makedoc' script, and a file `fdtxttex.tex' that should start a dialogue 
on trying `\MakeDocCorrectHook' if you can manage to run it ('WinShell'?). 
With other definitions of `\MakeDocCorrectHook'---see below---you can 
use this dialogue for arbitrary replacement jobs (as an application of 
'fifinddo' rather than 'makedoc').{\sloppy\par}

'fifinddo.pdf', 'makedoc.pdf', and 'niceverb.pdf' were typeset with the following 
typographical corrections in 'mdoccorr.cfg' that defines 
`\MakeDocCorrectHook':
\strut\hrule
\begingroup 
  \hfuzz=\textwidth \advance \hfuzz by 28pt
  \MakeOther\|\MakeOther\`\MakeOther\'\MakeOther\<
  \listinginput[5]{1}{mdoccorr.cfg}
\endgroup
\hrule\noindent\strut
This code also exemplifies the syntax 'niceverb' provides for writing 
about \LaTeX\ macros. It is typeset here with 'makedoc.sty' 
and then looks thus:
%  \sloppy %% 2010/03/29
\strut\hrule
\renewcommand*{\mdJobName}{mdoccorr}
\MakeInputJobDoc[cfg]{0}{\ProcessInputWith{comment}}
\hrule \noindent\strut
And this is the content of the intermediate generated file:
\hrule
\begingroup 
  \hfuzz=\textwidth \advance \hfuzz by 28pt
  \MakeOther\|\MakeOther\`\MakeOther\'\MakeOther\<
  \listinginput[5]{1}{mdoccorr.doc}
\endgroup
\hrule 
%  \fussy %% 2010/03/29

\subsection{Packages from other authors}
`substr.tex' should typeset a nicely formatted documentation 
of Harald Har\-der's 'substr.sty', see my own result `substr.pdf'. 
'substr.sty' is a rare case of the \lq`%% '\rq\ commenting style 
that 'nicetext' has used itself.

`arseneau.tex' should typeset nicely formatted documentations 
of a few packages by Donald Arseneau, see my own result `arseneau.pdf'. 
This demonstrates the usual \lq`% '\rq\ commenting style 
that 'makedoc' supports with v0.4.

\pagebreak              %% 2010/03/29
\ResetCodeLineNumbers   %% 2010/03/29
\section{Implementation}
\subsection{Preliminaries} 
Head of file (Legalese):
\sloppy
\renewcommand*{\mdJobName}{makedoc}
\ProcessLineMessage{}
\MakeInputJobDoc{22}{\ProcessInputWith{comment}}
The previous empty code line is the one \TeX\ insists to add at every 
end of a file it writes. %% todo TeXbook where? 2009/04/08

%% removed (TODO) 2010/03/15:
% \section{Examples: documentation of 'fifinddo'}
% \label{sec:fifinddo} 
%% removed 2010/03/10:
% \subsection{'makedoc.cfg'} 
% 'fifinddo.pdf' and 'makedoc.pdf' were typeset with the following 
% configuration file 'makedoc.cfg':
% \begingroup \MakeOther\|\MakeOther\`\MakeOther\'\MakeOther\<
%   %% <- TODO should be 'niceverb' command 2009/04/08
%   \listinginput[5]{1}{makedoc.cfg}
% \endgroup
%
%% TODO 'niceverb' example---to 'niceverb.tex'!? 2010/03/15
% \subsection{'mkfddoc.tex'}
% 'fifinddo.pdf' was typeset with the following 'makedoc' script file 
% 'mkfddoc.tex':
% \begingroup 
%   \MakeOther\|\MakeOther\`\MakeOther\'\MakeOther\<
%   \listinginput[5]{1}{mkfddoc.tex}
% \endgroup
% 
%
\end{document}

2009/04/12  more on examples
2009/04/15  exemplification of niceverb.sty by mdcorr.cfg
2009/04/21  === subsubsection -> === subsection
2010/03/08  moved `only' for better line break
2010/03/09  removed something from "Basics"
2010/03/10  more changes in "Basics", pdf stuff to makedoc.cfg, 
            makedoc.cfg no longer example; CodeDoc
2010/03/11  use \MakeCloseDoc; \hfuzz with \listinginput;
            tracing spurious `Label(s) may have changed'
2010/03/12  tests for hyperref compatibility completed
2010/03/13  use \MakeInputJobDoc; clarified ...; ctan.org/pkg
2010/03/14  updated ``Examples" and abstract; \href...easylatex
2010/03/15  ``styles supported"; abstract: txt->TeX; usage
2010/03/16  more on usage; mdcorr -> mdoccorr
2010/03/17  corr. mistake with \MakeDoc
2010/03/19  '' -> " 
2010/03/20  for niceverb v0.31
2010/03/21  for niceverb v0.32
2010/03/22  "may break"
2010/03/23  \noindent in example, moved, added mdoccorr.doc, 
            makedoc.tpl
2010/03/29  \ResetCodeLineNumbers, 
            examples and explanations for v0.4
2010/03/30  \listfiles test
 %% with pdf stuff and 'niceverb'
%% removed for niceverb v0.31 TODO!? 2010/03/20:
\sfcode`/=1001 %% TODO makedoc.cfg!? 2010/03/21
% \makeatletter %% TEST for hyperref compatibility 2010/03/11
%   \def\@testdef #1#2#3{%
%     \def\reserved@a{#3}%
%     \expandafter \ifx \csname #1@#2\endcsname
%    \reserved@a  \else \@tempswatrue \fi
%     \if@tempswa
%       \typeout{^^J*** Type `r' <input> to get around 
%                       \string\label\space issues! ***^^J}
%       \errorcontextlines=0
%       \show\reserved@a
%       \expandafter \show \csname #1@#2\endcsname
%     \fi
%    }
% \makeatother
\begin{document}
\title{'makedoc'---Preprocessing documentation by \TeX}
  %% 2009/04/10: \\---\\\ breaks TOC
\maketitle
\begin{abstract}\noindent
'makedoc' provides commands for generating \LaTeX\ input from a 
package file in order to typeset the latter's documentation 
(somewhat similar and opposite to 'docstrip')---with 
v0.3 \emph{a single one usually suffices}. 
Certain comment marks are removed, listing commands are inserted, 
and some (configurable) typographical `txt'$\to$\TeX\ corrections 
are applied.---This 
continues the policy of 'niceverb' to minimize documentation markup in 
package files. 'makedoc' extends and exemplifies the parsing package 
'fifinddo'. After an edit (and test) of your package, you get the new 
documentation in one run (or the usual number of runs) of the 
documentation driver file.---The present approach is meant to be an 
\emph{alternative} to the standard 'doc' package and its `\DocInput'. 
It provides \emph{less} than 'doc' does, rather deliberately. It may 
be helpful at least for the development of small packages, or at least 
at early stages.
\end{abstract}
\tableofcontents
\section{Introduction}
\emph{The abstract will not be repeated in this section.} Let me add 
instead that I was in dire need of such a package, I got stuck with my 
packages because I lost orientation in them, and I was unhappy with 
the forms of documentations of my other packages, and documenting them 
with the standard \LaTeX\ 'doc' system was not attractive for me 
(neither considered helpful). %% clarified 2010/03/13
I also worked on \emph{Windows} until September 2008, and I 
find a system like the present one still more attractive then using 
(learning!\@) other filtering utilities (see below on 'awk'). And I 
may work on \emph{Windows} once again and don't want to depend on 
installing some $\dots$ there---\emph{I really would like to have 
powerful tools for everything depending on nothing but \TeX\slash 
\LaTeX!}

\section{Prior work and what is new}
It is, of course, not a new idea to get around comment marks `%' to 
typeset the documentation. 'doc''s `\DocInput' does this by making `%' 
an ``ignored" character. This way you cannot use `%' for commenting 
comments (so 'doc' offers a ``new comment mark" 
`^'`^'`A'). %% TODO `^^A' suddenly failed 2010/03/15 -- "ligature"!?
You also cannot use `%' for commenting out code (that you are 
pondering---or using for debugging---only). %% clarified 2010/03/13

Moreover, 'doc' requires enclosing package code explicitly by 
environment commands (behind comment marks). Stephan I. B\"ottcher 
with his '\href{http://ctan.org/pkg/lineno}{lineno.sty}' 
and Grzegorz Murzynowski in \ctanpkgref{gmdoc}
aimed at doing away with this requirement. 
'lineno.sty' contains 'awk' scripts 
to remove starting comment marks and to insert listing commands. 
A file 'lineno.tex' is generated that typesets the documentation. 
By the way, 'lineno.sty' is full of discussions, but it is not 
'docstrip'ped---the maintainers never have received a complaint 
that inputting 'lineno.sty' were too slow. 

'gmdoc' seems to get around comment marks and insert listing commands 
\emph{while typesetting} by a refined version of `\DocInput', 
through some careful detecting and analysing comment marks, 
the approach resembles detection of lists in 'wiki.sty'.\footnote{See 
  'gmdoc.pdf' on &\DocInput. You can learn a lot from this 220 pages 
  document! I also find 
  \ctanpkgref{pauldoc} and \ctanpkgref{xdoc} inspiring.}
And this is a matter of principles---comparing the approaches of 
\emph{preprocessing} ('lineno.sty') and \emph{``smart typesetting"} 
('gmdoc', 'wiki'). Sometimes preprocessing seems to be simpler, 
sometimes detecting while typesetting. 
(Another example is the preprocessor 
\ctanpkgref{easylatex}
of which 'wiki.sty' is a much reduced ``while typesetting" variant.) 
``While typesetting" may be easier when single characters or 
sequences of two or three encode markup
information---but such detection can badly interfere with other 
packages etc. ``Preprocessing" may be easier when entire ``strings" 
of characters decide, which may be anywhere in a file line. 

'makedoc' chooses \emph{preprocessing}, as 'lineno.sty', but by 
\emph{\TeX}. There is a general discussion of this choice in the 
documentation of 'fifinddo'. Preprocessing here can be done in the 
same \LaTeX\ run as typesetting, though you can avoid 
incompatabilities with packages needed for typesetting 
(by inputting them only \emph{after} preprocessing). 

'lineno.sty' exemplifies why preprocessing with \emph{\TeX} may be 
preferable to preprocessing with other utilities: 
When I took over maintenance of 'lineno.sty', 
I needed hard work to get the 'awk' script running. 
The \emph{Munich} 'awk' seemed \emph{not} to behave as the \emph{Kiel} 
'awk' (I chose a Munich 'nawk' and reworked the script a little). 
\TeX\ seems to have better fixed functionality than other utilities! 

A different alternative to \LaTeX's 'doc' system is 
Paul Isambert's '\href{http://ctan.org/pkg/codedoc}{CodeDoc}' 
where the code environments extract package code in typesetting the 
documentation. %% added 2010/03/10

\section{Styles supported (parsers provided)}\label{sec:styles}
% \section{Styles of commenting '.sty's}
We find different styles of documenting \LaTeX\ packages. 
As the main aspects I consider 
(i)~\emph{telling code from comments} 
and (ii)~\emph{markup in comments}. 
(You may find more details on the next matters in the 
 ``implementation" section.)

\subsection{Telling code from comments}
\emph{Comment marks} (usually \lq`%'\rq\ in the case of \TeX) 
probably were named so to mark \emph{``comments"} as opposed 
to code $\dots$ great, but actually, in ``daily practice," 
they are so handy---and used---for ``commenting out" \emph{code}, 
i.e., \emph{managing code versions} in a simple way: 
one does not actually want to \emph{delete} code, 
one might want to use it another time, maybe for debugging
$\dots$ or to remind of earlier attempts that should not be tried 
again $\dots$

This is a problem for \emph{high-quality typesetting} of 
documentation. \emph{Code} should be typeset about as you see it on 
the \emph{screen}---\emph{monospaced}, this allows structuring by 
indenting, it is common practice to use a typewriter typeface for 
this. Real \emph{comments} should be typeset in \emph{high quality} as 
usual with \LaTeX. Little dilemmas therefore occur with \emph{``hidden 
code"} (``commented-out"). A comment mark starts the line, but 
obviously it is not really a comment and rather should be typeset 
like code (and otherwise they may break).           %% 2010/03/22
Another problem are comments at the \emph{end} of a 
\emph{code} line. Sometimes they are ``real comments" ('gmdoc' 
supports this style). But sometimes 
this is only another version of ``version management," code 
``commented-out."

I like the style of writing packages described before and use it all 
the time. I mark ``real comments" with \emph{two} adjacent comment 
marks and an ensuing space to distinguish them clearly from code 
commented out.
%% Adapted to v0.4 2010/03/29:
\emph{This style is presently the one supported by \textup{'makedoc'} 
      as default.}
This way only a line starting with 
|%% | is considered a ``real" comment line. The first three 
characters are removed, and the rest is typeset in high quality. 
Any other lines are typeset verbatim. The 'makedoc' \emph{parser} 
doing this has an ``identifier" |PPScomment| (``percent, percent, 
space"). Another identifier |comment| is a placeholder for 
the comment parser to be used, by default it is an alias for 
`PPScomment'. Lines just containing |%%| (without the space) may be 
used to suppress empty code lines preceding section titles and for 
keeping some visual, relieving space between code and comment lines.

The style I described previously may be considered ``unprofessional." 
The many \LaTeX\ packages documented using the 'doc'\slash'.dtx' 
system don't use comment marks for \emph{``commenting-out"}. 
Or one may mark code commented out by putting no space between the 
percent mark and the code. 
With v0.4 of 'makedoc', this style is supported as |PScomment|. 
You can directly call this as <main-parser> as described below, 
or you can switch to it by 
\[`\CopyFDconditionFromTo{PScomment}{comment}'\]

\subsection{Markup in comments}
Packages using the 'doc'\slash '.dtx' system as well as alternative 
highly developed systems mentioned above use (enhanced) usual 
\emph{\LaTeX} syntax for markup of comments. Other packages just use 
an \emph{ASCII} style \emph{without} any markup. My idea was to 
support the latter style by some `txt'$\to$\LaTeX\ functionality. 
'makedoc' does this using a file 'mdoccorr.cfg' which is very small 
right now.

I also thought of introducing another sort of ``decent" markup not 
needing much more space than the ``ASCII kernel" of the comments. 
This is to some extent implemented in 'niceverb.sty'. I thought of the 
syntax of editing \textit{Wikipedia} pages; this is partially 
implemented in 'wiki.sty' which unfortunately is not yet compatible 
with 'niceverb'.

But 'makedoc' implements one \textit{Wikipedia} feature in a different 
way than 'wiki.sty' (cf.~'wikicheat.pdf') that looks about as follows:
\begin{eqnarray*}
  \endcell\endcell`%% == Section =='\\
  \endcell\endcell`%% === Subsection ==='\\
  \endcell\endcell`%% ==== Subsubsection ===='
\end{eqnarray*}
i.e., you type `== <title> ==' in place of `\section{<title>}' etc.
The parser must replace `====<title3>===' before `===<title2>===' and 
the latter before `==<title1>=='. In fact, 'makedoc' provides three 
parsers for these situations:
\begin{description}
\cmdboxitem|\SectionLevelThreeParseInput| is the most general parser 
    offered. If it does not find two strings \lq`===='\rq\ enclosing 
    \emph{something}, it passes to
\cmdboxitem|\SectionLevelTwoParseInput| which unless finding 
    two strings `===' enclosing something passes to
\cmdboxitem|\SectionLevelOneParseInput| $\dots$ passes to the comment 
    detector |comment|. 
\end{description}


\section{Requirements}
'makedoc' requires \LaTeXe\ (supporting star forms of `\newcommand' 
etc.)\ as \TeX-format, the package 'fifinddo.sty' from the same 
directory (on CTAN etc.)\ as where 'makedoc.sty' is, and the 
\LaTeX-package 'moreverb' by Robin Fairbairns (after others)---it 
should be installed anyway, or you can get its latest version 
(v2.3, 2008/06/03?) from CTAN. 

'makedoc''s `.txt'$\to$\TeX\ functionality moreover needs a file 
'mdoccorr.cfg' that should have come along with 'makedoc.sty' and 
'fifinddo.sty'. You may need to have a modified copy of it in the 
directory of your main `.tex' file `<jobname>.tex' fitting special 
needs of your project. 

\section{Using 'makedoc' the simplest way}
In the most simple case, you are preparing documentation for a package 
file `<jobname>.sty' only, and you prepare a file `<jobname>.tex' 
containing 
\[`\title{\textsf{<jobname>}---a \LaTeX\ Package for <whatever>}'\]
and `\maketitle' etc.\ about your package.\footnote{With 'niceverb' 
    and &\title\ after &\begin{document}, you may replace 
    \lq&\textsf{<jobname>}\rq\ by \lq&'<jobname>&'\rq.}
The documentation will be produced by running `<jobname>.tex' with 
\LaTeX\ (e.g., \texttt{latex <jobname>.tex}).

First, `<jobname>.tex' must have |\usepackage{makedoc}| in its preamble. 
There are no package options. 

Second, to typeset the commented implementation from `<jobname>.sty', 
include in <jobname>.tex's `document' environment a line 
\[|\MakeInputJobDoc{<header-lines>}{\SectionLevelThreeParseInput}|\]
<header-lines> refers to a non-negative integer as follows: 
We think the most simple and useful way of typesetting the first lines 
of a package file including license and copyrights is ``depicting them 
as image," i.e., \textit{verbatim}. We could try to determine the 
number of these lines by parsing, but we won't do so soon. Please just 
count them and enter the number as <header-lines>---and change it 
until you can accept the outcome.

\section{Steps of advanced usage}
\subsection{Different main parsers (second mandatory argument)}

`\MakeInputJobDoc''s mandatory syntax actually is 
\[|\MakeInputJobDoc{<header-lines>}{<main-parser>}|\]
<main-parser> refers to the parsing macro that is applied to each 
input line whose number is greater than <header-lines>. 
Examples for <main-parser> are named in section~\ref{sec:styles} above. 
  %% TODO above/below macro 2010/03/15
`\SectionLevelThreeParseInput' is just the most general one. 
For \emph{efficiency} (!? or also to avoid problems?) you may 
replace `Three' by `Two' or by `One', if the `====' or the `===' 
feature is not used in `<jobname>.sty'. If the ``\textit{Wikipedia} 
sectioning" feature is not used at all, use 
\[|\MakeInputJobDoc{<header-lines>}{\ProcessInputWith{comment}}|\]
---provided you want to adopt the \lq`%% '\rq\ style of marking 
comments, cf.~section~\ref{sec:styles}. For the \lq`% '\rq style 
instead, use 
\[|\MakeInputJobDoc{<header-lines>}{\ProcessInputWith{PScomment}}|\]

\subsection{Different extensions (optional arguments)}
If your package to be documented is a \emph{class} `<jobname>.cls', 
a local configuration file `<jobname>.cfg' or something 
else---<jobname>.<ext-in>, e.g., <ext-in>=`cls' or <ext-in>=`cfg', 
use 
\[|\MakeInputJobDoc[<ext-in>]{<header>}{<parser>}|\]
Moreover, `\MakeInputJobDoc' writes an intermediate file 
`<jobname>.doc' and then `\input's it. If you do not like `doc' 
as extension for the written file name (maybe you use 
`<jobname>.doc' for something different already), preferring extension 
<ext-out>, use
\[|\MakeInputJobDoc[<ext-in>][<ext-out>]{<header>}{<parser>}|\]
Yes, you must state <ext-in> then as well, I can't help $\dots$

If even <jobname> is wrong in your view, see next step $\dots$

\subsection{Commands modifying &\MakeInputJobDoc's behaviour}
\label{sec:modimake}
Already <jobname> may not be what you want. E.g., you may want 
to collect documentations of some other files <job-1>, <job-2>, 
$\dots$ in a single <jobname>. Then precede `\MakeInputJobDoc'
with 
\[`\renewcommand*{\mdJobName}{<job-1>}'\]
etc.\ (please reason yourself about additional requirements \dots)
As a matter of fact, `\MakeInputJobDoc' reads 
\[`\mdJobName.<ext-in>' \mbox{\quad and writes\quad} 
  `\mdJobName.<ext-out>'\]
Stated another way, <jobname> above referred to |\mdJobName|.

`\MakeInputJobInput' moreover (by default) produces one dot 
per input line processed on screen to show progress. 
The reason is that `makedoc' issues the command 
|\ProcessLineMessage{\message{.}}|.
Already this trivial thing seems to slow down processing considerably 
(nowadays). `\MakeInputJobInput' will run faster if preceded by 
\[|\ProcessLineMessage{}|\]
which will suppress any message about processing.
However, the message may be helpful in trouble-shooting.

\subsection{Separating preprocessing from typesetting}
  %% extended 2010/03/16
To some surprise, I observe that `\MakeInputJobDoc' \emph{works.} 
This is quite a new discovery of mine (2010/03/13); 
before I thought that, for safety, preprocessing should happen 
inside a local group \emph{preceding} `\documentclass'.
|\MakeJobDoc| works like `\MakeInputJobDoc' described above, 
yet it just \emph{preprocesses} the package to be documented, 
waiting for an 
\[`\input{<jobname>.<ext-out>}'\] 
in the `document' environment to \emph{typeset} the documentation. 
So 'makedoc.tex' once had in its preamble
% \[`{\RequirePackage{makedoc} \MakeJobDoc{<header>}{<parser>}}'\]
% at the top of `<jobname>.tex' and `\input{<jobname>.<out-ext>}' later. 
\begin{eqnarray*}\endcell\endcell
  `{\RequirePackage{makedoc}'\cr    \endcell\endcell
  ` \ProcessLineMessage{}'\cr       \endcell\endcell
  ` \MakeJobDoc{22}{\ProcessInputWith{comment}}}'\cr
                                    \endcell\endcell
  `\documentclass{article}'
\end{eqnarray*} 
I did experience some truth in my earlier safety policy: 
With 'niceverb' ``running," `\MakeJobDoc' cannot (easily) be used 
in the `document' environment. `\MakeInputJobDoc' in fact does some 
'niceverb' switching (provided 'niceverb' has been loaded) 
when making use of `\MakeJobDoc'.
  %% <- verbose to improve line breaks 2010/03/16

Thinking of this ``safety" approach, especially grouping (`{\code}'), 
I had not much cared for compatibility with other packages 
in choosing 'makedoc' macro names. 

\subsection{Other 'makedoc' (and 'fifindo') script commands}
The next script commands may be considered of a lower level than 
`\MakeJobDoc' and `\MakeInputJobDoc', they underlie the latter 
commands. We also list commands from 'fifinddo.sty' that may be useful 
or, indeed, are needed for preparing package documentations. 
This may result in ideas on how do use the script commands for 
different purposes than for preparing package documentations---e.g., 
apply `txt'$\to$\TeX\ preprocessing to arbitrary text files. 

\subsubsection{Choosing parameter values for next preprocessing run}

This actually continues section~\ref{sec:modimake}.

\begin{description}
\cmdboxitem|\ResultFile{<output>}| (from 'fifinddo') 
    determines (and opens through the \TeX\ primitive `\openout') 
    the file <output> which will contain the result of 
    preprocessing the package file.
\cmdboxitem|\LaTeXresultFile{<output>}|---see next section.
\cmdboxitem|\Headerlines{<number>}| determines the <number> of lines 
    starting the input file to be copied \emph{verbatim} 
    (the first mandatory argument of `\MakeJobDoc'). 
\cmdboxitem|\MainDocParser{<parser>}| determines the <parser> 
    as in the \emph{second} mandatory argument of `\MakeJobDoc'.
\end{description}

We are now describing some parameters which rather must be switched 
``manually" instead of being modifiable by comfortable 'makedoc' 
script commands.

With the \emph{``Wikipedia sectioning"} feature, you may change the outcome 
regarding levels. Assume, e.g., the package file has titles along the 
scheme `== <title> ==' only, but these should become 
\emph{subsections} of the ``implementation" section of the 
corresponding `.tex' file. Then 
\[`\renewcommand*{\mdSectionLevelOne}{\string\subsection}'\]
-- see the implementation of the sectioning feature for details. 

There is a command 
\[|\NoEmptyInputLines| \mbox{\quad and a parameter macro\quad}
  |\OnEmptyInputLine|\] 
which is modified by the former. However, I cannot say much about them 
right now, I think they just were a difficult matter that I soon 
decided no longer to think about for a while (cf.\ the 
implementation). About the same holds for the hook |\EveryComment|.

The `txt'$\to$\TeX\ functionality is implemented through a hook 
\[|\MakeDocCorrectHook{<characters>}|\] 
'makedoc' initializes it as an alias of \LaTeX's `\@firstofone', i.e., 
it won't change <characters>. 'mdoccorr.cfg' should redefine it so it 
really ``corrects" <characters>. You might try other definitions of 
`\MakeDocCorrectHook' for different ``correcting" functions.
It should be \emph{noted} that (currently) 
`\MakeDocCorrectHook' must be \emph{expandable}, 'fifinddo.sty' 
provides setup for (expandable) chains of expandable replacements. 
The ``Wikipedia" sectioning feature moreover uses expandable 
trimming (single) surrounding spaces, this might be provided in a more 
general way.\footnote{%% TODO 2010/03/16
    The \ctanpkgref{trimspaces} package 
    has been a \emph{model} for this feature here. It cannot be used 
    directly here because it reads blank spaces as \TeX\ characters with 
    category code 10 while 'makedoc' reads blank spaces as ``other" 
    characters (category code 12) in order to \emph{keep} all blank spaces.}

\subsubsection{``Manual" insertions to the output file}
\begin{description}
\cmdboxitem|\WriteResult{<balanced>}| (from 'fifinddo') writes 
    <balanced> to <output> according to the earlier command 
    `\ResultFile{<output>}'.
\cmdboxitem|\WriteProvides| (from 'fifindo') writes a 
    `\ProvidesFile' line into <output> that declares the file 
    to be generated by 'fifindo'.
\cmdboxitem|\LaTeXresultFile{<output>}| issues 
    `\ResultFile{<output>}' and then writes a 
    `\ProvidesFile' line into <output> that declares the file 
    to be generated by 'makedoc'.
\end{description}

\subsubsection{Processing input and closing output}
\begin{description}
\cmdboxitem|\MakeDoc{<input>}|\hskip 0pt plus 4em
    reads 'mdoccorr.cfg' 
    (for `\MakeDocCorrectHook', see above),
    %% removed \LaTeXresult... 2010/03/17
    copies <number> according to `\HeaderLines' (see above) 
    from <input> into <output> (according to `\ResultFile'), 
    then processes the remaining lines of <input> according 
    to `\MainDocParser' (writing several things to <output>). 
    `\MakeDoc' invokes{\sloppy\par}
\cmdboxitem|\ProcessFileWith{<input>}{<loop-body>}| 
    (from 'fifindo') reads <input> line by line---each one stored as 
    macro |\fdInputLine| and applies <loop-body> to it. 
    \TeX's ``special" character codes (of characters listed in 
    macro `\dospecials') are replaced by 12 (``other") by 
    default---see the 'fifinddo' documentation.
\cmdboxitem|\CloseResultFile| (from 'fifinddo') 
    \hskip 0pt plus 1em
    \emph{closes} 
    <output> (using \TeX's primitive `\closeout'). 
\cmdboxitem|\MakeCloseDoc{<input>}| issues 
    `\MakeDoc{<input>}\CloseResultFile'.
\end{description}
%
Using `\MakeDoc' \emph{instead} of `\MakeCloseDoc' allows processing 
additional <input> files writing into the same <output>. Or maybe you 
want to add some additional lines manually to <output> using 
`\WriteResult'.

%% removed 2010/03/09:
% At least in the long run, using 'makedoc' should not imply commitment 
% to a certain design or to certain \LaTeX\ packages for typesetting 
% listings and documentations. Therefore, 'makedoc.cfg' (currently) 
% contains \emph{local} or \emph{personal choices}, but also 
% \emph{experiments} with future features of 'niceverb'. 
% Especially, (at present) the `packagecode' 
% environment that 'makedoc' `\write's must be chosen. 
% Currently this is the `listing' environment from 'moreverb' 
% with some modifications or extra settings. 
% It may be vital to `\MakeOther' the active characters from 'niceverb' 
% in the setup of `packagecode'. See the \emph{example} in 
% section~\ref{sec:fifinddo}.
% 
% Finally, 
% Each package file to be typeset needs its own little
% \emph{script} of 'makedoc' commands. 
% With v0.3, one or two of these should suffice. 

% It should fit into the preamble 
% of the main file for documenting the package (currently %% 2009/04/09
%   just 5 commands seem to suffice, see the \emph{example} in 
%   section~\ref{sec:fifinddo}). 
% The script commands are described 
% in the \dqtd{File handling} section of 'fifinddo.pdf' and in the 
% present section~\ref{sec:script} (and \ref{sec:emptylines}).
% As an alternative, you may prefer to have ``content only" (as much as 
% possible) in the main typesetting file and in its preamble only 
% `\input' a separate script file.
%% removed 2010/03/10:
% Yes, the idea of documenting a package \emph{here} is to have a 
% separate ``driver" file for typsetting the documentation. 
% It may contain an introduction and a guide for users. 
% The documentation of the package code that has been prepared by the 
% 'makedoc' script will be `\input'. Alternatively, the ``driver file" 
% could have title etc.\ only, or preamble and a minimal `document' 
% environment only. 
% 
% So there may be many files, which may look confusing, especially as 
% compared with the 'doc' procedure. However, 
% \begin{enumerate}
% \item ``One file distribution" still is possible thanks to the 
%       `filecontents' environment. 
% \item The 'makedoc' script can create a batch file (fitting the 
%       system, maybe using Will Robertson's 'ifplatform', or 
%       'texsys.cfg', or \dots) 
%       that removes certain auxiliary files or moves them to a 
%       certain directory. 
% \item I find it helpful to have rather little ``contentual" text 
%       in the package file. 
% \item The procedure now runs very smoothly, once the stumbling blocks 
%       have been overcome.\footnote{\hspace{1sp}%% TODO help in 'niceverb'!
%         'niceverb' v0.1 was too sloppy with 
%         some things, and self-documentation of 'makedoc.sty' was 
%         difficult---its parsing and that from 'verbatim' cannot 
%         distinguish between markup code and typeset code.}
% \end{enumerate}

\section{Examples}%%% (documentation of 'mdoccorr.cfg')}
%% moved here 2010/03/23
\subsection{'nicetext', especially 'mdoccorr.cfg'}
The documentations of 'fifinddo', 'makedoc', and 'niceverb' 
themselves are typeset using 'makedoc'.
'fifinddo.pdf' documents 'fifinddo.sty', typeset 
from 'fifinddo.tex', likewise 'makedoc.pdf' and 'niceverb.pdf'. 
% Section~\ref{sec:fifinddo} contains a listing of 
% 'makedoc.cfg' and 
% the 'makedoc' script file 'mkfddoc.tex' 
% especially made for 'fifinddo.pdf'. 
% 'fifinddo.doc', 'makedoc.doc', and 'niceverb.doc' are the \TeX\ input 
% files that were made with 'makedoc.sty'---I have only looked at them 
% when something was wrong (often syntax mistakes in typing). 
The Wikipedia syntax feature 
\begin{quote}
  `%% === subsection ===' 
\end{quote}
is used in 'fifinddo.sty' and 'niceverb.sty' only.

Along with 'makedoc' should come files `makedoc.tpl'---a template 
'makedoc' script, and a file `fdtxttex.tex' that should start a dialogue 
on trying `\MakeDocCorrectHook' if you can manage to run it ('WinShell'?). 
With other definitions of `\MakeDocCorrectHook'---see below---you can 
use this dialogue for arbitrary replacement jobs (as an application of 
'fifinddo' rather than 'makedoc').{\sloppy\par}

'fifinddo.pdf', 'makedoc.pdf', and 'niceverb.pdf' were typeset with the following 
typographical corrections in 'mdoccorr.cfg' that defines 
`\MakeDocCorrectHook':
\strut\hrule
\begingroup 
  \hfuzz=\textwidth \advance \hfuzz by 28pt
  \MakeOther\|\MakeOther\`\MakeOther\'\MakeOther\<
  \listinginput[5]{1}{mdoccorr.cfg}
\endgroup
\hrule\noindent\strut
This code also exemplifies the syntax 'niceverb' provides for writing 
about \LaTeX\ macros. It is typeset here with 'makedoc.sty' 
and then looks thus:
%  \sloppy %% 2010/03/29
\strut\hrule
\renewcommand*{\mdJobName}{mdoccorr}
\MakeInputJobDoc[cfg]{0}{\ProcessInputWith{comment}}
\hrule \noindent\strut
And this is the content of the intermediate generated file:
\hrule
\begingroup 
  \hfuzz=\textwidth \advance \hfuzz by 28pt
  \MakeOther\|\MakeOther\`\MakeOther\'\MakeOther\<
  \listinginput[5]{1}{mdoccorr.doc}
\endgroup
\hrule 
%  \fussy %% 2010/03/29

\subsection{Packages from other authors}
`substr.tex' should typeset a nicely formatted documentation 
of Harald Har\-der's 'substr.sty', see my own result `substr.pdf'. 
'substr.sty' is a rare case of the \lq`%% '\rq\ commenting style 
that 'nicetext' has used itself.

`arseneau.tex' should typeset nicely formatted documentations 
of a few packages by Donald Arseneau, see my own result `arseneau.pdf'. 
This demonstrates the usual \lq`% '\rq\ commenting style 
that 'makedoc' supports with v0.4.

\pagebreak              %% 2010/03/29
\ResetCodeLineNumbers   %% 2010/03/29
\section{Implementation}
\subsection{Preliminaries} 
Head of file (Legalese):
\sloppy
\renewcommand*{\mdJobName}{makedoc}
\ProcessLineMessage{}
\MakeInputJobDoc{22}{\ProcessInputWith{comment}}
The previous empty code line is the one \TeX\ insists to add at every 
end of a file it writes. %% todo TeXbook where? 2009/04/08

%% removed (TODO) 2010/03/15:
% \section{Examples: documentation of 'fifinddo'}
% \label{sec:fifinddo} 
%% removed 2010/03/10:
% \subsection{'makedoc.cfg'} 
% 'fifinddo.pdf' and 'makedoc.pdf' were typeset with the following 
% configuration file 'makedoc.cfg':
% \begingroup \MakeOther\|\MakeOther\`\MakeOther\'\MakeOther\<
%   %% <- TODO should be 'niceverb' command 2009/04/08
%   \listinginput[5]{1}{makedoc.cfg}
% \endgroup
%
%% TODO 'niceverb' example---to 'niceverb.tex'!? 2010/03/15
% \subsection{'mkfddoc.tex'}
% 'fifinddo.pdf' was typeset with the following 'makedoc' script file 
% 'mkfddoc.tex':
% \begingroup 
%   \MakeOther\|\MakeOther\`\MakeOther\'\MakeOther\<
%   \listinginput[5]{1}{mkfddoc.tex}
% \endgroup
% 
%
\end{document}

2009/04/12  more on examples
2009/04/15  exemplification of niceverb.sty by mdcorr.cfg
2009/04/21  === subsubsection -> === subsection
2010/03/08  moved `only' for better line break
2010/03/09  removed something from "Basics"
2010/03/10  more changes in "Basics", pdf stuff to makedoc.cfg, 
            makedoc.cfg no longer example; CodeDoc
2010/03/11  use \MakeCloseDoc; \hfuzz with \listinginput;
            tracing spurious `Label(s) may have changed'
2010/03/12  tests for hyperref compatibility completed
2010/03/13  use \MakeInputJobDoc; clarified ...; ctan.org/pkg
2010/03/14  updated ``Examples" and abstract; \href...easylatex
2010/03/15  ``styles supported"; abstract: txt->TeX; usage
2010/03/16  more on usage; mdcorr -> mdoccorr
2010/03/17  corr. mistake with \MakeDoc
2010/03/19  '' -> " 
2010/03/20  for niceverb v0.31
2010/03/21  for niceverb v0.32
2010/03/22  "may break"
2010/03/23  \noindent in example, moved, added mdoccorr.doc, 
            makedoc.tpl
2010/03/29  \ResetCodeLineNumbers, 
            examples and explanations for v0.4
2010/03/30  \listfiles test
 %% shared formatting settings
\begin{document}
\maketitle
\begin{abstract}\noindent
'niceverb.sty' provides very decent syntax (through active characters) 
for describing \LaTeX\ packages and the syntax of macros conforming to 
\LaTeX\ syntax conventions.
\end{abstract}
\tableofcontents

  %% TODO table listing of active characters
%% Were tests 2010/03/08:
% \section{Presenting Nasty's `Nasty' ``Nasty'' &\NVerb\ 'niceverb'}
% \section{Presenting \cs{NVerb} 'niceverb'}
\section{Presenting 'niceverb'}
\subsection{Purpose}
% \begin{abstract}\noindent
% The 'nicetext' bundle provides ``minimal" markup 
The 'niceverb' package provides ``minimal" markup for documenting \LaTeX\ 
packages, reducing the number of keystrokes/visible characters needed
% .\,.\,. %%% ... %% TODO nicedots 
(kind of poor man's WYSIWYG).\footnote{``What you see is what you 
  get." Novices are always warned that WYSIWYG is essentially 
  impossible with \LaTeX.} %% TODO UK FAQ 2010/03/11
% One feature---\verb'&\foo'%%% badly self-documenting, `&' fails
It conveniently handles command names in arguments of macros 
such as &\footnote or even of sectioning commands. 
% (`.aux'/`.toc' entries).
% 
% This is done by making some characters active. 
% 'niceverb.sty' thus resembles 'wiki.sty'; both are siblings. 
% \end{abstract}
If you use 'makedoc.sty' additionally, commands for typesetting a 
package's code are inserted automatically (just using \TeX). 
%%% \footnote{Stephan I. B\"ottcher used
%%% 'awk' instead to typeset the documentation of his 'lineno.sty'.} 
As opposed to tools that are rather common on UNIX/Linux, this 
operation should work at any \TeX\ installation, irrespective of 
platform.

Both packages may at least be useful while working at a very new package 
and may suffice with small, simple packages. After having edited your 
package's code 
%% <jobname> 2010/02/28:
(typically in a `.sty' file---<jobname>`.sty'), 
you just ``{`latex'}" the manual file 
(maybe some `.tex' file---<jobname>`.tex') 
and get instantly the corresponding updated documentation.

'niceverb' and 'makedoc' may also help to generate without much effort 
documentations of nowadays commonly expected typographical quality for 
packages that so far only had plain text documentations.

\subsection{Acknowledgement/Basic Ideas}
Three ideas of Stephan I. B\"ottcher's in documenting his 
\ctanpkgref{lineno}
inspired the present work: 
\begin{enumerate}
\item 
The markup and its definitions are short and simple, 
markup commands are placed at the right ``margin" 
of the ASCII file, 
so you hardly see them in reading the source file, 
you rather just read the text that will be printed. 
\item 
An 'awk' script removes the `%'s starting \emph{documentation} lines 
and inserts the commands for typesetting the package's \emph{code} 
(you don't see these commands in the source).\footnote{The 
  corresponding part of the ``present work" is 'makedoc.sty'.} 
  %% <- clarified 2010/03/11
\item 
An active character (\lq&|\rq) issues a `\string' \emph{and} switches 
to typewriter typeface for typesetting a command verbatim---so this 
works without changing category codes (which is the usual idea of 
typesetting code), therefore it works even in macro arguments.
\end{enumerate}

\subsection{The Commands and Features of 'niceverb'}
Actually, it is the main purpose of 'niceverb' to save you from 
``commands" $\dots$\par
Single quotes &`, &', ``less than" &< (accompanied 
with `>'), the ``vertical" &|, the hash mark `#', ampersand `&', 
and in an extended ``auto mode" even backslash `\' become `\active'
characters with ``special effects." 
% \qtd{&|$\dots$&|} (i.e., \GenCmdBox+|<code>|+) in general
% should highlight descriptions of user commands and their syntax. 

The package mainly aims at typesetting commands and descriptions of their 
syntax \emph{if the latter is ``standard \LaTeX-like"}, 
using ``meta-variables." A string to be 
typeset ``verbatim" thus is assumed to start with a single command like 
&\foo, maybe followed by stars (\lq`*'\rq) and pairs of 
square brackets (\lq`['<opt-arg>`]'\rq) 
or curly braces (\lq`{'<mand-arg>`}'\rq), 
where those pairs contain strings indicating the typical 
kinds of contents for the respective arguments of that command.
A typical example is this: 
\[\InlineCmdBox{&\foo*[<opt-arg>]{<mand-arg>}}\]
This was achieved by typing 
\[\HardVerbBox+&\foo*[<opt-arg>]{<mand-arg>}+\]
In ``auto mode" of the package, even typing 
\[\HardVerbBox+\foo*[<opt-arg>]{<mand-arg>}+\]
would have sufficed---WYSIWYG! I call such mixtures of 
\emph{verbatim} and ``meta-variables" \textit{\qtd{meta-code}}.

Outside macro arguments, you obtain the same by typing 
% \[\verb+`\foo*[<opt-arg>]{<mand-arg>}'+\]
\[\HardVerbBox+`\foo*[<opt-arg>]{<mand-arg>}'+\]

Details:
\begin{description}

\item[``Meta-variables:"] The package supports the ``angle 
brackets" style of ``meta-variables" (as with <meta-variable>). 
You just type \lq\verb'<bar>'\rq\ to get \lq<bar>\rq.

This works due to a sloppy variant `\NVerb' of `\verb'
which doesn't care about possible ligatures and definitions of active 
characters. Instead, it assumes that the ``verbatim" font doesn't 
contain ligatures anyway.\footnote{On the other hand, &\NVerb is more 
  \emph{careful} with 'niceverb''s special characters.}
\lq\verb'\verb+<foo>+'\rq, by contrast, just yields \lq\verb'<foo>'\rq.

Almost the same feature is offered by 'ltxguide.cls' which formats the 
basic guides from the \LaTeX\ Project Team. The present feature, 
however, also works in plain text outside verbatim mode. 
% On the other hand: without << feature

\item[Single quotes (left/right) for ``short verb:"]
The package ``assumes" that \emph{quoting} refers to 
\emph{code}, therefore \lq\verb+`foo'+\rq\ is typeset as 
\lq`foo'\rq, or (generally) |`<content>'| turns <content> 
into meta-code with the meta-variable feature as above. 
This somewhat resembles the &\MakeShortVerb feature of 'doc.sty'.
%% Moved up here 2010/02/28:
You can ``abuse" our %%% ``single quotes" 
feature just to get typewriter 
typeface.{\sloppy\par}%% not so useful here 2010/02/28:
% \footnote{In macro arguments this requires that the right 
% single quote &' is &\active.}

Problems with this feature will typically arise %%% fail %% 2010/02/28
when you try 
to typeset commands (and their syntax) in \emph{macro arguments}---e.g., 
$$\verb+\footnote{`\bar' is a celebrated fake example!}+$$
will try to \emph{execute} &\bar instead of typesetting it, giving 
an ``undefined" error or so. %% TODO try! 2010/02/28
\verb+\verb+ fails in the same situation, for the same reason. 
\lq\verb+&+\rq\ (&\footnote{&&&\bar<remaining>}) or 
``auto mode" (see below) may then work better.\footnote{&\bar indeed!} 
More generally, the quoting feature still works in macro arguments in 
the sense that you then have to mark difficult characters with `&' 
(simply as short for `\string'). However, it still won't work with 
curly braces that don't follow a command name 
(such \emph{pairs} of braces will simply get lost, 
 \emph{single} braces will give errors or so).%%%\footnote{`{group}'}

Double quotes and apostrophes should still work the usual way.
% %% TODO doesn't work, inside runs into `}' 2010/02/28:
% otherwise you could control the parsing mechanisms using curly braces 
% (outside and inside don't interact: `Harry{'}s' for \qtd{Harry's}).
For difficult cases, you can still use the standard `\verb' 
command from \LaTeX.
To get \emph{usual} single quotes, you can use their standard substitutes 
`\lq' and `\rq', or for pairs of them, 
|\qtd{<text>}| in place of `\lq <text>\rq'---or even `\lq <text>\rq\ '. 

\item[Single right quotes for &\textsf:]
Package names are (by some convention I often yet not always 
 see working) 
typeset with `\textsf'; 
it was natural to use a remaining case of using single quotes 
for abbreviating $$&\textsf{<text>}$$ by |'<text>'|.
% \footnote{%
% Font switching by sequences of single quotes is a feature of the 
% syntax for editing \textit{Wikipedia} pages and of 'wiki.sty'.}
%% <- undoubled 2010/02/28 ->
This idea of switching fonts continues font switching of 'wiki.sty'
which uses the syntax for editing {\it Wikipedia} pages 
(font switching by sequences of right single quotes).

\item[Verticals for setting-off command descriptions:]%%%
\hskip0pt plus 2em
\GenCmdBox+|<code>|+ works like \qtd{&`<code>&'} except putting 
the result into a \emph{framed box} (just as all around 
here)---or something else that you can achieve using some \emph{hooks} 
described with the implementation. There are variants like 
\GenCmdBox+\cmdboxitem|<code>|+.

\item[Ampersand shows command syntax \&c. even in arguments:]
\hfil E.g., type \lq\verb+&\foo{<arg>}+\rq\ to get 
\lq`\foo{<arg>}'\rq. This may be even more convenient for typing than 
the single quotes method, although looking somewhat strange.
However, in macro arguments this does not work with 
\emph{private letters} (`@' and `_' here), for this case, 
use |\cs{<characters>}| or |\cstx{<characters>}<parameters>|.%%%
% `&' may terminate \textit{verbatim} unexpectedly, being designed for 
% displaying ``\LaTeX-like command syntax" in the first instance.
\footnote{Moreover, && currently has a limited 'xspace' 
functionality only.}%%%\footnote{You can even use && for referring to 
%   active characters like && in footnotes etc.!}
%% <- said elsewhere now 2010/03/07

\begin{sloppypar}
This choice of `&' rests on the assumption that there won't be many 
tables in the documenation. You can restore the usual meaning of `&' 
by `\MakeNormal\&' and turn the present special meaning on again by 
\[`\MakeActive\&' \mbox{\quad or\quad } 
  `\MakeActiveLet\&\CmdSyntaxVerb'\]
You could also 
redefine (&\renewcommand) &\descriptionlabel using `\CmdSyntaxVerb' 
(the ``normal command" that is equivalent to `&', its ``permanent 
 alias") 
so \verb+\item[\foo]+ works as wanted.
\end{sloppypar}

\textbf{Another} feature of 'niceverb''s `&' is getting 
(some of the) special characters    %% 2010/03/20
(as listed in the standard macro `\dospecials') verbatim in arguments 
(where `\verb' and the like fail). It just acts similarly as \TeX's 
(as listed in the standard macro `\dospecials') verbatim in arguments 
(where `\verb' and the like fail). It just acts similarly as \TeX's 
 primitive `\string' (which it actually invokes---cf. discussion on the 
 left quote feature above). 

\item[``Auto mode" typesets commands verbatim unless .\,.\,.]
\begin{sloppypar}
In~``auto mode," the backslash \lq`\'\rq\ is an active character that 
builds a command name from the ensuing letters and typesets the 
command (and its syntax, allowing meta-variables) verbatim. 
However, there are some exceptions, which are collected in a macro 
|\niceverbNoVerbList|. &\begin, &\end, and &\item belong to this list, 
you can redefine (`\renewcommand') it, or add <macros> to it by
|\AddToMacro{\niceverbNoVerbList}{<macros>}|
There is also a command |\NormalCommand{<letters>}| \emph{issuing} the 
command `\<letters>' instead of typesetting it.
Since auto mode is somewhat dangerous, you have to start it explicitly 
by |\AutoCmdSyntaxVerb|. You can end it by |\EndAutoCmdSyntaxVerb|.
|\AutoCmdInput{<file>}| is probably most important. 
\end{sloppypar}

Auto mode is motivated by the observation that there are package files 
containing their documentation as pure (well-readable) ASCII 
text---contain\-ing the names of the new commands without any kind of 
quotation marks or verbatim commands. 
Auto mode should typeset such documentation just from the same ASCII 
text.

\item[Hash mark \lq&#\rq\ comes verbatim.]
No macro definitions are expected in the `document' 
environment.\footnote{This idea appeared 2009 on the 'LATEX-L' 
                      mailing list. It may be wrong, 
                      as I have sometimes experienced $\dots$}
                      %% <- changed 2010/03/11
Rather, \lq`#'\rq\ is an active character for taking the next 
character (assuming it is a digit) to form a reference to a 
\emph{macro parameter}---\lq`#1'\rq\ becomes \lq#1\rq\---WYSIWYG 
indeed! (So the general syntax is |#<digit>|.)
\item[Escaping from 'niceverb' (generally).] 
     To get rid of the functionality of some active character <char> 
     (\qtd{&&}, single quote, ampersand, hash mark---not 
      ``auto mode," see above) here, use |\MakeNormal\<char>|---may 
     be within a group. To revive it again, use |\MakeActive\<char>|. 
     This may fail when a different package overtook the active <char> 
     (but I expect more failures then), in this case 
     |\MakeActiveLet\<char>\<perm-alias>| 
     revives the 'niceverb' meaning of <char>
     where `\<perm-alias>' is the ``permanent alias" for that active 
     <char> according to the documentation below. 
     E.g., `\LQverb' is the ``permanent alias" for active single left 
     quote, 'niceverb' activates it by 
     \NVerb+\MakeActiveLet\'\LQverb+.---You can turn off 'niceverb' 
     syntax \emph{alltogether} by |\noNiceVerb| and revive it 
     by |\useNiceVerb| (without ``auto mode").{\sloppy\par}

     \textbf{Right Quotes:} Disabling\slash reviving replacement 
     of `\textsf' by single right quotes requires 
     \[|\nvRightQuoteNormal| \mbox{\quad or\quad } |\nvRightQuoteSansSerif|\] 
     respectively.
\end{description}

\subsection{Examples}
The file 'mdoccorr.cfg' providing some `.txt'$\to$\LaTeX\ 
functionality---i.e., typographical corrections---documents itself 
using 'niceverb' syntax. Its code and the documentation that is 
typeset from it are in the \qtd{examples} section of 
'makedoc.pdf'.---Moreover, 
the documentation 'niceverb.pdf' of 'niceverb.sty' was 
typeset from 'niceverb.tex' and 'niceverb.sty' using 'niceverb' 
syntax, likewise 'fifinddo.pdf' and 'makedoc.pdf'. 
The example of 'niceverb' shows the most frequent use of the `&' 
feature.{\sloppy\par}

'nicetext' bundle release v0.4 contains a file 'substr.tex' 
that should typeset the documentation of the version of 
Harald Harders'
'substr.sty'\footnote{\url{http://ctan.org/pkg/substr}}
that your \TeX\ finds first, as well as 'arseneau.tex' 
typesetting a few packages by Donald Arseneau. 
The outcomes (with me) are 'substr.pdf' and 'arseneau.pdf'.
These are the first applications of 'niceverb''s ``auto mode" to 
(unmodified) third-party package files.
(I also made a more ambitious documentation of Donald Arseneau's 
 'import.sty v3.0' before I found that CTAN already has a nicely 
 typeset documentation of 'import.sty v5.2'.)

%% removed 2010/03/11:
% It seems to me that I could type so many pages on 'fifinddo' and 
% 'makedoc' in little more than a week 
% % (2009/04/12, much of which was needed for debugging and reworking concepts) 
% only due to the ``minimal" \emph{verbatim} and syntax-display syntax. 
% 
\subsection{What is Wrong with the Present Version}
\begin{enumerate}
\item 'niceverb.sty' should be an extension of 'wiki.sty'; 
      yet their font selection mechanisms are currently not compatible. 
      %% 2010/02/28:
      Especially, the feature of \[\hbox\bgroup|''<text>''|\egroup\] 
      %% <- failed with \mbox as of 2010/03/23, first two rq missing 
      %%    2010/03/29
      replacing 
      `\textit{<text>}' or `\emph{<text>}' may be considered missing. 
\item Font switching or horizontal spacing may fail in certain 
      situations.
%       (parentheses, titles, footnotes; 
      You can correct spacing by \lq`\ '\rq. 
        %% <- \qtd{`&\ '}.
% \item 
% The feature of mixing high-quality-typeset comments into the 
% package code listing is implemented in a very rudimentary way only. 
% % just allowing for `\subsection's. 
% The ``comment detector" detects Wikipedia-style subsection titles 
% instead of lines beginning with percent characters.\footnote{%
% Percent characters will definitely not be ``ignored" as with &\DocInput, 
% rather they will hide rests of \emph{documentation} lines as usually, 
% while they will be typeset verbatim in \emph{package code} lines.} 
% Switching between plain and verbatim typesetting in the package 
% listings isn't settled yet, since there are different styles of using 
% percent symbols. I have sometimes used double percent symbols 
% (\lq\verb+%%+\rq) 
% for commenting text and single ones just for ``reversible deletion of 
% code," while usually single percent symbols indicate commenting text 
% indeed. Double percent symbols may, by contrast, mean that the text remains 
% visible in the `.sty' file only, suppressed in the typeset 
% documentation ('lineno.sty').
% For a while, it may be necessary to provide replacing macros for each 
% package separately instead of providing a single macro package 
% managing all of them. 
% \item 
% The code listing currently uses the `listing' and `listingcont' 
% environments of 'moreverb.sty'; 
% the code font and the line numbers may be too large. 
\item The ``vertical" character \qtd{&|} produces inline boxes 
      only at present. It might as well provide a version of the 
      `decl' tabular environment of 'ltxguide.cls'. 
%% changes 2010/03/10
%       coloured\slash framed boxes instead (2009/04/09). They have 
%       their merits! See 'fifinddo.pdf'  and 'makedoc.pdf'. However, 
%       they 
      The inline boxes
      badly deal with long command names and many arguments.
      Doubled verticals could ensure the `decl' mode. 
      Moreover, such a box might issue an \emph{index} entry.
\item One may have \emph{opposite} ideas about using quotes---maybe 
      rather `"<code>"' should typeset <code> \textit{verbatim}.
      There might be a package option for this. If ordinary 
      \qtd{\NVerb'``<text>"'} still should work, awful tricks as now with 
      the right quote feature would be needed. %% TODO 2010/03/06
% \item ``Auto mode" has \emph{not} been tested on a serious application yet. 
%% partially improved 2010/02/28:
% \item % 'niceverb''s font switching tricks sometimes turn against their 
%       % inventor (and other users?). There must be some switching 
%       % ``off'' (and ``on'' again).%
%       %   \footnote{\hspace{1sp}'fifinddo'\slash\hspace{1sp}'makedoc'
%       %     %% <- TODO oh, oh! 2009/04/11
%       %     allow inserting such commands from a driver script, 
%       %     invisible in the file that contains the ``contentual'' 
%       %     documentation.}
%       % Also, there 
%       There
%       might better help with weird errors, 
%       some syntax checks might intercept earlier. 
% 
%       Similarly, some choices reflect a %% rather OK 2010/02/28
%       personal style and should be modifiable, especially by package 
%       options.\footnote{Please sponsor the project or support it 
%         otherwise!}
\item Certain difficulties with typesetting code in macro arguments 
      may be overcome easily using $\varepsilon$\mbox{-}\TeX\ 
      features, I need to find out $\dots$
\end{enumerate}

\section{Implementation of the Markup Syntax}
\typeout{niceverb.tex 2010/04/05 documenting niceverb.sty}
\title{\textsf{niceverb.sty}\\---\\Minimizing 
  Markup\\for Documenting \LaTeX\ packages%%% \thanks{This 
%     manual describes package version
%     version 0.2 as of April 09, 2009%%%\fileversion\ as of \filedate\ 
%     .}}%%%of the package.}%
}
% \listfiles 2010/03/19
{ \RequirePackage{makedoc} \ProcessLineMessage{} %% 2010/03/11
  \MakeJobDoc{19}{\SectionLevelThreeParseInput}  }
\documentclass{article}%% TODO paper dimensions!?
\typeout{makedoc.tex 2010/03/30 documentation for `makedoc.sty'}
\listfiles
\RequirePackage{makedoc}
\documentclass{article}
\typeout{makedoc.tex 2010/03/30 documentation for `makedoc.sty'}
\listfiles
\RequirePackage{makedoc}
\documentclass{article}
\input{makedoc.cfg} %% with pdf stuff and 'niceverb'
%% removed for niceverb v0.31 TODO!? 2010/03/20:
\sfcode`/=1001 %% TODO makedoc.cfg!? 2010/03/21
% \makeatletter %% TEST for hyperref compatibility 2010/03/11
%   \def\@testdef #1#2#3{%
%     \def\reserved@a{#3}%
%     \expandafter \ifx \csname #1@#2\endcsname
%    \reserved@a  \else \@tempswatrue \fi
%     \if@tempswa
%       \typeout{^^J*** Type `r' <input> to get around 
%                       \string\label\space issues! ***^^J}
%       \errorcontextlines=0
%       \show\reserved@a
%       \expandafter \show \csname #1@#2\endcsname
%     \fi
%    }
% \makeatother
\begin{document}
\title{'makedoc'---Preprocessing documentation by \TeX}
  %% 2009/04/10: \\---\\\ breaks TOC
\maketitle
\begin{abstract}\noindent
'makedoc' provides commands for generating \LaTeX\ input from a 
package file in order to typeset the latter's documentation 
(somewhat similar and opposite to 'docstrip')---with 
v0.3 \emph{a single one usually suffices}. 
Certain comment marks are removed, listing commands are inserted, 
and some (configurable) typographical `txt'$\to$\TeX\ corrections 
are applied.---This 
continues the policy of 'niceverb' to minimize documentation markup in 
package files. 'makedoc' extends and exemplifies the parsing package 
'fifinddo'. After an edit (and test) of your package, you get the new 
documentation in one run (or the usual number of runs) of the 
documentation driver file.---The present approach is meant to be an 
\emph{alternative} to the standard 'doc' package and its `\DocInput'. 
It provides \emph{less} than 'doc' does, rather deliberately. It may 
be helpful at least for the development of small packages, or at least 
at early stages.
\end{abstract}
\tableofcontents
\section{Introduction}
\emph{The abstract will not be repeated in this section.} Let me add 
instead that I was in dire need of such a package, I got stuck with my 
packages because I lost orientation in them, and I was unhappy with 
the forms of documentations of my other packages, and documenting them 
with the standard \LaTeX\ 'doc' system was not attractive for me 
(neither considered helpful). %% clarified 2010/03/13
I also worked on \emph{Windows} until September 2008, and I 
find a system like the present one still more attractive then using 
(learning!\@) other filtering utilities (see below on 'awk'). And I 
may work on \emph{Windows} once again and don't want to depend on 
installing some $\dots$ there---\emph{I really would like to have 
powerful tools for everything depending on nothing but \TeX\slash 
\LaTeX!}

\section{Prior work and what is new}
It is, of course, not a new idea to get around comment marks `%' to 
typeset the documentation. 'doc''s `\DocInput' does this by making `%' 
an ``ignored" character. This way you cannot use `%' for commenting 
comments (so 'doc' offers a ``new comment mark" 
`^'`^'`A'). %% TODO `^^A' suddenly failed 2010/03/15 -- "ligature"!?
You also cannot use `%' for commenting out code (that you are 
pondering---or using for debugging---only). %% clarified 2010/03/13

Moreover, 'doc' requires enclosing package code explicitly by 
environment commands (behind comment marks). Stephan I. B\"ottcher 
with his '\href{http://ctan.org/pkg/lineno}{lineno.sty}' 
and Grzegorz Murzynowski in \ctanpkgref{gmdoc}
aimed at doing away with this requirement. 
'lineno.sty' contains 'awk' scripts 
to remove starting comment marks and to insert listing commands. 
A file 'lineno.tex' is generated that typesets the documentation. 
By the way, 'lineno.sty' is full of discussions, but it is not 
'docstrip'ped---the maintainers never have received a complaint 
that inputting 'lineno.sty' were too slow. 

'gmdoc' seems to get around comment marks and insert listing commands 
\emph{while typesetting} by a refined version of `\DocInput', 
through some careful detecting and analysing comment marks, 
the approach resembles detection of lists in 'wiki.sty'.\footnote{See 
  'gmdoc.pdf' on &\DocInput. You can learn a lot from this 220 pages 
  document! I also find 
  \ctanpkgref{pauldoc} and \ctanpkgref{xdoc} inspiring.}
And this is a matter of principles---comparing the approaches of 
\emph{preprocessing} ('lineno.sty') and \emph{``smart typesetting"} 
('gmdoc', 'wiki'). Sometimes preprocessing seems to be simpler, 
sometimes detecting while typesetting. 
(Another example is the preprocessor 
\ctanpkgref{easylatex}
of which 'wiki.sty' is a much reduced ``while typesetting" variant.) 
``While typesetting" may be easier when single characters or 
sequences of two or three encode markup
information---but such detection can badly interfere with other 
packages etc. ``Preprocessing" may be easier when entire ``strings" 
of characters decide, which may be anywhere in a file line. 

'makedoc' chooses \emph{preprocessing}, as 'lineno.sty', but by 
\emph{\TeX}. There is a general discussion of this choice in the 
documentation of 'fifinddo'. Preprocessing here can be done in the 
same \LaTeX\ run as typesetting, though you can avoid 
incompatabilities with packages needed for typesetting 
(by inputting them only \emph{after} preprocessing). 

'lineno.sty' exemplifies why preprocessing with \emph{\TeX} may be 
preferable to preprocessing with other utilities: 
When I took over maintenance of 'lineno.sty', 
I needed hard work to get the 'awk' script running. 
The \emph{Munich} 'awk' seemed \emph{not} to behave as the \emph{Kiel} 
'awk' (I chose a Munich 'nawk' and reworked the script a little). 
\TeX\ seems to have better fixed functionality than other utilities! 

A different alternative to \LaTeX's 'doc' system is 
Paul Isambert's '\href{http://ctan.org/pkg/codedoc}{CodeDoc}' 
where the code environments extract package code in typesetting the 
documentation. %% added 2010/03/10

\section{Styles supported (parsers provided)}\label{sec:styles}
% \section{Styles of commenting '.sty's}
We find different styles of documenting \LaTeX\ packages. 
As the main aspects I consider 
(i)~\emph{telling code from comments} 
and (ii)~\emph{markup in comments}. 
(You may find more details on the next matters in the 
 ``implementation" section.)

\subsection{Telling code from comments}
\emph{Comment marks} (usually \lq`%'\rq\ in the case of \TeX) 
probably were named so to mark \emph{``comments"} as opposed 
to code $\dots$ great, but actually, in ``daily practice," 
they are so handy---and used---for ``commenting out" \emph{code}, 
i.e., \emph{managing code versions} in a simple way: 
one does not actually want to \emph{delete} code, 
one might want to use it another time, maybe for debugging
$\dots$ or to remind of earlier attempts that should not be tried 
again $\dots$

This is a problem for \emph{high-quality typesetting} of 
documentation. \emph{Code} should be typeset about as you see it on 
the \emph{screen}---\emph{monospaced}, this allows structuring by 
indenting, it is common practice to use a typewriter typeface for 
this. Real \emph{comments} should be typeset in \emph{high quality} as 
usual with \LaTeX. Little dilemmas therefore occur with \emph{``hidden 
code"} (``commented-out"). A comment mark starts the line, but 
obviously it is not really a comment and rather should be typeset 
like code (and otherwise they may break).           %% 2010/03/22
Another problem are comments at the \emph{end} of a 
\emph{code} line. Sometimes they are ``real comments" ('gmdoc' 
supports this style). But sometimes 
this is only another version of ``version management," code 
``commented-out."

I like the style of writing packages described before and use it all 
the time. I mark ``real comments" with \emph{two} adjacent comment 
marks and an ensuing space to distinguish them clearly from code 
commented out.
%% Adapted to v0.4 2010/03/29:
\emph{This style is presently the one supported by \textup{'makedoc'} 
      as default.}
This way only a line starting with 
|%% | is considered a ``real" comment line. The first three 
characters are removed, and the rest is typeset in high quality. 
Any other lines are typeset verbatim. The 'makedoc' \emph{parser} 
doing this has an ``identifier" |PPScomment| (``percent, percent, 
space"). Another identifier |comment| is a placeholder for 
the comment parser to be used, by default it is an alias for 
`PPScomment'. Lines just containing |%%| (without the space) may be 
used to suppress empty code lines preceding section titles and for 
keeping some visual, relieving space between code and comment lines.

The style I described previously may be considered ``unprofessional." 
The many \LaTeX\ packages documented using the 'doc'\slash'.dtx' 
system don't use comment marks for \emph{``commenting-out"}. 
Or one may mark code commented out by putting no space between the 
percent mark and the code. 
With v0.4 of 'makedoc', this style is supported as |PScomment|. 
You can directly call this as <main-parser> as described below, 
or you can switch to it by 
\[`\CopyFDconditionFromTo{PScomment}{comment}'\]

\subsection{Markup in comments}
Packages using the 'doc'\slash '.dtx' system as well as alternative 
highly developed systems mentioned above use (enhanced) usual 
\emph{\LaTeX} syntax for markup of comments. Other packages just use 
an \emph{ASCII} style \emph{without} any markup. My idea was to 
support the latter style by some `txt'$\to$\LaTeX\ functionality. 
'makedoc' does this using a file 'mdoccorr.cfg' which is very small 
right now.

I also thought of introducing another sort of ``decent" markup not 
needing much more space than the ``ASCII kernel" of the comments. 
This is to some extent implemented in 'niceverb.sty'. I thought of the 
syntax of editing \textit{Wikipedia} pages; this is partially 
implemented in 'wiki.sty' which unfortunately is not yet compatible 
with 'niceverb'.

But 'makedoc' implements one \textit{Wikipedia} feature in a different 
way than 'wiki.sty' (cf.~'wikicheat.pdf') that looks about as follows:
\begin{eqnarray*}
  \endcell\endcell`%% == Section =='\\
  \endcell\endcell`%% === Subsection ==='\\
  \endcell\endcell`%% ==== Subsubsection ===='
\end{eqnarray*}
i.e., you type `== <title> ==' in place of `\section{<title>}' etc.
The parser must replace `====<title3>===' before `===<title2>===' and 
the latter before `==<title1>=='. In fact, 'makedoc' provides three 
parsers for these situations:
\begin{description}
\cmdboxitem|\SectionLevelThreeParseInput| is the most general parser 
    offered. If it does not find two strings \lq`===='\rq\ enclosing 
    \emph{something}, it passes to
\cmdboxitem|\SectionLevelTwoParseInput| which unless finding 
    two strings `===' enclosing something passes to
\cmdboxitem|\SectionLevelOneParseInput| $\dots$ passes to the comment 
    detector |comment|. 
\end{description}


\section{Requirements}
'makedoc' requires \LaTeXe\ (supporting star forms of `\newcommand' 
etc.)\ as \TeX-format, the package 'fifinddo.sty' from the same 
directory (on CTAN etc.)\ as where 'makedoc.sty' is, and the 
\LaTeX-package 'moreverb' by Robin Fairbairns (after others)---it 
should be installed anyway, or you can get its latest version 
(v2.3, 2008/06/03?) from CTAN. 

'makedoc''s `.txt'$\to$\TeX\ functionality moreover needs a file 
'mdoccorr.cfg' that should have come along with 'makedoc.sty' and 
'fifinddo.sty'. You may need to have a modified copy of it in the 
directory of your main `.tex' file `<jobname>.tex' fitting special 
needs of your project. 

\section{Using 'makedoc' the simplest way}
In the most simple case, you are preparing documentation for a package 
file `<jobname>.sty' only, and you prepare a file `<jobname>.tex' 
containing 
\[`\title{\textsf{<jobname>}---a \LaTeX\ Package for <whatever>}'\]
and `\maketitle' etc.\ about your package.\footnote{With 'niceverb' 
    and &\title\ after &\begin{document}, you may replace 
    \lq&\textsf{<jobname>}\rq\ by \lq&'<jobname>&'\rq.}
The documentation will be produced by running `<jobname>.tex' with 
\LaTeX\ (e.g., \texttt{latex <jobname>.tex}).

First, `<jobname>.tex' must have |\usepackage{makedoc}| in its preamble. 
There are no package options. 

Second, to typeset the commented implementation from `<jobname>.sty', 
include in <jobname>.tex's `document' environment a line 
\[|\MakeInputJobDoc{<header-lines>}{\SectionLevelThreeParseInput}|\]
<header-lines> refers to a non-negative integer as follows: 
We think the most simple and useful way of typesetting the first lines 
of a package file including license and copyrights is ``depicting them 
as image," i.e., \textit{verbatim}. We could try to determine the 
number of these lines by parsing, but we won't do so soon. Please just 
count them and enter the number as <header-lines>---and change it 
until you can accept the outcome.

\section{Steps of advanced usage}
\subsection{Different main parsers (second mandatory argument)}

`\MakeInputJobDoc''s mandatory syntax actually is 
\[|\MakeInputJobDoc{<header-lines>}{<main-parser>}|\]
<main-parser> refers to the parsing macro that is applied to each 
input line whose number is greater than <header-lines>. 
Examples for <main-parser> are named in section~\ref{sec:styles} above. 
  %% TODO above/below macro 2010/03/15
`\SectionLevelThreeParseInput' is just the most general one. 
For \emph{efficiency} (!? or also to avoid problems?) you may 
replace `Three' by `Two' or by `One', if the `====' or the `===' 
feature is not used in `<jobname>.sty'. If the ``\textit{Wikipedia} 
sectioning" feature is not used at all, use 
\[|\MakeInputJobDoc{<header-lines>}{\ProcessInputWith{comment}}|\]
---provided you want to adopt the \lq`%% '\rq\ style of marking 
comments, cf.~section~\ref{sec:styles}. For the \lq`% '\rq style 
instead, use 
\[|\MakeInputJobDoc{<header-lines>}{\ProcessInputWith{PScomment}}|\]

\subsection{Different extensions (optional arguments)}
If your package to be documented is a \emph{class} `<jobname>.cls', 
a local configuration file `<jobname>.cfg' or something 
else---<jobname>.<ext-in>, e.g., <ext-in>=`cls' or <ext-in>=`cfg', 
use 
\[|\MakeInputJobDoc[<ext-in>]{<header>}{<parser>}|\]
Moreover, `\MakeInputJobDoc' writes an intermediate file 
`<jobname>.doc' and then `\input's it. If you do not like `doc' 
as extension for the written file name (maybe you use 
`<jobname>.doc' for something different already), preferring extension 
<ext-out>, use
\[|\MakeInputJobDoc[<ext-in>][<ext-out>]{<header>}{<parser>}|\]
Yes, you must state <ext-in> then as well, I can't help $\dots$

If even <jobname> is wrong in your view, see next step $\dots$

\subsection{Commands modifying &\MakeInputJobDoc's behaviour}
\label{sec:modimake}
Already <jobname> may not be what you want. E.g., you may want 
to collect documentations of some other files <job-1>, <job-2>, 
$\dots$ in a single <jobname>. Then precede `\MakeInputJobDoc'
with 
\[`\renewcommand*{\mdJobName}{<job-1>}'\]
etc.\ (please reason yourself about additional requirements \dots)
As a matter of fact, `\MakeInputJobDoc' reads 
\[`\mdJobName.<ext-in>' \mbox{\quad and writes\quad} 
  `\mdJobName.<ext-out>'\]
Stated another way, <jobname> above referred to |\mdJobName|.

`\MakeInputJobInput' moreover (by default) produces one dot 
per input line processed on screen to show progress. 
The reason is that `makedoc' issues the command 
|\ProcessLineMessage{\message{.}}|.
Already this trivial thing seems to slow down processing considerably 
(nowadays). `\MakeInputJobInput' will run faster if preceded by 
\[|\ProcessLineMessage{}|\]
which will suppress any message about processing.
However, the message may be helpful in trouble-shooting.

\subsection{Separating preprocessing from typesetting}
  %% extended 2010/03/16
To some surprise, I observe that `\MakeInputJobDoc' \emph{works.} 
This is quite a new discovery of mine (2010/03/13); 
before I thought that, for safety, preprocessing should happen 
inside a local group \emph{preceding} `\documentclass'.
|\MakeJobDoc| works like `\MakeInputJobDoc' described above, 
yet it just \emph{preprocesses} the package to be documented, 
waiting for an 
\[`\input{<jobname>.<ext-out>}'\] 
in the `document' environment to \emph{typeset} the documentation. 
So 'makedoc.tex' once had in its preamble
% \[`{\RequirePackage{makedoc} \MakeJobDoc{<header>}{<parser>}}'\]
% at the top of `<jobname>.tex' and `\input{<jobname>.<out-ext>}' later. 
\begin{eqnarray*}\endcell\endcell
  `{\RequirePackage{makedoc}'\cr    \endcell\endcell
  ` \ProcessLineMessage{}'\cr       \endcell\endcell
  ` \MakeJobDoc{22}{\ProcessInputWith{comment}}}'\cr
                                    \endcell\endcell
  `\documentclass{article}'
\end{eqnarray*} 
I did experience some truth in my earlier safety policy: 
With 'niceverb' ``running," `\MakeJobDoc' cannot (easily) be used 
in the `document' environment. `\MakeInputJobDoc' in fact does some 
'niceverb' switching (provided 'niceverb' has been loaded) 
when making use of `\MakeJobDoc'.
  %% <- verbose to improve line breaks 2010/03/16

Thinking of this ``safety" approach, especially grouping (`{\code}'), 
I had not much cared for compatibility with other packages 
in choosing 'makedoc' macro names. 

\subsection{Other 'makedoc' (and 'fifindo') script commands}
The next script commands may be considered of a lower level than 
`\MakeJobDoc' and `\MakeInputJobDoc', they underlie the latter 
commands. We also list commands from 'fifinddo.sty' that may be useful 
or, indeed, are needed for preparing package documentations. 
This may result in ideas on how do use the script commands for 
different purposes than for preparing package documentations---e.g., 
apply `txt'$\to$\TeX\ preprocessing to arbitrary text files. 

\subsubsection{Choosing parameter values for next preprocessing run}

This actually continues section~\ref{sec:modimake}.

\begin{description}
\cmdboxitem|\ResultFile{<output>}| (from 'fifinddo') 
    determines (and opens through the \TeX\ primitive `\openout') 
    the file <output> which will contain the result of 
    preprocessing the package file.
\cmdboxitem|\LaTeXresultFile{<output>}|---see next section.
\cmdboxitem|\Headerlines{<number>}| determines the <number> of lines 
    starting the input file to be copied \emph{verbatim} 
    (the first mandatory argument of `\MakeJobDoc'). 
\cmdboxitem|\MainDocParser{<parser>}| determines the <parser> 
    as in the \emph{second} mandatory argument of `\MakeJobDoc'.
\end{description}

We are now describing some parameters which rather must be switched 
``manually" instead of being modifiable by comfortable 'makedoc' 
script commands.

With the \emph{``Wikipedia sectioning"} feature, you may change the outcome 
regarding levels. Assume, e.g., the package file has titles along the 
scheme `== <title> ==' only, but these should become 
\emph{subsections} of the ``implementation" section of the 
corresponding `.tex' file. Then 
\[`\renewcommand*{\mdSectionLevelOne}{\string\subsection}'\]
-- see the implementation of the sectioning feature for details. 

There is a command 
\[|\NoEmptyInputLines| \mbox{\quad and a parameter macro\quad}
  |\OnEmptyInputLine|\] 
which is modified by the former. However, I cannot say much about them 
right now, I think they just were a difficult matter that I soon 
decided no longer to think about for a while (cf.\ the 
implementation). About the same holds for the hook |\EveryComment|.

The `txt'$\to$\TeX\ functionality is implemented through a hook 
\[|\MakeDocCorrectHook{<characters>}|\] 
'makedoc' initializes it as an alias of \LaTeX's `\@firstofone', i.e., 
it won't change <characters>. 'mdoccorr.cfg' should redefine it so it 
really ``corrects" <characters>. You might try other definitions of 
`\MakeDocCorrectHook' for different ``correcting" functions.
It should be \emph{noted} that (currently) 
`\MakeDocCorrectHook' must be \emph{expandable}, 'fifinddo.sty' 
provides setup for (expandable) chains of expandable replacements. 
The ``Wikipedia" sectioning feature moreover uses expandable 
trimming (single) surrounding spaces, this might be provided in a more 
general way.\footnote{%% TODO 2010/03/16
    The \ctanpkgref{trimspaces} package 
    has been a \emph{model} for this feature here. It cannot be used 
    directly here because it reads blank spaces as \TeX\ characters with 
    category code 10 while 'makedoc' reads blank spaces as ``other" 
    characters (category code 12) in order to \emph{keep} all blank spaces.}

\subsubsection{``Manual" insertions to the output file}
\begin{description}
\cmdboxitem|\WriteResult{<balanced>}| (from 'fifinddo') writes 
    <balanced> to <output> according to the earlier command 
    `\ResultFile{<output>}'.
\cmdboxitem|\WriteProvides| (from 'fifindo') writes a 
    `\ProvidesFile' line into <output> that declares the file 
    to be generated by 'fifindo'.
\cmdboxitem|\LaTeXresultFile{<output>}| issues 
    `\ResultFile{<output>}' and then writes a 
    `\ProvidesFile' line into <output> that declares the file 
    to be generated by 'makedoc'.
\end{description}

\subsubsection{Processing input and closing output}
\begin{description}
\cmdboxitem|\MakeDoc{<input>}|\hskip 0pt plus 4em
    reads 'mdoccorr.cfg' 
    (for `\MakeDocCorrectHook', see above),
    %% removed \LaTeXresult... 2010/03/17
    copies <number> according to `\HeaderLines' (see above) 
    from <input> into <output> (according to `\ResultFile'), 
    then processes the remaining lines of <input> according 
    to `\MainDocParser' (writing several things to <output>). 
    `\MakeDoc' invokes{\sloppy\par}
\cmdboxitem|\ProcessFileWith{<input>}{<loop-body>}| 
    (from 'fifindo') reads <input> line by line---each one stored as 
    macro |\fdInputLine| and applies <loop-body> to it. 
    \TeX's ``special" character codes (of characters listed in 
    macro `\dospecials') are replaced by 12 (``other") by 
    default---see the 'fifinddo' documentation.
\cmdboxitem|\CloseResultFile| (from 'fifinddo') 
    \hskip 0pt plus 1em
    \emph{closes} 
    <output> (using \TeX's primitive `\closeout'). 
\cmdboxitem|\MakeCloseDoc{<input>}| issues 
    `\MakeDoc{<input>}\CloseResultFile'.
\end{description}
%
Using `\MakeDoc' \emph{instead} of `\MakeCloseDoc' allows processing 
additional <input> files writing into the same <output>. Or maybe you 
want to add some additional lines manually to <output> using 
`\WriteResult'.

%% removed 2010/03/09:
% At least in the long run, using 'makedoc' should not imply commitment 
% to a certain design or to certain \LaTeX\ packages for typesetting 
% listings and documentations. Therefore, 'makedoc.cfg' (currently) 
% contains \emph{local} or \emph{personal choices}, but also 
% \emph{experiments} with future features of 'niceverb'. 
% Especially, (at present) the `packagecode' 
% environment that 'makedoc' `\write's must be chosen. 
% Currently this is the `listing' environment from 'moreverb' 
% with some modifications or extra settings. 
% It may be vital to `\MakeOther' the active characters from 'niceverb' 
% in the setup of `packagecode'. See the \emph{example} in 
% section~\ref{sec:fifinddo}.
% 
% Finally, 
% Each package file to be typeset needs its own little
% \emph{script} of 'makedoc' commands. 
% With v0.3, one or two of these should suffice. 

% It should fit into the preamble 
% of the main file for documenting the package (currently %% 2009/04/09
%   just 5 commands seem to suffice, see the \emph{example} in 
%   section~\ref{sec:fifinddo}). 
% The script commands are described 
% in the \dqtd{File handling} section of 'fifinddo.pdf' and in the 
% present section~\ref{sec:script} (and \ref{sec:emptylines}).
% As an alternative, you may prefer to have ``content only" (as much as 
% possible) in the main typesetting file and in its preamble only 
% `\input' a separate script file.
%% removed 2010/03/10:
% Yes, the idea of documenting a package \emph{here} is to have a 
% separate ``driver" file for typsetting the documentation. 
% It may contain an introduction and a guide for users. 
% The documentation of the package code that has been prepared by the 
% 'makedoc' script will be `\input'. Alternatively, the ``driver file" 
% could have title etc.\ only, or preamble and a minimal `document' 
% environment only. 
% 
% So there may be many files, which may look confusing, especially as 
% compared with the 'doc' procedure. However, 
% \begin{enumerate}
% \item ``One file distribution" still is possible thanks to the 
%       `filecontents' environment. 
% \item The 'makedoc' script can create a batch file (fitting the 
%       system, maybe using Will Robertson's 'ifplatform', or 
%       'texsys.cfg', or \dots) 
%       that removes certain auxiliary files or moves them to a 
%       certain directory. 
% \item I find it helpful to have rather little ``contentual" text 
%       in the package file. 
% \item The procedure now runs very smoothly, once the stumbling blocks 
%       have been overcome.\footnote{\hspace{1sp}%% TODO help in 'niceverb'!
%         'niceverb' v0.1 was too sloppy with 
%         some things, and self-documentation of 'makedoc.sty' was 
%         difficult---its parsing and that from 'verbatim' cannot 
%         distinguish between markup code and typeset code.}
% \end{enumerate}

\section{Examples}%%% (documentation of 'mdoccorr.cfg')}
%% moved here 2010/03/23
\subsection{'nicetext', especially 'mdoccorr.cfg'}
The documentations of 'fifinddo', 'makedoc', and 'niceverb' 
themselves are typeset using 'makedoc'.
'fifinddo.pdf' documents 'fifinddo.sty', typeset 
from 'fifinddo.tex', likewise 'makedoc.pdf' and 'niceverb.pdf'. 
% Section~\ref{sec:fifinddo} contains a listing of 
% 'makedoc.cfg' and 
% the 'makedoc' script file 'mkfddoc.tex' 
% especially made for 'fifinddo.pdf'. 
% 'fifinddo.doc', 'makedoc.doc', and 'niceverb.doc' are the \TeX\ input 
% files that were made with 'makedoc.sty'---I have only looked at them 
% when something was wrong (often syntax mistakes in typing). 
The Wikipedia syntax feature 
\begin{quote}
  `%% === subsection ===' 
\end{quote}
is used in 'fifinddo.sty' and 'niceverb.sty' only.

Along with 'makedoc' should come files `makedoc.tpl'---a template 
'makedoc' script, and a file `fdtxttex.tex' that should start a dialogue 
on trying `\MakeDocCorrectHook' if you can manage to run it ('WinShell'?). 
With other definitions of `\MakeDocCorrectHook'---see below---you can 
use this dialogue for arbitrary replacement jobs (as an application of 
'fifinddo' rather than 'makedoc').{\sloppy\par}

'fifinddo.pdf', 'makedoc.pdf', and 'niceverb.pdf' were typeset with the following 
typographical corrections in 'mdoccorr.cfg' that defines 
`\MakeDocCorrectHook':
\strut\hrule
\begingroup 
  \hfuzz=\textwidth \advance \hfuzz by 28pt
  \MakeOther\|\MakeOther\`\MakeOther\'\MakeOther\<
  \listinginput[5]{1}{mdoccorr.cfg}
\endgroup
\hrule\noindent\strut
This code also exemplifies the syntax 'niceverb' provides for writing 
about \LaTeX\ macros. It is typeset here with 'makedoc.sty' 
and then looks thus:
%  \sloppy %% 2010/03/29
\strut\hrule
\renewcommand*{\mdJobName}{mdoccorr}
\MakeInputJobDoc[cfg]{0}{\ProcessInputWith{comment}}
\hrule \noindent\strut
And this is the content of the intermediate generated file:
\hrule
\begingroup 
  \hfuzz=\textwidth \advance \hfuzz by 28pt
  \MakeOther\|\MakeOther\`\MakeOther\'\MakeOther\<
  \listinginput[5]{1}{mdoccorr.doc}
\endgroup
\hrule 
%  \fussy %% 2010/03/29

\subsection{Packages from other authors}
`substr.tex' should typeset a nicely formatted documentation 
of Harald Har\-der's 'substr.sty', see my own result `substr.pdf'. 
'substr.sty' is a rare case of the \lq`%% '\rq\ commenting style 
that 'nicetext' has used itself.

`arseneau.tex' should typeset nicely formatted documentations 
of a few packages by Donald Arseneau, see my own result `arseneau.pdf'. 
This demonstrates the usual \lq`% '\rq\ commenting style 
that 'makedoc' supports with v0.4.

\pagebreak              %% 2010/03/29
\ResetCodeLineNumbers   %% 2010/03/29
\section{Implementation}
\subsection{Preliminaries} 
Head of file (Legalese):
\sloppy
\renewcommand*{\mdJobName}{makedoc}
\ProcessLineMessage{}
\MakeInputJobDoc{22}{\ProcessInputWith{comment}}
The previous empty code line is the one \TeX\ insists to add at every 
end of a file it writes. %% todo TeXbook where? 2009/04/08

%% removed (TODO) 2010/03/15:
% \section{Examples: documentation of 'fifinddo'}
% \label{sec:fifinddo} 
%% removed 2010/03/10:
% \subsection{'makedoc.cfg'} 
% 'fifinddo.pdf' and 'makedoc.pdf' were typeset with the following 
% configuration file 'makedoc.cfg':
% \begingroup \MakeOther\|\MakeOther\`\MakeOther\'\MakeOther\<
%   %% <- TODO should be 'niceverb' command 2009/04/08
%   \listinginput[5]{1}{makedoc.cfg}
% \endgroup
%
%% TODO 'niceverb' example---to 'niceverb.tex'!? 2010/03/15
% \subsection{'mkfddoc.tex'}
% 'fifinddo.pdf' was typeset with the following 'makedoc' script file 
% 'mkfddoc.tex':
% \begingroup 
%   \MakeOther\|\MakeOther\`\MakeOther\'\MakeOther\<
%   \listinginput[5]{1}{mkfddoc.tex}
% \endgroup
% 
%
\end{document}

2009/04/12  more on examples
2009/04/15  exemplification of niceverb.sty by mdcorr.cfg
2009/04/21  === subsubsection -> === subsection
2010/03/08  moved `only' for better line break
2010/03/09  removed something from "Basics"
2010/03/10  more changes in "Basics", pdf stuff to makedoc.cfg, 
            makedoc.cfg no longer example; CodeDoc
2010/03/11  use \MakeCloseDoc; \hfuzz with \listinginput;
            tracing spurious `Label(s) may have changed'
2010/03/12  tests for hyperref compatibility completed
2010/03/13  use \MakeInputJobDoc; clarified ...; ctan.org/pkg
2010/03/14  updated ``Examples" and abstract; \href...easylatex
2010/03/15  ``styles supported"; abstract: txt->TeX; usage
2010/03/16  more on usage; mdcorr -> mdoccorr
2010/03/17  corr. mistake with \MakeDoc
2010/03/19  '' -> " 
2010/03/20  for niceverb v0.31
2010/03/21  for niceverb v0.32
2010/03/22  "may break"
2010/03/23  \noindent in example, moved, added mdoccorr.doc, 
            makedoc.tpl
2010/03/29  \ResetCodeLineNumbers, 
            examples and explanations for v0.4
2010/03/30  \listfiles test
 %% with pdf stuff and 'niceverb'
%% removed for niceverb v0.31 TODO!? 2010/03/20:
\sfcode`/=1001 %% TODO makedoc.cfg!? 2010/03/21
% \makeatletter %% TEST for hyperref compatibility 2010/03/11
%   \def\@testdef #1#2#3{%
%     \def\reserved@a{#3}%
%     \expandafter \ifx \csname #1@#2\endcsname
%    \reserved@a  \else \@tempswatrue \fi
%     \if@tempswa
%       \typeout{^^J*** Type `r' <input> to get around 
%                       \string\label\space issues! ***^^J}
%       \errorcontextlines=0
%       \show\reserved@a
%       \expandafter \show \csname #1@#2\endcsname
%     \fi
%    }
% \makeatother
\begin{document}
\title{'makedoc'---Preprocessing documentation by \TeX}
  %% 2009/04/10: \\---\\\ breaks TOC
\maketitle
\begin{abstract}\noindent
'makedoc' provides commands for generating \LaTeX\ input from a 
package file in order to typeset the latter's documentation 
(somewhat similar and opposite to 'docstrip')---with 
v0.3 \emph{a single one usually suffices}. 
Certain comment marks are removed, listing commands are inserted, 
and some (configurable) typographical `txt'$\to$\TeX\ corrections 
are applied.---This 
continues the policy of 'niceverb' to minimize documentation markup in 
package files. 'makedoc' extends and exemplifies the parsing package 
'fifinddo'. After an edit (and test) of your package, you get the new 
documentation in one run (or the usual number of runs) of the 
documentation driver file.---The present approach is meant to be an 
\emph{alternative} to the standard 'doc' package and its `\DocInput'. 
It provides \emph{less} than 'doc' does, rather deliberately. It may 
be helpful at least for the development of small packages, or at least 
at early stages.
\end{abstract}
\tableofcontents
\section{Introduction}
\emph{The abstract will not be repeated in this section.} Let me add 
instead that I was in dire need of such a package, I got stuck with my 
packages because I lost orientation in them, and I was unhappy with 
the forms of documentations of my other packages, and documenting them 
with the standard \LaTeX\ 'doc' system was not attractive for me 
(neither considered helpful). %% clarified 2010/03/13
I also worked on \emph{Windows} until September 2008, and I 
find a system like the present one still more attractive then using 
(learning!\@) other filtering utilities (see below on 'awk'). And I 
may work on \emph{Windows} once again and don't want to depend on 
installing some $\dots$ there---\emph{I really would like to have 
powerful tools for everything depending on nothing but \TeX\slash 
\LaTeX!}

\section{Prior work and what is new}
It is, of course, not a new idea to get around comment marks `%' to 
typeset the documentation. 'doc''s `\DocInput' does this by making `%' 
an ``ignored" character. This way you cannot use `%' for commenting 
comments (so 'doc' offers a ``new comment mark" 
`^'`^'`A'). %% TODO `^^A' suddenly failed 2010/03/15 -- "ligature"!?
You also cannot use `%' for commenting out code (that you are 
pondering---or using for debugging---only). %% clarified 2010/03/13

Moreover, 'doc' requires enclosing package code explicitly by 
environment commands (behind comment marks). Stephan I. B\"ottcher 
with his '\href{http://ctan.org/pkg/lineno}{lineno.sty}' 
and Grzegorz Murzynowski in \ctanpkgref{gmdoc}
aimed at doing away with this requirement. 
'lineno.sty' contains 'awk' scripts 
to remove starting comment marks and to insert listing commands. 
A file 'lineno.tex' is generated that typesets the documentation. 
By the way, 'lineno.sty' is full of discussions, but it is not 
'docstrip'ped---the maintainers never have received a complaint 
that inputting 'lineno.sty' were too slow. 

'gmdoc' seems to get around comment marks and insert listing commands 
\emph{while typesetting} by a refined version of `\DocInput', 
through some careful detecting and analysing comment marks, 
the approach resembles detection of lists in 'wiki.sty'.\footnote{See 
  'gmdoc.pdf' on &\DocInput. You can learn a lot from this 220 pages 
  document! I also find 
  \ctanpkgref{pauldoc} and \ctanpkgref{xdoc} inspiring.}
And this is a matter of principles---comparing the approaches of 
\emph{preprocessing} ('lineno.sty') and \emph{``smart typesetting"} 
('gmdoc', 'wiki'). Sometimes preprocessing seems to be simpler, 
sometimes detecting while typesetting. 
(Another example is the preprocessor 
\ctanpkgref{easylatex}
of which 'wiki.sty' is a much reduced ``while typesetting" variant.) 
``While typesetting" may be easier when single characters or 
sequences of two or three encode markup
information---but such detection can badly interfere with other 
packages etc. ``Preprocessing" may be easier when entire ``strings" 
of characters decide, which may be anywhere in a file line. 

'makedoc' chooses \emph{preprocessing}, as 'lineno.sty', but by 
\emph{\TeX}. There is a general discussion of this choice in the 
documentation of 'fifinddo'. Preprocessing here can be done in the 
same \LaTeX\ run as typesetting, though you can avoid 
incompatabilities with packages needed for typesetting 
(by inputting them only \emph{after} preprocessing). 

'lineno.sty' exemplifies why preprocessing with \emph{\TeX} may be 
preferable to preprocessing with other utilities: 
When I took over maintenance of 'lineno.sty', 
I needed hard work to get the 'awk' script running. 
The \emph{Munich} 'awk' seemed \emph{not} to behave as the \emph{Kiel} 
'awk' (I chose a Munich 'nawk' and reworked the script a little). 
\TeX\ seems to have better fixed functionality than other utilities! 

A different alternative to \LaTeX's 'doc' system is 
Paul Isambert's '\href{http://ctan.org/pkg/codedoc}{CodeDoc}' 
where the code environments extract package code in typesetting the 
documentation. %% added 2010/03/10

\section{Styles supported (parsers provided)}\label{sec:styles}
% \section{Styles of commenting '.sty's}
We find different styles of documenting \LaTeX\ packages. 
As the main aspects I consider 
(i)~\emph{telling code from comments} 
and (ii)~\emph{markup in comments}. 
(You may find more details on the next matters in the 
 ``implementation" section.)

\subsection{Telling code from comments}
\emph{Comment marks} (usually \lq`%'\rq\ in the case of \TeX) 
probably were named so to mark \emph{``comments"} as opposed 
to code $\dots$ great, but actually, in ``daily practice," 
they are so handy---and used---for ``commenting out" \emph{code}, 
i.e., \emph{managing code versions} in a simple way: 
one does not actually want to \emph{delete} code, 
one might want to use it another time, maybe for debugging
$\dots$ or to remind of earlier attempts that should not be tried 
again $\dots$

This is a problem for \emph{high-quality typesetting} of 
documentation. \emph{Code} should be typeset about as you see it on 
the \emph{screen}---\emph{monospaced}, this allows structuring by 
indenting, it is common practice to use a typewriter typeface for 
this. Real \emph{comments} should be typeset in \emph{high quality} as 
usual with \LaTeX. Little dilemmas therefore occur with \emph{``hidden 
code"} (``commented-out"). A comment mark starts the line, but 
obviously it is not really a comment and rather should be typeset 
like code (and otherwise they may break).           %% 2010/03/22
Another problem are comments at the \emph{end} of a 
\emph{code} line. Sometimes they are ``real comments" ('gmdoc' 
supports this style). But sometimes 
this is only another version of ``version management," code 
``commented-out."

I like the style of writing packages described before and use it all 
the time. I mark ``real comments" with \emph{two} adjacent comment 
marks and an ensuing space to distinguish them clearly from code 
commented out.
%% Adapted to v0.4 2010/03/29:
\emph{This style is presently the one supported by \textup{'makedoc'} 
      as default.}
This way only a line starting with 
|%% | is considered a ``real" comment line. The first three 
characters are removed, and the rest is typeset in high quality. 
Any other lines are typeset verbatim. The 'makedoc' \emph{parser} 
doing this has an ``identifier" |PPScomment| (``percent, percent, 
space"). Another identifier |comment| is a placeholder for 
the comment parser to be used, by default it is an alias for 
`PPScomment'. Lines just containing |%%| (without the space) may be 
used to suppress empty code lines preceding section titles and for 
keeping some visual, relieving space between code and comment lines.

The style I described previously may be considered ``unprofessional." 
The many \LaTeX\ packages documented using the 'doc'\slash'.dtx' 
system don't use comment marks for \emph{``commenting-out"}. 
Or one may mark code commented out by putting no space between the 
percent mark and the code. 
With v0.4 of 'makedoc', this style is supported as |PScomment|. 
You can directly call this as <main-parser> as described below, 
or you can switch to it by 
\[`\CopyFDconditionFromTo{PScomment}{comment}'\]

\subsection{Markup in comments}
Packages using the 'doc'\slash '.dtx' system as well as alternative 
highly developed systems mentioned above use (enhanced) usual 
\emph{\LaTeX} syntax for markup of comments. Other packages just use 
an \emph{ASCII} style \emph{without} any markup. My idea was to 
support the latter style by some `txt'$\to$\LaTeX\ functionality. 
'makedoc' does this using a file 'mdoccorr.cfg' which is very small 
right now.

I also thought of introducing another sort of ``decent" markup not 
needing much more space than the ``ASCII kernel" of the comments. 
This is to some extent implemented in 'niceverb.sty'. I thought of the 
syntax of editing \textit{Wikipedia} pages; this is partially 
implemented in 'wiki.sty' which unfortunately is not yet compatible 
with 'niceverb'.

But 'makedoc' implements one \textit{Wikipedia} feature in a different 
way than 'wiki.sty' (cf.~'wikicheat.pdf') that looks about as follows:
\begin{eqnarray*}
  \endcell\endcell`%% == Section =='\\
  \endcell\endcell`%% === Subsection ==='\\
  \endcell\endcell`%% ==== Subsubsection ===='
\end{eqnarray*}
i.e., you type `== <title> ==' in place of `\section{<title>}' etc.
The parser must replace `====<title3>===' before `===<title2>===' and 
the latter before `==<title1>=='. In fact, 'makedoc' provides three 
parsers for these situations:
\begin{description}
\cmdboxitem|\SectionLevelThreeParseInput| is the most general parser 
    offered. If it does not find two strings \lq`===='\rq\ enclosing 
    \emph{something}, it passes to
\cmdboxitem|\SectionLevelTwoParseInput| which unless finding 
    two strings `===' enclosing something passes to
\cmdboxitem|\SectionLevelOneParseInput| $\dots$ passes to the comment 
    detector |comment|. 
\end{description}


\section{Requirements}
'makedoc' requires \LaTeXe\ (supporting star forms of `\newcommand' 
etc.)\ as \TeX-format, the package 'fifinddo.sty' from the same 
directory (on CTAN etc.)\ as where 'makedoc.sty' is, and the 
\LaTeX-package 'moreverb' by Robin Fairbairns (after others)---it 
should be installed anyway, or you can get its latest version 
(v2.3, 2008/06/03?) from CTAN. 

'makedoc''s `.txt'$\to$\TeX\ functionality moreover needs a file 
'mdoccorr.cfg' that should have come along with 'makedoc.sty' and 
'fifinddo.sty'. You may need to have a modified copy of it in the 
directory of your main `.tex' file `<jobname>.tex' fitting special 
needs of your project. 

\section{Using 'makedoc' the simplest way}
In the most simple case, you are preparing documentation for a package 
file `<jobname>.sty' only, and you prepare a file `<jobname>.tex' 
containing 
\[`\title{\textsf{<jobname>}---a \LaTeX\ Package for <whatever>}'\]
and `\maketitle' etc.\ about your package.\footnote{With 'niceverb' 
    and &\title\ after &\begin{document}, you may replace 
    \lq&\textsf{<jobname>}\rq\ by \lq&'<jobname>&'\rq.}
The documentation will be produced by running `<jobname>.tex' with 
\LaTeX\ (e.g., \texttt{latex <jobname>.tex}).

First, `<jobname>.tex' must have |\usepackage{makedoc}| in its preamble. 
There are no package options. 

Second, to typeset the commented implementation from `<jobname>.sty', 
include in <jobname>.tex's `document' environment a line 
\[|\MakeInputJobDoc{<header-lines>}{\SectionLevelThreeParseInput}|\]
<header-lines> refers to a non-negative integer as follows: 
We think the most simple and useful way of typesetting the first lines 
of a package file including license and copyrights is ``depicting them 
as image," i.e., \textit{verbatim}. We could try to determine the 
number of these lines by parsing, but we won't do so soon. Please just 
count them and enter the number as <header-lines>---and change it 
until you can accept the outcome.

\section{Steps of advanced usage}
\subsection{Different main parsers (second mandatory argument)}

`\MakeInputJobDoc''s mandatory syntax actually is 
\[|\MakeInputJobDoc{<header-lines>}{<main-parser>}|\]
<main-parser> refers to the parsing macro that is applied to each 
input line whose number is greater than <header-lines>. 
Examples for <main-parser> are named in section~\ref{sec:styles} above. 
  %% TODO above/below macro 2010/03/15
`\SectionLevelThreeParseInput' is just the most general one. 
For \emph{efficiency} (!? or also to avoid problems?) you may 
replace `Three' by `Two' or by `One', if the `====' or the `===' 
feature is not used in `<jobname>.sty'. If the ``\textit{Wikipedia} 
sectioning" feature is not used at all, use 
\[|\MakeInputJobDoc{<header-lines>}{\ProcessInputWith{comment}}|\]
---provided you want to adopt the \lq`%% '\rq\ style of marking 
comments, cf.~section~\ref{sec:styles}. For the \lq`% '\rq style 
instead, use 
\[|\MakeInputJobDoc{<header-lines>}{\ProcessInputWith{PScomment}}|\]

\subsection{Different extensions (optional arguments)}
If your package to be documented is a \emph{class} `<jobname>.cls', 
a local configuration file `<jobname>.cfg' or something 
else---<jobname>.<ext-in>, e.g., <ext-in>=`cls' or <ext-in>=`cfg', 
use 
\[|\MakeInputJobDoc[<ext-in>]{<header>}{<parser>}|\]
Moreover, `\MakeInputJobDoc' writes an intermediate file 
`<jobname>.doc' and then `\input's it. If you do not like `doc' 
as extension for the written file name (maybe you use 
`<jobname>.doc' for something different already), preferring extension 
<ext-out>, use
\[|\MakeInputJobDoc[<ext-in>][<ext-out>]{<header>}{<parser>}|\]
Yes, you must state <ext-in> then as well, I can't help $\dots$

If even <jobname> is wrong in your view, see next step $\dots$

\subsection{Commands modifying &\MakeInputJobDoc's behaviour}
\label{sec:modimake}
Already <jobname> may not be what you want. E.g., you may want 
to collect documentations of some other files <job-1>, <job-2>, 
$\dots$ in a single <jobname>. Then precede `\MakeInputJobDoc'
with 
\[`\renewcommand*{\mdJobName}{<job-1>}'\]
etc.\ (please reason yourself about additional requirements \dots)
As a matter of fact, `\MakeInputJobDoc' reads 
\[`\mdJobName.<ext-in>' \mbox{\quad and writes\quad} 
  `\mdJobName.<ext-out>'\]
Stated another way, <jobname> above referred to |\mdJobName|.

`\MakeInputJobInput' moreover (by default) produces one dot 
per input line processed on screen to show progress. 
The reason is that `makedoc' issues the command 
|\ProcessLineMessage{\message{.}}|.
Already this trivial thing seems to slow down processing considerably 
(nowadays). `\MakeInputJobInput' will run faster if preceded by 
\[|\ProcessLineMessage{}|\]
which will suppress any message about processing.
However, the message may be helpful in trouble-shooting.

\subsection{Separating preprocessing from typesetting}
  %% extended 2010/03/16
To some surprise, I observe that `\MakeInputJobDoc' \emph{works.} 
This is quite a new discovery of mine (2010/03/13); 
before I thought that, for safety, preprocessing should happen 
inside a local group \emph{preceding} `\documentclass'.
|\MakeJobDoc| works like `\MakeInputJobDoc' described above, 
yet it just \emph{preprocesses} the package to be documented, 
waiting for an 
\[`\input{<jobname>.<ext-out>}'\] 
in the `document' environment to \emph{typeset} the documentation. 
So 'makedoc.tex' once had in its preamble
% \[`{\RequirePackage{makedoc} \MakeJobDoc{<header>}{<parser>}}'\]
% at the top of `<jobname>.tex' and `\input{<jobname>.<out-ext>}' later. 
\begin{eqnarray*}\endcell\endcell
  `{\RequirePackage{makedoc}'\cr    \endcell\endcell
  ` \ProcessLineMessage{}'\cr       \endcell\endcell
  ` \MakeJobDoc{22}{\ProcessInputWith{comment}}}'\cr
                                    \endcell\endcell
  `\documentclass{article}'
\end{eqnarray*} 
I did experience some truth in my earlier safety policy: 
With 'niceverb' ``running," `\MakeJobDoc' cannot (easily) be used 
in the `document' environment. `\MakeInputJobDoc' in fact does some 
'niceverb' switching (provided 'niceverb' has been loaded) 
when making use of `\MakeJobDoc'.
  %% <- verbose to improve line breaks 2010/03/16

Thinking of this ``safety" approach, especially grouping (`{\code}'), 
I had not much cared for compatibility with other packages 
in choosing 'makedoc' macro names. 

\subsection{Other 'makedoc' (and 'fifindo') script commands}
The next script commands may be considered of a lower level than 
`\MakeJobDoc' and `\MakeInputJobDoc', they underlie the latter 
commands. We also list commands from 'fifinddo.sty' that may be useful 
or, indeed, are needed for preparing package documentations. 
This may result in ideas on how do use the script commands for 
different purposes than for preparing package documentations---e.g., 
apply `txt'$\to$\TeX\ preprocessing to arbitrary text files. 

\subsubsection{Choosing parameter values for next preprocessing run}

This actually continues section~\ref{sec:modimake}.

\begin{description}
\cmdboxitem|\ResultFile{<output>}| (from 'fifinddo') 
    determines (and opens through the \TeX\ primitive `\openout') 
    the file <output> which will contain the result of 
    preprocessing the package file.
\cmdboxitem|\LaTeXresultFile{<output>}|---see next section.
\cmdboxitem|\Headerlines{<number>}| determines the <number> of lines 
    starting the input file to be copied \emph{verbatim} 
    (the first mandatory argument of `\MakeJobDoc'). 
\cmdboxitem|\MainDocParser{<parser>}| determines the <parser> 
    as in the \emph{second} mandatory argument of `\MakeJobDoc'.
\end{description}

We are now describing some parameters which rather must be switched 
``manually" instead of being modifiable by comfortable 'makedoc' 
script commands.

With the \emph{``Wikipedia sectioning"} feature, you may change the outcome 
regarding levels. Assume, e.g., the package file has titles along the 
scheme `== <title> ==' only, but these should become 
\emph{subsections} of the ``implementation" section of the 
corresponding `.tex' file. Then 
\[`\renewcommand*{\mdSectionLevelOne}{\string\subsection}'\]
-- see the implementation of the sectioning feature for details. 

There is a command 
\[|\NoEmptyInputLines| \mbox{\quad and a parameter macro\quad}
  |\OnEmptyInputLine|\] 
which is modified by the former. However, I cannot say much about them 
right now, I think they just were a difficult matter that I soon 
decided no longer to think about for a while (cf.\ the 
implementation). About the same holds for the hook |\EveryComment|.

The `txt'$\to$\TeX\ functionality is implemented through a hook 
\[|\MakeDocCorrectHook{<characters>}|\] 
'makedoc' initializes it as an alias of \LaTeX's `\@firstofone', i.e., 
it won't change <characters>. 'mdoccorr.cfg' should redefine it so it 
really ``corrects" <characters>. You might try other definitions of 
`\MakeDocCorrectHook' for different ``correcting" functions.
It should be \emph{noted} that (currently) 
`\MakeDocCorrectHook' must be \emph{expandable}, 'fifinddo.sty' 
provides setup for (expandable) chains of expandable replacements. 
The ``Wikipedia" sectioning feature moreover uses expandable 
trimming (single) surrounding spaces, this might be provided in a more 
general way.\footnote{%% TODO 2010/03/16
    The \ctanpkgref{trimspaces} package 
    has been a \emph{model} for this feature here. It cannot be used 
    directly here because it reads blank spaces as \TeX\ characters with 
    category code 10 while 'makedoc' reads blank spaces as ``other" 
    characters (category code 12) in order to \emph{keep} all blank spaces.}

\subsubsection{``Manual" insertions to the output file}
\begin{description}
\cmdboxitem|\WriteResult{<balanced>}| (from 'fifinddo') writes 
    <balanced> to <output> according to the earlier command 
    `\ResultFile{<output>}'.
\cmdboxitem|\WriteProvides| (from 'fifindo') writes a 
    `\ProvidesFile' line into <output> that declares the file 
    to be generated by 'fifindo'.
\cmdboxitem|\LaTeXresultFile{<output>}| issues 
    `\ResultFile{<output>}' and then writes a 
    `\ProvidesFile' line into <output> that declares the file 
    to be generated by 'makedoc'.
\end{description}

\subsubsection{Processing input and closing output}
\begin{description}
\cmdboxitem|\MakeDoc{<input>}|\hskip 0pt plus 4em
    reads 'mdoccorr.cfg' 
    (for `\MakeDocCorrectHook', see above),
    %% removed \LaTeXresult... 2010/03/17
    copies <number> according to `\HeaderLines' (see above) 
    from <input> into <output> (according to `\ResultFile'), 
    then processes the remaining lines of <input> according 
    to `\MainDocParser' (writing several things to <output>). 
    `\MakeDoc' invokes{\sloppy\par}
\cmdboxitem|\ProcessFileWith{<input>}{<loop-body>}| 
    (from 'fifindo') reads <input> line by line---each one stored as 
    macro |\fdInputLine| and applies <loop-body> to it. 
    \TeX's ``special" character codes (of characters listed in 
    macro `\dospecials') are replaced by 12 (``other") by 
    default---see the 'fifinddo' documentation.
\cmdboxitem|\CloseResultFile| (from 'fifinddo') 
    \hskip 0pt plus 1em
    \emph{closes} 
    <output> (using \TeX's primitive `\closeout'). 
\cmdboxitem|\MakeCloseDoc{<input>}| issues 
    `\MakeDoc{<input>}\CloseResultFile'.
\end{description}
%
Using `\MakeDoc' \emph{instead} of `\MakeCloseDoc' allows processing 
additional <input> files writing into the same <output>. Or maybe you 
want to add some additional lines manually to <output> using 
`\WriteResult'.

%% removed 2010/03/09:
% At least in the long run, using 'makedoc' should not imply commitment 
% to a certain design or to certain \LaTeX\ packages for typesetting 
% listings and documentations. Therefore, 'makedoc.cfg' (currently) 
% contains \emph{local} or \emph{personal choices}, but also 
% \emph{experiments} with future features of 'niceverb'. 
% Especially, (at present) the `packagecode' 
% environment that 'makedoc' `\write's must be chosen. 
% Currently this is the `listing' environment from 'moreverb' 
% with some modifications or extra settings. 
% It may be vital to `\MakeOther' the active characters from 'niceverb' 
% in the setup of `packagecode'. See the \emph{example} in 
% section~\ref{sec:fifinddo}.
% 
% Finally, 
% Each package file to be typeset needs its own little
% \emph{script} of 'makedoc' commands. 
% With v0.3, one or two of these should suffice. 

% It should fit into the preamble 
% of the main file for documenting the package (currently %% 2009/04/09
%   just 5 commands seem to suffice, see the \emph{example} in 
%   section~\ref{sec:fifinddo}). 
% The script commands are described 
% in the \dqtd{File handling} section of 'fifinddo.pdf' and in the 
% present section~\ref{sec:script} (and \ref{sec:emptylines}).
% As an alternative, you may prefer to have ``content only" (as much as 
% possible) in the main typesetting file and in its preamble only 
% `\input' a separate script file.
%% removed 2010/03/10:
% Yes, the idea of documenting a package \emph{here} is to have a 
% separate ``driver" file for typsetting the documentation. 
% It may contain an introduction and a guide for users. 
% The documentation of the package code that has been prepared by the 
% 'makedoc' script will be `\input'. Alternatively, the ``driver file" 
% could have title etc.\ only, or preamble and a minimal `document' 
% environment only. 
% 
% So there may be many files, which may look confusing, especially as 
% compared with the 'doc' procedure. However, 
% \begin{enumerate}
% \item ``One file distribution" still is possible thanks to the 
%       `filecontents' environment. 
% \item The 'makedoc' script can create a batch file (fitting the 
%       system, maybe using Will Robertson's 'ifplatform', or 
%       'texsys.cfg', or \dots) 
%       that removes certain auxiliary files or moves them to a 
%       certain directory. 
% \item I find it helpful to have rather little ``contentual" text 
%       in the package file. 
% \item The procedure now runs very smoothly, once the stumbling blocks 
%       have been overcome.\footnote{\hspace{1sp}%% TODO help in 'niceverb'!
%         'niceverb' v0.1 was too sloppy with 
%         some things, and self-documentation of 'makedoc.sty' was 
%         difficult---its parsing and that from 'verbatim' cannot 
%         distinguish between markup code and typeset code.}
% \end{enumerate}

\section{Examples}%%% (documentation of 'mdoccorr.cfg')}
%% moved here 2010/03/23
\subsection{'nicetext', especially 'mdoccorr.cfg'}
The documentations of 'fifinddo', 'makedoc', and 'niceverb' 
themselves are typeset using 'makedoc'.
'fifinddo.pdf' documents 'fifinddo.sty', typeset 
from 'fifinddo.tex', likewise 'makedoc.pdf' and 'niceverb.pdf'. 
% Section~\ref{sec:fifinddo} contains a listing of 
% 'makedoc.cfg' and 
% the 'makedoc' script file 'mkfddoc.tex' 
% especially made for 'fifinddo.pdf'. 
% 'fifinddo.doc', 'makedoc.doc', and 'niceverb.doc' are the \TeX\ input 
% files that were made with 'makedoc.sty'---I have only looked at them 
% when something was wrong (often syntax mistakes in typing). 
The Wikipedia syntax feature 
\begin{quote}
  `%% === subsection ===' 
\end{quote}
is used in 'fifinddo.sty' and 'niceverb.sty' only.

Along with 'makedoc' should come files `makedoc.tpl'---a template 
'makedoc' script, and a file `fdtxttex.tex' that should start a dialogue 
on trying `\MakeDocCorrectHook' if you can manage to run it ('WinShell'?). 
With other definitions of `\MakeDocCorrectHook'---see below---you can 
use this dialogue for arbitrary replacement jobs (as an application of 
'fifinddo' rather than 'makedoc').{\sloppy\par}

'fifinddo.pdf', 'makedoc.pdf', and 'niceverb.pdf' were typeset with the following 
typographical corrections in 'mdoccorr.cfg' that defines 
`\MakeDocCorrectHook':
\strut\hrule
\begingroup 
  \hfuzz=\textwidth \advance \hfuzz by 28pt
  \MakeOther\|\MakeOther\`\MakeOther\'\MakeOther\<
  \listinginput[5]{1}{mdoccorr.cfg}
\endgroup
\hrule\noindent\strut
This code also exemplifies the syntax 'niceverb' provides for writing 
about \LaTeX\ macros. It is typeset here with 'makedoc.sty' 
and then looks thus:
%  \sloppy %% 2010/03/29
\strut\hrule
\renewcommand*{\mdJobName}{mdoccorr}
\MakeInputJobDoc[cfg]{0}{\ProcessInputWith{comment}}
\hrule \noindent\strut
And this is the content of the intermediate generated file:
\hrule
\begingroup 
  \hfuzz=\textwidth \advance \hfuzz by 28pt
  \MakeOther\|\MakeOther\`\MakeOther\'\MakeOther\<
  \listinginput[5]{1}{mdoccorr.doc}
\endgroup
\hrule 
%  \fussy %% 2010/03/29

\subsection{Packages from other authors}
`substr.tex' should typeset a nicely formatted documentation 
of Harald Har\-der's 'substr.sty', see my own result `substr.pdf'. 
'substr.sty' is a rare case of the \lq`%% '\rq\ commenting style 
that 'nicetext' has used itself.

`arseneau.tex' should typeset nicely formatted documentations 
of a few packages by Donald Arseneau, see my own result `arseneau.pdf'. 
This demonstrates the usual \lq`% '\rq\ commenting style 
that 'makedoc' supports with v0.4.

\pagebreak              %% 2010/03/29
\ResetCodeLineNumbers   %% 2010/03/29
\section{Implementation}
\subsection{Preliminaries} 
Head of file (Legalese):
\sloppy
\renewcommand*{\mdJobName}{makedoc}
\ProcessLineMessage{}
\MakeInputJobDoc{22}{\ProcessInputWith{comment}}
The previous empty code line is the one \TeX\ insists to add at every 
end of a file it writes. %% todo TeXbook where? 2009/04/08

%% removed (TODO) 2010/03/15:
% \section{Examples: documentation of 'fifinddo'}
% \label{sec:fifinddo} 
%% removed 2010/03/10:
% \subsection{'makedoc.cfg'} 
% 'fifinddo.pdf' and 'makedoc.pdf' were typeset with the following 
% configuration file 'makedoc.cfg':
% \begingroup \MakeOther\|\MakeOther\`\MakeOther\'\MakeOther\<
%   %% <- TODO should be 'niceverb' command 2009/04/08
%   \listinginput[5]{1}{makedoc.cfg}
% \endgroup
%
%% TODO 'niceverb' example---to 'niceverb.tex'!? 2010/03/15
% \subsection{'mkfddoc.tex'}
% 'fifinddo.pdf' was typeset with the following 'makedoc' script file 
% 'mkfddoc.tex':
% \begingroup 
%   \MakeOther\|\MakeOther\`\MakeOther\'\MakeOther\<
%   \listinginput[5]{1}{mkfddoc.tex}
% \endgroup
% 
%
\end{document}

2009/04/12  more on examples
2009/04/15  exemplification of niceverb.sty by mdcorr.cfg
2009/04/21  === subsubsection -> === subsection
2010/03/08  moved `only' for better line break
2010/03/09  removed something from "Basics"
2010/03/10  more changes in "Basics", pdf stuff to makedoc.cfg, 
            makedoc.cfg no longer example; CodeDoc
2010/03/11  use \MakeCloseDoc; \hfuzz with \listinginput;
            tracing spurious `Label(s) may have changed'
2010/03/12  tests for hyperref compatibility completed
2010/03/13  use \MakeInputJobDoc; clarified ...; ctan.org/pkg
2010/03/14  updated ``Examples" and abstract; \href...easylatex
2010/03/15  ``styles supported"; abstract: txt->TeX; usage
2010/03/16  more on usage; mdcorr -> mdoccorr
2010/03/17  corr. mistake with \MakeDoc
2010/03/19  '' -> " 
2010/03/20  for niceverb v0.31
2010/03/21  for niceverb v0.32
2010/03/22  "may break"
2010/03/23  \noindent in example, moved, added mdoccorr.doc, 
            makedoc.tpl
2010/03/29  \ResetCodeLineNumbers, 
            examples and explanations for v0.4
2010/03/30  \listfiles test
 %% shared formatting settings
\begin{document}
\maketitle
\begin{abstract}\noindent
'niceverb.sty' provides very decent syntax (through active characters) 
for describing \LaTeX\ packages and the syntax of macros conforming to 
\LaTeX\ syntax conventions.
\end{abstract}
\tableofcontents

  %% TODO table listing of active characters
%% Were tests 2010/03/08:
% \section{Presenting Nasty's `Nasty' ``Nasty'' &\NVerb\ 'niceverb'}
% \section{Presenting \cs{NVerb} 'niceverb'}
\section{Presenting 'niceverb'}
\subsection{Purpose}
% \begin{abstract}\noindent
% The 'nicetext' bundle provides ``minimal" markup 
The 'niceverb' package provides ``minimal" markup for documenting \LaTeX\ 
packages, reducing the number of keystrokes/visible characters needed
% .\,.\,. %%% ... %% TODO nicedots 
(kind of poor man's WYSIWYG).\footnote{``What you see is what you 
  get." Novices are always warned that WYSIWYG is essentially 
  impossible with \LaTeX.} %% TODO UK FAQ 2010/03/11
% One feature---\verb'&\foo'%%% badly self-documenting, `&' fails
It conveniently handles command names in arguments of macros 
such as &\footnote or even of sectioning commands. 
% (`.aux'/`.toc' entries).
% 
% This is done by making some characters active. 
% 'niceverb.sty' thus resembles 'wiki.sty'; both are siblings. 
% \end{abstract}
If you use 'makedoc.sty' additionally, commands for typesetting a 
package's code are inserted automatically (just using \TeX). 
%%% \footnote{Stephan I. B\"ottcher used
%%% 'awk' instead to typeset the documentation of his 'lineno.sty'.} 
As opposed to tools that are rather common on UNIX/Linux, this 
operation should work at any \TeX\ installation, irrespective of 
platform.

Both packages may at least be useful while working at a very new package 
and may suffice with small, simple packages. After having edited your 
package's code 
%% <jobname> 2010/02/28:
(typically in a `.sty' file---<jobname>`.sty'), 
you just ``{`latex'}" the manual file 
(maybe some `.tex' file---<jobname>`.tex') 
and get instantly the corresponding updated documentation.

'niceverb' and 'makedoc' may also help to generate without much effort 
documentations of nowadays commonly expected typographical quality for 
packages that so far only had plain text documentations.

\subsection{Acknowledgement/Basic Ideas}
Three ideas of Stephan I. B\"ottcher's in documenting his 
\ctanpkgref{lineno}
inspired the present work: 
\begin{enumerate}
\item 
The markup and its definitions are short and simple, 
markup commands are placed at the right ``margin" 
of the ASCII file, 
so you hardly see them in reading the source file, 
you rather just read the text that will be printed. 
\item 
An 'awk' script removes the `%'s starting \emph{documentation} lines 
and inserts the commands for typesetting the package's \emph{code} 
(you don't see these commands in the source).\footnote{The 
  corresponding part of the ``present work" is 'makedoc.sty'.} 
  %% <- clarified 2010/03/11
\item 
An active character (\lq&|\rq) issues a `\string' \emph{and} switches 
to typewriter typeface for typesetting a command verbatim---so this 
works without changing category codes (which is the usual idea of 
typesetting code), therefore it works even in macro arguments.
\end{enumerate}

\subsection{The Commands and Features of 'niceverb'}
Actually, it is the main purpose of 'niceverb' to save you from 
``commands" $\dots$\par
Single quotes &`, &', ``less than" &< (accompanied 
with `>'), the ``vertical" &|, the hash mark `#', ampersand `&', 
and in an extended ``auto mode" even backslash `\' become `\active'
characters with ``special effects." 
% \qtd{&|$\dots$&|} (i.e., \GenCmdBox+|<code>|+) in general
% should highlight descriptions of user commands and their syntax. 

The package mainly aims at typesetting commands and descriptions of their 
syntax \emph{if the latter is ``standard \LaTeX-like"}, 
using ``meta-variables." A string to be 
typeset ``verbatim" thus is assumed to start with a single command like 
&\foo, maybe followed by stars (\lq`*'\rq) and pairs of 
square brackets (\lq`['<opt-arg>`]'\rq) 
or curly braces (\lq`{'<mand-arg>`}'\rq), 
where those pairs contain strings indicating the typical 
kinds of contents for the respective arguments of that command.
A typical example is this: 
\[\InlineCmdBox{&\foo*[<opt-arg>]{<mand-arg>}}\]
This was achieved by typing 
\[\HardVerbBox+&\foo*[<opt-arg>]{<mand-arg>}+\]
In ``auto mode" of the package, even typing 
\[\HardVerbBox+\foo*[<opt-arg>]{<mand-arg>}+\]
would have sufficed---WYSIWYG! I call such mixtures of 
\emph{verbatim} and ``meta-variables" \textit{\qtd{meta-code}}.

Outside macro arguments, you obtain the same by typing 
% \[\verb+`\foo*[<opt-arg>]{<mand-arg>}'+\]
\[\HardVerbBox+`\foo*[<opt-arg>]{<mand-arg>}'+\]

Details:
\begin{description}

\item[``Meta-variables:"] The package supports the ``angle 
brackets" style of ``meta-variables" (as with <meta-variable>). 
You just type \lq\verb'<bar>'\rq\ to get \lq<bar>\rq.

This works due to a sloppy variant `\NVerb' of `\verb'
which doesn't care about possible ligatures and definitions of active 
characters. Instead, it assumes that the ``verbatim" font doesn't 
contain ligatures anyway.\footnote{On the other hand, &\NVerb is more 
  \emph{careful} with 'niceverb''s special characters.}
\lq\verb'\verb+<foo>+'\rq, by contrast, just yields \lq\verb'<foo>'\rq.

Almost the same feature is offered by 'ltxguide.cls' which formats the 
basic guides from the \LaTeX\ Project Team. The present feature, 
however, also works in plain text outside verbatim mode. 
% On the other hand: without << feature

\item[Single quotes (left/right) for ``short verb:"]
The package ``assumes" that \emph{quoting} refers to 
\emph{code}, therefore \lq\verb+`foo'+\rq\ is typeset as 
\lq`foo'\rq, or (generally) |`<content>'| turns <content> 
into meta-code with the meta-variable feature as above. 
This somewhat resembles the &\MakeShortVerb feature of 'doc.sty'.
%% Moved up here 2010/02/28:
You can ``abuse" our %%% ``single quotes" 
feature just to get typewriter 
typeface.{\sloppy\par}%% not so useful here 2010/02/28:
% \footnote{In macro arguments this requires that the right 
% single quote &' is &\active.}

Problems with this feature will typically arise %%% fail %% 2010/02/28
when you try 
to typeset commands (and their syntax) in \emph{macro arguments}---e.g., 
$$\verb+\footnote{`\bar' is a celebrated fake example!}+$$
will try to \emph{execute} &\bar instead of typesetting it, giving 
an ``undefined" error or so. %% TODO try! 2010/02/28
\verb+\verb+ fails in the same situation, for the same reason. 
\lq\verb+&+\rq\ (&\footnote{&&&\bar<remaining>}) or 
``auto mode" (see below) may then work better.\footnote{&\bar indeed!} 
More generally, the quoting feature still works in macro arguments in 
the sense that you then have to mark difficult characters with `&' 
(simply as short for `\string'). However, it still won't work with 
curly braces that don't follow a command name 
(such \emph{pairs} of braces will simply get lost, 
 \emph{single} braces will give errors or so).%%%\footnote{`{group}'}

Double quotes and apostrophes should still work the usual way.
% %% TODO doesn't work, inside runs into `}' 2010/02/28:
% otherwise you could control the parsing mechanisms using curly braces 
% (outside and inside don't interact: `Harry{'}s' for \qtd{Harry's}).
For difficult cases, you can still use the standard `\verb' 
command from \LaTeX.
To get \emph{usual} single quotes, you can use their standard substitutes 
`\lq' and `\rq', or for pairs of them, 
|\qtd{<text>}| in place of `\lq <text>\rq'---or even `\lq <text>\rq\ '. 

\item[Single right quotes for &\textsf:]
Package names are (by some convention I often yet not always 
 see working) 
typeset with `\textsf'; 
it was natural to use a remaining case of using single quotes 
for abbreviating $$&\textsf{<text>}$$ by |'<text>'|.
% \footnote{%
% Font switching by sequences of single quotes is a feature of the 
% syntax for editing \textit{Wikipedia} pages and of 'wiki.sty'.}
%% <- undoubled 2010/02/28 ->
This idea of switching fonts continues font switching of 'wiki.sty'
which uses the syntax for editing {\it Wikipedia} pages 
(font switching by sequences of right single quotes).

\item[Verticals for setting-off command descriptions:]%%%
\hskip0pt plus 2em
\GenCmdBox+|<code>|+ works like \qtd{&`<code>&'} except putting 
the result into a \emph{framed box} (just as all around 
here)---or something else that you can achieve using some \emph{hooks} 
described with the implementation. There are variants like 
\GenCmdBox+\cmdboxitem|<code>|+.

\item[Ampersand shows command syntax \&c. even in arguments:]
\hfil E.g., type \lq\verb+&\foo{<arg>}+\rq\ to get 
\lq`\foo{<arg>}'\rq. This may be even more convenient for typing than 
the single quotes method, although looking somewhat strange.
However, in macro arguments this does not work with 
\emph{private letters} (`@' and `_' here), for this case, 
use |\cs{<characters>}| or |\cstx{<characters>}<parameters>|.%%%
% `&' may terminate \textit{verbatim} unexpectedly, being designed for 
% displaying ``\LaTeX-like command syntax" in the first instance.
\footnote{Moreover, && currently has a limited 'xspace' 
functionality only.}%%%\footnote{You can even use && for referring to 
%   active characters like && in footnotes etc.!}
%% <- said elsewhere now 2010/03/07

\begin{sloppypar}
This choice of `&' rests on the assumption that there won't be many 
tables in the documenation. You can restore the usual meaning of `&' 
by `\MakeNormal\&' and turn the present special meaning on again by 
\[`\MakeActive\&' \mbox{\quad or\quad } 
  `\MakeActiveLet\&\CmdSyntaxVerb'\]
You could also 
redefine (&\renewcommand) &\descriptionlabel using `\CmdSyntaxVerb' 
(the ``normal command" that is equivalent to `&', its ``permanent 
 alias") 
so \verb+\item[\foo]+ works as wanted.
\end{sloppypar}

\textbf{Another} feature of 'niceverb''s `&' is getting 
(some of the) special characters    %% 2010/03/20
(as listed in the standard macro `\dospecials') verbatim in arguments 
(where `\verb' and the like fail). It just acts similarly as \TeX's 
(as listed in the standard macro `\dospecials') verbatim in arguments 
(where `\verb' and the like fail). It just acts similarly as \TeX's 
 primitive `\string' (which it actually invokes---cf. discussion on the 
 left quote feature above). 

\item[``Auto mode" typesets commands verbatim unless .\,.\,.]
\begin{sloppypar}
In~``auto mode," the backslash \lq`\'\rq\ is an active character that 
builds a command name from the ensuing letters and typesets the 
command (and its syntax, allowing meta-variables) verbatim. 
However, there are some exceptions, which are collected in a macro 
|\niceverbNoVerbList|. &\begin, &\end, and &\item belong to this list, 
you can redefine (`\renewcommand') it, or add <macros> to it by
|\AddToMacro{\niceverbNoVerbList}{<macros>}|
There is also a command |\NormalCommand{<letters>}| \emph{issuing} the 
command `\<letters>' instead of typesetting it.
Since auto mode is somewhat dangerous, you have to start it explicitly 
by |\AutoCmdSyntaxVerb|. You can end it by |\EndAutoCmdSyntaxVerb|.
|\AutoCmdInput{<file>}| is probably most important. 
\end{sloppypar}

Auto mode is motivated by the observation that there are package files 
containing their documentation as pure (well-readable) ASCII 
text---contain\-ing the names of the new commands without any kind of 
quotation marks or verbatim commands. 
Auto mode should typeset such documentation just from the same ASCII 
text.

\item[Hash mark \lq&#\rq\ comes verbatim.]
No macro definitions are expected in the `document' 
environment.\footnote{This idea appeared 2009 on the 'LATEX-L' 
                      mailing list. It may be wrong, 
                      as I have sometimes experienced $\dots$}
                      %% <- changed 2010/03/11
Rather, \lq`#'\rq\ is an active character for taking the next 
character (assuming it is a digit) to form a reference to a 
\emph{macro parameter}---\lq`#1'\rq\ becomes \lq#1\rq\---WYSIWYG 
indeed! (So the general syntax is |#<digit>|.)
\item[Escaping from 'niceverb' (generally).] 
     To get rid of the functionality of some active character <char> 
     (\qtd{&&}, single quote, ampersand, hash mark---not 
      ``auto mode," see above) here, use |\MakeNormal\<char>|---may 
     be within a group. To revive it again, use |\MakeActive\<char>|. 
     This may fail when a different package overtook the active <char> 
     (but I expect more failures then), in this case 
     |\MakeActiveLet\<char>\<perm-alias>| 
     revives the 'niceverb' meaning of <char>
     where `\<perm-alias>' is the ``permanent alias" for that active 
     <char> according to the documentation below. 
     E.g., `\LQverb' is the ``permanent alias" for active single left 
     quote, 'niceverb' activates it by 
     \NVerb+\MakeActiveLet\'\LQverb+.---You can turn off 'niceverb' 
     syntax \emph{alltogether} by |\noNiceVerb| and revive it 
     by |\useNiceVerb| (without ``auto mode").{\sloppy\par}

     \textbf{Right Quotes:} Disabling\slash reviving replacement 
     of `\textsf' by single right quotes requires 
     \[|\nvRightQuoteNormal| \mbox{\quad or\quad } |\nvRightQuoteSansSerif|\] 
     respectively.
\end{description}

\subsection{Examples}
The file 'mdoccorr.cfg' providing some `.txt'$\to$\LaTeX\ 
functionality---i.e., typographical corrections---documents itself 
using 'niceverb' syntax. Its code and the documentation that is 
typeset from it are in the \qtd{examples} section of 
'makedoc.pdf'.---Moreover, 
the documentation 'niceverb.pdf' of 'niceverb.sty' was 
typeset from 'niceverb.tex' and 'niceverb.sty' using 'niceverb' 
syntax, likewise 'fifinddo.pdf' and 'makedoc.pdf'. 
The example of 'niceverb' shows the most frequent use of the `&' 
feature.{\sloppy\par}

'nicetext' bundle release v0.4 contains a file 'substr.tex' 
that should typeset the documentation of the version of 
Harald Harders'
'substr.sty'\footnote{\url{http://ctan.org/pkg/substr}}
that your \TeX\ finds first, as well as 'arseneau.tex' 
typesetting a few packages by Donald Arseneau. 
The outcomes (with me) are 'substr.pdf' and 'arseneau.pdf'.
These are the first applications of 'niceverb''s ``auto mode" to 
(unmodified) third-party package files.
(I also made a more ambitious documentation of Donald Arseneau's 
 'import.sty v3.0' before I found that CTAN already has a nicely 
 typeset documentation of 'import.sty v5.2'.)

%% removed 2010/03/11:
% It seems to me that I could type so many pages on 'fifinddo' and 
% 'makedoc' in little more than a week 
% % (2009/04/12, much of which was needed for debugging and reworking concepts) 
% only due to the ``minimal" \emph{verbatim} and syntax-display syntax. 
% 
\subsection{What is Wrong with the Present Version}
\begin{enumerate}
\item 'niceverb.sty' should be an extension of 'wiki.sty'; 
      yet their font selection mechanisms are currently not compatible. 
      %% 2010/02/28:
      Especially, the feature of \[\hbox\bgroup|''<text>''|\egroup\] 
      %% <- failed with \mbox as of 2010/03/23, first two rq missing 
      %%    2010/03/29
      replacing 
      `\textit{<text>}' or `\emph{<text>}' may be considered missing. 
\item Font switching or horizontal spacing may fail in certain 
      situations.
%       (parentheses, titles, footnotes; 
      You can correct spacing by \lq`\ '\rq. 
        %% <- \qtd{`&\ '}.
% \item 
% The feature of mixing high-quality-typeset comments into the 
% package code listing is implemented in a very rudimentary way only. 
% % just allowing for `\subsection's. 
% The ``comment detector" detects Wikipedia-style subsection titles 
% instead of lines beginning with percent characters.\footnote{%
% Percent characters will definitely not be ``ignored" as with &\DocInput, 
% rather they will hide rests of \emph{documentation} lines as usually, 
% while they will be typeset verbatim in \emph{package code} lines.} 
% Switching between plain and verbatim typesetting in the package 
% listings isn't settled yet, since there are different styles of using 
% percent symbols. I have sometimes used double percent symbols 
% (\lq\verb+%%+\rq) 
% for commenting text and single ones just for ``reversible deletion of 
% code," while usually single percent symbols indicate commenting text 
% indeed. Double percent symbols may, by contrast, mean that the text remains 
% visible in the `.sty' file only, suppressed in the typeset 
% documentation ('lineno.sty').
% For a while, it may be necessary to provide replacing macros for each 
% package separately instead of providing a single macro package 
% managing all of them. 
% \item 
% The code listing currently uses the `listing' and `listingcont' 
% environments of 'moreverb.sty'; 
% the code font and the line numbers may be too large. 
\item The ``vertical" character \qtd{&|} produces inline boxes 
      only at present. It might as well provide a version of the 
      `decl' tabular environment of 'ltxguide.cls'. 
%% changes 2010/03/10
%       coloured\slash framed boxes instead (2009/04/09). They have 
%       their merits! See 'fifinddo.pdf'  and 'makedoc.pdf'. However, 
%       they 
      The inline boxes
      badly deal with long command names and many arguments.
      Doubled verticals could ensure the `decl' mode. 
      Moreover, such a box might issue an \emph{index} entry.
\item One may have \emph{opposite} ideas about using quotes---maybe 
      rather `"<code>"' should typeset <code> \textit{verbatim}.
      There might be a package option for this. If ordinary 
      \qtd{\NVerb'``<text>"'} still should work, awful tricks as now with 
      the right quote feature would be needed. %% TODO 2010/03/06
% \item ``Auto mode" has \emph{not} been tested on a serious application yet. 
%% partially improved 2010/02/28:
% \item % 'niceverb''s font switching tricks sometimes turn against their 
%       % inventor (and other users?). There must be some switching 
%       % ``off'' (and ``on'' again).%
%       %   \footnote{\hspace{1sp}'fifinddo'\slash\hspace{1sp}'makedoc'
%       %     %% <- TODO oh, oh! 2009/04/11
%       %     allow inserting such commands from a driver script, 
%       %     invisible in the file that contains the ``contentual'' 
%       %     documentation.}
%       % Also, there 
%       There
%       might better help with weird errors, 
%       some syntax checks might intercept earlier. 
% 
%       Similarly, some choices reflect a %% rather OK 2010/02/28
%       personal style and should be modifiable, especially by package 
%       options.\footnote{Please sponsor the project or support it 
%         otherwise!}
\item Certain difficulties with typesetting code in macro arguments 
      may be overcome easily using $\varepsilon$\mbox{-}\TeX\ 
      features, I need to find out $\dots$
\end{enumerate}

\section{Implementation of the Markup Syntax}
\typeout{niceverb.tex 2010/04/05 documenting niceverb.sty}
\title{\textsf{niceverb.sty}\\---\\Minimizing 
  Markup\\for Documenting \LaTeX\ packages%%% \thanks{This 
%     manual describes package version
%     version 0.2 as of April 09, 2009%%%\fileversion\ as of \filedate\ 
%     .}}%%%of the package.}%
}
% \listfiles 2010/03/19
{ \RequirePackage{makedoc} \ProcessLineMessage{} %% 2010/03/11
  \MakeJobDoc{19}{\SectionLevelThreeParseInput}  }
\documentclass{article}%% TODO paper dimensions!?
\typeout{makedoc.tex 2010/03/30 documentation for `makedoc.sty'}
\listfiles
\RequirePackage{makedoc}
\documentclass{article}
\input{makedoc.cfg} %% with pdf stuff and 'niceverb'
%% removed for niceverb v0.31 TODO!? 2010/03/20:
\sfcode`/=1001 %% TODO makedoc.cfg!? 2010/03/21
% \makeatletter %% TEST for hyperref compatibility 2010/03/11
%   \def\@testdef #1#2#3{%
%     \def\reserved@a{#3}%
%     \expandafter \ifx \csname #1@#2\endcsname
%    \reserved@a  \else \@tempswatrue \fi
%     \if@tempswa
%       \typeout{^^J*** Type `r' <input> to get around 
%                       \string\label\space issues! ***^^J}
%       \errorcontextlines=0
%       \show\reserved@a
%       \expandafter \show \csname #1@#2\endcsname
%     \fi
%    }
% \makeatother
\begin{document}
\title{'makedoc'---Preprocessing documentation by \TeX}
  %% 2009/04/10: \\---\\\ breaks TOC
\maketitle
\begin{abstract}\noindent
'makedoc' provides commands for generating \LaTeX\ input from a 
package file in order to typeset the latter's documentation 
(somewhat similar and opposite to 'docstrip')---with 
v0.3 \emph{a single one usually suffices}. 
Certain comment marks are removed, listing commands are inserted, 
and some (configurable) typographical `txt'$\to$\TeX\ corrections 
are applied.---This 
continues the policy of 'niceverb' to minimize documentation markup in 
package files. 'makedoc' extends and exemplifies the parsing package 
'fifinddo'. After an edit (and test) of your package, you get the new 
documentation in one run (or the usual number of runs) of the 
documentation driver file.---The present approach is meant to be an 
\emph{alternative} to the standard 'doc' package and its `\DocInput'. 
It provides \emph{less} than 'doc' does, rather deliberately. It may 
be helpful at least for the development of small packages, or at least 
at early stages.
\end{abstract}
\tableofcontents
\section{Introduction}
\emph{The abstract will not be repeated in this section.} Let me add 
instead that I was in dire need of such a package, I got stuck with my 
packages because I lost orientation in them, and I was unhappy with 
the forms of documentations of my other packages, and documenting them 
with the standard \LaTeX\ 'doc' system was not attractive for me 
(neither considered helpful). %% clarified 2010/03/13
I also worked on \emph{Windows} until September 2008, and I 
find a system like the present one still more attractive then using 
(learning!\@) other filtering utilities (see below on 'awk'). And I 
may work on \emph{Windows} once again and don't want to depend on 
installing some $\dots$ there---\emph{I really would like to have 
powerful tools for everything depending on nothing but \TeX\slash 
\LaTeX!}

\section{Prior work and what is new}
It is, of course, not a new idea to get around comment marks `%' to 
typeset the documentation. 'doc''s `\DocInput' does this by making `%' 
an ``ignored" character. This way you cannot use `%' for commenting 
comments (so 'doc' offers a ``new comment mark" 
`^'`^'`A'). %% TODO `^^A' suddenly failed 2010/03/15 -- "ligature"!?
You also cannot use `%' for commenting out code (that you are 
pondering---or using for debugging---only). %% clarified 2010/03/13

Moreover, 'doc' requires enclosing package code explicitly by 
environment commands (behind comment marks). Stephan I. B\"ottcher 
with his '\href{http://ctan.org/pkg/lineno}{lineno.sty}' 
and Grzegorz Murzynowski in \ctanpkgref{gmdoc}
aimed at doing away with this requirement. 
'lineno.sty' contains 'awk' scripts 
to remove starting comment marks and to insert listing commands. 
A file 'lineno.tex' is generated that typesets the documentation. 
By the way, 'lineno.sty' is full of discussions, but it is not 
'docstrip'ped---the maintainers never have received a complaint 
that inputting 'lineno.sty' were too slow. 

'gmdoc' seems to get around comment marks and insert listing commands 
\emph{while typesetting} by a refined version of `\DocInput', 
through some careful detecting and analysing comment marks, 
the approach resembles detection of lists in 'wiki.sty'.\footnote{See 
  'gmdoc.pdf' on &\DocInput. You can learn a lot from this 220 pages 
  document! I also find 
  \ctanpkgref{pauldoc} and \ctanpkgref{xdoc} inspiring.}
And this is a matter of principles---comparing the approaches of 
\emph{preprocessing} ('lineno.sty') and \emph{``smart typesetting"} 
('gmdoc', 'wiki'). Sometimes preprocessing seems to be simpler, 
sometimes detecting while typesetting. 
(Another example is the preprocessor 
\ctanpkgref{easylatex}
of which 'wiki.sty' is a much reduced ``while typesetting" variant.) 
``While typesetting" may be easier when single characters or 
sequences of two or three encode markup
information---but such detection can badly interfere with other 
packages etc. ``Preprocessing" may be easier when entire ``strings" 
of characters decide, which may be anywhere in a file line. 

'makedoc' chooses \emph{preprocessing}, as 'lineno.sty', but by 
\emph{\TeX}. There is a general discussion of this choice in the 
documentation of 'fifinddo'. Preprocessing here can be done in the 
same \LaTeX\ run as typesetting, though you can avoid 
incompatabilities with packages needed for typesetting 
(by inputting them only \emph{after} preprocessing). 

'lineno.sty' exemplifies why preprocessing with \emph{\TeX} may be 
preferable to preprocessing with other utilities: 
When I took over maintenance of 'lineno.sty', 
I needed hard work to get the 'awk' script running. 
The \emph{Munich} 'awk' seemed \emph{not} to behave as the \emph{Kiel} 
'awk' (I chose a Munich 'nawk' and reworked the script a little). 
\TeX\ seems to have better fixed functionality than other utilities! 

A different alternative to \LaTeX's 'doc' system is 
Paul Isambert's '\href{http://ctan.org/pkg/codedoc}{CodeDoc}' 
where the code environments extract package code in typesetting the 
documentation. %% added 2010/03/10

\section{Styles supported (parsers provided)}\label{sec:styles}
% \section{Styles of commenting '.sty's}
We find different styles of documenting \LaTeX\ packages. 
As the main aspects I consider 
(i)~\emph{telling code from comments} 
and (ii)~\emph{markup in comments}. 
(You may find more details on the next matters in the 
 ``implementation" section.)

\subsection{Telling code from comments}
\emph{Comment marks} (usually \lq`%'\rq\ in the case of \TeX) 
probably were named so to mark \emph{``comments"} as opposed 
to code $\dots$ great, but actually, in ``daily practice," 
they are so handy---and used---for ``commenting out" \emph{code}, 
i.e., \emph{managing code versions} in a simple way: 
one does not actually want to \emph{delete} code, 
one might want to use it another time, maybe for debugging
$\dots$ or to remind of earlier attempts that should not be tried 
again $\dots$

This is a problem for \emph{high-quality typesetting} of 
documentation. \emph{Code} should be typeset about as you see it on 
the \emph{screen}---\emph{monospaced}, this allows structuring by 
indenting, it is common practice to use a typewriter typeface for 
this. Real \emph{comments} should be typeset in \emph{high quality} as 
usual with \LaTeX. Little dilemmas therefore occur with \emph{``hidden 
code"} (``commented-out"). A comment mark starts the line, but 
obviously it is not really a comment and rather should be typeset 
like code (and otherwise they may break).           %% 2010/03/22
Another problem are comments at the \emph{end} of a 
\emph{code} line. Sometimes they are ``real comments" ('gmdoc' 
supports this style). But sometimes 
this is only another version of ``version management," code 
``commented-out."

I like the style of writing packages described before and use it all 
the time. I mark ``real comments" with \emph{two} adjacent comment 
marks and an ensuing space to distinguish them clearly from code 
commented out.
%% Adapted to v0.4 2010/03/29:
\emph{This style is presently the one supported by \textup{'makedoc'} 
      as default.}
This way only a line starting with 
|%% | is considered a ``real" comment line. The first three 
characters are removed, and the rest is typeset in high quality. 
Any other lines are typeset verbatim. The 'makedoc' \emph{parser} 
doing this has an ``identifier" |PPScomment| (``percent, percent, 
space"). Another identifier |comment| is a placeholder for 
the comment parser to be used, by default it is an alias for 
`PPScomment'. Lines just containing |%%| (without the space) may be 
used to suppress empty code lines preceding section titles and for 
keeping some visual, relieving space between code and comment lines.

The style I described previously may be considered ``unprofessional." 
The many \LaTeX\ packages documented using the 'doc'\slash'.dtx' 
system don't use comment marks for \emph{``commenting-out"}. 
Or one may mark code commented out by putting no space between the 
percent mark and the code. 
With v0.4 of 'makedoc', this style is supported as |PScomment|. 
You can directly call this as <main-parser> as described below, 
or you can switch to it by 
\[`\CopyFDconditionFromTo{PScomment}{comment}'\]

\subsection{Markup in comments}
Packages using the 'doc'\slash '.dtx' system as well as alternative 
highly developed systems mentioned above use (enhanced) usual 
\emph{\LaTeX} syntax for markup of comments. Other packages just use 
an \emph{ASCII} style \emph{without} any markup. My idea was to 
support the latter style by some `txt'$\to$\LaTeX\ functionality. 
'makedoc' does this using a file 'mdoccorr.cfg' which is very small 
right now.

I also thought of introducing another sort of ``decent" markup not 
needing much more space than the ``ASCII kernel" of the comments. 
This is to some extent implemented in 'niceverb.sty'. I thought of the 
syntax of editing \textit{Wikipedia} pages; this is partially 
implemented in 'wiki.sty' which unfortunately is not yet compatible 
with 'niceverb'.

But 'makedoc' implements one \textit{Wikipedia} feature in a different 
way than 'wiki.sty' (cf.~'wikicheat.pdf') that looks about as follows:
\begin{eqnarray*}
  \endcell\endcell`%% == Section =='\\
  \endcell\endcell`%% === Subsection ==='\\
  \endcell\endcell`%% ==== Subsubsection ===='
\end{eqnarray*}
i.e., you type `== <title> ==' in place of `\section{<title>}' etc.
The parser must replace `====<title3>===' before `===<title2>===' and 
the latter before `==<title1>=='. In fact, 'makedoc' provides three 
parsers for these situations:
\begin{description}
\cmdboxitem|\SectionLevelThreeParseInput| is the most general parser 
    offered. If it does not find two strings \lq`===='\rq\ enclosing 
    \emph{something}, it passes to
\cmdboxitem|\SectionLevelTwoParseInput| which unless finding 
    two strings `===' enclosing something passes to
\cmdboxitem|\SectionLevelOneParseInput| $\dots$ passes to the comment 
    detector |comment|. 
\end{description}


\section{Requirements}
'makedoc' requires \LaTeXe\ (supporting star forms of `\newcommand' 
etc.)\ as \TeX-format, the package 'fifinddo.sty' from the same 
directory (on CTAN etc.)\ as where 'makedoc.sty' is, and the 
\LaTeX-package 'moreverb' by Robin Fairbairns (after others)---it 
should be installed anyway, or you can get its latest version 
(v2.3, 2008/06/03?) from CTAN. 

'makedoc''s `.txt'$\to$\TeX\ functionality moreover needs a file 
'mdoccorr.cfg' that should have come along with 'makedoc.sty' and 
'fifinddo.sty'. You may need to have a modified copy of it in the 
directory of your main `.tex' file `<jobname>.tex' fitting special 
needs of your project. 

\section{Using 'makedoc' the simplest way}
In the most simple case, you are preparing documentation for a package 
file `<jobname>.sty' only, and you prepare a file `<jobname>.tex' 
containing 
\[`\title{\textsf{<jobname>}---a \LaTeX\ Package for <whatever>}'\]
and `\maketitle' etc.\ about your package.\footnote{With 'niceverb' 
    and &\title\ after &\begin{document}, you may replace 
    \lq&\textsf{<jobname>}\rq\ by \lq&'<jobname>&'\rq.}
The documentation will be produced by running `<jobname>.tex' with 
\LaTeX\ (e.g., \texttt{latex <jobname>.tex}).

First, `<jobname>.tex' must have |\usepackage{makedoc}| in its preamble. 
There are no package options. 

Second, to typeset the commented implementation from `<jobname>.sty', 
include in <jobname>.tex's `document' environment a line 
\[|\MakeInputJobDoc{<header-lines>}{\SectionLevelThreeParseInput}|\]
<header-lines> refers to a non-negative integer as follows: 
We think the most simple and useful way of typesetting the first lines 
of a package file including license and copyrights is ``depicting them 
as image," i.e., \textit{verbatim}. We could try to determine the 
number of these lines by parsing, but we won't do so soon. Please just 
count them and enter the number as <header-lines>---and change it 
until you can accept the outcome.

\section{Steps of advanced usage}
\subsection{Different main parsers (second mandatory argument)}

`\MakeInputJobDoc''s mandatory syntax actually is 
\[|\MakeInputJobDoc{<header-lines>}{<main-parser>}|\]
<main-parser> refers to the parsing macro that is applied to each 
input line whose number is greater than <header-lines>. 
Examples for <main-parser> are named in section~\ref{sec:styles} above. 
  %% TODO above/below macro 2010/03/15
`\SectionLevelThreeParseInput' is just the most general one. 
For \emph{efficiency} (!? or also to avoid problems?) you may 
replace `Three' by `Two' or by `One', if the `====' or the `===' 
feature is not used in `<jobname>.sty'. If the ``\textit{Wikipedia} 
sectioning" feature is not used at all, use 
\[|\MakeInputJobDoc{<header-lines>}{\ProcessInputWith{comment}}|\]
---provided you want to adopt the \lq`%% '\rq\ style of marking 
comments, cf.~section~\ref{sec:styles}. For the \lq`% '\rq style 
instead, use 
\[|\MakeInputJobDoc{<header-lines>}{\ProcessInputWith{PScomment}}|\]

\subsection{Different extensions (optional arguments)}
If your package to be documented is a \emph{class} `<jobname>.cls', 
a local configuration file `<jobname>.cfg' or something 
else---<jobname>.<ext-in>, e.g., <ext-in>=`cls' or <ext-in>=`cfg', 
use 
\[|\MakeInputJobDoc[<ext-in>]{<header>}{<parser>}|\]
Moreover, `\MakeInputJobDoc' writes an intermediate file 
`<jobname>.doc' and then `\input's it. If you do not like `doc' 
as extension for the written file name (maybe you use 
`<jobname>.doc' for something different already), preferring extension 
<ext-out>, use
\[|\MakeInputJobDoc[<ext-in>][<ext-out>]{<header>}{<parser>}|\]
Yes, you must state <ext-in> then as well, I can't help $\dots$

If even <jobname> is wrong in your view, see next step $\dots$

\subsection{Commands modifying &\MakeInputJobDoc's behaviour}
\label{sec:modimake}
Already <jobname> may not be what you want. E.g., you may want 
to collect documentations of some other files <job-1>, <job-2>, 
$\dots$ in a single <jobname>. Then precede `\MakeInputJobDoc'
with 
\[`\renewcommand*{\mdJobName}{<job-1>}'\]
etc.\ (please reason yourself about additional requirements \dots)
As a matter of fact, `\MakeInputJobDoc' reads 
\[`\mdJobName.<ext-in>' \mbox{\quad and writes\quad} 
  `\mdJobName.<ext-out>'\]
Stated another way, <jobname> above referred to |\mdJobName|.

`\MakeInputJobInput' moreover (by default) produces one dot 
per input line processed on screen to show progress. 
The reason is that `makedoc' issues the command 
|\ProcessLineMessage{\message{.}}|.
Already this trivial thing seems to slow down processing considerably 
(nowadays). `\MakeInputJobInput' will run faster if preceded by 
\[|\ProcessLineMessage{}|\]
which will suppress any message about processing.
However, the message may be helpful in trouble-shooting.

\subsection{Separating preprocessing from typesetting}
  %% extended 2010/03/16
To some surprise, I observe that `\MakeInputJobDoc' \emph{works.} 
This is quite a new discovery of mine (2010/03/13); 
before I thought that, for safety, preprocessing should happen 
inside a local group \emph{preceding} `\documentclass'.
|\MakeJobDoc| works like `\MakeInputJobDoc' described above, 
yet it just \emph{preprocesses} the package to be documented, 
waiting for an 
\[`\input{<jobname>.<ext-out>}'\] 
in the `document' environment to \emph{typeset} the documentation. 
So 'makedoc.tex' once had in its preamble
% \[`{\RequirePackage{makedoc} \MakeJobDoc{<header>}{<parser>}}'\]
% at the top of `<jobname>.tex' and `\input{<jobname>.<out-ext>}' later. 
\begin{eqnarray*}\endcell\endcell
  `{\RequirePackage{makedoc}'\cr    \endcell\endcell
  ` \ProcessLineMessage{}'\cr       \endcell\endcell
  ` \MakeJobDoc{22}{\ProcessInputWith{comment}}}'\cr
                                    \endcell\endcell
  `\documentclass{article}'
\end{eqnarray*} 
I did experience some truth in my earlier safety policy: 
With 'niceverb' ``running," `\MakeJobDoc' cannot (easily) be used 
in the `document' environment. `\MakeInputJobDoc' in fact does some 
'niceverb' switching (provided 'niceverb' has been loaded) 
when making use of `\MakeJobDoc'.
  %% <- verbose to improve line breaks 2010/03/16

Thinking of this ``safety" approach, especially grouping (`{\code}'), 
I had not much cared for compatibility with other packages 
in choosing 'makedoc' macro names. 

\subsection{Other 'makedoc' (and 'fifindo') script commands}
The next script commands may be considered of a lower level than 
`\MakeJobDoc' and `\MakeInputJobDoc', they underlie the latter 
commands. We also list commands from 'fifinddo.sty' that may be useful 
or, indeed, are needed for preparing package documentations. 
This may result in ideas on how do use the script commands for 
different purposes than for preparing package documentations---e.g., 
apply `txt'$\to$\TeX\ preprocessing to arbitrary text files. 

\subsubsection{Choosing parameter values for next preprocessing run}

This actually continues section~\ref{sec:modimake}.

\begin{description}
\cmdboxitem|\ResultFile{<output>}| (from 'fifinddo') 
    determines (and opens through the \TeX\ primitive `\openout') 
    the file <output> which will contain the result of 
    preprocessing the package file.
\cmdboxitem|\LaTeXresultFile{<output>}|---see next section.
\cmdboxitem|\Headerlines{<number>}| determines the <number> of lines 
    starting the input file to be copied \emph{verbatim} 
    (the first mandatory argument of `\MakeJobDoc'). 
\cmdboxitem|\MainDocParser{<parser>}| determines the <parser> 
    as in the \emph{second} mandatory argument of `\MakeJobDoc'.
\end{description}

We are now describing some parameters which rather must be switched 
``manually" instead of being modifiable by comfortable 'makedoc' 
script commands.

With the \emph{``Wikipedia sectioning"} feature, you may change the outcome 
regarding levels. Assume, e.g., the package file has titles along the 
scheme `== <title> ==' only, but these should become 
\emph{subsections} of the ``implementation" section of the 
corresponding `.tex' file. Then 
\[`\renewcommand*{\mdSectionLevelOne}{\string\subsection}'\]
-- see the implementation of the sectioning feature for details. 

There is a command 
\[|\NoEmptyInputLines| \mbox{\quad and a parameter macro\quad}
  |\OnEmptyInputLine|\] 
which is modified by the former. However, I cannot say much about them 
right now, I think they just were a difficult matter that I soon 
decided no longer to think about for a while (cf.\ the 
implementation). About the same holds for the hook |\EveryComment|.

The `txt'$\to$\TeX\ functionality is implemented through a hook 
\[|\MakeDocCorrectHook{<characters>}|\] 
'makedoc' initializes it as an alias of \LaTeX's `\@firstofone', i.e., 
it won't change <characters>. 'mdoccorr.cfg' should redefine it so it 
really ``corrects" <characters>. You might try other definitions of 
`\MakeDocCorrectHook' for different ``correcting" functions.
It should be \emph{noted} that (currently) 
`\MakeDocCorrectHook' must be \emph{expandable}, 'fifinddo.sty' 
provides setup for (expandable) chains of expandable replacements. 
The ``Wikipedia" sectioning feature moreover uses expandable 
trimming (single) surrounding spaces, this might be provided in a more 
general way.\footnote{%% TODO 2010/03/16
    The \ctanpkgref{trimspaces} package 
    has been a \emph{model} for this feature here. It cannot be used 
    directly here because it reads blank spaces as \TeX\ characters with 
    category code 10 while 'makedoc' reads blank spaces as ``other" 
    characters (category code 12) in order to \emph{keep} all blank spaces.}

\subsubsection{``Manual" insertions to the output file}
\begin{description}
\cmdboxitem|\WriteResult{<balanced>}| (from 'fifinddo') writes 
    <balanced> to <output> according to the earlier command 
    `\ResultFile{<output>}'.
\cmdboxitem|\WriteProvides| (from 'fifindo') writes a 
    `\ProvidesFile' line into <output> that declares the file 
    to be generated by 'fifindo'.
\cmdboxitem|\LaTeXresultFile{<output>}| issues 
    `\ResultFile{<output>}' and then writes a 
    `\ProvidesFile' line into <output> that declares the file 
    to be generated by 'makedoc'.
\end{description}

\subsubsection{Processing input and closing output}
\begin{description}
\cmdboxitem|\MakeDoc{<input>}|\hskip 0pt plus 4em
    reads 'mdoccorr.cfg' 
    (for `\MakeDocCorrectHook', see above),
    %% removed \LaTeXresult... 2010/03/17
    copies <number> according to `\HeaderLines' (see above) 
    from <input> into <output> (according to `\ResultFile'), 
    then processes the remaining lines of <input> according 
    to `\MainDocParser' (writing several things to <output>). 
    `\MakeDoc' invokes{\sloppy\par}
\cmdboxitem|\ProcessFileWith{<input>}{<loop-body>}| 
    (from 'fifindo') reads <input> line by line---each one stored as 
    macro |\fdInputLine| and applies <loop-body> to it. 
    \TeX's ``special" character codes (of characters listed in 
    macro `\dospecials') are replaced by 12 (``other") by 
    default---see the 'fifinddo' documentation.
\cmdboxitem|\CloseResultFile| (from 'fifinddo') 
    \hskip 0pt plus 1em
    \emph{closes} 
    <output> (using \TeX's primitive `\closeout'). 
\cmdboxitem|\MakeCloseDoc{<input>}| issues 
    `\MakeDoc{<input>}\CloseResultFile'.
\end{description}
%
Using `\MakeDoc' \emph{instead} of `\MakeCloseDoc' allows processing 
additional <input> files writing into the same <output>. Or maybe you 
want to add some additional lines manually to <output> using 
`\WriteResult'.

%% removed 2010/03/09:
% At least in the long run, using 'makedoc' should not imply commitment 
% to a certain design or to certain \LaTeX\ packages for typesetting 
% listings and documentations. Therefore, 'makedoc.cfg' (currently) 
% contains \emph{local} or \emph{personal choices}, but also 
% \emph{experiments} with future features of 'niceverb'. 
% Especially, (at present) the `packagecode' 
% environment that 'makedoc' `\write's must be chosen. 
% Currently this is the `listing' environment from 'moreverb' 
% with some modifications or extra settings. 
% It may be vital to `\MakeOther' the active characters from 'niceverb' 
% in the setup of `packagecode'. See the \emph{example} in 
% section~\ref{sec:fifinddo}.
% 
% Finally, 
% Each package file to be typeset needs its own little
% \emph{script} of 'makedoc' commands. 
% With v0.3, one or two of these should suffice. 

% It should fit into the preamble 
% of the main file for documenting the package (currently %% 2009/04/09
%   just 5 commands seem to suffice, see the \emph{example} in 
%   section~\ref{sec:fifinddo}). 
% The script commands are described 
% in the \dqtd{File handling} section of 'fifinddo.pdf' and in the 
% present section~\ref{sec:script} (and \ref{sec:emptylines}).
% As an alternative, you may prefer to have ``content only" (as much as 
% possible) in the main typesetting file and in its preamble only 
% `\input' a separate script file.
%% removed 2010/03/10:
% Yes, the idea of documenting a package \emph{here} is to have a 
% separate ``driver" file for typsetting the documentation. 
% It may contain an introduction and a guide for users. 
% The documentation of the package code that has been prepared by the 
% 'makedoc' script will be `\input'. Alternatively, the ``driver file" 
% could have title etc.\ only, or preamble and a minimal `document' 
% environment only. 
% 
% So there may be many files, which may look confusing, especially as 
% compared with the 'doc' procedure. However, 
% \begin{enumerate}
% \item ``One file distribution" still is possible thanks to the 
%       `filecontents' environment. 
% \item The 'makedoc' script can create a batch file (fitting the 
%       system, maybe using Will Robertson's 'ifplatform', or 
%       'texsys.cfg', or \dots) 
%       that removes certain auxiliary files or moves them to a 
%       certain directory. 
% \item I find it helpful to have rather little ``contentual" text 
%       in the package file. 
% \item The procedure now runs very smoothly, once the stumbling blocks 
%       have been overcome.\footnote{\hspace{1sp}%% TODO help in 'niceverb'!
%         'niceverb' v0.1 was too sloppy with 
%         some things, and self-documentation of 'makedoc.sty' was 
%         difficult---its parsing and that from 'verbatim' cannot 
%         distinguish between markup code and typeset code.}
% \end{enumerate}

\section{Examples}%%% (documentation of 'mdoccorr.cfg')}
%% moved here 2010/03/23
\subsection{'nicetext', especially 'mdoccorr.cfg'}
The documentations of 'fifinddo', 'makedoc', and 'niceverb' 
themselves are typeset using 'makedoc'.
'fifinddo.pdf' documents 'fifinddo.sty', typeset 
from 'fifinddo.tex', likewise 'makedoc.pdf' and 'niceverb.pdf'. 
% Section~\ref{sec:fifinddo} contains a listing of 
% 'makedoc.cfg' and 
% the 'makedoc' script file 'mkfddoc.tex' 
% especially made for 'fifinddo.pdf'. 
% 'fifinddo.doc', 'makedoc.doc', and 'niceverb.doc' are the \TeX\ input 
% files that were made with 'makedoc.sty'---I have only looked at them 
% when something was wrong (often syntax mistakes in typing). 
The Wikipedia syntax feature 
\begin{quote}
  `%% === subsection ===' 
\end{quote}
is used in 'fifinddo.sty' and 'niceverb.sty' only.

Along with 'makedoc' should come files `makedoc.tpl'---a template 
'makedoc' script, and a file `fdtxttex.tex' that should start a dialogue 
on trying `\MakeDocCorrectHook' if you can manage to run it ('WinShell'?). 
With other definitions of `\MakeDocCorrectHook'---see below---you can 
use this dialogue for arbitrary replacement jobs (as an application of 
'fifinddo' rather than 'makedoc').{\sloppy\par}

'fifinddo.pdf', 'makedoc.pdf', and 'niceverb.pdf' were typeset with the following 
typographical corrections in 'mdoccorr.cfg' that defines 
`\MakeDocCorrectHook':
\strut\hrule
\begingroup 
  \hfuzz=\textwidth \advance \hfuzz by 28pt
  \MakeOther\|\MakeOther\`\MakeOther\'\MakeOther\<
  \listinginput[5]{1}{mdoccorr.cfg}
\endgroup
\hrule\noindent\strut
This code also exemplifies the syntax 'niceverb' provides for writing 
about \LaTeX\ macros. It is typeset here with 'makedoc.sty' 
and then looks thus:
%  \sloppy %% 2010/03/29
\strut\hrule
\renewcommand*{\mdJobName}{mdoccorr}
\MakeInputJobDoc[cfg]{0}{\ProcessInputWith{comment}}
\hrule \noindent\strut
And this is the content of the intermediate generated file:
\hrule
\begingroup 
  \hfuzz=\textwidth \advance \hfuzz by 28pt
  \MakeOther\|\MakeOther\`\MakeOther\'\MakeOther\<
  \listinginput[5]{1}{mdoccorr.doc}
\endgroup
\hrule 
%  \fussy %% 2010/03/29

\subsection{Packages from other authors}
`substr.tex' should typeset a nicely formatted documentation 
of Harald Har\-der's 'substr.sty', see my own result `substr.pdf'. 
'substr.sty' is a rare case of the \lq`%% '\rq\ commenting style 
that 'nicetext' has used itself.

`arseneau.tex' should typeset nicely formatted documentations 
of a few packages by Donald Arseneau, see my own result `arseneau.pdf'. 
This demonstrates the usual \lq`% '\rq\ commenting style 
that 'makedoc' supports with v0.4.

\pagebreak              %% 2010/03/29
\ResetCodeLineNumbers   %% 2010/03/29
\section{Implementation}
\subsection{Preliminaries} 
Head of file (Legalese):
\sloppy
\renewcommand*{\mdJobName}{makedoc}
\ProcessLineMessage{}
\MakeInputJobDoc{22}{\ProcessInputWith{comment}}
The previous empty code line is the one \TeX\ insists to add at every 
end of a file it writes. %% todo TeXbook where? 2009/04/08

%% removed (TODO) 2010/03/15:
% \section{Examples: documentation of 'fifinddo'}
% \label{sec:fifinddo} 
%% removed 2010/03/10:
% \subsection{'makedoc.cfg'} 
% 'fifinddo.pdf' and 'makedoc.pdf' were typeset with the following 
% configuration file 'makedoc.cfg':
% \begingroup \MakeOther\|\MakeOther\`\MakeOther\'\MakeOther\<
%   %% <- TODO should be 'niceverb' command 2009/04/08
%   \listinginput[5]{1}{makedoc.cfg}
% \endgroup
%
%% TODO 'niceverb' example---to 'niceverb.tex'!? 2010/03/15
% \subsection{'mkfddoc.tex'}
% 'fifinddo.pdf' was typeset with the following 'makedoc' script file 
% 'mkfddoc.tex':
% \begingroup 
%   \MakeOther\|\MakeOther\`\MakeOther\'\MakeOther\<
%   \listinginput[5]{1}{mkfddoc.tex}
% \endgroup
% 
%
\end{document}

2009/04/12  more on examples
2009/04/15  exemplification of niceverb.sty by mdcorr.cfg
2009/04/21  === subsubsection -> === subsection
2010/03/08  moved `only' for better line break
2010/03/09  removed something from "Basics"
2010/03/10  more changes in "Basics", pdf stuff to makedoc.cfg, 
            makedoc.cfg no longer example; CodeDoc
2010/03/11  use \MakeCloseDoc; \hfuzz with \listinginput;
            tracing spurious `Label(s) may have changed'
2010/03/12  tests for hyperref compatibility completed
2010/03/13  use \MakeInputJobDoc; clarified ...; ctan.org/pkg
2010/03/14  updated ``Examples" and abstract; \href...easylatex
2010/03/15  ``styles supported"; abstract: txt->TeX; usage
2010/03/16  more on usage; mdcorr -> mdoccorr
2010/03/17  corr. mistake with \MakeDoc
2010/03/19  '' -> " 
2010/03/20  for niceverb v0.31
2010/03/21  for niceverb v0.32
2010/03/22  "may break"
2010/03/23  \noindent in example, moved, added mdoccorr.doc, 
            makedoc.tpl
2010/03/29  \ResetCodeLineNumbers, 
            examples and explanations for v0.4
2010/03/30  \listfiles test
 %% shared formatting settings
\begin{document}
\maketitle
\begin{abstract}\noindent
'niceverb.sty' provides very decent syntax (through active characters) 
for describing \LaTeX\ packages and the syntax of macros conforming to 
\LaTeX\ syntax conventions.
\end{abstract}
\tableofcontents

  %% TODO table listing of active characters
%% Were tests 2010/03/08:
% \section{Presenting Nasty's `Nasty' ``Nasty'' &\NVerb\ 'niceverb'}
% \section{Presenting \cs{NVerb} 'niceverb'}
\section{Presenting 'niceverb'}
\subsection{Purpose}
% \begin{abstract}\noindent
% The 'nicetext' bundle provides ``minimal" markup 
The 'niceverb' package provides ``minimal" markup for documenting \LaTeX\ 
packages, reducing the number of keystrokes/visible characters needed
% .\,.\,. %%% ... %% TODO nicedots 
(kind of poor man's WYSIWYG).\footnote{``What you see is what you 
  get." Novices are always warned that WYSIWYG is essentially 
  impossible with \LaTeX.} %% TODO UK FAQ 2010/03/11
% One feature---\verb'&\foo'%%% badly self-documenting, `&' fails
It conveniently handles command names in arguments of macros 
such as &\footnote or even of sectioning commands. 
% (`.aux'/`.toc' entries).
% 
% This is done by making some characters active. 
% 'niceverb.sty' thus resembles 'wiki.sty'; both are siblings. 
% \end{abstract}
If you use 'makedoc.sty' additionally, commands for typesetting a 
package's code are inserted automatically (just using \TeX). 
%%% \footnote{Stephan I. B\"ottcher used
%%% 'awk' instead to typeset the documentation of his 'lineno.sty'.} 
As opposed to tools that are rather common on UNIX/Linux, this 
operation should work at any \TeX\ installation, irrespective of 
platform.

Both packages may at least be useful while working at a very new package 
and may suffice with small, simple packages. After having edited your 
package's code 
%% <jobname> 2010/02/28:
(typically in a `.sty' file---<jobname>`.sty'), 
you just ``{`latex'}" the manual file 
(maybe some `.tex' file---<jobname>`.tex') 
and get instantly the corresponding updated documentation.

'niceverb' and 'makedoc' may also help to generate without much effort 
documentations of nowadays commonly expected typographical quality for 
packages that so far only had plain text documentations.

\subsection{Acknowledgement/Basic Ideas}
Three ideas of Stephan I. B\"ottcher's in documenting his 
\ctanpkgref{lineno}
inspired the present work: 
\begin{enumerate}
\item 
The markup and its definitions are short and simple, 
markup commands are placed at the right ``margin" 
of the ASCII file, 
so you hardly see them in reading the source file, 
you rather just read the text that will be printed. 
\item 
An 'awk' script removes the `%'s starting \emph{documentation} lines 
and inserts the commands for typesetting the package's \emph{code} 
(you don't see these commands in the source).\footnote{The 
  corresponding part of the ``present work" is 'makedoc.sty'.} 
  %% <- clarified 2010/03/11
\item 
An active character (\lq&|\rq) issues a `\string' \emph{and} switches 
to typewriter typeface for typesetting a command verbatim---so this 
works without changing category codes (which is the usual idea of 
typesetting code), therefore it works even in macro arguments.
\end{enumerate}

\subsection{The Commands and Features of 'niceverb'}
Actually, it is the main purpose of 'niceverb' to save you from 
``commands" $\dots$\par
Single quotes &`, &', ``less than" &< (accompanied 
with `>'), the ``vertical" &|, the hash mark `#', ampersand `&', 
and in an extended ``auto mode" even backslash `\' become `\active'
characters with ``special effects." 
% \qtd{&|$\dots$&|} (i.e., \GenCmdBox+|<code>|+) in general
% should highlight descriptions of user commands and their syntax. 

The package mainly aims at typesetting commands and descriptions of their 
syntax \emph{if the latter is ``standard \LaTeX-like"}, 
using ``meta-variables." A string to be 
typeset ``verbatim" thus is assumed to start with a single command like 
&\foo, maybe followed by stars (\lq`*'\rq) and pairs of 
square brackets (\lq`['<opt-arg>`]'\rq) 
or curly braces (\lq`{'<mand-arg>`}'\rq), 
where those pairs contain strings indicating the typical 
kinds of contents for the respective arguments of that command.
A typical example is this: 
\[\InlineCmdBox{&\foo*[<opt-arg>]{<mand-arg>}}\]
This was achieved by typing 
\[\HardVerbBox+&\foo*[<opt-arg>]{<mand-arg>}+\]
In ``auto mode" of the package, even typing 
\[\HardVerbBox+\foo*[<opt-arg>]{<mand-arg>}+\]
would have sufficed---WYSIWYG! I call such mixtures of 
\emph{verbatim} and ``meta-variables" \textit{\qtd{meta-code}}.

Outside macro arguments, you obtain the same by typing 
% \[\verb+`\foo*[<opt-arg>]{<mand-arg>}'+\]
\[\HardVerbBox+`\foo*[<opt-arg>]{<mand-arg>}'+\]

Details:
\begin{description}

\item[``Meta-variables:"] The package supports the ``angle 
brackets" style of ``meta-variables" (as with <meta-variable>). 
You just type \lq\verb'<bar>'\rq\ to get \lq<bar>\rq.

This works due to a sloppy variant `\NVerb' of `\verb'
which doesn't care about possible ligatures and definitions of active 
characters. Instead, it assumes that the ``verbatim" font doesn't 
contain ligatures anyway.\footnote{On the other hand, &\NVerb is more 
  \emph{careful} with 'niceverb''s special characters.}
\lq\verb'\verb+<foo>+'\rq, by contrast, just yields \lq\verb'<foo>'\rq.

Almost the same feature is offered by 'ltxguide.cls' which formats the 
basic guides from the \LaTeX\ Project Team. The present feature, 
however, also works in plain text outside verbatim mode. 
% On the other hand: without << feature

\item[Single quotes (left/right) for ``short verb:"]
The package ``assumes" that \emph{quoting} refers to 
\emph{code}, therefore \lq\verb+`foo'+\rq\ is typeset as 
\lq`foo'\rq, or (generally) |`<content>'| turns <content> 
into meta-code with the meta-variable feature as above. 
This somewhat resembles the &\MakeShortVerb feature of 'doc.sty'.
%% Moved up here 2010/02/28:
You can ``abuse" our %%% ``single quotes" 
feature just to get typewriter 
typeface.{\sloppy\par}%% not so useful here 2010/02/28:
% \footnote{In macro arguments this requires that the right 
% single quote &' is &\active.}

Problems with this feature will typically arise %%% fail %% 2010/02/28
when you try 
to typeset commands (and their syntax) in \emph{macro arguments}---e.g., 
$$\verb+\footnote{`\bar' is a celebrated fake example!}+$$
will try to \emph{execute} &\bar instead of typesetting it, giving 
an ``undefined" error or so. %% TODO try! 2010/02/28
\verb+\verb+ fails in the same situation, for the same reason. 
\lq\verb+&+\rq\ (&\footnote{&&&\bar<remaining>}) or 
``auto mode" (see below) may then work better.\footnote{&\bar indeed!} 
More generally, the quoting feature still works in macro arguments in 
the sense that you then have to mark difficult characters with `&' 
(simply as short for `\string'). However, it still won't work with 
curly braces that don't follow a command name 
(such \emph{pairs} of braces will simply get lost, 
 \emph{single} braces will give errors or so).%%%\footnote{`{group}'}

Double quotes and apostrophes should still work the usual way.
% %% TODO doesn't work, inside runs into `}' 2010/02/28:
% otherwise you could control the parsing mechanisms using curly braces 
% (outside and inside don't interact: `Harry{'}s' for \qtd{Harry's}).
For difficult cases, you can still use the standard `\verb' 
command from \LaTeX.
To get \emph{usual} single quotes, you can use their standard substitutes 
`\lq' and `\rq', or for pairs of them, 
|\qtd{<text>}| in place of `\lq <text>\rq'---or even `\lq <text>\rq\ '. 

\item[Single right quotes for &\textsf:]
Package names are (by some convention I often yet not always 
 see working) 
typeset with `\textsf'; 
it was natural to use a remaining case of using single quotes 
for abbreviating $$&\textsf{<text>}$$ by |'<text>'|.
% \footnote{%
% Font switching by sequences of single quotes is a feature of the 
% syntax for editing \textit{Wikipedia} pages and of 'wiki.sty'.}
%% <- undoubled 2010/02/28 ->
This idea of switching fonts continues font switching of 'wiki.sty'
which uses the syntax for editing {\it Wikipedia} pages 
(font switching by sequences of right single quotes).

\item[Verticals for setting-off command descriptions:]%%%
\hskip0pt plus 2em
\GenCmdBox+|<code>|+ works like \qtd{&`<code>&'} except putting 
the result into a \emph{framed box} (just as all around 
here)---or something else that you can achieve using some \emph{hooks} 
described with the implementation. There are variants like 
\GenCmdBox+\cmdboxitem|<code>|+.

\item[Ampersand shows command syntax \&c. even in arguments:]
\hfil E.g., type \lq\verb+&\foo{<arg>}+\rq\ to get 
\lq`\foo{<arg>}'\rq. This may be even more convenient for typing than 
the single quotes method, although looking somewhat strange.
However, in macro arguments this does not work with 
\emph{private letters} (`@' and `_' here), for this case, 
use |\cs{<characters>}| or |\cstx{<characters>}<parameters>|.%%%
% `&' may terminate \textit{verbatim} unexpectedly, being designed for 
% displaying ``\LaTeX-like command syntax" in the first instance.
\footnote{Moreover, && currently has a limited 'xspace' 
functionality only.}%%%\footnote{You can even use && for referring to 
%   active characters like && in footnotes etc.!}
%% <- said elsewhere now 2010/03/07

\begin{sloppypar}
This choice of `&' rests on the assumption that there won't be many 
tables in the documenation. You can restore the usual meaning of `&' 
by `\MakeNormal\&' and turn the present special meaning on again by 
\[`\MakeActive\&' \mbox{\quad or\quad } 
  `\MakeActiveLet\&\CmdSyntaxVerb'\]
You could also 
redefine (&\renewcommand) &\descriptionlabel using `\CmdSyntaxVerb' 
(the ``normal command" that is equivalent to `&', its ``permanent 
 alias") 
so \verb+\item[\foo]+ works as wanted.
\end{sloppypar}

\textbf{Another} feature of 'niceverb''s `&' is getting 
(some of the) special characters    %% 2010/03/20
(as listed in the standard macro `\dospecials') verbatim in arguments 
(where `\verb' and the like fail). It just acts similarly as \TeX's 
(as listed in the standard macro `\dospecials') verbatim in arguments 
(where `\verb' and the like fail). It just acts similarly as \TeX's 
 primitive `\string' (which it actually invokes---cf. discussion on the 
 left quote feature above). 

\item[``Auto mode" typesets commands verbatim unless .\,.\,.]
\begin{sloppypar}
In~``auto mode," the backslash \lq`\'\rq\ is an active character that 
builds a command name from the ensuing letters and typesets the 
command (and its syntax, allowing meta-variables) verbatim. 
However, there are some exceptions, which are collected in a macro 
|\niceverbNoVerbList|. &\begin, &\end, and &\item belong to this list, 
you can redefine (`\renewcommand') it, or add <macros> to it by
|\AddToMacro{\niceverbNoVerbList}{<macros>}|
There is also a command |\NormalCommand{<letters>}| \emph{issuing} the 
command `\<letters>' instead of typesetting it.
Since auto mode is somewhat dangerous, you have to start it explicitly 
by |\AutoCmdSyntaxVerb|. You can end it by |\EndAutoCmdSyntaxVerb|.
|\AutoCmdInput{<file>}| is probably most important. 
\end{sloppypar}

Auto mode is motivated by the observation that there are package files 
containing their documentation as pure (well-readable) ASCII 
text---contain\-ing the names of the new commands without any kind of 
quotation marks or verbatim commands. 
Auto mode should typeset such documentation just from the same ASCII 
text.

\item[Hash mark \lq&#\rq\ comes verbatim.]
No macro definitions are expected in the `document' 
environment.\footnote{This idea appeared 2009 on the 'LATEX-L' 
                      mailing list. It may be wrong, 
                      as I have sometimes experienced $\dots$}
                      %% <- changed 2010/03/11
Rather, \lq`#'\rq\ is an active character for taking the next 
character (assuming it is a digit) to form a reference to a 
\emph{macro parameter}---\lq`#1'\rq\ becomes \lq#1\rq\---WYSIWYG 
indeed! (So the general syntax is |#<digit>|.)
\item[Escaping from 'niceverb' (generally).] 
     To get rid of the functionality of some active character <char> 
     (\qtd{&&}, single quote, ampersand, hash mark---not 
      ``auto mode," see above) here, use |\MakeNormal\<char>|---may 
     be within a group. To revive it again, use |\MakeActive\<char>|. 
     This may fail when a different package overtook the active <char> 
     (but I expect more failures then), in this case 
     |\MakeActiveLet\<char>\<perm-alias>| 
     revives the 'niceverb' meaning of <char>
     where `\<perm-alias>' is the ``permanent alias" for that active 
     <char> according to the documentation below. 
     E.g., `\LQverb' is the ``permanent alias" for active single left 
     quote, 'niceverb' activates it by 
     \NVerb+\MakeActiveLet\'\LQverb+.---You can turn off 'niceverb' 
     syntax \emph{alltogether} by |\noNiceVerb| and revive it 
     by |\useNiceVerb| (without ``auto mode").{\sloppy\par}

     \textbf{Right Quotes:} Disabling\slash reviving replacement 
     of `\textsf' by single right quotes requires 
     \[|\nvRightQuoteNormal| \mbox{\quad or\quad } |\nvRightQuoteSansSerif|\] 
     respectively.
\end{description}

\subsection{Examples}
The file 'mdoccorr.cfg' providing some `.txt'$\to$\LaTeX\ 
functionality---i.e., typographical corrections---documents itself 
using 'niceverb' syntax. Its code and the documentation that is 
typeset from it are in the \qtd{examples} section of 
'makedoc.pdf'.---Moreover, 
the documentation 'niceverb.pdf' of 'niceverb.sty' was 
typeset from 'niceverb.tex' and 'niceverb.sty' using 'niceverb' 
syntax, likewise 'fifinddo.pdf' and 'makedoc.pdf'. 
The example of 'niceverb' shows the most frequent use of the `&' 
feature.{\sloppy\par}

'nicetext' bundle release v0.4 contains a file 'substr.tex' 
that should typeset the documentation of the version of 
Harald Harders'
'substr.sty'\footnote{\url{http://ctan.org/pkg/substr}}
that your \TeX\ finds first, as well as 'arseneau.tex' 
typesetting a few packages by Donald Arseneau. 
The outcomes (with me) are 'substr.pdf' and 'arseneau.pdf'.
These are the first applications of 'niceverb''s ``auto mode" to 
(unmodified) third-party package files.
(I also made a more ambitious documentation of Donald Arseneau's 
 'import.sty v3.0' before I found that CTAN already has a nicely 
 typeset documentation of 'import.sty v5.2'.)

%% removed 2010/03/11:
% It seems to me that I could type so many pages on 'fifinddo' and 
% 'makedoc' in little more than a week 
% % (2009/04/12, much of which was needed for debugging and reworking concepts) 
% only due to the ``minimal" \emph{verbatim} and syntax-display syntax. 
% 
\subsection{What is Wrong with the Present Version}
\begin{enumerate}
\item 'niceverb.sty' should be an extension of 'wiki.sty'; 
      yet their font selection mechanisms are currently not compatible. 
      %% 2010/02/28:
      Especially, the feature of \[\hbox\bgroup|''<text>''|\egroup\] 
      %% <- failed with \mbox as of 2010/03/23, first two rq missing 
      %%    2010/03/29
      replacing 
      `\textit{<text>}' or `\emph{<text>}' may be considered missing. 
\item Font switching or horizontal spacing may fail in certain 
      situations.
%       (parentheses, titles, footnotes; 
      You can correct spacing by \lq`\ '\rq. 
        %% <- \qtd{`&\ '}.
% \item 
% The feature of mixing high-quality-typeset comments into the 
% package code listing is implemented in a very rudimentary way only. 
% % just allowing for `\subsection's. 
% The ``comment detector" detects Wikipedia-style subsection titles 
% instead of lines beginning with percent characters.\footnote{%
% Percent characters will definitely not be ``ignored" as with &\DocInput, 
% rather they will hide rests of \emph{documentation} lines as usually, 
% while they will be typeset verbatim in \emph{package code} lines.} 
% Switching between plain and verbatim typesetting in the package 
% listings isn't settled yet, since there are different styles of using 
% percent symbols. I have sometimes used double percent symbols 
% (\lq\verb+%%+\rq) 
% for commenting text and single ones just for ``reversible deletion of 
% code," while usually single percent symbols indicate commenting text 
% indeed. Double percent symbols may, by contrast, mean that the text remains 
% visible in the `.sty' file only, suppressed in the typeset 
% documentation ('lineno.sty').
% For a while, it may be necessary to provide replacing macros for each 
% package separately instead of providing a single macro package 
% managing all of them. 
% \item 
% The code listing currently uses the `listing' and `listingcont' 
% environments of 'moreverb.sty'; 
% the code font and the line numbers may be too large. 
\item The ``vertical" character \qtd{&|} produces inline boxes 
      only at present. It might as well provide a version of the 
      `decl' tabular environment of 'ltxguide.cls'. 
%% changes 2010/03/10
%       coloured\slash framed boxes instead (2009/04/09). They have 
%       their merits! See 'fifinddo.pdf'  and 'makedoc.pdf'. However, 
%       they 
      The inline boxes
      badly deal with long command names and many arguments.
      Doubled verticals could ensure the `decl' mode. 
      Moreover, such a box might issue an \emph{index} entry.
\item One may have \emph{opposite} ideas about using quotes---maybe 
      rather `"<code>"' should typeset <code> \textit{verbatim}.
      There might be a package option for this. If ordinary 
      \qtd{\NVerb'``<text>"'} still should work, awful tricks as now with 
      the right quote feature would be needed. %% TODO 2010/03/06
% \item ``Auto mode" has \emph{not} been tested on a serious application yet. 
%% partially improved 2010/02/28:
% \item % 'niceverb''s font switching tricks sometimes turn against their 
%       % inventor (and other users?). There must be some switching 
%       % ``off'' (and ``on'' again).%
%       %   \footnote{\hspace{1sp}'fifinddo'\slash\hspace{1sp}'makedoc'
%       %     %% <- TODO oh, oh! 2009/04/11
%       %     allow inserting such commands from a driver script, 
%       %     invisible in the file that contains the ``contentual'' 
%       %     documentation.}
%       % Also, there 
%       There
%       might better help with weird errors, 
%       some syntax checks might intercept earlier. 
% 
%       Similarly, some choices reflect a %% rather OK 2010/02/28
%       personal style and should be modifiable, especially by package 
%       options.\footnote{Please sponsor the project or support it 
%         otherwise!}
\item Certain difficulties with typesetting code in macro arguments 
      may be overcome easily using $\varepsilon$\mbox{-}\TeX\ 
      features, I need to find out $\dots$
\end{enumerate}

\section{Implementation of the Markup Syntax}
\typeout{niceverb.tex 2010/04/05 documenting niceverb.sty}
\title{\textsf{niceverb.sty}\\---\\Minimizing 
  Markup\\for Documenting \LaTeX\ packages%%% \thanks{This 
%     manual describes package version
%     version 0.2 as of April 09, 2009%%%\fileversion\ as of \filedate\ 
%     .}}%%%of the package.}%
}
% \listfiles 2010/03/19
{ \RequirePackage{makedoc} \ProcessLineMessage{} %% 2010/03/11
  \MakeJobDoc{19}{\SectionLevelThreeParseInput}  }
\documentclass{article}%% TODO paper dimensions!?
\input{makedoc.cfg} %% shared formatting settings
\begin{document}
\maketitle
\begin{abstract}\noindent
'niceverb.sty' provides very decent syntax (through active characters) 
for describing \LaTeX\ packages and the syntax of macros conforming to 
\LaTeX\ syntax conventions.
\end{abstract}
\tableofcontents

  %% TODO table listing of active characters
%% Were tests 2010/03/08:
% \section{Presenting Nasty's `Nasty' ``Nasty'' &\NVerb\ 'niceverb'}
% \section{Presenting \cs{NVerb} 'niceverb'}
\section{Presenting 'niceverb'}
\subsection{Purpose}
% \begin{abstract}\noindent
% The 'nicetext' bundle provides ``minimal" markup 
The 'niceverb' package provides ``minimal" markup for documenting \LaTeX\ 
packages, reducing the number of keystrokes/visible characters needed
% .\,.\,. %%% ... %% TODO nicedots 
(kind of poor man's WYSIWYG).\footnote{``What you see is what you 
  get." Novices are always warned that WYSIWYG is essentially 
  impossible with \LaTeX.} %% TODO UK FAQ 2010/03/11
% One feature---\verb'&\foo'%%% badly self-documenting, `&' fails
It conveniently handles command names in arguments of macros 
such as &\footnote or even of sectioning commands. 
% (`.aux'/`.toc' entries).
% 
% This is done by making some characters active. 
% 'niceverb.sty' thus resembles 'wiki.sty'; both are siblings. 
% \end{abstract}
If you use 'makedoc.sty' additionally, commands for typesetting a 
package's code are inserted automatically (just using \TeX). 
%%% \footnote{Stephan I. B\"ottcher used
%%% 'awk' instead to typeset the documentation of his 'lineno.sty'.} 
As opposed to tools that are rather common on UNIX/Linux, this 
operation should work at any \TeX\ installation, irrespective of 
platform.

Both packages may at least be useful while working at a very new package 
and may suffice with small, simple packages. After having edited your 
package's code 
%% <jobname> 2010/02/28:
(typically in a `.sty' file---<jobname>`.sty'), 
you just ``{`latex'}" the manual file 
(maybe some `.tex' file---<jobname>`.tex') 
and get instantly the corresponding updated documentation.

'niceverb' and 'makedoc' may also help to generate without much effort 
documentations of nowadays commonly expected typographical quality for 
packages that so far only had plain text documentations.

\subsection{Acknowledgement/Basic Ideas}
Three ideas of Stephan I. B\"ottcher's in documenting his 
\ctanpkgref{lineno}
inspired the present work: 
\begin{enumerate}
\item 
The markup and its definitions are short and simple, 
markup commands are placed at the right ``margin" 
of the ASCII file, 
so you hardly see them in reading the source file, 
you rather just read the text that will be printed. 
\item 
An 'awk' script removes the `%'s starting \emph{documentation} lines 
and inserts the commands for typesetting the package's \emph{code} 
(you don't see these commands in the source).\footnote{The 
  corresponding part of the ``present work" is 'makedoc.sty'.} 
  %% <- clarified 2010/03/11
\item 
An active character (\lq&|\rq) issues a `\string' \emph{and} switches 
to typewriter typeface for typesetting a command verbatim---so this 
works without changing category codes (which is the usual idea of 
typesetting code), therefore it works even in macro arguments.
\end{enumerate}

\subsection{The Commands and Features of 'niceverb'}
Actually, it is the main purpose of 'niceverb' to save you from 
``commands" $\dots$\par
Single quotes &`, &', ``less than" &< (accompanied 
with `>'), the ``vertical" &|, the hash mark `#', ampersand `&', 
and in an extended ``auto mode" even backslash `\' become `\active'
characters with ``special effects." 
% \qtd{&|$\dots$&|} (i.e., \GenCmdBox+|<code>|+) in general
% should highlight descriptions of user commands and their syntax. 

The package mainly aims at typesetting commands and descriptions of their 
syntax \emph{if the latter is ``standard \LaTeX-like"}, 
using ``meta-variables." A string to be 
typeset ``verbatim" thus is assumed to start with a single command like 
&\foo, maybe followed by stars (\lq`*'\rq) and pairs of 
square brackets (\lq`['<opt-arg>`]'\rq) 
or curly braces (\lq`{'<mand-arg>`}'\rq), 
where those pairs contain strings indicating the typical 
kinds of contents for the respective arguments of that command.
A typical example is this: 
\[\InlineCmdBox{&\foo*[<opt-arg>]{<mand-arg>}}\]
This was achieved by typing 
\[\HardVerbBox+&\foo*[<opt-arg>]{<mand-arg>}+\]
In ``auto mode" of the package, even typing 
\[\HardVerbBox+\foo*[<opt-arg>]{<mand-arg>}+\]
would have sufficed---WYSIWYG! I call such mixtures of 
\emph{verbatim} and ``meta-variables" \textit{\qtd{meta-code}}.

Outside macro arguments, you obtain the same by typing 
% \[\verb+`\foo*[<opt-arg>]{<mand-arg>}'+\]
\[\HardVerbBox+`\foo*[<opt-arg>]{<mand-arg>}'+\]

Details:
\begin{description}

\item[``Meta-variables:"] The package supports the ``angle 
brackets" style of ``meta-variables" (as with <meta-variable>). 
You just type \lq\verb'<bar>'\rq\ to get \lq<bar>\rq.

This works due to a sloppy variant `\NVerb' of `\verb'
which doesn't care about possible ligatures and definitions of active 
characters. Instead, it assumes that the ``verbatim" font doesn't 
contain ligatures anyway.\footnote{On the other hand, &\NVerb is more 
  \emph{careful} with 'niceverb''s special characters.}
\lq\verb'\verb+<foo>+'\rq, by contrast, just yields \lq\verb'<foo>'\rq.

Almost the same feature is offered by 'ltxguide.cls' which formats the 
basic guides from the \LaTeX\ Project Team. The present feature, 
however, also works in plain text outside verbatim mode. 
% On the other hand: without << feature

\item[Single quotes (left/right) for ``short verb:"]
The package ``assumes" that \emph{quoting} refers to 
\emph{code}, therefore \lq\verb+`foo'+\rq\ is typeset as 
\lq`foo'\rq, or (generally) |`<content>'| turns <content> 
into meta-code with the meta-variable feature as above. 
This somewhat resembles the &\MakeShortVerb feature of 'doc.sty'.
%% Moved up here 2010/02/28:
You can ``abuse" our %%% ``single quotes" 
feature just to get typewriter 
typeface.{\sloppy\par}%% not so useful here 2010/02/28:
% \footnote{In macro arguments this requires that the right 
% single quote &' is &\active.}

Problems with this feature will typically arise %%% fail %% 2010/02/28
when you try 
to typeset commands (and their syntax) in \emph{macro arguments}---e.g., 
$$\verb+\footnote{`\bar' is a celebrated fake example!}+$$
will try to \emph{execute} &\bar instead of typesetting it, giving 
an ``undefined" error or so. %% TODO try! 2010/02/28
\verb+\verb+ fails in the same situation, for the same reason. 
\lq\verb+&+\rq\ (&\footnote{&&&\bar<remaining>}) or 
``auto mode" (see below) may then work better.\footnote{&\bar indeed!} 
More generally, the quoting feature still works in macro arguments in 
the sense that you then have to mark difficult characters with `&' 
(simply as short for `\string'). However, it still won't work with 
curly braces that don't follow a command name 
(such \emph{pairs} of braces will simply get lost, 
 \emph{single} braces will give errors or so).%%%\footnote{`{group}'}

Double quotes and apostrophes should still work the usual way.
% %% TODO doesn't work, inside runs into `}' 2010/02/28:
% otherwise you could control the parsing mechanisms using curly braces 
% (outside and inside don't interact: `Harry{'}s' for \qtd{Harry's}).
For difficult cases, you can still use the standard `\verb' 
command from \LaTeX.
To get \emph{usual} single quotes, you can use their standard substitutes 
`\lq' and `\rq', or for pairs of them, 
|\qtd{<text>}| in place of `\lq <text>\rq'---or even `\lq <text>\rq\ '. 

\item[Single right quotes for &\textsf:]
Package names are (by some convention I often yet not always 
 see working) 
typeset with `\textsf'; 
it was natural to use a remaining case of using single quotes 
for abbreviating $$&\textsf{<text>}$$ by |'<text>'|.
% \footnote{%
% Font switching by sequences of single quotes is a feature of the 
% syntax for editing \textit{Wikipedia} pages and of 'wiki.sty'.}
%% <- undoubled 2010/02/28 ->
This idea of switching fonts continues font switching of 'wiki.sty'
which uses the syntax for editing {\it Wikipedia} pages 
(font switching by sequences of right single quotes).

\item[Verticals for setting-off command descriptions:]%%%
\hskip0pt plus 2em
\GenCmdBox+|<code>|+ works like \qtd{&`<code>&'} except putting 
the result into a \emph{framed box} (just as all around 
here)---or something else that you can achieve using some \emph{hooks} 
described with the implementation. There are variants like 
\GenCmdBox+\cmdboxitem|<code>|+.

\item[Ampersand shows command syntax \&c. even in arguments:]
\hfil E.g., type \lq\verb+&\foo{<arg>}+\rq\ to get 
\lq`\foo{<arg>}'\rq. This may be even more convenient for typing than 
the single quotes method, although looking somewhat strange.
However, in macro arguments this does not work with 
\emph{private letters} (`@' and `_' here), for this case, 
use |\cs{<characters>}| or |\cstx{<characters>}<parameters>|.%%%
% `&' may terminate \textit{verbatim} unexpectedly, being designed for 
% displaying ``\LaTeX-like command syntax" in the first instance.
\footnote{Moreover, && currently has a limited 'xspace' 
functionality only.}%%%\footnote{You can even use && for referring to 
%   active characters like && in footnotes etc.!}
%% <- said elsewhere now 2010/03/07

\begin{sloppypar}
This choice of `&' rests on the assumption that there won't be many 
tables in the documenation. You can restore the usual meaning of `&' 
by `\MakeNormal\&' and turn the present special meaning on again by 
\[`\MakeActive\&' \mbox{\quad or\quad } 
  `\MakeActiveLet\&\CmdSyntaxVerb'\]
You could also 
redefine (&\renewcommand) &\descriptionlabel using `\CmdSyntaxVerb' 
(the ``normal command" that is equivalent to `&', its ``permanent 
 alias") 
so \verb+\item[\foo]+ works as wanted.
\end{sloppypar}

\textbf{Another} feature of 'niceverb''s `&' is getting 
(some of the) special characters    %% 2010/03/20
(as listed in the standard macro `\dospecials') verbatim in arguments 
(where `\verb' and the like fail). It just acts similarly as \TeX's 
(as listed in the standard macro `\dospecials') verbatim in arguments 
(where `\verb' and the like fail). It just acts similarly as \TeX's 
 primitive `\string' (which it actually invokes---cf. discussion on the 
 left quote feature above). 

\item[``Auto mode" typesets commands verbatim unless .\,.\,.]
\begin{sloppypar}
In~``auto mode," the backslash \lq`\'\rq\ is an active character that 
builds a command name from the ensuing letters and typesets the 
command (and its syntax, allowing meta-variables) verbatim. 
However, there are some exceptions, which are collected in a macro 
|\niceverbNoVerbList|. &\begin, &\end, and &\item belong to this list, 
you can redefine (`\renewcommand') it, or add <macros> to it by
|\AddToMacro{\niceverbNoVerbList}{<macros>}|
There is also a command |\NormalCommand{<letters>}| \emph{issuing} the 
command `\<letters>' instead of typesetting it.
Since auto mode is somewhat dangerous, you have to start it explicitly 
by |\AutoCmdSyntaxVerb|. You can end it by |\EndAutoCmdSyntaxVerb|.
|\AutoCmdInput{<file>}| is probably most important. 
\end{sloppypar}

Auto mode is motivated by the observation that there are package files 
containing their documentation as pure (well-readable) ASCII 
text---contain\-ing the names of the new commands without any kind of 
quotation marks or verbatim commands. 
Auto mode should typeset such documentation just from the same ASCII 
text.

\item[Hash mark \lq&#\rq\ comes verbatim.]
No macro definitions are expected in the `document' 
environment.\footnote{This idea appeared 2009 on the 'LATEX-L' 
                      mailing list. It may be wrong, 
                      as I have sometimes experienced $\dots$}
                      %% <- changed 2010/03/11
Rather, \lq`#'\rq\ is an active character for taking the next 
character (assuming it is a digit) to form a reference to a 
\emph{macro parameter}---\lq`#1'\rq\ becomes \lq#1\rq\---WYSIWYG 
indeed! (So the general syntax is |#<digit>|.)
\item[Escaping from 'niceverb' (generally).] 
     To get rid of the functionality of some active character <char> 
     (\qtd{&&}, single quote, ampersand, hash mark---not 
      ``auto mode," see above) here, use |\MakeNormal\<char>|---may 
     be within a group. To revive it again, use |\MakeActive\<char>|. 
     This may fail when a different package overtook the active <char> 
     (but I expect more failures then), in this case 
     |\MakeActiveLet\<char>\<perm-alias>| 
     revives the 'niceverb' meaning of <char>
     where `\<perm-alias>' is the ``permanent alias" for that active 
     <char> according to the documentation below. 
     E.g., `\LQverb' is the ``permanent alias" for active single left 
     quote, 'niceverb' activates it by 
     \NVerb+\MakeActiveLet\'\LQverb+.---You can turn off 'niceverb' 
     syntax \emph{alltogether} by |\noNiceVerb| and revive it 
     by |\useNiceVerb| (without ``auto mode").{\sloppy\par}

     \textbf{Right Quotes:} Disabling\slash reviving replacement 
     of `\textsf' by single right quotes requires 
     \[|\nvRightQuoteNormal| \mbox{\quad or\quad } |\nvRightQuoteSansSerif|\] 
     respectively.
\end{description}

\subsection{Examples}
The file 'mdoccorr.cfg' providing some `.txt'$\to$\LaTeX\ 
functionality---i.e., typographical corrections---documents itself 
using 'niceverb' syntax. Its code and the documentation that is 
typeset from it are in the \qtd{examples} section of 
'makedoc.pdf'.---Moreover, 
the documentation 'niceverb.pdf' of 'niceverb.sty' was 
typeset from 'niceverb.tex' and 'niceverb.sty' using 'niceverb' 
syntax, likewise 'fifinddo.pdf' and 'makedoc.pdf'. 
The example of 'niceverb' shows the most frequent use of the `&' 
feature.{\sloppy\par}

'nicetext' bundle release v0.4 contains a file 'substr.tex' 
that should typeset the documentation of the version of 
Harald Harders'
'substr.sty'\footnote{\url{http://ctan.org/pkg/substr}}
that your \TeX\ finds first, as well as 'arseneau.tex' 
typesetting a few packages by Donald Arseneau. 
The outcomes (with me) are 'substr.pdf' and 'arseneau.pdf'.
These are the first applications of 'niceverb''s ``auto mode" to 
(unmodified) third-party package files.
(I also made a more ambitious documentation of Donald Arseneau's 
 'import.sty v3.0' before I found that CTAN already has a nicely 
 typeset documentation of 'import.sty v5.2'.)

%% removed 2010/03/11:
% It seems to me that I could type so many pages on 'fifinddo' and 
% 'makedoc' in little more than a week 
% % (2009/04/12, much of which was needed for debugging and reworking concepts) 
% only due to the ``minimal" \emph{verbatim} and syntax-display syntax. 
% 
\subsection{What is Wrong with the Present Version}
\begin{enumerate}
\item 'niceverb.sty' should be an extension of 'wiki.sty'; 
      yet their font selection mechanisms are currently not compatible. 
      %% 2010/02/28:
      Especially, the feature of \[\hbox\bgroup|''<text>''|\egroup\] 
      %% <- failed with \mbox as of 2010/03/23, first two rq missing 
      %%    2010/03/29
      replacing 
      `\textit{<text>}' or `\emph{<text>}' may be considered missing. 
\item Font switching or horizontal spacing may fail in certain 
      situations.
%       (parentheses, titles, footnotes; 
      You can correct spacing by \lq`\ '\rq. 
        %% <- \qtd{`&\ '}.
% \item 
% The feature of mixing high-quality-typeset comments into the 
% package code listing is implemented in a very rudimentary way only. 
% % just allowing for `\subsection's. 
% The ``comment detector" detects Wikipedia-style subsection titles 
% instead of lines beginning with percent characters.\footnote{%
% Percent characters will definitely not be ``ignored" as with &\DocInput, 
% rather they will hide rests of \emph{documentation} lines as usually, 
% while they will be typeset verbatim in \emph{package code} lines.} 
% Switching between plain and verbatim typesetting in the package 
% listings isn't settled yet, since there are different styles of using 
% percent symbols. I have sometimes used double percent symbols 
% (\lq\verb+%%+\rq) 
% for commenting text and single ones just for ``reversible deletion of 
% code," while usually single percent symbols indicate commenting text 
% indeed. Double percent symbols may, by contrast, mean that the text remains 
% visible in the `.sty' file only, suppressed in the typeset 
% documentation ('lineno.sty').
% For a while, it may be necessary to provide replacing macros for each 
% package separately instead of providing a single macro package 
% managing all of them. 
% \item 
% The code listing currently uses the `listing' and `listingcont' 
% environments of 'moreverb.sty'; 
% the code font and the line numbers may be too large. 
\item The ``vertical" character \qtd{&|} produces inline boxes 
      only at present. It might as well provide a version of the 
      `decl' tabular environment of 'ltxguide.cls'. 
%% changes 2010/03/10
%       coloured\slash framed boxes instead (2009/04/09). They have 
%       their merits! See 'fifinddo.pdf'  and 'makedoc.pdf'. However, 
%       they 
      The inline boxes
      badly deal with long command names and many arguments.
      Doubled verticals could ensure the `decl' mode. 
      Moreover, such a box might issue an \emph{index} entry.
\item One may have \emph{opposite} ideas about using quotes---maybe 
      rather `"<code>"' should typeset <code> \textit{verbatim}.
      There might be a package option for this. If ordinary 
      \qtd{\NVerb'``<text>"'} still should work, awful tricks as now with 
      the right quote feature would be needed. %% TODO 2010/03/06
% \item ``Auto mode" has \emph{not} been tested on a serious application yet. 
%% partially improved 2010/02/28:
% \item % 'niceverb''s font switching tricks sometimes turn against their 
%       % inventor (and other users?). There must be some switching 
%       % ``off'' (and ``on'' again).%
%       %   \footnote{\hspace{1sp}'fifinddo'\slash\hspace{1sp}'makedoc'
%       %     %% <- TODO oh, oh! 2009/04/11
%       %     allow inserting such commands from a driver script, 
%       %     invisible in the file that contains the ``contentual'' 
%       %     documentation.}
%       % Also, there 
%       There
%       might better help with weird errors, 
%       some syntax checks might intercept earlier. 
% 
%       Similarly, some choices reflect a %% rather OK 2010/02/28
%       personal style and should be modifiable, especially by package 
%       options.\footnote{Please sponsor the project or support it 
%         otherwise!}
\item Certain difficulties with typesetting code in macro arguments 
      may be overcome easily using $\varepsilon$\mbox{-}\TeX\ 
      features, I need to find out $\dots$
\end{enumerate}

\section{Implementation of the Markup Syntax}
\input{niceverb.doc}
\end{document}

HISTORY

2009/04/09  adjusted to new doc-generation method
2009/04/12  examples, 'awk' lower-case
2009/04/15  example 'mdcorr.cfg', abstract, 
            \pagebreak to implementation
2010/02/27  replaced `|' by `+' with \verb 
            so `|' works as announced
2010/02/28  "Missing:" ''...'' 'wiki' feature, 
            somethings aren't missing anymore 
            (or otherwise removed); more on quotes; 
            applying |...| 
2010/03/05  \SimpleVerb -> \NVerb; after intro on `&' quotes as well
2010/03/06  typo in ``examples''; removed makedoc.cfg sample; 
            more on `&'
2010/03/07  without \listfiles
2010/03/09  hyperref ... \input{mdcorr.cfg}!, |...| settled
2010/03/10  moved pdf stuff to 'makedoc.cfg'; 
            do use 'mdcorr.cfg' for demo; future of |
2010/03/11  applied \MakeJobDoc and shortened preamble; 
            various minor doc changes
2010/03/12  ``Ampersand" improved; \noNiceVerb + \useNiceVerb
2010/03/14  use \InlineCmdBox and \HardVerbBox; |...| described
2010/03/18  \AddToMacro; ``auto mode" tested seriously (substr.sty) 
            - \AutoCmdInput
2010/03/19  line break changes; '' -> " 
2010/03/20  testing niceverb v0.31
2010/03/23  `mdoccorr.cfg' example again
2010/03/27  ``auto mode,"
2010/03/29  \mbox -> \hbox in display; arseneau.tex/pdf
2010/04/05  Harder -> Harders

\end{document}

HISTORY

2009/04/09  adjusted to new doc-generation method
2009/04/12  examples, 'awk' lower-case
2009/04/15  example 'mdcorr.cfg', abstract, 
            \pagebreak to implementation
2010/02/27  replaced `|' by `+' with \verb 
            so `|' works as announced
2010/02/28  "Missing:" ''...'' 'wiki' feature, 
            somethings aren't missing anymore 
            (or otherwise removed); more on quotes; 
            applying |...| 
2010/03/05  \SimpleVerb -> \NVerb; after intro on `&' quotes as well
2010/03/06  typo in ``examples''; removed makedoc.cfg sample; 
            more on `&'
2010/03/07  without \listfiles
2010/03/09  hyperref ... \input{mdcorr.cfg}!, |...| settled
2010/03/10  moved pdf stuff to 'makedoc.cfg'; 
            do use 'mdcorr.cfg' for demo; future of |
2010/03/11  applied \MakeJobDoc and shortened preamble; 
            various minor doc changes
2010/03/12  ``Ampersand" improved; \noNiceVerb + \useNiceVerb
2010/03/14  use \InlineCmdBox and \HardVerbBox; |...| described
2010/03/18  \AddToMacro; ``auto mode" tested seriously (substr.sty) 
            - \AutoCmdInput
2010/03/19  line break changes; '' -> " 
2010/03/20  testing niceverb v0.31
2010/03/23  `mdoccorr.cfg' example again
2010/03/27  ``auto mode,"
2010/03/29  \mbox -> \hbox in display; arseneau.tex/pdf
2010/04/05  Harder -> Harders

\end{document}

HISTORY

2009/04/09  adjusted to new doc-generation method
2009/04/12  examples, 'awk' lower-case
2009/04/15  example 'mdcorr.cfg', abstract, 
            \pagebreak to implementation
2010/02/27  replaced `|' by `+' with \verb 
            so `|' works as announced
2010/02/28  "Missing:" ''...'' 'wiki' feature, 
            somethings aren't missing anymore 
            (or otherwise removed); more on quotes; 
            applying |...| 
2010/03/05  \SimpleVerb -> \NVerb; after intro on `&' quotes as well
2010/03/06  typo in ``examples''; removed makedoc.cfg sample; 
            more on `&'
2010/03/07  without \listfiles
2010/03/09  hyperref ... \input{mdcorr.cfg}!, |...| settled
2010/03/10  moved pdf stuff to 'makedoc.cfg'; 
            do use 'mdcorr.cfg' for demo; future of |
2010/03/11  applied \MakeJobDoc and shortened preamble; 
            various minor doc changes
2010/03/12  ``Ampersand" improved; \noNiceVerb + \useNiceVerb
2010/03/14  use \InlineCmdBox and \HardVerbBox; |...| described
2010/03/18  \AddToMacro; ``auto mode" tested seriously (substr.sty) 
            - \AutoCmdInput
2010/03/19  line break changes; '' -> " 
2010/03/20  testing niceverb v0.31
2010/03/23  `mdoccorr.cfg' example again
2010/03/27  ``auto mode,"
2010/03/29  \mbox -> \hbox in display; arseneau.tex/pdf
2010/04/05  Harder -> Harders

\end{document}

HISTORY

2009/04/09  adjusted to new doc-generation method
2009/04/12  examples, 'awk' lower-case
2009/04/15  example 'mdcorr.cfg', abstract, 
            \pagebreak to implementation
2010/02/27  replaced `|' by `+' with \verb 
            so `|' works as announced
2010/02/28  "Missing:" ''...'' 'wiki' feature, 
            somethings aren't missing anymore 
            (or otherwise removed); more on quotes; 
            applying |...| 
2010/03/05  \SimpleVerb -> \NVerb; after intro on `&' quotes as well
2010/03/06  typo in ``examples''; removed makedoc.cfg sample; 
            more on `&'
2010/03/07  without \listfiles
2010/03/09  hyperref ... \input{mdcorr.cfg}!, |...| settled
2010/03/10  moved pdf stuff to 'makedoc.cfg'; 
            do use 'mdcorr.cfg' for demo; future of |
2010/03/11  applied \MakeJobDoc and shortened preamble; 
            various minor doc changes
2010/03/12  ``Ampersand" improved; \noNiceVerb + \useNiceVerb
2010/03/14  use \InlineCmdBox and \HardVerbBox; |...| described
2010/03/18  \AddToMacro; ``auto mode" tested seriously (substr.sty) 
            - \AutoCmdInput
2010/03/19  line break changes; '' -> " 
2010/03/20  testing niceverb v0.31
2010/03/23  `mdoccorr.cfg' example again
2010/03/27  ``auto mode,"
2010/03/29  \mbox -> \hbox in display; arseneau.tex/pdf
2010/04/05  Harder -> Harders
