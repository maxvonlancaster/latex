\ProvidesFile{wikicheat}[2008/07/10 Cheatsheet for wiki.sty]
% \def\title{Cheatsheet for the \textsf{wiki} Package}
\def\title{\textbf{Cheatsheet} for the \textbf{\textsf{wiki}} Package}
\def\author{Uwe L\"uck}
\documentclass[12pt,a4paper]{article}
\nofiles
\pagestyle{empty}
\makeatletter
\@ifundefined{UndeclareTextCommand}{}{\usepackage{upquote}} %% 2008/07/02
\makeatother
\usepackage{wiki}
% \let\WE\relax \let\WF\relax \let\WH\relax
\let\WE\wikiEnvironments
\let\WF\wikiFonts
\let\WH\wikiHeadings
\def\SP#1{\multicolumn{3}{\VR c\VR}{% 
%   \parbox{11cm}{% 
  \parbox{13cm}{% 
    \footnotesize\it\sloppy\SU#1\SD}}}
\def\SL{/\hskip.16667em}
% \setbox0\hbox{(}\showthe\ht0\showthe\dp0 %% height = 3x depth
\def\SU{\rule{0pt}{.99em}}%%%{1.1em}}
\def\HSU{\rule{0pt}{.75em}}
\def\SD{\rule[-.33em]{0pt}{0pt}}%%%45em]{0pt}{0pt}}
% \def\CS#1{{\tt\textbackslash #1}} %% 2008/07/02
{\makeatletter \gdef\CS#1{{\tt\@backslashchar#1}}}
\let\VR| \catcode`\|\active \def|{\verb|}
\begin{document}
% \maketitle

\enlargethispage{3\normalbaselineskip}
\vspace*{-\topmargin} %% !? 
\vspace*{-\headheight}\vspace*{-\headsep}\vspace*{-\topsep}\vspace*{-\partopsep}
\begin{center}
\large \title\\[.4ex] \normalsize 
Some \LaTeX-Markup through some Wiki-Markup\\[.8ex]
            \author, \today 
% \end{center}

\vspace{\topsep}
\vspace*{\partopsep}
% \vspace{1.5ex}
% \vspace{1ex}
% \hrule
% \vspace{1ex}

% \begin{center}
\leavevmode\kern-1em
\begin{tabular}{\VR l\VR l\VR l\VR}
\hline\hline
\SU\SD\bf Code for \textsf{wiki.sty}&\bf What you get& \bf like \LaTeX-code\\
\hline\hline
|''italic''|&\WF ''italic''&|\textit{italic}|\SU\\
|'''bold'''|&\WF '''bold'''&|\textbf{bold}|%%%\\
\SD\\
% |'''''bf+it'''''|&\WF '''''bf+it'''''
%                  &|\textbf{\textit{bf+it}}|\SD\\
% |'''''B+I'''''|&\WF '''''B+I''''' &|\textbf{\textit{B+I}}|\SD\\
\hline
\SP{``Italic'' and ``bold'' may be combined, nested, overlap. 
% Automatic italic correction cannot be suppressed by \CS{nocorr}. 
% Implementation uses \LaTeX\ internals; no groups are formed. 
Quotation marks 
% (even German) 
usually are not 
affected.---Turn
% {\tt '\string{\string}''} may help with 
% apostrophes.---Turn 
font feature on/off by \CS{wikiFonts}\SL\CS{nowikiFonts}; disable feature 
entirely by package option {\tt noFonts}.}\\
\hline\hline
|== Level 1 ==|&\Large\bf\HSU 1\quad Level 1&|\section{Level 1}|\SU\\
|=== Level 2 ===|&\large\bf 1.1\quad Level 2&|\subsection{Level 2}|\SU\\
|==== Level 3 ====|&\bf 1.1.1\quad Level 3&|\subsubsection{Level 3}|\\
|===== Level 4 =====|&\bf Level 4 &|\paragraph{Level 4}|\SD\\
|====== Level 5 ======|&\bf Level 5 &|\subparagraph{Level 5}|\SD\\
\hline
\SP{Turn heading feature on/off by \CS{wikiHeadings}\SL\CS{nowikiHeadings}; 
disable feature entirely by package option {\tt noHeadings}.}\\
\hline\hline
&&|\begin{itemize}|\SU\\
|* One|&\textbullet\hspace\labelsep One &|\item One|\\[2\parsep]
|* Two|&\textbullet\hspace\labelsep Two &|\item Two|\\
&&|\end{itemize}|\SD\\
\hline
&&|\begin{enumerate}|\SU\\
|# One|&1.\hspace\labelsep One &|\item One|\\[2\parsep]
|# Two|&2.\hspace\labelsep Two &|\item Two|\\
&&|\end{enumerate}|\SD\\
\hline
&&|\begin{description}|\SU\\
|;[club]  explain|&\textbf{club}\hspace\labelsep  explain 
                  &|\item[club]  explain|\\[2\parsep]
|;[widow] explain|&\textbf{widow}\hspace\labelsep explain 
                  &|\item[widow] explain|\\
&&|\end{description}|\SD\\
\hline
           &          &|She wrote:|\SU\\
|She wrote:|&\smash{\raisebox{\topsep}{She wrote:}}&|\begin{quote}|\\
|:Indeed|  &\hspace\leftmargini Indeed&|Indeed|\\
           &          &|\end{quote}|\SD\\
\hline
                 &               &|Type|\SU\\
|Type|           &\smash{\raisebox{\topsep}{Type}}&|\begin{verbatim}|\\
| \typeout{OK!}| &|\typeout{OK!}|&|\typeout{OK!}|\\
                 &               &|\end{verbatim}|\SD\\
\hline
\SP{A \CS{begin} is executed when {\tt*} etc.\ is first character of a 
code line; an indent starts {\tt verbatim}. 
\CS{end} is executed at code line without indent and not beginning with 
{\tt*} etc.---Turn this feature on by \CS{wikiEnvironments}; turn off by 
\CS{nowikiEnvironments}, e.g., to get around incompatibilities 
(lists feature is most dangerous). 
Disable feature entirely by package option {\tt noEnvironments}.
Package option {\tt noVerbatim} instead may usually suffice, 
if you don't want to have any {\tt verbatim}.}\\
\hline
\end{tabular}
\kern-1em
\end{center}

% \section{Test}\subsection{Sub-Test}\subsubsection{Subsub-Test}
% \paragraph{Para}\subparagraph{Sub-Para}

\footnotesize
% \vspace{2ex}
\sloppy
\noindent \WE\WF %%% '''Please note:'''
''Note:''
% * 
Explicit '''turning on''' by %%% one of 
%   \[|\wikiEnvironments|,\quad |\wikiFonts|,\quad |\wikiHeadings|\] 
|\wikiEnvironments| %%% , |\wikiFonts|, |\wikiHeadings|,
etc.\ 
% or by |\wikimarkup| (which executes the former three) is ''required''; 
is '''required''';
e.g., after %%% |\maketitle| and 
|\tableofcontents|.
|\wikimarkup| '''activates ''all''''' the features. 
% * 
|\nowikimarkup| %%% executes all of 
%   \[|\nowikiEnvironments|,\quad |\nowikiFonts|,\quad |\nowikiHeadings|.\] 
% |\nowikiEnvironments|, |\nowikiFonts|, |\nowikiHeadings|. 
% Use them to get around incompatible code.
'''''dis''ables all''' of them, e.g., to get around incompatibilities. 

\end{document}

