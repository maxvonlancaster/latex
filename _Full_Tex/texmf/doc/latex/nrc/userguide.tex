\begin{filecontents}{nrcdoc.sty}

\ProvidesPackage{nrcdoc}[2002/03/14 v2.3 collection of commands for
                                    NRC documentation]
%
\newcommand\checkNRCDOCdate[1]{\@ifl@ter{sty}{nrcdoc}{#1}%
  {}%
  {\ClassWarningNoLine{nrcdoc}{An old version of nrcdoc.sty has been detected%
                         \MessageBreak %
                         Please *delete* nrcdoc.sty, to allow LaTeX to
                         \MessageBreak generate a new version}}%
}

% converted to a 2e package, rf 1998/04/05
% also added \Status and provision for dealing with pdf output
%
% usage:
%        <processor> "\def\Status{<value>} \input <file>"
%
% for example:
%        pdflatex "\def\Status{1} \input user_guide"
%
% (that's un*x syntax, but the intent is (i hope) plain
\providecommand{\Status}{0}
\ifcase\Status
  % default case (0): using ordinary latex
  \PackageInfo{nrcdoc}{Regular LaTeX run}
  \def\href#1#2{#2}
\or
  % case 1: using pdflatex to generate a copy of the file
  \PackageInfo{nrcdoc}{Generating hypertext pdf output}
  \RequirePackage[pdftex]{hyperref}
\fi

\RequirePackage{url}
\newcommand{\email}{\begingroup \urlstyle{tt}\Url}
%
\def\EmailURL#1{\href{mailto:#1}{\email{#1}}}
\def\ctanaddr#1{\href{ftp://ctan.tug.org/tex-archive/#1}{\path{#1}}}


\def\ps@myheadings{\let\@mkboth\@gobbletwo
                   \def\@oddhead{{\sl\rightmark}\hfil \rm\thepage}%
                   \def\@oddfoot{Nov. 2002\hfill}%
                   \def\@evenhead{\rm \thepage\hfil\sl\leftmark}%
                   \def\@evenfoot{Oct. 2002\hfill}%
                   \def\sectionmark##1{}\def\subsectionmark##1{}}

\pagestyle{myheadings}

\markboth{{\rm NRC:} \LaTeX{} User Guide for Journals}
         {{\rm NRC:} \LaTeX{} User Guide for Journals}


%% Dimensions:

%% CT> to get holes at left without touching text, changed 
%%       a. margins from -3.5pc to -2pc
%%       b. textheight from 23cm to 24cm
%% CT> 28 SEP 00: makes no sense to use these margins for the version
%%                to be posted on CTAN. Dimens now revert to original
%%                value of  -3.5pc for left/rightmargins. The 
%%                textheight, however, will remain at 24cm.
%%
\setlength{\textwidth}{19cm}
\setlength{\textheight}{24cm}
\setlength{\oddsidemargin}{-3.5pc}
\setlength{\evensidemargin}{-3.5pc}
\setlength{\headsep}{12pt}
\setlength{\topmargin}{-3.5pc}
\setlength{\columnsep}{1.5pc}


%% Something Robin doesn't like but it's handy for tables:

%% Spacing commands for {tabular} (from TTN 2,3:10 -- Claudio
%%                                                    Beccari): 

\newcommand\T{\rule{0pt}{2.6ex}}            % = `top' strut
\newcommand\B{\rule[-1.2ex]{0pt}{0pt}}      % = `bottom' strut


%% Fonts:

\font\bfsl=cmssi17
\font\sectiontt=cmtt12 scaled\magstep1
\font\stt=cmtt10 scaled 850


%% Macros:

\newcommand{\NRC}{{\small NRC}}
\newcommand{\blash}{{\stt\char'134}}

\newcommand{\creditline}{{\footnotesize 
                           Carleton Production Centre (9/00)}}

%% def'n for \LaTeXe taken from TUGboat's tugboat.cmn file:
\newcount\TestCount     \newbox\TestBox
\def\La{\TestCount=\the\fam \leavevmode L%
        \setbox\TestBox=\hbox{$\fam\TestCount\scriptstyle A$}%
        \kern-.5\wd\TestBox\raise.42ex\box\TestBox}
\def\LaTeX{\La\kern-.15em\TeX}
\def\LaTeXe{\LaTeX{}\kern.05em2$_{\textstyle\varepsilon}$}


%% \let-statements:

\let\en=\enspace
\let\nl=\newline

\end{filecontents}

%%%%%%%%%%%%%%%%%%%%%%%%%%%%%%%%%%%

\documentclass{article}
\usepackage{nrcdoc}

% check we're running against a copy of nrcdoc.sty generated from this
% copy of userguide.tex.  first we check that we're running against
% something that's modern enough to have the command for checking
%
\expandafter\ifx\csname checkNRCDOCdate\endcsname\relax
  \typeout{*** An old version of nrcdoc.sty has been detected}
  \typeout{*** Please *delete* nrcdoc.sty to allow LaTeX to generate a
    new version}
\else
%
% we are using a recent enough version: use the checking command to
% compare with the date in the \ProvidePackage command in the package;
% the date here *must* be updated, in the standard latex format, in
% parallel with the date in the package.
%
  \checkNRCDOCdate{2001/11/15}
\fi

\begin{document}

%% Titleblock:
\twocolumn[
\begin{center}
       \bfsl National Research Council \\
       \vspace{.75pc}
             Research Press \\
       \vspace{.5pc}
       \Large\bf \LaTeX{} User Guide for Journals\footnotemark
\end{center}

\vspace{1pc}

]

\footnotetext{This document is available from your local CTAN site in
   \texttt{macros/latex/contrib/supported/nrc/} in both \texttt{.ps}
   and \texttt{.pdf} formats.}


%% Section 1:
\section{Basic coding}

All \LaTeX{} documents must include these three commands:

\begin{verbatim}
   \documentclass{...}
    ...                   <-- `preamble'
   \begin{document}
    ...                   <-- `body'
    ...
   \end{document}
\end{verbatim}

\begin{enumerate} \itemsep=0pt
   \item The \verb|documentclass| must be specified, either with a
         generic package, such as \verb|article| or a specific
         package, such as \verb|nrc1|).

   \item Between the first two commands comes the material known as
         the `preamble', which includes additional packages (and their
         options), as well as any file-specific macros.

   \item Between the 2nd and 3rd commands comes the actual contents of
         the article --- known as the `body' of the file.
\end{enumerate}

\noindent Commands are usually of the following types:

\begin{enumerate} \itemsep=0pt
   \item {\bf control sequences} begin with a backslash ( \blash\ ). 

   \item {\bf environments} use matching \verb|\begin{...}| and\break
         \verb|\end{...}| commands (i.e., \verb|\begin{document}| must
         eventually be matched with \verb|\end{document}|).

   \item {\bf optional} arguments are within square brackets \verb|[...]|
\end{enumerate}


%% Section 2:
\section{Document classes and options}

\NRC{} journals are set either in full-width, using the \texttt{nrc1}
class, or in 2-column format, using the \texttt{nrc2} document class.

The following represent the options for both document classes. Note
that most articles will {\bf not} require all of them; as well, some
options are only for \texttt{nrc2}.

\begin{verbatim}
   \documentclass[<options here>]{<class here>}
                  author              nrc1
            PLUS    usecmfonts     OR nrc2
             OR     type1rest       
                  french           
                  nonumbib
    nrc1 only:    leqno
    nrc2 only:    reqno

\end{verbatim}

\newpage

\noindent To combine options, insert a comma between each option:

\begin{verbatim}
  \documentclass[type1rest,genTeX,nonumbib]{nrc2}
\end{verbatim}

The following sections describe the options available to both classes,
and then those options which are specific to \texttt{nrc1} or
\texttt{nrc2} only. Options specific to in-house production work are
described separately; see section~\ref{inhouse}.


%% Section 2.1:
\subsection{Options for both classes}

{\bf Note:} Do not load options or packages which are never accessed;
their presence implies they are required and may cause unnecessary
searches for coded material which is not, in fact, present.

For the convenience of authors, the publicly available class files for
authors have a number of options already automated:
%
   \begin{verbatim}
      author,genTeX,type1rest,usecmfonts
   \end{verbatim}
%
Additionally, a number of in-house diagnostic messages are turned off,
so as not to interfere with processing. Where a default selection is
not appropriate, the following provides information on these options.


\begin{description} \itemsep=0pt
  \item [\texttt{author}]
        This option selects a configuration appropriate for author use
        of the class; it is enabled by default when a
        publicly distributed copy of the class is loaded.

        The \texttt{author} option automatically invokes the
        \texttt{genTeX} option, as well as one of the two font
        options: \texttt{type1rest} or \texttt{usecmfonts}. See below
        for details.


   \item [{\tt genTeX}]

        By use of this option, you declare that you are using a
        generic/public domain version of \TeX{}; it {\bf must} be
        used in conjunction with one of \texttt{usecmfonts} and
        \texttt{type1rest} (see below).

        The \texttt{author} option automatically selects the
        \texttt{genTeX} option.


   \item [{\tt type1rest}] 

        By use of this option, you declare that you have access to a
        PostScript printer or other interpreter, and may use ``basic''
        PostScript fonts to make a rough approximation to those that
        will be used in the published paper. Do not use this option in
        combination with the \texttt{usecmfonts} option.

        If the \texttt{author} option has been selected, the class
        will automatically decide whether to use \texttt{usecmfonts}
        or \texttt{type1rest} by default. The automatic selection may
        be incorrect, but you may over-ride it by specifying the
        \texttt{type1rest} option.


   \item [{\tt usecmfonts}] 

        By use of this option, you declare that you do not have the
        ``basic'' PostScript fonts to make a rough approximation to
        those that will be used in the published paper: the \verb|cm|
        (\emph{Computer Modern}) fonts that this option selects are
        available in any generic/public domain \TeX{} distribution, so
        this option will always be available. Do not use this option
        in combination with the \texttt{type1rest} option.

        If the \texttt{author} option has been selected, the class
        will automatically decide whether to use \texttt{usecmfonts}
        or \texttt{type1rest} by default. The automatic selection may
        be incorrect, but you may over-ride it by specifying the
        \texttt{usecmfonts} option.


   \item [{\tt french}]

         For French-language articles.\label{frenchclassoption} It
         configures the \NRC{} document classes to print all the
         auto\-mated textual ele\-ments (in the author IDbox, for
         figures, tables, biblio\-graphies, etc.)\ in French as per
         \NRC{} speci\-fi\-cations. To be used in concert with:

\begin{verbatim}
   \usepackage[english,french]{babel}
\end{verbatim}

         See section~\ref{babelinfo} for details on using the
         \textsf{babel} package in all articles, regardless of
         language. For two-column bilingual texts, see
         section~\ref{bilingpream}.


   \item [{\tt nonumbib}] 

         This option removes numbers (i.e., `labels') from
         \verb|\bibitem| entries; material is set with a hanging
         indent. The default is to have numbered bibliographic
         entries.
\end{description}


%% Section 2.2:
\subsection{Options specific to only one class}

\begin{description} \itemsep=0pt
   \item [{\tt leqno}]  \hfill for \texttt{nrc1} only

         This option puts equation numbers to the left margin;
         default is to have them on the right.

   \item [{\tt reqno}] \hfill for \texttt{nrc2} only

         This option puts equation numbers to the right margin;
         default is to have them on the left.
\end{description}

\noindent One other difference between the packages relates to the
position of the \verb|\maketitle| command. See
page~\pageref{maketitle} for details.


%% Section 3:
\section{The preamble}

The preamble area of the file should specify all packages and macros
definitions (if any) used; macros in particular should not be defined
within the body of the file.


%% Section 3.1:
\subsection{Useful packages}

The following packages and their options are often necessary (but note
that \textsf{babel} is necessary at all times). Note that package
order can be significant, on occasion. In particular, it is
recommended that graphics packages be loaded immediately after the
document class.

\begin{description} \itemsep=0pt
   \item [{\tt \blash usepackage\char`\{graphicx\char`\}}]

         Graphics (figures, illustrations,\break and so on) should be
         included using the standard \LaTeX{} \texttt{graphicx}
         package: use of this package permits portability across
         different implementations of \TeX{} (which itself has no
         graphics primitive commands).

         Your \TeX/\LaTeX{} system may offer documentation; otherwise
         see \textsl{The \LaTeX{} Graphics Companion\/} (see
         section~\ref{documentation}).

         Remember to specify the printer driver as a package option,
         if necessary (e.g., \verb|dvips|, \verb|dvipsone|, etc.).

   \item [{\tt \blash usepackage[figuresright]\char`\{rotating\char`\}}]

         \quad The \verb|rotating|\break package\footnote{For the
         NRC's production use, the standard distribution version of
         this file is currently under revision. Authors can proceed
         with the public version of \texttt{rotating.sty}.} is only
         required when tables or figures need to be placed in
         landscape (`sideways') position. It must be preceded by the
         \verb|graphicx| package.

         The \verb|figuresright| option ensures that rotated
         elements are oriented with the caption at the left margin of
         the regular page, in accord with \NRC{} style requirements. 

   \item [{\tt \blash usepackage\char`\{amsmath\char`\}}]

         A great deal of additional functionality is provided with
         this package, and the \NRC{} document classes have been
         designed to work smoothly with it. Using \texttt{amsmath}
         automatically loads the following packages: \texttt{amstext},
         \texttt{amsgen}, and \texttt{amsbsy}. The \texttt{amsmath}
         package is described in great detail in three documents:

   \begin{enumerate} \itemsep=0pt
     \item The revised version of chap.~8 of {\sl The \LaTeX{}
           Companion} is a good guide (see section~\ref{documentation}
           for details). As printed in the book, chap.~8 is out of
           date and should not be referenced.

      \item A useful set of documentation is available via the {\small
            AMS}'s web site. {\small URL}
            % \texttt text broken so it doesn't run into the margin and
            % disappear; would be better to use the url package, but...
            \texttt{http://www.ams.org/tex/ amslatex.html} offers a
            ``Short Math Guide'', a set of frequently asked questions
            (with answers), as well as various ``Readme'' files.
 
      \item Current \LaTeX{} installations have a file
            \texttt{amsldoc}; it is also available via the {\small
            AMS} web site (see above). The material is essentially the
            same as the revised chap.~8 mentioned above, but the
            differences in presentation are sometimes useful to
            consider.
\end{enumerate}

   \item [{\tt \blash usepackage\char`\{amssymb\char`\}}]

         This package automatically loads the 
         \texttt{amsfonts} package. Together, they provide 
         special characters and symbols frequently used by the
         mathematics and scientific communities. 

         The \texttt{amsfonts} package provides access to the
         following font sets: Euler (Fraktur, Roman, Script,
         extensions); Computer Modern (math bold ital\-ic, symbol,
         extensions, in the smaller sizes); Cyrillic (lightfact, bold,
         italic, smallcaps); and the \texttt{msam} and \texttt{msbm}
         fonts, familiar to Rev\TeX{} users.


   \item [{\tt \blash usepackage\char`\{bm\char`\}}] 

         This package simplifies the use of bold symbols and other
         objects in math mode. It defines a single command,
         \verb|\bm{...}|, which is used in math mode, and causes its
         argument to be typeset in the appropriate math bold font.
         See \verb|bm.sty| documentation for details.\label{bm}

         {\bf Note:} If a warning message about too many math
         alphabets arises, insert the following code {\bf above} the
         \verb|\usepackage{bm}| line:

\begin{verbatim}
   \newcommand\hmmax{0} % default 3
   \usepackage{bm}
\end{verbatim}
%
         See \verb|bm.sty| documentation for details.


   \item [{\tt \blash usepackage\char`\{cite\char`\}}]

         Authors should use this package, which enhances the default
         options available in \LaTeX{} (e.g., \verb|\citen| prints
         cross-references without brackets). See \verb|cite.sty|
         documentation for additional features.

   \item [{\tt \blash usepackage[...,...]\char`\{babel\char`\}}]

         The \textsf{babel}\label{babelinfo} package is used to manage
         the typesetting requirements of multilingual documents.
         Different cultures have different typesetting conventions,
         and \textsf{babel} enables the \LaTeX{} user to apply the
         appropriate conventions to the different parts of a
         multilingual document. Since \NRC{} publications are
         typically multilingual, \textsf{babel} has an important
         r\^ole to play in their preparation.

   \begin{itemize} \itemsep=0pt
      \item English-language articles have a French-language
            `Resum\'e', which requires French hyphenation and
            punctuation. Insert the following line --- and notice the
            order of the language options:

\begin{verbatim}
   \usepackage[french,english]{babel}
\end{verbatim}

      \item French-language articles have an English-language
            `Abstract', which requires English hyphenation and
            punctuation. Insert the following line and again notice
            the order of the language
            options:\footnote{\textsf{Babel}, like so many computer
            programs, believes that the last should be first, and
            therefore proceeds backwards through its options, thus
            making the last one typed the one selected first.}

\begin{verbatim}
  \usepackage[english,french]{babel}
\end{verbatim}

          \textbf{In addition to} the \textsf{babel} package,
          French-language articles must \textbf{also} include the
          \verb|french| option to the document class, as mentioned
          earlier in section~\ref{frenchclassoption}.

          The \textsf{babel} package invokes French hyphenation
          patterns\footnote{Use \texttt{\blash-} to signal additional
          potential hyphenation points.} as well as some of the
          (European) French typesetting conventions (e.g., space before
          some punctuation).
   \end{itemize}
\end{description}

\noindent These packages and others can be found either on your
machine or can be acquired from {\small CTAN}, the Comprehensive
\TeX{} Archive Network; use the search facilities at
\url{www.ctan.org/search}. The \NRC{} document classes and this
documentation can be found on {\small CTAN} in
\url{macros/latex/contrib/supported/nrc/}.


%% Section 3.2:
\subsection{Additional macros}

\begin{enumerate} \itemsep=0pt
   \item Avoid creating too many personal macros, which may conflict
         with the \NRC{} document classes (and possibly with other
         packages) and thereby slow in-house processing of files.
         Where these are used, macros not actively invoked in the
         file should be pruned out.

   \item Whenever possible, define your macros using the \LaTeX{}
         \verb|\newcommand| mechanism, rather than the \TeX{} primitive
         \verb|\def|; this way, \LaTeX{} itself will detect any name
         clashes you may innocently introduce. 

   \item Move all non-\NRC{} preamble material, including author
         macros, to the end of the file (after \verb|\end{document}|),
         rather than deleting it. Reintroduce only what is needed. 

   \item Where author macros are needed, they should all be gathered
         at the top of the file in the preamble area, after all
         packages have been loaded, and clearly marked as being author
         macros:\label{authmacros}

\begin{verbatim}
%%%%%%%% Author macros begin:
...
... 
%%%%%%%% Author macros end
\end{verbatim}

   \item Similarly, move all \verb|\let| statements to the preamble
         area, where they are immediately visible to the editor. 
\end{enumerate}


%% Section 4:
\section{The body}

All articles have the following elements:

\begin{enumerate} \itemsep=0pt
   \item titleblock and author information 

   \item abstract and resum\'e

   \item headings and subheadings

   \item text

   \item bibliography
\end{enumerate}

\noindent Most articles also include some or all of the following
elements: 

\begin{enumerate} \itemsep=0pt
   \item in-line and display mathematics 

   \item enumerated lists

   \item tables

   \item figures and illustrations (e.g., PostScript)

   \item footnotes

   \item offset quoted passages

   \item acknowledgements
\end{enumerate}


%% Section 4.1:
\subsection{Titleblock and author information}

All elements of the titleblock are taken care of via macros, some of
which have optional arguments, to allow for variant forms and for
`labelling' of bits of text (for cross-referencing).

\begin{description} \itemsep=0pt
   \item [{\tt \blash title\char`\{...\char`\}}] \mbox{}

         Lines will be broken automatically; line breaks can be forced
         by using \verb|\\|. 

         For sub- and superscripts, use \verb|\textsubscript| and
         \verb|\textsuperscript|, respectively.

         Notes on the title may be inserted using the standard \LaTeX{}
         command \verb|\thanks|. The text will print in the IDbox, and
         numbering will be synchronised with that of the IDbox notes.

   \item [{\tt \blash author[...]\char`\{Author1\char`\}}] \mbox{}

         {\sl Author1\/}'s name as it will appear in the titleblock
         and the right running head. The optional argument allows a
         different form of the author's name to appear in the IDbox at
         page-bottom (e.g., full first name in titleblock, but
         initials only in IDbox).

         Each author is specified separately. The document class will
         automatically insert `and' (Fr.\ `et') between the last and
         second-last names, both in the titleblock and in the IDbox
         area.

   \item [{\tt \blash address[label1]\char`\{addr1\char`\}}] \mbox{}

         Address for {\sl Author1\/}; this should be immediately below
         the \verb|\author| entry. Each author's address is specified
         separately. If no address is specified before the next author
         entry, the immediately preceding address is used (an address
         for the first author {\bf must} therefore be specified).

         The optional \verb|[label1]| argument (any sequence of
         letters can be used) does not print anything; it makes it
         possible to relate one address to several authors (this is
         similar to \LaTeX's \verb|\label| and \verb|\ref| commands).
\end{description}

Below is an example of using the author-address cross-referencing
macros. Notice that \verb|\address| always refers to the immediately
preceding \verb|\author|.
         
\begin{description} \itemsep=0pt
   \item \verb|\author{Author1}|\quad defines first author

   \item \verb|\address[label1]{Address1}|\quad defines an address for
         {\sl Author1\/} and tags it as \verb|[label1]|

   \item \verb|\author{Author2}|\quad this author doesn't have an
         address, so the IDbox will use the immediately preceeding
         author's address (i.e., {\sl Address1}

   \item \verb|\author{Author3}|\quad defines a third author

   \item \verb|\address{Address3}|\quad third author is at an address
         of their own

   \item \verb|\author{Author4}|\quad defines a fourth author

   \item \verb|\address[label1]|\quad fourth author uses the address
         associated with \verb|[label1]|, above

         The net result (assuming the \verb|breakaddress| option has
         been used) is the following:

         {\bf Author1, Author2, and Author4.} \\ 
         Address1.

         {\bf Author3.} \\ 
         Address3.

   \item [{\sl IDbox address notes}] \mbox{}
 
         Author addresses can be augmented with additional information
         via numbered notes, which appear below all the authors' names
         and main addresses. 

   \begin{description}
      \item [{\tt \blash correspond\char`\{...\char`\}}] \mbox{}

            Usually one author is selected from all co-authors to deal
            with correspondence. Insert the author's e-mail address
            into the argument. The following text will be prepended:
            `Corresponding author (email: ).' and in French,
            `Auteur correspondant (courriel\,: )'. 


      \item [{\tt \blash present\char`\{...\char`\}}] \mbox{}

            Authors may wish to indicate a temporary or current
            address, different from the main one provided via
            \verb|\address|. This note will begin with the text:
            `Present address:' (Fr.\ `Adresse actuelle\,:'). 


      \item [\mbox{}{\tt \blash AddressNote\char`\{...\char`\}}]
            Obsolete. See \verb|\IDnote|, below.


      \item [{\tt \blash IDnote\char`\{...\char`\}}] \mbox{}

            Allows the user to input any information they wish, with
            no automatic text being added (formerly called
            \verb|\AddressNote|, which implied a more restricted
            use). Both macros are available but \verb|\IDnote| gives
            the user a more generic macro name to remember and use for
            any type of note within the IDbox. 
   \end{description}

   \item [{\tt \blash dedication\char`\{...\char`\}}] \mbox{}

         Will print a dedicatory text (in italics) in the IDbox area,
         between the `Received/Accepted' and the author/address
         sections.

   \item [{\tt \blash abbreviations\char`\{...\char`\}}] \mbox{}

         Automatic printing of the word `\textit{Abbrevia\-tions: \/}'
         (Fr. `\textit{Abbr\'eviations\,: \/}'), followed by text
         input inside the argument (the curly braces). Forced line
         breaks (via \verb|\\|) can be used to separate entries.

   \item [{\tt \blash shortauthor\char`\{author list\char`\}}]
         \mbox{} 

         Shortened list of author names for use in the right running
         header. This macro must be present (its absence generates an
         error message), and should appear below all the
         author-address information

\newpage

   \item [{\tt \blash maketitle}] \mbox{}\label{maketitle}

         This command activates the titleblock commands.

         The \verb|nrc1| class requires this command to appear
         \textbf{before} the abstract/resum\'e block of text.

         The \verb|nrc2| class requires this command to appear
         \textbf{after} the abstract/resum\'e block of text.

   \item [{\tt \blash maketitle*}] \hfill [an \NRC\ macro]\mbox{}

         With the \verb|nrc2| class only, when Abstracts/Resumes spill
         over to a second page, a horizontal rule may be needed before
         the regular article text begins. To generate this rule use
         \verb|\maketitle*| instead of \verb|\maketitle|.
\end{description}


%% Section 4.2:
\subsection{Abstracts/R\'esum\'es}

The syntax is the normal one expected for environments: a matched set
of either \verb|{abstract}| or \verb|{resume}|:

\begin{description} \itemsep=0pt
   \item [{\tt \blash begin\char`\{abstract\char`\}\ ... \blash
         end\char`\{abstract\char`\}}]  \hfill English

   \item [{\tt \blash begin\char`\{resume\char`\}\ ... \blash
         end\char`\{resume\char`\}}]    \hfill French
\end{description}

\noindent Some journals may require the following, which should appear
\textbf{inside} the abstract environments:

\begin{description} \itemsep=0pt
   \item [{\tt \blash keywords\char`\{...\char`\}}]  \hfill [an \NRC\
      macro]\mbox{} 

         Automatically prints `\textit{Keywords:}', followed by
         whatever text is input inside the argument (the curly
         braces).

   \item [{\tt \blash motscles\char`\{...\char`\}}]  \hfill [an \NRC\
      macro]\mbox{} 

         Automatically prints `\textit{Mots cl\'es\,:}', followed by
         whatever text is input inside the argument (the curly
         braces).

   \item [{\tt \blash PACS\char`\{...\char`\}}]  \hfill [an \NRC\
      macro]\mbox{} 

         Automatically prints `PACS Nos.:' (Fr. `PACS
         N\textsuperscript{os}\,:'), followed by whatever material is
         input inside the argument.

   \item [{\tt \blash PACS*\char`\{...\char`\}}]  \hfill [an \NRC\
      macro]\mbox{} 

         Automatically prints `PACS No.:' (Fr. `PACS
         N\textsuperscript{o}\,:'), followed by whatever material is
         input inside the argument.
\end{description}


%% Section 4.3:
\subsection{Headings and subheadings}

Five levels, numbered automatically.\footnote{An alternate set of
headings commands also exists: \texttt{\blash Asection},
\texttt{\blash Bsection}, \texttt{\blash Csection}, \texttt{\blash
Dsection}, and \texttt{\blash Esection}, for levels 1 through 5,
respectively.} Line breaks can be forced by using \verb|\\|. To
suppress numbering (e.g., for `Acknowledgements'), use an asterisk
before the opening curly brace: \verb|\section*{Acknowledgements}|.

For sub- and superscripts in section titles, use the macros
\verb|\textsubscript| and \verb|\textsuperscript|, respectively.

\begin{enumerate} \itemsep=0pt
   \item {\tt \blash section\char`\{...\char`\}} Level-1 heading

   \item {\tt \blash subsection\char`\{...\char`\}} Level-2 heading

   \item {\tt \blash subsubsection\char`\{...\char`\}} Level-3
   heading 

   \item {\tt \blash subsubsubsection\char`\{...\char`\}} Level-4
   heading \hfill [\NRC]

   \item {\tt \blash paragraph\char`\{...\char`\}} Level-5 heading
\end{enumerate}


%% Section 4.4:
\subsection{Text}

Same as the default \LaTeX{} commands:

\begin{description} \itemsep=0pt
   \item [{\tt \blash begin\char`\{quote\char`\}\ ... \blash
         end\char`\{quote\char`\}}]  

   \item [{\tt \blash begin\char`\{enumerate\char`\}\ ... \blash
         end\char`\{enumerate\char`\}}]  

   \item [{\tt \blash begin\char`\{itemize\char`\}\ ... \blash
         end\char`\{itemize\char`\}}]  

   \item [{\tt \blash begin\char`\{description\char`\}\ ... \blash
         end\char`\{description\char`\}}]  

   \item [{\tt \blash footnote\char`\{...\char`\}}] 
\end{description}

\noindent Where lists must be flushed to the left margin, there are
two \NRC-specific environments to use:

\begin{description} \itemsep=0pt
   \item [{\tt \blash begin\char`\{flenumerate\char`\}\ ... \blash
         end\char`\{flenumerate\char`\}}]\hfill [\NRC] \mbox{}

         Generates a numbered list (first level only) with labels
         flushed to the left margin. No nesting possible.

   \item [{\tt \blash begin\char`\{flitemize\char`\}\ ... \blash
         end\char`\{flitemize\char`\}}]\hfill [\NRC] \mbox{}

         Generates a bulletted list (first level only) with labels
         flushed to the left margin. No nesting possible.
\end{description}


%% Section 4.4.1:
\subsubsection{Column switching in \texttt{nrc2}}

On occasion, material for 2-column journals is best set full-width,
interrupting the two text columns. For equations, the following
customized code will achieve this effect:

\begin{verbatim}
   \begin{FullWidth}[0.5]
   \LeftColumnBar
       <equation to span both columns>
   \RightColumnBar
   \end{FullWidth}
\end{verbatim}

The following description provides details of each step:

\begin{description} \itemsep=0pt
   \item [{\tt \blash begin\char`\{FullWidth\char`\}\ ... \blash
         end\char`\{FullWidth\char`\}}]

         This environment encloses the material which is to span the
         two columns. The text for the two columns immediately above
         this environment will be balanced. Text immediately below
         the environment will resume the 2-column layout. 

         The optional \texttt{[0.5]} argument (`one half' in this
         example) is an adjustment factor, affecting the split between
         left and right columns. Default is `1.0', the units are
         nominal line depths in the default font size; increasing the
         factor tends to increase the number of lines in the left
         column.

   \item  [{\tt \blash LeftColumnBar}] \mbox{}

          This draws a rule below the left column of the 2-column text
          which is above the full-width material. 

   \item  [{\tt \blash RightColumnBar}] \mbox{}

          This draws a rule above the right column of 2-column text
          which is below the full-width material.

   \item  [{\tt \blash BalanceColumns[0.5]}] \mbox{}

          This command is used at the end of a file, in order to
          balance the final page. 

          The optional argument serves the same purpose and is used
          in the same way as that for \verb|\begin{FullWidth}|.
\end{description}

Where both equations and text must span two columns, the text portions
must \textit{additionally} be enclosed in the following environment:

\begin{verbatim}
   ...
   \begin{WideText}
       <text material>
   \end{WideText}
   ...
\end{verbatim}
%
The text will then be flush left, and indented 3em from the right
margin, as per \NRC{} requirements, while the equation numbers will be
flushed to the right (where that is the style). The two combined will
look like this:

\begin{verbatim}
   \begin{FullWidth}[0]
   \LeftColumnBar
   \begin{equation}
       <equation material here>
   \end{equation}

   \begin{WideText}
       <text material>
   \end{WideText}

   \begin{equation}
       <equation material here>
   \end{equation}
   \RightColumnBar
   \end{FullWidth}
\end{verbatim}


%% Section 4.5:
\subsection{Mathematics}

In addition to default \LaTeX{} commands, the \texttt{amsmath} package
provides a number of useful enhancements. Consult both the revised
chap.~8 of \textsl{The \LaTeX{} Companion} and the AMS\LaTeX{}
documentation for details (see section~\ref{documentation}). See also
\verb|bm.sty| for enhanced font handling inside math mode
(section~\ref{bm}).

\begin{description} \itemsep=0pt
   \item [{\tt \blash begin\char`\{equation\char`\}\ ... \blash
         end\char`\{equation\char`\}}] \mbox{} 

   \item [{\tt \blash begin\char`\{array\char`\}\ ... \blash
         end\char`\{array\char`\}}] \mbox{} 

   \item [{\tt \blash begin\char`\{subequation\char`\}\ ... \blash
         end\char`\{subequation\char`\}}]  \mbox{}

        This is an \verb|amsmath| environment {\bf within which}
        other environments are used, to achieve lettered sub-equation
        numbers: e.g., 1.2a, 1.2b, etc. There is a file
        \texttt{subeqn.tex} on most \TeX{} installations with
        additional details.
\end{description}

\noindent In addition to the usual environments, two \NRC\ macros
allow further customisation of the form of the equation number:

\begin{description} \itemsep=0pt
   \item [{\tt \blash
         numberby\char`\{...\char`\}\char`\{...\char`\}}] \hfill [an
         \NRC\ macro]\mbox{}

         For articles where equation numbers must include section
         numbers (e.g., equation 3.2 is the second equation in
         section~3), input the following in the preamble area, after
         all \verb|\usepackage| commands:\footnote{This avoids having
         to load the {\tt amsmath} package in order to access the {\tt
         \blash numberwithin} command, which does the same thing.} 

\begin{verbatim} 
   \numberby{equation}{section}
\end{verbatim} 

         The first argument contains what element is being numbered;
         the second indicates that section numbers should be included
         in that numbering.

   \item [{\tt \blash eqnoformat}] \hfill [an \NRC\ macro]\mbox{}

         The \verb|nrc2| class default format is for square brackets
         around equation numbers. Other options (i.e., parenthe\-ses)
         can be achieved via the following definition (placed in the
         preamble, after all packages have been loaded):

\begin{verbatim}
   \renewcommand{\eqnoformat}[1]{(#1)}
\end{verbatim}
\end{description}


%% Section 4.6:
\subsection{Tables, Figures, Captions}

The following float parameters are useful to include in the preamble
area: 

\begin{description} \itemsep=0pt
   \item [nrc1] 
\begin{verbatim}
  \renewcommand{\topfraction}{.95}
  \renewcommand{\textfraction}{.05}
  \renewcommand{\floatpagefraction}{.95}
\end{verbatim}

   \item [nrc2] 
\begin{verbatim}
  \renewcommand{\topfraction}{.95}
  \renewcommand{\floatpagefraction}{.95}
  \renewcommand{\dbltopfraction}{.95}
  \renewcommand{\textfraction}{.05}
  \renewcommand{\dblfloatpagefraction}{.95}
\end{verbatim}
\end{description}


%% Section 4.6.1:
\subsubsection{Tables}

\begin{description} \itemsep=0pt
   \item [{\tt \blash begin\char`\{table\char`\}\ ... \blash
         end\char`\{table\char`\}}]  \mbox{}

         For regular tables, up to same width as text (page-wide or
         column-wide).

   \item [{\tt \blash begin\char`\{table*\char`\}\ ... \blash
         end\char`\{table*\char`\}}] \mbox{}

         For tables spanning two columns. 

   \item [{\tt \blash begin\char`\{sidewaystable\char`\}\ ... \blash 
         end\char`\{sidewaystable\char`\}}] \mbox{}

         For rotating a table sideways. To ensure rotation is in
         correct direction, remember to add the \verb|figuresright|
         option to the \verb|rotating| package in the Preamble.

   \item [{\tt \blash begin\char`\{sidewaystable*\char`\}\ ... \blash 
         end\char`\{sidewaystable*\char`\}}] \mbox{}

         For rotating a table sideways across a 2-column page. As for
         above, make sure the \verb|figuresright| option has been
         added to the \verb|rotating| package, loaded in the Preamble.

   \item [{\tt \blash begin\char`\{tabular\char`\}\ ... \blash
         end\char`\{tabular\char`\}}]  \mbox{}

         {\bf Note:} use \verb|\tabcolsep| to slightly reduce/increase
         intercolumn space. 


   \item [{\tt \blash hline*}]\hfill  [an \NRC\ macro] \mbox{}

         For the thicker lines at top/bottom of tables. Regular 1pt
         rules accessed via default \verb|\hline| command.
\end{description}


%% Section 4.6.2:
\subsubsection{Figures}

\begin{description} \itemsep=0pt
   \item [{\tt \blash begin\char`\{figure\char`\}\ ... \blash
         end\char`\{figure\char`\}}]  \mbox{}

         For regular figures, up to same width as text (page-wide or
         column-wide).

   \item [{\tt \blash begin\char`\{figure*\char`\}\ ... \blash
         end\char`\{figure*\char`\}}] \mbox{}

         For figures spanning two columns.
\end{description}


%% Section 4.6.3:
\subsubsection{Captions}

\begin{description} \itemsep=0pt
   \item [{\tt \blash topcaption\char`\{...\char`\}}]  \hfill [an
         \NRC\ macro]\mbox{} 

         \NRC\ style has all captions at the top of their table or
         figure. Syntax is the same as for \LaTeX{} default
         \verb|\caption|.

        Captions will be automatically sized to the width of their
        table, provided the file is processed at least twice. Should
        a caption require a different width, the following code, used
        inside the float environment, will work:

\begin{verbatim}
   \setlength{\captionwidth}{ ... }
   \topcaption{ ... }
\end{verbatim}

   \item [{\tt \blash sepcaption\char`\{...\char`\}}] \hfill [an \NRC\
         macro]\mbox{} 

         For captions separated from their full-column or full-page
         tables and figures. Caption will appear at top of next column
         or next page. Usage is \textbf{different} from regular
         caption commands in that it must appear in its own separate
         \texttt{table} environment:
%
\begin{verbatim}
   \begin{table}
      \sepcaption{.....}
   \end{table}

   \begin{table}
      <contents of table>
   \end{table}
\end{verbatim}
\end{description}


%% Section 4.6.4:
\subsubsection{Centering}

To centre any of these elements inside a float, the \verb|\centering|
command is preferred to the \verb|{center}| environment, which adds
its own vertical space and hence interferes with \NRC{} spacing
requirements.


%% Section 4.7:
\subsection{Cross-Referencing}

As with all cross-referencing codes, process file 2 or even 3 times,
to ensure that all citations and labels have been resolved, until the
following message no longer appears:\footnote{Note, however, that if
cross-referencing codes are incomplete, \textbf{[?]} will remain. In
such cases, the warning message reads: \texttt{LaTeX Warning: There
were undefined references}.}

\begin{verbatim}
   LaTeX Warning: Label(s) may have changed. 
   Rerun to get cross-references right. 
\end{verbatim}

\noindent Components of an article affected by cross-referencing ---
and thus requiring several processing runs, include the following:

\begin{itemize} \itemsep=0pt
   \item cross-refs to tables and figures, and to equations, etc. \\
         (via \verb|\ref|, \verb|\pageref|, and \verb|\label|)

   \item automatic sizing of caption widths for floats \\
         (via \verb|\topcaption|)

   \item bibliographies \\
         (via \verb|\cite| et al., and \verb|\bibitem|)
\end{itemize}

To ensure that cross-referencing via the \verb|\label| and \verb|\ref|
commands is correctly associated with the matching tables and figures,
it is recommended that the \verb|\label| command appear
\textbf{inside} the closing parentheses of caption commands.


%% Section 4.8:
\subsection{Bibliography} 

The default macros \verb|\cite| and \verb|\bibitem| are usually
adequate for citations and bibliography entries. Additional
flexibility can be had by using the \verb|cite| package; for more
complicated requirements, authors may choose to use the \verb|natbib|
package. Authors using the latter must remember to submit their
\verb|.bib| file, and are reminded to use the referencing style and
order of elements appropriate to the journal to which they are
submitting.\footnote{As yet, there is no \texttt{nrc.bst} file for the
NRC.}

The main \verb|cite| commands are:

\begin{description} \itemsep=0pt
   \item [{\tt \blash cite\char`\{...\char`\}}] \mbox{}

         To produce cross-referencing digit or digits inside square
         brackets.

   \item [{\tt \blash citen\char`\{...\char`\}}] \mbox{}

         To produce cross-referencing digit with {\bf no} brackets.
\end{description}

\noindent Further options are described in the \verb|cite|
documentation. However, one in particular will be noted here, as some
\NRC{} journals require in-line citations to appear within
parentheses, rather than square brackets. If this is the case, the
following code should be added to the preamble (after all packages,
including \verb|cite|, have been loaded):

\begin{verbatim}
   \renewcommand{\citeleft}{(}
   \renewcommand{\citeright}{)}
\end{verbatim}

The references/bibliography section at the end of the paper uses the
default \LaTeX{} commands:

\begin{verbatim}
   \begin{thebibliography}{99}
     \bibitem[...]{...}

     \bibitem[...]{...}

   \end{thebibliography}
\end{verbatim}


%% Section 4.9:
\subsection{Appendices}

Where only the word `Appendix' is needed, use the command
\verb|\section*{Appendix}| (note that the asterisk suppresses any
section numbering, either by digit or letter). If equations within the
Appendix are to restart at `1', insert

\begin{verbatim}
   \setcounter{equation}{0}
\end{verbatim}

If more than the word `Appendix' is to appear, then the
\verb|\section| command must be augmented by either \verb|\appendix|
or \verb|\appendix*|.

The \verb|\appendix| command (unmodified) behaves as in the standard
\LaTeX{} classes; so, for `A. Title of First Appendix', the following
code is used:
%
\begin{verbatim}
   \appendix
   \section{Title of First Appendix}
\end{verbatim}

For `Appendix A:' + a heading (and then `Appendix B:' + its heading,
etc.), the following code will do the job (notice that the word
`Appendix' is \textbf{not} input):

\begin{verbatim}
   \appendix*
   \section{A subheading}
    ...
   \section{Next subheading}
\end{verbatim}

Both \verb|\appendix| and \verb|\appendix*| preserve \verb|\numberby|
commands, as one might expect: equations in appendix~A are numbered
`A.1', `A.2', etc. To ensure numbering is correctly applied throughout
all appendices, insert \verb|\numberby| {\bf before} the
\verb|\section| command.


%% Section 5:
\section{Resources}

The following documentation, newsgroups, and web pages are useful
source to consult for help, news, and updates. Keep in mind, however,
that conflicts may arise when 


%% Section 5.1
\subsection{Books and articles\label{documentation}}

\begin{description}  \itemsep=0pt \raggedright \frenchspacing
   \item [{\sl The \LaTeX{} Companion:}] by Michel Goossens, Frank
         Mittelbach, and Alexander Samarin (Addison-Wesley, 1994).

         Contains many details to assist users. Caveats:

         Chapter~8 is no longer valid --- a revised version is
         available in both {\tt .ps} and {\tt .pdf} formats from
         {\small CTAN}.\footnote{CTAN = Comprehensive \TeX{} Archive
         Network; a list of site addresses can be found on the {\small
         TUG} home page {\tt www.tug.org}. Follow the links to {\tt
         /tex-archive/info/companion-rev}.}

         As well, the sections on graphics and colour have been
         superseded by material in {\sl The \LaTeX{} Graphics
         Companion\/}.

   \item [{\sl The \LaTeX{} Graphics Companion:}]{\sl Illustrating
         Documents\break with \TeX{} and PostScript\/}, by Michel
         Goossens, Sebastian Rahtz, and Frank Mittelbach
         (Addison-Wesley, 1997).

   \item [{\sl Math into \LaTeX:}] {\sl An Introduction to \LaTeX{}
         and AMS\LaTeX{}}, by George Gr\"atzer (Birkh\"auser, Boston
         and Springer Verlag, New York, 1996).

   \item [{\sl First Steps in \LaTeX:}] by George Gr\"atzer
          (Birkh\"auser, Boston, 1999).

   \item [{\sl The \TeX book:}] by Donald E.~Knuth (Addison-Wesley,
         1986). 

   \item [{\sl \LaTeX:}] {\sl A Document Preparation System ---
         User's Guide and Reference Manual\/}, by Leslie
         Lamport (Addison-Wesley, 1994, 2nd ed). 

   \item [{\sl A Guide to \LaTeXe:}] by Helmut Kopka and Patrick
         W.~Daly (Addison-Wesley, 1998, 3rd ed).

   \item [{\rm Michael Downes:}] ``Breaking equations,'' {\sl TUGboat}
         18,3 (Sept 1997): 182--194.

   \item [{\rm Keith Reckdahl:}] ``Using EPS graphics in \LaTeXe\
         documents,'' {\sl TUGboat} 17,1 (March 1996): 43--53.

   \item [{\rm Keith Reckdahl:}] ``Using EPS graphics in \LaTeXe\ documents,
         Part 2: Floating figures, boxed figures, captions, and math
         in figures,'' {\sl TUGboat} 17,3 (Sept.\ 1996): 288--310.

         The latest version of the Reckdahl material can be found on
         {\small CTAN} in \texttt{info/epslatex} in both \texttt{.ps}
         and \texttt{.pdf} formats.
\end{description}


%% Section 5.2
\subsection{Electronic resources}

\begin{description} \itemsep=0pt
   \item [\href{http://www.tug.org/}{www.tug.org}:] the most complete
         stepping-stone to the world-wide \TeX{} community, including
         the {\small CTAN} archives, user groups, news, and so on.

   \item [comp.text.tex] a general all-purpose newgroup for \TeX{}
         users. Consult your local technical support group to see
         if newsgroup access if available via your browser. 

   \item [FAQ:] put together by the UK \TeX{} Users Group; available
         via the {\small TUG} web page.

   \item [Listserv lists:] there are a great number of specialised
         lists. Consult the {\small TUG} web pages for details.

   \item [http://groups.google.com/] holds an archive of usenet
         discussions, and may be used to review current topics of
         concern, or to search for answers to specific questions.
         Unfortunately, the service does {\bf not} offer facilities
         for posting to usenet, at present.
\end{description}

\newpage


%% Section 6:
\section{In-house Coding for Articles\label{inhouse}}

A template file with all the main preamble lines of code already
input, is available (see Appendix~B). At the top of the new file,
insert the contents of:
%
\begin{verbatim}
   nrc-opening.tex
\end{verbatim}
%
and begin to {\bf un}comment those lines which are pertinent for the
file. There are brief notes in the template, indicating the purpose of
each macro line, along with cross-references to pages in these
guidelines. Delete or leave commented those lines which are not
relevant to the file.

{\bf Note:} Only invoke those packages and/or macros which are present
in the file; for example, it is misleading to load a graphics package
if there are no figures in the file.



%% Section 6.1
\subsection{Changing class option choices}

Included in the main class options are some which are intended for
authors only; remove any of the following options before processing
author files in-house:

\begin{verbatim}
   author   genTeX   type1rest   usecmfonts   
\end{verbatim}

\noindent On the other hand, there are a number of class options
related to various stages of in-house production and thus intended
only for \NRC{} editorial staff. Below is a list of these options,
followed by a brief description of their purpose:

\begin{verbatim}
   \documentclass[<options here>]{<class here>}
                  breakaddress        nrc1
  for nrc2 only:  twocolid         OR nrc2
  for nrc2 only:  twocolid*
                  preprint      
                  proof
                  pagnf
                  trimmarks
                  finalverso
\end{verbatim}

\begin{description} \itemsep=0pt
   \item [{\tt breakaddress}] 
%
         This option affects the author IDbox at the bottom of the
         titlepage. It inserts a linebreak between the author name and
         address; the default setting has them print on the same line.

   \item [{\tt twocolid}] 
%
         For \texttt{nrc2} only. This option affects author
         information (the IDbox at page bottom): the text spans both
         columns.\footnote{The default is to set all IDbox material
         into the bottom of the left column.}

   \item [{\tt twocolid*}] 
%
         For \texttt{nrc2} only. This variation for the IDbox also
         spans both columns, but the material inside is itself set up
         in two columns.

   \item [{\tt preprint}] 
%
         This affects headers and footers, omitting such items as
         dates, page numbers, and so on. For any additional text in
         running heads (e.g., `Rapid Communication'), use
         \verb|\shortauthor|.

   \item [{\tt proof}] 
%
         Prints a centred footer on every page with the following
         text: `Proof/\'Epreuve'.\label{proof-pagnf}

        {\bf Note:} Comment out when {\small DOI} line must appear at
        bottom centre of opening page. See section~\ref{filenumber}.


   \item [{\tt pagnf}] 
%
         Prints a centred footer on every page with the following
         text: `Pagination not final/Pagination non
         finale'.

        {\bf Note:} Comment out when {\small DOI} line must appear at
        bottom centre of opening page. See section~\ref{filenumber}.


   \item [{\tt trimmarks}] 
%
         Prints cropmarks at all four corners. Note that trimmarks for
         \texttt{nrc2} are off the regular $8.5 \times 11$-inch paper,
         but will be visible if oversized paper is used.


   \item [{\tt finalverso}] 
%
         Specifies that the paper should end on a recto page (creating
         a blank, unnumbered page if the text doesn't for itself; the
         blank page does {\bf not} appear in the paper's page count.

\end{description}


%% Section 6.2
\subsection{Additional packages}

A number of additional packages are included in the template file {\tt
nrc-opening.tex}. Uncomment those packages which will be needed for
each specific article. 


\begin{description} \itemsep=0pt
   \item [{\tt \blash usepackage\char`\{color\char`\}}]

         The \verb|color| package is used for in-house production of
         reversed out text (white on black). Ensure that no driver
         option is specified here, as it would over-ride the in-house
         printer set-up. See section \ref{specialtitles} for details.

   \item [{\tt \blash usepackage\char`\{dcolumn\char`\}}]

         At present, this remains commented out as the \NRC's need for
         left-justified decimal alignment is not possible via
         \texttt{dcolumn}. Left-justified alignment at present is
         achieved by using the \verb|\llap|, \verb|\rlap|, and
         \verb|\phantom| commands.

   \item [{\tt \blash usepackage\char`\{url\char`\}}]

         Inserts line breaks into e-mail and website addresses. The
         package \textbf{and} its additional line of code must be
         uncommented.

   \item [{\tt \blash usepackage\char`\{array\char`\}}]

         Allows for raggedright columns in tables. The package
         \textbf{and} its additional line of code must be uncommented.

   \item [{\tt \blash usepackage\char`\{cases\char`\}}]

         Makes it possible for a left curly brace to span several
         lines of equations. The package \textbf{and} its additional
         line of code must be uncommented.

   \item [personal macros] 

         Over time, it may become apparent that some small
         modifications or shorthands are used in almost all
         papers. Until such changes are incorporated into the document
         classes, these should not be inserted into each article file
         but rather stored in a separate file, loaded via the
         \verb|\usepackage| command and inserted after all other
         packages.
\end{description}

\newpage


%% Section 6.3
\subsection{Package to remove}

If the user has specified
%
\begin{verbatim}
   \usepackage[T1]{fontenc}
\end{verbatim}
%
so as to enable French-language hyphenation to work when using {\small
CM} or restricted Type~1 fonts, the package invocation should be
deleted (the \NRC{} classes supply their own \verb|fontenc|
invocation).


%% Section 6.4
\subsection{Additional macros}

Following the loading of all packages and their options, files may
contain additional macros from the author (see
page~\pageref{authmacros} for instructions provided to authors). These
should be clearly marked off with, for example, a row of
\verb|%%|~signs both above and below. Keep in mind the potential for
author definitions to interfere or over-ride journal macros and
specifications; for example, authors may have commands to specify page
dimensions, or fonts for sections, or numbering schemes. Where these
do not collide with journal requirements, they can probably be safely
retained. However, where there is interference, journal definitions
take precedence. Ideally, authors will increasingly switch to using
the \NRC's document classes and reduce the chances of such problems.


%% Section 6.5
\subsection{Other additions in the preamble area}

Some if not all of the following macros are used by the \NRC's
in-house production team, and not by the author. They are input in the
file after all packages have been loaded, and before the
\verb|\begin{document}| statement:

\begin{verbatim}
   \setcounter{page}{<number>}
   \journal{<abbrev.>}
   \journalcode{<acro>}
   \volyear{<vol no.>}[<copyright year>]{<year>}
   \filenumber{<file no.>}

   \received{<complete date>}
   \revreceived{<complete date>}
   \accepted{<complete date>}
   \revaccepted{<complete date>}
   \IDdate or \IDdates{<Addit'nal text + date info>}
   \webpub{<complete date>}
   \commdate{<complete date>}
   
   \assoced{<name of assoc. ed.>}
   \corred{<name of correspond. ed.>}
\end{verbatim}

\begin{description} \itemsep=0pt
   \item [{\tt \blash
         setcounter\char`\{page\char`\}\char`\{...\char`\}}] \mbox{}  

         Insert starting page number for article. The information will
         be printed on the titlepage (bottom left) and in the running
         head; the complete page range will be calculated and inserted
         automatically when the file is run a second time.

\newpage


   \item [{\tt \blash journal\char`\{...\char`\}}] \mbox{}

         Specific journal abbreviations must be entered via this macro
         (e.g., \verb|Can. J. Civ. Eng.|).
         The \verb|\journal| command records the web address that
         will be used for this paper when it is published on the web;
         note that the \verb|\journalcode| command may be used as an
         alternative to \verb|\journal|.

         See Appendix~A for complete list of journal abbreviations.


   \item [{\tt \blash journalcode\char`\{...\char`\}}] \mbox{}

         The argument is the ``journal acronym'' (see table in
         Appendix~A for a list).  This acronym identifies the journal,
         and the \verb|\journalcode| command uses it to set the
         journal abbreviation and the web site addresses; note that
         the \verb|\journal| command may be used as an alternative to
         \verb|\journalcode|.


   \item [{\tt \blash
         volyear\char`\{...\char`\}[...]\char`\{...\char`\}}] \mbox{} 

         First argument is for the volume number. The second
         (optional) argument specifies the copyright year; if the
         argument is not present, the copyright year is assumed to be
         the same as the production year. The third argument
         specifies the publication year, which is used in the
         titlepage footer and in the left running head.


   \item [{\tt \blash filenumber\char`\{...\char`\}}] \mbox{}

         Insert the \NRC's file number here.\label{filenumber} The
         number will be appended to the canned text \verb|10-1139/|,
         which appears bottom centre of the
         opening page. If \verb|\filenumber| is missing, the following
         will be printed: {\it Zxx-xxx\/}. 

         {\bf Note:} The {\small DOI} line prints in the same location
         as text for the class options \verb|proof| and
         \verb|pagnf|. Comment these class options out once it becomes
         necessary to have the {\small DOI} line print at bottom
         centre of the opening page. See section~\ref{proof-pagnf}.


   \item [{\tt \blash filenumber*\char`\{...\char`\}}] \mbox{}

         Add an asterisk to the \verb|\filenumber| macro when it is
         necessary for the filenumber to be prefixed to the page
         numbers, in addition to appearing in the DOI line.

         All page numbers in the headers, and on the opening page at
         the bottom left will have the filenumber prefixed to them. 

         {\bf Note:} The filenumber will {\bf not} be prefixed to any
         page cross-references (via the \verb|\pageref| macro).


   \item [{\tt \blash received\char`\{...\char`\}}] \mbox{}

         Insert date as per journal style --- e.g., June 6, 2001 ---
         but without a final period (it is automatically inserted).
         The word `Received' (Fr.\ `Re\c cu le') will be automatically
         generated; however, the date must be input in French (e.g., 6
         juin 2001). This text appears in the author IDbox area.
         
\newpage

   \item [{\tt \blash revreceived\char`\{...\char`\}}] \mbox{}

         Same instructions as for \texttt{\blash received}. The text
         `Revision received' (Fr.\ `R\'evision re\c cue le') will be
         automatically generated. This text appears in the author
         IDbox area.
         

   \item [{\tt \blash accepted\char`\{...\char`\}}] \mbox{}

         Same instructions as for \texttt{\blash received}. The word
         `Accepted' (Fr.\ `Accept\'e le') will be automatically
         generated. This text appears in the author IDbox area.


   \item [{\tt \blash revaccepted\char`\{...\char`\}}] \mbox{}

         Same instructions as for \texttt{\blash received}. The text
         `Revision accepted' (Fr.\ `R\'evision accept\'ee le') will be
         automatically generated. This text appears in the author
         IDbox area.


   \item [{\tt \blash IDdates\char`\{...\char`\}}] \mbox{}

         Unlike \verb|\received| and \verb|\accepted|, no canned text or
         final punctuation is included, allowing the user to insert 
         customised text and/or date information, which appears in the
         author IDbox area.  An alias, \verb|\IDdate|, is also
         available.


   \item [{\tt \blash webpub\char`\{...\char`\}}] \mbox{}

         Insert date of publication at the \NRC{} website as per
         journal style --- e.g., June 6, 2001 --- but without a final
         period (it is automatically inserted). The text will appear
         in the author IDbox area. 

         For English-language articles, the text `Published on the
         {\small NRC} Research Press Web site at \verb|webaddress| on
         \verb|date|' will be automatically generated. The website
         address is generated by using either the \verb|\journal| or
         \verb|\journalcode| macros.

         For French-language articles, the text `Publi\'{e} sur le
         site Web des Presses scientifiques du {\small CNRC}, \`a
         \verb|webaddress|, le \verb|date|' will be automatically
         generated. Note that the date must be input in French (e.g.,
         6 juin 2001).


   \item [{\tt \blash commdate\char`\{...\char`\}}] \mbox{}

         Insert date as per journal style --- e.g., June 6, 2001 ---
         but without a final period (it is automatically inserted).
         The text `Written discussion of this article is welcomed and
         will be received by the Editor until' (Fr.\ `Les commentaires
         sur le contenu de cet article doivent \^{e}tre envoy\'{e}s au
         directeur scientifique de la revue avant le') will be
         automatically generated; however, the date must be input in
         French (e.g., 6 juin 2001). This text appears in the author
         IDbox area.  


   \item [{\tt \blash assoced\char`\{...\char`\}}] \mbox{}

         Insert name of associate editor, without a final period.  The
         text `Paper handled by Associate Editor' (Fr.\ `Production de
         l'article coordonn\'{e}e par le directeur scientifique
         associ\'{e}') will be automatically generated. This text
         appears in the author IDbox area.

\newpage


   \item [{\tt \blash corred\char`\{...\char`\}}] \mbox{}

         Insert name of associate editor, without a final period.  The
         text `Corresponding Editor:' (Fr.\ `Directeur scientifique
         correspondant\,:') will be automatically generated. This text
         appears in the author IDbox~area.
\end{description}


%% Section 6.6
\subsection{Special titleblocks\label{specialtitles}}

Some journal material requires a special heading: a solid black stripe
with reversed-out white lettering. The white-on-black effect requires
the presence of a special package in the preamble area, immediately
below the graphics package, in addition to the special title coding:

\begin{verbatim}
   \usepackage{color}
\end{verbatim}

\noindent Since both the \texttt{graphicx} package and \texttt{color}
share the same option, it is possible to merge them into one line:

\begin{verbatim}
   \usepackage{graphicx,color}
\end{verbatim}

\noindent Having added the \texttt{color} package, the actual special
title command will now work. There are two versions of the command:

\begin{description} \itemsep=0pt
   \item [{\tt \blash specialtitle}] \mbox{}

         This allows the regular titleblock (\verb|\title|, etc.) to
         be included with the special title; for example, a review
         article with its own title.
%
\begin{verbatim} 
\begin{document}
\specialtitle{REVIEW/SYNTH\`ESE}
\title{Regular article title}
\author{Someone's name here}
\address{Someplace nice and warm}
   \correspond
\shortauthor{Review/Synth\`ese}
\maketitle
\end{verbatim}

   \item [{\tt \blash specialtitle*}]\mbox{}

         The regular article titleblock cannot be used with this
         variant; for example, an editorial or other non-article
         material.
%
\begin{verbatim} 
\begin{document}
\specialtitle*{EDITORIAL/\'EDITORIAL}
\shortauthor{Editorial/\'Editorial}
\maketitle
\end{verbatim}

         For non-articles, the headers and footers are changed by
         using the \verb|\pagestyle{nrcplain}| command. The page
         numbers will appear at bottom centre, the \NRC{} Canada
         copyright footer is suppressed, and the running heads are
         suppressed entirely. For further adjustments to pagination, 
         see `Miscellaneous adjustments'.
\end{description}
          

%% Section 6.7
\subsection{Translations of abstracts/resum\'es}

The following lines are inserted at the end of each abstract or
resum\'e, before the \verb|\end{...}| statement:

\begin{description} \itemsep=0pt
   \item [{\tt \blash translation}] generates the text: `[Journal
         translation]'.

   \item [{\tt \blash traduit}] generates the text: `[Traduit par la
         r\'edaction]'.

   \item [{\tt \blash Traduit}] generates the text: `[Traduit par la
         R\'edaction]'.
\end{description}

\noindent Note that author files will only have one: an abstract or a
resum\'e. It is useful to insert a suitable \verb|\vspace| to
represent the approximate space the translation would require, so that
page breaks will not be unduly affected by the additional text.


%% Section 6.8
\subsection{Miscellaneous adjustments}

\begin{enumerate} \itemsep=0pt
   \item Journals requiring more space between lines will need the
         following command inserted into the preamble area:
%
\begin{verbatim}
   \easebaselines
\end{verbatim}

         This command will also adjust the inter-row spacing within
         tables (value of \verb|\arraystretch| increases to 1.05).


   \item For {\bf roman numerals}, with only page numbers in the
         footers, insert the following lines at the end of the
         preamble, just above the \verb|\begin{document}| line (notice
         that, in this example, pagination will begin with roman iii):
%
\begin{verbatim}
   \pagestyle{nrcplain}
   \pagenumbering{roman}
   \setcounter{page}{3}
\end{verbatim}

   \item To add parentheses (or any other design element) to (roman)
         page numbers, insert the following just before the
         \verb|\setcounter{page}{...}| command:
%
\begin{verbatim}
   \renewcommand\thepage{(\roman{page})}
\end{verbatim}

   \item For full-width text spanning two columns, the default left
         and right margins can be altered by using the following
         optional argument to the \verb|{WideText}| environment
         (recall that the default values are 0em on the left, 3em on
         the right):

\begin{verbatim}
\begin{WideText}[<l.margin>][<r.margin>]
  <text here>
\end{WideText}
\end{verbatim}
\end{enumerate}


%% Section 6.9
\subsection{Two-column bilingual texts}

Special coding at both the top of the file and around the bilingual
paragraphs is required.


%% Section 6.9.1 
\subsubsection{In the preamble\label{bilingpream}}

First, load the appropriate package and options. These are added after
the \verb|\documentclass| line in the preamble.

\newpage

\begin{enumerate} \itemsep=0pt
   \item if main (left column) language is English:

         \verb|\usepackage[french,english]{babel}|

         As English is the default, there is no need to specify it as
         an option to the document class.

   \item if main (left column) language is French, there is an
         additional option to add to the document class line:

         \verb|\documentclass[french]{nrc1}|

         \verb|\usepackage[english,french]{babel}|

         See sections~\ref{frenchclassoption} and \ref{babelinfo},
         which also discuss the \textsf{babel} package.
\end{enumerate}


%% Section 6.9.2
\subsubsection{In the bilingual text}

The next step is to code the English and French texts so that the tops
of matching paragraphs align horizontally. One set of codes surrounds
the entire bilingual set of paragraphs; another set of codes is put
around each matched set of English-French paragraphs.

\begin{verbatim}
1.  \begin{par-text}[<language>]
2.      \begin{par-para}
3.         ... <English paragraph> ... 
4.      \othercol
5.         ... <French paragraph> ... 
6.      \end{par-para}
7.
8.      \begin{par-para}
9.         ... <English paragraph> ... 
10.     \othercol
11.        ... <French paragraph> ... 
12.     \end{par-para}
13. \end{par-text}
\end{verbatim}

\begin{description} \itemsep=0pt
   \item [lines 1, 13:] The entire 2-column section of matching
         paragraphs is set inside a \verb|par-text| environment. This
         separates the parallel text portion of a file from other
         elements that may not require such formatting: headings or
         other text, graphics inclusions, etc. The \verb|par-text|
         command has one required option (language) and two width
         options (to change from default values).

   \item [line 1: \tt{[<language>]}] When the left-column language
         is\break English, \verb|[french]| is the right-column
         `option'.

         When left-column language is French, the right-column
         language option is \verb|[english]|. {\bf Also required} is
         the preamble code specified in section~\ref{bilingpream},
         item~2.

   \item [line 1:] The default column widths, which do {\bf not} need
         to be specified, are:

   \hspace{1pc}\begin{tabular}{lccc}
             & left col & inter-col & right col \\ [2pt]
    \tt nrc1 &  15.5pc  & 1.5pc     & 17pc    \\
    \tt nrc2 &  20.75pc & 1.5pc     & 20.75pc \\
   \end{tabular}

         To change left-column width and inter-column space (the right
         column width is calculated, based on these other two
         dimensions): 

\begin{verbatim}
\begin{par-text}[<lang.>][<dimen1>][<dimen2>]
\end{verbatim}

         where \verb|<dimen1>| is the new left column width, and
         \verb|<dimen2>| is the new inter-column space. If only the
         inter-column space, \verb|<dimen2>|, is to change,
         \verb|<dimen1>| must still be inserted, in order for the
         program to find the 2nd set of square brackets.

   \item [lines 2 and 6:] matching English/French paragraphs are set
         inside a \verb|par-para| environment

   \item [lines 3 and 5:] the matched sets of paragraphs (which can of
         course be longer than one line!) 

   \item [lines 4 and 10:] to signal the end of the left-column
         paragraph and the start of its matching right column
         paragraph, use \verb|\othercol|

   \item [line 7:] to make it easier to read the source file, separate
         each set of \verb|{par-para}| with a blank line or a {\tt \%}
         sign
\end{description}

\noindent {\bf Note:} The default vertical space between lines is set
at 1pc ($\approx$ one blank line). To change this at specific points,
explicit \verb|\vspace{...}| commands can be inserted between sets of
\verb|{par-para}|. To change this for the entire file, insert
\verb|\parallelparsep{<some dimen.>}| before \verb|{par-text}|.


%% Section 7:
\section{Final notes}

\begin{enumerate} \itemsep=0pt
   \item There is no dotless-j available in Adobe PostScript fonts.  
         The only dotless-j appears in math mode --- $\jmath$ ---
         which is accessed by \verb|$\jmath$|.

   \item Only English and French hyphenation are currently set up;
         English is the default, French is invoked by \textsf{babel},
         and the class selects French or English hyphenation as
         necessary, as it goes along (provided the appropriate
         hyphenation patterns have been installed).
\end{enumerate}

\clearpage


\section*{Appendix A}

\subsection*{Journal reference grid}

{\footnotesize
\begin{tabular}{@{}lllll@{}}
\hline \hline 
Journal name  
   & Journal abbreviation
   & Journal 
   & English website 
   & French website \T \\ 
   & & acronym & & \B \\ 
\hline 
Biochemistry and Cell Biology\T              
   & Biochem. Cell Biol.         
   & bcb                 
   & http://bcb.nrc.ca                                    
   & http://bbc.cnrc.ca \\ [2pt]
%
Canadian Geotechnical Journal                      
   & Can. Geotech. J.            
   & cgj                 
   & http://cgj.nrc.ca                                    
   & http://rcg.cnrc.ca \\  [2pt]
%
Canadian Journal of Botany                         
   & Can. J. Bot.                
   & cjb              
   & http://canjbot.nrc.ca                                    
   & http://revcanbot.cnrc.ca    \\  [2pt]
%
Canadian Journal of Chemistry                      
   & Can. J. Chem.               
   & cjc                 
   & http://canjchem.nrc.ca                                    
   & http://revcanchim.cnrc.ca \\  [2pt]
%
Canadian Journal of Civil Engineering              
   & Can. J. Civ. Eng.          
   & cjce                        
   & http://cjce.nrc.ca                                    
   & http://rcgc.cnrc.ca  \\  [2pt]
%
Canadian Journal of Earth Sciences                 
   & Can. J. Earth Sci.          
   & cjes                   
   & http://cjes.nrc.ca                                    
   & http://rcst.cnrc.ca  \\  [2pt]
%
Canadian Journal of Fisheries
   & Can. J. Fish. Aquat. Sci.   
   & cjfas             
   & http://cjfas.nrc.ca                                    
   & http://jcsha.cnrc.ca    \\
\quad and Aquatic Sciences \\  [2pt]
%
Canadian Journal of Forest Research                
   & Can. J. For. Res.           
   & cjfr                                     
   & http://cjfr.nrc.ca                                    
   & http://rcrf.cnrc.ca  \\  [2pt]
%
Canadian Journal of Microbiology                   
   & Can. J. Microbiol.          
   & cjm                       
   & http://cjm.nrc.ca                                    
   & http://rcm.cnrc.ca \\  [2pt]
%
Canadian Journal of Physics                        
   & Can. J. Phys.               
   & cjp  
   & http://cjp.nrc.ca                                    
   & http://rcp.cnrc.ca \\  [2pt]
%
Canadian Journal of Physiology
   & Can. J. Physiol.            
   & cjpp                     
   & http://cjpp.nrc.ca                                    
   & http://rcpp.cnrc.ca   \\
\quad and Pharmacology \\  [2pt]
%
Canadian Journal of Zoology                        
   & Can. J. Zool.               
   & cjz               
   & http://cjz.nrc.ca                                    
   & http://rcz.cnrc.ca  \\  [2pt]
%
Environmental Reviews                              
   & Environ. Rev.               
   & er                                    
   & http://er.nrc.ca                                    
   & http://de.cnrc.ca \\  [2pt]
%
Genome                                             
   & Genome
   & gen   
   & http://genome.nrc.ca                                    
   & http://genome.cnrc.ca \\  [2pt]
%
Journal of Environmental 
   & J. Environ. Eng. Sci.
   & jees                                
   & http://jees.nrc.ca                                    
   & http://rgse.cnrc.ca \\
\quad Engineering and Science \B \\ 
\hline \hline 
\end{tabular}
}

\clearpage

\onecolumn

\section*{Appendix B}

\subsection*{Canned Preamble: \texttt{nrc-opening.tex}}

\begin{verbatim}
%%%%%%%%%%%%%%%%%%%%%%%%%%%%%%%%%%%%%
%%%%%%%%%%%%%%%%%%%%%%%%%%%%%%%%%%%%%
%%
%% Typeset by                  , Research Press, NRC
%% Date: 
%% NRC, <name of journal>
%% 
%%%%%%%%%%%%%%
%%
%% 1. See original preamble material (at bottom of file) for
%%    details on source of current .tex file: conversion
%%    from word-processing program or author-generated TeX
%%    code. 
%%
%% 2. This template includes most options and packages used by 
%%    all the NRC journals. UNcomment those packages and options
%%    which are REQUIRED. 
%% 
%%%%%%%%%%%%%%%%%%%%%%%%%%%%%%%%%%%%%
%%%%%%%%%%%%%%%%%%%%%%%%%%%%%%%%%%%%%


%% 1. Class file (nrc1 or nrc2) + options (see userguide, pp.1-2; p.9):
\documentclass[%% french,        %% use with \usepackage[french]{babel}
               %% leqno,         %% only for nrc1 (default is right eqno)
               %% reqno,         %% only for nrc2 (default is left eqno)
               %% nonumbib,      %% biblio entries without nos.
%
               %% breakaddress,  %% linebreak btwn author(s) + address(es)
               %% twocolid,      %% IDbox spans 2 cols
               %% twocolid*,     %% 2-col IDbox
               %% preprint,      %% removes identifying nos. from headers/footers
               %% proof,         %% `Proof/Epreuve' in footer
               %% pagnf,         %% `Pagination not final/Pagination non finale'
               %% trimmarks,     %% add trimmarks
               %% finalverso,    %% final blank verso NOT included in pagerange
]{nrc2}                          %% choose one: nrc1 or nrc2

%% NOTE: authors may use the following options, which should be
%%       DELETED once the file comes in-house:
%%      
%%          usecmfonts    type1rest     genTeX


%% 2. Frequently used packages -- see pp.2-3 of userguide: 
%%    a. graphics-related:
%%    \usepackage{graphicx}       %% color not usually needed
%%    \usepackage[figuresright]{rotating} %% for landscape tables

%%    b. math-related: 
%%    \usepackage{amsmath}        %% math macros in wide use
%%    \usepackage{amssymb}        %% additional math symbols
%%    \usepackage{dcolumn}        %% decimal alignment for tables
%%    \usepackage{bm}             %% `bold math' via \bm command

%%    c. for website addresses:
%%    \usepackage{url}            %% inserts linebreaks automatically
%%    \NRCurl{url}

%%    d. biblio-related:
%%    \usepackage{cite}           %% enhances options for \cite commands

%%    e. for English-language papers:
%%    \usepackage[french,english]{babel}  

%%    f. for French-language papers: 
%%    \usepackage[english,french]{babel}  %% remember to add french as a
                                          %% CLASS option, above
%%    g. for ragged-right tables:
%%    \usepackage{array}
%%    \newcommand{\PreserveBackslash}[1]{\let\temp=\\#1\let\\=\temp}
%%    \let\PBS=\PreserveBackslash

%%    h. for left curly brace to span several lines of equations:
%%    \usepackage{cases}
%%    \expandafter\let\csname numc@left\expandafter\endcsname\csname 
%%                 z@\endcsname


%% 3. Resetting float parameters: 
%%    a. in nrc1:
%%    \renewcommand{\topfraction}{.95}
%%    \renewcommand{\textfraction}{.05}
%%    \renewcommand{\floatpagefraction}{.95}

%%    b. in nrc2:
%%    \renewcommand{\topfraction}{.95}
%%    \renewcommand{\floatpagefraction}{.95}
%%    \renewcommand{\dbltopfraction}{.95}
%%    \renewcommand{\textfraction}{.05}
%%    \renewcommand{\dblfloatpagefraction}{.95}


%% 4. Resetting journal-specific parameters:
%%    a. eqn nos. with section nos.:
%%    \numberby {equation}{section}
%%    \setcounter{equation}{0}

%%    b. in-line citations to use ( ) instead of default [ ]:
%%   \renewcommand{\citeleft}{(}
%%   \renewcommand{\citeright}{)}

%%    c. for JEES (to expand inter-line spacing; see p.12 of guide):
%%    \easebaselines


%% 5. Miscellaneous macros to always have available:
%%    a. shorthands:
\let\p=\phantom
\let\mc=\multicolumn

%%    b. struts for vertical spacing above/below rules in tables:
%%%%%%%%%%%%%%%%%%  beginning of Claudio Beccari's code:
%% Spacing commands for {tabular} (from TTN 2,3:10 -- Claudio
%%                                                    Beccari): 
%% Usage: a. use \T to put space below a line 
%%           (e.g., at top of a `cell' of text)
%%        b. use \B to put space above a line 
%%           (e.g., at bottom of a `cell' of text)
\newcommand\T{\rule{0pt}{2.6ex}}            % = `top' strut
\newcommand\B{\rule[-1.2ex]{0pt}{0pt}}      % = `bottom' strut
%%%%%%%%%%%%%%%%%%  end of Claudio's code 

%%%%%%%%%%%%%%%%%%%%%%%%%%%%%%%%%%%%%   end of class and package 
%%%%%%%%%%%%%%%%%%%%%%%%%%%%%%%%%%%%%   options, additional macros


%% Journal-specific information for opening page -- pp.9-11 of guide:
%% a. numbers:
\setcounter{page}{1}             %% replace 1 with starting page no.
\volyear{XX}{2001}               %% volume, year of journal 
\journal{}                       %% jrnl. abbrev. (see App.A of guide)
\journalcode{}                   %% jrnl. acro    (see App.A of guide)
\filenumber{}                    %% NRC file number
%% \filenumber*{}                %% prefixes \filenumber to all page nos.
                                 %% NOTE: COMMENT OUT class options
                                 %%             pagnf
                                 %%             proof
                                 %%       once no longer needed


%% b. dates:
\received{}                      %% insert date, no period
\revreceived{}                   %% <same>
\accepted{}                      %% <same>
\revaccepted{}                   %% <same>
%% \IDdates{}                       %% <same>. Use for `Revised ...' etc.
%% \webpub{}                        %% insert date
%% \commdate{}                      %% <same>


%% c. miscellaneous:
%%   \assoced{}                  %% insert name of Associate ed.
%%   \corred{}                   %% insert name of Corresponding ed.
%%   \dedication{}               %% insert text as neede
%%   \abbreviations{}            %% insert as needed


\begin{document}

%% Reversed titlebar -- see p.11 of userguide:
%% \specialtitle{}      %% for black stripe + text + regular title
%% \specialtitle*{}     %% black stripe + text only




%% Title, Author(s), Address(es) -- see p.4 of userguide for 
%%    various options to save time and keyboarding, esp. where
%%    authors share same address(s). 

\title{} 

%% Author 1:
\author[J.L. Humar]{John Larry Humar} %% opt. arg. ONLY if IDbox
                                      %% name is diff. from
                                      %% titleblock name 
\address{}                            %% address of 1st author


%% Author 2:
\author{M.A. Rahgozar}
\address{}

%% Author 3:
\author{Fred Murray}
\address{}


\shortauthor{Humar, Rahgozar, and Murray} %% for headers

%%%%%%%%
%% This line goes here in nrc1.
%% \maketitle			
%%%%%%%%


%% Abstract/Resume area -- see pp.5,12 of userguide:
\begin{abstract}
   Abstract text
%% \keywords{}
%% \translation 
\end{abstract}

\begin{resume}
   Texte du resume
%% \motscles{}
%% \Traduit %% or \traduit
\end{resume}

%%%%%%%%
%% This line goes here in nrc2.
%% \maketitle			
%%%%%%%%

%%%%%%%%%%%%%%%%%%%%%%   END OF TEMPLATE   %%%%%%%%%%%%%%%%%%%%%%

%% Ch. -- 11 NOV 02

\end{verbatim}

\end{document}


%% END OF FILE
