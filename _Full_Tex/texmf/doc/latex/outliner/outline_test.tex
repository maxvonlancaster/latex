% A test file for Victor Eijkhout's "outliner.sty".
\documentclass{article}
\usepackage{outliner}

% Page breaks wanted for levels 0 and 1:
\OutlinePageBreaks{1}
% level 0
\OutlineLevelStart0{\section*{#1}}
% level 1
\OutlineLevelStart1{\subsection*{#1}}
% level 2
\OutlineLevelStart2{\begin{itemize}\item#1.}
\OutlineLevelCont2{\item#1.}
\OutlineLevelEnd2{\end{itemize}}
% level 3
\OutlineLevelStart3{\par\noindent{\sl#1:}}

\begin{document}
\title{Test file for Victor's outline macros}
\author{Victor Eijkhout, {\tt victor@eijkhout.net}}
\maketitle

\begin{Outline}
\Level 0 {Outline test}
\Level 1 {First section}
This is just an example, remember?
\Level 2 {A point}
But it should be sufficient.
\Level 2 {Another point}
Even if you just want to copy from this file.
\Level 3 {something} is useful, \Level 3 {another thing} is not.
\Level 1 {Another section, another page}
\Level 2 {This one has this} \Level 2{And that.}

%%% for include files:
\SetBaseLevel{1}
\Level 0 {This could be an included chapter}
by another author.
\Level 1 {This point is level 1 by the author, 2 by inclusion}
Yadah yadah yadah.
\Level 1 {Another level 1 point made level 2}
Note that this time there is no page break.
\Level 0 {Another chapter, by another author}
More text.
% return to the main show:
\SetBaseLevel 0
%%%

\Level 0 {Conclusions}
So, do you like it?
\end{Outline}

\end{document}