\documentstyle[german,picinpar]{article}
\setlength{\parskip}{.66\baselineskip}
\setlength{\parindent}{0pt}
\begin{document}
\section*{Beispiele zu {\tt picinpar.sty}}
\font\yn=cmss17 scaled \magstep5 %oder sonst was gro"ses (yinit?)
\begin{verbatim}
\begin{window}[0,l,{\yn V},{}]
or einigen Jahren wurde von ...
... Abschnitts erfolgten automatisch.
\end{window}
\end{verbatim}

\begin{window}[0,l,{\yn V},{}]
or einigen Jahren wurde von Donald E.~Knuth im TUGboat ein kleines
Problem mit der Bitte um L"osung vorgestellt. Es handelte sich darum,
in einem Paragraphen ein Fenster zu erzeugen, in das man beliebigen Text
oder eine Zeichnung hineinsetzen kann. Prompt kamen dann in den folgenden
Ausgaben L"osungsvorschl"age: Einer von DEK pers"onlich, der andere von
Alan Hoenig. Der letztgenannte brachte die elegantere L"osung, die keine
manuellen Korrekturen mehr notwendig machte. Sein Makro verlangte lediglich
in den Parametern Informationen "uber die Breite und H"ohe der
freizulassenden Stelle im Paragraphen. Die Einz"uge und der Satz der
Fragmente des Abschnitts erfolgten automatisch.
\end{window}

\begin{verbatim}
\begin{figwindow}[2,r,{
\unitlength1cm
\begin{picture}(3,1.4)
\put(0.7,0.7){\circle*{0.2}}     \put(0.7,0.7){\circle{1.2}}
\put(0.7,0.7){\vector(0,1){0.6}} \put(2.5,0.7){\circle*{0.5}}
\end{picture}
},{Kreise und Pfeile}]
Was leisten nun diese Macros ...
... sieht hierbei wie folgt aus:
\end{figwindow}
\end{verbatim}

\begin{figwindow}[2,r,{
\unitlength1cm
\begin{picture}(3,1.4)
\put(0.7,0.7){\circle*{0.2}}     \put(0.7,0.7){\circle{1.2}}
\put(0.7,0.7){\vector(0,1){0.6}} \put(2.5,0.7){\circle*{0.5}}
\end{picture}
},{Kreise und Pfeile}]
Was leisten nun diese Macros? Einen kleinen Eindruck hat man schon
zu Anfang dieses Artikels bekommen: der erste Buchstabe des Absatzes
ist in einer anderen Schriftgr"o"se gedruckt und in den Absatz eingepa"st.
Es mu"s aber nicht unbedingt Text sein, der so einger"uckt wird. Es kann
ebenso eine \verb?picture?--Umgebung sein, die so in den Absatz eingef"ugt
wird. Das rechts stehende Beispiel, das aus "'\LaTeX , Eine Einf"uhrung"'
bekannt sein d"urfte, ist in einer \verb?minipage?--Umgebung gesetzt
worden. Die Eingabe im Text sieht hierbei wie oben aus.
\end{figwindow}

\begin{verbatim}
\begin{tabwindow}[1,r,{
\begin{tabular}[t]{|r|l|r@{:}l|}
\hline
1&HSV&12&0\\
...
\end{tabular}
},{Tabelle}]
Auch der Satz von Tabellen in ...
... w"urde sie etwas durchh"angen.

Nachdem die Tastatur gequ"alt ...
... noch die Nummerierung stimmt.
\end{tabwindow}
\end{verbatim}

\begin{tabwindow}[1,r,{
\begin{tabular}[t]{|r|l|r@{:}l|}
\hline
1&HSV&12&0\\
\hline
2&MSV&11&1\\
\hline
3&VfB&10&2\\
\hline
4&SVW&9&3\\
\hline
5&1.\ FCK&8&4\\
\hline
\end{tabular}
},{Tabelle}]
Auch der Satz von Tabellen in ein Fenster ist kein gro"ses Problem.
Man nimmt einfach eine Tabelle, oder denkt sich auch eine aus.
Dann wird daf"ur Sorge getragen, da"s sie auch mit der unteren
Zeile ausgerichtet wird. Ansonsten w"urde sie etwas durchh"angen.

Nachdem die Tastatur gequ"alt und das Hirn zermartert wurde kommt
dann so etwas heraus, wie es hier rechts zu sehen ist. Sogar die
Tabellenbezeichnung ist erschienen. Aber wirklich interessant wird's,
wenn auch bei der n"achsten Tabelle noch die Nummerierung stimmt.
\end{tabwindow}

\begin{verbatim}
\begin{tabwindow}[2,l,{
\begin{tabular}[t]{|r|l|r@{:}l|}
...
\end{tabular}
},{Tabelle}]
Was wahnsinnig ...
... sagt's denn?
\end{tabwindow}
\end{verbatim}

\begin{tabwindow}[2,l,{
\begin{tabular}[t]{|r|l|r@{:}l|}
\hline
1&HSV&12&0\\
\hline
2&MSV&11&1\\
\hline
3&VfB&10&2\\
\hline
4&SVW&9&3\\
\hline
5&1.\ FCK&8&4\\
\hline
\end{tabular}
},{Tabelle}]
Was wahnsinnig auflockernd wirkt, ist der st"andige Positionswechsel
von Abbildungen auf einer Seite. Das Auge wird nicht m"ude bei
der Erkundung der typographischen Vielfalt, die in gedruckten
Werken enthalten ist. Aber man sollte sich nichtsdestotrotz doch
auf die wesentlichen Mitteilungen konzentrieren, die mit derartigen
Werken vermittelt werden sollen.

Was sind aber nun die essentiellen Informationen? Naja, ein kleines
{\tt l} und eine {\tt 2} bewirkt ein g"anzlich anderes Erscheinungsbild
der Realit"at, wie sie mit den Augen eines Fu"sballfans der sechziger
Jahre gesehen sein k"onnte. Bevor wir es vergessen: Da war doch noch
die Unsicherheit wegen der Nummerierung von Tabellen. Na also, wer
sagt's denn?
\end{tabwindow}

\begin{verbatim}
\begin{tabwindow}[4,c,{
\begin{tabular}[t]{|r|l|r@{:}l|r@{:}l|}
...
\end{tabular}
},{Tabelle}]
\sloppy
Aber jetzt wird alles auf ...
... genug der Beispiele.
\end{tabwindow}
\end{verbatim}

\begin{tabwindow}[4,c,{
\begin{tabular}[t]{|r|l|r@{:}l|r@{:}l|}
\hline
1&HSV&12&0&14&1\\
\hline
2&MSV&11&1&10&4\\
\hline
3&VfB&10&2&12&9\\
\hline
4&SVW&9&3&11&9\\
\hline
5&1.\ FCK&8&4&10&10\\
\hline
\end{tabular}
},{Tabelle}]
\sloppy
Aber jetzt wird alles auf die Spitze getrieben. Oder vielmehr
in die Mitte gesetzt. Ja doch, auch der zentrierte Satz einer Tabelle
ist ebenfalls kein gro"ses Problem. Ein Problem ist jedoch, wie denn
der Text gelesen werden soll. Erst die linke Spalte und dann die
rechte oder einfach von links nach rechts? Oft kann man es nicht
falsch machen. Das tr"ostet doch ungemein!

Nun noch ein kleiner Hinweis in eigener Sache:
Beim zentrierten Satz ist es schon besser, wenn der Abstand zwischen
den Abschnitten auf $0pt$ gesetzt wird oder nur ein Paragraph verwendet
wird. Der Grund? Die Abst"ande k"onnen unabh"angig vom Zeilenabstand
sein und so ist ein optisch korrekter Satz der Spalten rechts und
links vom Fenster etwas m"uhsam {\tt :-)}. Um nichts dem Zufall zu
"uberlassen wird dies gleich von {\tt picinpar} erledigt.
\end{tabwindow}

Jetzt noch der Fall, bei dem eine Abbildung zentriert in den Text gesetzt
werden soll, aber rechts und links so wenig Platz ($<=72pt$) bleibt,
da"s dort die Probleme mit den Trennungen zu gro"s werden w"urden. Als
Beispiel wird im Fenster die \TeX{}--Eingabe zu diesem Fall gezeigt. Um
gleich etwaigen Problemen vorzubeugen, wird zun"achst wie folgt vorgegangen:

{\parskip0pt\topsep0pt
\begin{verbatim}
\newbox\pppbox
\setbox\pppbox=\vbox{\hsize=11cm
\begin{verbatim}
\begin{figwindow}[4,c,{\wframepic{ppp}
},
{Die Eingabe zu diesem Textteil!}]
Diese Vorgehensweise ist wegen der Benutzung
...
dem Einsatz im Wege.
\end{figwindow}
\end{verbatim}
\verb?\end{verbatim}?
\begin{verbatim}
}
\end{verbatim}
}
\newbox\pppbox
\setbox\pppbox=\vbox{\hsize=11cm
\begin{verbatim}
\begin{figwindow}[4,c,{\wframepic{ppp}
},
{Die Eingabe zu diesem Textteil!}]
Diese Vorgehensweise ist wegen der Benutzung
...
dem Einsatz im Wege.
\end{figwindow}
\end{verbatim}
}
\begin{figwindow}[4,c,{\wframepic{ppp}},
{Die Eingabe zu diesem Textteil!}]
Diese Vorgehensweise ist wegen der Benutzung von {\tt figwindow} und
der {\tt verbatim} dargestellten Eingabe notwendig. Aber ist ja auch egal.
Wichtig ist einfach, da"s festgestellt wird, ob rechts und links vom
zentrierten Bild noch genug Platz frei ist, um dort Text ohne gro"se
Probleme zu setzen. Die Wahl von $72pt$ ist mehr oder weniger willk"urlich
vorgenommen worden. Letztlich h"angt diese Grenze ja auch von der
verwendeten Sprache ab. Falls also nicht genug Platz an den Seiten
vorhanden ist, dann wird einfach nach dem Fenster mit dem Text fortgefahren.
Noch mal schnell ein Blick auf die Nummerierung der Abbildungen...
Ja, stimmt genau. Jetzt steht also dem Einsatz nichts im Wege.
\end{figwindow}

\begin{window}[3,r,{\arraycolsep=2.5pt \def\arraystretch{.75}
$\left(\begin{array}{ccccccccc}%
	 *&*&*&*&&&&&\\
        **&*&*&*&*&&&&\\ \cline{3-6}
	 *&*&\multicolumn{1}{|c}{*}&*&*&\multicolumn{1}{c|}{*}&&&\\
	 &*&\multicolumn{1}{|c}{*}&*&*&\multicolumn{1}{c|}{*}&*&&\\
	 &&\multicolumn{1}{|c}{*}&*&*&\multicolumn{1}{c|}{*}&*&*&\\ \cline{3-6}
	 &&&*&*&*&*&*&*\\
	 &&&&*&*&*&*&*\\
	 &&&&&*&*&*&*\\
	 &&&&&&*&*&*
	 \end{array}
   \right)$},{}]
Doch halt! Es handelt sich hier ja um \TeX{}. Und das ist Grund genug, auch
noch kurz auf mathematischen Formelsatz einzugehen. Sowohl im Fenster als
auch im Text neben dem Fenster k"onnen Formeln erscheinen, deren Dimensionen
beim Satz weitestgehend ber"ucksichtigt werden. Da ist also rechts ein
{\tt array} und hier folgt eine Formel:

\def\formel#1#2#3#4#5{#1(#4)-#1(#3)=
\int_{#3}^{#4}\sum_{j=0}^{#5}#2(x_j)
\prod_{{k=0}\atop{k\not= j}}^{#5}
{x-x_k\over x_j-x_k}\; dx}
$\formel Ffabn$

Der zentrierte Satz von Formeln neben dem Text ist bisher mit
{\tt picinpar.sty} nicht m"oglich, vielleicht kommt es aber eines
Tages. Und "uberhaupt hat sich gerade der Mathematiksatz als gro"ses
Problem bei der Entwicklung herausgestellt, da doch viel {\tt glue} in
den {\tt boxes} verwendet wird. Aber es geht ja doch so halbwegs.
\end{window}
\vfill
\centerline{Friedhelm Sowa, Heinrich--Heine--Universit"at D"usseldorf,
Universit"atsrechenzentrum}
\centerline{Email: sowa@convex.rz.uni-duesseldorf.de {\sl oder}
tex@ze8.rz.uni-duesseldorf.de}
\end{document}
