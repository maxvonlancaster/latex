\documentclass[a4paper]{ltxdoc}
\usepackage[colorlinks,
            bookmarks,
            hyperindex=false,
            pdfauthor={Nicola L.C. Talbot},
            pdftitle={probsoln: creating problem sheets optionally with solutions}]{hyperref}

\usepackage{color}
\usepackage{creatdtx}
\ifmakedtx{
\usepackage{probsoln}
\RecordChanges
\PageIndex
\CheckSum{886}

\renewcommand{\usage}[1]{\textit{\hyperpage{#1}}}
\renewcommand{\main}[1]{\hyperpage{#1}}
\newcommand{\see}[2]{\emph{see} #1}
\makeatletter
\def\index@prologue{\section*{Index}}
\makeatother
\CodelineNumbered
\newcommand{\dq}[1]{``#1''}

\definecolor{defbackground}{rgb}{0.8,1,1}

\newsavebox\defsbox
\newlength\defwidth
\newenvironment{definition}{%
\setlength{\fboxsep}{4pt}\setlength{\fboxrule}{1.25pt}%
\begin{lrbox}{\defsbox}%
\setlength\defwidth\linewidth
\addtolength\defwidth{-2\fboxrule}%
\addtolength\defwidth{-2\fboxsep}%
\begin{minipage}{\defwidth}\flushleft
}{%
\end{minipage}
\end{lrbox}%
\vskip10pt
\noindent
\fcolorbox{black}{defbackground}{\usebox\defsbox}%
\vskip10pt
}

\newlength\boxlength
\newsavebox\importantbox
\newenvironment{important}{%
\setlength{\fboxrule}{4pt}%
\setlength\boxlength{\linewidth}%
\addtolength\boxlength{-2\fboxsep}%
\addtolength\boxlength{-2\fboxrule}%
\begin{lrbox}{\importantbox}%
\begin{minipage}{\boxlength}%
}{\end{minipage}\end{lrbox}%
\par\vskip10pt\noindent
\fcolorbox{red}{white}{\usebox{\importantbox}}\par\vskip10pt}
}
{
\usepackage{html}
\usepackage{makeidx}
\renewcommand{\LaTeX}{LaTeX}
\renewcommand{\LaTeXe}{LaTeX2e}
\newcommand{\dq}[1]{"#1"}
\newenvironment{important}{\textbf{NOTE:} }{}
\newenvironment{definition}{\htmlrule}{\htmlrule}
\html{\newcommand{\SpecialMainIndex}[1]{\index{#1@\cs{#1}}}}
}

\newcommand*{\ics}[1]{\cs{#1}\SpecialMainIndex{#1}}

\ifmakedtx{
\newcommand{\pkgopt}[1]{\textsf{#1}\index{package options>#1=\textsf{#1}|hyperpage}}
\newcommand*{\sty}[1]{\textsf{#1}}
\newcommand*{\isty}[1]{\sty{#1}\index{#1 package=\sty{#1} package|hyperpage}}
\newcommand*{\env}[1]{\textsf{#1}}
\newcommand{\ienv}[1]{\env{#1}\index{#1 environment=\env{#1} environment|hyperpage}}
\newcommand*{\bool}[1]{\textsf{#1}\index{#1 boolean variable=\textsf{#1} boolean variable|hyperpage}}
\newcommand{\csdef}[1]{\cs{#1}}
}
{
\newcommand{\pkgopt}[1]{\texttt{#1}\index{package options!#1@\texttt{#1}}}
\newcommand{\sty}[1]{\texttt{#1}}
\newcommand{\isty}[1]{\sty{#1}\index{#1@\sty{#1}}}
\newcommand{\env}[1]{\texttt{#1}}
\newcommand{\ienv}[1]{\env{#1}\index{#1 environment@\env{#1} environment}}
\newcommand*{\bool}[1]{\texttt{#1}\index{#1 boolean variable@\texttt{#1} boolean variable}}
\newcommand{\csdef}[1]{\ics{#1}}
\renewcommand{\DescribeEnv}[1]{\index{#1 environment@\env{#1} environment}}
\newcommand{\toTop}{\mbox{}\par\htmlref{Top}{top}}
}

\begin{document}\ifmakedtx{}{\label{top}}
\MakeShortVerb{"}
\DeleteShortVerb{\|}

 \title{probsoln v3.0: creating problem sheets optionally with solutions}
 \author{Nicola L.C. Talbot\\[10pt]
School of Computing Sciences\\
University of East Anglia\\
Norwich. Norfolk\\
NR4 7TJ. United Kingdom.\\
\url{http://theoval.cmp.uea.ac.uk/~nlct/}}

 \date{26th August 2008}
 \maketitle
\tableofcontents

 \section{Introduction}
The \sty{probsoln} package is designed for teachers or
lecturers who want to create problem sheets for their 
students.  This package was designed with specifically 
mathematics problems in mind, but can be used for other 
subjects as well.  The idea is to create a file containing a
large number of problems with their solutions which can be 
read in by \LaTeX, and then select a number of problems to
typeset.  This means that once the database has been set up,
each year you can easily create a new problem sheet that is
sufficiently different from the previous year, thus
preventing the temptation of current students seeking out
the previous year's students, and checking out their
answers.  There is also an option that can be passed to the
package to determine whether or not the solutions should be
printed.  In this way, one file can either produce the
student's version or the teacher's version.
\ifmakedtx{}{\toTop}

\section{Package Options}\label{sec:pkgopt}
The following options may be passed to this package:
\begin{description}
\item[\pkgopt{answers}]  Show the answers
\item[\pkgopt{noanswers}] Don't show the answers (default)
\item[\pkgopt{draft}] Display the label and dataset name when a problem is used
\item[\pkgopt{final}] Don't display label and dataset name when a problem is used
\end{description}
\ifmakedtx{}{\toTop}

\section{Showing and Hiding Solutions}\label{sec:showanswers}

In addition to the \pkgopt{answers} and \pkgopt{noanswers} package
options, it is also possible to show or suppress the solutions
using \DescribeMacro{\showanswers}\cs{showanswers} and
\DescribeMacro{\hideanswers}\cs{hideanswers}, respectively.
The boolean variable \bool{showanswers} determines whether the
answers should be displayed. You can use this value with the
\isty{ifthen} package to specify different text depending on 
whether the solutions should be displayed. For example:
\begin{verbatim}
Assignment 1\ifthenelse{\boolean{showanswers}}{ (Solution Sheet)}{}
\end{verbatim}
Alternatively you can use \ics{ifshowanswers}\ldots\cs{else}\ldots
\cs{fi}:
\begin{verbatim}
Assignment 1\ifshowanswers\space (Solution Sheet)\fi
\end{verbatim}

For longer passages, you can use the environments \DescribeEnv{onlyproblem}\env{onlyproblem}
and \DescribeEnv{onlysolution}\env{onlysolution}. For example:
\begin{verbatim}
\begin{onlyproblem}%
What is the derivative of $f(x) = x^2$?
\end{onlyproblem}%
\begin{onlysolution}%
$f'(x) = 2x$
\end{onlysolution}
\end{verbatim}
The above will only display the question if \bool{showanswers}
is false and will only display the solution if \bool{showanswers}
is true. If you want the question to appear in the answer
sheet as well as the solution, then don't put the question in
the \env{onlyproblem} environment:
\begin{verbatim}
What is the derivative of $f(x) = x^2$?
\begin{onlysolution}%
Solution: $f'(x) = 2x$
\end{onlysolution}
\end{verbatim}
\begin{important}
You can't use verbatim text in the body of the
\ienv{onlyproblem} or \ienv{onlysolution} environments.
If you need verbatim-like text, then try packages such as
\isty{alltt}. Remember that you also can't use verbatim
text in command arguments.
\end{important}
\ifmakedtx{}{\toTop}

\section{General Formatting Commands}\label{sec:formatting}

The commands and environments described in this section are
provided to assist formatting problems and their solutions.
\DescribeEnv{solution}
\begin{definition}
\verb|\begin{solution}|\meta{text}\verb|\end{solution}|
\end{definition}
This is equivalent to 
\verb|\par\noindent\textbf{\solutionname}:|\meta{text}
where \DescribeMacro{\solutionname}\cs{solutionname} defaults
to \dq{Solution}. Note that you must place the \env{solution}
environment inside the \ienv{onlysolution} environment or
between \ics{ifshowanswers}\ldots\cs{fi} to ensure that it
is suppressed when the solutions are not wanted. (See
\ifmakedtx{\autoref{sec:showanswers}}{\htmlref{Showing and
Hiding Solutions}{sec:showanswers}}.) 


Note that the \sty{probsoln} package will only define the 
\env{solution} environment if it is not already defined.

\DescribeEnv{textenum}
\begin{definition}
\verb|\begin{textenum}|\ldots\verb|\end{textenum}|
\end{definition}
The \env{textenum} environment is like the \ienv{enumerate} 
environment but is in-line. It uses the same counter that the
\env{enumerate} environment would use at that level so the
question can be compact but the answer can use \env{enumerate}
instead. For example:
\begin{verbatim}
\begin{onlyproblem}%
Differentiate the following: 
\begin{textenum}
\item $f(x)=2^x$;
\item $f(x)=\cot(x)$
\end{textenum}
\end{onlyproblem}
\begin{onlysolution}
\begin{enumerate}
\item 
\begin{align*}
f(x) &= 2^x = \exp(\ln(x^2)) =\exp(2\ln(x))\\
f'(x) &= \exp(2\ln(x))\times \frac{2}{x}\\
 &= f(x)\frac{2}{x}
\end{align*}
\item
\begin{align*}
f(x) &= \cot(x) = (\tan(x))^{-2}\\
f'(x) &= -(\tan(x))^{-2}\times\sec^2(x)\\
&=-\csc^2x
\end{align*}
\end{enumerate}
\end{onlysolution}
\end{verbatim}
In this example, the items in the question are brief, so an
\ienv{enumerate} environment would result in a lot of 
unnecessary white space,
but the answers require more space, so an \env{enumerate}
environment is more appropriate. Since the \ienv{textenum}
environment uses the same counters as the \env{enumerate}
environment, the question and answer sheets use consistent
labelling. Note that there are other packages available on
CTAN that you can use to create in-line lists. Check the
\ifmakedtx{\href{http://www.tex.ac.uk/tex-archive/help/Catalogue/bytopic.html\#enumeration}{TeX 
Catalogue}}{\htmladdnormallink{TeX 
Catalogue}{http://www.tex.ac.uk/tex-archive/help/Catalogue/bytopic.html\#enumeration}} for
further details.

\DescribeMacro{\correctitem}\DescribeMacro{\incorrectitem}
\begin{definition}
\csdef{correctitem}\\
\csdef{incorrectitem}
\end{definition}
You can use the commands \cs{correctitem} and \cs{incorrectitem} 
in place of \ics{item}. If the solutions are suppressed, these
commands behave in the same way as \cs{item}, otherwise they
format the item label using one of the commands:
\DescribeMacro{\correctitemformat}\DescribeMacro{\incorrectitemformat}
\begin{definition}
\csdef{correctitemformat}\marg{label}\\
\csdef{incorrectitemformat}\marg{label}
\end{definition}
For example:
\begin{verbatim}
Under which of the following functions does $S=\{a_1,a_2\}$
become a probability space?
\begin{enumerate}
\incorrectitem $P(a_1)=\frac{1}{3}$, $P(a_2)=\frac{1}{2}$
\correctitem $P(a_1)=\frac{3}{4}$, $P(a_2)=\frac{1}{4}$
\correctitem $P(a_1)=1$, $P(a_2)=0$
\incorrectitem $P(a_1)=\frac{5}{4}$, $P(a_2)=-\frac{1}{4}$
\end{enumerate}
\end{verbatim}
\ifmakedtx{}{\toTop}

\section{Defining a Problem}\label{sec:defproblem}

It is possible to construct a problem sheet with solutions using
the commands described in the previous sections, however it is
also possible to define a set of problems for later use. In this
way you can create an external file containing many problems some
or all of which can be loaded and used in a document. The
\sty{probsoln} package has a default data set labelled
\dq{default} in which you can store problems or you can create
multiple data sets. You can then iterate through each problem
in a problem set. You can use a previously defined problem more
than once, which means that by judicious use of \ienv{onlyproblem},
\ienv{onlysolution} or the \bool{showanswers} boolean variable
in conjunction with \ics{showanswers} and \ics{hideanswers},
you can print the solutions in a different location to the
questions (for example in an appendix).

\DescribeEnv{defproblem}
\begin{definition}
\verb|\begin{defproblem}|\oarg{n}\marg{label}\meta{definition}\verb|\end{defproblem}|
\end{definition}
This defines the problem whose label is given by \meta{label}.
The label must be unique for a given data set and should not
contain active characters. (Active characters include the special
characters such as \$ and \&, but some packages may make other
symbols active. For example, the \isty{ngerman} and \isty{babel}
packages make certain punctuation active. Check the relevant 
package documentation for details.)

If \env{defproblem} occurs in the document or is included via
\cs{input} or \cs{include}, then the problem will be added to
the default data set. If \env{defproblem} occurs in an external
file that is loaded using one of the commands defined in
\ifmakedtx{\autoref{sec:load}}{\htmlref{Loading Problems From 
External Files}{sec:load}} then the problem will be added to
the specified data set.

The contents of the \env{defproblem} environment should be the
text that defines the problem. This may include any of the
commands defined in 
\ifmakedtx{\autoref{sec:showanswers} and 
\autoref{sec:formatting}}{\htmlref{Showing and Hiding 
Solutions}{sec:showanswers} and \htmlref{General Formatting 
Commands}{sec:formatting}}.

The problem may optionally take \meta{n} arguments (where 
\meta{n} is from 0 to 9). The arguments can be referenced
in the definition via \texttt{\#1},\ldots,\texttt{\#9}.
If \meta{n} is omitted then the problem doesn't take any
arguments.
The following example defines a problem with one argument:
\begin{verbatim}
\begin{defproblem}[1]{diffsin}
Differentiate $f(x)=\sin(#1x)$.
\begin{onlysolution}%
\begin{solution}
$f'(x) = #1\cos(#1x)$
\end{solution}
\end{onlysolution}
\end{defproblem}
\end{verbatim}

\DescribeMacro{\newproblem}
\begin{definition}
\csdef{newproblem}\oarg{n}\marg{label}\marg{problem}\marg{solution}
\end{definition}
This is a shortcut command for:
\begin{ttfamily}\obeylines
\cs{begin}\{defproblem\}\oarg{n}\marg{label}\%
\meta{problem}\%
\cs{begin}\{onlysolution\}\%
\cs{begin}\{solution\}\%
\meta{solution}\%
\cs{end}\{solution\}\%
\cs{end}\{onlysolution\}\%
\cs{end}\{defproblem\}
\end{ttfamily}
For example:
\begin{verbatim}
\newproblem[1]{diffsin}{%
\(f(x) = \sin(#1x)\)
}{%
\(f'(x) = #1\cos(#1x)\)
}
\end{verbatim}
is equivalent to
\begin{verbatim}
\begin{defproblem}[1]{diffcos}%
\(f(x) = \cos(#1x)\)
\begin{onlysolution}%
\begin{solution}%
\(f'(x) = -#1\sin(#1x)\)
\end{solution}%
\end{onlysolution}%
\end{defproblem}
\end{verbatim}
(In this example, the argument will need to be a positive number
to avoid a double minus in the answer. If you want to perform
floating point arithmetic on the arguments, then try the
\isty{fp} package.)

\DescribeMacro{\newproblem*}
\begin{definition}
\csdef{newproblem*}\oarg{n}\marg{label}\marg{definition}
\end{definition}
This is a shortcut for:
\begin{ttfamily}\obeylines
\cs{begin}\{defproblem\}\oarg{n}\marg{label}\%
\meta{definition}\%
\cs{end}\{defproblem\}
\end{ttfamily}

\begin{important}
Verbatim text must not be used in the contents of \ienv{defproblem}
or in any of the arguments to \cs{newproblem} or
\cs{newproblem*}. If you need verbatim-like text consider using
\cs{texttt} or the \isty{alltt} package.
\end{important}
\ifmakedtx{}{\toTop}

\section{Using a Problem}\label{sec:useproblem}

Once you have defined a problem using \ienv{defproblem} or
\cs{newproblem} (see 
\ifmakedtx{\autoref{sec:defproblem}}{\htmlref{Defining a 
Problem}{sec:defproblem}}), you can later display the problem 
using:
\DescribeMacro{\useproblem}
\begin{definition}
\csdef{useproblem}\oarg{data set}\marg{label}\marg{arg\ifmakedtx{$_1$}{\HTMLcode{SUB}{1}}}\ldots
\marg{arg\ifmakedtx{$_N$}{\HTMLcode{SUB}{N}}}
\end{definition}
where \meta{data set} is the name of the data set that contains
the problem (the default data set is used if omitted), 
\meta{label} is the label identifying the required problem and
\meta{arg\ifmakedtx{$_1$}{\HTMLcode{SUB}{1}}}, \ldots, 
\meta{arg\ifmakedtx{$N$}{\HTMLcode{SUB}{N}}} 
are the arguments to pass to the problem, if the problem was 
defined to have arguments (where \ifmakedtx{$N$}{N} is the number 
of arguments specified when the problem was defined).

For example, in the previous section the problem \texttt{diffcos} 
was defined to have one argument, so it can be used as follows:
\begin{verbatim}
\useproblem{diffcos}{3}
\end{verbatim}
This will be equivalent to:
\begin{verbatim}
\(f(x) = \cos(3x)\)
\begin{onlysolution}%
\begin{solution}%
\(f'(x) = -3\sin(3x)\)
\end{solution}%
\end{onlysolution}%
\end{verbatim}
\ifmakedtx{}{\toTop}

\section{Loading Problems From External Files}\label{sec:load}

You can store all your problem definitions (see
\ifmakedtx{\autoref{sec:defproblem}}{\htmlref{Defining New 
Problems}{sec:defproblem}}) in an external file. 
These problems can all be appended to the default data set by
including the file via \ics{input} or they can be appended
to other data sets using one of the commands described below.
Once you have loaded all the required problems, you can
iterate through the data sets using the commands described
in \ifmakedtx{\autoref{sec:foreach}}{\htmlref{Iterating 
Through Datasets}{sec:foreach}}. Note that the commands below
will create a new data set, if the named data set doesn't
exist.

\DescribeMacro{\loadallproblems}
\begin{definition}
\csdef{loadallproblems}\oarg{data set}\marg{filename}
\end{definition}
This will load all problems defined in \meta{filename} and
append them to the specified data set, in the order in which
they are defined in the file. If \meta{data set} is
omitted, the default data set will be used. If \meta{data set}
doesn't exist, it will be created.

\DescribeMacro{\loadselectedproblems}
\begin{definition}
\csdef{loadselectedproblems}\oarg{data set}\marg{labels}\marg{filename}
\end{definition}
This is like \cs{loadproblems}, but only those problems whose
label is listed in the comma-separated list \meta{labels} are
loaded. For example, if I have some problems defined in the
file \texttt{derivatives.tex}, then
\begin{verbatim}
\loadselectedproblems{diffsin,diffcos}{derivatives}
\end{verbatim}
will only load the problems whose labels are \texttt{diffsin}
and \texttt{diffcos}, respectively. All the other problems in 
the file will remain undefined.

\DescribeMacro{\loadrandomproblems}
\begin{definition}
\csdef{loadrandomproblems}\oarg{data set}\marg{n}\marg{filename}
\end{definition}
This randomly loads \meta{n} problems from \meta{filename} and
adds them to the given data set. If \meta{data set} is omitted,
the default data set is assumed. Note that the problems will be
added to the data set in a random order, not in the order in
which they were defined. There must be at least \meta{n} problems
defined in \meta{filename}.

\begin{important}
It is generally not a good idea to place anything in 
\meta{filename} that is not inside the body of \ienv{defproblem} 
or in the arguments to \ics{newproblem} or \ics{newproblem*}.
All the commands in this section input the external file within
a local scope, so commands definitions would need to be made
global to have any effect. In addition, \cs{loadrandomproblems}
has to load each file twice, which means that anything outside
a problem definition will be parsed twice.
\end{important}

\ifmakedtx{}{\toTop}

\section{Iterating Through Datasets}\label{sec:foreach}

Once you have defined all your problems for a given data set,
you can use an individual problem with \ics{useproblem}
(see \ifmakedtx{\autoref{sec:useproblem}}{\htmlref{Using a 
Problem}{sec:useproblem}}) but it is more likely that you will
want to iterate through all the problems so that you don't need
to remember the labels of all the problems you have defined.

\DescribeMacro{\foreachproblem}
\begin{definition}
\csdef{foreachproblem}\oarg{data set}\marg{body}
\end{definition}
This does \meta{body} for each problem in the given data set.
If \meta{data set} is omitted, the default data set is used.
Within \meta{body} you can use 
\DescribeMacro{\thisproblem}\cs{thisproblem} to use the
current problem and \DescribeMacro{\thisproblemlabel}\cs{thisproblemlabel} 
to access the current label. If the problem requires arguments,
you will be prompted for them, so if you want to use this
approach you will need to use \LaTeX\ in interactive mode. If
you do provide arguments, they will be stored in the event that
you need to iterate through the data set again. The
arguments will be included in \cs{thisproblem}, so you only
need to use \cs{thisproblem} without having to specify
\ics{useproblem}. For example, to iterate through all problems
in the default data set:
\begin{verbatim}
\begin{enumerate}
\foreachproblem{\item\thisproblem}
\end{enumerate}
\end{verbatim}

\DescribeMacro{\foreachdataset}
\begin{definition}
\csdef{foreachdataset}\marg{cmd}\marg{body}
\end{definition}
This does \meta{body} for each of the defined data sets. Within
\meta{body}, \meta{cmd} will be set to the name of the current
data set. For example, to display all problems in all data sets:
\begin{verbatim}
\begin{enumerate}
\foreachdataset{\thisdataset}{%
\foreachproblem[\thisdataset]{\item\thisproblem}}
\end{enumerate}
\end{verbatim}

Suppose I have two external files called
\texttt{derivatives.tex} and \texttt{probspaces.tex} which
define problems using both \ienv{onlyproblem} and 
\ienv{onlysolution} for example:
\begin{verbatim}
\begin{defproblem}{cosxsqsinx}%
\begin{onlyproblem}%
$y = \cos(x^2)\sin x$.%
\end{onlyproblem}%
\begin{onlysolution}%
\[\frac{dy}{dx} = -\sin(x^2)2x\sin x + \cos(x^2)\cos x\]
\end{onlysolution}%
\end{defproblem}
\end{verbatim}
I can write a document that creates two data sets, one for
the derivative problems and one for the problems about
probability spaces. I can then use \ics{hideanswers} and
iterate through the require data set to produce the problems.
Later, I can use \ics{showanswers} and iterate over all problems defined in both data
sets to produce the chapter containing all the answers. When
displaying the questions, I have taken advantage of the fact that
I can cross-reference items within an \ienv{enumerate} environment,
and redefined \ics{theenumi} to label the questions according to
the chapter. The cross-reference label is constructed from
the problem label and is referenced in the answer section to
ensure that the answers have the same label as the questions.
\begin{verbatim}
\documentclass{report}
\usepackage{probsoln}
\begin{document}
\hideanswers
\chapter{Differentiation}
% randomly select 25 problems from derivatives.tex and add to
% the data set called 'deriv'
\loadrandomproblems[deriv]{25}{derivatives}

% Display the problems
\renewcommand{\theenumi}{\thechapter.\arabic{enumi}}
\begin{enumerate}
\foreachproblem[deriv]{\item\label{prob:\thisproblemlabel}\thisproblem}
\end{enumerate}
% You may need to change \theenumi back here

\chapter{Probability Spaces}
% randomly select 25 problems from probspaces.tex and add to
% the data set called 'spaces'
\loadrandomproblems[spaces]{25}{probspaces}

% Display the problems
\renewcommand{\theenumi}{\thechapter.\arabic{enumi}}
\begin{enumerate}
\foreachproblem[spaces]{\item\label{prob:\thisproblemlabel}\thisproblem}
\end{enumerate}
% You may need to change \theenumi back here

\appendix

\chapter{Solutions}
\showanswers
\begin{itemize}
\foreachdataset{\thisdataset}{%
\foreachproblem[\thisdataset]{\item[\ref{prob:\thisproblemlabel}]\thisproblem}
}
\end{itemize}

\end{document}
\end{verbatim}

\ifmakedtx{}{\toTop}

\section{Random Number Generator}\label{sec:random}

This package provides a pseudo-random number generator that is used
by \ics{loadrandomproblems}.

\DescribeMacro{\PSNrandseed}
\begin{definition}
\csdef{PSNrandseed}\marg{n}
\end{definition}
This sets the seed to \meta{n} which must be a non-zero integer.
For example, to generate a different set of random numbers
every time you \LaTeX\ your document,\footnote{assuming you
leave at least a minute between runs.} put the following in your
preamble:
\begin{verbatim}
\PSNrandseed{\time}
\end{verbatim}
or to generate a different set of random numbers every year you
\LaTeX\ your document:
\begin{verbatim}
\PSNrandseed{\year}
\end{verbatim}

\DescribeMacro{\PSNgetrandseed}
\begin{definition}
\csdef{PSNgetrandseed}\marg{register}
\end{definition}
This stores the current seed in the count register specified by 
\meta{register}.
For example:
\begin{verbatim}
\newcount\myseed
\PSNgetrandseed{\myseed}
\end{verbatim}

\DescribeMacro{\PSNrandom}
\begin{definition}
\csdef{PSNrandom}\marg{register}\marg{n}
\end{definition}
Generates a random integer from 1 to \meta{n} and stores in 
the count register specified by \meta{register}. For example,
the following generates an integer from 1 to 10 and stores it
in the register \cs{myreg}:
\begin{verbatim}
\newcount\myreg
\PSNrandom{\myreg}{10}
\end{verbatim}

\DescribeMacro{\random}
\begin{definition}
\csdef{random}\marg{counter}\marg{min}\marg{max}
\end{definition}
Generates a random integer from \meta{min} to \meta{max} and
stores in the given counter. For example, the following generates
a random number between 3 and 8 (inclusive) and stores it in
the counter \texttt{myrand}.
\begin{verbatim}
\newcounter{myrand}
\random{myrand}{3}{8}
\end{verbatim}

\DescribeMacro{\doforrandN}
\begin{definition}
\csdef{doforrandN}\marg{n}\marg{cmd}\marg{list}\marg{text}
\end{definition}
Randomly selects \meta{n} values from the comma-separated list
given by \meta{list} and iterates through this subset. On
each iteration it sets \meta{cmd} to the current value and
does \meta{text}. For example, the following will load a
randomly selected problem from two of the listed files (where
\texttt{file1.tex}, \texttt{file2.tex} and \texttt{file3.tex}
are files containing at least one problem):
\begin{verbatim}
\doforrandN{2}{\thisfile}{file1,file2,file3}{%
\loadrandomproblems{1}{\thisfile}}
\end{verbatim}

\ifmakedtx{}{\toTop}

\section{Compatibility With Versions Prior to 3.0}

Version 3.0 of the \sty{probsoln} package completely changed the
structure of the package, but the commands described in this
section have been provided to maintain compatibility with 
earlier versions. The only problems that are likely to occur are
those where commands are contained within groups. This will 
effect any commands that are contained in external files that are
outside of the arguments to \ics{newproblem} and \ics{newproblem*}.
However, since the external files had to be parsed twice in
order to load the problems, this shouldn't be an issue as adding
anything other than problem definitions in those files would
be problematic anyway.

The other likely difference is where the random generator is
used in a group. This includes commands such as 
\ics{selectrandomly}. For example, if your document contained
something like:
\begin{verbatim}
\begin{enumerate}
\selectrandomly{file1}{8}

\item Solve the following:
\begin{enumerate}
\selectrandomly{file2}{4}
\end{enumerate}

\selectrandomly{file3}{2}
\end{enumerate}
\end{verbatim}
Then using versions prior to v3.0 will produce a different
set of random numbers since the second \cs{selectrandomly}
is in a different level of grouping. If you want to ensure
that the document produces exactly the same random set with
the new version as with the old version, you will need to
get and set the random number seed. For example, the above
would need to be modified so that it becomes:
\begin{verbatim}
\begin{enumerate}
\selectrandomly{file1}{8}

\item Solve the following:
\newcount\oldseed
\PSNgetrandseed{\oldseed}
\begin{enumerate}
\selectrandomly{file2}{4}
\end{enumerate}
\PSNrandseed{\oldseed}

\selectrandomly{file3}{2}
\end{enumerate}
\end{verbatim}

\DescribeMacro{\selectrandomly}
\begin{definition}
\csdef{selectrandomly}\marg{filename}\marg{n}
\end{definition}
This is now equivalent to:
\begin{ttfamily}\obeylines
\{\cs{loadrandomproblems}\oarg{filename}\marg{n}\marg{filename}\}\%
\cs{foreachproblem}\oarg{filename}\marg{\cs{PSNitem}\cs{thisproblem}\cs{endPSNitem}}
\end{ttfamily}

\DescribeMacro{\selectallproblems}
\begin{definition}
\csdef{selectallproblems}\marg{filename}
\end{definition}
This is now equivalent to:
\begin{ttfamily}\obeylines
\{\cs{loadallproblems}\oarg{filename}\marg{filename}\}\%
\cs{foreachproblem}\oarg{filename}\marg{\cs{PSNitem}\cs{thisproblem}\cs{endPSNitem}}
\end{ttfamily}

Note that in both the above cases, a new data set is created
with the same name as the file name.

\ifmakedtx{}{\toTop}

 \StopEventually{\clearpage\phantomsection\addcontentsline{toc}{section}{Index}\PrintIndex}

\ifmakedtx{}{\printindex\toTop}
\end{document}
