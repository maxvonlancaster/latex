% Copyright 2005, 2006 TeX Users Group.
% 
% Copying and distribution of this file, with or without modification,
% are permitted in any medium, without royalty.

\documentclass[11pt]{article}

% we have to change the font encoding for Lucida.
\usepackage[T1]{fontenc}
\usepackage{textcomp} % to get the right copyright, etc.

% use Lucida fonts for both text and math.
\usepackage[altbullet]{lucidabr}     % get larger bullet
\DeclareEncodingSubset{TS1}{hlh}{1}  % including \oldstylenums

% other features we'll use.
\usepackage{framed}
\reversemarginpar
\addtolength\marginparwidth{20pt}

% live url's if pdf.
\usepackage{ifpdf}
\ifpdf
  \usepackage[breaklinks,colorlinks,linkcolor=black,citecolor=black,
              pagecolor=black,urlcolor=black]{hyperref}
\else
  \usepackage{url}
\fi

\newcommand*\pkg[1]{\textsf{#1}}
\newcommand*\opt[1]{\texttt{#1}}
\newcommand*\cs[1]{\texttt{\char`\\#1}}

\pagestyle{headings}

\newcommand\demotext{%
  For \textsterling 45, almost anything can
  be found floating in fields. 
%  !`THE DAZED BROWN FOX QUICKLY GAVE 12345--67890 JUMPS!
  --- ?`But aren't Kafka's Schlo\ss{} and
  \AE sop's \OE uvres often na\"\i ve vis-\`a-vis the d\ae monic
  ph\oe nix's official r\^ole in fluffy souffl\'es?}

\newcommand*\demotextsc{\textsc{Sphinx of black quartz, judge my vow}.}

\newcommand*\demotextosf{\oldstylenums{0123456789}.}

\newcommand*\raggedmarginpar[1]{\marginpar{\raggedright\hspace{0pt}#1}}

\newcommand*\demo[2]{%
   \par\leavevmode\raggedmarginpar{#1}%
   \begin{minipage}[t]{\linewidth}%
   \normalfont#2\demotext
   \end{minipage}%
}

\newcommand*\demosc[2]{%
  \par\leavevmode\raggedmarginpar{#1}{\normalfont#2\demotext
  \newline\demotextsc\par}%
}
\newcommand*\demoscosf[2]{%
  \par\leavevmode\raggedmarginpar{#1}{\normalfont#2\demotext
  \newline\demotextsc\space\demotextosf\par}%
}
\newcommand*\demoosf[2]{%
  \par\leavevmode\raggedmarginpar{#1}{\normalfont#2\demotext
  \newline\space\demotextosf\par}%
}


\title{Using the Lucida fonts with \LaTeX}
\author{\TeX\ Users Group\\[2pt]\url{http://tug.org/lucida}}
\begin{document}
\maketitle


\section{Introduction}

{\def\thefootnote{}
% article.cls uses 1.8em for the footnote indent.
\footnotetext{\kern-1.8em \textregistered\ Lucida is a trademark of
Bigelow \& Holmes Inc.\ registered in the U.S. Patent \& Trademark
Office and other jurisdictions.}
}

This document contains examples of the Lucida fonts available through
TUG.  They are divided into two sets, \textit{basic} and
\textit{complete}, as displayed in the following sections.

For more information about Lucida and \TeX, and an order form for the
fonts, please see \url{http://tug.org/lucida}.


\section{\LaTeX\ macro support for Lucida}

The Lucida support primarily consists of two packages: \pkg{lucidabr}
and \pkg{lucbmath}. The former changes both running text and math to use
Lucida, whereas the latter only changes the math font setup in case you
want to use a different text font with the Lucida math fonts.

You may already have the macro packages installed as they are part of
most \TeX{} distributions---try running the example below.

If it complains that \texttt{lucidabr.sty} is not found, you must
install the package; it's available on CTAN in
\url{http://www.ctan.org/tex-archive/macros/latex/contrib/psnfssx/lucidabr},
and (of course) also included in the TUG distribution when you order the
fonts.

\subsection{Basic example}

The packages do \emph{not} support \LaTeX's (and \TeX's) default
encoding (OT1).  Supported encodings are T1, LY1, and TS1 (partial).
What this means is that you have to use the \pkg{fontenc} package to
switch the default.

Here's a small example:

\begin{verbatim}
\documentclass{article}
\usepackage[T1]{fontenc}
\usepackage{textcomp}
\usepackage{lucidabr}
\begin{document}
Here's some text. And here's some math:
\[
  \phi(x)=\int_{-\infty}^{x} e^{-x^{2}/2}
\]
Euro and copyright symbols are available:
\texteuro \textcopyright \textbullet.
\end{document}
\end{verbatim}
This results in the following output:
\begin{framed}
Here's some text. And here's some math:
\[
  \phi(x)=\int_{-\infty}^{x} e^{-x^{2}/2}
\]
Euro, copyright, and bullet symbols are available:
\texteuro \textcopyright \textbullet.
\end{framed}

\subsection{More details}

If the example runs ok, but produces no output, try refreshing the
``filename database'' (e.g., run \texttt{mktexlsr}).  Also, of course,
you must actually purchase the fonts!  (All the metrics and support
files are on CTAN, but not the \texttt{.pfb} files containing the actual
outlines.)

It's best to load the \pkg{textcomp} package with \pkg{lucidabr}, or
some symbols, notably \cs{textcopyright}, will be synthesized instead of
coming from the fonts.  We don't load \pkg{textcomp} automatically,
since loading such fundamental packages behind the scenes can cause
hard-to-debug trouble.

Furthermore, the default \cs{textbullet} is quite small; the more normal
one above is generated by specifying the \opt{altbullet} option when
loading \pkg{lucidabr}.

By default, oldstyle figures from the \pkg{textcomp} package, accessed
with the \verb|\oldstylenums| command, are disabled for the Lucida fonts
since they do not exist in all shapes. In order for \verb|\oldstylenums|
to work, you must add the line
\begin{verbatim}
\DeclareEncodingSubset{TS1}{hlh}{1}
\end{verbatim}
to your preamble \emph{after} loading the \pkg{textcomp} package. The
font family \texttt{hlhj} provides the oldstyle figures by default but
there exists no bold italic version of these figures and the italic
versions are only available if you buy the complete font set.

Now, let's take a more systematic look at the fonts.


\section{The \textsf{basic} font set}

The basic set of fonts contains all the math fonts (shown in the
accompanying \texttt{lucida-amsmath} document), a set of text fonts with
accompanying small caps, and a monospaced font for code examples.  The
idea is that this is sufficient for mathematical papers and typical text
usage.

The roman text font is Lucida Bright. It comes with small caps and
oldstyle figures only in the upright shapes:

\begin{quote}
\demoscosf{LucidaBright}{}

\demo{LucidaBright-Italic}{\itshape}

\demo{LucidaBright-Oblique}{\slshape}

\demoscosf{LucidaBright-Demi}{\bfseries}

\demo{LucidaBright-DemiItalic}{\bfseries\itshape}
\end{quote}

\noindent The basic set also contains Lucida Sans Typewriter in various
shapes and series:
\begin{quote}
\demo{LucidaSans-Typewriter}{\ttfamily\raggedright}

\demo{LucidaSans-Typewriter Oblique}{\ttfamily\slshape\raggedright}

\demo{LucidaSans-Typewriter Bold}{\ttfamily\bfseries\raggedright}

\demo{LucidaSans-Typewriter BoldOblique}
     {\ttfamily\bfseries\slshape\raggedright}
\end{quote}

\section{The \textsf{complete} font set}

The complete font set includes (naturally) all the basic fonts, and
assorted other text font variations, starting with the full sans serif
variant, Lucida Sans:
\begin{quote}
\demo{LucidaSans}{\sffamily}

\demo{LucidaSans-Italic}{\sffamily\itshape}

\demo{LucidaSans-Demi}{\sffamily\bfseries}

\demo{LucidaSans-DemiItalic}
     {\sffamily\bfseries\itshape}
\end{quote}

LucidaSans also exists in an ultra bold version, which you have to select
manually with \verb|\fontseries{ub}\selectfont|.
\begin{quote}
\demo{LucidaSans-Bold}{\sffamily\fontseries{ub}\selectfont}

\demo{LucidaSans-BoldItalic}
     {\sffamily\itshape\fontseries{ub}\selectfont}
\end{quote}


A second, seriffed, typewriter font is included as well. By default the
\pkg{lucidabr} package chooses Lucida Sans Typewriter for typewriter but
you can change that by giving the option \opt{seriftt}, as in:\\
\verb|\usepackage[seriftt]{lucidabr}|

\begin{quote}
\demo{Lucida Typewriter}{\fontfamily{hlct}\selectfont\raggedright}

\demo{Lucida Typewriter Oblique}
     {\fontfamily{hlct}\selectfont\slshape\raggedright}

\demo{Lucida Typewriter Bold}{\fontfamily{hlct}\selectfont\bfseries\raggedright}

\demo{Lucida Typewriter BoldOblique}
     {\fontfamily{hlct}\selectfont\slshape\bfseries\raggedright}
\end{quote}

Lucida Fax is a complete text font. By giving the option \opt{fax} to
\pkg{lucidabr} this becomes the default roman font. There are no small
caps or oldstyle figures for this font.
\begin{quote}
\demo{LucidaFax}{\fontfamily{hlx}\selectfont}

\demo{LucidaFax-Italic}
     {\fontfamily{hlx}\selectfont\itshape}

\demo{LucidaFax-Bold}{\fontfamily{hlx}\selectfont\bfseries}

\demo{LucidaFax-BoldItalic}
     {\fontfamily{hlx}\selectfont\itshape\bfseries}
\end{quote}

Lucida Casual exists in two versions only: medium upright and
medium italic. You can still make it the default text font by giving the
option \opt{casual} to \pkg{lucidabr}.
\begin{quote}
\demo{LucidaCasual}{\fontfamily{hlcn}\selectfont}

\demo{LucidaCasual-Italic}
     {\fontfamily{hlcn}\selectfont\itshape}
\end{quote}


Finally, three more esoteric fonts are included. Lucida Calligraphy
contains italic oldstyle figures, which are used if available when
selecting \verb|\oldstylenums| for Lucida Bright. If you choose the
option \opt{calligraphic} for \pkg{lucidabr}, Lucida Calligraphy will be
the default roman font. A similar option \opt{handwriting} makes Lucida
Handwriting the default roman font.

\begin{quote}
\demo{LucidaCalligraphy-Italic}
     {\fontfamily{hlce}\selectfont}

\demo{LucidaHandwriting-Italic}
     {\fontfamily{hlcw}\selectfont}

\demo{Lucida Blackletter}
     {\fontfamily{hlcf}\selectfont}
\end{quote}

This is all the Lucida fonts available with \TeX\ support.  For more
information about Lucida and \TeX, and an order form for the fonts,
please see \url{http://tug.org/lucida}, and thanks.

\end{document}
