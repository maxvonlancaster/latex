\title{\texttt{PST-Calendar} - plotting calendars\\
\footnotesize{version \PSTfileversion}}
\author{%
Manuel Luque\thanks{\url{Mluque5130@aol.com}} and
Herbert Vo\ss\thanks{\url{voss@pstricks.de}}}
\date{\today}
\maketitle

\begin{abstract}
\texttt{pst-calendar} stellt zwei Makros f�r das Erstellen von Kalenderbl�ttern
oder Kalender-Dodekaedern. Der Monat und das Jahr k�nnen frei gew�hlt werden,
wobei Kalender nur zwischen 2000 und 2099 m�glich sind. Die Berechnung
erfolgt ausschlie�lich mit den Makros des \LPack{fp}-Pakets.
\end{abstract}

%\tableofcontents


\section{\LPack{pst-calendar} -- Verschiedene Kalenderoperationen}

Das Paket hat die drei Optionen \Loption{french}, \Loption{english} und \Loption{ngerman}.
Standardm��ig wird \verb+english+ ausgew�hlt, womit die Monats- und WOchennamen dann
in eben dieser Sprache ausgegeben werden. F�r die Dokumentation wurde \verb+ngerman+
ausgew�hlt. 


Dieses Paket bietet keine Kalenderoperationen im mathematischen Sinne, sondern erlaubt
die Ausgabe von Kalenderbl�ttern. \LPack{pst-calendar} verf�gt �ber speziellen Sprachanpassungen
f�r Englisch (Standard),Franz�sisch und Deutsch. Dies bezieht sich auch auf die Optionen, die
in Tabelle~\ref{tab:calendrier:options} zusammengestellt sind.
Es existiert keine spezielle
\TeX-Version, sodass hier spezielle Anpassungen n�tig werden, wenn nicht mit \LaTeX\ 
gearbeitet wird.

\Lmsyn{psCalendar}\OptArg\\
\Lmsyn{psCalDodecaeder}\OptArg

Ohne Angabe von Daten wird immer vom aktuellen Tagesdatum ausgegangen und der entsprechende Monat
f�r \Lmcs{psCalendar} und die entsprechenden Monate f�r \Lmcs{psCalDodecaeder} ausgegeben.
\begin{table}[htb]
\caption{Zusammenstellung der verf�gbaren Optionen f�r \texttt{calendrierfp}}\label{tab:calendrier:options}
\begin{tabularx}{\linewidth}{@{}ll@{}c>{\RaggedRight}X@{}}
\emph{Name}           & \emph{Wert}   & \emph{Vorgabe}    & \emph{Bedeutung}\\\hline
\Loption{Jahr} & \Larga{Integer} & \verb+\number\year+ & betrachteter Jahr\\
\Loption{Monat} & \Larga{$1\ldots 12$} & \verb+\number\month+ & betrachteter Monat\\
\Loption{MonatT} & \Larga{$1\ldots 12$} & \verb+\number\month+ & Markierung des aktuellen Tages im angegebenen Monat\\
\Loption{Tag} & \Larga{$1\ldots 31$} & \verb+\number\day+ & betrachteter Monat\\
\Loption{style} & \Larga{Monat} & \verb+\number\month+ & vorne liegender Monat beim Dodekaeder
\end{tabularx}
\end{table}


\bgroup
\begin{LTXexample}[pos=t]
\psscalebox{0.5}{\psCalendar}
\psscalebox{0.5}{\psCalendar[Jahr=2006,Monat=1]}
\psscalebox{0.5}{\psCalendar[Jahr=2006,Tag=23,Monat=2,MonatT=2]}
\end{LTXexample}
\egroup

Das einzige Problem sind die Feiertage, die bislang noch ausschlie�lich auf den franz�sichen
Vorgaben aufbauen. Die entsprechenden Kode-Sequenzen in dem Paket sind jedoch leicht zu
erkennen und entsprechend zu �ndern.  Die Ausgabe eines kompletten Jahres l�sst sich
mithilfe des \Lmcs{multido}-Makros leicht realisieren.


\begin{center}
\bgroup
\multido{\iM=1+3}{4}{%
  \multido{\iMM=\iM+1}{3}{\psscalebox{0.5}{\psCalendar[Jahr=2006,Monat=\iMM]}}\\}
\egroup
\end{center}

\begin{lstlisting}
\multido{\iM=1+3}{4}{%
  \multido{\iMM=\iM+1}{3}{\psscalebox{0.5}{\psCalendar[Jahr=2006,Monat=\iMM]}}\\}
\end{lstlisting}

Eine weitaus ansprechendere Ausgabe erm�glicht \Lmcs{psCalDodecaeder}, welcher die
Monate auf den Seiten eines Dodekaeders\index{Dodekaeder} anordnet. Ohne Angabe irgendeines
Monates beziehungsweise Jahres wird grunds�tzlich vom Januar des aktuellen Jahres
ausgegangen. Mit der Option \Loption{style}\verb+=+\Larga{Monat} kann man jederzeit
einen anderen Monat auf die Vorderseite platzieren.

\bgroup
\begin{LTXexample}[pos=t]
\psscalebox{0.2}{\psCalDodecaeder}\hfill
\psscalebox{0.2}{\psCalDodecaeder[Jahr=2006,Monat=1]}\hfill
\psscalebox{0.2}{\psCalDodecaeder[Jahr=2006,style=April]}
\end{LTXexample}
\egroup

\begin{center}
\bgroup
\psscalebox{0.2}{\psCalDodecaeder[Jahr=2006,style=Januar]}\hfill
\psscalebox{0.2}{\psCalDodecaeder[Jahr=2006,style=Februar]}\hfill
\psscalebox{0.2}{\psCalDodecaeder[Jahr=2006,style=Maerz]}\\
\psscalebox{0.2}{\psCalDodecaeder[Jahr=2006,style=April]}\hfill
\psscalebox{0.2}{\psCalDodecaeder[Jahr=2006,style=Mai]}\hfill
\psscalebox{0.2}{\psCalDodecaeder[Jahr=2006,style=Juni]}\\
\psscalebox{0.2}{\psCalDodecaeder[Jahr=2006,style=Juli]}\hfill
\psscalebox{0.2}{\psCalDodecaeder[Jahr=2006,style=August]}\hfill
\psscalebox{0.2}{\psCalDodecaeder[Jahr=2006,style=September]}\\
\psscalebox{0.2}{\psCalDodecaeder[Jahr=2006,style=Oktober]}\hfill
\psscalebox{0.2}{\psCalDodecaeder[Jahr=2006,style=November]}\hfill
\psscalebox{0.2}{\psCalDodecaeder[Jahr=2006,style=Dezember]}
\egroup
\end{center}

\begin{lstlisting}
\psscalebox{0.2}{\psCalDodecaeder[Jahr=2006,style=Januar]}\hfill
\psscalebox{0.2}{\psCalDodecaeder[Jahr=2006,style=Februar]}\hfill
\psscalebox{0.2}{\psCalDodecaeder[Jahr=2006,style=Maerz]}\\
\psscalebox{0.2}{\psCalDodecaeder[Jahr=2006,style=April]}\hfill
\psscalebox{0.2}{\psCalDodecaeder[Jahr=2006,style=Mai]}\hfill
\psscalebox{0.2}{\psCalDodecaeder[Jahr=2006,style=Juni]}\\
\psscalebox{0.2}{\psCalDodecaeder[Jahr=2006,style=Juli]}\hfill
\psscalebox{0.2}{\psCalDodecaeder[Jahr=2006,style=August]}\hfill
\psscalebox{0.2}{\psCalDodecaeder[Jahr=2006,style=September]}\\
\psscalebox{0.2}{\psCalDodecaeder[Jahr=2006,style=Oktober]}\hfill
\psscalebox{0.2}{\psCalDodecaeder[Jahr=2006,style=November]}\hfill
\psscalebox{0.2}{\psCalDodecaeder[Jahr=2006,style=Dezember]}
\end{lstlisting}

\iffalse
\nocite{*}
\bibliographystyle{plain}
\bibliography{pst-labo-doc}
\fi

\endinput