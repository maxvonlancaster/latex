
\documentclass[12pt]{article}
\usepackage{pstricks,pst-xkey,pst-jtree}

\rightskip=0pt plus 2em

\newdimen\dimpuba
\newdimen\dimpubb
\newdimen\dimpubc
\newdimen\dimpubd
\newcount\exno
\exno=0

\def\ex{%
   \vskip2.5em
   \allowbreak\noindent
   \global\advance\exno by 1
   \bgroup
   \parindent=0pt
   \rightskip=0pt minus 3em % allow overwide displays if necessary
   (\the\exno)\quad
}
\def\xe{\par\egroup}

\begin{document}
\psset{nodesepA=.8ex,nodesepB=.4ex}

\centerline{\huge Examples}

\vskip1.5em

\noindent The source file for this pdf file is {\it
pst-jtree-examples.tex}. It is very simply formatted (no packages
other than pstricks, pst-xkey, and pst-jtree are used) in order
to facilitate user experimentation with pst-jtree typesetting.
Provided that PSTricks and PST-XKey are installed, users should
be able to run this file with little difficulty if PST-jTree is
installed or simply if \hbox{\it pst-jtree.tex\/} and \hbox{\it
pst-jtree.sty\/} are in the same directory that \hbox{\it
pst-jtree-examples.tex\/} is in. \hbox{\it
pst-jtree-examples.pdf\/} shows what the source produces. For
those interested in Tex tree formatting, but not LaTex, \hbox{\it
pst-jtree-examples.tex\/} can be very simply converted into a
Tex file.

All of the complex examples in the documentation file {\it
pst-jtree-doc.pdf\/} are given here, as well as many of the
simpler ones. They appear in the same order that they appear in
the documentation. But the numbering is sequential and differs from
the numbering in the documentation.

\ex
\jtree[xunit=2.2em,yunit=1em]
\def\:{:[scaleby=2.3]}%
\! = \:!a :{is} {rotten}.
\!a = :{the} :{cheese} :{that } \:!b :{ate} {\it t}.
\!b = :{the} :{rat} :{that } :{John \ } :{killed} {\it t}.
\endjtree
\xe

\ex
\jtree
\! = {a}
   :{b} {c}
   :{d} {e}
   :{f} {g}
   :{h} {i}.
\endjtree
\xe

\ex
\jtree
\! = :{a} :{b} :{c} :{d} {e}.
\endjtree
\xe

\ex
\jtree
\defbranch<colonB>(1)(-.5)
\! = {a}
   :{b} {c}
   :{d} {e}
   :{f} {g}.
\endjtree
\xe

\ex
\jtree[xunit=2.5em,yunit=1em]
\defbranch<Left>(2.3)(1)
\! = <Left>!a ^<right> :{is} {rotten}.
\!a = :{the} :{cheese} :{that}
   <Left>!b ^<right> :{ate} {\it t}.
\!b = :{the} :{rat} :{that}
   <Left>!c ^<right> :{killed} {\it t}.
\!c = :{the} :{cat} :{that} :{John} :{owned} {\it t}.
\endjtree
\xe

\ex
\jtree[xunit=3em,yunit=3em]
\! = <left>[scaleby=2 1,
            arrows=->]{a}
   ^<left>[linewidth=2pt]{b}
   ^<right>[scaleby=1 2,
      linestyle=dashed]{c}
   ^<right>[scaleby=1.5 1,
      linestyle=dotted,linewidth=1pt]{d}
   ^<right>[scaleby=2 .5,doubleline=true]{e}.
\endjtree
\xe

\ex
\def\GO#1#2#3{$\rm GO\bigl({#1},[_{Path}\,
   FROM({#2}),TO({#3})]\bigr)$}%
\jtree[xunit=3em,yunit=1em]
\defstuff[a]{\multiline
   \it The cup slid from John to Mary\cr
   \GO{\bf cup}{\bf John}{\bf Mary}\cr
   IP\endmultiline}
\defstuff[b]{\multiline
   \it The cup\cr
   \bf cup\cr
   NP\endmultiline}
\defstuff[c]{\multiline
   \it slid from John to Mary\cr
   \GO{\mit x}{\bf John}{\bf Mary}\cr
   IP\endmultiline}
\defstuff[d]{\multiline
   \it The\cr
   $\perp$\cr
  \quad N$\rm _{SPEC}$\endmultiline}
\defstuff[e]{\multiline
   \it cup\cr
   \bf cup\cr
   N\endmultiline}
\def\Fracture{<vert>{\omit\psframebox[framesep=.4ex]
   {Fracture$\vphantom j$}}}%
\! = {\stuff[a]}
   \Fracture
   :\jtbig{\stuff[b]}!a \jtbig{\stuff[c]}
   \Fracture
   :{$\vdots$} {$\vdots$}.
\!a = \Fracture
   :{\stuff[d]} {\stuff[e]}.
\endjtree
\xe

\ex % Dowty
\jtree[xunit=4em,yunit=2em,everylabel=\strut\it]
\! = {John walks from Boston to Detroit\rm, t, 4}
   :{John\rm, T} {walk from Boston
                     to Detroit\rm, IV,7}[labeloffset=1.5em]
   :\jtlong{walk from Boston\rm, IV,7}!a
      {to Detroit\rm, IV/IV, 5}
   :[scaleby=.8]{to\rm, IAV/T}() [scaleby=.8]{Detroit\rm, T}.
\!a = :{walk\rm, IV} {from Boston\rm, IV/IV, 5}
   :{from\rm, IAV/T} {Boston\rm, T}.
\endjtree
\xe

\ex % Nunes
\jtree[xunit=3em,yunit=1.4em]
\def\what{[which claim]$_k$}%
\! = {CP$_1$}
   <left>[scaleby=1.7]{CP$_2$}!a ^<right>[scaleby=2.5]{C$'$}
   :{$\rm was+Q$} {TP}
   <vartri>[scaleby=1.6,triratio=.4]
      {he$_i$ willing to discuss \what}.
\!a = :{\what} {CP$_2$}
   <vartri>[scaleby=1.6,triratio=.6]
      {that John$_i$ made \what}.
\endjtree
\xe

\ex % Andrews
\jtree[xunit=2.2em,yunit=.7em]
\def\\{[labelgapb=-.5ex]}%
\! = {S}
   :({NP}\\{that book}@A1 ) \jtwide{S/NP}
   :({NP}\\{I}) {VP/NP}
   :({V}\\{want}) {VP/NP}
   :{\it to} {VP/NP}
   <left>({V}\\{ask}) ^<vert>({NP}\\{Mary}) ^<wideright>{VP/NP}
   :{\it to} {VP/NP}
   <left>({V}\\{tell}) ^<vert>({NP}\\{Tom}) ^<wideright>{VP/NP}
   :{\it to} {VP/NP}
   :({V}\\{read}) {NP/NP}\\{e}@A2 .
\nccurve[angleA=210,angleB=-90]{->}{A2}{A1}
\endjtree
\xe

\ex % Andrews (second version)
\jtree[xunit=2.2em,yunit=.7em]
\def\\{<shortvert>}%
\! = {S}
   :({NP}<vartri>{that book}@A1 ) \jtwide{S/NP}
   :({NP}\\{I}) {VP/NP}
   :({V}\\{want}) {VP/NP}
   :{\it to} {VP/NP}
   <left>({V}\\{ask}) ^<vert>({NP}\\{Mary}) ^<wideright>{VP/NP}
   :{\it to} {VP/NP}
   <left>({V}\\{tell}) ^<vert>({NP}\\{Tom}) ^<wideright>{VP/NP}
   :{\it to} {VP/NP}
   :({V}\\{read}) {NP/NP}\\{e}@A2 .
\nccurve[angleA=210,angleB=-90]{->}{A2}{A1}
\endjtree
\xe

\ex % Merchant
a.\quad
%
\jtree[xunit=2.6em,yunit=1em]
\def\\{[labelgapb=-1.2ex]}%
\everymath={\rm}%
\! = {CP}
   :({XP}\\{$\scriptstyle [wh]$}@A1 ) {$C'$}
   :({$C^0$}\\{$\scriptstyle [wh,Q]$}) {$\langle\, TP\,\rangle$}
   <tri>{\dots\quad\rnode[b]{A2}{\it t}\quad\dots}.
\nccurve[angleA=-150,angleB=-90,ncurv=1]{->}{A2}{A1}
\endjtree
%
\hfil
b.\quad
\jtree[xunit=2.6em,yunit=1em]
\def\\{[labelgapb=-1.2ex]}%
\everymath={\rm}%
\! = {CP}
   :({XP}\\{$\scriptstyle [wh]$}@A1 )  {$C'$}
   :({$C^0$}\\{$\scriptstyle [wh,Q]$}) {$\langle\, TP\,\rangle$}
   <tri>{\dots\quad\rnode[b]{A2}{\it t}\quad \dots}.
   \ncbar[angleA=-90,angleB=-90,armA=1em,
   armB=1em,linearc=.6ex]{->}{A2}{A1}
\endjtree
\xe

\ex % Chung, first example
a.\quad
%
\jtree[xunit=2.6em,yunit=1em]
\def\\{[labelgapb=-1.2ex]}%
\everymath={\rm}%
\! = {CP}
   :({XP}\\{$\scriptstyle [wh]$}@A1 ) {$C'$}
   :({$C^0$}\\{$\scriptstyle [wh,Q]$}) {$\langle\, TP\,\rangle$}
   <tri>{\dots\quad\rnode[b]{A2}{\it t}\quad\dots}.
\nccurve[angleA=-150,angleB=-90,ncurv=1]{->}{A2}{A1}
\endjtree
%
\hfil
b.\quad
\jtree[xunit=2.6em,yunit=1em]
\def\\{[labelgapb=-1.2ex]}%
\everymath={\rm}%
\! = {CP}
   :({XP}\\{$\scriptstyle [wh]$}@A1 )  {$C'$}
   :({$C^0$}\\{$\scriptstyle [wh,Q]$}) {$\langle\, TP\,\rangle$}
   <tri>{\dots\quad\rnode[b]{A2}{\it t}\quad \dots}.
   \ncbar[angleA=-90,angleB=-90,armA=1em,
   armB=1em,linearc=.6ex]{->}{A2}{A1}
\endjtree
\xe

\ex % Chung, second example
\jtree[xunit=2em,yunit=1.4em,labelgapb=0,triratio=0,
   arrowscale=1.6 1]
\deftriangle<tri>(1.8)(1)(-.5)
\defbranch<colonB>(1)(-.5)
\! = {CP}
   :{\sc WH}@A {C$'$}
   <tri>{\rlap{V}}@AA ^<tri>[triratio=.55]{CP}
   :{\it t}@B {C$'$}
   <tri>{\rlap{V}}@BB ^<tri>[triratio=.55]{CP}
   :{\it t}@C {C$'$}
   <tri>{\rlap{V}}@CC .
\psset{linewidth=1pt,ncurvB=1.1,nodesepA=0,nodesepB=.7ex,
   angleA=-90,angleB=180,offsetA=.5ex}
\nccurve{A}{AA}
\nccurve{B}{BB}
\nccurve{C}{CC}
\psset{offsetA=-.5ex,arrows=->}
\nccurve{A}{AA}
\nccurve{B}{BB}
\nccurve{C}{CC}
\endjtree
\xe

\ex % Caponigo and Schutze
\jtree[xunit=5em,yunit=2em]
\! = {IP}
   <tri>{\triline{sono stati\hfil}}  ^<tri>[triratio=.95]{FP}
   :{F$_{\rlap{$\scriptstyle\rm [+strong]$}}$}!a
      {Voice$_{\rlap{$\scriptstyle\rm Pass$}}$}
   :{Voice\rlap{$_{\rm Pass}$}}@A2  {$\rm Agr_OP$}
   :{DP}!b  {${\rm Agr_O}'$}
   :[scaleby=.8 1]{$\rm Agr_O$}@A3  [scaleby=.8 1]{VP}
   <tri>[scaleby=.4 .7]
      {\rnode{A5}{$t_i$}\hskip1ex \rnode{A6}{$t_m$}}.
\!a = <shortvert>{arrestati$_i$}@A1 .
\!b = <shortvert>{alcuni uomini$_m$}@A4 .
\psset{arrows=->}
\nccurve[angleA=225,angleB=-45]{A2}{A1}
\nccurve[angleA=200,angleB=-90,ncurv=1.5]{A3}{A2}
\nccurve[angleA=-130,angleB=-70]{A5}{A3}
\nccurve[angleA=-130,angleB=-70,linestyle=dashed]{A6}{A4}
\endjtree
\xe

\ex % Uriagereka
\jtree[xunit=1.8em,yunit=.9em]
\def\*{\xleaders\hbox{\kern1pt.\kern1pt}\hfil}%
\deftriangle<triA>(1.3)(1)(-1/2)
\! = {F$'$}
   <wideleft>{F}!a
      ^<wideright>{\hbox to 2em{\*}}{$\rm Agr_O P$}
   <left>{DP\rlap{\rnode{B2}{$\;\equiv\;$}\it me}}@B1
      ^<wideright>{$\rm Agr_O P$}
   <left>  ^<wideright>{$\rm {Agr_O}'$}
   <left>{$\rm Agr_O$}!b  ^<wideright>{AsP}
   <triA>{\triline
      {\* \rnode[b]{D1}{t} \* DP \* \rnode[b]{D2}{t} \*}}
   <tri>[scaleby=.8 1.3]{\triline{\* \rnode[b]{E1}{\it a} \*}}.
\!a = :{F}!a1 @A1 .
\!a1 = :{F}!a2 @A2 .
\!a2 = :{V}@A3 {F}.
\psset{scaleby=.5 1}
\!b = :{V}@C1 {$\emptyset$}.
\psset{arrows=->}
\ncarc[arcangle=50]{C1}{A3}
\ncarc[arcangle=50]{B1}{A2:t}
\ncarc[arcangle=50]{D1}{C1}
\nccurve[angleA=-75,angleB=-90,ncurv=1.2,nodesepB=1ex]{D2}{B2}
\nccurve[angleA=-145,angleB=-100,ncurv=1]{E1}{A1:t}
\endjtree
\xe

\ex % Richards
\jtree[xunit=1.5em,yunit=1.1em,labelgap=0]
\def\A#1{\pnode(0,.3){A#1}}%
\def\B#1{\pnode(0,-.3){B#1}}%
\! = :{car} {\omit\A1}
   :({\omit\A0}{Op})
   :{that}
   :{I}
   :{T}
   :{know} {\omit\B1}
   :({\omit\B0}{who})
   :{C}
   :{Pro}
   :{to}() \jtjot
   :{persuade}() \jtjot
   :{V} {\omit\A2}
   :\jtlong{\omit\A4}!a
   :{Pro}
   :{to}
   :{talk} {\omit\B2}
   :\jtlong{\omit\B3}!b
   :{V}
   :{about}() {\omit\A3}{\it t}.
\!a = :{owners}() \jtjot
   :{of} {\omit\A5}{\it t}.
\!b = :{to} {\omit\B4}{\it t}.
\psset{linestyle=dashed,linewidth=.3ex,linecolor=blue,nodesep=0}
\def\fudge{.5}%
\ncline[nodesepA=-\fudge]{A0}{A1}
\ncline[nodesepB=-\fudge]{A1}{A3}
\ncline{A2}{A4}
\ncline[nodesepB=-\fudge]{A4}{A5}
\psset{linestyle=dotted,linewidth=.5ex,linecolor=red}
\ncline[nodesepA=\fudge]{B0}{B1}
\ncline{B1}{B2}
\ncline{B2}{B3}
\ncline[nodesepB=\fudge]{B3}{B4}
\endjtree
\xe

\ex % Zubizaretta
\jtree[xunit=2.5em,yunit=2em]
\def\ovalstuff{\vtop{\hbox{arg$^3$}%
   \hbox to 4em{(\thinspace \leaders\hrule\hfil\ NP)}}}%
\! = {V}@B1
   <left>{face}({arg$^1$}@B2 )
      ^<right>{$\rm [_V\,$leggere]}({arg$^2$,})
      ^<right>[scaleby=3 1,branch=\blank]
         {}{\ovalnode[framesep=1ex,boxsep=false]{K}{\ovalstuff}}.
\psset{arrows=->,nodesepA=0}
\nccurve[angleA=150,angleB=180,ncurv=1.2]{B2}{B1}
\nccurve[angleA=90,angleB=0,ncurv=.8]{K}{B1}
\endjtree
\xe

\begingroup % Koopman

\def\scaleA{[scaleby=1.6 1]}%
\def\scaleB{[scaleby=.6 1,doubleline=true,doublesep=.1ex]}%
\def\mkovalnode{\rput(-1ex,-.8)
   {\ovalnode[framesep=\psxunit]{K}{\hskip2em}}}%

\ex
\jtree[xunit=3em,yunit=1.5em]
\def\scaleA{[scaleby=1.6 1]}%
\def\scaleB{[scaleby=.6 1,doubleline=true,doublesep=.1ex]}%
\def\mkovalnode{\rput(-1ex,-.8)
   {\ovalnode[framesep=\psxunit]{K}{\hskip2em}}}%
\! = {$\rm Agr'$}
   :\scaleA{Agr}!a \scaleA{\bf VP}
   <vert>[linestyle=dashed,linewidth=1pt]{\bf V}
   :{${\rm [_V\,e]}_i$}@A1 {T}.
\!a = :{V$_i$}!b {Agr}
   <vert>{[e]$_j$}@A2 .
\!b = {\omit\mkovalnode}
   :\scaleB{${\rm [_{Agr}\,Agr]}_j$}@A3 \scaleB{$\rm [_T\,T]$}.
\psset{angleA=-90,angleB=-45,arrows=->}
\nccurve[nodesepB=0]{A1}{K}
\nccurve[ncurv=1.3]{A2}{A3}
\endjtree
\xe
\vskip2em

\ex
\jtree[xunit=4.8em,yunit=2em]
\def\mkovalnode{\rput(-1ex,-.8){\ovalnode[framesep=\psxunit]{K}
   {\hskip2em}}}%
\! = {$\rm Agr'$}
   :\scaleA{Agr}!a \scaleA{\bf VP}
   <vert>[linestyle=dashed,linewidth=1pt]{\bf V}
   :{${\rm [_V\,e]}_i$}@A1 {T}.
\!a = :{V$_i$}!b {Agr}
   <vert>{[e]$_j$}@A2 .
\!b = {\omit\mkovalnode}
   :\scaleB{${\rm [_{Agr}\,Agr]}_j$}@A3 \scaleB{$\rm [_T\,T]$}.
\psset{angleA=-90,angleB=-45,arrows=->}
\nccurve[nodesepB=0]{A1}{K}
\nccurve[ncurv=1.3]{A2}{A3}
\endjtree
\xe
\vskip5em

\ex  \small
\jtree[xunit=2.6em,yunit=1em]
\def\mkovalnode{\rput(-1ex,-.8){\ovalnode[framesep=\psxunit]{K}
   {\hskip2em}}}%
\! = {$\rm Agr'$}
   :\scaleA{Agr}!a \scaleA{\bf VP}
   <vert>[linestyle=dashed,linewidth=1pt]{\bf V}
   :{${\rm [_V\,e]}_i$}@A1 {T}.
\!a = :{V$_i$}!b {Agr}
   <vert>{[e]$_j$}@A2 .
\!b = {\omit\mkovalnode}
   :\scaleB{${\rm [_{Agr}\,Agr]}_j$}@A3 \scaleB{$\rm [_T\,T]$}.
\psset{angleA=-90,angleB=-45,arrows=->}
\nccurve[nodesepB=0]{A1}{K}
\nccurve[ncurv=1.3]{A2}{A3}
\endjtree
\xe
\vskip2em

\endgroup % Koopman

\ex % McCawley
\jtree[xunit=2.45em,yunit=1.4em,dirA=(1:-1),nodesep=0]
\def\\{[labelgapb=-4pt]}%
\def\V{$\rm \overline V$}%
\! = {S}
   <wideleft>{S}!a ^<vert>{and} ^<wideright>{S}
   :({NP}<shortvert>{Fred}) {\V}
   :({V}\\{knows}) {NP}
   <tri>{a man} ^<right>
   <right>[scaleby=3.5 1,branch=\blank]{NP}@A3 !b ^{S}
   :({NP}<shortvert>{who}) {\V}@A2
   <left>({V}\\{repairs}).
\!a = :({NP}<shortvert>{John}) {\V}@A1
   <left>{V}\\{sells}.
\!b = <vartri>{washing machines}.
\nccurve[angleB=150,ncurvB=1.4]{A2:b}{A3:t}
\nccurve[angleB=135,ncurvA=.5,ncurvB=2.6]{A1:b}{A3:t}
\endjtree
\xe

\vfil\break

\ex % Frampton, verb raising
\def\\#1\par{\bigskip #1\par\nobreak\vskip1.5em \hfil}
\def\Vj{V$\mskip-5mu _j$}
\psset{xunit=3.5em,yunit=1.5em,treevshift=1.3em}
\hfil Theories of verb raising
\medskip
\leftskip=2em

\\Trace theory

\jtree[scaleby=.8]
\! = :{Agr} :{Tns} :{V}() \etc.
\endjtree
\quad $\longrightarrow$\quad
\jtree
\! = :{Agr} !a
   :{$t_i$}
   :{$t_j$}() \etc.
\psset{scaleby=.3 .4}
\!a = :{Agr} {Tns$_i$}
   :{Tns} {\Vj}.
\endjtree

\\Copy theory

\jtree[scaleby=.8]
\! = :{Agr} :{Tns} :{V}() \etc.
\endjtree
\quad $\longrightarrow$\quad
\jtree
\! = :{Agr}!a [scaleby=1.45]
   :{Tns$_i$}!b
   :{\Vj}() \etc.
\psset{scaleby=.3 .4}
\!a = :{Agr} {Tns$_i$}
   :{Tns} {\Vj}.
\!b = :{Tns} {\Vj}.
\endjtree

\\Shared structure

\jtree[scaleby=.8]
\! = :{Agr} :{Tns} :{V}() \etc.
\endjtree
\quad $\longrightarrow$\quad
\jtree
\! = :{Agr}@A1 !a
   :{Tns}@A2 !b
   :{V}@A3 \etc.
\psset{scaleby=.3 .4,angleA=-35,angleB=125,ncurv=1.5,nodesep=0}
\!a = <left>{Agr}.
\!b = <left>{Tns}.
\nccurve{->}{A1:b}{A2:t}
\nccurve{->}{A2:b}{A3:t}
\endjtree
\xe

\ex % Frampton, Bulgarian
\kern2em
\jtree[xunit=2.4em,yunit=1.2em,style=arrows2,nodesep=0]
\def\broken{[branch=\brokenbranch,scaleby=1.6]}%
\def\stub{<right>[scaleby=.5,arrows=-]}%
\def\\#1{\rput[bl](.6ex,.4ex){\it #1}}%
\! = {\omit\\a}@A1
   \stub @K1  ^<right>{\omit\\b}
   :{C$_2$}()  \broken @A2
   \stub @K2  ^<right>@A3
   \stub @K3  ^<right>
   :{C$_1$}()  \broken
   :{ubil}  {\omit\\c}@A4
   :{kolko}  {\omit\\d}
   :{studenti}  {\omit\\e}@A5
   :{ot} :{koi} {strani}.
\psset{dirA=(1:1),angleB=90,ncurvA=.6,ncurvB=1}
\nccurve{K1}{A5}
\nccurve{-}{K2}{A5}
\nccurve{K3}{A4}
\psset{dirA=(-1:-1),dirB=(-1:-1),ncurv=4,arrows=-}
\nccurve{A1}{K1}
\nccurve{A2}{K2}
\nccurve{A3}{K3}
\endjtree
\xe

\ex % Wilder, right node raising
\jtree[dirA=(1:-1),nodesepA=0,nodesepB=.8ex,
   xunit=2.2em,yunit=1em,style=arrows2]
\! = :!a {\rnode{K1}{knew}}.
\!a = :!b {\rnode{O1}{owned}}.
\!b = :!c {\rnode{C1}{cats}}.
\!c =
   :\jtlong !d [scaleby=1.8]
   :{and}() [scaleby=2.4]
   :{he}() @K2
   <left>\jtjot !e .
\!d =
   :{she}() @K3
   <left>\jtjot !f .
\!e =
   :{a woman}[labeloffset=-1ex]
   :{who}() @O2
   <left>@C2
   <left>{four}.
\!f =
   :{a man}
   :{who}() @O3
   <left>@C3
   <left>{three}.
\psset{linestyle=dashed,arrows=<-}
\nccurve[angleB=-10,ncurvB=2,ncurvA=1.2]{O2}{O1}
\nccurve[angleB=-90,ncurvA=1.4]{O3}{O1}
\nccurve[angleB=-10,ncurvB=1.8,ncurvA=1.6]{K2}{K1}
\nccurve[angleB=-90,ncurvA=1.4]{K3}{K1}
\nccurve[angleB=-90,ncurvA=1.4]{C3}{C1}
\nccurve[angleB=-10,ncurvB=1.8,ncurvA=1.6]{C2}{C1}
\endjtree
\xe

\ex % Frampton, bilinks
\def\bilink(#1,#2)(#3,#4){%
   \pcarc(#1,#2)(#3,#4)%
   \pcarc[linestyle=dashed](#3,#4)(#1,#2)%
}%
\kern5em
\jtree[xunit=3em,yunit=1.8em,style=arrows2,
   dirA=(-1:-1),branch=\bilink,nodesep=3pt,
   arcangle=10,offset=1pt,labelgapt=!3pt]
\def\@{\pscircle(0,0){3pt}}%
\! =
   {\pnode{A1}\@}
   <right>{\omit\@\pnode(.8,-.8){A3}}
   :({\omit\@}{C}) [scaleby=1.6,arcangle=7]{\omit\@}
   :({\omit\@}{see})
      ({\omit\@\pnode{A2}}{who}[labeloffset=.8em]).
\nccurve[angleB=225,ncurvA=1.95,ncurvB=1,offset=1.6pt]{A1}{A2}
\nccurve[angleB=227,ncurvA=2,ncurvB=1.02,offset=-1.6pt,
   linestyle=dashed,arrows=<-]{A1}{A2}
\rput(1.8,-1.8){\pscircle*[linecolor=white]{1em}}%
\rput(1.8,-1.8){\dots}
\endjtree
\xe

\end{document}

