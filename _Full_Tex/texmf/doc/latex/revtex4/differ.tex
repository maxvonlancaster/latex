%% ****** Start of file authguide.tex ****** %
%%
%%   This file is part of the APS files in the REVTeX 4 distribution.
%%   Version 4.0 of REVTeX, August 2001
%%
%%   Copyright (c) 2000, 2001 The American Physical Society.
%%
%%   See the REVTeX 4 README file for restrictions and more information.
%%
\documentclass[%
%prl%
%,preprint%
,twocolumn%
,secnumarabic%
%,tightenlines%
,amssymb,aps,prl,nobibnotes]{revtex4}
\usepackage{docs}
%\usepackage{acrofont}%NOTE: Comment out this line for the release version!
%\usepackage[colorlinks=true,linkcolor=blue]{hyperref}%
%\nofiles
\expandafter\ifx\csname package@font\endcsname\relax\else
 \expandafter\expandafter
 \expandafter\usepackage
 \expandafter\expandafter
 \expandafter{\csname package@font\endcsname}%
\fi
\DeclareRobustCommand\substyle{\name@idx{document substyle}}%
\DeclareRobustCommand\classoption{\name@idx{document class option}}%
\DeclareRobustCommand\classname{\name@idx{document class}}%
\def\name@idx#1#2{%
 {\ttfamily#2}%
 \index{#2\space#1=\string\ttt{#2}\space#1}\index{#1>#2=\string\ttt{#2}}%
}%

\DeclareRobustCommand\revtex{REV\TeX}
\begin{document}
\title{Differences between \revtex~4 and \revtex~3}%
\author{American Physical Society}%
\email{revtex4@aps.org}
\affiliation{1 Research Road, Ridge, NY 11961}
\date{August 2001}%
\maketitle
\tableofcontents

\section{Introduction}
This document gives a brief summary of how \revtex~4 is different from
what authors may already be familiar with.  The two primary design
goals for \revtex~4 are to 1) move to \LaTeXe\ and 2) improve the
markup so that infomation can be more reliably extracted for the
editorial and production processes. Both of these goals require that
authors comfortable with earlier versions of \revtex\ change their
habits. In addition, authors may already be familiar with the standard
\classname{article.cls} in \LaTeXe. \revtex~4 differs in some
important ways from this class as well. For more complete
documentation on \revtex~4, see the main \textit{\revtex~4 Author's
Guide}.  The most important changes are in the markup of the front
matter (title, authors, affiliations, abstract, etc.). Please see
Sec.~\ref{sec:front}.

\section{Version of \LaTeX}
The most obvious difference between \revtex~4 and \revtex~3 is that
\revtex~4 works solely with \LaTeXe; it is not useable as a \LaTeX2.09 package.
Furthermore, \revtex~4 requires an up-to-date \LaTeX\ installation
(1996/06/01 or later); its use under older versions is not supported.

\section{Class Options and Defaults}
Many of the class options in \revtex~3 have been retained in
\revtex~4. However, the default behavior for these options can be
different than in \revtex~3. Currently, there is only one society
option, \classoption{aps}, and this is the default. Furthermore, the
selection of a journal (such as \classoption{prl}) will automatically
set the society as well (this will be true even after other societies
are added).

In \revtex~3, it was necessary to invoke the \classoption{floats}, but
this is the default for \classoption{aps} journal in
\revtex~4. \revtex~4 introduces two new class options,
\classoption{endfloats} and \classoption{endfloats*} for moving floats
to the end of the paper.

The preamble commands \cmd{\draft} and \cmd{\tighten} have been replaced
with new class options \classoption{draft} and
\classoption{tightenlines}, respectively. The \cmd{\preprint} command
is now used only for specifying institutional report numbers (typeset
in the upper-righthand corner of the first page); it no longer
influences whether PACS numbers are displayed below the abstract. PACS
display is controlled by the \classoption{showpacs} and
\classoption{noshowpacs} (default) class options.

Paper size options (\classoption{letter}, \classoption{a4paper}, etc.)
work in \revtex~4. The text ``Typeset by \revtex'' no longer appears
by default - the option \classoption{byrevtex} will place this text in
the lower-lefthand corner of the first page.

\section{One- and Two-column formatting}

\revtex~4 has excellent support for achieving the two-column
formatting in the \textit{Physical~Review} and \textit{Reviews of
Modern Physics} styles. It will balance the columns
automatically. Whereas \revtex~3 had the \cmd{\widetext} and
\cmd{\narrowtext} commands for switching between one- and two-cloumn
modes, \revtex~4 simply has a \env{widetext} environment,
\envb{widetext} \dots \enve{widetext}. One-column formatting can be
specified by choosing either the \classoption{onecolumn} or
\classoption{preprint} class option (the \revtex~3 option
\classoption{manuscript} no longer exists). Two-column formatting is
the default for most journal styles, but can be specified with the
\classoption{twocolumn} option. Note that the spacing for
\classoption{preprint} is now set to 1.5, rather than full
double-spacing. The \classoption{tightenlines} option can be used to
reduce this to single spacing.


\section{Front Matter Markup}
\label{sec:front}

\revtex~4 has substantially changed how the front matter for an article
is marked up. These are the most significant differences between
\revtex~4 and other systems for typesetting manuscripts. It is
essential that authors new to \revtex~4 be familiar with these changes.

\subsection{Authors, Affiliations, and Author Notes}
\revtex~4 has substantially changed the markup of author names,
affiliations, and author notes (footnotes giving additional
information about the author such as a permanent address or an email
address).
\begin{itemize}
\item Each author name should appear separately in
individual \cmd\author\ macros. 

\item Email addresses should be marked up using the \cmd\email\ macro.

\item Alternative affiliation information should be marked up using
the \cmd\altaffiliation\ macro.

\item URLs for author home pages can be specified with a
\cmd\homepage\ macro.

\item The \cmd\thanks\ macro should only be used if one of the above
don't apply.

\item \cmd{\email}, \cmd{\homepage}, \cmd{\altaffiliation}, and
\cmd{\thanks} commands are grouped together under a single footnote for
each author. These footnotes can either appear at the bottom of the
first page of the article or as the first entries in the
bibliography. The journal style controls this placement, but it may be
overridden by using the class options \classoption{bibnotes} and
\classoption{nobibnotes}. Note that these footnotes are treated
differently than the other footnotes in the article.

\item The grouping of authors by affiliations is accomplished
automatically. Each affiliation should be in its own
\cmd{\affiliation} command. Multiple \cmd{\affiliation},
\cmd{\email}, \cmd{\homepage}, \cmd{\altaffiliation}, and \cmd{\thanks}
commands can be applied to each author. The macro \cmd\and\ has been
eliminated.

\item \cmd\affiliation\ commmands apply to all previous authors that
don't have an affiliation already declared. Furthermore, for any
particular author, the \cmd\affilation\ must follow any \cmd{\email},
\cmd{\homepage}, \cmd{\altaffiliation}, or \cmd{\thanks} commands for
that author.

\item Footnote-style associations of authors with affilitations should
not be done via explicit superscripts; rather, the class option
\classoption{superscriptaddress} should be used to accomplish this
automatically.

\item A collaboration for a group of authors can be given using the
\cmd\collaboration\ command.

\end{itemize}

Table~\ref{tab:front} summarizes some common pitfalls in moving from
\revtex~3 to \revtex~4.
\begin{table*}
\begin{ruledtabular}
\begin{tabular}{lll}
\textbf{\revtex~3 Markup} & \textbf{\revtex~4 Markup} & \textbf{Explanation}\\
& & \\
\verb+\author{Author One and Author Two}+ & \verb+\author{Author One}+ & One name per\\
& \verb+\author{Author Two}+  & \verb+\author+ \\
& & \\
\verb+\author{Author One$^{1}$}+ & \verb+\author{Author One}+& Use \classoption{superscriptaddress}\\
\dots &\dots &  class option \\
\verb+\address{$^{1}$APS}+ &\verb+\affiliation{APS}+ & \\
& & \\
\verb+\thanks{Permanent address...}+ & \verb+\altaffiliation{}+& Use most
specific macro \\
\verb+\thanks{Electronic address: user@domain.edu}+ &
\verb+\email{user@domain.edu}+& available\\
\verb+\thanks{http://publish.aps.org/}+ &
\verb+\homepage{http://publish.aps.org/}+& \\
\end{tabular}
\end{ruledtabular}
\caption{Common mistakes in marking up the front matter}
\label{tab:front}
\end{table*}


\subsection{Abstracts}
\revtex~4, like \revtex~3, uses the \env{abstract} environment
\envb{abstract} \dots \enve{abstract} for the abstract. The
\env{abstract} environment must appear before the \cmd{\maketitle}
command in \revtex~4. The abstract will be formatted
appropriately for either one-column (preprint) or two-column
formatting. In particular, in the two-column case, the abstract will
automatically be placed in a single column that spans the width of the
page. It is unnecessary to use a \cmd{\minipage} or any other macro to
achieve this result.


\section{Citations and References}

\revtex~4 uses the same \cmd{\cite},\cmd{\ref}, and \cmd{\bibitem}
commmands as standard \LaTeX\ and \revtex~3. Citation handling is
based upon Patick Daly's \classname{natbib} package. The
\env{references} environment is no longer used. Instead, use the
standard \LaTeXe\ environment \env{thebibliography}.

Two new \BibTeX\ files have been included with \revtex~4,
\file{apsrev.bst} and \file{apsrmp.bst}. These will format references
in the style of \textit{Physical Review} and \textit{Reviews of Modern
Physics} respectively. In addition, these \BibTeX\ styles
automatically apply a special macro \cmd{\bibinfo} to each element of the
bibliography to make it easier to extract information for use in the
editorial and production processes. Authors are strongly urged to use
\BibTeX\ to manage their bibliographies so that the \cmd{\bibinfo}
directives will be automatically included.  Other bibliography styles
can be specified by using the \cmd\bibliographystyle\ command, but
unlike standard \LaTeXe, you must give this command \emph{before} the
\envb{document} statement.

Please note that the package \classname{cite.sty} is not needed with
\revtex~4 and is incompatible.

\section{Footnotes and Tablenotes}
\label{sec:foot}

\revtex~4 uses the standard \cmd{\footnote} macro for
footnotes. Footnotes can either appear on the bottom of the page on
which they occur or they can appear as entries at the end of the
bibliography. As with author notes, the journal style option controls
the placement; however, this can be overridden with the class options
\classoption{footinbib} and \classoption{nofootinbib}.

Within a table, the \cmd{\footnote} command behaves differently. Footnotes
appear at the bottom of the table. \cmd{\footnotemark} and
\cmd{\footnotetext} are also available within the table environment so
that multiple table entries can share the same footnote text.  There
is no longer a need to use a \cmd{\tablenote}, \cmd{\tablenotemark},
and \cmd{\tablenotetext} macros.

\section{Section Commands}

The title in a \cmd\section\marg{title} command will be automatically
uppercased in \revtex~4. To prevent a particular letter from being
uppercased, enclose it in curly braces.

\section{Figures}

Figures should be enclosed within either a \env{figure} or \env{figure*}
environment (the latter will cause the figure to span the full width
of the page in two-column mode). \LaTeXe\ has two convenient packages
for including the figure file itself: \classname{graphics} and
\classname{graphicx}. These two packages both define a macro
\cmd{\includegraphics} which calls in the figure. They differ in how
arguments for rotation, translation, and scaling are specified. The
package \classname{epsfig} has been re-implemented to use these
\classname{graphicx} package. The package \classname{epsfig} provides
an interface similar to that under the \revtex~3 \classoption{epsf}
class option. Authors should use these standard
\LaTeXe\ packages rather than some other alternative.

\section{Tables}

Short tables should be enclosed within either a \env{table} or \env{table*}
environmnent (the latter will cause the table to span the full width
of the page in two-column mode). The heart of the table is the
\env{tabular} environment. This will behave for the most part as in
standard \LaTeXe. Note that \revtex~4 no longer automatically adds
double (Scotch) rules around tables. Nor does the \env{tabular}
environment set various table parameters as before. Instead, a new
environment \env{ruledtabular} provides this functionality. This
environment should surround the \env{tabular} environment:
\begin{verbatim}
\begin{table}
\caption{...}
\label{tab:...}
\begin{ruledtabular}
\begin{tabular}
...
\end{tabular}
\end{ruledtabular}
\end{table}
\end{verbatim}

Under \revtex~3, tables automatically break across pages. \revtex~4
provides some of this functionality. However, this requires adding the
table a float placement option of [H] (meaning put the table
``here'') to the \envb{table} command.

Long tables are more robustly handled by using the
\classname{longtable.sty} package included with the standard \LaTeXe\
distribution (put \verb+\usepackage{longtable}+ in the preamble). This
package gives precise control over the layout of the table. \revtex~4
goes out of its way to provide patches so that the \env{longtable}
environment will work within a two-column format. A new
\env{longtable*} environment is also provided for long tables that are
too wide for a narrow column. (Note that the \env{table*} and
\env{longtable*} environments should always be used rather than
attempting to use the \env{widetext} environment.)

To create tables with columns of numbers aligned on decimal points,
load the standard \LaTeXe\ \classname{dcolumn} package and use the
\verb+d+ column specifier. The content of each cell in the column is
implicitly in math mode: Use of math delimiters (\verb+$+) is unnecessary
in a \verb+d+ column.

Footnotes within a table can be specified with the
\cmd{\footnote} command (see Sec.~\ref{sec:foot}). 

\section{Font selection}

The largest difference between \revtex~3 and \revtex~4 with respect to
fonts is that \revtex~4 allows one use the \LaTeXe\ font commands such
as \cmd{\textit}, \cmd{\texttt}, \cmd{\textbf} etc. These commands
should be used in place of the basic \TeX/\LaTeX\ 2.09 font commands
such as \cmd{\it}, \cmd{\tt}, \cmd{\bf}, etc. The new font commands
better handle subtleties such as italic correction and scaling in
super- and subscripts.

\section{Math and Symbols}

\revtex~4 depends more heavily on packages from the standard \LaTeXe\
distribution and AMS-\LaTeX\ than \revtex~3 did. Thus, \revtex~4 users
should make sure their \LaTeXe\ distributions are up to date and they
should install AMS-\LaTeX\ 2.0 as well. In general, if any fine control of
equation layout, special math symbols, or other specialized math
constructs are needed, users should look to the \classname{amsmath}
package (see the AMS-\LaTeX\ documentation).

\revtex~4  provides a small number of additional diacritics, symbols,
and bold parentheses. Table~\ref{tab:revsymb} summarizes this.

\begin{table}
\caption{Special \revtex~4 symbols, accents, and boldfaced parentheses 
defined in \file{revsymb.sty}}
\label{tab:revsymb}
\begin{ruledtabular}
\begin{tabular}{ll|ll}
\cmd\lambdabar & $\lambdabar$ &\cmd\openone & $\openone$\\
\cmd\altsuccsim & $\altsuccsim$ & \cmd\altprecsim & $\altprecsim$ \\
\cmd\alt & $\alt$ & \cmd\agt & $\agt$ \\
\cmd\tensor\ x & $\tensor x$ & \cmd\overstar\ x & $\overstar x$ \\
\cmd\loarrow\ x & $\loarrow x$ & \cmd\roarrow\ x & $\roarrow x$  \\
\cmd\biglb\ ( \cmd\bigrb ) & $\biglb( \bigrb)$ &
\cmd\Biglb\ ( \cmd\Bigrb )& $\Biglb( \Bigrb)$ \\
& & \\
\cmd\bigglb\ ( \cmd\biggrb ) & $\bigglb( \biggrb)$ &
\cmd\Bigglb\ ( \cmd\Biggrb\ ) & $\Bigglb( \Biggrb)$ \\
\end{tabular}
\end{ruledtabular}
\end{table}

Here is a partial list of the more notable changes between \revtex~3
and \revtex~4 math:
\begin{itemize}
\item Bold math characters should now be handle via the standard
\LaTeXe\ \classname{bm} package (use \cmd{\bm} instead of \cmd{\bbox}).
\cmd{\bm} will handle Greek letters and other symbols.

\item Use the class options \classoption{amsmath},
\classoption{amsfonts} and \classoption{amssymb} to get even more math
fonts and symbols. \cmd{\mathfrak} and \cmd{\mathbb} will, for instance, give
Fraktur and Blackboard Bold symbols.

\item Use the \classoption{fleqn} class option for making equation
flush left or right. \cmd{\FL} and \cmd{\FR} are no longer provided.

\item In place of \cmd{\eqnum}, load the \classname{amsmath} package
[\verb+\usepackage{amsmath}+] and use \cmd{\tag}.

\item In place of \cmd{\case}, use \cmd{\textstyle}\cmd{\frac}.

\item In place of the \env{mathletters} environment, load the
\classname{amsmath} package and use \env{subequations} environment.

\item In place of \cmd{\slantfrac}, use \cmd{\frac}.

\item The macros \cmd{\corresponds}, \cmd{\overdots}, and
\cmd{\overcirc} have been removed. See Table~\ref{tab:obsolete}.

\end{itemize}

\section{Obsolete \revtex~3.1 commands}

Table~\ref{tab:obsolete} summarizes more differences between \revtex~4
and \revtex~3, particularly which \revtex~3 commands are now obsolete.

\begin{table*}
\caption{Differences between \revtex~3.1 and \revtex~4
markup}\label{tab:diff31}
\label{tab:obsolete}
\begin{ruledtabular}
\begin{tabular}{lp{330pt}}
\textbf{\revtex~3.1 command}&\textbf{\revtex~4 replacement}
\lrstrut\\
\cmd\documentstyle\oarg{options}\aarg{\classname{revtex}}&\cmd\documentclass\oarg{options}\aarg{\classname{revtex4}}
\\
option  \classoption{manuscript}& \classoption{preprint}
\\
\cmd\tighten\ preamble command & \classoption{tightenlines} class option
\\
\cmd\draft\ preamble command   & \classoption{draft} class option
\\
\cmd\author                    & \cmd\author\marg{name} may appear
multiple times; each signifies a new author name.\\
                               & \cmd\collaboration\marg{name}:
Collaboration name (should appear after last \cmd\author)\\
                               & \cmd\homepage\marg{URL}: URL for preceding author\\
                               & \cmd\email\marg{email}: email
address for preceding author\\
                               & \cmd{\altaffiliation}: alternate
affiliation for preceding \cmd\author\\
\cmd\thanks                    & \cmd\thanks, but use only for
information not covered by \cmd{\email}, \cmd{\homepage}, or \cmd{\altaffilitiation}\\
\cmd\and                       & obsolete, remove this command\\
\cmd\address                   & \cmd\affiliation\marg{institution}\ gives the affiliation for the group of authors above\\
                               & \cmd\affiliation\oarg{note} lets you specify a footnote to this institution\\
                               & \cmd\noaffiliation\ signifies that the above authors have no affiliation\\

\cmd\preprint                  & \cmd\preprint\marg{number} can appear multiple times, and must precede \cmd\maketitle\\
\cmd\pacs                      & \cmd\pacs\ must precede \cmd\maketitle\\
\env{abstract} environment     & \env{abstract} environment must precede \cmd\maketitle\\
\cmd\wideabs                       & obsolete, remove this command\\
\cmd\maketitle                 & \cmd\maketitle\ must follow
\emph{all} front matter data commands\\
\cmd\narrowtext                & obsolete, remove this command\\
\cmd\mediumtext                & obsolete, remove this command\\
\cmd\widetext                  & obsolete, replace with \env{widetext} environment\\
\cmd\FL                        & obsolete, remove this command\\
\cmd\FR                        & obsolete, remove this command\\
\cmd\eqnum                     & replace with \cmd\tag, load \classname{amsmath}\\
\env{mathletters}              & replace with \env{subequations}, load
\classname{amsmath}\\
\env{tabular} environment      & No longer puts in doubled-rules. Enclose \env{tabular} in \env{ruledtabular} to get old behavior.\\
\env{quasitable} environment   & obsolete, \env{tabular} environment no longer
puts in rules\\
\env{references} environment   & replace with \env{thebibliography}\verb+{}+\\
\cmd\case                      & replace with \cmd\textstyle\cmd\frac\\
\cmd\slantfrac                 & replace with \cmd\frac\\
\cmd\tablenote                 & replace with \cmd\footnote\\
\cmd\tablenotemark             & replace with \cmd\footnotemark\\
\cmd\tablenotetext             & replace with \cmd\footnotetext\lrstrut\\
\cmd\overcirc                  & Use standard \LaTeXe\ \cmd\mathring\ \\
\cmd\overdots                  & Use \cmd\dddot\ with \classoption{amsmath}\\
\cmd\corresponds               & Use \cmd\triangleq\ with \classoption{amssymb}\\
\classoption{epsf} class option & \verb+\usepackage{epsfig}+\\
\end{tabular}
\end{ruledtabular}
\end{table*}


\section{Converting a \revtex~3.1 Document to \revtex~4}\label{sec:conv31}%

\revtex~3 documents can be converted to \revtex~4 rather
straightforwardly. The following checklist covers most of the major
steps involved.

\begin{itemize}
\item Change \cmd\documentstyle\verb+{revtex}+ to
\cmd\documentclass\verb+{revtex4}+, and run the document under
\LaTeXe\ instead of \LaTeX2.09.

\item
Replace the \cmd\draft\ command with the \classoption{draft} class option.

\item
Replace the \cmd\tighten\ command with the \classoption{tightenlines}
class option.

\item
For each \cmd\author\ command, split the multiple authors into
individual \cmd\author\ commands. Remove any instances of \cmd\and.

\item For superscript-style associations between authors and
affiliations, remove explicit superscripts and use the
\classoption{superscriptaddress} class option.

\item
Use \cmd\affiliation\ instead of \cmd\address.

\item
Put \cmd\maketitle\ after the \env{abstract} environment and any
\cmd\pacs\ commands.

\item If double-ruled table borders are desired, enclose \env{tabular}
enviroments in \env{ruledtabular} environments.

\item
Convert long tables to \env{longtable}, and load the
\classname{longtable} package. Alternatively, give the \env{table}
an [H] float placement parameter so that the table will break automatically.

\item
Replace any instances of the \cmd\widetext\ and \cmd\narrowtext\
commands with the \env{widetext} environment.
Usually, the \envb{widetext} statement will replace the \cmd\widetext\
command, and the \enve{widetext} statement replaces the matching
\cmd\narrowtext\ command.

Note in this connection that due to a curious feature of \LaTeX\
itself, \revtex~4 having a \env{widetext} environment means that it
also has a definition for the \cmd\widetext\ command, even though the
latter cammand is not intended to be used in your document.
Therefore, it is particularly important to remove
all \cmd\widetext\ commands when converting to \revtex~4.

\item
Remove all obsolete commands: \cmd\FL, \cmd\FR, \cmd\narrowtext, and
\cmd\mediumtext\ (see Table~\ref{tab:diff31}).

\item
Replace \cmd\case\ with \cmd\frac. If a fraction needs to be set
in text style despite being in a display equation, use the
construction \cmd\textstyle\cmd\frac.  Note that \cmd\frac\ does not
support the syntax \cmd\case\verb+1/2+.

\item
Replace \cmd\slantfrac\ with \cmd\frac.

\item
Change \cmd\frak\ to \cmd\mathfrak\marg{char}\index{Fraktur} and
\cmd\Bbb\ to \cmd\mathbb\marg{char}\index{Blackboard Bold}, and invoke
one of the class options \classoption{amsfonts} or
\classoption{amssymb}.

\item
Replace environment \env{mathletters} with environment
\env{subequations} and load the \classname{amsmath} package.

\item
Replace \cmd\eqnum\ with \cmd\tag\ and load the \classname{amsmath} package.

\item
Replace \cmd\bbox\ with \cmd\bm\ and load the \classname{bm} package.

\item
If using the \cmd\text\ command, load the \classname{amsmath} package.

\item
If using the \verb+d+ column specifier in \env{tabular} environments,
load the \classname{dcolumn} package. Under \classname{dcolumn}, the
content of each \verb+d+ column cell is implicitly in math mode:
remove any \verb+$+ math delimiters appearing in cells in a \verb+d+
column.

\item
Replace \cmd\tablenote\ with \cmd\footnote, \cmd\tablenotemark\ with
\cmd\footnotemark, and \cmd\tablenotetext\ with \cmd\footnotetext.

\item
Replace \envb{references} with \envb{thebibliography}\verb+{}+;
\enve{references} with \enve{thebibliography}.
\end{itemize}
\end{document}
