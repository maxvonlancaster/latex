%
% http://www.adobe.com/devnet/acrobat/pdfs/PDF32000_2008.pdf
%
% makeindex < aebpro_man.idx > aebpro_man.ind
\documentclass{article}
\usepackage[fleqn]{amsmath}
\usepackage[
    web={centertitlepage,designv,
        forcolorpaper,latextoc,pro},%usesf,
        aebxmp,eforms
]{aeb_pro}
\usepackage[dvipsone,showembeds]{graphicxsp}
\usepackage{aeb_mlink}
\usepackage{array}
%\usepackage{myriadpro}
\usepackage[altbullet]{lucidbry}
\usepackage{rmannot}

\usepackage{makeidx}
\makeindex
\usepackage{acroman}

\usepackage[active]{srcltx}

\def\expath{../examples}

\urlstyle{rm}

%\def\tutpath{doc/tutorial}
%\def\tutpathi{tutorial}

\DeclareDocInfo
{
    university={\AcroTeX.Net},
    title={ The \texorpdfstring{\texttt{rmannot} Package\\[1em]}{rmannot Package: }
        Rich Media Annotations\texorpdfstring{\\[1em]}{ }for Acrobat 9 Pro},
    author={D. P. Story},
    email={dpstory@acrotex.net},
    subject={Documentation for AeB Pro from AcroTeX},
    talksite={\url{www.acrotex.net}},
    version={1.0},
    keywords={Rich Media Annotations, SWF, FLV, MP3, AcroTeX, AcroFlex, LaTeX},
    copyrightStatus=True,
    copyrightNotice={Copyright (C) \the\year, D. P. Story},
    copyrightInfoURL={http://www.acrotex.net}
}
\nocopyright
\copyrightyears{2008-\the\year}
\def\dps{$\hbox{$\mathfrak D$\kern-.3em\hbox{$\mathfrak P$}%
   \kern-.6em \hbox{$\mathcal S$}}$}

\universityLayout{fontsize=Large}
\titleLayout{fontsize=LARGE}
\authorLayout{fontsize=Large}
\tocLayout{fontsize=Large,color=aeb}
\sectionLayout{indent=-62.5pt,fontsize=large,color=aeb}
\subsectionLayout{indent=-31.25pt,color=aeb}
\subsubsectionLayout{indent=0pt,color=aeb}
\subsubDefaultDing{\texorpdfstring{$\bullet$}{\textrm\textbullet}}

%\pagestyle{empty}
\parindent0pt\parskip\medskipamount

\newcommand{\myRMFiles}{C:/Users/Public/Documents/My TeX Files/%
    tex/latex/aeb/aebpro/rmannot/RMfiles}
\saveNamedPath{AcroAd}{\myRMFiles/Acro_Advertiser.swf}
\saveNamedPath{horse1}{\myRMFiles/horse1.flv}
\saveNamedPath{trek}{\myRMFiles/trek.mp3}
\saveNamedPath{AcroLimerick}{\myRMFiles/AcroTeX_limerick.mp3}
\makePoster[hiresbb]{AcroAd_poster}{\expath/AcroAd_poster}
\makePoster[hiresbb]{aebmovie_poster}{\expath/aebmovie_poster}
\makePoster[hiresbb]{horse1_poster}{\expath/horse1_poster}

\optionalPageMatter
{%
    \makebox[\linewidth][c]{%
        \rmAnnot[poster=AcroAd_poster,enabled=pageopen]%
            {.5\linewidth}{.5\linewidth*\ratio{265bp}{612bp}}{AcroAd}}%
}

\definePath{\bgPath}{"C:/Users/Public/Documents/ManualBGs/Manual_BG_Print_AeB.pdf"}
\begin{docassembly}
\addWatermarkFromFile({
    bOnTop:false,
    cDIPath:\bgPath
});
\executeSave();
\end{docassembly}

\def\puncPt#1{#1}

\begin{document}

\maketitle

\selectColors{linkColor=black}
\tableofcontents
\selectColors{linkColor=webgreen}

\section{Introduction}

Beginning with version 9, \textbf{Adobe Reader} and \textbf{Acrobat}
contain an embedded \textbf{Adobe Flash Player} that will play SWF,
FLV, and MP3 files. A new annotation type, called a \emph{rich media
annotation}, was developed to manage these media file types in a PDF
file.

The \texttt{rmannot} package supports the creation of rich media
annotations (a \texttt{RichMedia} annotation type), and the
embedding of SWF, FLV, and MP3 files in a PDF. SWF animations, FLV
video, and MP3 sound can then be played within a PDF viewed within
version 9 (or later) of \textbf{Adobe Reader} or \textbf{Acrobat}.\footnote{The
\texttt{rmannot} package was written, in part, to support the
{\AcroFLeX} Graphing package.}

Source material for the creation of this package is the document
\textsl{Adobe Supplement to the ISO 32000}, June 2008. This document
contains the PDF specification---the so called, BaseLevel~1.7,
ExtensionLevel~3 specification---of the rich media annotation.

\newtopic\textbf{Examples.} In addition to the examples that ship with the
\textsf{rmannot} package, there are numerous examples of \textsf{rmannot}
on my \href{http://www.math.uakron.edu/~dpstory/pdfblog.html}{{\AcroTeX} PDF
Blog}. I wrote a whole series of articles on the \textbf{Rich Media Annotation}
(Blogs \#1--11) using \textbf{AeB Pro} and \textsf{rmannot}. Additional
examples may appear, in time, on my \href{http://www.math.uakron.edu/~dpstory/aebblog.html}{AeB
Blog}

\section{Requirements}

The requirements for your {\LaTeX} system, and well as any other
software, is highlighted in this section.

\subsection{{\LaTeX} Package Requirements}

The following packages, in addition to the standard {\LaTeX}
distribution, are required:
\begin{enumerate}
  \item The \texttt{xkeyval} package is used to set up the key-value
      pairs of the \cs{rmAnnot} command. Get a recent version.
  \item AeB (\AcroTeX{} eDucation Bundle) The most recent version.
      In particular the \texttt{eforms} package and its companion
      package \texttt{insdljs}. The AeB Pro package is recommended. (All
      the demo files use AeB Pro.)
  \item The \texttt{graphicxsp} Package. The latest version, I made
      some slight modifications of this package for \texttt{rmannot}.
      This package allows the embedding of poster graphics for use in
      the appearances of the annotations when they are not activated.
\end{enumerate}
If you don't have AeB or \texttt{graphicxsp}, you can obtain the
latest versions from The AeB Home page\footnote{AeB:
\url{http://www.math.uakron.edu/~dpstory/webeq.html}} and the
GraphicxSP home page.\footnote{GraphicxSP:
\url{http://www.math.uakron.edu/~dpstory/graphicxsp.html}} The
installation instructions for AeB must be read very closely as there
are certain JavaScript files that must be copied to the correct location
on your local hard drive. The AeB Pro package can be obtained from
its home page as well.\footnote{AeB Pro:
\url{http://www.math.uakron.edu/~dpstory/aeb_pro.html}} It too has a
JavaScript file that needs to be installed, placement is important.


\subsection{PDF Creator Requirements}

The \textsf{rmannot} package supports \textbf{Acrobat Distiller 9.0} (or
later) as the PDF creator. The document author must have
\textbf{Acrobat 9.0 Pro} and its companion application
\textbf{Distiller}. The document author typically uses dvips to
produce a PostScript file, which is then distilled to obtain a PDF.

\section{Installation}

The installation is simple enough. Unzip \texttt{rmannot.zip} in a
folder that is on your {\LaTeX} search path.  Refresh your filename
database, if appropriate.

I am perhaps the last one using YandY, but if there is anyone else,
there is one other thing to do. The distribution comes with the
default poster file for the MP3 file; the name of this file is
\texttt{ramp3poster.eps} (found in the \texttt{graphics} subfolder).
For YandY users, this file needs to be copied to a folder on the
\texttt{PSPATH}. If you don't know what I'm talking about, follow
the steps below.

Open \textsf{dviwindo}, and go to \texttt{Preferences > Environment}
and choose \texttt{PSPATH} from the drop down menu. Add the path
\begin{Verbatim}[fontsize=\small]
    C:\yandy\tex\latex\contrib\rmannot\graphics\\
\end{Verbatim}
at the end of your \texttt{PSPATH} string.\footnote{If your
\textsf{YandY System} installation is elsewhere, enter that path.}
It is important to have the double backslash at the end of the path.
This tells the \textsf{YandY System} to search all subfolders for
the graphics files. When you are finished, your \texttt{PSPATH}
should look something like this:
\begin{Verbatim}[fontsize=\small]
    C:\yandy\ps;C:\yandy\tex\latex\contrib\rmannot\graphics\\
\end{Verbatim}
Be sure to separate these paths by a semicolon.

\textbf{\textcolor{red}{Important:}} In recent versions of Acrobat,
security restrictions have been put in place to prevent
\textbf{Distiller} from reading files (the PostScript \textbf{file}
operator does not work). Fortunately, Distiller has a switch that
turns off this particular restriction. To successfully use this
package, therefore, you need to run Distiller by using the
\texttt{-F} command line switch. I personally use the WinEdt
application as my text editor,\footnote{WinEdt home page:
\url{www.winedt.com}} and have defined a Distiller button on my
toolbar. The Distiller button executes the following WinEdt macro.
\begin{Verbatim}[fontsize=\footnotesize]
Run(|"c:\Program Files\Adobe\Acrobat 9.0\Acrobat\acrodist.exe" -F "%P\%N.ps"|,
    '%P',0,0,'%N.ps - Distiller',1,1);
\end{Verbatim}
Note the use of the \texttt{-F} switch for \texttt{acrodist.exe}. If
this package is used to create rich media annotations without the
\texttt{-F} switch, you typically get the following error message in
the Distiller log file
\begin{Verbatim}[fontsize=\small]
%%[ Error: undefinedfilename; OffendingCommand: file ]%%
\end{Verbatim}
This tells you that either you have not started Distiller with the
\texttt{-F} command line switch, or Distiller can't find one of the
files that the \textbf{file} operator was trying to read.

\newtopic \textbf{Mac OS Users.} The above comments on the \texttt{-F} command line switch
is for Windows users, Mac OS user must choose the \texttt{AllowPSFileOps} user preference, this is located
in the \texttt{plist}, possibly located at
\begin{Verbatim}[fontsize=\small]
/Users/[User]/Library/Preferences/com.adobe.distiller9.plist
\end{Verbatim}
You can use Spotlight, the search utility on Mac, to search for \texttt{com.adobe.distiller}.
This finds the file \texttt{com.adobe.distiller9.plist}. Clicking on this find,
Spotlight opens \texttt{com.adobe.distiller9.plist} in the \texttt{plist} editor, see \hyperref[plist]{Figure~\ref*{plist}}.
If necessary, click on the arrow next to the Root to expand the
choices, then click the up and down arrows at the far
right in the \texttt{AllowPSFileOps} row to select Yes as the value.
\begin{figure}[hbt]\setlength{\fboxsep}{0pt}
\begin{center}
\fbox{\includegraphics[width=.75\linewidth]{plistEditor}}
\caption{com.adobe.distiller9.plist}\label{plist}
\end{center}
\end{figure}

% \section{Options of this Package}


\section{Setting the Paths and Posters}

The paths to SWF/FLV/MP3 files are required to appear in the preamble, and any poster
graphics are required to appear in the preamble as well.

\subsection{Setting the Paths}

There are two types of paths: System paths to resources needed by
\textbf{Acrobat Distiller}, and media paths to the files used in the
document.

\paragraph*{System Paths.} This package uses \textbf{Acrobat Distiller~9.0}
(or later), and requires the document author to have \textbf{Acrobat 9 Pro}.
In the \textbf{Acrobat} program folder is a \texttt{Multimedia Skins} folder.
This folder contains skins (SWF files) used in providing playing
controls to FLV video files, and in the \texttt{Players} subfolder
you will find VideoPlayer.swf and AudioPlayer.swf. The former plays
FLV files with an appropriate skin for user controls, the latter
plays MP3 files. The document author needs to set these paths to
these files, which are passed on to the distiller. For this purpose
use \Com{pathToSkins}. The default definition follows:
\begin{dCmd*}{\linewidth}
\pathToSkins{%
    C:/Program Files/Adobe/Acrobat 9.0/Acrobat/Multimedia Skins}
\end{dCmd*}
This is the path on my WinXP system. The path for the Mac OS may look like
\begin{dCmd*}{\linewidth}
\pathToSkins{/Applications/Adobe\ Acrobat\ 9\ Pro/Adobe\ Acrobat\
    Pro.app/Contents/MacOS/Multimedia\ Skins}
\end{dCmd*}
These paths may differ from platform to platform. Note what the path is to
the Multimedia Skins folder. The command \cs{pathToSkins} also
defines the path to the \texttt{Players} subfolder. Future releases
of \textbf{Acrobat} may change the name of the folders, so a
\Com{pathToPlayers} command is also provided; as with
\cs{pathToSkins}, \cs{pathToPlayers} takes one argument, the path to
the players.

\handpoint The \textsf{rmannot} distribution comes with a
\texttt{rmannot.cfg} file. In this file, you can set the path to the
multimedia skins (\cs{pathToSkins}) on your system. Read the
instructions contained in this file.

\paragraph*{Document Media Paths.}

Each media file (SWF, FLV, MP3) must be declared in the preamble
using the \Com{saveNamedPath} command.
\begin{dCmd}{.67\linewidth}
\saveNamedPath[<mime_type>]{<name>}{<path>}
\end{dCmd}
The first optional argument \texttt{<mime\_type>} is normally not
needed. It is the mime type of the file. Currently, only SWF, FLV
and MP3 files are supported, and the extension of the file name is
isolated to determine the mime type.  The second parameter
\texttt{<name>} is a \emph{unique} name that will be used to
reference this media file. Finally, \texttt{<path>} is full and absolute path
to the media file. The path includes the file name and extension.

For example,
\begin{dCmd*}{.85\linewidth}
\saveNamedPath{mySWF}{C:/myMedia/AcroFlex3_demo.swf}
\saveNamedPath{fishing}{C:/myMedia/100_0239.flv}
\saveNamedPath{summertime}{C:/myMedia/Summertime.mp3}
\end{dCmd*}

Once the paths are defined in this way, the media files are
referenced using their given names. This has a couple of purposes.
\begin{enumerate}
    \item The names are used to determine if the media file has
        already been embedded in the document. Though the media clip
        may be used in several rich media annotations, the \texttt{rmannot}
        attempts to embed a media file only once.
    \item The command \cs{saveNamePath} uses
        \cs{hyper@normalise}, of the \texttt{hyperref} package, to
        ``sanitize'' special characters, so the path may contain
        characters that normally have special meaning to {\LaTeX}.
    \item Defining the path once leads to a consistent reference to
        the file paths, and reduces the chance of typos.
\end{enumerate}

A brief example to illustrate the use of the names assigned by the
\cs{saveNamedPath} follows:
\begin{dCmd*}{.75\linewidth}
\rmAnnot{200bp}{200bp}{mySWF}
\end{dCmd*}
See \Nameref{rmAnnot} for additional details on the \texttt{poster}
key and the \cs{rmAnnot} command.

The above example would use the default poster image to give a
visual of the annotation when it is not activated. The next section
discusses how to define and implement your own poster image.

\paragraph*{Defining a RM Path.}

The resources (\texttt{.flv}, \texttt{.swf}, \texttt{.mp3} files, for example)
for your Flash application my reside on your local computer or in the Internet.
As a way of reducing the amount of typing, you can use \Com{defineRMPath}
to define common paths to your resources.
\begin{dCmd}{.67\linewidth}
\defineRMPath{<\name>}{<path>}
\end{dCmd}
The command uses \cs{hyper@normalise} (of \textsf{hyperref}) to ``sanitize'' the path.
The first argument \verb!<\name>! is the name of the command to be created, and
\verb!<path>! is the path. After the definition \verb!<\name>! expands to
\verb!<path>!. For example,
\begin{dCmd*}{.85\linewidth}
\defineRMPath{\myRMFiles}{C:/myMedia}
\saveNamedPath{mySWF}{\myRMFiles/AcroFlex3_demo.swf}
\saveNamedPath{fishing}{\myRMFiles/100_0239.flv}
\saveNamedPath{summertime}{\myRMFiles/Summertime.mp3}
\end{dCmd*}
We first define a path to our resources, then save those paths along with the file names.

You can use \cs{defineRMPath} to define URLs as well
{\small
\begin{dCmd*}{\linewidth}
\defineRMPath{\myRMURLs}{http://www.example.com/~dpspeaker/videos}
\end{dCmd*}
}
Now, \cs{myRMURLs} points to your common video resources on the Internet.




\subsection{Creating Posters}\label{createPosters}

The \cs{rmAnnot} command has a \texttt{poster} key that is
recognized as part of optional key-value pairs. The use of the
\texttt{poster} key is optional, if you do not specify one, one will
be generated for you. (More on the default poster appearance is
presented below.) The poster image is visible when the rich media
annotation is not activated.

To create a poster for your rich media annotation, use a graphics
application (Adobe Illustrator, Adobe Photoshop, etc.), and save as
an EPS file. Move this file to your source file folder. Let's call
this file \texttt{cool\_poster.eps}. In the preamble place the command
\begin{dCmd*}{.5\linewidth}
\makePoster{myCP}{cool_poster}
\end{dCmd*}
The first argument is a \emph{unique name} for the graphic, the
second argument is the path name of the graphic (without the
extension). The name is used as the value of the \texttt{poster}
key.

The command actually has an optional first argument. This argument
is passed to the command \cs{includegraphics} (of the \texttt{graphicx}
package). The general syntax of the command is
\begin{dCmd}{\linewidth}
\makePoster[<options>]{<name>}{<path_to_EPS>}
\end{dCmd}
The command uses the \texttt{graphicxsp} package to embed the file
in the PDF document. The graphical image can then be used multiple
times in many annotations.

The graphic itself should have the same \emph{aspect ratio} as the rich
media annotation; this is important if the graphic contains text or
images that would get otherwise distorted.

Example,

\begin{dCmd*}{.75\linewidth}
\rmAnnot[poster=myCP]{200bp}{200bp}{mySWF}
\end{dCmd*}
See \Nameref{rmAnnot} for additional discussion of the \texttt{poster}
key and \cs{rmAnnot}.

\paragraph*{Default Poster Image.} The \texttt{rmannot} package has
default poster appearance. This poster appearance takes one of two
forms. If the media file is MP3, an image of the AudioPlayer control
bar is used; otherwise it is dynamically generated (with the correct
dimensions) using the following PostScript operators:
\begin{dCmd*}{.8\linewidth}
\defaultPoster
{%
    .7529 setgray
    0 0 \this@width\space\this@height\space rectfill
    10 \adj@measure 10 \adj@measure moveto .4 setgray
    /Helvetica \this@height\space 10 div selectfont
    (\rma@posternote) show
}
\end{dCmd*}
The commands \cs{this@width} and \cs{this@height} are the width and
height of the annotation. The command \cs{adj@measure} converts a
measurement to a proportion of the smaller of the two measurements
\cs{this@width} and \cs{this@height}.\footnote{The code presented
here is a simplified version of the actual code found in
\texttt{rmannot.dtx}. The definition of the default poster has a
number of macros that can be redefined to change the placement of
text, the color, size of the font, etc. See \texttt{rmannot.dtx}
for details.}

Note that, in the above code, some text is generated in the lower
left corner of the annotation, the text is \cs{rma@posternote}. This
command is populated by the value of the \texttt{posternote} key of
the optional argument of \cs{rmAnnot}. The default value of
\texttt{posternote} is \texttt{AcroTeX Flash} or \texttt{AcroTeX
Video}, depending on the file type of the media. This can be changed
through the \texttt{posternote} key.

The default poster itself can be redefined by a document author who
is schooled in PostScript things, perhaps if only to change colors,
or font, or location of the poster note.

\section{\texorpdfstring{\protect\cs{rmAnnot}}{\CMD{rmAnnot}} and its Options}\label{rmAnnot}

The \cs{rmAnnot} command creates a rich media annotation, new to
version 9 of \textbf{Acrobat}/\textbf{Adobe Reader}. Media files (SWF, FLV, or MP3)
can be either embedded in the document, or linked via a URL, and
played. \textbf{Acrobat}/\textbf{Adobe Reader} have a built-in Flash player that plays
SWF, FLV and MP3 files.

\goodbreak
Media files in other formats need to be converted to one of these
three supported formats.\footnote{The new \textbf{Acrobat 9 Pro Extended} can
convert media files to FLV, but embed the converted file in the PDF,
so we cannot really use that re-encoded file with our
\texttt{rmannot} package. Adobe Flash Video Encoder converts many
movie formats to FLV format, which can, in turn, be used in this
package. Other utilities may be available as shareware or
commercialware.}

\subsection{\texorpdfstring{\protect\cs{rmAnnot}}{\CMD{rmAnnot}} Command}

The primary command of this package is \Com{rmAnnot}, which has four
arguments, one optional and three required.
\begin{dCmd}{.75\linewidth}
\rmAnnot[<options>]{<width>}{<height>}{<name>}
\end{dCmd}
The \texttt{<width>} and \texttt{<height>} parameters are what they
are, the width and height to be used in the rich media annotation.
The aspect ratio should be the same as the aspect ratio of the
Flash media. The annotation can be resized using either
\cs{resizebox} or \cs{scalebox} of the \texttt{graphicx} package to
get the physical dimensions you want.

\textbf{For MP3 Files.} After a careful measurement, the aspect
ratio (width/height) of the MP3 \texttt{AudioPlayer} control bar is
about 9.6. In some of the demo files, I've been using a width of
\texttt{268bp} and a height of \texttt{28bp}, and resize the
annotation to what is desired. Use \texttt{268bp} and \texttt{28bp}
for the width and height of an MP3 file, and resize.

The \texttt{<name>} argument references a media file defined by the
\cs{saveNamedPath} in the preamble.

The \texttt{<options>} are discussed in the subsection that follows.

\subsubsection{\texorpdfstring{\protect\cs{rmAnnot}}{\CMD{rmAnnot}} Options}

The \cs{rmAnnot} command has many key-value pairs that are passed to
it through its first optional argument. Most of these key-value
pairs correspond to options available through the user interface of
\textbf{Acrobat}. Below is a listing of the key-values, and a brief
description of each.

\begin{itemize}
    \item\texttt{name}: The name of the annotation. If none is
        supplied, then \verb!aebRM\therm@Cnt! is used, where
        \texttt{rm@Cnt} is a {\LaTeX} counter that is incremented each
        time \cs{rmAnnot} is expanded.

    \item\texttt{enabled}: The enabled key determines when the
        annotation is activated, possible values are \texttt{onclick},
        \texttt{pageopen}, and \texttt{pagevisible}.
    \begin{itemize}
        \item \texttt{onclick}: The annotation is activated when the
            user clicks on the annotation, or is activated through
        JavaScript.
        \item \texttt{pageopen}: The annotation is activated when
            the page containing the annotation is opened.
        \item\relax\texttt{pagevisible}: The annotation is activated
            when the page containing the annotation becomes visible.
            (Useful for continuous page mode.)
    \end{itemize}
    The default is \texttt{onclick}.

    \item\texttt{deactivated}: The enabled key determines when the
        annotation is activate, possible values are \texttt{onclick},
        \texttt{pageopen}, and \texttt{pagevisible}.
    \begin{itemize}
        \item \texttt{onclick}: The annotation is deactivated by
            user script or by right-clicking the annotation and choosing
            Disable Content.
        \item \texttt{pageclose}: The annotation is deactivated when
            the page containing the annotation is closed.
        \item \texttt{pageinvisible}: The annotation is deactivated
            when the page containing the annotation becomes invisible.
            (Useful for continuous page mode.)
    \end{itemize}
    The default is \texttt{onclick}.

    \item \texttt{windowed}: A Boolean, which if \texttt{true}, the
        media is played in a floating window. The default is
        \texttt{false}, the media is played in the annotation on
        the page. For information on how to set the floating
        window parameters, see \mlNameref{winparams}.
    \item \texttt{url}: A Boolean, which if \texttt{true}, the media
        is to be interpreted as an URL. The default is \texttt{false},
        the media is embedded from the local hard drive and embedded in
        the PDF file.
    \item \texttt{borderwidth}: The borderwidth determines whether a
        border is drawn around the annotation when it is activated.
        Possible values are \texttt{none}, \texttt{thin},
        \texttt{medium}, and \texttt{thick}. The default is \texttt{none}.
    \item \texttt{poster}: The name of a poster graphic created by
        \cs{makePoster}. See the section \Nameref{createPosters} for
        additional details.
    \item \texttt{posternote}: When the poster key is not given, the
        default poster is generated. A short note of text appears in the
        lower left-corner. The text for that note can be passed to the
        default poster appearance through \texttt{posternote}. See
        \mlNameref{createPosters} for additional details.
    \item \texttt{transparentBG}: This option is available for SWF
        files only. Quoting the \emph{Adobe Supplement}
        document, ``A flag that indicates whether the page
        content is displayed through the transparent areas of
        the rich media content (where the alpha value is less
        than 1.0). If \texttt{true}, the rich media artwork is
        composited over the page content using an alpha channel.
        If false, the rich media artwork is drawn over an opaque
        background prior to composition over the page content.''
        The default is \texttt{false}.
    \item \texttt{passcontext}: A Boolean, if \texttt{true}, passes
        right-click context to Flash. Should be used only if there is a
        way of deactivating the annotation, perhaps through JavaScript.
        Recognized only for SWF files. The default is \texttt{false}.

        SWF file developers can select this option to replace the
        Acrobat context menu with the context menu of the
        originating SWF file. When the user right-clicks the SWF
        file, the available options are from the originating file.

    \item \texttt{skin}: For playing a FLV file, seven different
        skins are available for the user to control the video,
        \texttt{skin1}, \texttt{skin2}, \texttt{skin3}, \texttt{skin4},
        \texttt{skin5}, \texttt{skin6}, and \texttt{skin7}. Another
        possible value is \texttt{none}, for no skin. In the latter
        case, the media is played when activated, but there is no user
        interface to control the play. As for the description of each of
        the skins,
    \begin{itemize}
        \item \texttt{skin1}: All Controls
        \item \texttt{skin2}: Play, Stop, Forward, Rewind, Seek, Mute, and Volume
        \item \texttt{skin3}: Play
        \item \texttt{skin4}: Play and Mute
        \item \texttt{skin5}: Play, Seek, and Mute
        \item \texttt{skin6}: Play, Seek, and Stop
        \item \texttt{skin7}: Play, Stop, Seek, Mute, and Volume
        \item \texttt{none}: No Controls
    \end{itemize}
    \item \texttt{skinAutoHide}: A Boolean, if \texttt{true}, the skin auto hides.
        Only valid for FLV files.
    \item \texttt{skinBGColor} The color of the skin. The value is a color in hex format.
    The default is \texttt{0x5F5F5F}.  Only valid for FLV files.
    \item \texttt{skinBGAlpha}: The alpha level of the skin, a number between 0 and 1. The default
    is 0.75. Only valid for FLV files.
    \item \texttt{volume} The initial volume level of the video
    file, a number between 0 (muted) and 1
    (max volume). The default is 1.0. Only valid for FLV files.
    \item \texttt{speed}: Description quoted from the \textsl{Adobe
        Supplement} document. ``A positive number specifying the speed to be used
        when running the animation. A value greater than one shortens
        the time it takes to play the animation, or effectively speeds
        up the animation.'' The default is 1.
    \item\texttt{playcount}: Description quoted from the \textsl{Adobe
        Supplement} document. ``An integer specifying the play count for
        this animation style. A nonnegative integer represents the
        number of times the animation is played. A negative integer
        indicates that the animation is infinitely repeated.'' The
        default is -1.
    \item\texttt{cuepoints}: If the video is encoded with cue points, you
    can associate a JavaScript action with each. The value of \texttt{cuepoints}
    is a comma delimited list of cue points. See the paragraph
    \Nameref{cuepoints} for more details.
    \item \texttt{resources}: Use this key to list all files that
        are required to run a SWF file.  The value of the resources key
        is a comma delimited list of path names created by the \cs{saveNamedPath}
        command. \emph{The files referenced within this key are embedded in the PDF.}
        Files that are on the Internet---and are played from the Internet---should not be listed here.
    \item \texttt{flashvars}: Flash developers can use the \texttt{flashvars}
    key to add ActionScript variables for the SWF file. See the discussion
    of \textit{{\NoHyper\nameref{NameCmds}}} below.
\end{itemize}

\paragraph*{Some Name commands}\label{NameCmds} Within the optional parameters of the \cs{rmAnnot}
command, two convenience commands, \Com{Name} and \Com{urlName}, are
used with the \texttt{flashvars} key.

The \cs{Name} command may be used to set the value of a flash
variable. \cs{Name} has one argument, the symbolic name of a file
embedded by \cs{saveNamedPath}. The expansion of
\verb!\Name{<name>}! will appear in the Resources tab of the Edit
Flash dialog box. For example, if we define \texttt{myVid} as
\begin{Verbatim}[fontsize=\small]
    \defineRMPath{\myRMFiles}{C:/acrotex/video}
    \saveNamedPath{myVid}{\myRMFiles/assets/myVid.flv}
\end{Verbatim}
then \verb!\Name{myVid}! expands to \texttt{myVid.flv}. If the path is grouped
with braces, like so,
\begin{Verbatim}[fontsize=\small]
    \saveNamedPath{myVid}{\myRMFiles/{assets/myVid.flv}}
\end{Verbatim}
then \verb!\Name{myVid}! expands to \texttt{assets/myVid.flv}. This latter form
corresponds to adding a directory using the Add Directory button on the
Resources tab of the Edit Flash dialog box.

We can then use \cs{Name} as follows:
\begin{Verbatim}
    \rmAnnot[flashvars={source=\Name{myVid}},
        resources={myVid}]{320bp}{240bp}{mySWF}
\end{Verbatim}
where mySWF is the name of an SWF application that takes a flash variable named \texttt{source}, the value
of the variable is the video to be played.

The \cs{urlName} command is designed for resources on the Internet, and which are passed to the
SWF application with a flash variable.
\begin{Verbatim}[fontsize=\small]
    \defineRMPath{\myRMURLs}{http://www.example.com/~dpspeaker/videos}
    \saveNamedPath{myVid}{\myRMURLs/myVid.flv}
\end{Verbatim}
The expansion of \verb!\urlName{myVid}! is
\begin{verbatim}
    http://www.example.com/~dpspeaker/videos/myVid.flv
\end{verbatim}
We can then use \cs{urlName} as follows:
\begin{Verbatim}
    \rmAnnot[flashvars={source=\urlName{myVid}
    }]{320bp}{240bp}{mySWF}
\end{Verbatim}
Note that we don't list myVid as a resource, we just pass the URL to
\texttt{mySWF} as a flash variable.

\paragraph*{On Cue Points}\label{cuepoints} A cue point is any significant
moment in time occurring within a video clip. Cue points can be embedded
in the FLV using \textbf{Adobe Flash Professional}, or some other video encoder.

The value of the \texttt{cuepoints} key is a list of cue points data, a
``typical example'' is
\begin{Verbatim}[fontsize=\footnotesize]
\newcommand{\myCuePoints}{%
    {type=nav,name=Chapter1,time=0,action={console.println("Chapter1")}},%
    {type=nav,name=Chapter2,time=1883,action={console.println("Chapter2")}},%
    {type=nav,name=Chapter3,time=5197,action={console.println("Chapter3")}},%
    {type=nav,name=Chapter4,time=6817,action={console.println("Chapter4")}},%
    {type=nav,name=Chapter5,time=9114,action={console.println("Chapter6")}},%
    {type=nav,name=Chapter6,time=12712,action={console.println("Chapter6")}}
}
\end{Verbatim}
\textbf{\textcolor{red}{Comments:}} Having made such a definition, we then say
\verb!cuepoints={\myCuePoints}!, note that \cs{myCuePoints} must be enclosed in
braces. Note also in the above example, that the comment character
(\texttt{\%}) is used after each comma (\texttt{,}) in a line break. Because of the way the
argument is initially parsed, these comment characters are needed.

Each of the cue points is a comma-delimited list of key-value pairs; the
keys are \texttt{type}, \texttt{name}, \texttt{time}, and \texttt{action}.
Each of these are briefly described.

\begin{itemize}
    \item \texttt{type}: Possible values for this key are \texttt{nav} and
    \texttt{event}, and describes the type of cue point this is.
    \begin{itemize}
        \item \texttt{type=nav}: Navigation cue points enable users to seek
        to a specified part of a file. Embed Navigation cue points in the
        FLV stream and FLV metadata packet when the FLV file is encoded.

        \item[] Navigation cue points create a keyframe at the specified
        cue point location, so you can use code to move a video player�s
        playhead to that location. You can set particular points in an FLV
        file where you might want users to seek. For example, your video
        might have multiple chapters or segments, and you can control the
        video by embedding navigation cue points in the video
        file.\footnote{Taken in part from
        \url{http://www.peachpit.com/articles/article.aspx?p=663087}}
    \item \texttt{type=event}: Event cue points can also be embedded in
    your FLV stream and FLV metadata packet when video clip is encoded.
    You can write code to handle the events that are triggered at
    specified points during FLV playback.\footnote{Ibid.}
    \end{itemize}
    \item \texttt{name=<name>}: The name of the cue point
    \item \texttt{time=<time>}: The time in milliseconds the cue point
    occurs.
    \item \texttt{action=<JS\_action>} The JavaScript code that is executed
    when this cue point is reached.
\end{itemize}


\subsubsection{Setting the Floating Window Parameters}\label{winparams}

When the \texttt{windowed} key is set to \texttt{true}, the rich
media annotation appears in a floating window. Use the
\Com{setWindowDimPos} command to set the dimensions of the window
and its positioning.

\begin{dCmd}{.67\linewidth}
\setWindowDimPos{<key-value pairs>}
\end{dCmd}
\CmdLoc This command may be placed anywhere and will take affect for the next rich media annotation
created by \cs{rmAnnot}.

\PD There are a number of key-value pairs for setting the floating window; the default values are
normally adequate for most applications.

\begin{itemize}
\item\texttt{width}: The width is described by three \emph{key-value pairs}, \texttt{default},
    \texttt{max}, and \texttt{min}, measured in default user space units. Default values:
    \texttt{default}: \texttt{288}, \texttt{max}:\texttt{576}, \texttt{min}: \texttt{72}.

    For example, \verb!width={default=300,max=600,min=80}!.

\item\texttt{height}: The height is described by three \emph{key-value pairs}, \texttt{default},
    \texttt{max}, and \texttt{min}, measured in default user space units. Default values:
    \texttt{default}: \texttt{216}, \texttt{max}:\texttt{432}, \texttt{min}: \texttt{72}.

    For example, \verb!height={default=300,max=600,min=80}!.

\item\texttt{position}: The position of the floating window is
    described by four key-value pairs.
\begin{itemize}
    \item\texttt{halign}: The \texttt{halign} parameter describes
        the horizontal alignment of the window. Valid values are
        \texttt{near}, \texttt{center} and \texttt{far}. The default is
        \texttt{far}. For languages that read from left-to-right, a
        value of near refers to the left edge of the viewing window;
        whereas far refers to the right edge of the viewing window. (For
        right-to-left reading languages, the description of
        \texttt{near} and \texttt{far} are reversed.)

    \item\texttt{valign}: The \texttt{valign} parameter describes
        the vertical alignment of the window. Valid values are
        \texttt{near}, \texttt{center} and \texttt{far}. The default is
        \texttt{near}.

    \item\texttt{hoffset}: The description of \texttt{hoffset} is
        paraphrased from the \textsl{Adobe Supplement} document: The
        offset from the alignment point specified by the \texttt{halign}
        key. A positive value for \texttt{hoffset}, when \texttt{halign}
        is either \texttt{near} or \texttt{center}, offsets the position
        towards the \texttt{far} direction. A positive value for
        \texttt{hoffset}, when \texttt{halign} is \texttt{far}, offsets
        the position towards the \texttt{near} direction. The default is
        18.
    \item\texttt{voffset}: The description of \texttt{voffset} is
        paraphrased from the \textsl{Adobe Supplement}
        document: The offset from the alignment point
        specified by the \texttt{valign} key. A positive
        value for \texttt{voffset}, when \texttt{valign} is
        either \texttt{near} or \texttt{center}, offsets the
        position towards the \texttt{far} direction. A positive value
        for \texttt{voffset}, when \texttt{valign} is \texttt{far}, offsets the
        position towards the \texttt{near} direction. The default is
        18.
\end{itemize}
\end{itemize}
In layman's terms the combination of \texttt{halign=far, valign=near} puts
the floating window in the upper right corner of the active window of
Adobe Reader/Acrobat, assuming a left-to-right reading language.  The
values of \texttt{voffset=18,hoffset=18}, moves the floating window 18
points down and 18 points to the left. That would be its initial position.

\textbf{\textcolor{red}{Note}}: This feature, the positioning of the
window, never worked in Version~9, but has been implemented for
Version~10.

The \Com{resetWindowDimPos} command can be used to reset the
floating window parameters to their default values.


\begin{dCmd}{.5\linewidth}
\resetWindowDimPos
\end{dCmd}

\subsection{Examples}

In this section, several examples are presented that illustrate the \cs{rmAnnot}
and some of the key-value pairs.

\subsubsection{Posters}

The poster is an image that is displayed when the rich media annotation is not activated.
If a poster is not specified using the \texttt{poster} key, one is supplied for it.
Consider the following Flash animation.
\begin{center}
    \resizebox{!}{.75in}{\rmAnnot{612bp}{265bp}{AcroAd}}\quad
    \resizebox{!}{.75in}{\rmAnnot[poster=AcroAd_poster]{612bp}{265bp}{AcroAd}}%
\end{center}
Above are two rich media annotations, each running the same SWF
file. The one on the left uses the default poster, the one on the
right uses a custom poster. In the annotation on the left, you see
the default \texttt{posternote}, this can be changed using the
\texttt{posternote} key.

The custom poster was obtained by viewing the SWF file in Adobe
Flash Player~9, then printing one of the frames to Adobe PDF,
cropping the PDF, then saving the resulting PDF as an EPS file.
After you crop the printed image, you can determine its dimensions
by moving your mouse to the lower-left corner; the width and height
values should appear. Use these in setting up your annotation.

The verbatim listing for the two above annotations is found below.
\begin{dCmd*}{\linewidth}
\begin{center}
    \resizebox{!}{.75in}{\rmAnnot{612bp}{265bp}{AcroAd}}\quad
    \resizebox{!}{.75in}{%
        \rmAnnot[poster=AcroAd_poster]{612bp}{265bp}{AcroAd}}
\end{center}
\end{dCmd*}
The poster \texttt{AcroAd\_poster} was defined in the preamble of this document.

Below is the same video, the one on the left is a generic poster
created from a {\LaTeX} source file, then saved as an EPS file, the
one on the right was obtained from the poster page generated by
\textbf{Acrobat}. (See the paragraph below,
\hyperref[acroposter]{page~\pageref*{acroposter}}, for details on
how this was done.)
\begin{center}
\resizebox{2in}{!}{%
    \rmAnnot[poster=aebmovie_poster]{209bp}{157bp}{horse1}}\quad
\resizebox{2in}{!}{%
    \rmAnnot[poster=horse1_poster]{209bp}{157bp}{horse1}}
\end{center}
The verbatim listing for the two above annotations follows:
\begin{dCmd*}{\linewidth}
\resizebox{2in}{!}{%
    \rmAnnot[poster=aebmovie_poster]{209bp}{157bp}{horse1}}\quad
\resizebox{2in}{!}{%
    \rmAnnot[poster=horse1_poster]{209bp}{157bp}{horse1}}
\end{dCmd*}
Posters and media files are embedded only once, so using the same
poster and/or media file multiple times does not increase the file
size significantly.

For MP3 files, the default poster is an EPS file that is an image of
the player control bar, the example below shows the MP3 poster and audio
player.
\begin{center}
        \resizebox{!}{14bp}{\rmAnnot{268bp}{28bp}{trek}}
\end{center}
The code for the above annotation follows:
\begin{dCmd*}{.75\linewidth}
\resizebox{!}{14bp}{\rmAnnot{268bp}{28bp}{trek}}
\end{dCmd*}
A custom poster can be inserted using the \texttt{poster} key, as
usual.

\paragraph*{The Acrobat Pro generated poster.}\label{acroposter}
To acquire the same poster image that \textbf{Acrobat} generates,
use the following steps:
\begin{enumerate}
    \item Open \textbf{Acrobat}
    \item Drag and drop your SWF or FLV file onto an empty
        \textbf{Acrobat} window
    \item Press \textbf{Ctrl-P}, or select {File > Print}
    \item Select \textbf{Adobe PDF} as the printer
    \item Select \textbf{Choose paper source by PDF page size}
    \item Select \textbf{Use custom paper size when needed}
    \item Press \textbf{OK}
    \item A new PDF should be created, and it should be the same
        size as the poster image
    \item Choose {File > Save As}, select \texttt{Encapsulated
        PostScript (*.eps)} as the \textbf{Save as type}
    \item Press \textbf{Save}, and save to an appropriate folder.
\end{enumerate}

\subsubsection{Skin Options}

When a FLV video file is used, the video is played by the
VideoPlayer.swf and uses one of the seven standard skins.
Customizing information is actually passed using FlashVars. (For FLV
files, the user does not have access to the FlashVars, the
application, in this case, this package, uses the FlashVars.)
Customizing options include a choice of skin, setting the auto hide
flag, a choice of the color of the skin, setting the opacity of the
skin and setting the initial volume level. The following illustrates
some of the options on a short FLV video with a horse theme.
\begin{center}
\resizebox{2in}{!}{\rmAnnot[posternote=All Controls]{209bp}{157bp}{horse1}}\quad
\resizebox{2in}{!}{\rmAnnot[posternote={skin6: Play, Seek, Stop},skin=skin6,
    skinBGColor=0xFF0000,skinBGAlpha=0.25]{209bp}{157bp}{horse1}}
\end{center}
The video on the left shows the default settings (default skin, skin
alpha, volume level, etc.), while the same video on the right uses
skin6, with skin color of \texttt{0xFF0000} (red) and skin alpha
level set to 0.25.

\begin{center}
        \setLinkText[\A{\JS{%
            var rm=this.getAnnotRichMedia({nPage: this.pageNum, cName: "acrolimerick"});\r
            if (rm.activated) rm.callAS("multimedia_play");\r
            else rm.activated=true;
        }}]{\includegraphics[width=2in]{AeB_Logo}}\\[1ex]
        \resizebox{!}{14bp}{\rmAnnot[name=acrolimerick]{268bp}{28bp}{AcroLimerick}}
\end{center}

\bigskip

That's all for now, I simply must get back to my retirement. \dps

\end{document}
