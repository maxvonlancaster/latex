%%%%%%%%%%%%%%%%%%%%%%%%%%%%%%%%%%%%%%%%%%%%%%%%%%%%%%%%%%%%%%%%%%%%%
%% This is a (brief) model paper for submission to the RSC using
%% standard LaTeX packages
%%%%%%%%%%%%%%%%%%%%%%%%%%%%%%%%%%%%%%%%%%%%%%%%%%%%%%%%%%%%%%%%%%%%%
\documentclass[a4paper]{article}

%%%%%%%%%%%%%%%%%%%%%%%%%%%%%%%%%%%%%%%%%%%%%%%%%%%%%%%%%%%%%%%%%%%%%
%% A few core packages which are really essential
%%%%%%%%%%%%%%%%%%%%%%%%%%%%%%%%%%%%%%%%%%%%%%%%%%%%%%%%%%%%%%%%%%%%%
\usepackage[T1]{fontenc} % Modern font encoding
\usepackage{float}       % For creating charts, graphs and schemes
\usepackage{helvet}      % Helvetica font for sans serif
\usepackage{mathptmx}    % Times font ("Word-like")
\usepackage{rsc}         % Loads natbib, etc.
\usepackage{setspace}    % For double-spacing
\AtBeginDocument{\doublespacing}

%%%%%%%%%%%%%%%%%%%%%%%%%%%%%%%%%%%%%%%%%%%%%%%%%%%%%%%%%%%%%%%%%%%%%
%% New floats are created for standard chemistry graphics
%%%%%%%%%%%%%%%%%%%%%%%%%%%%%%%%%%%%%%%%%%%%%%%%%%%%%%%%%%%%%%%%%%%%%
\newfloat{chart}{htbp}{loc}
\newfloat{graph}{htbp}{loh}
\newfloat{scheme}{htbp}{los}

%%%%%%%%%%%%%%%%%%%%%%%%%%%%%%%%%%%%%%%%%%%%%%%%%%%%%%%%%%%%%%%%%%%%%
%% Place any additional packages needed here.  Only include packages
%% which are essential, to avoid problems later.
%%%%%%%%%%%%%%%%%%%%%%%%%%%%%%%%%%%%%%%%%%%%%%%%%%%%%%%%%%%%%%%%%%%%%
\usepackage[version=3]{mhchem} % Formula subscripts using \ce{}

%%%%%%%%%%%%%%%%%%%%%%%%%%%%%%%%%%%%%%%%%%%%%%%%%%%%%%%%%%%%%%%%%%%%%
%% If issues arise when submitting your manuscript, you may want to
%% un-comment the next line.  This provides information on the
%% version of every file you have used.
%%%%%%%%%%%%%%%%%%%%%%%%%%%%%%%%%%%%%%%%%%%%%%%%%%%%%%%%%%%%%%%%%%%%%
%%\listfiles

%%%%%%%%%%%%%%%%%%%%%%%%%%%%%%%%%%%%%%%%%%%%%%%%%%%%%%%%%%%%%%%%%%%%%
%% Place any additional macros here.  Please use \newcommand* where
%% possible, and avoid layout-changing macros (which are not used
%% when typesetting).
%%%%%%%%%%%%%%%%%%%%%%%%%%%%%%%%%%%%%%%%%%%%%%%%%%%%%%%%%%%%%%%%%%%%%
\newcommand*{\mycommand}[1]{\texttt{\emph{#1}}}

%%%%%%%%%%%%%%%%%%%%%%%%%%%%%%%%%%%%%%%%%%%%%%%%%%%%%%%%%%%%%%%%%%%%%
%% Meta-data block
%% ---------------
%% Give details for all authors here, using the standard LaTeX
%% method.
%%%%%%%%%%%%%%%%%%%%%%%%%%%%%%%%%%%%%%%%%%%%%%%%%%%%%%%%%%%%%%%%%%%%%
\author{%
  Andrew N. Other\thanks{%
    Department of Chemistry, Unknown University, Unknown Town%
  }%
  \and
  I. Ken Groupleader\thanks{%
    Department of Chemistry, Unknown University, Unknown Town,
    E-mail: \texttt{i.k.groupleader@unknown.uu}
  }%
  \and
  Susanne K. Laborator\thanks{%
    Lead Discovery, BigPharma, Big Town, USA
  }%
}

%%%%%%%%%%%%%%%%%%%%%%%%%%%%%%%%%%%%%%%%%%%%%%%%%%%%%%%%%%%%%%%%%%%%%
%% The document title should be given as usual
%%%%%%%%%%%%%%%%%%%%%%%%%%%%%%%%%%%%%%%%%%%%%%%%%%%%%%%%%%%%%%%%%%%%%
\title{A demonstration for submission to the RSC}

\begin{document}

%%%%%%%%%%%%%%%%%%%%%%%%%%%%%%%%%%%%%%%%%%%%%%%%%%%%%%%%%%%%%%%%%%%%%
%% Make a title, of course
%%%%%%%%%%%%%%%%%%%%%%%%%%%%%%%%%%%%%%%%%%%%%%%%%%%%%%%%%%%%%%%%%%%%%
\maketitle

\begin{abstract}
  This is an example document for writing a submission to the
  Royal Society of Chemistry. It does not seek to reproduce the
  layout of a published paper, but instead shows how the standard
  \LaTeX\ tools can be used to create a manuscript.
\end{abstract}

%%%%%%%%%%%%%%%%%%%%%%%%%%%%%%%%%%%%%%%%%%%%%%%%%%%%%%%%%%%%%%%%%%%%%
%% Start the main part of the manuscript here.
%%%%%%%%%%%%%%%%%%%%%%%%%%%%%%%%%%%%%%%%%%%%%%%%%%%%%%%%%%%%%%%%%%%%%
\section{Introduction}
This is a paragraph of text to fill the introduction of the
demonstration file.

\section{Results and discussion}

\subsection{Outline}

The document layout should follow the style of the journal concerned.
Consult the appropriate Instructions for Authors to see what is
required. Some journals, for example, do not allow sections in
articles.

\subsection{References}

The rsc bibliography style requires the \textsf{natbib} package,
which is loaded automatically by the \textsf{rsc} package.
References can be made using the normal method; if the
\textsf{natmove} package is available, citations can be placed
before any punctuation, and they will be moved as necessary
\cite{Mena2000,Abernethy2003}. The \textsf{natmove} package is
part of the \textsf{achemso} bundle.

The use of \textsf{natbib} allows the use of the various citation
commands of that package: \citeauthor{Abernethy2003} have shown
something, in \citeyear{Cotton1999}, or as given by
Ref.~\citenum{Mena2000}. If you encounter problems with the
citation macros, please check that your copy of \textsf{natbib}
is up to date.

Multiple citations to be combined into a list can be given as
a single citation.  This uses the \textsf{mciteplus} package
\cite{Arduengo1992,*Eisenstein2005,*Arduengo1994}.  Citations
other than the first of the list should be indicated with a star.
If the \textsf{mciteplus} package is not installed, the standard
bibliography tools will still work but starred references will be
ignored.

\subsection{Floats}

Setting up new float types is demonstrated in the premable, where
charts, graphs and schemes are defined.  As illustrated, the float
is ``here'' if possible (Scheme~\ref{sch:example}).
\begin{scheme}
  \centering
  Your scheme graphic would go here: \texttt{.eps} format\\
  for \LaTeX\, or \texttt{.pdf} (or \texttt{.png}) for pdf\LaTeX\\
  \textsc{ChemDraw} files are best saved as \texttt{.eps} files;\\
  these can be scaled without loss of quality, and can be\\
  converted to \texttt{.pdf} files easily using \texttt{eps2pdf}.\\
  %\includegraphics{graphic}
  \caption{An example scheme}
  \label{sch:example}
\end{scheme}

\begin{figure}
  \centering
  As well as the standard float types \texttt{table}\\
  and \texttt{figure}, the premable defines\\
  \texttt{scheme}, \texttt{chart} and \texttt{graph}.
  \caption{An example figure}
  \label{fgr:example}
\end{figure}

Charts, figures and schemes do not necessarily have to be labelled or
captioned.  However, tables should always have a title. It is
possible to include a number and label for a graphic without any
title, using an empty argument to the \texttt{\textbackslash caption}
macro.

The use of the different floating environments is not required, but
it is intended to make document preparation easier for authors. In
general, you should place your graphics where they make logical
sense; the production process will move them if needed.

\subsection{Math(s)}

This file does not load any particular additional support for
mathematics.  If the author \emph{needs} things like
\textsf{amsmath}, they should be loaded in the preamble.  However,
the basics should work fine.  Some inline material $ y = mx + c$
followed by some display. \[ A = \pi r^2 \]

It is possible to label equations in the usual way.
\begin{equation}
  \frac{\mathrm{d}}{\mathrm{d}x} r^2 = 2r
\end{equation}
This can also be used to have equations containing graphical
content.
\begin{equation}
  \textrm{Some content}
  %\includegraphics{graphic}
  \label{eqn:graphic}
\end{equation}

\section{Experimental}

The usual experimental details should appear here.  This could
include a table, which can be referenced as Table~\ref{tbl:example}.
Do not worry about the appearance of the table: this will be altered
during production.
\begin{table}
  \centering
  \caption{An example table}
  \label{tbl:example}
  \begin{tabular}{ll}
    \hline
    Header one & Header two \\
    \hline
    Entry one & Entry two \\
    Entry three & Entry four \\
    Entry five & Entry five \\
    Entry seven & Entry eight \\
    \hline
  \end{tabular}
\end{table}

Adding notes to tables can be complicated.  Perhaps the easiest
method is to generate these using the basic
\texttt{\textbackslash textsuperscript} and
\texttt{\textbackslash emph} macros, as illustrated
(\ref{tbl:notes}).
\begin{table}
  \caption{A table with notes}
  \label{tbl:notes}
  \begin{tabular}{ll}
    \hline
    Header one & Header two \\
    \hline
    Entry one\textsuperscript{\emph{a}}
      & Entry two \\
    Entry three\textsuperscript{\emph{b}}
      & Entry four \\
  \hline
  \end{tabular}

  \textsuperscript{\emph{a}} Some text;
  \textsuperscript{\emph{b}} Some more text.
\end{table}

The example file also loads the \textsf{mhchem} package, so
that formulas are easy to input: \texttt{\textbackslash
\ce\{H2SO4\}} gives \ce{H2SO4}.  See the use in the
bibliography file (when using titles in the references
section).

The use of new commands should be limited to simple things which will
not interfere with the production process.  For example,
\texttt{\textbackslash mycommand} has been defined in this example,
to give italic, monospaced text: \mycommand{some text}.

%%%%%%%%%%%%%%%%%%%%%%%%%%%%%%%%%%%%%%%%%%%%%%%%%%%%%%%%%%%%%%%%%%%%%
%% The appropriate \bibliographystyle and \bibliography commands
%% should be placed here.
%%%%%%%%%%%%%%%%%%%%%%%%%%%%%%%%%%%%%%%%%%%%%%%%%%%%%%%%%%%%%%%%%%%%%
\bibliographystyle{rsc}
\bibliography{rsc}

\end{document}
