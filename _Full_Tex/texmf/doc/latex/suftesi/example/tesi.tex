% !TEX encoding = UTF-8 Unicode
%	copyright (C) Ivan Valbusa 2010
%
%	Questo file è distribuito assieme alla classe suftesi nella cartella suftesi.zip

\documentclass{suftesi}

\usepackage[T1]{fontenc}                
\usepackage[utf8]{inputenc}% Fate sempre attenzione alla codifica di input!
\usepackage[polutonikogreek,german,english,italian]{babel}
\usepackage{graphicx}
   \graphicspath{{Immagini/}}
\usepackage{booktabs}% per le tabelle
\usepackage{emptypage}%elimina la testatina sulle pagine bianche
\usepackage{varioref}% riferimenti incrociati
\usepackage{lipsum}% Produce testo fittizio
 \usepackage{makeidx}
     \addto\captionsitalian{\renewcommand{\indexname}{Indice dei nomi}}
       \makeindex
% biblatex settings
\usepackage[babel,italian=guillemets]{csquotes}
\usepackage[
	style=philosophy-verbose,%philosophy-modern, philosophy-classic
	]{biblatex}
   \bibliography{bibliografia}
%\nocite{*}
% FRONTESPIZIO
%****************************************
% (1)	Compilare Tesi.tex
% (2)	Compilare il file Tesi-fr.tex generato al punto (1)
% (3)	Compilare Tesi.tex
 \usepackage{frontespizio}% 

% RIFERIMENTI INCROCIATI
\usepackage{hyperref}
\hypersetup{colorlinks=true,citecolor=blue,draft=false}
\begin{document}
%***********************************************************
%
% 			IMPORTANTE!
%
%***********************************************************
% DECOMMENTARE LE DUE RIGHE SEGUENTI
% SE SI METTE ``FRENCH'' TRA LE OPZIONI DI BABEL
%\selectlanguage{italian}
%\frenchspacing


%%			frontespizio
%***************************************************************************
\begin{frontespizio} 
\Universita{Paperopoli} 
\Logo{logo} 
\Facolta{Pennutologia} 
\Corso{Belle Lettere} 
\Annoaccademico{2030--2031} 
\Titoletto{Tesi di laurea magistrale} 
\Titolo{La mia tesi:\\ una lunga serie di risultati\\
difficilissimi e complicatissimi} 
\Sottotitolo{Alcune considerazioni mutevoli} 
\Candidato[PP999999]{Paperino Paolino} 
\Relatore{Giovanni Episcopo} 
\Relatore{Pippo Cluvio}
\Correlatore{Ugo Frogio} 
\Correlatore{Ubaldo Kutuzu} 
\end{frontespizio}

%%			Colophon
%***************************************************************************
\colophon[<sistema operativo>]{Nome Cognome}{<specifiche sui font utilizzati e/o sulla data di discussione della tesi, ecc.>}

%***************************************************************************
%%			MATERIALE INIZIALE
%***************************************************************************
\frontmatter
% Indici
\tableofcontents
\listoftables
\listoffigures
% Ringraziamenti
% !TEX encoding = UTF-8 Unicode

%*****************
% Ringraziamenti
%****************


\chapter{Ringraziamenti}

\lipsum[1-3]



% Sigle
	 \defbibnote{prenote:shorthands}{Breve testo introduttivo alle sigle}
	 \defbibnote{postnote:shorthands}{Possiamo anche inserire un'avvertenza alla fine delle sigle, come questa. In genere non serve. Potrebbe essere utile nel caso si voglia usare un apparato di sigle molto complesso.}
		 \printshorthands[prenote=prenote:shorthands,postnote=postnote:shorthands]
		 \addcontentsline{toc}{chapter}{Sigle}
% Introduzione 
% !TEX encoding = UTF-8 Unicode

%**************
% Introduzione
%**************

\chapter{Introduzione}

\lipsum[1-3]

%***************************************************************************
%%				CAPITOLI
%***************************************************************************
\mainmatter
\part{uno}
% !TEX encoding = UTF-8 Unicode
\chapter{Il primo capitolo della mia tesi}

\chapterintro

Questa è un'introduzione non numerata.


\section{Citazioni fuori testo e note}

Ecco una citazione fuori teso, o ``in display'':
\begin{quotation}
\lipsum[11]
\end{quotation}

Questa è una citazione di una massima:
\begin{quote}
Meglio un uovo oggi che una gallina domani.\footnote{Una delle massime più conosciute.}
\end{quote}

Questa è una citazione di una terzina dantesca:
\begin{verse}
Nel mezzo del cammin di nostra vita\\\marginpar{I primi tre versi della \emph{Divina Commedia}}
mi ritrovai per una selva oscura,\\
ché la diritta via era smarrita.
\end{verse}


\section{Documenti multilingua}

\subsection{Greco antico}

   \begin{otherlanguage*}{polutonikogreek}
Th| p`anta dido`ush|
ka'i >apolambano`ush| f`usei
<o pepaideum`enos ka'i
a>id`hmwn l`egei; <<d'os, <'o
j`eleis, >ap`olabe, <'o
j`eleic>>. L`egei d'e to~uto
o>u katajrasun`omenos, >all'a
peijarq~wn m`onon ka'i
e>uno~wn a>ut~h|.
   \end{otherlanguage*}
   
 \subsection{Inglese}
 
 \begin{otherlanguage*}{english}
The Categories are the clue to the discovery of, in view of these considerations, the Ideal. To avoid all misapprehension, it is necessary to explain that our concepts would thereby be made to contradict, in view of these considerations, the things in themselves. As any dedicated reader can clearly see, our sense perceptions (and to avoid all misapprehension, it is necessary to explain that this is the case) are a representation of space. Since all of natural causes are problematic, the objects in space and time are by their very nature contradictory.
\end{otherlanguage*}


\section{Citazioni}

\cite{Ethica,Bringhurst:1996,Descartes:1897,Facchinetti:2009a,Galilei1,kant:kpv}

% Section
%****************************
\section{Elenchi puntati e numerati e descrizioni}\label{sec:elenchi}


\subsection{Elenchi puntati}
\begin{itemize} 
\item Kant
\item Cartesio
\item Leibniz
\item Platone
\end{itemize}

 \begin{itemize}
 \item[-] Kant
 \item[-] Cartesio
 \item[-] Leibniz
 \item[-] Platone
 \end{itemize}

\subsection{Elenchi numerati}
\begin{enumerate}
 \item Kant
 \item Cartesio
% \item Leibniz
% \item Platone
 \end{enumerate}
 
\subsection{Descrizioni} 
\begin{description}
 \item[Kant]
 \item[Cartesio]
 \item[Leibniz]
 \item[Platone]
 \end{description}
 
  
 \section{Figure e tabelle}

\subsection{Inserimento delle figure}

Si veda la figura \ref{fig:knuth-lamport}.
\begin{figure}[h]
\centering
\includegraphics[width=3cm]{don}
\quad
\includegraphics[width=3cm]{lamport}
\caption[Donald Knuth e Laslie Lamport]{Donald Knuth e Laslie Lamport, ovvero i padri di \TeX{} e \LaTeX.}
\label{fig:knuth-lamport}
\end{figure}


\subsection{Inserimento di tabelle}

La tabella~\vref{tab:gabbia} raccoglie le misure dei diversi elementi
della pagina della classe \textsf{suftesi}.
\begin{table}[h]
\centering
\begin{tabular}{lc}
\toprule
Elemento                 & Valore (pt)\\\midrule
Giustezza                & 312 \\
Altezza del testo        & 624 \\
Margine interno          & 95 \\
Margine esterno          & 190 \\
Margine superiore        & 74 \\
Margine inferiore        & 147 \\
Proporzioni del testo    & $1:2$\\
Proporzioni della pagina & $1/\sqrt{2}$\\
\bottomrule
\end{tabular}
\caption[Misure degli elementi della pagina]{Valore in punti dei diversi elementi della pagina.}
\label{tab:gabbia}
\end{table}
 
 % !TEX encoding = UTF-8 Unicode
\chapter{Il secondo capitolo della mia tesi}
\lipsum[2-3]


 % !TEX encoding = UTF-8 Unicode
\chapter{Il terzo capitolo della mia tesi}
\lipsum[3-5]

 % !TEX encoding = UTF-8 Unicode
\chapter{Il quarto capitolo della mia tesi}
\lipsum[1-3]

 
%***************************************************************************
%%			MATERIALE FINALE
%***************************************************************************
\backmatter
% !TEX encoding = UTF-8 Unicode

%************
% Conclusione
%************

\chapter{Conclusione}

\lipsum[1-3]


%***************************************************************************
%	Appendici
%***************************************************************************
%\nomeappendici{Appendici}
\appendix
\chapter{Prima}
\lipsum
\chapter{Seconda}
\lipsum
\chapter{Terza}
\lipsum

%***************************************************************************
% Sigle
%***************************************************************************
\printshorthands

%***************************************************************************
% 	Bibliografia
%***************************************************************************
\cleardoublepage
\phantomsection % solo se si usa hyperref
\addcontentsline{toc}{chapter}{\bibname}
\printbibheading
Breve testo introduttivo per spiegare in che modo sono state organizzate le bibliografie. Questo è solo un semplice esempio. Con il pacchetto \textsf{biblatex} si possono comporre bibliografie molto più complesse. È bene però non abusare di queste potenzialità se non strettamente necessario: \emph{la bibliografia è un mezzo, non un fine}.
% Bibliografia primaria
\defbibnote{primaria}{\sffamily Breve testo introduttivo per la bibliografia secondaria. In genere non serve, ma per bibliografie molto complesse potrebbe essere utile.}
\phantomsection % solo se si usa hyperref
\addcontentsline{toc}{section}{Bibliografia primaria}
\printbibliography[keyword=primaria,title=Bibliografia primaria,heading=subbibliography,prenote=primaria]
% Bibliografia secondaria
\defbibnote{secondaria}{\sffamily Breve testo introduttivo per la bibliografia primaria. In genere non serve, ma per bibliografie molto complesse potrebbe essere utile.}
\phantomsection % solo se si usa hyperref
\addcontentsline{toc}{section}{Bibliografia secondaria}
\printbibliography[keyword=secondaria,title=Bibliografia secondaria,heading=subbibliography,prenote=secondaria]
%***************************************************************************
% Indice dei nomi
%***************************************************************************
\phantomsection % solo se si usa hyperref
\addcontentsline{toc}{chapter}{Indice dei nomi}
\printindex



\end{document}  