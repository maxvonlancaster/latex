% Eine kurze Einfuehrung in die Verwendung von syntax.tex!
%
% syntax.tex ermoeglicht das automatische Setzen von Syntaxdiagrammen fuer 
% Progammiersprachen durch spezielle Befehle in LaTeX-Dokumenten 
%
% Bernd Worsch 7.7.97 / 27.1.00
%
% 

\documentclass{article}
\usepackage{german}
\usepackage{fancybox}

% Einbinden von syntax.tex
% Dieses File ist dazu gedacht, im Vorspann von LaTeX Dokumenten vermittels
% % Dieses File ist dazu gedacht, im Vorspann von LaTeX Dokumenten vermittels
% % Dieses File ist dazu gedacht, im Vorspann von LaTeX Dokumenten vermittels
% \include{syntax.tex} eingebunden zu werden. Es stellt dann die Umgebung
% syntaxdiagramm zur Verfuegung, mit deren Hilfe Syntaxdiagramme fuer
% z.B. Pascal erzeugt werden koennen. 
%
% syntax.tex benoetigt nur noch folgendes LaTeX Paket:
%   - fancybox.sty
% (Es muss im Vorspann des aufrufenden Dokumentes eingebunden werden.) 
%
% Beispiele zur Verwendung siehe: syntaxtest.tex 
% zur naeheren Dokumentation siehe: syntaxintro.tex
% zur Entwicklung des Codes siehe: syntaxdevel.tex
%
% Bernd Worsch 29.6.97 
%
% 05.07.97  \Leer als vertikale Stuetze
%           groessere Abstaende in Alternativen-Umgebung 
%
% to-do-list:
%   -- Oval fuer SynVar ueber picture erzeugen -> fancybox unnoetig
%   -- Alternativen-Parameter automatisch bestimmen
%   -- Dokumentation (syntaxuse.tex)
%   -- Kommentare im Quelltext
%   -- Als sty-File implementieren



\typeout{Verwende: 'syntax.tex' v0.02 von Bernd Worsch}



%%%% Anlegen der benoetigten Laengen und Boxen 

\newlength{\breite}
\newlength{\hoehe}
\newlength{\tiefe}
\newlength{\gesamthoehe}
\newlength{\tmpLength}

\newsavebox{\Messbox}

\newlength{\LineLengStan}
\newlength{\LineThikStan}
\setlength{\LineLengStan}{4mm}
\setlength{\LineThikStan}{\fboxrule}

\newsavebox{\Centerbox}
\newlength{\Centercor}

\newlength{\VertSkipStan}
\setlength{\VertSkipStan}{3mm}

\newsavebox{\upperbox}
\newsavebox{\lowerbox}
\newsavebox{\togetherbox}
\newlength{\VertLineLength}
\newlength{\VertSkip}

\newlength{\upperVertLineLength}
\newlength{\lowerVertLineLength}

\newlength{\tabulcor}

%%%%% Beginn Syntaxdiagramm Befehle und Umgebungen

\newenvironment{Syntaxdiagramm}[1][\PfeilR]%
 {%
%
%%%% Befehl zum Vermessen von TeX-Strukturen
%
%\newcommand{\Messe}[1]
% {
%  \sbox{\Messbox}{##1}
%
%  \setlength{\breite}{\wd\Messbox}
%  \setlength{\hoehe}{\ht\Messbox}
%  \setlength{\tiefe}{\dp\Messbox}
%  \setlength{\gesamthoehe}{\hoehe}
%  \addtolength{\gesamthoehe}{\tiefe}
%  \setlength{\tmpLength}{\hoehe}
%  \addtolength{\tmpLength}{-\tiefe}
%  \setlength{\tmpLength}{-0.5\tmpLength} 
%
%  \fbox
%   {
%    \begin{tabular}{ll}
%      Breite:      & \the\breite      \\
%      Hoehe:       & \the\hoehe       \\
%      Tiefe:       & \the\tiefe       \\
%      Gesamthoehe: & \the\gesamthoehe \\
%      CenterCor:   & \the\tmpLength
%    \end{tabular}
% 
%    \usebox{\Messbox}
%   }
% }
%
%%%% Befehl zum Justieren von Boxen 
%
\newcommand{\CenterToBaseLine}[1]
  {%
   \sbox{\Centerbox}{##1}%
   \setlength{\Centercor}{\ht\Centerbox}%
   \addtolength{\Centercor}{-\dp\Centerbox}%
   \setlength{\Centercor}{-0.5\Centercor}%
   \addtolength{\Centercor}{0.5\LineThikStan}%
   \raisebox{\Centercor}{\usebox{\Centerbox}}%
  }%
%
%%%% Grundelemente eines Syntaxdiagramms
%
\newcommand{\Leer}%
  {\rule[-1.3ex]{0pt}{2.5ex}}%
%
\newcommand{\Linie}%
  {\rule{\LineLengStan}{\LineThikStan}}%
%
\newcommand{\KurzeLinie}%
  {\rule{0.5\LineLengStan}{\LineThikStan}}%
%
\newcommand{\Tri}
  {%
   \CenterToBaseLine
   {%
     \setlength{\unitlength}{1pt}%
     \begin{picture}(0,0)%
       \put(0,0){\vector(0,1){0}}%
     \end{picture}%
   }%
  }%
%
\newcommand{\TriR}
  {%
   \CenterToBaseLine
    {%
     \setlength{\unitlength}{1pt}%
     \begin{picture}(0,0)%
       \put(0,0){\vector(1,0){0}}%
     \end{picture}%
    }%
  }%
%
\newcommand{\TriD}
  {%
   \CenterToBaseLine
    {%
     \setlength{\unitlength}{1pt}%
     \begin{picture}(0,0)%
       \put(0,0){\vector(0,-1){0}}%
     \end{picture}%
    }%
  }%
%
\newcommand{\TriL}
  {%
   \CenterToBaseLine
    {%
     \setlength{\unitlength}{1pt}%
     \begin{picture}(0,0)%
       \put(0,0){\vector(-1,0){0}}%
     \end{picture}%
    }%
  }%
%
\newcommand{\Dots}
  {%
   \CenterToBaseLine
    {%
     ...%
    }%
  }%
%
\newcommand{\SynVar}[1]%
  {%
   \CenterToBaseLine%
    {%
     \fbox%
      {%
       \rule{0.1em}{0pt}%
       \rule[-0.3ex]{0pt}{2ex}%
       ##1%
       \rule{0.1em}{0pt}%
      }%
    }%
  }%
%
\newcommand{\TerSym}[1]%
  {%
   \CenterToBaseLine%
    {%
     \ovalbox%
      {%
       \rule{0.2em}{0pt}%
       \rule[-0.2ex]{0pt}{1.8ex}%
       ##1%
       \rule{0.2em}{0pt}%
      }%
    }%
  }%
%
%%%% Kombinierte Grafische Elemente
%
\newcommand{\Ellipse}%
 {%
  \Linie\ \Dots\ \Linie%
 }%
%
\newcommand{\PfeilR}%
 {%
  \Linie\TriR%
 }%
%
\newcommand{\PfeilL}%
 {%
  \TriL\Linie%
 }%
%
\newcommand{\SynVarR}[1]%
 {%
  \PfeilR\SynVar{##1}\Linie%
 }%
%
\newcommand{\SynVarL}[1]%
 {%
  \Linie\SynVar{##1}\PfeilL%
 }%
%
\newcommand{\TerSymR}[1]%
 {%
  \PfeilR\TerSym{##1}\Linie%
 }%
%
\newcommand{\TerSymL}[1]%
 {%
  \Linie\TerSym{##1}\PfeilL%
 }%
%
%%%% Vertikale grafische Elemente
%
\newcommand{\LinieO}[1]%
 {%
  \CenterToBaseLine%
   {%
    \raisebox{0pt}[0pt][0pt]%
     {%
      \makebox[0mm][c]{\rule{\LineThikStan}{##1}}%
     }%
   }%
 }%
%
\newcommand{\LinieU}[1]%
 {%
  \CenterToBaseLine%
   {%
    \raisebox{0pt}[0pt][0pt]%
     {%
      \makebox[0mm][c]{\rule[-##1]{\LineThikStan}{##1}}%
     }%
   }%
 }%
%
\newcommand{\PfeilO}[1]%
 {%
  \CenterToBaseLine%
   {%
    \raisebox{0pt}[0pt][0pt]%
     {%
      \makebox[0mm][c]{\TriD}%
      \makebox[0mm][c]{\rule{\LineThikStan}{##1}}%
     }%
   }%
 }%
%
\newcommand{\PfeilU}[1]%
 {%
  \CenterToBaseLine%
   {%
    \raisebox{0pt}[0pt][0pt]%
     {%
      \makebox[0mm][c]{\Tri}%
      \makebox[0mm][c]{\rule[-##1]{\LineThikStan}{##1}}%
     }%
   }%
 }%
%
%%%% Strukturen
%
%%% Wiederholung
%
\newcommand{\Wiederholung}[2]%
 {%
  \sbox{\upperbox}{##1}%
  \sbox{\lowerbox}{##2}%
  \setlength{\VertLineLength}{\dp\upperbox}%
  \addtolength{\VertLineLength}{\ht\lowerbox}%
  \addtolength{\VertLineLength}{\VertSkipStan}%
  \addtolength{\VertLineLength}{0.5\LineThikStan}%
  \setlength{\VertSkip}{\VertSkipStan}%
  \addtolength{\VertSkip}{\dp\upperbox}%
%
  \Linie%
  \PfeilU{\VertLineLength}%
% 
  \renewcommand{\arraystretch}{0}%
  \begin{tabular}[t]%              % Zwei Eintraege
   {%
    @{\hrulefill}%                 % Verbindungslinie 1
    c%                             % Eintraege Zentrieren
    @{\hrulefill}%                 % Verbindungslinie 2
   }%
   \PfeilR\usebox{\upperbox}\Linie\\[\VertSkip]  % Dies wiederholen
   \Linie\usebox{\lowerbox}\PfeilL%              % Dies in Gegenrichtung
  \end{tabular}%
%
  \LinieU{\VertLineLength}%
  \Linie%
 }%
%
%%% Alternative
%
\newcommand{\Alternative}[2]%
 {%
  \sbox{\upperbox}{##1}%
  \sbox{\lowerbox}{##2}%
%
  \setlength{\upperVertLineLength}{\dp\upperbox}%
  \addtolength{\upperVertLineLength}{0.5\VertSkipStan}%
  \addtolength{\upperVertLineLength}{0.5\LineThikStan}%
%
  \setlength{\lowerVertLineLength}{\ht\lowerbox}%
  \addtolength{\lowerVertLineLength}{0.5\VertSkipStan}%
  \addtolength{\lowerVertLineLength}{0.5\LineThikStan}%
%
  \setlength{\VertSkip}{\VertSkipStan}%
  \addtolength{\VertSkip}{\dp\upperbox}%
%
  \Linie%
  \LinieO{\upperVertLineLength}%
  \LinieU{\lowerVertLineLength}%
% 
  \renewcommand{\arraystretch}{0}%
  \addtolength{\upperVertLineLength}{-0.5\LineThikStan}%
  \raisebox{\upperVertLineLength}%
   {% 
    \begin{tabular}[t]%                 % Zwei Eintraege
     {%
      @{\hrulefill}%                 % Verbindungslinie 1
      c%                             % Eintraege Zentrieren
      @{\hrulefill}%                 % Verbindungslinie 2
     }%
     \PfeilR\usebox{\upperbox}\Linie\\[\VertSkip]  % Weg1
     \PfeilR\usebox{\lowerbox}\Linie%              % Weg2
    \end{tabular}%
   }%
  \renewcommand{\arraystretch}{1}%
  \addtolength{\upperVertLineLength}{0.5\LineThikStan}%
%
  \LinieO{\upperVertLineLength}%
  \LinieU{\lowerVertLineLength}%
  \Linie%
 }%
%
%%% Stapel
%
\newcommand{\Stapel}[2]%
 {%
  \sbox{\upperbox}{##1}%
  \sbox{\lowerbox}{##2}%
%
  \setlength{\upperVertLineLength}{\dp\upperbox}%
  \addtolength{\upperVertLineLength}{0.5\VertSkipStan}%
  \addtolength{\upperVertLineLength}{0.5\LineThikStan}%
%
  \setlength{\lowerVertLineLength}{\ht\lowerbox}%
  \addtolength{\lowerVertLineLength}{0.5\VertSkipStan}%
  \addtolength{\lowerVertLineLength}{0.5\LineThikStan}%
%
  \setlength{\VertSkip}{0.5\VertSkipStan}%
  \addtolength{\VertSkip}{\dp\upperbox}%
  \addtolength{\VertSkip}{-1\LineThikStan}% 
%
  \Linie%
  \LinieO{\upperVertLineLength}% 
  \addtolength{\lowerVertLineLength}{-0.5\LineThikStan}%
  \raisebox{\upperVertLineLength}{\KurzeLinie}%
  \addtolength{\lowerVertLineLength}{0.5\LineThikStan}%
  \LinieU{\lowerVertLineLength}%
%
  \renewcommand{\arraystretch}{0}%
  \addtolength{\upperVertLineLength}{-0.5\LineThikStan}%
  \raisebox{\upperVertLineLength}%
   {% 
    \addtolength{\upperVertLineLength}{0.5\LineThikStan}%
    \begin{tabular}[t]%              % Zwei Eintraege
     {%
      @{\hrulefill}%   % Verbindungslinie 1
      c%               % Eintraege Zentrieren
      @{\hrulefill}%   % Verbindungslinie 2
     }%
     \PfeilR\usebox{\upperbox}\Linie\\[\VertSkip]  % oberer Eintrag
     \PfeilL\\[0.5\VertSkipStan]%
     \PfeilR\usebox{\lowerbox}\Linie%              % unterer Eintrag
    \end{tabular}%
   }%
  \renewcommand{\arraystretch}{1}%
  \addtolength{\upperVertLineLength}{0.5\LineThikStan}%
%
  \LinieO{\upperVertLineLength}%
  \addtolength{\lowerVertLineLength}{-0.5\LineThikStan}%
  \raisebox{-\lowerVertLineLength}{\KurzeLinie}%
  \addtolength{\lowerVertLineLength}{0.5\LineThikStan}%
  \LinieU{\lowerVertLineLength}%
  \Linie%
 }%
%
%%%% Die Alternativen-Umgebung
%
\newenvironment{Alternativen}[2]%
 {%
  \renewcommand{\arraystretch}{2}%
  \setlength{\tabulcor}{9.3pt}%
  \sbox{\upperbox}{##1}%
  \sbox{\lowerbox}{##2}%
%
  \begin{lrbox}{\togetherbox}% 
    \begin{tabular}[t]%              % Zwei Eintraege
     {%
      @{\hrulefill}%                 % Verbindungslinie 1
      c%                             % Eintraege Zentrieren
      @{\hrulefill}%                 % Verbindungslinie 2
     }%
     \PfeilR\usebox{\upperbox}\Linie\\             % erster Weg 
 }%
%                                                  % weitere Wege
 {%
     \PfeilR\usebox{\lowerbox}\Linie%              % letzter Weg
    \end{tabular}%
  \end{lrbox}% 
%
  \setlength{\upperVertLineLength}{0.5\dp\togetherbox}%
  \addtolength{\upperVertLineLength}{0.5\ht\togetherbox}%
  \addtolength{\upperVertLineLength}{-\ht\upperbox}%
  \addtolength{\upperVertLineLength}{0.5\LineThikStan}%
  \addtolength{\upperVertLineLength}{-\tabulcor}%
%
  \setlength{\lowerVertLineLength}{0.5\dp\togetherbox}%
  \addtolength{\lowerVertLineLength}{0.5\ht\togetherbox}%
  \addtolength{\lowerVertLineLength}{-\dp\lowerbox}%
%
  \Linie%
  \LinieO{\upperVertLineLength}%
  \LinieU{\lowerVertLineLength}%
% 
  \CenterToBaseLine{\usebox{\togetherbox}}%
%  
  \LinieO{\upperVertLineLength}%
  \LinieU{\lowerVertLineLength}%
  \Linie%
 }%
%
%%%%% Ende Syntaxdiagramm Befehle und Umgebungen
%
  \providecommand{\ContSym}{#1}
  \renewcommand{\ContSym}{#1}
  \begin{lrbox}{\Messbox}%
 }%
 {%
  \end{lrbox}%
  \setlength{\hoehe}{\ht\Messbox}%
  \addtolength{\hoehe}{\VertSkipStan}%
  \setlength{\tiefe}{\dp\Messbox}%
  \addtolength{\tiefe}{\VertSkipStan}%
  \raisebox{0pt}[\hoehe][\tiefe]%
   {%
    \Linie%
    \usebox{\Messbox}%
    \ContSym%
   }%
 }%
%
%%%%% --Und Schluss--

 eingebunden zu werden. Es stellt dann die Umgebung
% syntaxdiagramm zur Verfuegung, mit deren Hilfe Syntaxdiagramme fuer
% z.B. Pascal erzeugt werden koennen. 
%
% syntax.tex benoetigt nur noch folgendes LaTeX Paket:
%   - fancybox.sty
% (Es muss im Vorspann des aufrufenden Dokumentes eingebunden werden.) 
%
% Beispiele zur Verwendung siehe: syntaxtest.tex 
% zur naeheren Dokumentation siehe: syntaxintro.tex
% zur Entwicklung des Codes siehe: syntaxdevel.tex
%
% Bernd Worsch 29.6.97 
%
% 05.07.97  \Leer als vertikale Stuetze
%           groessere Abstaende in Alternativen-Umgebung 
%
% to-do-list:
%   -- Oval fuer SynVar ueber picture erzeugen -> fancybox unnoetig
%   -- Alternativen-Parameter automatisch bestimmen
%   -- Dokumentation (syntaxuse.tex)
%   -- Kommentare im Quelltext
%   -- Als sty-File implementieren



\typeout{Verwende: 'syntax.tex' v0.02 von Bernd Worsch}



%%%% Anlegen der benoetigten Laengen und Boxen 

\newlength{\breite}
\newlength{\hoehe}
\newlength{\tiefe}
\newlength{\gesamthoehe}
\newlength{\tmpLength}

\newsavebox{\Messbox}

\newlength{\LineLengStan}
\newlength{\LineThikStan}
\setlength{\LineLengStan}{4mm}
\setlength{\LineThikStan}{\fboxrule}

\newsavebox{\Centerbox}
\newlength{\Centercor}

\newlength{\VertSkipStan}
\setlength{\VertSkipStan}{3mm}

\newsavebox{\upperbox}
\newsavebox{\lowerbox}
\newsavebox{\togetherbox}
\newlength{\VertLineLength}
\newlength{\VertSkip}

\newlength{\upperVertLineLength}
\newlength{\lowerVertLineLength}

\newlength{\tabulcor}

%%%%% Beginn Syntaxdiagramm Befehle und Umgebungen

\newenvironment{Syntaxdiagramm}[1][\PfeilR]%
 {%
%
%%%% Befehl zum Vermessen von TeX-Strukturen
%
%\newcommand{\Messe}[1]
% {
%  \sbox{\Messbox}{##1}
%
%  \setlength{\breite}{\wd\Messbox}
%  \setlength{\hoehe}{\ht\Messbox}
%  \setlength{\tiefe}{\dp\Messbox}
%  \setlength{\gesamthoehe}{\hoehe}
%  \addtolength{\gesamthoehe}{\tiefe}
%  \setlength{\tmpLength}{\hoehe}
%  \addtolength{\tmpLength}{-\tiefe}
%  \setlength{\tmpLength}{-0.5\tmpLength} 
%
%  \fbox
%   {
%    \begin{tabular}{ll}
%      Breite:      & \the\breite      \\
%      Hoehe:       & \the\hoehe       \\
%      Tiefe:       & \the\tiefe       \\
%      Gesamthoehe: & \the\gesamthoehe \\
%      CenterCor:   & \the\tmpLength
%    \end{tabular}
% 
%    \usebox{\Messbox}
%   }
% }
%
%%%% Befehl zum Justieren von Boxen 
%
\newcommand{\CenterToBaseLine}[1]
  {%
   \sbox{\Centerbox}{##1}%
   \setlength{\Centercor}{\ht\Centerbox}%
   \addtolength{\Centercor}{-\dp\Centerbox}%
   \setlength{\Centercor}{-0.5\Centercor}%
   \addtolength{\Centercor}{0.5\LineThikStan}%
   \raisebox{\Centercor}{\usebox{\Centerbox}}%
  }%
%
%%%% Grundelemente eines Syntaxdiagramms
%
\newcommand{\Leer}%
  {\rule[-1.3ex]{0pt}{2.5ex}}%
%
\newcommand{\Linie}%
  {\rule{\LineLengStan}{\LineThikStan}}%
%
\newcommand{\KurzeLinie}%
  {\rule{0.5\LineLengStan}{\LineThikStan}}%
%
\newcommand{\Tri}
  {%
   \CenterToBaseLine
   {%
     \setlength{\unitlength}{1pt}%
     \begin{picture}(0,0)%
       \put(0,0){\vector(0,1){0}}%
     \end{picture}%
   }%
  }%
%
\newcommand{\TriR}
  {%
   \CenterToBaseLine
    {%
     \setlength{\unitlength}{1pt}%
     \begin{picture}(0,0)%
       \put(0,0){\vector(1,0){0}}%
     \end{picture}%
    }%
  }%
%
\newcommand{\TriD}
  {%
   \CenterToBaseLine
    {%
     \setlength{\unitlength}{1pt}%
     \begin{picture}(0,0)%
       \put(0,0){\vector(0,-1){0}}%
     \end{picture}%
    }%
  }%
%
\newcommand{\TriL}
  {%
   \CenterToBaseLine
    {%
     \setlength{\unitlength}{1pt}%
     \begin{picture}(0,0)%
       \put(0,0){\vector(-1,0){0}}%
     \end{picture}%
    }%
  }%
%
\newcommand{\Dots}
  {%
   \CenterToBaseLine
    {%
     ...%
    }%
  }%
%
\newcommand{\SynVar}[1]%
  {%
   \CenterToBaseLine%
    {%
     \fbox%
      {%
       \rule{0.1em}{0pt}%
       \rule[-0.3ex]{0pt}{2ex}%
       ##1%
       \rule{0.1em}{0pt}%
      }%
    }%
  }%
%
\newcommand{\TerSym}[1]%
  {%
   \CenterToBaseLine%
    {%
     \ovalbox%
      {%
       \rule{0.2em}{0pt}%
       \rule[-0.2ex]{0pt}{1.8ex}%
       ##1%
       \rule{0.2em}{0pt}%
      }%
    }%
  }%
%
%%%% Kombinierte Grafische Elemente
%
\newcommand{\Ellipse}%
 {%
  \Linie\ \Dots\ \Linie%
 }%
%
\newcommand{\PfeilR}%
 {%
  \Linie\TriR%
 }%
%
\newcommand{\PfeilL}%
 {%
  \TriL\Linie%
 }%
%
\newcommand{\SynVarR}[1]%
 {%
  \PfeilR\SynVar{##1}\Linie%
 }%
%
\newcommand{\SynVarL}[1]%
 {%
  \Linie\SynVar{##1}\PfeilL%
 }%
%
\newcommand{\TerSymR}[1]%
 {%
  \PfeilR\TerSym{##1}\Linie%
 }%
%
\newcommand{\TerSymL}[1]%
 {%
  \Linie\TerSym{##1}\PfeilL%
 }%
%
%%%% Vertikale grafische Elemente
%
\newcommand{\LinieO}[1]%
 {%
  \CenterToBaseLine%
   {%
    \raisebox{0pt}[0pt][0pt]%
     {%
      \makebox[0mm][c]{\rule{\LineThikStan}{##1}}%
     }%
   }%
 }%
%
\newcommand{\LinieU}[1]%
 {%
  \CenterToBaseLine%
   {%
    \raisebox{0pt}[0pt][0pt]%
     {%
      \makebox[0mm][c]{\rule[-##1]{\LineThikStan}{##1}}%
     }%
   }%
 }%
%
\newcommand{\PfeilO}[1]%
 {%
  \CenterToBaseLine%
   {%
    \raisebox{0pt}[0pt][0pt]%
     {%
      \makebox[0mm][c]{\TriD}%
      \makebox[0mm][c]{\rule{\LineThikStan}{##1}}%
     }%
   }%
 }%
%
\newcommand{\PfeilU}[1]%
 {%
  \CenterToBaseLine%
   {%
    \raisebox{0pt}[0pt][0pt]%
     {%
      \makebox[0mm][c]{\Tri}%
      \makebox[0mm][c]{\rule[-##1]{\LineThikStan}{##1}}%
     }%
   }%
 }%
%
%%%% Strukturen
%
%%% Wiederholung
%
\newcommand{\Wiederholung}[2]%
 {%
  \sbox{\upperbox}{##1}%
  \sbox{\lowerbox}{##2}%
  \setlength{\VertLineLength}{\dp\upperbox}%
  \addtolength{\VertLineLength}{\ht\lowerbox}%
  \addtolength{\VertLineLength}{\VertSkipStan}%
  \addtolength{\VertLineLength}{0.5\LineThikStan}%
  \setlength{\VertSkip}{\VertSkipStan}%
  \addtolength{\VertSkip}{\dp\upperbox}%
%
  \Linie%
  \PfeilU{\VertLineLength}%
% 
  \renewcommand{\arraystretch}{0}%
  \begin{tabular}[t]%              % Zwei Eintraege
   {%
    @{\hrulefill}%                 % Verbindungslinie 1
    c%                             % Eintraege Zentrieren
    @{\hrulefill}%                 % Verbindungslinie 2
   }%
   \PfeilR\usebox{\upperbox}\Linie\\[\VertSkip]  % Dies wiederholen
   \Linie\usebox{\lowerbox}\PfeilL%              % Dies in Gegenrichtung
  \end{tabular}%
%
  \LinieU{\VertLineLength}%
  \Linie%
 }%
%
%%% Alternative
%
\newcommand{\Alternative}[2]%
 {%
  \sbox{\upperbox}{##1}%
  \sbox{\lowerbox}{##2}%
%
  \setlength{\upperVertLineLength}{\dp\upperbox}%
  \addtolength{\upperVertLineLength}{0.5\VertSkipStan}%
  \addtolength{\upperVertLineLength}{0.5\LineThikStan}%
%
  \setlength{\lowerVertLineLength}{\ht\lowerbox}%
  \addtolength{\lowerVertLineLength}{0.5\VertSkipStan}%
  \addtolength{\lowerVertLineLength}{0.5\LineThikStan}%
%
  \setlength{\VertSkip}{\VertSkipStan}%
  \addtolength{\VertSkip}{\dp\upperbox}%
%
  \Linie%
  \LinieO{\upperVertLineLength}%
  \LinieU{\lowerVertLineLength}%
% 
  \renewcommand{\arraystretch}{0}%
  \addtolength{\upperVertLineLength}{-0.5\LineThikStan}%
  \raisebox{\upperVertLineLength}%
   {% 
    \begin{tabular}[t]%                 % Zwei Eintraege
     {%
      @{\hrulefill}%                 % Verbindungslinie 1
      c%                             % Eintraege Zentrieren
      @{\hrulefill}%                 % Verbindungslinie 2
     }%
     \PfeilR\usebox{\upperbox}\Linie\\[\VertSkip]  % Weg1
     \PfeilR\usebox{\lowerbox}\Linie%              % Weg2
    \end{tabular}%
   }%
  \renewcommand{\arraystretch}{1}%
  \addtolength{\upperVertLineLength}{0.5\LineThikStan}%
%
  \LinieO{\upperVertLineLength}%
  \LinieU{\lowerVertLineLength}%
  \Linie%
 }%
%
%%% Stapel
%
\newcommand{\Stapel}[2]%
 {%
  \sbox{\upperbox}{##1}%
  \sbox{\lowerbox}{##2}%
%
  \setlength{\upperVertLineLength}{\dp\upperbox}%
  \addtolength{\upperVertLineLength}{0.5\VertSkipStan}%
  \addtolength{\upperVertLineLength}{0.5\LineThikStan}%
%
  \setlength{\lowerVertLineLength}{\ht\lowerbox}%
  \addtolength{\lowerVertLineLength}{0.5\VertSkipStan}%
  \addtolength{\lowerVertLineLength}{0.5\LineThikStan}%
%
  \setlength{\VertSkip}{0.5\VertSkipStan}%
  \addtolength{\VertSkip}{\dp\upperbox}%
  \addtolength{\VertSkip}{-1\LineThikStan}% 
%
  \Linie%
  \LinieO{\upperVertLineLength}% 
  \addtolength{\lowerVertLineLength}{-0.5\LineThikStan}%
  \raisebox{\upperVertLineLength}{\KurzeLinie}%
  \addtolength{\lowerVertLineLength}{0.5\LineThikStan}%
  \LinieU{\lowerVertLineLength}%
%
  \renewcommand{\arraystretch}{0}%
  \addtolength{\upperVertLineLength}{-0.5\LineThikStan}%
  \raisebox{\upperVertLineLength}%
   {% 
    \addtolength{\upperVertLineLength}{0.5\LineThikStan}%
    \begin{tabular}[t]%              % Zwei Eintraege
     {%
      @{\hrulefill}%   % Verbindungslinie 1
      c%               % Eintraege Zentrieren
      @{\hrulefill}%   % Verbindungslinie 2
     }%
     \PfeilR\usebox{\upperbox}\Linie\\[\VertSkip]  % oberer Eintrag
     \PfeilL\\[0.5\VertSkipStan]%
     \PfeilR\usebox{\lowerbox}\Linie%              % unterer Eintrag
    \end{tabular}%
   }%
  \renewcommand{\arraystretch}{1}%
  \addtolength{\upperVertLineLength}{0.5\LineThikStan}%
%
  \LinieO{\upperVertLineLength}%
  \addtolength{\lowerVertLineLength}{-0.5\LineThikStan}%
  \raisebox{-\lowerVertLineLength}{\KurzeLinie}%
  \addtolength{\lowerVertLineLength}{0.5\LineThikStan}%
  \LinieU{\lowerVertLineLength}%
  \Linie%
 }%
%
%%%% Die Alternativen-Umgebung
%
\newenvironment{Alternativen}[2]%
 {%
  \renewcommand{\arraystretch}{2}%
  \setlength{\tabulcor}{9.3pt}%
  \sbox{\upperbox}{##1}%
  \sbox{\lowerbox}{##2}%
%
  \begin{lrbox}{\togetherbox}% 
    \begin{tabular}[t]%              % Zwei Eintraege
     {%
      @{\hrulefill}%                 % Verbindungslinie 1
      c%                             % Eintraege Zentrieren
      @{\hrulefill}%                 % Verbindungslinie 2
     }%
     \PfeilR\usebox{\upperbox}\Linie\\             % erster Weg 
 }%
%                                                  % weitere Wege
 {%
     \PfeilR\usebox{\lowerbox}\Linie%              % letzter Weg
    \end{tabular}%
  \end{lrbox}% 
%
  \setlength{\upperVertLineLength}{0.5\dp\togetherbox}%
  \addtolength{\upperVertLineLength}{0.5\ht\togetherbox}%
  \addtolength{\upperVertLineLength}{-\ht\upperbox}%
  \addtolength{\upperVertLineLength}{0.5\LineThikStan}%
  \addtolength{\upperVertLineLength}{-\tabulcor}%
%
  \setlength{\lowerVertLineLength}{0.5\dp\togetherbox}%
  \addtolength{\lowerVertLineLength}{0.5\ht\togetherbox}%
  \addtolength{\lowerVertLineLength}{-\dp\lowerbox}%
%
  \Linie%
  \LinieO{\upperVertLineLength}%
  \LinieU{\lowerVertLineLength}%
% 
  \CenterToBaseLine{\usebox{\togetherbox}}%
%  
  \LinieO{\upperVertLineLength}%
  \LinieU{\lowerVertLineLength}%
  \Linie%
 }%
%
%%%%% Ende Syntaxdiagramm Befehle und Umgebungen
%
  \providecommand{\ContSym}{#1}
  \renewcommand{\ContSym}{#1}
  \begin{lrbox}{\Messbox}%
 }%
 {%
  \end{lrbox}%
  \setlength{\hoehe}{\ht\Messbox}%
  \addtolength{\hoehe}{\VertSkipStan}%
  \setlength{\tiefe}{\dp\Messbox}%
  \addtolength{\tiefe}{\VertSkipStan}%
  \raisebox{0pt}[\hoehe][\tiefe]%
   {%
    \Linie%
    \usebox{\Messbox}%
    \ContSym%
   }%
 }%
%
%%%%% --Und Schluss--

 eingebunden zu werden. Es stellt dann die Umgebung
% syntaxdiagramm zur Verfuegung, mit deren Hilfe Syntaxdiagramme fuer
% z.B. Pascal erzeugt werden koennen. 
%
% syntax.tex benoetigt nur noch folgendes LaTeX Paket:
%   - fancybox.sty
% (Es muss im Vorspann des aufrufenden Dokumentes eingebunden werden.) 
%
% Beispiele zur Verwendung siehe: syntaxtest.tex 
% zur naeheren Dokumentation siehe: syntaxintro.tex
% zur Entwicklung des Codes siehe: syntaxdevel.tex
%
% Bernd Worsch 29.6.97 
%
% 05.07.97  \Leer als vertikale Stuetze
%           groessere Abstaende in Alternativen-Umgebung 
%
% to-do-list:
%   -- Oval fuer SynVar ueber picture erzeugen -> fancybox unnoetig
%   -- Alternativen-Parameter automatisch bestimmen
%   -- Dokumentation (syntaxuse.tex)
%   -- Kommentare im Quelltext
%   -- Als sty-File implementieren



\typeout{Verwende: 'syntax.tex' v0.02 von Bernd Worsch}



%%%% Anlegen der benoetigten Laengen und Boxen 

\newlength{\breite}
\newlength{\hoehe}
\newlength{\tiefe}
\newlength{\gesamthoehe}
\newlength{\tmpLength}

\newsavebox{\Messbox}

\newlength{\LineLengStan}
\newlength{\LineThikStan}
\setlength{\LineLengStan}{4mm}
\setlength{\LineThikStan}{\fboxrule}

\newsavebox{\Centerbox}
\newlength{\Centercor}

\newlength{\VertSkipStan}
\setlength{\VertSkipStan}{3mm}

\newsavebox{\upperbox}
\newsavebox{\lowerbox}
\newsavebox{\togetherbox}
\newlength{\VertLineLength}
\newlength{\VertSkip}

\newlength{\upperVertLineLength}
\newlength{\lowerVertLineLength}

\newlength{\tabulcor}

%%%%% Beginn Syntaxdiagramm Befehle und Umgebungen

\newenvironment{Syntaxdiagramm}[1][\PfeilR]%
 {%
%
%%%% Befehl zum Vermessen von TeX-Strukturen
%
%\newcommand{\Messe}[1]
% {
%  \sbox{\Messbox}{##1}
%
%  \setlength{\breite}{\wd\Messbox}
%  \setlength{\hoehe}{\ht\Messbox}
%  \setlength{\tiefe}{\dp\Messbox}
%  \setlength{\gesamthoehe}{\hoehe}
%  \addtolength{\gesamthoehe}{\tiefe}
%  \setlength{\tmpLength}{\hoehe}
%  \addtolength{\tmpLength}{-\tiefe}
%  \setlength{\tmpLength}{-0.5\tmpLength} 
%
%  \fbox
%   {
%    \begin{tabular}{ll}
%      Breite:      & \the\breite      \\
%      Hoehe:       & \the\hoehe       \\
%      Tiefe:       & \the\tiefe       \\
%      Gesamthoehe: & \the\gesamthoehe \\
%      CenterCor:   & \the\tmpLength
%    \end{tabular}
% 
%    \usebox{\Messbox}
%   }
% }
%
%%%% Befehl zum Justieren von Boxen 
%
\newcommand{\CenterToBaseLine}[1]
  {%
   \sbox{\Centerbox}{##1}%
   \setlength{\Centercor}{\ht\Centerbox}%
   \addtolength{\Centercor}{-\dp\Centerbox}%
   \setlength{\Centercor}{-0.5\Centercor}%
   \addtolength{\Centercor}{0.5\LineThikStan}%
   \raisebox{\Centercor}{\usebox{\Centerbox}}%
  }%
%
%%%% Grundelemente eines Syntaxdiagramms
%
\newcommand{\Leer}%
  {\rule[-1.3ex]{0pt}{2.5ex}}%
%
\newcommand{\Linie}%
  {\rule{\LineLengStan}{\LineThikStan}}%
%
\newcommand{\KurzeLinie}%
  {\rule{0.5\LineLengStan}{\LineThikStan}}%
%
\newcommand{\Tri}
  {%
   \CenterToBaseLine
   {%
     \setlength{\unitlength}{1pt}%
     \begin{picture}(0,0)%
       \put(0,0){\vector(0,1){0}}%
     \end{picture}%
   }%
  }%
%
\newcommand{\TriR}
  {%
   \CenterToBaseLine
    {%
     \setlength{\unitlength}{1pt}%
     \begin{picture}(0,0)%
       \put(0,0){\vector(1,0){0}}%
     \end{picture}%
    }%
  }%
%
\newcommand{\TriD}
  {%
   \CenterToBaseLine
    {%
     \setlength{\unitlength}{1pt}%
     \begin{picture}(0,0)%
       \put(0,0){\vector(0,-1){0}}%
     \end{picture}%
    }%
  }%
%
\newcommand{\TriL}
  {%
   \CenterToBaseLine
    {%
     \setlength{\unitlength}{1pt}%
     \begin{picture}(0,0)%
       \put(0,0){\vector(-1,0){0}}%
     \end{picture}%
    }%
  }%
%
\newcommand{\Dots}
  {%
   \CenterToBaseLine
    {%
     ...%
    }%
  }%
%
\newcommand{\SynVar}[1]%
  {%
   \CenterToBaseLine%
    {%
     \fbox%
      {%
       \rule{0.1em}{0pt}%
       \rule[-0.3ex]{0pt}{2ex}%
       ##1%
       \rule{0.1em}{0pt}%
      }%
    }%
  }%
%
\newcommand{\TerSym}[1]%
  {%
   \CenterToBaseLine%
    {%
     \ovalbox%
      {%
       \rule{0.2em}{0pt}%
       \rule[-0.2ex]{0pt}{1.8ex}%
       ##1%
       \rule{0.2em}{0pt}%
      }%
    }%
  }%
%
%%%% Kombinierte Grafische Elemente
%
\newcommand{\Ellipse}%
 {%
  \Linie\ \Dots\ \Linie%
 }%
%
\newcommand{\PfeilR}%
 {%
  \Linie\TriR%
 }%
%
\newcommand{\PfeilL}%
 {%
  \TriL\Linie%
 }%
%
\newcommand{\SynVarR}[1]%
 {%
  \PfeilR\SynVar{##1}\Linie%
 }%
%
\newcommand{\SynVarL}[1]%
 {%
  \Linie\SynVar{##1}\PfeilL%
 }%
%
\newcommand{\TerSymR}[1]%
 {%
  \PfeilR\TerSym{##1}\Linie%
 }%
%
\newcommand{\TerSymL}[1]%
 {%
  \Linie\TerSym{##1}\PfeilL%
 }%
%
%%%% Vertikale grafische Elemente
%
\newcommand{\LinieO}[1]%
 {%
  \CenterToBaseLine%
   {%
    \raisebox{0pt}[0pt][0pt]%
     {%
      \makebox[0mm][c]{\rule{\LineThikStan}{##1}}%
     }%
   }%
 }%
%
\newcommand{\LinieU}[1]%
 {%
  \CenterToBaseLine%
   {%
    \raisebox{0pt}[0pt][0pt]%
     {%
      \makebox[0mm][c]{\rule[-##1]{\LineThikStan}{##1}}%
     }%
   }%
 }%
%
\newcommand{\PfeilO}[1]%
 {%
  \CenterToBaseLine%
   {%
    \raisebox{0pt}[0pt][0pt]%
     {%
      \makebox[0mm][c]{\TriD}%
      \makebox[0mm][c]{\rule{\LineThikStan}{##1}}%
     }%
   }%
 }%
%
\newcommand{\PfeilU}[1]%
 {%
  \CenterToBaseLine%
   {%
    \raisebox{0pt}[0pt][0pt]%
     {%
      \makebox[0mm][c]{\Tri}%
      \makebox[0mm][c]{\rule[-##1]{\LineThikStan}{##1}}%
     }%
   }%
 }%
%
%%%% Strukturen
%
%%% Wiederholung
%
\newcommand{\Wiederholung}[2]%
 {%
  \sbox{\upperbox}{##1}%
  \sbox{\lowerbox}{##2}%
  \setlength{\VertLineLength}{\dp\upperbox}%
  \addtolength{\VertLineLength}{\ht\lowerbox}%
  \addtolength{\VertLineLength}{\VertSkipStan}%
  \addtolength{\VertLineLength}{0.5\LineThikStan}%
  \setlength{\VertSkip}{\VertSkipStan}%
  \addtolength{\VertSkip}{\dp\upperbox}%
%
  \Linie%
  \PfeilU{\VertLineLength}%
% 
  \renewcommand{\arraystretch}{0}%
  \begin{tabular}[t]%              % Zwei Eintraege
   {%
    @{\hrulefill}%                 % Verbindungslinie 1
    c%                             % Eintraege Zentrieren
    @{\hrulefill}%                 % Verbindungslinie 2
   }%
   \PfeilR\usebox{\upperbox}\Linie\\[\VertSkip]  % Dies wiederholen
   \Linie\usebox{\lowerbox}\PfeilL%              % Dies in Gegenrichtung
  \end{tabular}%
%
  \LinieU{\VertLineLength}%
  \Linie%
 }%
%
%%% Alternative
%
\newcommand{\Alternative}[2]%
 {%
  \sbox{\upperbox}{##1}%
  \sbox{\lowerbox}{##2}%
%
  \setlength{\upperVertLineLength}{\dp\upperbox}%
  \addtolength{\upperVertLineLength}{0.5\VertSkipStan}%
  \addtolength{\upperVertLineLength}{0.5\LineThikStan}%
%
  \setlength{\lowerVertLineLength}{\ht\lowerbox}%
  \addtolength{\lowerVertLineLength}{0.5\VertSkipStan}%
  \addtolength{\lowerVertLineLength}{0.5\LineThikStan}%
%
  \setlength{\VertSkip}{\VertSkipStan}%
  \addtolength{\VertSkip}{\dp\upperbox}%
%
  \Linie%
  \LinieO{\upperVertLineLength}%
  \LinieU{\lowerVertLineLength}%
% 
  \renewcommand{\arraystretch}{0}%
  \addtolength{\upperVertLineLength}{-0.5\LineThikStan}%
  \raisebox{\upperVertLineLength}%
   {% 
    \begin{tabular}[t]%                 % Zwei Eintraege
     {%
      @{\hrulefill}%                 % Verbindungslinie 1
      c%                             % Eintraege Zentrieren
      @{\hrulefill}%                 % Verbindungslinie 2
     }%
     \PfeilR\usebox{\upperbox}\Linie\\[\VertSkip]  % Weg1
     \PfeilR\usebox{\lowerbox}\Linie%              % Weg2
    \end{tabular}%
   }%
  \renewcommand{\arraystretch}{1}%
  \addtolength{\upperVertLineLength}{0.5\LineThikStan}%
%
  \LinieO{\upperVertLineLength}%
  \LinieU{\lowerVertLineLength}%
  \Linie%
 }%
%
%%% Stapel
%
\newcommand{\Stapel}[2]%
 {%
  \sbox{\upperbox}{##1}%
  \sbox{\lowerbox}{##2}%
%
  \setlength{\upperVertLineLength}{\dp\upperbox}%
  \addtolength{\upperVertLineLength}{0.5\VertSkipStan}%
  \addtolength{\upperVertLineLength}{0.5\LineThikStan}%
%
  \setlength{\lowerVertLineLength}{\ht\lowerbox}%
  \addtolength{\lowerVertLineLength}{0.5\VertSkipStan}%
  \addtolength{\lowerVertLineLength}{0.5\LineThikStan}%
%
  \setlength{\VertSkip}{0.5\VertSkipStan}%
  \addtolength{\VertSkip}{\dp\upperbox}%
  \addtolength{\VertSkip}{-1\LineThikStan}% 
%
  \Linie%
  \LinieO{\upperVertLineLength}% 
  \addtolength{\lowerVertLineLength}{-0.5\LineThikStan}%
  \raisebox{\upperVertLineLength}{\KurzeLinie}%
  \addtolength{\lowerVertLineLength}{0.5\LineThikStan}%
  \LinieU{\lowerVertLineLength}%
%
  \renewcommand{\arraystretch}{0}%
  \addtolength{\upperVertLineLength}{-0.5\LineThikStan}%
  \raisebox{\upperVertLineLength}%
   {% 
    \addtolength{\upperVertLineLength}{0.5\LineThikStan}%
    \begin{tabular}[t]%              % Zwei Eintraege
     {%
      @{\hrulefill}%   % Verbindungslinie 1
      c%               % Eintraege Zentrieren
      @{\hrulefill}%   % Verbindungslinie 2
     }%
     \PfeilR\usebox{\upperbox}\Linie\\[\VertSkip]  % oberer Eintrag
     \PfeilL\\[0.5\VertSkipStan]%
     \PfeilR\usebox{\lowerbox}\Linie%              % unterer Eintrag
    \end{tabular}%
   }%
  \renewcommand{\arraystretch}{1}%
  \addtolength{\upperVertLineLength}{0.5\LineThikStan}%
%
  \LinieO{\upperVertLineLength}%
  \addtolength{\lowerVertLineLength}{-0.5\LineThikStan}%
  \raisebox{-\lowerVertLineLength}{\KurzeLinie}%
  \addtolength{\lowerVertLineLength}{0.5\LineThikStan}%
  \LinieU{\lowerVertLineLength}%
  \Linie%
 }%
%
%%%% Die Alternativen-Umgebung
%
\newenvironment{Alternativen}[2]%
 {%
  \renewcommand{\arraystretch}{2}%
  \setlength{\tabulcor}{9.3pt}%
  \sbox{\upperbox}{##1}%
  \sbox{\lowerbox}{##2}%
%
  \begin{lrbox}{\togetherbox}% 
    \begin{tabular}[t]%              % Zwei Eintraege
     {%
      @{\hrulefill}%                 % Verbindungslinie 1
      c%                             % Eintraege Zentrieren
      @{\hrulefill}%                 % Verbindungslinie 2
     }%
     \PfeilR\usebox{\upperbox}\Linie\\             % erster Weg 
 }%
%                                                  % weitere Wege
 {%
     \PfeilR\usebox{\lowerbox}\Linie%              % letzter Weg
    \end{tabular}%
  \end{lrbox}% 
%
  \setlength{\upperVertLineLength}{0.5\dp\togetherbox}%
  \addtolength{\upperVertLineLength}{0.5\ht\togetherbox}%
  \addtolength{\upperVertLineLength}{-\ht\upperbox}%
  \addtolength{\upperVertLineLength}{0.5\LineThikStan}%
  \addtolength{\upperVertLineLength}{-\tabulcor}%
%
  \setlength{\lowerVertLineLength}{0.5\dp\togetherbox}%
  \addtolength{\lowerVertLineLength}{0.5\ht\togetherbox}%
  \addtolength{\lowerVertLineLength}{-\dp\lowerbox}%
%
  \Linie%
  \LinieO{\upperVertLineLength}%
  \LinieU{\lowerVertLineLength}%
% 
  \CenterToBaseLine{\usebox{\togetherbox}}%
%  
  \LinieO{\upperVertLineLength}%
  \LinieU{\lowerVertLineLength}%
  \Linie%
 }%
%
%%%%% Ende Syntaxdiagramm Befehle und Umgebungen
%
  \providecommand{\ContSym}{#1}
  \renewcommand{\ContSym}{#1}
  \begin{lrbox}{\Messbox}%
 }%
 {%
  \end{lrbox}%
  \setlength{\hoehe}{\ht\Messbox}%
  \addtolength{\hoehe}{\VertSkipStan}%
  \setlength{\tiefe}{\dp\Messbox}%
  \addtolength{\tiefe}{\VertSkipStan}%
  \raisebox{0pt}[\hoehe][\tiefe]%
   {%
    \Linie%
    \usebox{\Messbox}%
    \ContSym%
   }%
 }%
%
%%%%% --Und Schluss--


%Rundung fuer ovalbox einstellen
\cornersize{1}

\title{\syn \\Eine "Ubersicht}
\author{Bernd Worsch}
\date{7. Juli 1997}

\newcommand{\syn}{syntax.tex}
\newcommand{\vers}{0.02}
\newcommand{\dx}{\rule{2mm}{0pt}}

\begin{document}

\maketitle

\tableofcontents

\section{Einleitung}
Zur Beschreibung von Zusammenh"angen in formal sprachlichen Systemen, 
findet h"aufig die sogenannte Bachus--Naur--Form kurz BNF Verwendung.
Von dieser (textlichen) Darstellung ausgehend wird gerade im Zusammenhang
mit h"oheren Programmiersprachen gerne eine grafische Beschreibung zur
Verdeutlichung herangezogen.
Solche \emph{Syntaxdiagramme} sind intuitiv verst"andlich und verdeutlichen
auf anschauliche Weise den formalen Aufbau der jeweiligen Programmiersprache.

Das File \syn\ erm"oglicht das einfache Erstellen solcher Diagramme in
\LaTeX -Dokumenten.
Statt mit geometrischen Objekten (Linien, Kreise, etc.) zu arbeiten, 
nutzt der Anwender dazu Befehle, die schon im Quelldokument die zu 
illustrierenden Kontrollstrukturen wiederspiegeln.  
 
\section{Bevor es funktioniert \dots}
Um die M"oglichkeiten von \syn\ nutzen zu k"onnen mu"s die Datei vor dem
ersten zu setzenden Syntaxdiagramm durch 
\verb+% Dieses File ist dazu gedacht, im Vorspann von LaTeX Dokumenten vermittels
% % Dieses File ist dazu gedacht, im Vorspann von LaTeX Dokumenten vermittels
% % Dieses File ist dazu gedacht, im Vorspann von LaTeX Dokumenten vermittels
% \include{syntax.tex} eingebunden zu werden. Es stellt dann die Umgebung
% syntaxdiagramm zur Verfuegung, mit deren Hilfe Syntaxdiagramme fuer
% z.B. Pascal erzeugt werden koennen. 
%
% syntax.tex benoetigt nur noch folgendes LaTeX Paket:
%   - fancybox.sty
% (Es muss im Vorspann des aufrufenden Dokumentes eingebunden werden.) 
%
% Beispiele zur Verwendung siehe: syntaxtest.tex 
% zur naeheren Dokumentation siehe: syntaxintro.tex
% zur Entwicklung des Codes siehe: syntaxdevel.tex
%
% Bernd Worsch 29.6.97 
%
% 05.07.97  \Leer als vertikale Stuetze
%           groessere Abstaende in Alternativen-Umgebung 
%
% to-do-list:
%   -- Oval fuer SynVar ueber picture erzeugen -> fancybox unnoetig
%   -- Alternativen-Parameter automatisch bestimmen
%   -- Dokumentation (syntaxuse.tex)
%   -- Kommentare im Quelltext
%   -- Als sty-File implementieren



\typeout{Verwende: 'syntax.tex' v0.02 von Bernd Worsch}



%%%% Anlegen der benoetigten Laengen und Boxen 

\newlength{\breite}
\newlength{\hoehe}
\newlength{\tiefe}
\newlength{\gesamthoehe}
\newlength{\tmpLength}

\newsavebox{\Messbox}

\newlength{\LineLengStan}
\newlength{\LineThikStan}
\setlength{\LineLengStan}{4mm}
\setlength{\LineThikStan}{\fboxrule}

\newsavebox{\Centerbox}
\newlength{\Centercor}

\newlength{\VertSkipStan}
\setlength{\VertSkipStan}{3mm}

\newsavebox{\upperbox}
\newsavebox{\lowerbox}
\newsavebox{\togetherbox}
\newlength{\VertLineLength}
\newlength{\VertSkip}

\newlength{\upperVertLineLength}
\newlength{\lowerVertLineLength}

\newlength{\tabulcor}

%%%%% Beginn Syntaxdiagramm Befehle und Umgebungen

\newenvironment{Syntaxdiagramm}[1][\PfeilR]%
 {%
%
%%%% Befehl zum Vermessen von TeX-Strukturen
%
%\newcommand{\Messe}[1]
% {
%  \sbox{\Messbox}{##1}
%
%  \setlength{\breite}{\wd\Messbox}
%  \setlength{\hoehe}{\ht\Messbox}
%  \setlength{\tiefe}{\dp\Messbox}
%  \setlength{\gesamthoehe}{\hoehe}
%  \addtolength{\gesamthoehe}{\tiefe}
%  \setlength{\tmpLength}{\hoehe}
%  \addtolength{\tmpLength}{-\tiefe}
%  \setlength{\tmpLength}{-0.5\tmpLength} 
%
%  \fbox
%   {
%    \begin{tabular}{ll}
%      Breite:      & \the\breite      \\
%      Hoehe:       & \the\hoehe       \\
%      Tiefe:       & \the\tiefe       \\
%      Gesamthoehe: & \the\gesamthoehe \\
%      CenterCor:   & \the\tmpLength
%    \end{tabular}
% 
%    \usebox{\Messbox}
%   }
% }
%
%%%% Befehl zum Justieren von Boxen 
%
\newcommand{\CenterToBaseLine}[1]
  {%
   \sbox{\Centerbox}{##1}%
   \setlength{\Centercor}{\ht\Centerbox}%
   \addtolength{\Centercor}{-\dp\Centerbox}%
   \setlength{\Centercor}{-0.5\Centercor}%
   \addtolength{\Centercor}{0.5\LineThikStan}%
   \raisebox{\Centercor}{\usebox{\Centerbox}}%
  }%
%
%%%% Grundelemente eines Syntaxdiagramms
%
\newcommand{\Leer}%
  {\rule[-1.3ex]{0pt}{2.5ex}}%
%
\newcommand{\Linie}%
  {\rule{\LineLengStan}{\LineThikStan}}%
%
\newcommand{\KurzeLinie}%
  {\rule{0.5\LineLengStan}{\LineThikStan}}%
%
\newcommand{\Tri}
  {%
   \CenterToBaseLine
   {%
     \setlength{\unitlength}{1pt}%
     \begin{picture}(0,0)%
       \put(0,0){\vector(0,1){0}}%
     \end{picture}%
   }%
  }%
%
\newcommand{\TriR}
  {%
   \CenterToBaseLine
    {%
     \setlength{\unitlength}{1pt}%
     \begin{picture}(0,0)%
       \put(0,0){\vector(1,0){0}}%
     \end{picture}%
    }%
  }%
%
\newcommand{\TriD}
  {%
   \CenterToBaseLine
    {%
     \setlength{\unitlength}{1pt}%
     \begin{picture}(0,0)%
       \put(0,0){\vector(0,-1){0}}%
     \end{picture}%
    }%
  }%
%
\newcommand{\TriL}
  {%
   \CenterToBaseLine
    {%
     \setlength{\unitlength}{1pt}%
     \begin{picture}(0,0)%
       \put(0,0){\vector(-1,0){0}}%
     \end{picture}%
    }%
  }%
%
\newcommand{\Dots}
  {%
   \CenterToBaseLine
    {%
     ...%
    }%
  }%
%
\newcommand{\SynVar}[1]%
  {%
   \CenterToBaseLine%
    {%
     \fbox%
      {%
       \rule{0.1em}{0pt}%
       \rule[-0.3ex]{0pt}{2ex}%
       ##1%
       \rule{0.1em}{0pt}%
      }%
    }%
  }%
%
\newcommand{\TerSym}[1]%
  {%
   \CenterToBaseLine%
    {%
     \ovalbox%
      {%
       \rule{0.2em}{0pt}%
       \rule[-0.2ex]{0pt}{1.8ex}%
       ##1%
       \rule{0.2em}{0pt}%
      }%
    }%
  }%
%
%%%% Kombinierte Grafische Elemente
%
\newcommand{\Ellipse}%
 {%
  \Linie\ \Dots\ \Linie%
 }%
%
\newcommand{\PfeilR}%
 {%
  \Linie\TriR%
 }%
%
\newcommand{\PfeilL}%
 {%
  \TriL\Linie%
 }%
%
\newcommand{\SynVarR}[1]%
 {%
  \PfeilR\SynVar{##1}\Linie%
 }%
%
\newcommand{\SynVarL}[1]%
 {%
  \Linie\SynVar{##1}\PfeilL%
 }%
%
\newcommand{\TerSymR}[1]%
 {%
  \PfeilR\TerSym{##1}\Linie%
 }%
%
\newcommand{\TerSymL}[1]%
 {%
  \Linie\TerSym{##1}\PfeilL%
 }%
%
%%%% Vertikale grafische Elemente
%
\newcommand{\LinieO}[1]%
 {%
  \CenterToBaseLine%
   {%
    \raisebox{0pt}[0pt][0pt]%
     {%
      \makebox[0mm][c]{\rule{\LineThikStan}{##1}}%
     }%
   }%
 }%
%
\newcommand{\LinieU}[1]%
 {%
  \CenterToBaseLine%
   {%
    \raisebox{0pt}[0pt][0pt]%
     {%
      \makebox[0mm][c]{\rule[-##1]{\LineThikStan}{##1}}%
     }%
   }%
 }%
%
\newcommand{\PfeilO}[1]%
 {%
  \CenterToBaseLine%
   {%
    \raisebox{0pt}[0pt][0pt]%
     {%
      \makebox[0mm][c]{\TriD}%
      \makebox[0mm][c]{\rule{\LineThikStan}{##1}}%
     }%
   }%
 }%
%
\newcommand{\PfeilU}[1]%
 {%
  \CenterToBaseLine%
   {%
    \raisebox{0pt}[0pt][0pt]%
     {%
      \makebox[0mm][c]{\Tri}%
      \makebox[0mm][c]{\rule[-##1]{\LineThikStan}{##1}}%
     }%
   }%
 }%
%
%%%% Strukturen
%
%%% Wiederholung
%
\newcommand{\Wiederholung}[2]%
 {%
  \sbox{\upperbox}{##1}%
  \sbox{\lowerbox}{##2}%
  \setlength{\VertLineLength}{\dp\upperbox}%
  \addtolength{\VertLineLength}{\ht\lowerbox}%
  \addtolength{\VertLineLength}{\VertSkipStan}%
  \addtolength{\VertLineLength}{0.5\LineThikStan}%
  \setlength{\VertSkip}{\VertSkipStan}%
  \addtolength{\VertSkip}{\dp\upperbox}%
%
  \Linie%
  \PfeilU{\VertLineLength}%
% 
  \renewcommand{\arraystretch}{0}%
  \begin{tabular}[t]%              % Zwei Eintraege
   {%
    @{\hrulefill}%                 % Verbindungslinie 1
    c%                             % Eintraege Zentrieren
    @{\hrulefill}%                 % Verbindungslinie 2
   }%
   \PfeilR\usebox{\upperbox}\Linie\\[\VertSkip]  % Dies wiederholen
   \Linie\usebox{\lowerbox}\PfeilL%              % Dies in Gegenrichtung
  \end{tabular}%
%
  \LinieU{\VertLineLength}%
  \Linie%
 }%
%
%%% Alternative
%
\newcommand{\Alternative}[2]%
 {%
  \sbox{\upperbox}{##1}%
  \sbox{\lowerbox}{##2}%
%
  \setlength{\upperVertLineLength}{\dp\upperbox}%
  \addtolength{\upperVertLineLength}{0.5\VertSkipStan}%
  \addtolength{\upperVertLineLength}{0.5\LineThikStan}%
%
  \setlength{\lowerVertLineLength}{\ht\lowerbox}%
  \addtolength{\lowerVertLineLength}{0.5\VertSkipStan}%
  \addtolength{\lowerVertLineLength}{0.5\LineThikStan}%
%
  \setlength{\VertSkip}{\VertSkipStan}%
  \addtolength{\VertSkip}{\dp\upperbox}%
%
  \Linie%
  \LinieO{\upperVertLineLength}%
  \LinieU{\lowerVertLineLength}%
% 
  \renewcommand{\arraystretch}{0}%
  \addtolength{\upperVertLineLength}{-0.5\LineThikStan}%
  \raisebox{\upperVertLineLength}%
   {% 
    \begin{tabular}[t]%                 % Zwei Eintraege
     {%
      @{\hrulefill}%                 % Verbindungslinie 1
      c%                             % Eintraege Zentrieren
      @{\hrulefill}%                 % Verbindungslinie 2
     }%
     \PfeilR\usebox{\upperbox}\Linie\\[\VertSkip]  % Weg1
     \PfeilR\usebox{\lowerbox}\Linie%              % Weg2
    \end{tabular}%
   }%
  \renewcommand{\arraystretch}{1}%
  \addtolength{\upperVertLineLength}{0.5\LineThikStan}%
%
  \LinieO{\upperVertLineLength}%
  \LinieU{\lowerVertLineLength}%
  \Linie%
 }%
%
%%% Stapel
%
\newcommand{\Stapel}[2]%
 {%
  \sbox{\upperbox}{##1}%
  \sbox{\lowerbox}{##2}%
%
  \setlength{\upperVertLineLength}{\dp\upperbox}%
  \addtolength{\upperVertLineLength}{0.5\VertSkipStan}%
  \addtolength{\upperVertLineLength}{0.5\LineThikStan}%
%
  \setlength{\lowerVertLineLength}{\ht\lowerbox}%
  \addtolength{\lowerVertLineLength}{0.5\VertSkipStan}%
  \addtolength{\lowerVertLineLength}{0.5\LineThikStan}%
%
  \setlength{\VertSkip}{0.5\VertSkipStan}%
  \addtolength{\VertSkip}{\dp\upperbox}%
  \addtolength{\VertSkip}{-1\LineThikStan}% 
%
  \Linie%
  \LinieO{\upperVertLineLength}% 
  \addtolength{\lowerVertLineLength}{-0.5\LineThikStan}%
  \raisebox{\upperVertLineLength}{\KurzeLinie}%
  \addtolength{\lowerVertLineLength}{0.5\LineThikStan}%
  \LinieU{\lowerVertLineLength}%
%
  \renewcommand{\arraystretch}{0}%
  \addtolength{\upperVertLineLength}{-0.5\LineThikStan}%
  \raisebox{\upperVertLineLength}%
   {% 
    \addtolength{\upperVertLineLength}{0.5\LineThikStan}%
    \begin{tabular}[t]%              % Zwei Eintraege
     {%
      @{\hrulefill}%   % Verbindungslinie 1
      c%               % Eintraege Zentrieren
      @{\hrulefill}%   % Verbindungslinie 2
     }%
     \PfeilR\usebox{\upperbox}\Linie\\[\VertSkip]  % oberer Eintrag
     \PfeilL\\[0.5\VertSkipStan]%
     \PfeilR\usebox{\lowerbox}\Linie%              % unterer Eintrag
    \end{tabular}%
   }%
  \renewcommand{\arraystretch}{1}%
  \addtolength{\upperVertLineLength}{0.5\LineThikStan}%
%
  \LinieO{\upperVertLineLength}%
  \addtolength{\lowerVertLineLength}{-0.5\LineThikStan}%
  \raisebox{-\lowerVertLineLength}{\KurzeLinie}%
  \addtolength{\lowerVertLineLength}{0.5\LineThikStan}%
  \LinieU{\lowerVertLineLength}%
  \Linie%
 }%
%
%%%% Die Alternativen-Umgebung
%
\newenvironment{Alternativen}[2]%
 {%
  \renewcommand{\arraystretch}{2}%
  \setlength{\tabulcor}{9.3pt}%
  \sbox{\upperbox}{##1}%
  \sbox{\lowerbox}{##2}%
%
  \begin{lrbox}{\togetherbox}% 
    \begin{tabular}[t]%              % Zwei Eintraege
     {%
      @{\hrulefill}%                 % Verbindungslinie 1
      c%                             % Eintraege Zentrieren
      @{\hrulefill}%                 % Verbindungslinie 2
     }%
     \PfeilR\usebox{\upperbox}\Linie\\             % erster Weg 
 }%
%                                                  % weitere Wege
 {%
     \PfeilR\usebox{\lowerbox}\Linie%              % letzter Weg
    \end{tabular}%
  \end{lrbox}% 
%
  \setlength{\upperVertLineLength}{0.5\dp\togetherbox}%
  \addtolength{\upperVertLineLength}{0.5\ht\togetherbox}%
  \addtolength{\upperVertLineLength}{-\ht\upperbox}%
  \addtolength{\upperVertLineLength}{0.5\LineThikStan}%
  \addtolength{\upperVertLineLength}{-\tabulcor}%
%
  \setlength{\lowerVertLineLength}{0.5\dp\togetherbox}%
  \addtolength{\lowerVertLineLength}{0.5\ht\togetherbox}%
  \addtolength{\lowerVertLineLength}{-\dp\lowerbox}%
%
  \Linie%
  \LinieO{\upperVertLineLength}%
  \LinieU{\lowerVertLineLength}%
% 
  \CenterToBaseLine{\usebox{\togetherbox}}%
%  
  \LinieO{\upperVertLineLength}%
  \LinieU{\lowerVertLineLength}%
  \Linie%
 }%
%
%%%%% Ende Syntaxdiagramm Befehle und Umgebungen
%
  \providecommand{\ContSym}{#1}
  \renewcommand{\ContSym}{#1}
  \begin{lrbox}{\Messbox}%
 }%
 {%
  \end{lrbox}%
  \setlength{\hoehe}{\ht\Messbox}%
  \addtolength{\hoehe}{\VertSkipStan}%
  \setlength{\tiefe}{\dp\Messbox}%
  \addtolength{\tiefe}{\VertSkipStan}%
  \raisebox{0pt}[\hoehe][\tiefe]%
   {%
    \Linie%
    \usebox{\Messbox}%
    \ContSym%
   }%
 }%
%
%%%%% --Und Schluss--

 eingebunden zu werden. Es stellt dann die Umgebung
% syntaxdiagramm zur Verfuegung, mit deren Hilfe Syntaxdiagramme fuer
% z.B. Pascal erzeugt werden koennen. 
%
% syntax.tex benoetigt nur noch folgendes LaTeX Paket:
%   - fancybox.sty
% (Es muss im Vorspann des aufrufenden Dokumentes eingebunden werden.) 
%
% Beispiele zur Verwendung siehe: syntaxtest.tex 
% zur naeheren Dokumentation siehe: syntaxintro.tex
% zur Entwicklung des Codes siehe: syntaxdevel.tex
%
% Bernd Worsch 29.6.97 
%
% 05.07.97  \Leer als vertikale Stuetze
%           groessere Abstaende in Alternativen-Umgebung 
%
% to-do-list:
%   -- Oval fuer SynVar ueber picture erzeugen -> fancybox unnoetig
%   -- Alternativen-Parameter automatisch bestimmen
%   -- Dokumentation (syntaxuse.tex)
%   -- Kommentare im Quelltext
%   -- Als sty-File implementieren



\typeout{Verwende: 'syntax.tex' v0.02 von Bernd Worsch}



%%%% Anlegen der benoetigten Laengen und Boxen 

\newlength{\breite}
\newlength{\hoehe}
\newlength{\tiefe}
\newlength{\gesamthoehe}
\newlength{\tmpLength}

\newsavebox{\Messbox}

\newlength{\LineLengStan}
\newlength{\LineThikStan}
\setlength{\LineLengStan}{4mm}
\setlength{\LineThikStan}{\fboxrule}

\newsavebox{\Centerbox}
\newlength{\Centercor}

\newlength{\VertSkipStan}
\setlength{\VertSkipStan}{3mm}

\newsavebox{\upperbox}
\newsavebox{\lowerbox}
\newsavebox{\togetherbox}
\newlength{\VertLineLength}
\newlength{\VertSkip}

\newlength{\upperVertLineLength}
\newlength{\lowerVertLineLength}

\newlength{\tabulcor}

%%%%% Beginn Syntaxdiagramm Befehle und Umgebungen

\newenvironment{Syntaxdiagramm}[1][\PfeilR]%
 {%
%
%%%% Befehl zum Vermessen von TeX-Strukturen
%
%\newcommand{\Messe}[1]
% {
%  \sbox{\Messbox}{##1}
%
%  \setlength{\breite}{\wd\Messbox}
%  \setlength{\hoehe}{\ht\Messbox}
%  \setlength{\tiefe}{\dp\Messbox}
%  \setlength{\gesamthoehe}{\hoehe}
%  \addtolength{\gesamthoehe}{\tiefe}
%  \setlength{\tmpLength}{\hoehe}
%  \addtolength{\tmpLength}{-\tiefe}
%  \setlength{\tmpLength}{-0.5\tmpLength} 
%
%  \fbox
%   {
%    \begin{tabular}{ll}
%      Breite:      & \the\breite      \\
%      Hoehe:       & \the\hoehe       \\
%      Tiefe:       & \the\tiefe       \\
%      Gesamthoehe: & \the\gesamthoehe \\
%      CenterCor:   & \the\tmpLength
%    \end{tabular}
% 
%    \usebox{\Messbox}
%   }
% }
%
%%%% Befehl zum Justieren von Boxen 
%
\newcommand{\CenterToBaseLine}[1]
  {%
   \sbox{\Centerbox}{##1}%
   \setlength{\Centercor}{\ht\Centerbox}%
   \addtolength{\Centercor}{-\dp\Centerbox}%
   \setlength{\Centercor}{-0.5\Centercor}%
   \addtolength{\Centercor}{0.5\LineThikStan}%
   \raisebox{\Centercor}{\usebox{\Centerbox}}%
  }%
%
%%%% Grundelemente eines Syntaxdiagramms
%
\newcommand{\Leer}%
  {\rule[-1.3ex]{0pt}{2.5ex}}%
%
\newcommand{\Linie}%
  {\rule{\LineLengStan}{\LineThikStan}}%
%
\newcommand{\KurzeLinie}%
  {\rule{0.5\LineLengStan}{\LineThikStan}}%
%
\newcommand{\Tri}
  {%
   \CenterToBaseLine
   {%
     \setlength{\unitlength}{1pt}%
     \begin{picture}(0,0)%
       \put(0,0){\vector(0,1){0}}%
     \end{picture}%
   }%
  }%
%
\newcommand{\TriR}
  {%
   \CenterToBaseLine
    {%
     \setlength{\unitlength}{1pt}%
     \begin{picture}(0,0)%
       \put(0,0){\vector(1,0){0}}%
     \end{picture}%
    }%
  }%
%
\newcommand{\TriD}
  {%
   \CenterToBaseLine
    {%
     \setlength{\unitlength}{1pt}%
     \begin{picture}(0,0)%
       \put(0,0){\vector(0,-1){0}}%
     \end{picture}%
    }%
  }%
%
\newcommand{\TriL}
  {%
   \CenterToBaseLine
    {%
     \setlength{\unitlength}{1pt}%
     \begin{picture}(0,0)%
       \put(0,0){\vector(-1,0){0}}%
     \end{picture}%
    }%
  }%
%
\newcommand{\Dots}
  {%
   \CenterToBaseLine
    {%
     ...%
    }%
  }%
%
\newcommand{\SynVar}[1]%
  {%
   \CenterToBaseLine%
    {%
     \fbox%
      {%
       \rule{0.1em}{0pt}%
       \rule[-0.3ex]{0pt}{2ex}%
       ##1%
       \rule{0.1em}{0pt}%
      }%
    }%
  }%
%
\newcommand{\TerSym}[1]%
  {%
   \CenterToBaseLine%
    {%
     \ovalbox%
      {%
       \rule{0.2em}{0pt}%
       \rule[-0.2ex]{0pt}{1.8ex}%
       ##1%
       \rule{0.2em}{0pt}%
      }%
    }%
  }%
%
%%%% Kombinierte Grafische Elemente
%
\newcommand{\Ellipse}%
 {%
  \Linie\ \Dots\ \Linie%
 }%
%
\newcommand{\PfeilR}%
 {%
  \Linie\TriR%
 }%
%
\newcommand{\PfeilL}%
 {%
  \TriL\Linie%
 }%
%
\newcommand{\SynVarR}[1]%
 {%
  \PfeilR\SynVar{##1}\Linie%
 }%
%
\newcommand{\SynVarL}[1]%
 {%
  \Linie\SynVar{##1}\PfeilL%
 }%
%
\newcommand{\TerSymR}[1]%
 {%
  \PfeilR\TerSym{##1}\Linie%
 }%
%
\newcommand{\TerSymL}[1]%
 {%
  \Linie\TerSym{##1}\PfeilL%
 }%
%
%%%% Vertikale grafische Elemente
%
\newcommand{\LinieO}[1]%
 {%
  \CenterToBaseLine%
   {%
    \raisebox{0pt}[0pt][0pt]%
     {%
      \makebox[0mm][c]{\rule{\LineThikStan}{##1}}%
     }%
   }%
 }%
%
\newcommand{\LinieU}[1]%
 {%
  \CenterToBaseLine%
   {%
    \raisebox{0pt}[0pt][0pt]%
     {%
      \makebox[0mm][c]{\rule[-##1]{\LineThikStan}{##1}}%
     }%
   }%
 }%
%
\newcommand{\PfeilO}[1]%
 {%
  \CenterToBaseLine%
   {%
    \raisebox{0pt}[0pt][0pt]%
     {%
      \makebox[0mm][c]{\TriD}%
      \makebox[0mm][c]{\rule{\LineThikStan}{##1}}%
     }%
   }%
 }%
%
\newcommand{\PfeilU}[1]%
 {%
  \CenterToBaseLine%
   {%
    \raisebox{0pt}[0pt][0pt]%
     {%
      \makebox[0mm][c]{\Tri}%
      \makebox[0mm][c]{\rule[-##1]{\LineThikStan}{##1}}%
     }%
   }%
 }%
%
%%%% Strukturen
%
%%% Wiederholung
%
\newcommand{\Wiederholung}[2]%
 {%
  \sbox{\upperbox}{##1}%
  \sbox{\lowerbox}{##2}%
  \setlength{\VertLineLength}{\dp\upperbox}%
  \addtolength{\VertLineLength}{\ht\lowerbox}%
  \addtolength{\VertLineLength}{\VertSkipStan}%
  \addtolength{\VertLineLength}{0.5\LineThikStan}%
  \setlength{\VertSkip}{\VertSkipStan}%
  \addtolength{\VertSkip}{\dp\upperbox}%
%
  \Linie%
  \PfeilU{\VertLineLength}%
% 
  \renewcommand{\arraystretch}{0}%
  \begin{tabular}[t]%              % Zwei Eintraege
   {%
    @{\hrulefill}%                 % Verbindungslinie 1
    c%                             % Eintraege Zentrieren
    @{\hrulefill}%                 % Verbindungslinie 2
   }%
   \PfeilR\usebox{\upperbox}\Linie\\[\VertSkip]  % Dies wiederholen
   \Linie\usebox{\lowerbox}\PfeilL%              % Dies in Gegenrichtung
  \end{tabular}%
%
  \LinieU{\VertLineLength}%
  \Linie%
 }%
%
%%% Alternative
%
\newcommand{\Alternative}[2]%
 {%
  \sbox{\upperbox}{##1}%
  \sbox{\lowerbox}{##2}%
%
  \setlength{\upperVertLineLength}{\dp\upperbox}%
  \addtolength{\upperVertLineLength}{0.5\VertSkipStan}%
  \addtolength{\upperVertLineLength}{0.5\LineThikStan}%
%
  \setlength{\lowerVertLineLength}{\ht\lowerbox}%
  \addtolength{\lowerVertLineLength}{0.5\VertSkipStan}%
  \addtolength{\lowerVertLineLength}{0.5\LineThikStan}%
%
  \setlength{\VertSkip}{\VertSkipStan}%
  \addtolength{\VertSkip}{\dp\upperbox}%
%
  \Linie%
  \LinieO{\upperVertLineLength}%
  \LinieU{\lowerVertLineLength}%
% 
  \renewcommand{\arraystretch}{0}%
  \addtolength{\upperVertLineLength}{-0.5\LineThikStan}%
  \raisebox{\upperVertLineLength}%
   {% 
    \begin{tabular}[t]%                 % Zwei Eintraege
     {%
      @{\hrulefill}%                 % Verbindungslinie 1
      c%                             % Eintraege Zentrieren
      @{\hrulefill}%                 % Verbindungslinie 2
     }%
     \PfeilR\usebox{\upperbox}\Linie\\[\VertSkip]  % Weg1
     \PfeilR\usebox{\lowerbox}\Linie%              % Weg2
    \end{tabular}%
   }%
  \renewcommand{\arraystretch}{1}%
  \addtolength{\upperVertLineLength}{0.5\LineThikStan}%
%
  \LinieO{\upperVertLineLength}%
  \LinieU{\lowerVertLineLength}%
  \Linie%
 }%
%
%%% Stapel
%
\newcommand{\Stapel}[2]%
 {%
  \sbox{\upperbox}{##1}%
  \sbox{\lowerbox}{##2}%
%
  \setlength{\upperVertLineLength}{\dp\upperbox}%
  \addtolength{\upperVertLineLength}{0.5\VertSkipStan}%
  \addtolength{\upperVertLineLength}{0.5\LineThikStan}%
%
  \setlength{\lowerVertLineLength}{\ht\lowerbox}%
  \addtolength{\lowerVertLineLength}{0.5\VertSkipStan}%
  \addtolength{\lowerVertLineLength}{0.5\LineThikStan}%
%
  \setlength{\VertSkip}{0.5\VertSkipStan}%
  \addtolength{\VertSkip}{\dp\upperbox}%
  \addtolength{\VertSkip}{-1\LineThikStan}% 
%
  \Linie%
  \LinieO{\upperVertLineLength}% 
  \addtolength{\lowerVertLineLength}{-0.5\LineThikStan}%
  \raisebox{\upperVertLineLength}{\KurzeLinie}%
  \addtolength{\lowerVertLineLength}{0.5\LineThikStan}%
  \LinieU{\lowerVertLineLength}%
%
  \renewcommand{\arraystretch}{0}%
  \addtolength{\upperVertLineLength}{-0.5\LineThikStan}%
  \raisebox{\upperVertLineLength}%
   {% 
    \addtolength{\upperVertLineLength}{0.5\LineThikStan}%
    \begin{tabular}[t]%              % Zwei Eintraege
     {%
      @{\hrulefill}%   % Verbindungslinie 1
      c%               % Eintraege Zentrieren
      @{\hrulefill}%   % Verbindungslinie 2
     }%
     \PfeilR\usebox{\upperbox}\Linie\\[\VertSkip]  % oberer Eintrag
     \PfeilL\\[0.5\VertSkipStan]%
     \PfeilR\usebox{\lowerbox}\Linie%              % unterer Eintrag
    \end{tabular}%
   }%
  \renewcommand{\arraystretch}{1}%
  \addtolength{\upperVertLineLength}{0.5\LineThikStan}%
%
  \LinieO{\upperVertLineLength}%
  \addtolength{\lowerVertLineLength}{-0.5\LineThikStan}%
  \raisebox{-\lowerVertLineLength}{\KurzeLinie}%
  \addtolength{\lowerVertLineLength}{0.5\LineThikStan}%
  \LinieU{\lowerVertLineLength}%
  \Linie%
 }%
%
%%%% Die Alternativen-Umgebung
%
\newenvironment{Alternativen}[2]%
 {%
  \renewcommand{\arraystretch}{2}%
  \setlength{\tabulcor}{9.3pt}%
  \sbox{\upperbox}{##1}%
  \sbox{\lowerbox}{##2}%
%
  \begin{lrbox}{\togetherbox}% 
    \begin{tabular}[t]%              % Zwei Eintraege
     {%
      @{\hrulefill}%                 % Verbindungslinie 1
      c%                             % Eintraege Zentrieren
      @{\hrulefill}%                 % Verbindungslinie 2
     }%
     \PfeilR\usebox{\upperbox}\Linie\\             % erster Weg 
 }%
%                                                  % weitere Wege
 {%
     \PfeilR\usebox{\lowerbox}\Linie%              % letzter Weg
    \end{tabular}%
  \end{lrbox}% 
%
  \setlength{\upperVertLineLength}{0.5\dp\togetherbox}%
  \addtolength{\upperVertLineLength}{0.5\ht\togetherbox}%
  \addtolength{\upperVertLineLength}{-\ht\upperbox}%
  \addtolength{\upperVertLineLength}{0.5\LineThikStan}%
  \addtolength{\upperVertLineLength}{-\tabulcor}%
%
  \setlength{\lowerVertLineLength}{0.5\dp\togetherbox}%
  \addtolength{\lowerVertLineLength}{0.5\ht\togetherbox}%
  \addtolength{\lowerVertLineLength}{-\dp\lowerbox}%
%
  \Linie%
  \LinieO{\upperVertLineLength}%
  \LinieU{\lowerVertLineLength}%
% 
  \CenterToBaseLine{\usebox{\togetherbox}}%
%  
  \LinieO{\upperVertLineLength}%
  \LinieU{\lowerVertLineLength}%
  \Linie%
 }%
%
%%%%% Ende Syntaxdiagramm Befehle und Umgebungen
%
  \providecommand{\ContSym}{#1}
  \renewcommand{\ContSym}{#1}
  \begin{lrbox}{\Messbox}%
 }%
 {%
  \end{lrbox}%
  \setlength{\hoehe}{\ht\Messbox}%
  \addtolength{\hoehe}{\VertSkipStan}%
  \setlength{\tiefe}{\dp\Messbox}%
  \addtolength{\tiefe}{\VertSkipStan}%
  \raisebox{0pt}[\hoehe][\tiefe]%
   {%
    \Linie%
    \usebox{\Messbox}%
    \ContSym%
   }%
 }%
%
%%%%% --Und Schluss--

 eingebunden zu werden. Es stellt dann die Umgebung
% syntaxdiagramm zur Verfuegung, mit deren Hilfe Syntaxdiagramme fuer
% z.B. Pascal erzeugt werden koennen. 
%
% syntax.tex benoetigt nur noch folgendes LaTeX Paket:
%   - fancybox.sty
% (Es muss im Vorspann des aufrufenden Dokumentes eingebunden werden.) 
%
% Beispiele zur Verwendung siehe: syntaxtest.tex 
% zur naeheren Dokumentation siehe: syntaxintro.tex
% zur Entwicklung des Codes siehe: syntaxdevel.tex
%
% Bernd Worsch 29.6.97 
%
% 05.07.97  \Leer als vertikale Stuetze
%           groessere Abstaende in Alternativen-Umgebung 
%
% to-do-list:
%   -- Oval fuer SynVar ueber picture erzeugen -> fancybox unnoetig
%   -- Alternativen-Parameter automatisch bestimmen
%   -- Dokumentation (syntaxuse.tex)
%   -- Kommentare im Quelltext
%   -- Als sty-File implementieren



\typeout{Verwende: 'syntax.tex' v0.02 von Bernd Worsch}



%%%% Anlegen der benoetigten Laengen und Boxen 

\newlength{\breite}
\newlength{\hoehe}
\newlength{\tiefe}
\newlength{\gesamthoehe}
\newlength{\tmpLength}

\newsavebox{\Messbox}

\newlength{\LineLengStan}
\newlength{\LineThikStan}
\setlength{\LineLengStan}{4mm}
\setlength{\LineThikStan}{\fboxrule}

\newsavebox{\Centerbox}
\newlength{\Centercor}

\newlength{\VertSkipStan}
\setlength{\VertSkipStan}{3mm}

\newsavebox{\upperbox}
\newsavebox{\lowerbox}
\newsavebox{\togetherbox}
\newlength{\VertLineLength}
\newlength{\VertSkip}

\newlength{\upperVertLineLength}
\newlength{\lowerVertLineLength}

\newlength{\tabulcor}

%%%%% Beginn Syntaxdiagramm Befehle und Umgebungen

\newenvironment{Syntaxdiagramm}[1][\PfeilR]%
 {%
%
%%%% Befehl zum Vermessen von TeX-Strukturen
%
%\newcommand{\Messe}[1]
% {
%  \sbox{\Messbox}{##1}
%
%  \setlength{\breite}{\wd\Messbox}
%  \setlength{\hoehe}{\ht\Messbox}
%  \setlength{\tiefe}{\dp\Messbox}
%  \setlength{\gesamthoehe}{\hoehe}
%  \addtolength{\gesamthoehe}{\tiefe}
%  \setlength{\tmpLength}{\hoehe}
%  \addtolength{\tmpLength}{-\tiefe}
%  \setlength{\tmpLength}{-0.5\tmpLength} 
%
%  \fbox
%   {
%    \begin{tabular}{ll}
%      Breite:      & \the\breite      \\
%      Hoehe:       & \the\hoehe       \\
%      Tiefe:       & \the\tiefe       \\
%      Gesamthoehe: & \the\gesamthoehe \\
%      CenterCor:   & \the\tmpLength
%    \end{tabular}
% 
%    \usebox{\Messbox}
%   }
% }
%
%%%% Befehl zum Justieren von Boxen 
%
\newcommand{\CenterToBaseLine}[1]
  {%
   \sbox{\Centerbox}{##1}%
   \setlength{\Centercor}{\ht\Centerbox}%
   \addtolength{\Centercor}{-\dp\Centerbox}%
   \setlength{\Centercor}{-0.5\Centercor}%
   \addtolength{\Centercor}{0.5\LineThikStan}%
   \raisebox{\Centercor}{\usebox{\Centerbox}}%
  }%
%
%%%% Grundelemente eines Syntaxdiagramms
%
\newcommand{\Leer}%
  {\rule[-1.3ex]{0pt}{2.5ex}}%
%
\newcommand{\Linie}%
  {\rule{\LineLengStan}{\LineThikStan}}%
%
\newcommand{\KurzeLinie}%
  {\rule{0.5\LineLengStan}{\LineThikStan}}%
%
\newcommand{\Tri}
  {%
   \CenterToBaseLine
   {%
     \setlength{\unitlength}{1pt}%
     \begin{picture}(0,0)%
       \put(0,0){\vector(0,1){0}}%
     \end{picture}%
   }%
  }%
%
\newcommand{\TriR}
  {%
   \CenterToBaseLine
    {%
     \setlength{\unitlength}{1pt}%
     \begin{picture}(0,0)%
       \put(0,0){\vector(1,0){0}}%
     \end{picture}%
    }%
  }%
%
\newcommand{\TriD}
  {%
   \CenterToBaseLine
    {%
     \setlength{\unitlength}{1pt}%
     \begin{picture}(0,0)%
       \put(0,0){\vector(0,-1){0}}%
     \end{picture}%
    }%
  }%
%
\newcommand{\TriL}
  {%
   \CenterToBaseLine
    {%
     \setlength{\unitlength}{1pt}%
     \begin{picture}(0,0)%
       \put(0,0){\vector(-1,0){0}}%
     \end{picture}%
    }%
  }%
%
\newcommand{\Dots}
  {%
   \CenterToBaseLine
    {%
     ...%
    }%
  }%
%
\newcommand{\SynVar}[1]%
  {%
   \CenterToBaseLine%
    {%
     \fbox%
      {%
       \rule{0.1em}{0pt}%
       \rule[-0.3ex]{0pt}{2ex}%
       ##1%
       \rule{0.1em}{0pt}%
      }%
    }%
  }%
%
\newcommand{\TerSym}[1]%
  {%
   \CenterToBaseLine%
    {%
     \ovalbox%
      {%
       \rule{0.2em}{0pt}%
       \rule[-0.2ex]{0pt}{1.8ex}%
       ##1%
       \rule{0.2em}{0pt}%
      }%
    }%
  }%
%
%%%% Kombinierte Grafische Elemente
%
\newcommand{\Ellipse}%
 {%
  \Linie\ \Dots\ \Linie%
 }%
%
\newcommand{\PfeilR}%
 {%
  \Linie\TriR%
 }%
%
\newcommand{\PfeilL}%
 {%
  \TriL\Linie%
 }%
%
\newcommand{\SynVarR}[1]%
 {%
  \PfeilR\SynVar{##1}\Linie%
 }%
%
\newcommand{\SynVarL}[1]%
 {%
  \Linie\SynVar{##1}\PfeilL%
 }%
%
\newcommand{\TerSymR}[1]%
 {%
  \PfeilR\TerSym{##1}\Linie%
 }%
%
\newcommand{\TerSymL}[1]%
 {%
  \Linie\TerSym{##1}\PfeilL%
 }%
%
%%%% Vertikale grafische Elemente
%
\newcommand{\LinieO}[1]%
 {%
  \CenterToBaseLine%
   {%
    \raisebox{0pt}[0pt][0pt]%
     {%
      \makebox[0mm][c]{\rule{\LineThikStan}{##1}}%
     }%
   }%
 }%
%
\newcommand{\LinieU}[1]%
 {%
  \CenterToBaseLine%
   {%
    \raisebox{0pt}[0pt][0pt]%
     {%
      \makebox[0mm][c]{\rule[-##1]{\LineThikStan}{##1}}%
     }%
   }%
 }%
%
\newcommand{\PfeilO}[1]%
 {%
  \CenterToBaseLine%
   {%
    \raisebox{0pt}[0pt][0pt]%
     {%
      \makebox[0mm][c]{\TriD}%
      \makebox[0mm][c]{\rule{\LineThikStan}{##1}}%
     }%
   }%
 }%
%
\newcommand{\PfeilU}[1]%
 {%
  \CenterToBaseLine%
   {%
    \raisebox{0pt}[0pt][0pt]%
     {%
      \makebox[0mm][c]{\Tri}%
      \makebox[0mm][c]{\rule[-##1]{\LineThikStan}{##1}}%
     }%
   }%
 }%
%
%%%% Strukturen
%
%%% Wiederholung
%
\newcommand{\Wiederholung}[2]%
 {%
  \sbox{\upperbox}{##1}%
  \sbox{\lowerbox}{##2}%
  \setlength{\VertLineLength}{\dp\upperbox}%
  \addtolength{\VertLineLength}{\ht\lowerbox}%
  \addtolength{\VertLineLength}{\VertSkipStan}%
  \addtolength{\VertLineLength}{0.5\LineThikStan}%
  \setlength{\VertSkip}{\VertSkipStan}%
  \addtolength{\VertSkip}{\dp\upperbox}%
%
  \Linie%
  \PfeilU{\VertLineLength}%
% 
  \renewcommand{\arraystretch}{0}%
  \begin{tabular}[t]%              % Zwei Eintraege
   {%
    @{\hrulefill}%                 % Verbindungslinie 1
    c%                             % Eintraege Zentrieren
    @{\hrulefill}%                 % Verbindungslinie 2
   }%
   \PfeilR\usebox{\upperbox}\Linie\\[\VertSkip]  % Dies wiederholen
   \Linie\usebox{\lowerbox}\PfeilL%              % Dies in Gegenrichtung
  \end{tabular}%
%
  \LinieU{\VertLineLength}%
  \Linie%
 }%
%
%%% Alternative
%
\newcommand{\Alternative}[2]%
 {%
  \sbox{\upperbox}{##1}%
  \sbox{\lowerbox}{##2}%
%
  \setlength{\upperVertLineLength}{\dp\upperbox}%
  \addtolength{\upperVertLineLength}{0.5\VertSkipStan}%
  \addtolength{\upperVertLineLength}{0.5\LineThikStan}%
%
  \setlength{\lowerVertLineLength}{\ht\lowerbox}%
  \addtolength{\lowerVertLineLength}{0.5\VertSkipStan}%
  \addtolength{\lowerVertLineLength}{0.5\LineThikStan}%
%
  \setlength{\VertSkip}{\VertSkipStan}%
  \addtolength{\VertSkip}{\dp\upperbox}%
%
  \Linie%
  \LinieO{\upperVertLineLength}%
  \LinieU{\lowerVertLineLength}%
% 
  \renewcommand{\arraystretch}{0}%
  \addtolength{\upperVertLineLength}{-0.5\LineThikStan}%
  \raisebox{\upperVertLineLength}%
   {% 
    \begin{tabular}[t]%                 % Zwei Eintraege
     {%
      @{\hrulefill}%                 % Verbindungslinie 1
      c%                             % Eintraege Zentrieren
      @{\hrulefill}%                 % Verbindungslinie 2
     }%
     \PfeilR\usebox{\upperbox}\Linie\\[\VertSkip]  % Weg1
     \PfeilR\usebox{\lowerbox}\Linie%              % Weg2
    \end{tabular}%
   }%
  \renewcommand{\arraystretch}{1}%
  \addtolength{\upperVertLineLength}{0.5\LineThikStan}%
%
  \LinieO{\upperVertLineLength}%
  \LinieU{\lowerVertLineLength}%
  \Linie%
 }%
%
%%% Stapel
%
\newcommand{\Stapel}[2]%
 {%
  \sbox{\upperbox}{##1}%
  \sbox{\lowerbox}{##2}%
%
  \setlength{\upperVertLineLength}{\dp\upperbox}%
  \addtolength{\upperVertLineLength}{0.5\VertSkipStan}%
  \addtolength{\upperVertLineLength}{0.5\LineThikStan}%
%
  \setlength{\lowerVertLineLength}{\ht\lowerbox}%
  \addtolength{\lowerVertLineLength}{0.5\VertSkipStan}%
  \addtolength{\lowerVertLineLength}{0.5\LineThikStan}%
%
  \setlength{\VertSkip}{0.5\VertSkipStan}%
  \addtolength{\VertSkip}{\dp\upperbox}%
  \addtolength{\VertSkip}{-1\LineThikStan}% 
%
  \Linie%
  \LinieO{\upperVertLineLength}% 
  \addtolength{\lowerVertLineLength}{-0.5\LineThikStan}%
  \raisebox{\upperVertLineLength}{\KurzeLinie}%
  \addtolength{\lowerVertLineLength}{0.5\LineThikStan}%
  \LinieU{\lowerVertLineLength}%
%
  \renewcommand{\arraystretch}{0}%
  \addtolength{\upperVertLineLength}{-0.5\LineThikStan}%
  \raisebox{\upperVertLineLength}%
   {% 
    \addtolength{\upperVertLineLength}{0.5\LineThikStan}%
    \begin{tabular}[t]%              % Zwei Eintraege
     {%
      @{\hrulefill}%   % Verbindungslinie 1
      c%               % Eintraege Zentrieren
      @{\hrulefill}%   % Verbindungslinie 2
     }%
     \PfeilR\usebox{\upperbox}\Linie\\[\VertSkip]  % oberer Eintrag
     \PfeilL\\[0.5\VertSkipStan]%
     \PfeilR\usebox{\lowerbox}\Linie%              % unterer Eintrag
    \end{tabular}%
   }%
  \renewcommand{\arraystretch}{1}%
  \addtolength{\upperVertLineLength}{0.5\LineThikStan}%
%
  \LinieO{\upperVertLineLength}%
  \addtolength{\lowerVertLineLength}{-0.5\LineThikStan}%
  \raisebox{-\lowerVertLineLength}{\KurzeLinie}%
  \addtolength{\lowerVertLineLength}{0.5\LineThikStan}%
  \LinieU{\lowerVertLineLength}%
  \Linie%
 }%
%
%%%% Die Alternativen-Umgebung
%
\newenvironment{Alternativen}[2]%
 {%
  \renewcommand{\arraystretch}{2}%
  \setlength{\tabulcor}{9.3pt}%
  \sbox{\upperbox}{##1}%
  \sbox{\lowerbox}{##2}%
%
  \begin{lrbox}{\togetherbox}% 
    \begin{tabular}[t]%              % Zwei Eintraege
     {%
      @{\hrulefill}%                 % Verbindungslinie 1
      c%                             % Eintraege Zentrieren
      @{\hrulefill}%                 % Verbindungslinie 2
     }%
     \PfeilR\usebox{\upperbox}\Linie\\             % erster Weg 
 }%
%                                                  % weitere Wege
 {%
     \PfeilR\usebox{\lowerbox}\Linie%              % letzter Weg
    \end{tabular}%
  \end{lrbox}% 
%
  \setlength{\upperVertLineLength}{0.5\dp\togetherbox}%
  \addtolength{\upperVertLineLength}{0.5\ht\togetherbox}%
  \addtolength{\upperVertLineLength}{-\ht\upperbox}%
  \addtolength{\upperVertLineLength}{0.5\LineThikStan}%
  \addtolength{\upperVertLineLength}{-\tabulcor}%
%
  \setlength{\lowerVertLineLength}{0.5\dp\togetherbox}%
  \addtolength{\lowerVertLineLength}{0.5\ht\togetherbox}%
  \addtolength{\lowerVertLineLength}{-\dp\lowerbox}%
%
  \Linie%
  \LinieO{\upperVertLineLength}%
  \LinieU{\lowerVertLineLength}%
% 
  \CenterToBaseLine{\usebox{\togetherbox}}%
%  
  \LinieO{\upperVertLineLength}%
  \LinieU{\lowerVertLineLength}%
  \Linie%
 }%
%
%%%%% Ende Syntaxdiagramm Befehle und Umgebungen
%
  \providecommand{\ContSym}{#1}
  \renewcommand{\ContSym}{#1}
  \begin{lrbox}{\Messbox}%
 }%
 {%
  \end{lrbox}%
  \setlength{\hoehe}{\ht\Messbox}%
  \addtolength{\hoehe}{\VertSkipStan}%
  \setlength{\tiefe}{\dp\Messbox}%
  \addtolength{\tiefe}{\VertSkipStan}%
  \raisebox{0pt}[\hoehe][\tiefe]%
   {%
    \Linie%
    \usebox{\Messbox}%
    \ContSym%
   }%
 }%
%
%%%%% --Und Schluss--

+
in das eigene Dokument eingef"ugt werden. Da es in der vorliegenden
Version \vers\ noch auf einen Befehl des Pakets fancybox.sty 
zur"uckgreift, mu"s dieses am Anfang der eigenen Datei aktiviert werden.
Wie f"ur alle Pakete geschieht dies durch den Befehl: 
\verb+\usepackage{fancybox}+.

\section{Grundelemente von \syn}
Abgesehen von Hilfskonstrukten, insbesondere Linien und Pfeilen, 
bestehen Syntaxdiagramme aus Terminalsymbolen, den elementaren
Grundbausteinen einer Programmiersprache, und Syntaxvariablen.
Letztere stehen als Platzhalter f"ur syntaktische Konstruktionen. 

Beide Grundelemente stehen innerhalb der Umgebung \verb+Syntaxdiagramm+
mit den Befehlen \verb+\TerSym+ und \verb+\SynVar+ zur Verf"ugung.

Die Umgebung \verb+Syntaxdiagramm+ erzeugt zun"achst ein leeres 
Syntaxdiagramm, also einen von links nach rechts verlaufenden Pfeil. 
An inneren Befehlen stehen unter anderem
elementare Teilstrukturen von Syntaxdiagrammen bereit. Einige haben 
keine Parameter und erzeugen beispielsweise Linien oder Pfeile, 
\verb+\TerSym+ und \verb+\SynVar+ hingegen erhalten als Parameter 
einen Text. 
Da diese Befehle stets innerhalb des umgebenden Pfeils auftreten, 
wurden in der folgenden "Ubersicht die inneren Elemente durch 
Leerzeichen von der "au"seren 
Umgebung abgegrenzt. Diese Zwischenr"aume sind im endg"ultigen
Diagramm unerw"unscht. 
Befehle und Umgebungen in Syntaxdiagrammen sollten daher mit einem 
Kommentarzeichen (\verb+%+) 
beendet werden , falls Zeilenumbr"uche im Eingabetext vorgenommen werden.
\medskip

\begin{tabular}{l|c}
  \begin{tabular}{l}
    \verb+\begin{Syntaxdiagramm}+\\
    \verb+\end{Syntaxdiagramm}+
  \end{tabular}   & \begin{Syntaxdiagramm}\end{Syntaxdiagramm}         \\
  \hline
  \verb+\Linie+   & \begin{Syntaxdiagramm}\dx\Linie\dx\end{Syntaxdiagramm}\\
  \verb+\PfeilR+  & \begin{Syntaxdiagramm}\dx\PfeilR\dx\end{Syntaxdiagramm}\\
  \verb+\PfeilL+  & \begin{Syntaxdiagramm}\dx\PfeilL\dx\end{Syntaxdiagramm}\\
  \verb+\Ellipse+ & \begin{Syntaxdiagramm}\dx\Ellipse\dx\end{Syntaxdiagramm}\\
  \verb+\Leer+ & \begin{Syntaxdiagramm}\dx\Leer\dx\end{Syntaxdiagramm}\\
  \hline
  \verb+\TerSym{xxxx}+ & 
  \begin{Syntaxdiagramm}\dx\TerSym{xxxx}\dx\end{Syntaxdiagramm} \\
  \verb+\SynVar{xxxx}+ & 
  \begin{Syntaxdiagramm}\dx\SynVar{xxxx}\dx\end{Syntaxdiagramm} 
\end{tabular} 
\medskip
   
Aus obigen Grundelementen und vertikalen Linien werden alle Syntaxdiagramme 
aufgebaut. Die vertikalen Linien und Pfeile sind prinzipiell unter den 
Namen 
\verb+\LinieO{x}+,\verb+\LinieU{x}+,\verb+\PfeilO{x}+ und \verb+\PfeilU{x}+
ansprechbar, wobei \verb+x+ die jeweilige L"ange festlegt, sie sind
jedoch nur intern notwendig.  
Um die \LaTeX --Beschreibung eines Syntaxdiagramms kompakt zu halten 
existieren schlie"slich noch vier abk"urzende Schreibweisen. Es sind 
die zusammengesetzten Grundelemente:
\medskip

\begin{tabular}{l|c}
  \verb+\TerSymR{xxxx}+ & 
   \begin{Syntaxdiagramm}\dx\TerSymR{xxxx}\dx\end{Syntaxdiagramm} \\
  \verb+\TerSymL{xxxx}+ & 
   \begin{Syntaxdiagramm}\dx\TerSymL{xxxx}\dx\end{Syntaxdiagramm} \\
  \verb+\SynVarR{xxxx}+ & 
   \begin{Syntaxdiagramm}\dx\SynVarR{xxxx}\dx\end{Syntaxdiagramm} \\
  \verb+\SynVarL{xxxx}+ & 
   \begin{Syntaxdiagramm}\dx\SynVarL{xxxx}\dx\end{Syntaxdiagramm} \\
\end{tabular}
\medskip

\section{Strukturelemente von \syn}
Aus den Grundelementen allein lassen sich noch keine (bzw. sehr wenige)
Syntaxdiagramme aufbauen, es fehlt die M"oglichkeit diese Elemente zu
verbinden. Zu diesem Zweck stellt \syn\ Alternativen und Wiederholungen
bereit.
F"ur beide Strukturen existieren zun"achst Befehle, die je
zwei Parameter "ubernehmen. Im Fall des Befehls 
\verb+\Alternative{x1}{x2}+ steht \verb+x1+ f"ur den Teil des 
Strukturdiagramms der auf dem ersten Pfad, \verb+x2+
f"ur den Teil der auf dem zweiten Pfad durchlaufen wird.
Bei \verb+\Wiederholung{x1}{x2}+ entspricht \verb+x1+ dem wiederholten
Teil und \verb+x2+ allem, was im zur"ucklaufenden Pfad auftritt.
Die erzeugten Bilder wieder in einer Tabelle:
\medskip

\begin{tabular}{l|c}
  \verb+\Alternative{x1}{x2}+ & 
   \begin{Syntaxdiagramm}\dx\Alternative{x1}{x2}\dx\end{Syntaxdiagramm} \\
  \verb+\Wiederholung{x1}{x2}+ & 
   \begin{Syntaxdiagramm}\dx\Wiederholung{x1}{x2}\dx\end{Syntaxdiagramm} 
\end{tabular}
\medskip
 
F"ur Wiederholungen ist diese Konstruktion ausreichend, bei Alternativen
sollten jedoch mehr als zwei Pfade m"oglich sein, dies wird durch die
Umgebung \verb+Alternativen+ realisiert. Sie "ubernimmt "ahnlich obigen 
Befehlen zwei Parameter. \verb+x1+ enth"alt wie gehabt den ersten Pfad, 
\verb+x2+ hingegen den letzten der m"oglichen Pfade. 
Alle weiteren Pfade werden 
innerhalb der Umgebung getrennt durch \verb+\\+ angegeben, also wie in 
einer einspaltigen \verb+\tabular+--Umgebung, wobei auch auf die letzte
Zeile ein abschlie"sendes \verb+\\+ folgt.  
\medskip

\begin{tabular}{l|c}
  \begin{tabular}{l}
    \verb+\begin{Alternativen}+\\
    \verb+  {x1}{x2}+\\
    \verb+  x3\\+\\
    \verb+\end{Alternativen}+
  \end{tabular}  & 
  \begin{Syntaxdiagramm}\dx
    \begin{Alternativen}
      {x1}{x2}
      x3 \\
    \end{Alternativen}
  \dx\end{Syntaxdiagramm} 
\end{tabular}
\medskip

\section{Setzen von Syntaxdiagrammen}
Praktisch geht man nun so vor, da"s innerhalb der "au"seren Umgebung
im einfachsten Fall eine Folge von
Syntax--Variablen und Terminal--Symbolen aufgelistet wird. Hierzu
verwendet man die nach rechts gerichteten Varianten. 

\medskip

\begin{tabular}{l|c}
  \begin{tabular}{l}
    \verb+\begin{Syntaxdiagramm}%+\\
    \verb+  \TerSymR{xx}%+\\
    \verb+\end{Syntaxdiagramm}+
  \end{tabular}  & 
  \begin{Syntaxdiagramm}%
    \TerSymR{xx}%
  \end{Syntaxdiagramm} 
\end{tabular}
\medskip

\begin{tabular}{l|c}
  \begin{tabular}{l}
    \verb+\begin{Syntaxdiagramm}%+\\
    \verb+  \SynVarR{xx}%+\\
    \verb+\end{Syntaxdiagramm}+
  \end{tabular}  & 
  \begin{Syntaxdiagramm}%
    \SynVarR{xx}%
  \end{Syntaxdiagramm} 
\end{tabular}
\medskip

\begin{tabular}{l|c}
  \begin{tabular}{l}
    \verb+\begin{Syntaxdiagramm}%+\\
    \verb+  \TerSymR{xx}%+\\
    \verb+  \SynVarR{xx}%+\\
    \verb+\end{Syntaxdiagramm}+
  \end{tabular}  & 
  \begin{Syntaxdiagramm}%
    \TerSymR{xx}%
    \SynVarR{xx}%
  \end{Syntaxdiagramm} 
\end{tabular}
\medskip

Etwas anspruchsvoller k"onnen Syntax--Variablen und Terminal--Symbole in
Strukturen auftreten. In diesem Fall sorgt die umgebende Struktur f"ur die 
notwendigen Pfeile, man verwendet also die ungerichteten Befehle
\verb+\TerSym{}+ und \verb+\SynVar{}+:
\medskip

\begin{tabular}{l|c}
  \begin{tabular}{l}
    \verb+\begin{Syntaxdiagramm}%+\\
    \verb+  \Wiederholung%+\\
    \verb+    {\TerSym{x1}}%+\\
    \verb+    {\SynVar{x2}}%+\\
    \verb+\end{Syntaxdiagramm}+ 
  \end{tabular}  & 
  \begin{Syntaxdiagramm}%
    \Wiederholung%
      {\TerSym{x1}}%
      {\SynVar{x2}}%
  \end{Syntaxdiagramm} 
\end{tabular}
\medskip

\begin{tabular}{l|c}
  \begin{tabular}{l}
    \verb+\begin{Syntaxdiagramm}%+\\
    \verb+  \Alternative%+\\
    \verb+    {\TerSym{x1}}%+\\
    \verb+    {\SynVar{x2}}%+\\
    \verb+\end{Syntaxdiagramm}+ 
  \end{tabular}  & 
  \begin{Syntaxdiagramm}%
    \Alternative%
      {\TerSym{x1}}%
      {\SynVar{x2}}%
  \end{Syntaxdiagramm} 
\end{tabular}
\medskip

In der \verb+Alternativen+--Umgebung m"u"sen die inneren Pfade selbst f"ur
ihre Pfeile sorgen (hier: Terminal--Symbol x3).
\medskip

\begin{tabular}{l|c}
  \begin{tabular}{l}
    \verb+\begin{Syntaxdiagramm}%+\\
    \verb+  \begin{Alternativen}%+\\
    \verb+    {\TerSym{x1}}%+\\
    \verb+    {\SynVar{x2}}%+\\
    \verb+    \TerSymR{x3}\\%+\\
    \verb+  \end{Alternativen}%+\\
    \verb+\end{Syntaxdiagramm}+ 
  \end{tabular}  & 
    \begin{Syntaxdiagramm}%
      \begin{Alternativen}%
        {\TerSym{x1}}%
        {\SynVar{x2}}%
        \TerSymR{x3}\\%
      \end{Alternativen}%
    \end{Syntaxdiagramm}    
\end{tabular}
\medskip

Schlie"slich k"onnen auch Strukturen beliebig kombiniert und 
verschachtelt werden:
\medskip

\begin{tabular}{l|c}
  \begin{tabular}{l}
    \verb+\begin{Syntaxdiagramm}%+\\
    \verb+  \Alternative%+\\
    \verb+    {\Wiederholung%+\\
    \verb+      {\TerSym{x1}}%+\\
    \verb+      {\TerSym{x1}}}%+\\
    \verb+    {\SynVar{x2}}%+\\
    \verb+\end{Syntaxdiagramm}+ 
  \end{tabular}  & 
  \begin{Syntaxdiagramm}%
    \Alternative%
      {\Wiederholung%
        {\TerSym{x1}}%
        {\TerSym{x1}}}%
      {\SynVar{x2}}%
  \end{Syntaxdiagramm} 
\end{tabular}
\medskip

Die vielen Kommentarzeichen (\%) in der Eingabe sind nicht an allen Stellen
notwendig. Es ist aber in jedem Fall einfacher alle Zeilen mit einem \% zu
beenden, als jeweils zu "uberlegen, ob es eventuell weggelassen werden 
k"onnte. 
Sollte ein notwendiges \%-Zeichen vergessen werden, so erscheint an 
entsprechender Stelle im Diagramm ein Leerzeichen. Umgekehrt weisen solche
Br"uche im Diagramm auf vergessene Kommentarzeichen hin.
\medskip

\begin{tabular}{l|c}
  \begin{tabular}{l}
    \verb+\begin{Syntaxdiagramm}%+\\
    \verb+  \TerSymR{xx}+\\
    \verb+  \TerSymR{xx}+\\
    \verb+\end{Syntaxdiagramm}+
  \end{tabular}  & 
  \begin{Syntaxdiagramm}%
    \TerSymR{xx}
    \TerSymR{xx}
  \end{Syntaxdiagramm} 
\end{tabular}
\medskip  

Sollte ein Syntaxdiagramm zu lang f"ur eine einzelne Zeile werden, so 
kann ein Teildiagramm zu einer Syntax-Variable zusammengezogen
werden und diese in einem weiteren Diagramm aufgeschl"usselt werden.

Ist dies nicht erw"unscht, so kann die \verb+Syntaxdiagramm+-Umgebung
mit einem optionalen Parameter aufgerufen werden. Der Inhalt des
Parameters ersetzt dann als Ende--Symbol den abschlie"senden Pfeil.
Verwendet man den Befehl \verb+\Dots+ so kann man das Diagramm mit
einem weiteren in der n"achsten Zeile fortsetzen. Dieses kann durch 
eine voranstehendes \verb+\quad+ etwas einger"uckt werden.
\medskip

\begin{tabular}{l|c}
  \begin{tabular}{l}
    \verb+\begin{Syntaxdiagramm}[\Dots]%+\\
    \verb+  \TerSymR{x1}+\\
    \verb+\end{Syntaxdiagramm} +\\
    \verb+\quad+\\
    \verb+\begin{Syntaxdiagramm}%+\\
    \verb+  \TerSymR{x2}+\\
    \verb+\end{Syntaxdiagramm} +
  \end{tabular}  & 
  \begin{tabular}{l}
  \begin{Syntaxdiagramm}[\Dots]%
    \TerSymR{x1}%
  \end{Syntaxdiagramm} \\
  \quad
  \begin{Syntaxdiagramm}%
    \TerSymR{x2}%
  \end{Syntaxdiagramm} 
  \end{tabular}
\end{tabular}
\medskip  

Schlie"slich gibt es den Stapel Befehl der zwei Teile eines Diagramms
untereinander setzt und sie durch einen r"ucklaufenden Pfeil verkn"upft.
Oft ist das Aufteilen ohne r"ucklaufenden Pfeil allerdings sowohl
einfacher als auch "ubersichtlicher.
\medskip

\begin{tabular}{l|c}
  \begin{tabular}{l}
    \verb+\begin{Syntaxdiagramm}%+\\
    \verb+  \Stapel{\TerSym{x1}}%+\\
    \verb+         {\TerSym{x2}}%+\\
    \verb+\end{Syntaxdiagramm} +\\
  \end{tabular}  & 
  \begin{Syntaxdiagramm}%
    \Stapel{\TerSym{x1}}%
           {\TerSym{x2}}%
  \end{Syntaxdiagramm} 
\end{tabular}
\medskip  
  
\section{Weitere Hinweise}

\begin{itemize}
\item
Einige Konstruktionen erzeugen nur dann Verbindungslinien korrekter L"ange,
wenn auf den Pfaden Eintr"age mit der richtigen H"ohe bzw. Tiefe auftreten.
Im ersten Beispiel zur Umgebung \verb+Alternativen + ist dies nicht der Fall.
\item
In der Datei syntax.tex sind einige Anmerkungen und eine kleine Liste 
m"oglicher Erweiterungen. Wer Zeit und Lust hat darf sich gerne austoben
und das eine oder andere erledigen.
\item
Kritik, Anregungen und eigene Beitr"age sind in jedem Fall erw"unscht. 
Vielleicht finde ich bei ausreichendem Interesse sogar die Zeit noch ein 
wenig an diesem Paket zu basteln.
\end{itemize}

\section{Kontakt}

\begin{verbatim}

       --------------------------------------
                    Bernd Worsch
                 Doberanerstrasse 99 
                   18057 Rostock
          bernd.worsch@stud.uni-rostock.de
                  email preferred
       --------------------------biichi-ji---

\end{verbatim}

\end{document}








