%Format: PDF

\documentclass[a4paper,11pt]{article}

\addtolength{\textwidth}{10mm}
\addtolength{\oddsidemargin}{-5mm}
\setlength{\parindent}{0mm}
\setlength{\parskip}{.5\baselineskip}

\usepackage{tabvar}
%\usepackage[FlechesPS]{tabvar}
%\usepackage[FlechesMP]{tabvar}

\usepackage[latin1]{inputenc}
\usepackage[T1]{fontenc}
\usepackage[frenchb]{babel}

\IfFileExists{lmodern.sty}{\usepackage{lmodern}}{}

\begin{document}
\thispagestyle{empty}

\begin{center}
  \Large\bfseries Exemples de tableaux de variations avec \texttt{tabvar}
\end{center}

Un exemple simple : 
$\displaystyle f(x)=\frac{x^3+2}{2x} \qquad f'(x)=\frac{x^3-1}{x^2}$.

\[\begin{tabvar}{|C|CCCCCCCCC|} \hline
 x    &-\infty &   &-\sqrt[3]{2} &   & 0       &   & 1  &    &+\infty      
\\ \hline
f'(x) &        & - &             & - & \dbarre & - & 0  & +  & 
\\ \hline
\niveau{3}{3}f(x)  
      &+\infty                       &\decroit 
      &0                             &\decroit
      &\discont{-\infty}{<}{+\infty} &\decroit
      &\frac{3}{2}                   &\croit
      &+\infty
\\ \hline
\end{tabvar}\]

Le codage du tableau est le suivant :
\begin{verbatim}
\[\begin{tabvar}{|C|CCCCCCCCC|} \hline
 x    &-\infty &  &-\sqrt[3]{2} &  &0       &  & 1 &  &+\infty      
\\ \hline
f'(x) &        &- &             &- &\dbarre &- & 0 &+ & 
\\ \hline
\niveau{3}{3}f(x)  
      &+\infty                       &\decroit 
      &0                             &\decroit
      &\discont{-\infty}{<}{+\infty} &\decroit
      &\frac{3}{2}                   &\croit
      &+\infty
\\ \hline
\end{tabvar}\]
\end{verbatim}

L'argument optionnel de \verb|\discont| n'a pas �t� utilis�, on obtiendrait
une meilleure pr�sentation en lui donnant la valeur~1, ce qui �carterait
d'un interligne les valeurs $+\infty$ et~$-\infty$, mettant ainsi
les trois valeurs~$+\infty$ sur la m�me ligne.

D'autre part, $f(x)$ est plac� au niveau~3 par la commande \verb|\niveau|.
Si on souhaitait que $f(x)$ soit plac� plus bas, au niveau~2 par exemple,
il faudrait coder :\\
\verb|\niveau{2}{3}f(x)} &\niveau{3}{3}+\infty}|

Voici le r�sultat obtenu avec ces deux modifications :
\[\begin{tabvar}{|C|CCCCCCCCC|} \hline
 x    &-\infty &  &-\sqrt[3]{2} &  &0       &  & 1 &  &+\infty      
\\ \hline
f'(x) &        &- &             &- &\dbarre &- & 0 &+ & 
\\ \hline
\niveau{2}{3}f(x)  
      &\niveau{3}{3}+\infty             &\decroit 
      &0                                &\decroit
      &\discont[1]{-\infty}{<}{+\infty} &\decroit
      &\frac{3}{2}                      &\croit
      &+\infty
\\ \hline
\end{tabvar}\]

Une pr�sentation plus traditionnelle du tableau de variations serait la
suivante (on renonce � l'utilisation de \verb|\discont| et on remplace
la colonne \texttt{C} par trois colonnes \texttt{LCR}, la colonnne
centrale contenant une double barre).  On ajoute �galement des filets
verticaux pour les valeurs remarquables de la fonction ou de sa d�riv�e
gr�ce � la commande \verb|\barre{}|%
\footnote{Cette commande n'est disponible que depuis la version 1.1 (mai~2007)
  de \texttt{tabvar}.}
(argument \emph{obligatoire}, �ventuellement vide).

\[\begin{tabvar}{|C|CCCCLCRCCCC|} \hline
 x    &-\infty &  &-\sqrt[3]{2} &  & &0       & &  & 1         &  &+\infty
\\ \hline
f'(x) &        &- & \barre{}    &- & &\dbarre & &- & \barre{0} &+ & 
\\ \hline
\niveau{2}{3}f(x)  
      &\niveau{3}{3}+\infty                    &\decroit 
      &\barre{0}                               &\decroit
      &-\infty  &\dbarre &\niveau{3}{3}+\infty &\decroit
      &\barre{\frac{3}{2}}                     &\croit
      &+\infty
\\ \hline
\end{tabvar}\]

Le codage est le suivant :
\begin{verbatim}
\[\begin{tabvar}{|C|CCCCLCRCCCC|} \hline
 x    &-\infty &  &-\sqrt[3]{2} &  & &0       & &  & 1         &  &+\infty
\\ \hline
f'(x) &        &- & \barre{}    &- & &\dbarre & &- & \barre{0} &+ & 
\\ \hline
\niveau{2}{3}f(x)  
      &\niveau{3}{3}+\infty                    &\decroit 
      &\barre{0}                               &\decroit
      &-\infty  &\dbarre &\niveau{3}{3}+\infty &\decroit
      &\barre{\frac{3}{2}}                     &\croit
      &+\infty
\\ \hline
\end{tabvar}\]
\end{verbatim}

Noter la pr�sence de la seconde commande \verb|\niveau| pour
positionner le terme \verb|+\infty| au niveau~3 apr�s la discontinuit�.

\newpage
Un exemple de courbe param�tr�e :
$\displaystyle x(t)= t + \frac{1}{t}\qquad y(t) = t + \frac{1}{2t^2}$.

\[
\begin{tabvar}{|C|CCRCCCCCC|} \hline
 t    &-\infty &   &-1  &   & 0       &   & 1  &   &+\infty      
\\ \hline
x'(t) &        &+  & 0  & - & \dbarre & - & 0  & + & 
\\ \hline
\niveau{1}{3}
x(t) &-\infty                                &\croit 
           &-2                               &\decroit 
           &\discont[1]{-\infty}{<}{+\infty} &\decroit 
           &2                                &\croit
           &+\infty
\\ \hline
\niveau{1}{3}
y(t) &-\infty            &\croit 
           &-\frac{1}{2} &\croit
           &+\infty      &\decroit
           &\frac{3}{2}  &\croit
           &+\infty
\\ \hline
y'(t) &        &+  &2   & + & \dbarre & - & 0  &+   &             
\\ \hline
\end{tabvar}
\]

Le codage est le suivant :
\begin{verbatim}
\[\begin{tabvar}{|C|CCRCCCCCC|} \hline
 t    &-\infty &   &-1  &   & 0       &   & 1  &   &+\infty      
\\ \hline
x'(t) &        &+  & 0  & - & \dbarre & - & 0  & + & 
\\ \hline
\niveau{1}{3}
x(t) &-\infty                                &\croit 
           &-2                               &\decroit 
           &\discont[1]{-\infty}{<}{+\infty} &\decroit 
           &2                                &\croit
           &+\infty
\\ \hline
\niveau{1}{3}
y(t) &-\infty            &\croit 
           &-\frac{1}{2} &\croit
           &+\infty      &\decroit
           &\frac{3}{2}  &\croit
           &+\infty
\\ \hline
y'(t) &        &+  &2   & + & \dbarre & - & 0  &+   &             
\\ \hline
\end{tabvar}\]
\end{verbatim}

\newpage
Le m�me tableau de variations en pr�sentation \og traditionnelle \fg.

\[\begin{tabvar}{|C|CCCCRCLCCCC|} \hline
 t    &-\infty &  &-1        &  & &0       & &  & 1        &  &+\infty
\\ \hline
x'(t) &        &+ &\barre{0}
                             &- & &\dbarre & &- &\barre{0} &+ & 
\\ \hline
\niveau{1}{3}
x(t) &-\infty                                       &\croit 
           &\barre{-2}                              &\decroit 
           &-\infty  &\dbarre &\niveau{3}{3}+\infty &\decroit 
           &\barre{2}                               &\croit
           &+\infty
\\ \hline
\niveau{1}{3}
y(t) &-\infty                         &\croit 
           &-\frac{1}{2}              &\croit
           &+\infty &\dbarre &+\infty &\decroit
           &\barre{\frac{3}{2}}       &\croit
           &+\infty
\\ \hline
y'(t) &        &+ &2         &+ & &\dbarre & &- &\barre{0} &+ &             
\\ \hline
\end{tabvar}\]

Le codage est le suivant :
\begin{verbatim}
\[\begin{tabvar}{|C|CCCCRCLCCCC|} \hline
 t    &-\infty &  &-1        &  & &0       & &  & 1        &  &+\infty
\\ \hline
x'(t) &        &+ &\barre{0}
                             &- & &\dbarre & &- &\barre{0} &+ & 
\\ \hline
\niveau{1}{3}
x(t) &-\infty                                       &\croit 
           &\barre{-2}                              &\decroit 
           &-\infty  &\dbarre &\niveau{3}{3}+\infty &\decroit 
           &\barre{2}                               &\croit
           &+\infty
\\ \hline
\niveau{1}{3}
y(t) &-\infty                         &\croit 
           &-\frac{1}{2}              &\croit
           &+\infty &\dbarre &+\infty &\decroit
           &\barre{\frac{3}{2}}       &\croit
           &+\infty
\\ \hline
y'(t) &        &+ &2         &+ & &\dbarre & &- &\barre{0} &+ &             
\\ \hline
\end{tabvar}\]
\end{verbatim}

Noter que le type de la colonne $t=-1$ a d� �tre chang� de \texttt{R} �
\texttt{C} pour permettre l'ajout du filet vertical.

\newpage
Le m�me tableau encore, mais cette fois on utilise les fl�ches
dessin�es en MetaPost.
\begingroup
\renewcommand{\FlecheC}{\FlecheCm}
\renewcommand{\FlecheD}{\FlecheDm}
\renewcommand{\FlecheH}{\FlecheHm}

\[\begin{tabvar}{|C|CCRCRCLCCCC|} \hline
 t    &-\infty &  &-1  &   & &0       &  &   & 1 &   &+\infty      
\\ \hline
x'(t) &        &+ & 0  & - & &\dbarre &  & - & 0 & + & 
\\ \hline
\niveau{1}{3}
x(t) &-\infty                                       &\croit 
           &-2                                      &\decroit 
           &-\infty  &\dbarre &\niveau{3}{3}+\infty &\decroit 
           &2                                       &\croit
           &+\infty
\\ \hline
\niveau{1}{3}
y(t) &-\infty                         &\croit 
           &-\frac{1}{2}              &\croit
           &+\infty &\dbarre &+\infty &\decroit
           &\frac{3}{2}               &\croit
           &+\infty
\\ \hline
y'(t) &        &+  &2   & + & & \dbarre & & - & 0  &+   &             
\\ \hline
\end{tabvar}\]
\endgroup

Le choix entre les fl�ches MetaPost et celles de Michel \textsc{Bovani} se
fait normalement soit � l'aide des options 
%\texttt{FlechesMP} ou\texttt{FlechesPS} 
de \texttt{tabvar} (\verb|\usepackage[FlechesMP]{tabvar}|)
soit dans le pr�ambule ou dans le fichier \texttt{tabvar.cfg}, � l'aide du
drapeau \verb|\FlechesMP| : \verb|\FlechesMPtrue| pour les 
fl�ches MetaPost (par d�faut les fl�ches 
\og bovaniennes \fg{} sont utilis�es).

Un exemple de fonction non d�finie partout :
$\displaystyle f(x) = \sqrt{\frac{x-1}{x+1}}$. 

\[\begin{tabvar}{|C|CCRNLCC|} \hline
 x    &-\infty &       &-1       &\hspace*{15mm} & 1              &    &+\infty
\\ \hline
 f'(x) &       & +     &         &               &+\infty         & +  &
\\ \hline
\niveau{1}{2}
f(x)  &1     &\croit &+\infty   &               
          &\niveau{1}{2}0 &\croit & 1
\\ \hline
\end{tabvar}\]

Le codage est le suivant :
\begin{verbatim}
\[\begin{tabvar}{|C|CCRNLCC|} \hline
 x     &-\infty &  &-1 &\hspace*{15mm} & 1      &  &+\infty
\\ \hline
 f'(x) &        &+ &   &               &+\infty &+ &
\\ \hline
\niveau{1}{2}
f(x)  &1     &\croit &+\infty   &               
          &\niveau{1}{2}0 &\croit & 1
\\ \hline
\end{tabvar}\]
\end{verbatim}

La largeur de la colonne gris�e est fix�e �~15mm par le \verb|\hspace*{15mm}|
plac� dans une ligne quelconque du tableau. Certains visualiseurs
(Xdvi par exemple) n'affichent pas correctement les couleurs ;
en cas de doute, v�rifier sur une sortie PostScript ou PDF. 

Noter l'emploi d'une seconde commande \verb|\niveau{1}{2}| pour positionner
la valeur de~$f$ au point~1 (sans celle-ci, cette valeur serait plac�e
au niveau de la valeur pr�c�dente, ici~$+\infty$. 

Si on prolongeait la d�finition de $f$ en posant $f(x)=0$ sur $[-1,1]$
on aurait le tableau suivant :

\[\begin{tabvar}{|C|CCRCCCCC|} \hline
 x     &-\infty &       &        &-1      &   & 1      &  &+\infty
\\ \hline
 f'(x) &        &+      &        &\dbarre & 0 &+\infty &+ &
\\ \hline
\niveau{1}{2}
f(x)  &1        &\croit &+\infty &\niveau{1}{2}0 
                                   &\constante &0 &\croit & 1
\\ \hline
\end{tabvar}\]

Le codage est le suivant :

\begin{verbatim}
\[\begin{tabvar}{|C|CCRCCCCC|} \hline
 x     &-\infty &       &        &-1      &   & 1      &  &+\infty
\\ \hline
 f'(x) &        &+      &        &\dbarre & 0 &+\infty &+ &
\\ \hline
\niveau{1}{2}
f(x)  &1        &\croit &+\infty &\niveau{1}{2}0 
                                   &\constante &0 &\croit & 1
\\ \hline
\end{tabvar}\]
\end{verbatim}

\end{document}

%%% Local Variables: 
%%% mode: latex
%%% TeX-master: t
%%% coding: latin-1
%%% End: 
