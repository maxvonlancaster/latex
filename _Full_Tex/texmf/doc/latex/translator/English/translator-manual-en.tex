% Copyright 2007 by Till Tantau
%
% This file may be distributed and/or modified
%
% 1. under the LaTeX Project Public License and/or
% 2. under the GNU Public License.
%
% See the documentation file for more details.

\documentclass{ltxdoc}

\usepackage[left=2.25cm,right=2.25cm,top=2.5cm,bottom=2.5cm,nohead]{geometry}
\usepackage{makeidx}
\usepackage[latin1]{inputenc}
\usepackage{xcolor}
\usepackage{translator}

% Copyright 2007 by Till Tantau
%
% This file may be distributed and/or modified
%
% 1. under the LaTeX Project Public License and/or
% 2. under the GNU Public License.
%
% See the documentation file for more details.

\providecommand\href[2]{\texttt{#1}}

\colorlet{examplefill}{yellow!80!black}
\definecolor{graphicbackground}{rgb}{0.96,0.96,0.8}
\definecolor{codebackground}{rgb}{0.8,0.8,1}

\newenvironment{translatormanualentry}{\list{}{\leftmargin=2em\itemindent-\leftmargin\def\makelabel##1{\hss##1}}}{\endlist}
\newcommand\translatormanualentryheadline[1]{\itemsep=0pt\parskip=0pt\item\strut#1\par\topsep=0pt}
\newcommand\translatormanualbody{\parskip3pt}


\newenvironment{command}[1]{
  \begin{translatormanualentry}
    \extractcommand#1\@@
    \translatormanualbody
}
{
  \end{translatormanualentry}
}

\def\extractcommand#1#2\@@{%
  \translatormanualentryheadline{\declare{\texttt{\string#1}}#2}%
  \removeats{#1}%
  \index{\strippedat @\protect\myprintocmmand{\strippedat}}}


\renewenvironment{environment}[1]{
  \begin{translatormanualentry}
    \extractenvironement#1\@@
    \translatormanualbody
}
{
  \end{translatormanualentry}
}

\def\extractenvironement#1#2\@@{%
  \translatormanualentryheadline{{\ttfamily\char`\\begin\char`\{\declare{#1}\char`\}}#2}%
  \translatormanualentryheadline{{\ttfamily\ \ }\meta{environment contents}}%
  \translatormanualentryheadline{{\ttfamily\char`\\end\char`\{\declare{#1}\char`\}}}%
  \index{#1@\protect\texttt{#1} environment}%
  \index{Environments!#1@\protect\texttt{#1}}}



\newenvironment{package}[1]{
  \begin{translatormanualentry}
    \translatormanualentryheadline{{\ttfamily\char`\\usepackage\opt{[\meta{options}]}\char`\{\declare{#1}\char`\}}}
    \index{#1@\protect\texttt{#1} package}%
    \index{Packages and files!#1@\protect\texttt{#1}}%
    \translatormanualbody
}
{
  \end{translatormanualentry}
}



\newenvironment{filedescription}[1]{
  \begin{translatormanualentry}
    \translatormanualentryheadline{File {\ttfamily\declare{#1}}}%
    \index{#1@\protect\texttt{#1} file}%
    \index{Packages and files!#1@\protect\texttt{#1}}%
    \translatormanualbody
}
{
  \end{translatormanualentry}
}


\newenvironment{packageoption}[1]{
  \begin{translatormanualentry}
    \translatormanualentryheadline{{\ttfamily\char`\\usepackage[\declare{#1}]\char`\{translator\char`\}}}
    \index{#1@\protect\texttt{#1} package option}%
    \index{Package options for \textsc{translator}!#1@\protect\texttt{#1}}%
    \translatormanualbody
}
{
  \end{translatormanualentry}
}



\newcommand\opt[1]{{\color{black!50!green}#1}}
\newcommand\ooarg[1]{{\ttfamily[}\meta{#1}{\ttfamily]}}

\def\opt{\afterassignment\translatormanualopt\let\next=}
\def\translatormanualopt{\ifx\next\bgroup\bgroup\color{black!50!green}\else{\color{black!50!green}\next}\fi}



\def\beamer{\textsc{beamer}}
\def\pdf{\textsc{pdf}}
\def\translatorname{\textsc{translator}}
\def\tikzname{Ti\emph{k}Z}
\def\pstricks{\textsc{pstricks}}
\def\prosper{\textsc{prosper}}
\def\seminar{\textsc{seminar}}
\def\texpower{\textsc{texpower}}
\def\foils{\textsc{foils}}

{
  \makeatletter
  \global\let\myempty=\@empty
  \global\let\mygobble=\@gobble
  \catcode`\@=12
  \gdef\getridofats#1@#2\relax{%
    \def\getridtest{#2}%
    \ifx\getridtest\myempty%
      \expandafter\def\expandafter\strippedat\expandafter{\strippedat#1}
    \else%
      \expandafter\def\expandafter\strippedat\expandafter{\strippedat#1\protect\printanat}
      \getridofats#2\relax%
    \fi%
  }

  \gdef\removeats#1{%
    \let\strippedat\myempty%
    \edef\strippedtext{\stripcommand#1}%
    \expandafter\getridofats\strippedtext @\relax%
  }
  
  \gdef\stripcommand#1{\expandafter\mygobble\string#1}
}

\def\printanat{\char`\@}

\def\declare{\afterassignment\translatormanualdeclare\let\next=}
\def\translatormanualdeclare{\ifx\next\bgroup\bgroup\color{red!75!black}\else{\color{red!75!black}\next}\fi}

\def\example{\par\smallskip\noindent\textit{Example: }}

\def\itemoption#1{\item \declare{\texttt{#1}}%
  \indexoption{#1}%
}

\def\indexoption#1{%
  \index{#1@\protect\texttt{#1} option}%
  \index{Options!#1@\protect\texttt{#1}}%
}

\let\textoken=\command
\let\endtextoken=\endcommand

\def\myprintocmmand#1{\texttt{\char`\\#1}}

\makeatletter
\def\index@prologue{\section*{Index}\addcontentsline{toc}{section}{Index}
  This index only contains automatically generated entries, sorry. A good
  index should also contain carefully selected keywords. 
  \bigskip
}
\c@IndexColumns=2
  \def\theindex{\@restonecoltrue
    \columnseprule \z@  \columnsep 35\p@
    \twocolumn[\index@prologue]%
       \parindent -30pt
       \columnsep 15pt
       \parskip 0pt plus 1pt
       \leftskip 30pt
       \rightskip 0pt plus 2cm
       \small
       \def\@idxitem{\par}%
    \let\item\@idxitem \ignorespaces}
  \def\endtheindex{\onecolumn}
\def\noindexing{\let\index=\@gobble}

\makeatother


%%% Local Variables: 
%%% mode: latex
%%% TeX-master: "beameruserguide"
%%% End: 


\makeindex



\begin{document}

\title{The Translator Package\\
  Manual for Version \translatorversion\\[1mm]
\large\href{http://sourceforge.net/projects/latex-beamer}{\texttt{http://sourceforge.net/projects/latex-beamer}}}
\author{Till Tantau\\
  \href{mailto:tantau@users.sourceforge.net}{\texttt{tantau@users.sourceforge.net}}}


\maketitle

\vskip0pt plus 1fill

\parindent=0pt
Copyright 2007 by Till Tantau

\medskip  
Permission is granted to copy, distribute and/or modify all files of
this package under the terms of the \textsc{gnu} Public License, Version 2
or any later version published by the Free Software Foundation.
A copy of the license is included in the section entitled \textsc{gnu}
Public License.

\medskip  
Permission is also granted to distribute and/or modify all files of
this package under the conditions of the LaTeX
Project Public License, either version 1.3 of this license or (at
your option) any later version. A copy of the license is included in
the section entitled \LaTeX\ Project Public License. 

\clearpage

\tableofcontents

\clearpage


\section{Introduction}

\subsection{Overview of the Package}

The translator package is a \LaTeX\ package that provides a flexible
mechanism for translating individual words into different languages.
For example, it can be used to translate a word like ``figure'' into,
say, the German word ``Abbildung''. Such a translation mechanism is
useful when the author of some package would like to localize the
package such that texts are correctly translated into the language
preferred by the user. The translator package is \emph{not} intended
to be used to automatically translate more than a few words. 

You may wonder whether the translator package is really necessary
since there is the (very nice) |babel| package available for
\LaTeX. This package already provides translations for words like
``figure''. Unfortunately, the architecture of the babel package was
designed in such a way that there is no way of adding translations of
new words to the (very short) list of translations directly build into
babel.

The translator package was specifically designed to allow an easy
extension of the vocabulary. It is both possible to add new words that
should be translated and translations of these words.

The translator package can be used together with babel. In this case,
babel is used for language-specific things like special quotation
marks and input shortcuts, while translator is used for the
translation of words.  



\subsection{How to Read This Manual}

This manual explains the commands of the translator package and its
usage. The ``public'' commands and environments provided by the |translator|
package are described throughout the text. In each such description, the
described command, environment or option is printed in red. Text shown
in green is optional and can be left out.

In the following documentation, the installation is explained first,
followed by an overview of the basic concepts used. Then, I explain
the usage of the package.



\subsection{Contributing}

Since this package is about internationalization, it needs input from
people who can contribute translations to their native tongue.

In order to submit dictionaries, please do the following:
\begin{enumerate}
\item Read this manual and make sure you understand the basic concepts.
\item Find out whether the translations should be part of the
  translator package or part of another package. In general, submit
  translations and new keys to the translator project only if they are
  of public interest.

  For example, translations for keys like |figure| should be send to
  the translator project. Translations for keys that are part of a
  special package should be send to the author of the package.
\item If you are sure that the translations should go to the
  translator package, create a dictionary of the correct name (see
  this documentation once more).
\item Finally, submit the dictionary using the correct forum on the
  development site. 
\end{enumerate}




\subsection{Getting Help}

When you need help with the package, please do the following:

\begin{enumerate}
\item
  Read this manual, at least the part that has to do with your
  problem.
\item
  Consider rereading the manual, especially the part that has to do
  with your problem.
\item
  If that does not solve the problem, try having a look at the
  development page for translator (see the
  title of this document). Perhaps someone has already reported a
  similar problem and someone has found a solution.
\item
  On the website you will find numerous forums for getting
  help. There, you can write to help forums, file bug reports, join
  mailing lists, and so on.
\item
  Before you file a bug report, especially a bug report concerning the
  installation, make sure that this is really a bug. In particular,
  have a look at the |.log| file that results when you \TeX\ your
  files. This |.log| file should show that all the right files are
  loaded from the right directories. Nearly all installation problems
  can be resolved by looking at the |.log| file.
\item
  \emph{As a last resort} you can try to email me (the author). I do
  not mind getting emails, I simply get way too many of them. Because
  of this, I cannot guarantee that your emails will be answered timely
  or even at all. Your chances that your problem will be fixed are
  somewhat higher if you use the forums and mailing lists. (Naturally,
  I read the lists and answer questions when I have the time).
\item
  Please, do not phone me in my office. 
\end{enumerate}




\section{Installation}

This package is distributed under the \textsc{gpl} license and under
the \LaTeX\ public license, see Sections~\ref{section-gpl}
and~\ref{LPPL:LPPL}. 

Typically, the package will already be installed on your
system. Naturally, in this case you do not need to worry about the
installation process at all and you can skip the rest of this
section. 



\subsection{Package and Driver Versions}

This documentation is part of version \translatorversion\ of the
translator package. To use it, you just need a reasonably up-to-date
\LaTeX-system, but the requirements are rather low. There are no
special dependencies.


\subsection{Installing Prebundled Packages}

I do not create or manage prebundled packages of translator, but
perhaps other people might be willing to do so. If you have a
problem with installing such a bundle, you might wish to have a look
at the page of whoever created the package.


\subsection{Installation in a texmf Tree}

For a permanent installation, you place the files of the
the \textsc{translator} package in an appropriate |texmf| tree. 

When you ask \TeX\ to use a certain class or package, it usually looks
for the necessary files in so-called |texmf| trees. These trees
are simply huge directories that contain these files. By default,
\TeX\ looks for files in three different |texmf| trees:
\begin{itemize}
\item
  The root |texmf| tree, which is usually located at
  |/usr/share/texmf/| or |c:\texmf\| or somewhere similar.
\item
  The local  |texmf| tree, which is usually located at
  |/usr/local/share/texmf/| or |c:\localtexmf\| or somewhere similar.
\item
  Your personal  |texmf| tree, which is usually located in your home
  directory at |~/texmf/| or |~/Library/texmf/|.   
\end{itemize}

You should install the package either in the local tree or in
your personal tree, depending on whether you have write access to the
local tree. Installation in the root tree can cause problems, since an
update of the whole \TeX\ installation will replace this whole tree.

Inside whatever texmf-tree that you have chosen, create the
sub-directory |texmf/tex/latex/translator| and put everything from the
package in this directory.

Finally, you may need to run the program |texhash| to rebuild \TeX's
cache. In Mik\TeX, there is a menu option for this.


\subsection{Updating the Installation}

To update your installation from a previous version, all you need to
do is to replace everything in the directory |texmf/tex/latex/translator|
with the files of the new version. The easiest way to do this is to
first delete the old version and then proceed as described above.

Sometimes, there are changes in the syntax of certain command from
version to version. If things no longer work that used to work, you
may wish to have a look at the release notes and at the change log. 



\section{Basic Concepts}

\subsection{Keys}

The main purpose of the translator package is to provide translations
for \emph{keys}. Typically, a key is an English word like |Figure|
and the German translation for this key is ``Abbildung''.

For a concept like ``figures'' a single key typically is not enough:
\begin{enumerate}
\item
  It is sometimes necessary to translate a group of words like ``Table
  of figures'' as a whole. While these are three words in English,
  the German translation in just a single word:
  ``Abbildungsverzeichnis''.
\item
  Uppercase and lowercase letters may cause problems. Suppose we
  provide a translation for the key |Figure|. Then what happens
  when we want to use this word in normal text, spelled with a
  lowercase first letter? We could use \TeX's functions to turn the
  translation into lowercase, but that would be wrong with the German
  translation ``Abbildung'', which is always spelled with a capital
  letter.
\item 
  Plurals may also cause problems. If we know the translation for
  ``Figure'', that does not mean that we know the translation for
  ``Figures'' (which is ``Abbildungen'' in German).
\end{enumerate}

Because of these problems, there are many keys for the single concept
of ``figures'': |Figure|, |figure|, |Figures|, and |figures|. The
first key is used for translations of  ``figure'' when used
in a headline in singular. The last key is used for translations of
``figure'' when used in normal text in plural.

A key may contain spaces, so |Table of figures| is a permissible
key.

Keys are normally English texts whose English translation is the same
as the key, but this need not be the case. Theoretically, a key could
be anything. However, since the key is used as a last fallback when no
translation whatsoever is available, a key should be readable by itself.


\subsection{Language Names}

The translator package uses names for languages that are different
from the names used by other packages like babel. The reason for this
is that the names used by babel are a bit of a mess, so I decided to
clean things up for the translator package. However, mappings from
babel names to translator names are provided.

The names used by the translator package are the English names
commonly used for these languages. Thus, the name for the English
language is |English|, the name for German is |German|.

Variants of a language get their own name: The British version of
English is called |BritishEnglish|, the US-version is called
|AmericanEnglish|.

For German there is the special problem of pre-1998 as opposed to the
current (not yet fixed) spelling. The language |German| reflects the
current official spelling, but |German1997| refers to the spelling
used in 1997.


\subsection{Language Paths}

When you request a translation for a key, the translator package will
try to provide the translation for the current
\emph{language}. Examples of languages are German or English.

When the translator looks up the translation for the given key in the
current language, it may fail to find a translation. In this case, the
translator will try a fallback strategy: It keeps track of a
\emph{language path} and successively tries to find translations for
each language on this path.

Language paths are not only useful for fallbacks. They are also used
for situations where a language is a variant of another language. For
example, when the translator looks for the translation for a key in
Austrian, the language path starts with Austrian, followed by
German. Then, a dictionary for Austrian only needs to provide
translations for those keys where Austrian differs from German. 




\subsection{Dictionaries}

The translations of keys are typically provided by
\emph{dictionaries}. A dictionary contains the translations of a
specific set of keys into a specific language. For example, a
dictionary might contain the translations of the names of months
into the language German. Another dictionary might contain the
translations of the numbers into French.




\section{Usage}



\subsection{Basic Usage}

Here is a typical example of how to use the package:

\begin{verbatim}
\documentclass[german]{article}

\usepackage{babel}
\usepackage{some-package-that-uses-translator}

\begin{document}
...
\end{document}
\end{verbatim}

As can be seen, things really happen behind the scenes, so, typically,
you do not really need to do anything. It is the job of other package
to load the translator package, to load the dictionaries and to
request translations of keys.




\subsection{Providing Translations}

There are several commands to tell the translator package what the
translation of a given key is. As said before, as a normal author you
typically need not provide such translations explicitly, they are
loaded automatically. However, there are two situations in which you
need to provide translations:
\begin{enumerate}
\item You do not like the existing translation and you would like to
  provide a new one.
\item You are writing a dictionary.
\end{enumerate}

You provide a translation using one of the following commands:
\begin{command}{\newtranslation\oarg{options}\marg{key}\marg{translation}}
  This command defines the translation of \meta{key} to be
  \meta{translation} in the language specified by the \meta{options}.

  You can only use this command if the translation is really ``new''
  in the sense that no translation for the keys has yet been given for
  the language. If there is already a translation, an error message
  will be printed.
  
  The following \meta{options} may be given:
  \begin{itemize}
    \itemoption{to}|=|\meta{language}
    This options tells the translator, that the translation
    \meta{translation} of \meta{keys} applies to the language
    \meta{language}. 
  
    Inside a dictionary file (see Section~\ref{section-dictionaries}),
    this option is set automatically to the language of the
    dictionary.
  \end{itemize}

  \example
  |\newtranslation[to=German]{figure}{Abbildung}|

  \example
  |\newtranslation[to=German]{Figures}{Abbildungen}|
\end{command}

\begin{command}{\renewtranslation\oarg{options}\marg{key}\marg{translation}}
  This command works like |\newtranslation|, only it will redefine an
  existing translation.   
\end{command}

\begin{command}{\providetranslation\oarg{options}\marg{key}\marg{translation}}
  This command works like |\newtranslation|, but no error message will
  be printed if the translation already exists. It this case, the
  existing translation is not changed.

  This command should be used by dictionary authors since their
  translations should not overrule any translations given by document
  authors or other dictionary authors.
\end{command}

\begin{command}{\deftranslation\oarg{options}\marg{key}\marg{translation}}
  This command defines the translation ``no matter what''. An existing
  translation will be overwritten.

  This command should typically used by document authors to install
  their preferred translations.

  \example
  |\deftranslation[to=German]{figure}{Figur}|
\end{command}

Here is an example where a translation is provided by a document
author:

\begin{verbatim}
\documentclass[ngerman]{article}

\usepackage{babel}
\usepackage{some-package-that-uses-translator}

\deftranslation[to=German]{Sketch of proof}{Beweisskizze}

\begin{document}
  ...
\end{document}
\end{verbatim}





\subsection{Creating and Using Dictionaries}
\label{section-dictionaries}

Two kind of people will create \emph{dictionaries}: First, package
authors will create dictionaries containing translations for the
(new) keys used in the package. Second, document authors can create
their own private dictionaries that overrule settings from other
dictionaries or that provide missing translations.

There is not only one dictionary per language. Rather, many different
dictionaries may be used by \textsc{translator} when it tries to find a
translation. This makes it easy to add new translations: Instead of
having to change translator's main dictionaries (which involves, among
other things, the release of a new version of the |translator| package),
package authors can just add a new dictionary containing just the keys
needed for the package.

Dictionaries are named according to the following rule: The name of
the dictionary must start with its \emph{kind}. The kind tells
translator which kind of keys the dictionary contains. For example,
the dictionaries of the kind |translator-months-dictionary| contain
keys like |January| (note that this is a key, not a translation).
Following the kind, the name of a dictionary must have a dash. Then
comes the language for which the dictionary file provides
translations. Finally, the file name must end with |.dict|.

To continue the example of the month dictionary, for the German
language the dictionary is called
\begin{verbatim}
translator-months-dictionary-German.dict
\end{verbatim}
Its contents is the following:
\begin{verbatim}
\ProvidesDictionary{translator-months-dictionary}{German}

\providetranslation{January}{Januar}
\providetranslation{February}{Februar}
\providetranslation{March}{M\"arz}
\providetranslation{April}{April}
\providetranslation{May}{Mai}
\providetranslation{June}{Juni}
\providetranslation{July}{Juli}
\providetranslation{August}{August}
\providetranslation{September}{September}
\providetranslation{October}{Oktober}
\providetranslation{November}{November}
\providetranslation{December}{Dezember}
\end{verbatim}

Note that the |\providetranslation| command does not need the option
|[to=German]|. Inside a dictionary file \textsc{translator} will always set
the default translation language to the language provided by the
dictionary. However, you can still specify the language, if you
prefer. 

The |\ProvidesDictionary| command currently only prints a message in
the log-files.

\begin{command}{\ProvidesDictionary\marg{kind}\marg{language}\oarg{version}}
  This command currently only prints a message in the log-files. The
  format is the same as for \LaTeX's |\ProvidesPackage| command.
\end{command}

Dictionaries are stored in a decentralized manner: A special
dictionary for a package will typically be stored somewhere in the
vicinity of the package. For this reasons, \textsc{translator} needs to be
told which \emph{kinds} of dictionaries should be loaded and which
\emph{languages}  should be used. This is accomplished using the
following two commands:

\begin{command}{\usedictionary\marg{kind}}
  This command tells the |translator| package, that at the beginning of
  the document it should load all dictionaries of kind \meta{kind} for
  the languages used in the document. Note that the dictionaries are
  not loaded immediately, but only at the beginning of the document.

  If no dictionary of the given \emph{kind} exists for one of the
  language, nothing bad happens.

  Invocations of this command accumulate, that is, you can call it
  multiple times for different dictionaries.
\end{command}

\begin{command}{\uselanguage\marg{list of languages}}
  This command tells the |translator| package that it should load the
  dictionaries for all languages in the \meta{list of languages}. The
  dictionaries are loaded at the beginning of the document.
\end{command}

Here is an example of how all of this works: Suppose you wish to
create a new package for drawing, say, chess boards. Let us call this
package |chess|. In the file |chess.sty| we could now write the
following:

\begin{verbatim}
// This is chess.sty

\RequirePackage{translator}
\usedictionary{chess}

...

\newcommand\MoveKnight[2]{%
  ...
  \translate{knight}
  ...
}
\end{verbatim}

Now we create dictionaries like the following:
\begin{verbatim}
// This is chess-German.dict
\ProvidesDictionary{chess}{German}

\providetranslation{chess}{Schach}
\providetranslation{knight}{Springer}
\providetranslation{bishop}{L\"aufer}
...
\end{verbatim}

and

\begin{verbatim}
// This is chess-English.dict
\ProvidesDictionary{chess}{English}

\providetranslation{chess}{chass}
\providetranslation{knight}{knight}
\providetranslation{bishop}{bishop}
...
\end{verbatim}

Here are a few things to note:
\begin{itemize}
\item
  The package |chess.sty| does not use the command
  |\uselanguage|. After all, the package does not know (or care) about
  the language used in the final document. It only needs to tell the
  translator package that it will use the dictionary |chess|.
\item
  You may wonder why we need an English dictionary. After all, the
  keys themselves are the ultimate fallbacks if no other translation
  is available. The answer to this question is that, first of all,
  English should be treated like any other language. Second, there are
  some situations in which there is a ``better'' English translation
  than the key itself. An example is explained next.
\item
  The keys we chose may not be optimal. What happens, if some other
  package, perhaps on medieval architecture, also needs translations
  of knights and bishops. However, in this different context, the
  translations of knight and bishop are totally different, namely
  |Ritter| and |Bischof|.

  Thus, it might be a good idea to add something to the key to make it
  clear that the ``chess bishop'' is meant:
\begin{verbatim}
// This is chess-German.dict
\providetranslation{knight (chess)}{Springer}
\providetranslation{bishop (chess)}{L\"aufer}
\end{verbatim}

\begin{verbatim}
// This is chess-English.dict
\providetranslation{knight (chess)}{knight}
\providetranslation{bishop (chess)}{bishop}
\end{verbatim}
\end{itemize}


\subsection{Creating a User Dictionaries}

There are two ways of creating a personal set of translations. First,
you can simply add commands like
\begin{verbatim}
\deftranslation[to=German]{figure}{Figur}
\end{verbatim}
to your personal macro files.

Second, you can create a personal dictionary file as follows: In your
document you say
\begin{verbatim}
\documentclass[ngerman]{article}

\usepackage{translator}
\usedictionary{my-personal-dictionary}
\end{verbatim}
and then you create the following file somewhere where \TeX\ can find
it:
\begin{verbatim}
// This is file my-personal-dictionary-German.dict
\ProvidesDictionary{my-personal-dictionary}{German}

\deftranslation{figure}{Figur}
\end{verbatim}


\subsection{Translating Keys}

Once the dictionaries and languages have been setup, you can translate
keys using the following commands:

\begin{command}{\translate\oarg{options}\marg{key}}
  This command will insert the translation of the \meta{key} at the
  current position into the text. The command is robust.

  The translation process of \meta{key} works as follows:
  \textsc{Translator} iterates over all languages on the current
  \emph{language 
    path} (see Section~\ref{section-language-path}). For each language on
  the path, \textsc{translator} checks whether a translation is available
  for the \meta{key}. For the first language for which this is the
  case, the translation is used. If there is no translation available
  for any language on the path, the \meta{key} itself is used as the
  translation.

  \example |\caption{\translate{Figure}~2.}|

  The following options may be given:
  \begin{itemize}
    \itemoption{to}|=|\meta{language}
    This option overrules the language path setting and installs
    \meta{language} as the target language(s) for which \textsc{translator}
    tries to find a translation.
  \end{itemize}
\end{command}

\begin{command}{\translatelet\oarg{options}\marg{macro}\marg{key}}
  This command works like the |\translate| command, only it will not
  insert the translation into the text, but will set the macro
  \meta{macro} to the translation found by the |\translate| command.

  \example |\translatelet\localfigure{figure}|
\end{command}


\subsection{Language Path and Language Substitution}

\label{section-language-path}

\begin{command}{\languagepath\marg{language path}}
  This command sets the language path that is searched when
  \textsc{translator} looks for a key. 

  The default value of the language path is
  |\languagename,English|. The |\languagename| is the standard \TeX\
  macro that expands to the current language. Typically, this is
  exactly what you want and there is no real need to change this
  default language path.
\end{command}

There is a problem with the names used in the macro
|\languagename|. These names, like |ngerman|, are not the ones used by
\textsc{translator} and we somehow have to tell the translator about
aliases for cryptic language names like |ngerman|. This is done using
the following command:

\begin{command}{\languagealias\marg{name}\marg{language list}}
  This command tells the translator that the language \meta{name}
  should be replaced by the language in the \meta{language list}.

  \example |\languagealias{ngerman}{German}|
  
  \example |\languagealias{german}{German1997,German}|
\end{command}

For the languages used by the babel package, the aliases are
automatically setup, so you typically do not need to call either
|\languagepath| or |\languagealias|.



\subsection{Package Loading Process}

The translator package is loaded ``in stages'':
\begin{enumerate}
\item
  First, some package or the document author requests the the
  translator package is loaded.
\item
  The translator package allows options like |ngerman| to be
  given. These options cause the necessary aliases and the correct
  translator languages to be requested.
\item
  During the preamble, packages and the document author request
  creating dictionary kinds and certain languages to be used. There
  requests are protocoled.
\item
  At the beginning of the document (|\begin{document}|) the requested
  dictionary-language-pairs are loaded.
\end{enumerate}

The first thing that needs to be done is to load the
package. Typically, this is done automatically by some other package,
but you may wish to include it directly:

\begin{package}{translator}
  When you load the package, you can specify (multiple) babel
  languages as \meta{options}. The effect of giving such an option is
  the following: It causes the translator package to call
  |\uselanguage| for the appropriate translation of the babel language
  names to translator's language names. It also causes
  |\languagealias| to be called for the languages. 
\end{package}



\section{Licenses}

\subsection{The GNU Public License, Version 2}
\label{section-gpl}

\subsubsection{Preamble}

The licenses for most software are designed to take away your freedom to
share and change it.  By contrast, the \textsc{gnu} General Public License is
intended to guarantee your freedom to share and change free software---to
make sure the software is free for all its users.  This General Public
License applies to most of the Free Software Foundation's software and to
any other program whose authors commit to using it.  (Some other Free
Software Foundation software is covered by the \textsc{gnu} Library
General Public License instead.)  You can apply it to your programs, too.

When we speak of free software, we are referring to freedom, not price.
Our General Public Licenses are designed to make sure that you have the
freedom to distribute copies of free software (and charge for this service
if you wish), that you receive source code or can get it if you want it,
that you can change the software or use pieces of it in new free programs;
and that you know you can do these things.

To protect your rights, we need to make restrictions that forbid anyone to
deny you these rights or to ask you to surrender the rights.  These
restrictions translate to certain responsibilities for you if you
distribute copies of the software, or if you modify it.

For example, if you distribute copies of such a program, whether gratis or
for a fee, you must give the recipients all the rights that you have.  You
must make sure that they, too, receive or can get the source code.  And
you must show them these terms so they know their rights.

We protect your rights with two steps: (1) copyright the software, and (2)
offer you this license which gives you legal permission to copy,
distribute and/or modify the software.

Also, for each author's protection and ours, we want to make certain that
everyone understands that there is no warranty for this free software.  If
the software is modified by someone else and passed on, we want its
recipients to know that what they have is not the original, so that any
problems introduced by others will not reflect on the original authors'
reputations.

Finally, any free program is threatened constantly by software patents.
We wish to avoid the danger that redistributors of a free program will
individually obtain patent licenses, in effect making the program
proprietary.  To prevent this, we have made it clear that any patent must
be licensed for everyone's free use or not licensed at all.

The precise terms and conditions for copying, distribution and
modification follow.

\subsubsection{Terms and Conditions For Copying, Distribution and
  Modification}

\begin{enumerate}

\addtocounter{enumi}{-1}

\item 
This License applies to any program or other work which contains a notice
placed by the copyright holder saying it may be distributed under the
terms of this General Public License.  The ``Program'', below, refers to
any such program or work, and a ``work based on the Program'' means either
the Program or any derivative work under copyright law: that is to say, a
work containing the Program or a portion of it, either verbatim or with
modifications and/or translated into another language.  (Hereinafter,
translation is included without limitation in the term ``modification''.)
Each licensee is addressed as ``you''.

Activities other than copying, distribution and modification are not
covered by this License; they are outside its scope.  The act of
running the Program is not restricted, and the output from the Program
is covered only if its contents constitute a work based on the
Program (independent of having been made by running the Program).
Whether that is true depends on what the Program does.

\item You may copy and distribute verbatim copies of the Program's source
  code as you receive it, in any medium, provided that you conspicuously
  and appropriately publish on each copy an appropriate copyright notice
  and disclaimer of warranty; keep intact all the notices that refer to
  this License and to the absence of any warranty; and give any other
  recipients of the Program a copy of this License along with the Program.

You may charge a fee for the physical act of transferring a copy, and you
may at your option offer warranty protection in exchange for a fee.

\item
You may modify your copy or copies of the Program or any portion
of it, thus forming a work based on the Program, and copy and
distribute such modifications or work under the terms of Section 1
above, provided that you also meet all of these conditions:

\begin{enumerate}

\item 
You must cause the modified files to carry prominent notices stating that
you changed the files and the date of any change.

\item
You must cause any work that you distribute or publish, that in
whole or in part contains or is derived from the Program or any
part thereof, to be licensed as a whole at no charge to all third
parties under the terms of this License.

\item
If the modified program normally reads commands interactively
when run, you must cause it, when started running for such
interactive use in the most ordinary way, to print or display an
announcement including an appropriate copyright notice and a
notice that there is no warranty (or else, saying that you provide
a warranty) and that users may redistribute the program under
these conditions, and telling the user how to view a copy of this
License.  (Exception: if the Program itself is interactive but
does not normally print such an announcement, your work based on
the Program is not required to print an announcement.)

\end{enumerate}


These requirements apply to the modified work as a whole.  If
identifiable sections of that work are not derived from the Program,
and can be reasonably considered independent and separate works in
themselves, then this License, and its terms, do not apply to those
sections when you distribute them as separate works.  But when you
distribute the same sections as part of a whole which is a work based
on the Program, the distribution of the whole must be on the terms of
this License, whose permissions for other licensees extend to the
entire whole, and thus to each and every part regardless of who wrote it.

Thus, it is not the intent of this section to claim rights or contest
your rights to work written entirely by you; rather, the intent is to
exercise the right to control the distribution of derivative or
collective works based on the Program.

In addition, mere aggregation of another work not based on the Program
with the Program (or with a work based on the Program) on a volume of
a storage or distribution medium does not bring the other work under
the scope of this License.

\item
You may copy and distribute the Program (or a work based on it,
under Section 2) in object code or executable form under the terms of
Sections 1 and 2 above provided that you also do one of the following:

\begin{enumerate}

\item
Accompany it with the complete corresponding machine-readable
source code, which must be distributed under the terms of Sections
1 and 2 above on a medium customarily used for software interchange; or,

\item
Accompany it with a written offer, valid for at least three
years, to give any third party, for a charge no more than your
cost of physically performing source distribution, a complete
machine-readable copy of the corresponding source code, to be
distributed under the terms of Sections 1 and 2 above on a medium
customarily used for software interchange; or,

\item
Accompany it with the information you received as to the offer
to distribute corresponding source code.  (This alternative is
allowed only for noncommercial distribution and only if you
received the program in object code or executable form with such
an offer, in accord with Subsubsection b above.)

\end{enumerate}


The source code for a work means the preferred form of the work for
making modifications to it.  For an executable work, complete source
code means all the source code for all modules it contains, plus any
associated interface definition files, plus the scripts used to
control compilation and installation of the executable.  However, as a
special exception, the source code distributed need not include
anything that is normally distributed (in either source or binary
form) with the major components (compiler, kernel, and so on) of the
operating system on which the executable runs, unless that component
itself accompanies the executable.

If distribution of executable or object code is made by offering
access to copy from a designated place, then offering equivalent
access to copy the source code from the same place counts as
distribution of the source code, even though third parties are not
compelled to copy the source along with the object code.

\item
You may not copy, modify, sublicense, or distribute the Program
except as expressly provided under this License.  Any attempt
otherwise to copy, modify, sublicense or distribute the Program is
void, and will automatically terminate your rights under this License.
However, parties who have received copies, or rights, from you under
this License will not have their licenses terminated so long as such
parties remain in full compliance.

\item
You are not required to accept this License, since you have not
signed it.  However, nothing else grants you permission to modify or
distribute the Program or its derivative works.  These actions are
prohibited by law if you do not accept this License.  Therefore, by
modifying or distributing the Program (or any work based on the
Program), you indicate your acceptance of this License to do so, and
all its terms and conditions for copying, distributing or modifying
the Program or works based on it.

\item
Each time you redistribute the Program (or any work based on the
Program), the recipient automatically receives a license from the
original licensor to copy, distribute or modify the Program subject to
these terms and conditions.  You may not impose any further
restrictions on the recipients' exercise of the rights granted herein.
You are not responsible for enforcing compliance by third parties to
this License.

\item
If, as a consequence of a court judgment or allegation of patent
infringement or for any other reason (not limited to patent issues),
conditions are imposed on you (whether by court order, agreement or
otherwise) that contradict the conditions of this License, they do not
excuse you from the conditions of this License.  If you cannot
distribute so as to satisfy simultaneously your obligations under this
License and any other pertinent obligations, then as a consequence you
may not distribute the Program at all.  For example, if a patent
license would not permit royalty-free redistribution of the Program by
all those who receive copies directly or indirectly through you, then
the only way you could satisfy both it and this License would be to
refrain entirely from distribution of the Program.

If any portion of this section is held invalid or unenforceable under
any particular circumstance, the balance of the section is intended to
apply and the section as a whole is intended to apply in other
circumstances.

It is not the purpose of this section to induce you to infringe any
patents or other property right claims or to contest validity of any
such claims; this section has the sole purpose of protecting the
integrity of the free software distribution system, which is
implemented by public license practices.  Many people have made
generous contributions to the wide range of software distributed
through that system in reliance on consistent application of that
system; it is up to the author/donor to decide if he or she is willing
to distribute software through any other system and a licensee cannot
impose that choice.

This section is intended to make thoroughly clear what is believed to
be a consequence of the rest of this License.

\item
If the distribution and/or use of the Program is restricted in
certain countries either by patents or by copyrighted interfaces, the
original copyright holder who places the Program under this License
may add an explicit geographical distribution limitation excluding
those countries, so that distribution is permitted only in or among
countries not thus excluded.  In such case, this License incorporates
the limitation as if written in the body of this License.

\item
The Free Software Foundation may publish revised and/or new versions
of the General Public License from time to time.  Such new versions will
be similar in spirit to the present version, but may differ in detail to
address new problems or concerns.

Each version is given a distinguishing version number.  If the Program
specifies a version number of this License which applies to it and ``any
later version'', you have the option of following the terms and conditions
either of that version or of any later version published by the Free
Software Foundation.  If the Program does not specify a version number of
this License, you may choose any version ever published by the Free Software
Foundation.

\item
If you wish to incorporate parts of the Program into other free
programs whose distribution conditions are different, write to the author
to ask for permission.  For software which is copyrighted by the Free
Software Foundation, write to the Free Software Foundation; we sometimes
make exceptions for this.  Our decision will be guided by the two goals
of preserving the free status of all derivatives of our free software and
of promoting the sharing and reuse of software generally.

\end{enumerate}

\subsubsection{No Warranty}

\begin{enumerate}

\addtocounter{enumi}{9}

\item
Because the program is licensed free of charge, there is no warranty
for the program, to the extent permitted by applicable law.  Except when
otherwise stated in writing the copyright holders and/or other parties
provide the program ``as is'' without warranty of any kind, either expressed
or implied, including, but not limited to, the implied warranties of
merchantability and fitness for a particular purpose.  The entire risk as
to the quality and performance of the program is with you.  Should the
program prove defective, you assume the cost of all necessary servicing,
repair or correction.

\item
In no event unless required by applicable law or agreed to in writing
will any copyright holder, or any other party who may modify and/or
redistribute the program as permitted above, be liable to you for damages,
including any general, special, incidental or consequential damages arising
out of the use or inability to use the program (including but not limited
to loss of data or data being rendered inaccurate or losses sustained by
you or third parties or a failure of the program to operate with any other
programs), even if such holder or other party has been advised of the
possibility of such damages.
\end{enumerate}


\providecommand{\LPPLsection}{\subsection}
\providecommand{\LPPLsubsection}{\subsubsection}
\providecommand{\LPPLsubsubsection}{\subsubsection}
\providecommand{\LPPLparagraph}{\paragraph}


% The file lppl.tex, some minor typographic changes:

%
% $Id: translator-manual-en.tex,v 1.5 2007/03/11 17:48:46 tantau Exp $
%
% Copyright 1999 2002-2006 LaTeX3 Project
%    Everyone is allowed to distribute verbatim copies of this
%    license document, but modification of it is not allowed.
%
%
% If you wish to load it as part of a ``doc'' source, you have to
% ensure that a) % is a comment character and b) that short verb
% characters are being turned off, i.e.,
%
%   \DeleteShortVerb{\'}   % or whatever was made a shorthand
%   \MakePercentComment
%   %
% $Id: lppl.tex,v 1.3 2006/01/13 14:45:56 mittelba Exp $
%
% Copyright 1999 2002-2006 LaTeX3 Project
%    Everyone is allowed to distribute verbatim copies of this
%    license document, but modification of it is not allowed.
%
%
% We want it to be possible that this file can be processed by
% (pdf)LaTeX on its own, or that this file can be included in another
% LaTeX document without any modification whatsoever.
% Hence the little test below.
% 
\makeatletter
\ifx\@preamblecmds\@notprerr
  % In this case the preamble has already been processed so this file
  % is loaded as part of another document; just enclose everything in
  % a group
  \let\LPPLicense\bgroup
  \let\endLPPLicense\egroup
\else
  % In this case the preamble has not been processed yet so this file
  % is processed by itself.
  \documentclass{article}
  \let\LPPLicense\document
  \let\endLPPLicense\enddocument
\fi
\makeatother


\begin{LPPLicense}
  \newcommand{\LPPLsection}{\section*}
  \newcommand{\LPPLsubsection}{\subsection*}
  \newcommand{\LPPLsubsubsection}{\subsubsection*}
  \newcommand{\LPPLparagraph}{\paragraph*}
  \newcommand*{\LPPLfile}[1]{\texttt{#1}}
  \newcommand*{\LPPLdocfile}[1]{`\LPPLfile{#1.tex}'}
  \newcommand*{\LPPL}{\textsc{lppl}}

  \LPPLsection{The \LaTeX\ Project Public License}

  \emph{LPPL Version 1.3b  2006-01-07}

  \textbf{Copyright 1999  2002--2006 \LaTeX3 Project}
  \begin{quotation}
    Everyone is allowed to distribute verbatim copies of this
    license document, but modification of it is not allowed.
  \end{quotation}

  \LPPLsubsection{Preamble}
  
  The \LaTeX\ Project Public License (\LPPL) is the primary license
  under which the the \LaTeX\ kernel and the base \LaTeX\ packages are
  distributed.

  You may use this license for any work of which you hold the
  copyright and which you wish to distribute.  This license may be
  particularly suitable if your work is \TeX-related (such as a
  \LaTeX\ package), but you may use it with small modifications even
  if your work is unrelated to \TeX.

  The section `WHETHER AND HOW TO DISTRIBUTE WORKS UNDER THIS
  LICENSE', below, gives instructions, examples, and recommendations
  for authors who are considering distributing their works under this
  license.

  This license gives conditions under which a work may be distributed
  and modified, as well as conditions under which modified versions of
  that work may be distributed.

  We, the \LaTeX3 Project, believe that the conditions below give you
  the freedom to make and distribute modified versions of your work
  that conform with whatever technical specifications you wish while
  maintaining the availability, integrity, and reliability of that
  work.  If you do not see how to achieve your goal while meeting
  these conditions, then read the document \LPPLdocfile{cfgguide} and
  \LPPLdocfile{modguide} in the base \LaTeX\ distribution for suggestions.


  \LPPLsubsection{Definitions}
  In this license document the following terms are used:

  \begin{description}
  \item[Work] Any work being distributed under this License.

  \item[Derived Work] Any work that under any applicable law is
    derived from the Work.

  \item[Modification] Any procedure that produces a Derived Work under
    any applicable law -- for example, the production of a file
    containing an original file associated with the Work or a
    significant portion of such a file, either verbatim or with
    modifications and/or translated into another language.

  \item[Modify] To apply any procedure that produces a Derived Work
    under any applicable law.
    
  \item[Distribution] Making copies of the Work available from one
    person to another, in whole or in part.  Distribution includes
    (but is not limited to) making any electronic components of the
    Work accessible by file transfer protocols such as \textsc{ftp} or
    \textsc{http} or by shared file systems such as Sun's Network File
    System (\textsc{nfs}).

  \item[Compiled Work] A version of the Work that has been processed
    into a form where it is directly usable on a computer system.
    This processing may include using installation facilities provided
    by the Work, transformations of the Work, copying of components of
    the Work, or other activities.  Note that modification of any
    installation facilities provided by the Work constitutes
    modification of the Work.

  \item[Current Maintainer] A person or persons nominated as such
    within the Work.  If there is no such explicit nomination then it
    is the `Copyright Holder' under any applicable law.

  \item[Base Interpreter] A program or process that is normally needed
    for running or interpreting a part or the whole of the Work.
    
    A Base Interpreter may depend on external components but these are
    not considered part of the Base Interpreter provided that each
    external component clearly identifies itself whenever it is used
    interactively.  Unless explicitly specified when applying the
    license to the Work, the only applicable Base Interpreter is a
    `\LaTeX-Format' or in the case of files belonging to the
    `\LaTeX-format' a program implementing the `\TeX{} language'.
  \end{description}

  \LPPLsubsection{Conditions on Distribution and Modification}

  \begin{enumerate}
  \item Activities other than distribution and/or modification of the
    Work are not covered by this license; they are outside its scope.
    In particular, the act of running the Work is not restricted and
    no requirements are made concerning any offers of support for the
    Work.

  \item\label{distribute} You may distribute a complete, unmodified
    copy of the Work as you received it.  Distribution of only part of
    the Work is considered modification of the Work, and no right to
    distribute such a Derived Work may be assumed under the terms of
    this clause.

  \item You may distribute a Compiled Work that has been generated
    from a complete, unmodified copy of the Work as distributed under
    Clause~\ref{distribute} above, as long as that Compiled Work is
    distributed in such a way that the recipients may install the
    Compiled Work on their system exactly as it would have been
    installed if they generated a Compiled Work directly from the
    Work.

  \item\label{currmaint} If you are the Current Maintainer of the
    Work, you may, without restriction, modify the Work, thus creating
    a Derived Work.  You may also distribute the Derived Work without
    restriction, including Compiled Works generated from the Derived
    Work.  Derived Works distributed in this manner by the Current
    Maintainer are considered to be updated versions of the Work.

  \item If you are not the Current Maintainer of the Work, you may
    modify your copy of the Work, thus creating a Derived Work based
    on the Work, and compile this Derived Work, thus creating a
    Compiled Work based on the Derived Work.

  \item\label{conditions} If you are not the Current Maintainer of the
    Work, you may distribute a Derived Work provided the following
    conditions are met for every component of the Work unless that
    component clearly states in the copyright notice that it is exempt
    from that condition.  Only the Current Maintainer is allowed to
    add such statements of exemption to a component of the Work.
    \begin{enumerate}
    \item If a component of this Derived Work can be a direct
      replacement for a component of the Work when that component is
      used with the Base Interpreter, then, wherever this component of
      the Work identifies itself to the user when used interactively
      with that Base Interpreter, the replacement component of this
      Derived Work clearly and unambiguously identifies itself as a
      modified version of this component to the user when used
      interactively with that Base Interpreter.
     
    \item Every component of the Derived Work contains prominent
      notices detailing the nature of the changes to that component,
      or a prominent reference to another file that is distributed as
      part of the Derived Work and that contains a complete and
      accurate log of the changes.
  
    \item No information in the Derived Work implies that any persons,
      including (but not limited to) the authors of the original
      version of the Work, provide any support, including (but not
      limited to) the reporting and handling of errors, to recipients
      of the Derived Work unless those persons have stated explicitly
      that they do provide such support for the Derived Work.

    \item You distribute at least one of the following with the Derived Work:
      \begin{enumerate}
      \item A complete, unmodified copy of the Work; if your
        distribution of a modified component is made by offering
        access to copy the modified component from a designated place,
        then offering equivalent access to copy the Work from the same
        or some similar place meets this condition, even though third
        parties are not compelled to copy the Work along with the
        modified component;

      \item Information that is sufficient to obtain a complete,
        unmodified copy of the Work.
      \end{enumerate}
    \end{enumerate}
  \item If you are not the Current Maintainer of the Work, you may
    distribute a Compiled Work generated from a Derived Work, as long
    as the Derived Work is distributed to all recipients of the
    Compiled Work, and as long as the conditions of
    Clause~\ref{conditions}, above, are met with regard to the Derived
    Work.

  \item The conditions above are not intended to prohibit, and hence
    do not apply to, the modification, by any method, of any component
    so that it becomes identical to an updated version of that
    component of the Work as it is distributed by the Current
    Maintainer under Clause~\ref{currmaint}, above.

  \item Distribution of the Work or any Derived Work in an alternative
    format, where the Work or that Derived Work (in whole or in part)
    is then produced by applying some process to that format, does not
    relax or nullify any sections of this license as they pertain to
    the results of applying that process.
     
  \item \null
    \begin{enumerate}
    \item A Derived Work may be distributed under a different license
      provided that license itself honors the conditions listed in
      Clause~\ref{conditions} above, in regard to the Work, though it
      does not have to honor the rest of the conditions in this
      license.
      
    \item If a Derived Work is distributed under a different license,
      that Derived Work must provide sufficient documentation as part
      of itself to allow each recipient of that Derived Work to honor
      the restrictions in Clause~\ref{conditions} above, concerning
      changes from the Work.
    \end{enumerate}
  \item This license places no restrictions on works that are
    unrelated to the Work, nor does this license place any
    restrictions on aggregating such works with the Work by any means.

  \item Nothing in this license is intended to, or may be used to,
    prevent complete compliance by all parties with all applicable
    laws.
  \end{enumerate}

  \LPPLsubsection{No Warranty}

  There is no warranty for the Work.  Except when otherwise stated in
  writing, the Copyright Holder provides the Work `as is', without
  warranty of any kind, either expressed or implied, including, but
  not limited to, the implied warranties of merchantability and
  fitness for a particular purpose.  The entire risk as to the quality
  and performance of the Work is with you.  Should the Work prove
  defective, you assume the cost of all necessary servicing, repair,
  or correction.

  In no event unless required by applicable law or agreed to in
  writing will The Copyright Holder, or any author named in the
  components of the Work, or any other party who may distribute and/or
  modify the Work as permitted above, be liable to you for damages,
  including any general, special, incidental or consequential damages
  arising out of any use of the Work or out of inability to use the
  Work (including, but not limited to, loss of data, data being
  rendered inaccurate, or losses sustained by anyone as a result of
  any failure of the Work to operate with any other programs), even if
  the Copyright Holder or said author or said other party has been
  advised of the possibility of such damages.

  \LPPLsubsection{Maintenance of The Work}

  The Work has the status `author-maintained' if the Copyright Holder
  explicitly and prominently states near the primary copyright notice
  in the Work that the Work can only be maintained by the Copyright
  Holder or simply that it is `author-maintained'.

  The Work has the status `maintained' if there is a Current
  Maintainer who has indicated in the Work that they are willing to
  receive error reports for the Work (for example, by supplying a
  valid e-mail address). It is not required for the Current Maintainer
  to acknowledge or act upon these error reports.

  The Work changes from status `maintained' to `unmaintained' if there
  is no Current Maintainer, or the person stated to be Current
  Maintainer of the work cannot be reached through the indicated means
  of communication for a period of six months, and there are no other
  significant signs of active maintenance.

  You can become the Current Maintainer of the Work by agreement with
  any existing Current Maintainer to take over this role.

  If the Work is unmaintained, you can become the Current Maintainer
  of the Work through the following steps:
  \begin{enumerate}
  \item Make a reasonable attempt to trace the Current Maintainer (and
    the Copyright Holder, if the two differ) through the means of an
    Internet or similar search.
  \item If this search is successful, then enquire whether the Work is
    still maintained.
    \begin{enumerate}
    \item If it is being maintained, then ask the Current Maintainer
      to update their communication data within one month.
     
    \item\label{intention} If the search is unsuccessful or no action
      to resume active maintenance is taken by the Current Maintainer,
      then announce within the pertinent community your intention to
      take over maintenance.  (If the Work is a \LaTeX{} work, this
      could be done, for example, by posting to
      \texttt{comp.text.tex}.)
    \end{enumerate}
  \item {}
    \begin{enumerate}
    \item If the Current Maintainer is reachable and agrees to pass
      maintenance of the Work to you, then this takes effect
      immediately upon announcement.
     
    \item\label{announce} If the Current Maintainer is not reachable
      and the Copyright Holder agrees that maintenance of the Work be
      passed to you, then this takes effect immediately upon
      announcement.
    \end{enumerate}
  \item\label{change} If you make an `intention announcement' as
    described in~\ref{intention} above and after three months your
    intention is challenged neither by the Current Maintainer nor by
    the Copyright Holder nor by other people, then you may arrange for
    the Work to be changed so as to name you as the (new) Current
    Maintainer.
     
  \item If the previously unreachable Current Maintainer becomes
    reachable once more within three months of a change completed
    under the terms of~\ref{announce} or~\ref{change}, then that
    Current Maintainer must become or remain the Current Maintainer
    upon request provided they then update their communication data
    within one month.
  \end{enumerate}
  A change in the Current Maintainer does not, of itself, alter the
  fact that the Work is distributed under the \LPPL\ license.

  If you become the Current Maintainer of the Work, you should
  immediately provide, within the Work, a prominent and unambiguous
  statement of your status as Current Maintainer.  You should also
  announce your new status to the same pertinent community as
  in~\ref{intention} above.

  \LPPLsubsection{Whether and How to Distribute Works under This License}

  This section contains important instructions, examples, and
  recommendations for authors who are considering distributing their
  works under this license.  These authors are addressed as `you' in
  this section.

  \LPPLsubsubsection{Choosing This License or Another License}

  If for any part of your work you want or need to use
  \emph{distribution} conditions that differ significantly from those
  in this license, then do not refer to this license anywhere in your
  work but, instead, distribute your work under a different license.
  You may use the text of this license as a model for your own
  license, but your license should not refer to the \LPPL\ or
  otherwise give the impression that your work is distributed under
  the \LPPL.

  The document \LPPLdocfile{modguide} in the base \LaTeX\ distribution
  explains the motivation behind the conditions of this license.  It
  explains, for example, why distributing \LaTeX\ under the
  \textsc{gnu} General Public License (\textsc{gpl}) was considered
  inappropriate.  Even if your work is unrelated to \LaTeX, the
  discussion in \LPPLdocfile{modguide} may still be relevant, and authors
  intending to distribute their works under any license are encouraged
  to read it.

  \LPPLsubsubsection{A Recommendation on Modification Without Distribution}

  It is wise never to modify a component of the Work, even for your
  own personal use, without also meeting the above conditions for
  distributing the modified component.  While you might intend that
  such modifications will never be distributed, often this will happen
  by accident -- you may forget that you have modified that component;
  or it may not occur to you when allowing others to access the
  modified version that you are thus distributing it and violating the
  conditions of this license in ways that could have legal
  implications and, worse, cause problems for the community.  It is
  therefore usually in your best interest to keep your copy of the
  Work identical with the public one.  Many works provide ways to
  control the behavior of that work without altering any of its
  licensed components.

  \LPPLsubsubsection{How to Use This License}

  To use this license, place in each of the components of your work
  both an explicit copyright notice including your name and the year
  the work was authored and/or last substantially modified.  Include
  also a statement that the distribution and/or modification of that
  component is constrained by the conditions in this license.

  Here is an example of such a notice and statement:
  \begin{verbatim}
  %% pig.dtx
  %% Copyright 2005 M. Y. Name
  %
  % This work may be distributed and/or modified under the
  % conditions of the LaTeX Project Public License, either version 1.3
  % of this license or (at your option) any later version.
  % The latest version of this license is in
  %   http://www.latex-project.org/lppl.txt
  % and version 1.3 or later is part of all distributions of LaTeX
  % version 2005/12/01 or later.
  %
  % This work has the LPPL maintenance status `maintained'.
  % 
  % The Current Maintainer of this work is M. Y. Name.
  %
  % This work consists of the files pig.dtx and pig.ins
  % and the derived file pig.sty.
  \end{verbatim}
  
  Given such a notice and statement in a file, the conditions given in
  this license document would apply, with the `Work' referring to the
  three files `\LPPLfile{pig.dtx}', `\LPPLfile{pig.ins}', and
  `\LPPLfile{pig.sty}' (the last being generated from
  `\LPPLfile{pig.dtx}' using `\LPPLfile{pig.ins}'), the `Base
  Interpreter' referring to any `\LaTeX-Format', and both `Copyright
  Holder' and `Current Maintainer' referring to the person `M. Y.
  Name'.

  If you do not want the Maintenance section of \LPPL\ to apply to
  your Work, change `maintained' above into `author-maintained'.
  However, we recommend that you use `maintained' as the Maintenance
  section was added in order to ensure that your Work remains useful
  to the community even when you can no longer maintain and support it
  yourself.

  \LPPLsubsubsection{Derived Works That Are Not Replacements}

  Several clauses of the \LPPL\ specify means to provide reliability
  and stability for the user community. They therefore concern
  themselves with the case that a Derived Work is intended to be used
  as a (compatible or incompatible) replacement of the original
  Work. If this is not the case (e.g., if a few lines of code are
  reused for a completely different task), then clauses 6b and 6d
  shall not apply.

  \LPPLsubsubsection{Important Recommendations}

  \LPPLparagraph{Defining What Constitutes the Work}

  The \LPPL\ requires that distributions of the Work contain all the
  files of the Work.  It is therefore important that you provide a way
  for the licensee to determine which files constitute the Work.  This
  could, for example, be achieved by explicitly listing all the files
  of the Work near the copyright notice of each file or by using a
  line such as:

  \begin{verbatim}
    % This work consists of all files listed in manifest.txt.
  \end{verbatim}
   
  in that place.  In the absence of an unequivocal list it might be
  impossible for the licensee to determine what is considered by you
  to comprise the Work and, in such a case, the licensee would be
  entitled to make reasonable conjectures as to which files comprise
  the Work.

\end{LPPLicense}
\endinput

%   \MakePercentIgnore
%   \MakeShortVerb{\'}     % turn it on again if necessary
%
%
% By default the license is produced with \section* as the highest
% heading level. If this is not appropriate for the document in which
% it is included define the commands listed below before loading this
% document, e.g., for inclusion as a separate chapter define:
%
%  \providecommand{\LPPLsection}{\chapter*}
%  \providecommand{\LPPLsubsection}{\section*}
%  \providecommand{\LPPLsubsubsection}{\subsection*}
%  \providecommand{\LPPLparagraph}{\subsubsection*}
%
% 
% To allow cross-referencing the headings \label's have been attached
% to them, all starting with ``LPPL:''. As by default headings without
% numbers are produced, this will only allow page references.
% However, you can use the titleref package to produce textual
% references or you change the definitions of \LPPLsection, and
% friends to generated numbered headings.
%
%
% We want it to be possible that this file can be processed by
% (pdf)LaTeX on its own, or that this file can be included in another
% LaTeX document without any modification whatsoever.
% Hence the little test below.
%
%
\makeatletter
\ifx\@preamblecmds\@notprerr
  % In this case the preamble has already been processed so this file
  % is loaded as part of another document; just enclose everything in
  % a group
  \let\LPPLicense\bgroup
  \let\endLPPLicense\egroup
\else
  % In this case the preamble has not been processed yet so this file
  % is processed by itself.
  \documentclass{article}
  \let\LPPLicense\document
  \let\endLPPLicense\enddocument
\fi
\makeatother


\begin{LPPLicense}
  \providecommand{\LPPLsection}{\section*}
  \providecommand{\LPPLsubsection}{\subsection*}
  \providecommand{\LPPLsubsubsection}{\subsubsection*}
  \providecommand{\LPPLparagraph}{\paragraph*}
  \providecommand*{\LPPLfile}[1]{\texttt{#1}}
  \providecommand*{\LPPLdocfile}[1]{`\LPPLfile{#1.tex}'}
  \providecommand*{\LPPL}{\textsc{lppl}}

  \LPPLsection{The \LaTeX\ Project Public License, Version 1.3c 2006-05-20}
  \label{LPPL:LPPL}

%  \textbf{Copyright 1999, 2002--2006 \LaTeX3 Project}
%  \begin{quotation}
%    Everyone is allowed to distribute verbatim copies of this
%    license document, but modification of it is not allowed.
%  \end{quotation}

  \LPPLsubsection{Preamble}
  \label{LPPL:Preamble}
  
  The \LaTeX\ Project Public License (\LPPL) is the primary license
  under which the the \LaTeX\ kernel and the base \LaTeX\ packages are
  distributed.

  You may use this license for any work of which you hold the
  copyright and which you wish to distribute.  This license may be
  particularly suitable if your work is \TeX-related (such as a
  \LaTeX\ package), but it is written in such a way that you can use 
  it even if your work is unrelated to \TeX.

  The section `\textsc{wheter and how to distribute works under this
    license}', below, gives instructions, examples, and
  recommendations 
  for authors who are considering distributing their works under this
  license.

  This license gives conditions under which a work may be distributed
  and modified, as well as conditions under which modified versions of
  that work may be distributed.

  We, the \LaTeX3 Project, believe that the conditions below give you
  the freedom to make and distribute modified versions of your work
  that conform with whatever technical specifications you wish while
  maintaining the availability, integrity, and reliability of that
  work.  If you do not see how to achieve your goal while meeting
  these conditions, then read the document \LPPLdocfile{cfgguide} and
  \LPPLdocfile{modguide} in the base \LaTeX\ distribution for suggestions.


  \LPPLsubsection{Definitions}
  \label{LPPL:Definitions}

  In this license document the following terms are used:

  \begin{description}
  \item[Work] Any work being distributed under this License.

  \item[Derived Work] Any work that under any applicable law is
    derived from the Work.

  \item[Modification] Any procedure that produces a Derived Work under
    any applicable law -- for example, the production of a file
    containing an original file associated with the Work or a
    significant portion of such a file, either verbatim or with
    modifications and/or translated into another language.

  \item[Modify] To apply any procedure that produces a Derived Work
    under any applicable law.
    
  \item[Distribution] Making copies of the Work available from one
    person to another, in whole or in part.  Distribution includes
    (but is not limited to) making any electronic components of the
    Work accessible by file transfer protocols such as \textsc{ftp} or
    \textsc{http} or by shared file systems such as Sun's Network File
    System (\textsc{nfs}).

  \item[Compiled Work] A version of the Work that has been processed
    into a form where it is directly usable on a computer system.
    This processing may include using installation facilities provided
    by the Work, transformations of the Work, copying of components of
    the Work, or other activities.  Note that modification of any
    installation facilities provided by the Work constitutes
    modification of the Work.

  \item[Current Maintainer] A person or persons nominated as such
    within the Work.  If there is no such explicit nomination then it
    is the `Copyright Holder' under any applicable law.

  \item[Base Interpreter] A program or process that is normally needed
    for running or interpreting a part or the whole of the Work.
    
    A Base Interpreter may depend on external components but these are
    not considered part of the Base Interpreter provided that each
    external component clearly identifies itself whenever it is used
    interactively.  Unless explicitly specified when applying the
    license to the Work, the only applicable Base Interpreter is a
    `\LaTeX-Format' or in the case of files belonging to the
    `\LaTeX-format' a program implementing the `\TeX{} language'.
  \end{description}

  \LPPLsubsection{Conditions on Distribution and Modification}
  \label{LPPL:Conditions}

  \begin{enumerate}
  \item Activities other than distribution and/or modification of the
    Work are not covered by this license; they are outside its scope.
    In particular, the act of running the Work is not restricted and
    no requirements are made concerning any offers of support for the
    Work.

  \item\label{LPPL:item:distribute} You may distribute a complete, unmodified
    copy of the Work as you received it.  Distribution of only part of
    the Work is considered modification of the Work, and no right to
    distribute such a Derived Work may be assumed under the terms of
    this clause.

  \item You may distribute a Compiled Work that has been generated
    from a complete, unmodified copy of the Work as distributed under
    Clause~\ref{LPPL:item:distribute} above, as long as that Compiled Work is
    distributed in such a way that the recipients may install the
    Compiled Work on their system exactly as it would have been
    installed if they generated a Compiled Work directly from the
    Work.

  \item\label{LPPL:item:currmaint} If you are the Current Maintainer of the
    Work, you may, without restriction, modify the Work, thus creating
    a Derived Work.  You may also distribute the Derived Work without
    restriction, including Compiled Works generated from the Derived
    Work.  Derived Works distributed in this manner by the Current
    Maintainer are considered to be updated versions of the Work.

  \item If you are not the Current Maintainer of the Work, you may
    modify your copy of the Work, thus creating a Derived Work based
    on the Work, and compile this Derived Work, thus creating a
    Compiled Work based on the Derived Work.

  \item\label{LPPL:item:conditions} If you are not the Current Maintainer of the
    Work, you may distribute a Derived Work provided the following
    conditions are met for every component of the Work unless that
    component clearly states in the copyright notice that it is exempt
    from that condition.  Only the Current Maintainer is allowed to
    add such statements of exemption to a component of the Work.
    \begin{enumerate}
    \item If a component of this Derived Work can be a direct
      replacement for a component of the Work when that component is
      used with the Base Interpreter, then, wherever this component of
      the Work identifies itself to the user when used interactively
      with that Base Interpreter, the replacement component of this
      Derived Work clearly and unambiguously identifies itself as a
      modified version of this component to the user when used
      interactively with that Base Interpreter.
     
    \item Every component of the Derived Work contains prominent
      notices detailing the nature of the changes to that component,
      or a prominent reference to another file that is distributed as
      part of the Derived Work and that contains a complete and
      accurate log of the changes.
  
    \item No information in the Derived Work implies that any persons,
      including (but not limited to) the authors of the original
      version of the Work, provide any support, including (but not
      limited to) the reporting and handling of errors, to recipients
      of the Derived Work unless those persons have stated explicitly
      that they do provide such support for the Derived Work.

    \item You distribute at least one of the following with the Derived Work:
      \begin{enumerate}
      \item A complete, unmodified copy of the Work; if your
        distribution of a modified component is made by offering
        access to copy the modified component from a designated place,
        then offering equivalent access to copy the Work from the same
        or some similar place meets this condition, even though third
        parties are not compelled to copy the Work along with the
        modified component;

      \item Information that is sufficient to obtain a complete,
        unmodified copy of the Work.
      \end{enumerate}
    \end{enumerate}
  \item If you are not the Current Maintainer of the Work, you may
    distribute a Compiled Work generated from a Derived Work, as long
    as the Derived Work is distributed to all recipients of the
    Compiled Work, and as long as the conditions of
    Clause~\ref{LPPL:item:conditions}, above, are met with regard to the Derived
    Work.

  \item The conditions above are not intended to prohibit, and hence
    do not apply to, the modification, by any method, of any component
    so that it becomes identical to an updated version of that
    component of the Work as it is distributed by the Current
    Maintainer under Clause~\ref{LPPL:item:currmaint}, above.

  \item Distribution of the Work or any Derived Work in an alternative
    format, where the Work or that Derived Work (in whole or in part)
    is then produced by applying some process to that format, does not
    relax or nullify any sections of this license as they pertain to
    the results of applying that process.
     
  \item \null
    \begin{enumerate}
    \item A Derived Work may be distributed under a different license
      provided that license itself honors the conditions listed in
      Clause~\ref{LPPL:item:conditions} above, in regard to the Work, though it
      does not have to honor the rest of the conditions in this
      license.
      
    \item If a Derived Work is distributed under a different license,
      that Derived Work must provide sufficient documentation as part
      of itself to allow each recipient of that Derived Work to honor
      the restrictions in Clause~\ref{LPPL:item:conditions} above, concerning
      changes from the Work.
    \end{enumerate}
  \item This license places no restrictions on works that are
    unrelated to the Work, nor does this license place any
    restrictions on aggregating such works with the Work by any means.

  \item Nothing in this license is intended to, or may be used to,
    prevent complete compliance by all parties with all applicable
    laws.
  \end{enumerate}

  \LPPLsubsection{No Warranty}
  \label{LPPL:Warranty}

  There is no warranty for the Work.  Except when otherwise stated in
  writing, the Copyright Holder provides the Work `as is', without
  warranty of any kind, either expressed or implied, including, but
  not limited to, the implied warranties of merchantability and
  fitness for a particular purpose.  The entire risk as to the quality
  and performance of the Work is with you.  Should the Work prove
  defective, you assume the cost of all necessary servicing, repair,
  or correction.

  In no event unless required by applicable law or agreed to in
  writing will The Copyright Holder, or any author named in the
  components of the Work, or any other party who may distribute and/or
  modify the Work as permitted above, be liable to you for damages,
  including any general, special, incidental or consequential damages
  arising out of any use of the Work or out of inability to use the
  Work (including, but not limited to, loss of data, data being
  rendered inaccurate, or losses sustained by anyone as a result of
  any failure of the Work to operate with any other programs), even if
  the Copyright Holder or said author or said other party has been
  advised of the possibility of such damages.

  \LPPLsubsection{Maintenance of The Work}
  \label{LPPL:Maintenance}

  The Work has the status `author-maintained' if the Copyright Holder
  explicitly and prominently states near the primary copyright notice
  in the Work that the Work can only be maintained by the Copyright
  Holder or simply that it is `author-maintained'.

  The Work has the status `maintained' if there is a Current
  Maintainer who has indicated in the Work that they are willing to
  receive error reports for the Work (for example, by supplying a
  valid e-mail address). It is not required for the Current Maintainer
  to acknowledge or act upon these error reports.

  The Work changes from status `maintained' to `unmaintained' if there
  is no Current Maintainer, or the person stated to be Current
  Maintainer of the work cannot be reached through the indicated means
  of communication for a period of six months, and there are no other
  significant signs of active maintenance.

  You can become the Current Maintainer of the Work by agreement with
  any existing Current Maintainer to take over this role.

  If the Work is unmaintained, you can become the Current Maintainer
  of the Work through the following steps:
  \begin{enumerate}
  \item Make a reasonable attempt to trace the Current Maintainer (and
    the Copyright Holder, if the two differ) through the means of an
    Internet or similar search.
  \item If this search is successful, then enquire whether the Work is
    still maintained.
    \begin{enumerate}
    \item If it is being maintained, then ask the Current Maintainer
      to update their communication data within one month.
     
    \item\label{LPPL:item:intention} If the search is unsuccessful or
      no action to resume active maintenance is taken by the Current
      Maintainer, then announce within the pertinent community your
      intention to take over maintenance.  (If the Work is a \LaTeX{}
      work, this could be done, for example, by posting to
      \texttt{comp.text.tex}.)
    \end{enumerate}
  \item {}
    \begin{enumerate}
    \item If the Current Maintainer is reachable and agrees to pass
      maintenance of the Work to you, then this takes effect
      immediately upon announcement.
     
    \item\label{LPPL:item:announce} If the Current Maintainer is not
      reachable and the Copyright Holder agrees that maintenance of
      the Work be passed to you, then this takes effect immediately
      upon announcement.
    \end{enumerate}
  \item\label{LPPL:item:change} If you make an `intention
    announcement' as described in~\ref{LPPL:item:intention} above and
    after three months your intention is challenged neither by the
    Current Maintainer nor by the Copyright Holder nor by other
    people, then you may arrange for the Work to be changed so as to
    name you as the (new) Current Maintainer.
     
  \item If the previously unreachable Current Maintainer becomes
    reachable once more within three months of a change completed
    under the terms of~\ref{LPPL:item:announce}
    or~\ref{LPPL:item:change}, then that Current Maintainer must
    become or remain the Current Maintainer upon request provided they
    then update their communication data within one month.
  \end{enumerate}
  A change in the Current Maintainer does not, of itself, alter the
  fact that the Work is distributed under the \LPPL\ license.

  If you become the Current Maintainer of the Work, you should
  immediately provide, within the Work, a prominent and unambiguous
  statement of your status as Current Maintainer.  You should also
  announce your new status to the same pertinent community as
  in~\ref{LPPL:item:intention} above.

  \LPPLsubsection{Whether and How to Distribute Works under This License}
  \label{LPPL:Distribute}

  This section contains important instructions, examples, and
  recommendations for authors who are considering distributing their
  works under this license.  These authors are addressed as `you' in
  this section.

  \LPPLsubsubsection{Choosing This License or Another License}
  \label{LPPL:Choosing}

  If for any part of your work you want or need to use
  \emph{distribution} conditions that differ significantly from those
  in this license, then do not refer to this license anywhere in your
  work but, instead, distribute your work under a different license.
  You may use the text of this license as a model for your own
  license, but your license should not refer to the \LPPL\ or
  otherwise give the impression that your work is distributed under
  the \LPPL.

  The document \LPPLdocfile{modguide} in the base \LaTeX\ distribution
  explains the motivation behind the conditions of this license.  It
  explains, for example, why distributing \LaTeX\ under the
  \textsc{gnu} General Public License (\textsc{gpl}) was considered
  inappropriate.  Even if your work is unrelated to \LaTeX, the
  discussion in \LPPLdocfile{modguide} may still be relevant, and authors
  intending to distribute their works under any license are encouraged
  to read it.

  \LPPLsubsubsection{A Recommendation on Modification Without Distribution}
  \label{LPPL:WithoutDistribution}

  It is wise never to modify a component of the Work, even for your
  own personal use, without also meeting the above conditions for
  distributing the modified component.  While you might intend that
  such modifications will never be distributed, often this will happen
  by accident -- you may forget that you have modified that component;
  or it may not occur to you when allowing others to access the
  modified version that you are thus distributing it and violating the
  conditions of this license in ways that could have legal
  implications and, worse, cause problems for the community.  It is
  therefore usually in your best interest to keep your copy of the
  Work identical with the public one.  Many works provide ways to
  control the behavior of that work without altering any of its
  licensed components.

  \LPPLsubsubsection{How to Use This License}
  \label{LPPL:HowTo}

  To use this license, place in each of the components of your work
  both an explicit copyright notice including your name and the year
  the work was authored and/or last substantially modified.  Include
  also a statement that the distribution and/or modification of that
  component is constrained by the conditions in this license.

  Here is an example of such a notice and statement:
\begin{verbatim}
  %% pig.dtx
  %% Copyright 2005 M. Y. Name
  %
  % This work may be distributed and/or modified under the
  % conditions of the LaTeX Project Public License, either version 1.3
  % of this license or (at your option) any later version.
  % The latest version of this license is in
  %   http://www.latex-project.org/lppl.txt
  % and version 1.3 or later is part of all distributions of LaTeX
  % version 2005/12/01 or later.
  %
  % This work has the LPPL maintenance status `maintained'.
  % 
  % The Current Maintainer of this work is M. Y. Name.
  %
  % This work consists of the files pig.dtx and pig.ins
  % and the derived file pig.sty.
\end{verbatim}
  
  Given such a notice and statement in a file, the conditions given in
  this license document would apply, with the `Work' referring to the
  three files `\LPPLfile{pig.dtx}', `\LPPLfile{pig.ins}', and
  `\LPPLfile{pig.sty}' (the last being generated from
  `\LPPLfile{pig.dtx}' using `\LPPLfile{pig.ins}'), the `Base
  Interpreter' referring to any `\LaTeX-Format', and both `Copyright
  Holder' and `Current Maintainer' referring to the person `M. Y.
  Name'.

  If you do not want the Maintenance section of \LPPL\ to apply to
  your Work, change `maintained' above into `author-maintained'.
  However, we recommend that you use `maintained' as the Maintenance
  section was added in order to ensure that your Work remains useful
  to the community even when you can no longer maintain and support it
  yourself.

  \LPPLsubsubsection{Derived Works That Are Not Replacements}
  \label{LPPL:NotReplacements}

  Several clauses of the \LPPL\ specify means to provide reliability
  and stability for the user community. They therefore concern
  themselves with the case that a Derived Work is intended to be used
  as a (compatible or incompatible) replacement of the original
  Work. If this is not the case (e.g., if a few lines of code are
  reused for a completely different task), then clauses 6b and 6d
  shall not apply.

  \LPPLsubsubsection{Important Recommendations}
  \label{LPPL:Recommendations}

  \LPPLparagraph{Defining What Constitutes the Work}

  The \LPPL\ requires that distributions of the Work contain all the
  files of the Work.  It is therefore important that you provide a way
  for the licensee to determine which files constitute the Work.  This
  could, for example, be achieved by explicitly listing all the files
  of the Work near the copyright notice of each file or by using a
  line such as:
\begin{verbatim}
    % This work consists of all files listed in manifest.txt.
\end{verbatim}
  in that place.  In the absence of an unequivocal list it might be
  impossible for the licensee to determine what is considered by you
  to comprise the Work and, in such a case, the licensee would be
  entitled to make reasonable conjectures as to which files comprise
  the Work.

\end{LPPLicense}



\printindex

\end{document}
