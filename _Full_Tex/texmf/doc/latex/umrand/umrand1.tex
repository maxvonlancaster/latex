\newpage\setcounter{page}{1}
\font\f=umranda
\font\F=umranda at 20pt
{\def\make{\RandBox {4}
         font {\F} [0pt]
         (\C136) ([\C137]) (\C140)
         ([\C145])         ([\C141])
         (\C144) ([\C143]) (\C142)

          \vskip 2mm
		  \hrule
          \verbatiminput{b}
		  \vskip 2mm\hrule\hrule\vskip 1mm}
\fussy
\Thema{Umrandungen}
\Nr{9}

Heute gehts um Umrandungen, die Makros und der Font sind ziemlich
umfangreich, so da"s beide hier nur in der Anwendung gezeigt werden sollen.
Es sind definiert:
\begin{verbatim}
\def\C{\char'} % Oktaldarstellung
\font\f = umranda
\font\F = umranda at 20pt
\end{verbatim}
Und los gehts:
\vskip 2mm

\begin{bsp}
\RandBox {Dieses ist eine Quadrat-Testbox}
         font {\f} [0pt]
         (\C114) ([\C71\\\C73]\C71) (\C114)
         (\C72[\C74\\\C72])        ([\C74\\\C72]\C74)
         (\C114) (\C73[\C71\\\C73]) (\C114)
\end{bsp}
\make

\begin{bsp}
\RandBox {Es w"achst Gras "uber die Sache}
         font {\f} [0pt]
         () ([\C75\\\C77]\C75) ()
         (\C76[\C100\\\C76])           ([\C100\\\C76]\C100)
         () (\C77[\C75\\\C77]) ()
\end{bsp}
\make



\begin{bsp}
\RandBox {Ist das nicht zum Kringeln?}
         font {\f} [-0.3em]
         () ([\C65]) ()
         ([\C67])   ([\C67])
         () ([\C65]) ()
\end{bsp}
\make



\begin{bsp}
\RandBox {Eine sich schl"angelnde Schlange.}
         font {\F} [-0.3em]
         () ([\C127]) ()
         ([\C131])   ([\C135])
         () ([\C133]) ()
\end{bsp}
\make



\begin{bsp}
\RandBox {Dieses ist eine riesige Box mit Fischen}
         font {\f} [0.1em]
         (\C113) ([\C102]) (\C113)
         ([\C104])         ([\C110])
         (\C113) ([\C106]) (\C113)
\end{bsp}
\make



\begin{bsp}
\RandBox {Wellenlinien mit verschiedenen Ecken}
         font {\f} [0pt]
         (\C115) ([\C111]) (\C114)
         ([\C112])         ([\C112])
         (\C115) ([\C111]) (\C114)
\end{bsp}
\make



\begin{bsp}
\RandBox {\parbox{7cm}{Dieses ist eine riesige und gro"se Testbox
                       mit hoffentlich zwei Zeilen! Der Rand besteht
                       aus sich "uberlappenden Wellenteilen.}}
         font {\f} [-0.5em]
         (\C113) ([\C111]) (\C113)
         ([\C112])         ([\C112])
         (\C113) ([\C111]) (\C113)
\end{bsp}
\make



\begin{bsp}
\EinfachRand {Wieder mal eine (andere) Box} font {\F} char {\C43} [0pt]
\end{bsp}
\make



\begin{bsp}
\EinfachRand {Noch eine Box, diesmal mit Stacheldraht.}
    font {\F} char {\C41} [0.2em]
\end{bsp}
\make



\begin{bsp}
\EinfachRand {Und man sieht hier noch einen Kasten.}
    font {\F} char {\C42} [0.2em]
\end{bsp}
\make



\begin{bsp}
\EinfachRand {Yin und Yang} font {\f} char {\C10} [0.2em]
\end{bsp}
\make



\begin{bsp}
\EinfachRand {Nach einer Grafik von Apollonio.}
    font {\F} char {\C37} [0.2em]
\end{bsp}
\make



\begin{bsp}
\RandBox {Test in einer Box mit Buchstaben.}
         font {\tt} [0.2em]
         (W)        ([B\\C]B)       (X)
         ([D\\E\\F])                ([G\\H\\I])
         (Y)        ([J\\K]J)           (Z)
\end{bsp}
\make



\begin{bsp}
\RandBox {Wir machen das Unm"ogliche m"oglich.}
         font {\F} [0.1em]
         (\C3)              ([\C2\\\C6]\C2)         (\C1)
         ([\C4\\\C0]\C4)                        ([\C0\\\C4]\C0)
         (\C5)              ([\C6\\\C2]\C6)     (\C7)
\end{bsp}
\make



\begin{bsp}
\RandBox {Wunder dauern etwas l"anger.}
         font {\F} [0.2em]
         ()                 ([\C2\\\C6]\C2)         ()
         ([\C4\\\C0]\C4)                        ([\C0\\\C4]\C0)
         ()                 ([\C6\\\C2]\C6)     ()
\end{bsp}
\make



\begin{bsp}
\RandBox {Eine Box wie vorne, aber mit Abst"anden.}
         font {\F} [0.2em]
         ()         ([\C43])        ()
         ([\C43])                       ([\C43])
         ()         ([\C43])        ()
\end{bsp}
\make



\begin{bsp}
\RandBox {Mischmasch mit gedrehtem Yin und Yang.}
         font {\f} [0.2em]
         (\C7)      ([\C10\\\C12\\\C14\\\C16\\\C20\\\C22])      (\C5)
         ([\C21])                                               ([\C13])
         (\C1)      ([\C10\\\C12\\\C14\\\C16\\\C20\\\C22])      (\C3)
\end{bsp}
\make

\begin{bsp}
\EinfachRand {\parbox{7cm}{Dieses hier ist eine Parbox mit mehreren Zeilen
Text, die sp"ater umrahmt werden soll. Der Rahmen besteht nicht aus Linien
sondern aus Schmucksymbolen, die von \TeX\ automatisch zu einem Rahmen in
der erforderlichen Gr"o"se zusammengesetzt werden.}}
    font {\f} char {\C66} [-0.7em]
\end{bsp}
\make
}
\endinput