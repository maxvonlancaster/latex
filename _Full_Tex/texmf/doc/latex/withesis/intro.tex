% Pre-lim
% by Eric Benedict


\chapter{Introducing the {\tt withesis} \LaTeX{} Style Guide}
This manual is was written to test the {\tt withesis} style
file and to provide documentation for this style file.  

\section{History}
The
idea for this came from a similar manual written by James Darrell
McCauley and Scott Hucker in 1993 for the Purdue University thesis
style file.  Content ideas were liberally borrowed from this document.
The {\tt withesis} style file is based on the Purdue thesis file
written by Dave Kraynie and edited by Darrell McCauley.  This base was
edited to meet the format requirements of the University of 
Wisconsin--Madison and several additional new commands were created.
In addition, environments from the UW Mathematics Department were also
incorporated.

\section{Producing Your Thesis or Dissertation}
The {\tt withesis} style file will take care of most of the formatting
requirements for submitting your thesis or dissertation at the University
of Wisconsin-Madison.  There are some requirements on the printing of your
document.  From the Graduate School's {\em UW-Madison Guide To Preparing 
Your Doctoral Dissertation},
\begin{quote}\singlespace
Print your dissertation on a laser printer. (Some high quality dot-matrix
printers may be acceptable.) The printer must produce output that
meets all format and legibility requirements. A professional copy shop
can produce an acceptable copy to be submitted to the Graduate School.
Some copiers enlarge the original between one and two percent. To avoid
problems with margins, produce the original copy with margins larger than
the required minimum. Look carefully at the copy before paying for the
services and ask for pages to be recopied if necessary. Common flaws are:
smudges, copy lines, specks, missing pages, margin shifts, slanting of
the printed image on the page, and poor paper quality.
\end{quote}

\subsection{Required Paper}
The paper which is used for PhD Dissertations should be:
\begin{itemize}
\item 8-1/2 x 11 inches
\item High-quality, white
\item 20 pound weight, bond
\end{itemize}
 
While for Masters Theses, the paper should be:

\begin{itemize}
\item 8-1/2 x 11 inches
\item White
\item Acid-free or pH neutral
\item 20 pound weight
\item 25\% cotton bond minimum
\end{itemize}

Paper that meets these requirements can be purchased at book and stationery
stores.

\subsection{Copyright Page}
\label{copyright}
If you choose to retain and register copyright of the dissertation, prepare
a copyright page using the {\tt withesis} {\tt \verb|\copyrightpage|} command. 
Center the text in the bottom third of the page within the dissertation
margins. This page is not numbered. There is an additional fee for copyrighting
your dissertation which is payable at the bursars office along with the
microfilming and binding fee.

\subsection{Prechecks}
The Graduate School has reserved 9:00-9:30 each morning to answer specific formatting questions
(for example: use of tables, graphs and charts). You may bring in 8-10
pages to be reviewed. No appointment is necessary.

\subsection{Final Checks}
\sloppypar
For information about the final Graduate School review and about depositing
your dissertation in the library, see {\em The Three D's: Deadlines, Defending, 
Depositing Your Doctoral Dissertation} or look
at the web site 
\begin{quote}
{\tt http://www.wisc.edu/grad/gs/degrees/ddd.html}
\end{quote}

\section{Disclaimer}
This software and documentation is provided ``as is'' without any
express or implied warranty.
While care has been taken by the authors of this style file such that the
final product will probably meet the University of Wisconsin's formatting 
requirements this is not guaranteed. 
