\documentclass{article}
\title{The `ze' fonts
       (virtual T1 fonts)\\
       version 2.1}
\author{Robert Fuster\\
        \footnotesize Departament de Matem\`atica Aplicada\\
        \footnotesize Universitat Polit\`ecnica de Val\`encia\\
        \footnotesize 46071 Val\`encia (Spain)\\
        \footnotesize e-mail: \texttt{rfuster\@ mat.upv.es}\\
        \footnotesize URL: \texttt{HTTP://www.upv.es/\~{}rfuster}}
\date{\small June 24, 1997\\
      September 22, 1998\\
      July 17, 2000}
\begin{document}
\maketitle
\tableofcontents
\section{Description}
    The `zd' fonts by Constantin Kahn (\verb<kahn@math.uni-hannover.de<)
    are virtual T1 encoded Computer Modern fonts based on (OT1) Computer Modern,
    Times, and Helvetica fonts, intended for simulate `dc'
    fonts.\footnote{Waine Sullivan's `dm' fonts are another approach to
    the substitution of `dc' fonts by virtual ones.}

    Because `dc' fonts are now obsolete, I've adapted the Kahn's package to
    `ec' fonts. The resulting virtual fonts are named according to the ec
    fonts names, changing `ec' by `ze' (\verb+zerm1000.vf+
    simulates \verb+ecrm1000+, and so on)
                             
    Another virtual T1 encoded Computer Modern fonts are the almost european
    (`am') fonts, by Lars Engebretsen (\texttt{enge@nada.kth.se}). The main
    difference between `am' and `ze' fonts is that almost european fonts only
    use Computer Modern fonts to simulate the European Modern and, then,
    more characters are lost.
\section{Installation}
    The installation is very easy: you only need to put every file in its
    correct directory.
    \begin{itemize}
        \item Copy all the files from the \verb+vf+ directory where your
        dvi driver can find virtual files.\footnote{Your dvi driver must
        support virtual fonts, of course.}
        \item Copy all the files from the \verb+tfm+ directory where \TeX{}
        look for font metrics.
        \item Copy all the files from the \verb+texinput+ directory
        where \LaTeX{} look for input files.
        \item If you have the \verb+dvips+ driver, copy the files from the
        \verb+dvips+ directory where \verb+dvips+ looks for font mapping.
        \item \textbf{If you have an (very) old version of zefonts}, remove
        this files:
        \begin{itemize}
              \item  cmsy10o.vf
                     cmmi10o.vf
                     cmvtt10o.vf
                     cmbsy10o.vf
                     cmmib10o.vf and
                     cmbxi10f.vf
                    (from your virtual fonts path),
              \item ze.map and config.ze (from the directory where dvips
              search for font mapping).
        \end{itemize}
    \end{itemize}
\section{Warnings}
\begin{itemize}
\item
This is an unsupported software, and it comes without any kind of warranty.
\item
You can distribute it freely.
\item
My work for construct the `ze' fonts was a simple adaptation from the
`zd' fonts by C. Khan.
\end{itemize}
\end{document}
