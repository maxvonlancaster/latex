%    MetaUML: Tutorial, Reference and Test Suite
%
%    Copyright (c) 2005-2006 Ovidiu Gheorghies
%    Permission is granted to copy, distribute and/or modify this document
%    under the terms of the GNU Free Documentation License, Version 1.2
%    or any later version published by the Free Software Foundation;
%    with no Invariant Sections, no Front-Cover Texts, and no Back-Cover Texts.
%    A copy of the license is included in the section entitled "GNU
%    Free Documentation License".

\documentclass{article}

\usepackage[pdftex,colorlinks=true]{hyperref}
\usepackage{multicol}

\ifx\pdftexversion\undefined
  \usepackage[dvips]{graphicx}
\else
  \usepackage[pdftex]{graphicx}
   \DeclareGraphicsRule{*}{mps}{*}{}
\fi

\newcommand{\code}{\ttfamily}
\newcommand{\metauml}{MetaUML}

\setcounter{page}{1}

\begin{document}

\metauml: Tutorial, Reference and Test Suite

\begin{quote}
    Copyright \copyright 2005-2006 Ovidiu Gheorghie\c{s}.
    Permission is granted to copy, distribute and/or modify this document
    under the terms of the GNU Free Documentation License, Version 1.2
    or any later version published by the Free Software Foundation;
    with no Invariant Sections, no Front-Cover Texts, and no Back-Cover Texts.
    A copy of the license is included in the section entitled "GNU
    Free Documentation License".
\end{quote}

\pagebreak
This page is left intentionally blank.

\pagebreak
\title{\metauml: Tutorial, Reference and Test Suite}

\author{Ovidiu Gheorghie\c{s}}

\maketitle

\begin{abstract}
\metauml\ is a GNU GPL MetaPost library for typesetting UML diagrams, using a human-friendly textual notation. MetaUML offers a highly customizable, object-oriented API, designed with the ease of use in mind. Apart from being a reference, this manual is also a tutorial but, more importantly, a living example. You can look at its source code, getting direct accounts on ``how things are done''.
\end{abstract}

%\begin{keywords}
%MetaPost, TeX, LaTeX, UML, class diagram, state machine diagram,
%use case diagram, activity diagram
%\end{keywords}

\section{Introduction}

Here are a few diagrams created with MetaUML, just to give you a glimpse of its features:

\begin{multicols}{2}
\paragraph{A} Class Diagram\\
\includegraphics[scale=.55]{fig/appetizer.1}
\paragraph{B} Activity Diagram\\
\includegraphics[scale=.55]{fig/appetizer.2}
\paragraph{C} Notes\\
\includegraphics[scale=.55]{fig/appetizer.5}
\columnbreak
\paragraph{D} Use Case Diagram\\
\includegraphics[scale=.55]{fig/appetizer.3}
\paragraph{E} State Machine Diagram\\
\includegraphics[scale=.55]{fig/appetizer.4}
\paragraph{F} Package Diagram\\
\includegraphics[scale=.55]{fig/appetizer.6}
\end{multicols}

\pagebreak

The code which generates these diagrams is quite straightforward, combining a natural object-oriented parlance with the power of MetaPost equation solving; for more information on MetaPost see \cite {metapost}.

An UML class, for example, can be drawn as follows:

\begin{multicols}{2}
\begin{verbatim}
Class.A("MyClass")
  ("attr1: int", "attr2: int")
  ("method1(): void",
   "method2(): void");

A.nw = (0, 0); % optional, implied
drawObject(A);
\end{verbatim}
\columnbreak
\hspace{1cm}\includegraphics{fig/appetizer.7}
\end{multicols}

This piece of code creates an instance of {\code Class}, which will be afterward
identified as {\code A}. This object has the following content properties: a name
({\code MyClass}), a list of attributes ({\code attr1}, {\code attr2})
and a list of methods ({\code method1}, {\code method2}). The one thing remaining
before actually drawing {\code A} is to set its location.

\begin{figure}
\centering
\includegraphics{fig/properties.1}
\caption{Positioning properties of any MetaUML object (here a class object is depicted).}
\label{fig:properties}
\end{figure}

In {\code A.nw} we refer to the ``north-west'' of the class rectangle, that is
to its upper-left corner. In general, every MetaUML object has the positioning
properties given in figure \ref{fig:properties}. These properties are used to set
where to draw a given object, whether by assigning them absolute values, or by setting
them relatively to other objects. Suppose that we have defined two classes
{\code A} and {\code  B}. Then the following code would give a conceivable positioning:

\begin{multicols}{2}
\begin{verbatim}
A.nw = (0,0);
B.e = A.w + (-20, 0);
\end{verbatim}
\columnbreak
\includegraphics{fig/appetizer.8}
\end{multicols}

After the objects are drawn, one may draw links between them, such as inheritance
or association relations between classes in class diagrams, or transitions between states
in state machine diagrams. Whichever the purpose is, MetaUML provides a generic
way of drawing an edge in a diagram's graph:

\begin{verbatim}
link(how-to-draw-information)(path-to-draw);
\end{verbatim}

The ``how to draw information'' is actually an object which defines the style
of the line (e.g. solid, dashed) and the appearance of the heads (e.g. nothing, arrow, diamond).
One such object, called {\code inheritance}, defines a solid path ending in
a white triangle. The {\code path-to-draw} parameter is simply a MetaPost path.
For example, the following code can be used used to represent that class {\code B} is derived from {\code A}:

\begin{verbatim}
link(inheritance)(B.e -- A.w);
\end{verbatim}

Note that the direction of the path is important, and MetaUML uses it to determine the
type of adornment to attach at the link ends (if applicable). In our example, a white triangle,
denoting inheritance, points towards the end of the path, that is towards class {\code A}.

To sum up, we present a short code and the resulting diagram, typical for just about
everything else in MetaUML. The positioning of {\code A} does not need to be
explicitly set because ``floating'' objects are automatically positioned at {\code (0,0)} by their
draw method.

\begin{multicols}{2}

\begin{verbatim}
input metauml;
beginfig(1);
  Class.A("A")()();
  Class.B("B")()();
  B.e = A.w + (-20, 0);
  drawObjects(A, B);
  link(inheritance)(B.e -- A.w);
endfig;
end
\end{verbatim}
\columnbreak
\includegraphics{fig/appetizer.9}
\end{multicols}

From a user's perspective, this is all there is to MetaUML. With a reference describing how other
UML elements are created, one can set out to typeset arbitrary complex diagrams.

\section{Class Diagrams}

A class is created as follows:

\begin{verbatim}
Class.name(class-name)
  (list-of-attributes)
  (list-of-methods);
\end{verbatim}

The suffix {\code name} gives a name to the {\code Class} object (which, of course, represents an UML class).
The name of the UML class is a string given by {\code class-name};
the attributes are given as a comma separated list of strings, {\code list-of-attributes};
the methods are given as a comma separated list of strings, {\code list-of-attributes}.
The list of attributes and the list of methods may be void.

Each of the strings representing an attribute or a method may begin with a visibility marker: ``$+$'' for
public, ``\#'' for protected and ``$-$'' for private. MetaUML interprets this marker and renders a
graphic stereotype in form of a lock which may be opened, semi-closed and closed, respectively.

Here is an example:

\begin{multicols}{2}
\begin{verbatim}
Class.A("Point")
  ("#x:int", "#y:int")
  ("+set(x:int, y:int)",
   "+getX():int",
   "+getY():int",
   "-debug():void");
drawObject(A);
\end{verbatim}
\columnbreak
\includegraphics{fig/class.1}
\end{multicols}

\subsection{Stereotypes}

After a class is created, its stereotypes may be specified by using the macro {\code classStereotypes}:

\begin{verbatim}
classStereotypes.name(list-of-stereotypes);
\end{verbatim}

Here, {\code name} is the object name of a previously created class and {\code list-of-stereotypes}
is a comma separated list of strings. Here is an example:

\begin{multicols}{2}
\begin{verbatim}
Class.A("User")()();
classStereotypes.A("<<interface>>",
                   "<<home>>");

drawObject(A);
\end{verbatim}
\columnbreak
\hspace{1cm}\includegraphics{fig/class.2}
\end{multicols}

\subsection{Interfaces and Abstract Classes}

At times it is prefered to typeset the name of an interface in an oblique font, rather than using the ``interface'' stereotype. This can be easily achieved by using the macro:

\begin{verbatim}
Interface.name(class-name)
  (list-of-methods);
\end{verbatim}

Here is an example:

\begin{multicols}{2}
\begin{verbatim}
Interface.A("Observer")
    ("+update(src:Object)");

drawObject(A);
\end{verbatim}
\columnbreak
\hspace{1cm}\includegraphics{fig/class.11}
\end{multicols}

Note that {\code Interface} is a special kind of {\code Class}, the declaration code above being equivalent to:
\begin{verbatim}
EClass.A(iInterface)("Observer")()
      ("+update(src:Object)");
\end{verbatim}

Along the same line, here's how abstract classes can be drawn:

\begin{multicols}{2}
\begin{verbatim}
EClass.A(iAbstractClass)("Observable")
    ("observers: Observer[0..*]")
    ("+addObserver(o: Observer)",
     "+notify()");

drawObject(A);
\end{verbatim}
\columnbreak
\hspace{1cm}\includegraphics{fig/class.12}
\end{multicols}

If you prefer, you can use the syntactic sugar:

\begin{verbatim}
AbstractClass.A("Observable")
    ("observers: Observer[0..*]")
    ("+addObserver(o: Observer)",
     "+notify()");
\end{verbatim}

\subsection{Objects (or Class Instances)}

An UML object (or class instance) is created as follows:

\begin{verbatim}
Instance.name(object-name)
  (list-of-attributes);
\end{verbatim}

The suffix {\code name} gives a name to the {\code Instance} object. The name of the object (given by {\code object-name}) is typeset underlined. The attributes are given as a comma separated list of strings, {\code list-of-attributes}.

\begin{multicols}{2}
\begin{verbatim}
Instance.order("o: Order")
  ("name='book'", "{placed}", "{payed}");
drawObject(order);
\end{verbatim}
\columnbreak
\hspace{2cm}\includegraphics{fig/instance.1}
\end{multicols}


\subsection{Parametrized Classes (Templates)}

The most convenient way of typesetting a class template in \metauml\ is to use the macro {\code ClassTemplate}.
This macro creates a visual object which is appropriately positioned near the class object it adorns.

\begin{verbatim}
ClassTemplate.name(list-of-templates)
                  (class-object);
\end{verbatim}

The {\code name} is the name of the template object, {\code list-of-templates} is a comma separated list of strings and the {\code class-object} is the name of a class object.

Here is an example:

\begin{multicols}{2}
\begin{verbatim}
Class.A("Vector")()();
ClassTemplate.T("T", "size: int")(A);

drawObjects(A, T);
\end{verbatim}
\columnbreak
\hspace{1cm}\includegraphics{fig/class.3}
\end{multicols}

The macro {\code Template} can also be used to create a template object, but this time the resulting
object can be positioned freely.

\begin{verbatim}
Template.name(list-of-templates);
\end{verbatim}

Of course, one can specify both stereotypes and template parameters for a given class.

\subsection{Types of Links}

In this section we enumerate the relations that can be drawn between classes by means
of \metauml\ macros. Suppose that we have the declared two points, {\code A} (on the left)
and {\code B} (on the right):

\begin{verbatim}
pair A, B;
A = (0,0);
B = (50,0);
\end{verbatim}

\begin{tabular}{||l|c||}
\hline
{\code link(association)(X.e -- Y.w)} & \includegraphics{fig/class_diagrams.4} \\
\hline
{\code link(associationUni)(X.e -- Y.w)} & \includegraphics{fig/class_diagrams.5}  \\
\hline
{\code link(inheritance)(X.e -- Y.w)} & \includegraphics{fig/class_diagrams.6} \\
\hline
{\code link(aggregation)(X.e -- Y.w)} & \includegraphics{fig/class_diagrams.7} \\
\hline
{\code link(associationUni)(X.e -- Y.w)} & \includegraphics{fig/class_diagrams.8} \\
\hline
{\code link(composition)(X.e -- Y.w)} & \includegraphics{fig/class_diagrams.9} \\
\hline
{\code link(compositionUni)(X.e -- Y.w)} & \includegraphics{fig/class_diagrams.10} \\
\hline
{\code link(dependency)(X.e -- Y.w)} & \includegraphics{fig/class_diagrams.11} \\
\hline
\end{tabular}

\subsection{Associations}
In UML an association typically has two of association ends and may have a name specified for it.
In turn, each association end may specify a multiplicity, a role, a visibility, an ordering.
These entities are treated in \metauml\ as pictures having specific drawing information
(spacings, font).

The first method of creating association ``items'' is by giving them explicit names.
Having a name for an association item comes in handy when referring to its properties
is later needed (see the non UML-compliant diagram below). Note that the last parameter of the macro {\code item} is an equation which uses the item name to perform positioning.

\begin{multicols}{2}
\begin{verbatim}
Class.P("Person")()();
Class.C("Company")()();
% drawing code ommited

item.aName(iAssoc)("works for")
          (aName.s = .5[P.w, C.w]);
draw aName.n -- (aName.n + (20,20));
label.urt("association name" infont "tyxtt",
          aName.n + (20,20));
\end{verbatim}
\columnbreak
\hspace{1cm}\includegraphics[scale=.8]{fig/class_association.1}
\end{multicols}

However, giving names to every association item may become an annoying burden
(especially when there are many of them). Because of this, \metauml\ also allows for
``anonymous items''. In this case, the positioning is set by an equation
which refers to the anonymous item as {\code obj}.

\begin{multicols}{2}
\begin{verbatim}
% P and C defined as in the previous example

item(iAssoc)("employee")(obj.nw = P.s);
item(iAssoc)("1..*")(obj.ne = P.s);

% other items are drawn similarly
\end{verbatim}
\columnbreak
\hspace{3cm}\includegraphics{fig/class_association.2}
\end{multicols}

\subsection{Dependencies and Stereotypes}

Stereotypes are frequently used with dependencies. Below is an example.
\pagebreak

\begin{multicols}{2}
\begin{verbatim}
Class.F("Factory")()();
Class.O("Object")()();

O.n = F.s - (0, 50);
drawObjects(F, O);

clink(dependency)(F, O);
item(iStereo)("<<creates>>")(obj.w = .5[F.s,O.n])
\end{verbatim}
\columnbreak
\hspace{3cm}\includegraphics{fig/class_association.3}
\end{multicols}

\section{Notes}

A note is created as follows:

\begin{verbatim}
Note.name(list-of-lines);
\end{verbatim}

The suffix {\code name} is the name of the {\code Note} object. The comma separated list of strings, {\code list-of-lines}, gives the text contents of the note object, each string being drawn on its own line.
Here is an example:

\begin{multicols}{2}
\begin{verbatim}
Note.A("This note", "has two lines.");
drawObject(A);
\end{verbatim}
\columnbreak
\hspace{3cm}\includegraphics{fig/note.1}
\end{multicols}

\subsection{Attaching notes to diagram elements}

Notes can be attached to diagram elements by using a link of type {\code dashedLink}.

\begin{multicols}{2}
\begin{verbatim}
Note.A("This is a class");
Class.C("Object")()();

A.sw = C.ne + (20, 20);

drawObject(A, C);

clink(dashedLink)(A, C);
\end{verbatim}
\columnbreak
\hspace{1cm}\includegraphics{fig/note.2}
\end{multicols}

Now let us see a more complex example, which demontrates the ability of accessing sub-elements in a \metauml\ diagram.
\pagebreak

\begin{multicols}{2}
\begin{verbatim}
Note.nA("This is the class name");
Note.nB("This is a key attribute");
Note.nC("This is a nice method");

Class.C("Object")("+id:int")
        ("+clone()", "+serialize()");

topToBottom.left(10)(nA, nB, nC);
leftToRight(10)(C, nB);

drawObjects(C, nA, nB, nC);

clink(dashedLink)(C.namePict, nA); 
clink(dashedLink)(C.attributeStack.pict[0], nB); 
clink(dashedLink)(C.methodStack.pict[1], nC);
\end{verbatim}
\columnbreak
\hspace{1cm}\includegraphics{fig/note.3}
\end{multicols}

Macros like {\code leftToRight} and {\code topToBottom} are presented in section \ref{section:positioning}.

\section{Packages}

MetaUML allows for the creation of packages in various forms. Firstly, we have the option of writing the package name in the middle of the main box. Secondly, we can write the name on the tiny box above the main box, leaving the main box empty. Lastly, we can write the package name as in the second case, but the main box can have an arbitrary contents: classes, other packages, or even other UML items. 

The macro that creates a package has the following synopsis:

\begin{verbatim}
Package.name(package-name)(subitems-list);
\end{verbatim}

The parameter {\code package-name} is a string or a list of comma separated strings representing the package's name. The {\code subitems-list} parameter is used to specify the subitems (tipically classes or packages) of this package; its form is as a comma separated list of objects, which can be void.

\begin{multicols}{2}
\begin{verbatim}
Package.P("java.lang")();
drawObject(P);
\end{verbatim}
\columnbreak
\hspace{3cm}\includegraphics{fig/package.1}
\end{multicols}

Below is another example:

\begin{multicols}{2}
\begin{verbatim}
Package.P("An important", "package")();
drawObject(P);
\end{verbatim}
\columnbreak
\hspace{3cm}\includegraphics{fig/package.2}
\end{multicols}

If you wish to leave the main box empty, you can use the following code:

\begin{multicols}{2}
\begin{verbatim}
Package.P("java.lang")();
P.info.forceEmptyContent := 1;
drawObject(P);
\end{verbatim}
\columnbreak
\hspace{3cm}\includegraphics{fig/package.3}
\end{multicols}

The same effect as above can be achieved globally by doing:

\begin{verbatim}
iPackage.forceEmptyContent := 1;
\end{verbatim}

More information on MetaUML's way of managing global and per-object configuration data can be found in section \ref{section:infrastructure} and section \ref{section:customization}.

Here is an example involving items contained in a package.

\begin{multicols}{2}
\begin{verbatim}
Class.A("A")()();
Class.B("B")()();
Package.P("net.metauml")(A, B);

leftToRight(10)(A, B);

drawObject(P);
\end{verbatim}
\columnbreak
\hspace{3cm}\includegraphics{fig/package.4}
\end{multicols}

\subsection{Types of Links}

The nesting relation between packages is created by using the {\code nest} link information.

\begin{tabular}{||l|c||}
\hline
{\code link(nest)(X.e -- Y.w)} & \includegraphics{fig/package.5} \\
\hline
\end{tabular}

\section{Use Case Diagrams}

\subsection{Use Cases}
An use case is created by the macro {\code Usecase}:

\begin{verbatim}
Usecase.name(list-of-lines);
\end{verbatim}

The {\code list-of-lines} is a comma separated list of strings. These strings are placed
on top of each other, centered and surrounded by the appropriate visual UML notation.

Here is an use case example:

\begin{multicols}{2}
\begin{verbatim}
Usecase.U("Authenticate user",
          "by name, password");
drawObject(U);
\end{verbatim}
\columnbreak
\hspace{1cm}\includegraphics{fig/usecase.1}
\end{multicols}

\subsection{Actors}

An actor is created by the macro {\code Actor}:

\begin{verbatim}
Actor.name(list-of-lines);
\end{verbatim}

Here, {\code list-of-lines} represents the actor's name. For convenience, the name may be
given as a list of strings which are placed on top of each other, to provide support for
the situations when the role is quite long. Otherwise, giving a single string
as an argument to the Actor constructor is perfectly fine.

Here is an actor example:

\begin{multicols}{2}
\begin{verbatim}
Actor.A("User");
drawObject(A);
\end{verbatim}
\columnbreak
\hspace{1cm}\includegraphics{fig/usecase.2}
\end{multicols}

Note that one may prefer to draw diagram relations positioned relatively to
the visual representation of an actor (the ``human'') rather than relatively to the whole
actor object (which also includes the text). Because of that, MetaUML provides access
to the ``human'' of every actor object {\code actor} by means of the sub-object {\code actor.human}.

\begin{multicols}{2}
\begin{verbatim}
Actor.A("Administrator");
drawObject(A);
draw objectBox(A);
draw objectBox(A.human);
\end{verbatim}
\columnbreak
\hspace{1cm}\includegraphics{fig/usecase.3}
\end{multicols}

Note that in \metauml\ {\code objectBox(X)} is equivalent to {\code X.nw -- X.ne -- X.se -- X.sw -- cycle} for every object {\code X}. {\code A.human} is considered a \metauml\ object, so you can use expressions like {\code A.human.n} or {\code A.human.midx}.

\subsection{Types of Links}

Some of the types of links defined for class diagrams (such as inheritance, association etc.) can be used with similar semantics within use case diagrams.

\section{Activity Diagrams}

\subsection{Begin, End and Flow End}

The begin and the end of an activity diagram can be marked by using the macros {\code Begin}
and {\code End} or {\code FlowFinal}, respectively. The constructors of these visual objects take no parameters:

\begin{verbatim}
Begin.beginName;
End.endName;
\end{verbatim}

Below is an example:

\begin{multicols}{2}
\begin{verbatim}
Begin.b;
End.e;
FlowFinal.f;

leftToRight(20)(b, e, f);

drawObjects(b, e, f);
\end{verbatim}
\columnbreak
\hspace{1cm}\includegraphics{fig/activity.1}
\end{multicols}

\subsection{Activity}

An activity is constructed as follows:
\begin{verbatim}
Activity.name(list-of-strings);
\end{verbatim}

The parameter {\code list-of-strings} is a comma separated list of strings. These strings are
centered on top of each other to allow for the accommodation of a longer activity description
within a reasonable space.

An example is given below:

\begin{multicols}{2}
\begin{verbatim}
Activity.A("Learn MetaUML -",
           "the MetaPost UML library");
drawObject(A);
\end{verbatim}
\columnbreak
\hspace{1cm}\includegraphics{fig/activity.2}
\end{multicols}

\subsection{Fork and Join}

A fork or join is created by the macro:

\begin{verbatim}
Fork.name(type, length);
\end{verbatim}

The parameter {\code type} is a string and can be either of {\code "h"}, {\code "horiz"}, {\code "horizontal"} for horizontal bars, and either of {\code "v"}, {\code "vert"}, {\code "vertical"} for vertical bars. The {\code length} gives the bar's length.

\begin{multicols}{2}
\begin{verbatim}
Fork.forkA("h", 100);
Fork.forkB("v", 20);

leftToRight(10)(forkA, forkB);

drawObject(forkA, forkB);
\end{verbatim}
\columnbreak
\hspace{1cm}\includegraphics{fig/activity.3}
\end{multicols}

\subsection{Branch}

A branch is created by the macro:

\begin{verbatim}
Branch.name;
\end{verbatim}

Here is an example:

\begin{multicols}{2}
\begin{verbatim}
Branch.testA;

drawObject(testA);
\end{verbatim}
\columnbreak
\hspace{1cm}\includegraphics{fig/activity.4}
\end{multicols}


\subsection{Types of Links}

In activity diagrams, transitions between activities are needed. They are typeset
as in the example below. In section \ref{composite-states} such a transition
is showed. This type of link is also used for state machine diagrams.

\begin{verbatim}
link(transition)( pointA -- pointB );
\end{verbatim}

\section{State Diagrams}

The constructor of a state allows for aggregated sub-states:

\begin{verbatim}
State.name(state-name)(substates-list);
\end{verbatim}

The parameter {\code state-name} is a string or a list of comma separated strings representing
the state's name or description. The {\code substates-list} parameter is used to specify
the substates of this state as a comma separated list of objects; this list may be void.

An example of a simple state:

\begin{multicols}{2}
\begin{verbatim}
State.s("Take order")();
drawObject(s);
\end{verbatim}
\columnbreak
\hspace{1cm}\includegraphics{fig/state.1}
\end{multicols}


\subsection{Composite States}
\label{composite-states}

A composite state is defined by enumerating at the end of its constructor the inner
states. Interestingly enough, the composite state takes care of drawing the sub-states it
contains. The transitions must be drawn after the composite state, as seen in the
next example:

\begin{multicols}{2}
\begin{verbatim}
Begin.b;
End.e;
State.c("Component")();
State.composite("Composite")(b, e, c);

b.midx = e.midx = c.midx;
c.top = b.bottom - 20;
e.top = c.bottom - 20;

composite.info.drawNameLine := 1;
drawObject(composite);

link(transition)(b.s -- c.n);
link(transition)(c.s -- e.n);
\end{verbatim}
\columnbreak
\hspace{1cm}\includegraphics{fig/state.2}
\end{multicols}

\subsection{Internal Transitions}

Internal transitions can be specified by using the macro:
\begin{verbatim}
stateTransitions.name(list-transitions);
\end{verbatim}

Identifier {\code name} gives the state object whose internal transitions are being set,
and parameter {\code list-transitions} is a comma separated string list.


An example is given below:

\begin{multicols}{2}
\begin{verbatim}
State.s("An interesting state",
        "which is worth mentioning")();
stateTransitions.s(
  "OnEntry / Open eyes",
  "OnExit  / Sleep well");
s.info.drawNameLine := 1;

drawObject(s);
\end{verbatim}
\columnbreak
\hspace{1cm}\includegraphics{fig/state.3}
\end{multicols}

\subsection{Special States}

Similarly to the usage of {\code Begin} and {\code End} macros, one can define history states,
exit/entry point states and terminate pseudo-states, by using the following constructors.

\begin{verbatim}
History.nameA;
ExitPoint.nameB;
EntryPoint.nameC;
Terminate.nameD;
\end{verbatim}

\section{Drawing Paths}

The {\code link} macro is powerful enough to draw relations following arbitrary paths:

\begin{multicols}{2}
\begin{verbatim}
path cool;
cool := A.e .. A.e+(20,10) ..
        B.s+(20,-40) .. B.s+(-10,-30)
        -- B.s;
link(inheritance)(cool);

link(aggregationUni)
    (A.n ..(30,30)..B.w);
\end{verbatim}
\columnbreak
\hspace{1cm}\includegraphics{fig/paths.1}
\end{multicols}

Regardless of how amusing this feature might be, it does become a bit of a nuisance to
use it in its bare form. When typesetting UML diagrams in good style, one generally
uses rectangular paths. It is for this kind of style that \metauml\ offers extensive
support, providing a ``syntactic sugar'' for constructs which can otherwise be
done by hand, but with some extra effort.

\subsection{Manhattan Paths}

The ``Manhattan'' path macros generate a path between two points consisting of one
horizontal and one vertical segment. The macro {\code pathManhattanX} generates first a
horizontal segment, while the macro {\code pathManhattanY} generates first a
vertical segment. In \metauml\ it also matters the direction of a path, so you
can choose to reverse it by using {\code rpathManhattanX} and {\code rpathManhattanY}
(note the prefix ``{\code r}''):

\begin{verbatim}
pathManhattanX(A, B)
pathManhattanY(A, B)

rpathManhattanX(A, B)
rpathManhattanY(A, B)
\end{verbatim}

\pagebreak
Here is an example:

\begin{multicols}{2}
\begin{verbatim}
Class.A("A")()();
Class.B("B")()();

B.sw = A.ne + (10,10);
drawObjects(A, B);

link(aggregationUni)
   (rpathManhattanX(A.e, B.s));
link(inheritance)
   (pathManhattanY(A.n, B.w));
\end{verbatim}
\columnbreak
\hspace{1cm}\includegraphics{fig/paths.2}
\end{multicols}

\subsection{Stair Step Paths}

These path macros generate stair-like paths between two points.
The ``stair'' can ``rise'' first in the direction of $Ox$ axis ({\code pathStepX})
or in the direction of $Oy$ axis ({\code pathStepY}). How much should a step
rise is given by an additional parameter, {\code delta}. Again, the macros
prefixed with ``{\code r}'' reverse the direction of the path given by their
unprefixed counterparts.

\begin{verbatim}
pathStepX(A, B, delta)
pathStepY(A, B, delta)

rpathStepX(A, B, delta)
rpathStepY(A, B, delta)
\end{verbatim}

Here is an example:

\begin{multicols}{2}
\begin{verbatim}
stepX:=60;
link(aggregationUni)
   (pathStepX(A.e, B.e, stepX));

stepY:=20;
link(inheritance)
   (pathStepY(B.n, A.n, stepY));
\end{verbatim}
\columnbreak
\hspace{1cm}\includegraphics{fig/paths.3}
\end{multicols}

\subsection{Horizontal and Vertical Paths}

There are times when drawing horizontal or vertical links is required,
even when the objects are not properly aligned. To this aim, the following macros
are useful:

\begin{verbatim}
pathHorizontal(pA, untilX)
pathVertical(pA, untilY)

rpathHorizontal(pA, untilX)
rpathVertical(pA, untilY)
\end{verbatim}

A path created by {\code pathHorizonal} starts from the point {\code pA}
and continues horizontally until coordinate {\code untilX} is reached. The macro
{\code pathVertical} constructs the path dually, working vertically.
The prefix ``{\code r}'' reverses the direction of the path.

Usage example:

\begin{multicols}{2}
\begin{verbatim}
untilX := B.left;
link(association)
   (pathHorizontal(A.e, untilX));

untilY:= C.bottom;
link(association)
   (pathVertical(A.n, untilY));
\end{verbatim}
\columnbreak
\hspace{1cm}\includegraphics{fig/paths.4}
\end{multicols}

\subsection{Direct Paths}

A direct path can be created with {\code directPath}. The call {\code directPath(A, B)}
is equivalent to {\code A -{}-  B}.

\subsection{Paths between Objects}

Using the constructs presented above, it is clear that one can draw links between diagram
objects, using a code like:

\begin{verbatim}
link(transition)(directPath(objA.nw, objB.se));
\end{verbatim}

There are times however this may yield unsatisfactory visual results,
especially when the appearance of the object's corners is round. MetaUML provides the macro
{\code pathCut} whose aim is to limit a given path exactly to the region outside the actual
borders of the objects it connects. The macro's synopsis is:

\begin{verbatim}
pathCut(thePath)(objectA, objectB)
\end{verbatim}

Here, {\code thePath} is a given MetaPost path and {\code objectA} and {\code objectB}
are two \metauml\ objects. By contract, each \metauml\ object of type, say, {\code X}
defines a macro {\code X\_border} which returns the path that surrounds it. Because
of that, {\code pathCut} can make the appropriate modifications to {\code thePath}.

The following code demonstrates the benefits of the {\code pathCut} macro:

\begin{multicols}{2}
\begin{verbatim}
z = A.se + (30, -10);
link(transition)
   (pathCut(A, B)(A.c--z--B.c));
\end{verbatim}
\columnbreak
\hspace{1cm}\includegraphics{fig/paths.5}
\end{multicols}

\subsubsection{Direct Paths between Centers}

At times is quicker to just draw direct paths between the center of two objects,
minding of course the object margins. The macro which does this is {\code clink}:

\begin{verbatim}
clink(how-to-draw-information)(objA, objB);
\end{verbatim}

The parameter {\code how-to-draw-information} is the same as for the macro {\code link};
{\code objA} and {\code objB} are two \metauml\ objects.

Below is an example which involves the inheritance relation:

\begin{multicols}{2}
\begin{verbatim}
clink(inheritance)(A, B);
\end{verbatim}
\columnbreak
\hspace{1cm}\includegraphics{fig/paths.6}
\end{multicols}

\section{Arranging Diagram Items}
\label{section:positioning}

Using equations involving cardinal points, such as {\code A.nw = B.ne + (10,0)}, is
good enough for achieving the desired results. However, programs are best to
be written for human audience, rather than for compilers. It does become a bit
tiresome to think all the time of cardinal points and figure out the
direction of positive or negative offsets. Because of that, \metauml\ offers
syntactic sugar which allows for an easier understanding of the intent behind
the positioning code.

Suppose that we have three classes, {\code A}, {\code B}, {\code C} and their base class
{\code Base}. We want the base class to be at the top, and the derived classes to be
on a line below. A code like the following will do:

\begin{verbatim}
A.ne = B.nw + (20,0);
B.ne = C.nw + (20,0);
Base.s = B.n + (0,-20);
\end{verbatim}

Now, look at the code again. What strikes you is that you cannot visualize what it is all about, unless you really try --- decoding the intent line by line. What this code lacks is a feature called self-documenting: the code is good only if you can read it as a story and understand its meaning.

Perhaps the following version of the code will make the point. All you need to know is that the numeric argument represents a distance.

\begin{multicols}{2}
\begin{verbatim}
leftToRight(20)(A, B, C);
topToBottom(20)(Base, B);
\end{verbatim}
\columnbreak
\hspace{1cm}\includegraphics{fig/positioning.2}
\end{multicols}

Below there are examples which show how these macros can be used. Suppose that we have the following definitions for objects {\code X}, {\code Y}, and {\code Z}; also, let's assume that {\code spacing} is a numeric variable set to {\code 5}.

\begin{verbatim}
Picture.X("a");
Picture.Y("...");
Picture.Z("Cyan");
\end{verbatim}

\begin{tabular}{||l|c||}
\hline
{\code leftToRight.top(spacing)(X, Y, Z);} & \includegraphics{fig/positioning.3} \\
\hline
{\code leftToRight.midy(spacing)(X, Y, Z);} & \includegraphics{fig/positioning.4} \\
\hline
{\code leftToRight.bottom(spacing)(X, Y, Z);} & \includegraphics{fig/positioning.5} \\
\hline
{\code topToBottom.left(spacing)(X, Y, Z);} & \includegraphics{fig/positioning.6} \\
\hline
{\code topToBottom.midx(spacing)(X, Y, Z);} & \includegraphics{fig/positioning.7} \\
\hline
{\code topToBottom.right(spacing)(X, Y, Z);} & \includegraphics{fig/positioning.8} \\
\hline
\end{tabular} \\

To make typesetting even quicker in frequent usage scenarios, the following equivalent contructs are also allowed:

\begin{verbatim}
leftToRight.midy(spacing)(X, Y, Z);
leftToRight(spacing)(X, Y, Z);
\end{verbatim}

\begin{verbatim}
topToBottom.midx(spacing)(X, Y, Z);
topToBottom(spacing)(X, Y, Z);
\end{verbatim}

If you want to specify that some objects have a given property equal, while the distance between them is given elsewhere, you can use the macro {\code same}.
This macro accepts a variable number of parameters, but at least two. The following table gives the interpretation of the macro for a simple example.

\begin{tabular}{||l|l||}
\hline
{\code same.top(X, Y, Z);} & {\code X.top = Y.top = Z.top;} \\
\hline
{\code same.midy(X, Y, Z);} & {\code X.midy = Y.midy = Z.midy;} \\
\hline
{\code same.bottom(X, Y, Z);} & {\code X.bottom = Y.bottom = Z.bottom;} \\
\hline
{\code same.left(X, Y, Z);} & {\code X.left = Y.left = Z.left;} \\
\hline
{\code same.midx(X, Y, Z);} & {\code X.midx = Y.midx = Z.midx;} \\
\hline
{\code same.right(X, Y, Z);} & {\code X.right = Y.right = Z.right;} \\
\hline
\end{tabular} \\


To specify the relative position of two points more easily, one can use the macros {\code below}, {\code above}, {\code atright}, {\code atleft}. Let us assume that {\code A} and {\code B} are two points (objects of type {\code pair} in MetaPost). The following constructs are equivalent:

\begin{tabular}{||l|l||}
\hline
{\code B = A + (5,0);} & {\code B = atright(A, 5);} \\
{\code B = A - (5,0);} & {\code B = atleft(A, 5);} \\
{\code B = A + (0,5);} & {\code B = above(A, 5);} \\
{\code B = A - (0,5);} & {\code B = below(A, 5);} \\
\hline
\end{tabular}


\section{The MetaUML Infrastructure}
\label{section:infrastructure}

MetaPost is a macro language based on equation solving. Using it may seem quite
tricky at first for a programmer accustomed to modern object-oriented languages.
However, the great power of MetaPost consists in its versatility. Indeed, it is possible to write
a system which mimics quite well object-oriented behavior. Along this line, METAOBJ
(\cite{metaobj}) is a library worth mentioning: it provides a high-level objects
infrastructure along with a battery of predefined objects.

Surprisingly enough, \metauml\ does not use METAOBJ. Instead, it uses a custom written,
lightweight object-oriented infrastructure, provisionally called ``{\code util}''.
METAOBJ's facilities, although impressive, were perceived by me as being a bit too much
for what was initially intented as a quick way of getting some UML diagrams layed out.
Inspired by METAOBJ, ``{\code util}'' was designed to fulfill with minimal effort
the specific tasks needed to confortably position, allign or group visual objects
which include text.

Another library having some object-oriented traits is the {\code boxes}
library, which comes with the standard MetaPost distribution. Early versions of
MetaUML did use {\code boxes} as an infrastructure, but this approach had to be abandoned eventually.
The main reason was that it was difficult to achieve good visual results when stacking texts
(more on that further on). Also, it had a degree of flexibility which became apparent to be
insufficient.

\subsection{Motivation}

Suppose that we want to typeset two texts with their bottom lines aligned, using {\code boxit}:

\begin{multicols}{2}
\begin{verbatim}
boxit.a ("yummy");
boxit.b ("cool");

a.nw = (0,0); b.sw = a.se + (10,0);

drawboxed (a, b); % or drawunboxed(a,b)
draw a.sw -- b.se dashed evenly
   withpen pencircle scaled 1.1;
\end{verbatim}
\columnbreak
\hspace{1cm}\includegraphics{fig/boxes_vs_util.1}
\end{multicols}

Note that, despite supposedly having their bottoms alligned,
``yummy'' {\it looks} slightly higher than ``cool''. This would be unacceptable
in an UML class diagram, when roles are placed at the ends of a horizontal association.
Regardless of default spacing being smaller in the {\code util} library,
the very same unfortunate misalignment effect rears its ugly head:

\pagebreak

\begin{multicols}{2}
\begin{verbatim}
Picture.a("yummy");
Picture.b("cool");
% comment next line for unboxed
a.info.boxed := b.info.boxed := 1;

b.sw = a.se + (10,0);

drawObjects(a, b);
\end{verbatim}
\columnbreak
\hspace{1cm}\includegraphics{fig/boxes_vs_util.2}
\end{multicols}

However, the strong point of {\code util} is that we have a recourse to this problem:

\begin{multicols}{2}
\begin{verbatim}
iPict.ignoreNegativeBase := 1;

Picture.a("yummy");
Picture.b("cool");
% the rest the same as above
drawObjects(a, b);
\end{verbatim}
\columnbreak
\hspace{1cm}\includegraphics{fig/boxes_vs_util.3}
\end{multicols}

\subsection{The Picture Macro}

We have seen previously the line {\code iPict.ignoreNegativeBase := 1}.
Who is {\code iPict} and what is it doing in our program? \metauml\
aims at separating the ``business logic'' (what to draw) from the
``interface'' (how to draw). In order to achieve this, it records the ``how to draw''
information within the so-called {\code Info} structures. The object {\code iPict}
is an instance of {\code PictureInfo} structure, which has the following properties
(or attributes):
\begin{verbatim}
left, right, top, bottom
ignoreNegativeBase
boxed, borderColor
\end{verbatim}

The first four attributes specify how much space should be left around the
actual item to be drawn. The marvelous effect of {\code ignoreNegativeBase}
has just been shown (off), while the last two attributes control whether the border
should be drawn (when {\code boxed=1}) and if drawn, in which color.

There's one more thing: the font to typeset the text in. This is specified
in a {\code FontInfo} structure which has two attributes: the font name
and the font scale. This information is kept within the {\code PictureInfo} structure
as a contained attribute {\code iFont}. Both {\code FontInfo} and {\code PictureInfo}
have ``copy constructors'' which can be used to make copies. We have already
the effect of these copy constructors at work, when we used:

\begin{verbatim}
Picture.a("yummy");
a.info.boxed := 1;
\end{verbatim}

A copy of the default info for a picture, {\code iPict}, has been made within
the object {\code a} and can be accessed as {\code a.info}. Having a copy of the
info in each object may seem like an overkill, but it allows for a fine grained
control of the drawing mode of each individual object. This feature comes in very
handy when working with a large number of settings, as it is the case for \metauml.

Let us imagine for a moment that we have two types of text to write: one with a small font
and a small margin and one with a big font and a big margin. We could in theory
configure each individual object or set back and forth global parameters, but
this is far for convenient. It is preferable to have two sets of settings and specify
them explicitly when they are needed. The following code could be placed somewhere
in a configuration file and loaded before any {\code beginfig} macro:
\begin{verbatim}
PictureInfoCopy.iBig(iPict);
iBig.left := iBig.right := 20;
iBig.top := 10;
iBig.bottom := 1;
iBig.boxed := 1;
iBig.ignoreNegativeBase := 1;
iBig.iFont.name := defaultfont;
iBig.iFont.scale := 3;

PictureInfoCopy.iSmall(iPict);
iSmall.boxed := 1;
iSmall.borderColor := green;
\end{verbatim}

Below is an usage example of these definitions. Note the name of the macro: {\code EPicture}.
The prefix comes form ``explicit''  and it's used to acknowledge that the
``how to draw'' information is given explicitly --- as a parameter,
rather than defaulted to what's recorded in {\code iPict}, as with the {\code Picture} macro.
Having predefined configurations yields short, convenient code.

\begin{multicols}{2}
\begin{verbatim}
EPicture.a(iBig)("yummy");
EPicture.b(iSmall)("cool");
% you can still modify a.info, b.info

b.sw = a.se + (10,0);

drawObjects(a, b);
\end{verbatim}
\columnbreak
\hspace{1cm}\includegraphics{fig/picture_info.1}
\end{multicols}

\subsection{Stacking Objects}

It is possible to stack objects, much in the style of {\code setboxjoin}
from {\code boxes} library.

\pagebreak

\begin{multicols}{2}
\begin{verbatim}
Picture.a0("yummy");
Picture.a1("cool");
Picture.a2("fool");

setObjectJoin(pa.sw = pb.nw);
joinObjects(scantokens listArray(a)(3));

drawObjects(scantokens listArray(a)(3));
% or drawObjects (a0, a1, a2);
\end{verbatim}
\columnbreak
\hspace{1cm}\includegraphics{fig/object_stack.1}
\end{multicols}

The {\code listArray} macro provides here a shortcut for writing
{\code a0, a1, a2}. This macro is particularly useful for generic
code which does not know beforehand the number of elements to be drawn.
Having to write the {\code scantokens} keyword is admittedly a nuisance, but
this is required.

\subsection{The Group Macro}

It is possible to group objects in \metauml. This feature is the cornerstone
of \metauml, allowing for the easy development of complex objects, such as
composite stats in state machine diagrams.

Similarly to the macro {\code Picture}, the structure {\code GroupInfo}
is used for specifying group properties; its default instantiation is
{\code iGroup}. Furthermore, the macro {\code EGroup} explicitely sets the
layout information.

Here is an example:

\begin{multicols}{2}
\begin{verbatim}
iGroup.left:=20;
iGroup.right:=15;
iGroup.boxed:=1;
iPicture.boxed:=1;

Picture.a("yummy");
Picture.b("cool");
Picture.c("fool");

b.nw = a.nw + (20,20);  % A
c.nw = a.nw + (15, 40); % B

Group.g(a, b, c);
g.nw = (10,10); % C

drawObject(g);
\end{verbatim}
\columnbreak
\hspace{1cm}\includegraphics{fig/group.1}
\end{multicols}

Note that after some objects are grouped, they can all be drawn
by invoking the {\code drawObject} macro solely on the group that aggregates them.
Another important remark is that it is necessary only to set the relative
positioning of objects within a group (line A and B); afterward, one can
simply ``move'' the group to a given position (line C), and all the contained
objects will move along.

\subsection{The PictureStack Macro}

The {\code PictureStack} macro is a syntactic sugar for a set of pictures,
stacked according to predefined equations and grouped together.

\begin{multicols}{2}
\begin{verbatim}
iStack.boxed := 1;
iStack.iPict.boxed := 1;
PictureStack.myStack("foo",
  "bar: int" infont "tyxtt",
  "nicely-centered" infont defaultfont,
  "nice")("vcenter");

drawObject(myStack);
\end{verbatim}
\columnbreak
\hspace{1cm}\includegraphics{fig/picture_stack.1}
\end{multicols}

Note the last parameter of the macro {\code PictureStack}, here {\code vcenter}.
It is used to generate appropriate equations based on a descriptive name.
The spacing between individual picture objects is set by the field
{\code iStack.spacing}. Currently, the following alignment names are
defined: {\code vleft}, {\code vright}, {\code vcenter},
{\code vleftbase}, {\code vrightbase}, {\code vcenterbase}. All these
names refer to vertical alignment (the prefix ``{\code v}''); alignment can
be at left, right or centered. The variants having the suffix ``{\code base}'' align
the pictures so that {\code iStack.spacing} refer to the distance between the
bottom lines of the pictures. The unsuffixed variants use {\code iStack.spacing} as
the distance between one's bottom line and the next's top line.

The ``{\code base}'' alignment is particularly useful for stacking text, since it
offers better visual appearance when {\code iPict.ignoreNegativeBase} is set to {\code 1}.

\section{Components Design}

Each MetaUML component (e.g. {\code Picture}, {\code PictureStack}, {\code Class}) is
designed according to an established pattern. This section gives more insight
on this.

In order to draw a component, one must know the following information:
\begin{itemize}
\item what to draw, or what are the elements of a component.
\item how to draw, or how are the elements positioned in relation to each other within the component
\item where to draw
\end{itemize}

For example, in order to draw a picture object we must know, respectively:
\begin{itemize}
\item what is the text or the native picture that needs to be drawn
\item what are the margins that should be left around the contents
\item where is the picture to be drawn
\end{itemize}

Why do we bother with these questions? Why don't we just simply draw the picture
component as soon as it was created and get it over with?
That is, why doesn't the following code just work?

\begin{verbatim}
Picture.pict("foo");
\end{verbatim}

Well, although we have the answer to question 1 (what to draw),
we still need to have question 3 answered. The code below becomes thus a
necessity (actually, you are not forced to specify the positioning of an object,
because its draw method positions it to {\code (0,0)} by default):

\begin{verbatim}
% question 1: what to draw
Picture.pict("foo");

% question 3: where to draw
pict.nw = (10,10);

% now we can draw
drawObject(pict);
\end{verbatim}

How about question 2, how to draw? By default, this problem is addressed behind the
scenes by the component. This means, for the Picture object, that a native picture is created
from the given string, and around that picture certain margins are placed, by means of MetaPost equations.
(The margins come in handy when one wants to quickly place Picture objects near others,
so that the result doesn't look too cluttered.)
If these equations were defined within the Picture constructor, then an
usability problem would have appeared, because it wouldn't have been possible to modify the margins,
as in the code below:

\begin{verbatim}
% question 1: what to draw
Picture.pict("foo");

% question 2: how to draw
pict.info.left := 10;
pict.info.boxed := 1;

% question 3: where to draw
pict.nw = (0,0);

% now we can draw
drawObject(pict);
\end{verbatim}

To allow for this type of code, the equations that define the layout of the {\code Picture} object (here, what the margins are)
must be defined somewhere after the constructor. This is done by a macro called {\code Picture\_layout}.
This macro defines all the equations which link the ``what to draw'' information to the ``how to draw''
information (which in our case is taken from the {\code info} member, a copy of {\code iPict}).
Nevertheless, notice that {\code Picture\_layouts} is not explicitly invoked. To the user's
great relief, this is taken care of automatically within the {\code Picture\_draw} macro.

There are times however, when explicitly invoking a macro like {\code Picture\_layout}
becomes a necessity. This is because, by contract, it is only after the {\code layout}
macro is invoked that the final dimensions (width, height) of an object are
definitely and permanently known. Imagine that we have a component whose job is to
surround in a red-filled rectangle some other objects. This component
needs to know what the dimensions of the contained objects are, in order to be able to set
its own dimensions. At drawing time, the contained objects must not have been drawn already,
because the red rectangle of the container would overwrite them.
Therefore, the whole pseudo-code would be:
\begin{verbatim}
Create objects o1, o2, ... ok;
Create container c(o1, o2, ..., ok);
Optional: modify info-s for o1, o2, ... ok;
Optional: modify info for c;

layout c, requiring layout of o1, o2, ... ok;
establish where to draw c;
draw red rectangle defined by c;
draw components o1, o2, ...ok within c
\end{verbatim}

Note that an object mustn't be laid out more than once, because otherwise
inconsistent or superfluous equations would arise. To enforce this, by contract,
any object must keep record of whether its layout method has already been invoked,
and if the answer is affirmative, subsequent invocations of the layout macro would
do nothing. It is very important to mention that after the {\code layout} macro is
invoked over an object, modifying the {\code info} member of that object has
no subsequent effect, since the layout equations are declared and interpreted only once.

\subsection{Notes on the Implementation of Links}

\metauml\ considers edges in diagram graphs as links. A link is composed of a path and the
heads (possible none, one or two). For example, an association has no heads, and one must simply
draw along the path with a solid pen. An unidirectional aggregation has a solid path and two
heads: one is an arrow and the other is a diamond. So the template algorithm for drawing a link
is:

\begin{verbatim}
0. Reserve space for heads
1. Draw the path (except for the heads)
2. Draw head 1
3. Draw head 2
\end{verbatim}

Each of the UML link types define how the drawing should be done, in each of the
cases (1, 2 and 3). Consider the link type of unidirectional composition.
Its ``class'' is declared as:

\begin{verbatim}
vardef CompositionUniInfo@# =
  LinkInfo@#;

  @#widthA      = defaultRelationHeadWidth;
  @#heightA     = defaultRelationHeadHeight;
  @#drawMethodA = "drawArrow";

  @#widthB      = defaultRelationHeadWidth;
  @#heightB     = defaultRelationHeadHeight;
  @#drawMethodB = "drawDiamondBlack";

  @#drawMethod = "drawLine";
enddef;
\end{verbatim}

Using this definition, the actual description is created like this:

\begin{verbatim}
CompositionUniInfo.compositionUni;
\end{verbatim}

As shown previously, is is the macro {\code link} which
performs the actual drawing, using the link description information
which is given as parameter (generally called {\code iLink}).
For example, we can use:

\begin{verbatim}
link(aggregationUni)((0,0)--(40,0));
\end{verbatim}

%\begin{figure}
%\centering
%\includegraphics{fig/how-links-work.1}
%\caption{An example of a picture stack.}
%\label{fig:hlw}
%\end{figure}

Let us see now the inner workings of macro {\code link}. Its definition is:

\begin{verbatim}
vardef link(text iLink)(expr myPath)=
   LinkStructure.ls(myPath,
                    iLink.widthA, iLink.widthB);
   drawLinkStructure(ls)(iLink);
enddef;
\end{verbatim}

\begin{figure}
\centering
\begin{tabular}{l|l}
$AB$ & the path specified by the user \\
$|AA'|$ & {\code iLink.widthA}\\
$|BB'|$ & {\code iLink.widthB}
\end{tabular}
\includegraphics{fig/how-links-work.2}
\caption{Details on how a link is drawn by \metauml.}
\label{fig:hlw2}
\end{figure}

First, space is reserved for heads, by ``shortening'' the given path {\code myPath}
by {\code iLink.widthA} at the beginning and by {\code iLink.widthB} at the end.
After that, the shortened path is drawn with the ``method''
given by {\code iLink.drawMethod} and the heads with the ``methods''
{\code iLink.drawMethodA} and {\code iLink.drawMethodB},
respectively (figure \ref{fig:hlw2}).

\subsection{Object Definitions: Easier {\code generic\_declare}}

In MetaPost, if somebody wants to define something resembling a class in an object-oriented language,
named, say, {\code Person}, he would do something like this:

\begin{verbatim}
vardef Person@#(expr _name, _age)=
  % @# prefix can be seen as `this` pointer
  string @#name;
  numeric @#age;

  @#name := _name;
  @#age := _age;
enddef;
\end{verbatim}

This allows for the creation of instances (or objects) of class {\code Person} by using
declarations like:

\begin{verbatim}
Person.personA;
Person.personB;
\end{verbatim}

 However, if one also wants to able able to create indexed arrays of persons, such as
{\code Person.student0}, {\code Person.student1} etc., the definition of class
{\code Person} must read:

\begin{verbatim}
vardef Person@#(expr _name, _age)=
  _n_ := str @#;
  generic_declare(string) _n.name;
  generic_declare(numeric) _n.age;

  @#name := _name;
  @#age := _age;
enddef;
\end{verbatim}

This construction is rather inelegant. MetaUML offers alternative macros to achieve
the same effect, uncluttering the code by removing the need for the unaesthetic {\code \_n\_} and
{\code \_n}.

\begin{verbatim}
vardef Person@#(expr _name, _age)=
  attributes(@#);
  var(string) name;
  var(numeric) age;

  @#name := _name;
  @#age := _age;
enddef;
\end{verbatim}

\section{Customization in MetaUML: Examples}
\label{section:customization}

We have seen that in MetaUML the ``how to draw'' information is memorized into the so-called
``{\code Info}'' structures. For example, the default way in which a {\code Picture} object is
to be drawn is recorded into an instance of {\code PictureInfo}, named {\code iPict}. In this section we
present a case study involving the customization of {\code Class} objects. The customization of
any other \metauml\ objects works similarly. Here we cannot possibly present all the customization
options for all kinds of \metauml\ objects: this would take too long. Nevertheless, an interested reader can refer
to the top of the appropriate \metauml\ library file, where {\code Info} structures are defined.
For example, class diagram related definitions are in {\code metauml\_class.mp}, activity diagram
definitions are in {\code metauml\_activity.mp} etc.

\subsection{Global settings}

Let us assume that we do not particularly like the default foreground color of all classes, and wish
to change it so something yellowish. In this scenario, one would most likely want to change
the appropriate field in {\code iClass}:

\begin{verbatim}
iClass.foreColor := (.9, .9, 0);
\end{verbatim}

After this, we can obtain the following result:

\begin{multicols}{2}
\begin{verbatim}
Class.A("A")()();
Class.B("B")()();
Class.C("C")()();

B.w = A.e + (20,0);
C.n = .5[A.se, B.sw] + (0, -10);

drawObjects(A, B, C);
\end{verbatim}
\columnbreak
\hspace{1cm}\includegraphics{fig/class_customization.1}
\end{multicols}

\subsection{Individual settings}

When one wants to make modifications to the settings of one particular
{\code Class} objects, another strategy is more appropriate. How about having class
{\code C} stand out with a light blue foreground color, a bigger font size for the class name and a blue border?

\pagebreak
\begin{multicols}{2}
\begin{verbatim}
iPict.foreColor := (.9, .9, 0);

Class.A("A")()();
Class.B("B")()();
Class.C("C")()();
C.info.foreColor := (.9, .7, .7);
C.info.borderColor := green;
C.info.iName.iFont.scale := 2;

% positioning code ommited
drawObjects(A, B, C);
\end{verbatim}
\columnbreak
\hspace{1cm}\includegraphics{fig/class_customization.2}
\end{multicols}

As an aside, note that for each {\code Class} object its {\code info} member is created as
a copy of {\code iClass}: the actual drawing is performed using this copied
information. Because of that, one can modify the {\code info} member after the object
has been created and still get the desired results.

Another thing worth mentioning is that the {\code ClassInfo} structure contains
the {\code iName} member, which is an instance of {\code PictureInfo}. In our example we
do not want to modify the spacings around the {\code Picture} object,
but the characteristics of the font its contents is typeset into. To do that,
we modify the {\code iName.iFont} member, which by default is a copy of {\code iFont}
(an instance of {\code FontInfo}, defined in {\code util\_picture.mp}).
If, for example, we want to change the font the class name is rendered into, we would set
the attribute {\code iName.iFont.name} to a string representing a font name
on our system (as used with the MetaPost {\code infont} operator).


\subsection{Predefined settings}

This usage scenario is perhaps more interesting. Suppose that we have two
types of classes which we want to draw differently. Making the setting adjustments
for each individual class object would soon become a nuisance. \metauml's solution consists in the
ability of using predefined ``how to draw'' {\code Info} objects. Let us create such objects:

\begin{verbatim}
ClassInfoCopy.iHome(iClass);
iHome.foreColor := (0, .9, .9);

ClassInfo.iRemote;
iRemote.foreColor := (.9, .9, 0);
iRemote.borderColor := green;
\end{verbatim}

Object {\code iHome} is a copy of {\code iClass} (as it might have been set at
the time of the macro call). Object {\code iRemote} is created just as {\code iClass}
is originally created. We can now use these {\code Info} objects to easily set the
``how to draw'' information for classes. The result is depicted below,
please note the ``{\code E}'' prefix in {\code EClass}:

\begin{multicols}{2}
\begin{verbatim}
EClass.A(iHome)("UserHome")()();
EClass.B(iRemote)("UserRemote")()();
EClass.C(iHome)("CartHome")()();
EClass.D(iRemote)("CartRemote")()();
\end{verbatim}
\columnbreak
\hspace{1cm}\includegraphics{fig/class_customization.3}
\end{multicols}

\subsection{Extreme customization}

When another font (or font size) is used, one may also want to modify the spacings between the
attributes' and methods' baselines. Figure below is the result of the
(unlikely) code:

\begin{multicols}{2}
\begin{verbatim}
Class.A("Foo")
  ("a: int", "b: int")
  ("foo()", "bar()", "gar()");

A.info.iName.iFont.name := metauml_defaultFontBold;
A.info.iName.iFont.scale := 1.2;

A.info.iAttributeStack.iPict.iFont.scale := 0.8;
A.info.iAttributeStack.top := 10;
A.info.iAttributeStack.spacing := 11;

A.info.iMethodStack.iPict.iFont.scale := 2;
A.info.iMethodStack.spacing := 17;
A.info.iMethodStack.bottom := 10;

drawObject(A);
\end{verbatim}
\columnbreak
\hspace{4cm}\includegraphics{fig/class_customization.4}
\end{multicols}

\begin{verbatim}
\end{verbatim}

Both {\code iAttributeStack} and {\code iMethodStack} are instances of
{\code PictureStackInfo}, which is used to control the display of {\code PictureStack} objects.
%We can also customize the size and colors of the ``locks'' by setting {\code A.info.iLock}.

As font names, you can choose from the globally defined {\code metauml\_defaultFont}, {\code metauml\_defaultFontOblique}, {\code metauml\_defaultFontBold}, {\code metauml\_defaultFontBoldOblique}, or any other name of a font that is available on your system.

%\theendnotes

\bibliographystyle{apalike}

\begin{thebibliography}{1}
\bibitem[Roegel, 2002]{metaobj}
Roegel, D. (2002).
\newblock {The METAOBJ tutorial and reference manual}.
\newblock Available from {\code www.loria.fr/~roegel/TeX/momanual.pdf}.

\bibitem[Knuth, 1986]{knuth}
Knuth, D.~E. (1986).
\newblock {\em The {\TeX{}}book}.
\newblock Addison-Wesley Publishing Company.

\bibitem[Lamport, 1994]{lamport}
Lamport, L. (1994).
\newblock {\em {\LaTeX} a Document Preparation System}.
\newblock Addison-Wesley Publishing Company, 2nd edition.

%\bibitem[Gheorghies, 2005]{metaumlman}
%Gheorghies, O. (2005).
%\newblock {MetaUML: Tutorial, Reference and Test Suite}.
%\newblock Available from {\code http://metauml.sourceforge.net}.

\bibitem[Hobby, 1992]{metapost}
Hobby, J. (1992)
\newblock {A User's Manual for MetaPost}.
\newblock Available from {\code http://www.tug.org/tutorials/mp/}.

\bibitem[Gjelstad, 2001]{umlsty}
Gjelstad, E. (2001).
\newblock {uml.sty 0.09.09}.
\newblock Available from {\code http://heim.ifi.uio.no/\~{ }ellefg/uml.sty/}.

\bibitem[Diamantini, 1998]{pstumlsty}
Diamantini, M. (1998).
\newblock {Interface utilisateur du package pst-uml}.
\newblock Available from {\code http://perce.de/LaTeX/pst-uml/}.

\bibitem[Palmer, 1999]{umldoc}
Palmer, D. (1999).
\newblock {The umldoc UML Documentation Package}.
\newblock Available from {\code http://www.charvolant.org/\~{ }elements/}.

\bibitem[OMG, 2003]{XMI}
Object Management Group (2003).
\newblock {XML Metadata Interchange (XMI) Specification}.
\newblock Available from {\code http://www.omg.org/}.
\end{thebibliography}

\pagebreak
\pagebreak
\pagebreak
%    Part of the MetaUML manual
%    Copyright (c) 2005 Ovidiu Gheorghies
%
%    Permission is granted to copy, distribute and/or modify this document
%    under the terms of the GNU Free Documentation License, Version 1.2
%    or any later version published by the Free Software Foundation;
%    with no Invariant Sections, no Front-Cover Texts, and no Back-Cover Texts.
%    A copy of the license is included in the section entitled "GNU
%    Free Documentation License".

\newcommand{\metaumltest}[2]{Test #2 --- \\ \includegraphics{fig/test_#1.#2} \\ }

\section{Test suite}

\subsection{Low-level}
  \metaumltest{lowlevel}{1}
  \metaumltest{lowlevel}{2}

\subsection{Fonts}
  \metaumltest{font}{1}
  \metaumltest{font}{2}
  \metaumltest{font}{3}

\subsection{Cliparts}
  Locks ---\\
  \includegraphics{fig/cliparts.1}

\subsection{Util library}
  \subsubsection{Picture tests}
    \metaumltest{picture}{1}
    \metaumltest{picture}{2}
    \metaumltest{picture}{3}
    \metaumltest{picture}{4}
    \metaumltest{picture}{5}
    \metaumltest{picture}{6}

  \subsubsection{Group tests}
    \metaumltest{group}{1}
    \metaumltest{group}{2}

  \subsubsection{PictureStack tests}
    \metaumltest{picture_stack}{1}
    \metaumltest{picture_stack}{2}
    \metaumltest{picture_stack}{3}

  \subsubsection{Positioning tests}
    \metaumltest{positioning}{1}
    \metaumltest{positioning}{2}
    \metaumltest{positioning}{3}
    \metaumltest{positioning}{4}
    \metaumltest{positioning}{5}
    \metaumltest{positioning}{6}

\subsection{Class diagram}
  \subsubsection{Class tests}
    \metaumltest{class}{1}
    \metaumltest{class}{2}
    \metaumltest{class}{3}
    \metaumltest{class}{4}
    \metaumltest{class}{5}
    \metaumltest{class}{6}
    \metaumltest{class}{7}
    \metaumltest{class}{8}
    \metaumltest{class}{9}
    \metaumltest{class}{10}
    \metaumltest{class}{11}

  \subsubsection{Class template tests}
    \metaumltest{class_templates}{1}
    \metaumltest{class_templates}{2}
    \metaumltest{class_templates}{3}

  \subsubsection{Qualified Association tests}
    \metaumltest{class_qual_assoc}{1}
    \metaumltest{class_qual_assoc}{2}

\subsection{Package diagram}
\subsubsection{Package tests}
    \metaumltest{package}{1}
    \metaumltest{package}{2}

\subsection{Paths}
  \metaumltest{paths}{1}

\subsection{Behavioral diagrams}
  \subsubsection{Activity tests}
    \metaumltest{activity}{1}
    \metaumltest{activity}{2}

  \subsubsection{State Machine tests}
    \metaumltest{state}{1}
    \metaumltest{state}{2}
    \metaumltest{state}{3}
    \metaumltest{state}{4}
    \metaumltest{state}{5}

  \subsubsection{Usecase tests}
    \metaumltest{usecase}{1}
    \metaumltest{usecase}{2}
    \metaumltest{usecase}{3}
    \metaumltest{usecase}{4}
    \metaumltest{usecase}{5}
    \metaumltest{usecase}{6}
    \metaumltest{usecase}{7}
    \metaumltest{usecase}{8}
    \metaumltest{usecase}{9}

\subsection{Miscelaneous}
  \subsubsection{Notes}
    \metaumltest{note}{1}
    \metaumltest{note}{2}
  \subsubsection{Objects (Class Instances)}
    \metaumltest{instance}{1}

\subsection{User requests}
  Test 1 --- \\ \includegraphics[scale=.2]{fig/test_lars_issues.1} \\
  \metaumltest{lars_issues}{2}

\subsection{Skins}
    \metaumltest{skins}{1}
    \metaumltest{skins_global_defaults}{1}


\pagebreak
\section{GNU Free Documentation License}
 \begin{center}

       Version 1.2, November 2002


 Copyright \copyright 2000,2001,2002  Free Software Foundation, Inc.
 
 \bigskip
 
     51 Franklin St, Fifth Floor, Boston, MA  02110-1301  USA
  
 \bigskip
 
 Everyone is permitted to copy and distribute verbatim copies
 of this license document, but changing it is not allowed.
\end{center}


\begin{center}
{\bf\large Preamble}
\end{center}

The purpose of this License is to make a manual, textbook, or other
functional and useful document "free" in the sense of freedom: to
assure everyone the effective freedom to copy and redistribute it,
with or without modifying it, either commercially or noncommercially.
Secondarily, this License preserves for the author and publisher a way
to get credit for their work, while not being considered responsible
for modifications made by others.

This License is a kind of "copyleft", which means that derivative
works of the document must themselves be free in the same sense.  It
complements the GNU General Public License, which is a copyleft
license designed for free software.

We have designed this License in order to use it for manuals for free
software, because free software needs free documentation: a free
program should come with manuals providing the same freedoms that the
software does.  But this License is not limited to software manuals;
it can be used for any textual work, regardless of subject matter or
whether it is published as a printed book.  We recommend this License
principally for works whose purpose is instruction or reference.


\begin{center}
{\bf 1. APPLICABILITY AND DEFINITIONS}
%\addcontentsline{toc}{section}{1. APPLICABILITY AND DEFINITIONS}
\end{center}

This License applies to any manual or other work, in any medium, that
contains a notice placed by the copyright holder saying it can be
distributed under the terms of this License.  Such a notice grants a
world-wide, royalty-free license, unlimited in duration, to use that
work under the conditions stated herein.  The \textbf{"Document"}, below,
refers to any such manual or work.  Any member of the public is a
licensee, and is addressed as \textbf{"you"}.  You accept the license if you
copy, modify or distribute the work in a way requiring permission
under copyright law.

A \textbf{"Modified Version"} of the Document means any work containing the
Document or a portion of it, either copied verbatim, or with
modifications and/or translated into another language.

A \textbf{"Secondary Section"} is a named appendix or a front-matter section of
the Document that deals exclusively with the relationship of the
publishers or authors of the Document to the Document's overall subject
(or to related matters) and contains nothing that could fall directly
within that overall subject.  (Thus, if the Document is in part a
textbook of mathematics, a Secondary Section may not explain any
mathematics.)  The relationship could be a matter of historical
connection with the subject or with related matters, or of legal,
commercial, philosophical, ethical or political position regarding
them.

The \textbf{"Invariant Sections"} are certain Secondary Sections whose titles
are designated, as being those of Invariant Sections, in the notice
that says that the Document is released under this License.  If a
section does not fit the above definition of Secondary then it is not
allowed to be designated as Invariant.  The Document may contain zero
Invariant Sections.  If the Document does not identify any Invariant
Sections then there are none.

The \textbf{"Cover Texts"} are certain short passages of text that are listed,
as Front-Cover Texts or Back-Cover Texts, in the notice that says that
the Document is released under this License.  A Front-Cover Text may
be at most 5 words, and a Back-Cover Text may be at most 25 words.

A \textbf{"Transparent"} copy of the Document means a machine-readable copy,
represented in a format whose specification is available to the
general public, that is suitable for revising the document
straightforwardly with generic text editors or (for images composed of
pixels) generic paint programs or (for drawings) some widely available
drawing editor, and that is suitable for input to text formatters or
for automatic translation to a variety of formats suitable for input
to text formatters.  A copy made in an otherwise Transparent file
format whose markup, or absence of markup, has been arranged to thwart
or discourage subsequent modification by readers is not Transparent.
An image format is not Transparent if used for any substantial amount
of text.  A copy that is not "Transparent" is called \textbf{"Opaque"}.

Examples of suitable formats for Transparent copies include plain
ASCII without markup, Texinfo input format, LaTeX input format, SGML
or XML using a publicly available DTD, and standard-conforming simple
HTML, PostScript or PDF designed for human modification.  Examples of
transparent image formats include PNG, XCF and JPG.  Opaque formats
include proprietary formats that can be read and edited only by
proprietary word processors, SGML or XML for which the DTD and/or
processing tools are not generally available, and the
machine-generated HTML, PostScript or PDF produced by some word
processors for output purposes only.

The \textbf{"Title Page"} means, for a printed book, the title page itself,
plus such following pages as are needed to hold, legibly, the material
this License requires to appear in the title page.  For works in
formats which do not have any title page as such, "Title Page" means
the text near the most prominent appearance of the work's title,
preceding the beginning of the body of the text.

A section \textbf{"Entitled XYZ"} means a named subunit of the Document whose
title either is precisely XYZ or contains XYZ in parentheses following
text that translates XYZ in another language.  (Here XYZ stands for a
specific section name mentioned below, such as \textbf{"Acknowledgements"},
\textbf{"Dedications"}, \textbf{"Endorsements"}, or \textbf{"History"}.)  
To \textbf{"Preserve the Title"}
of such a section when you modify the Document means that it remains a
section "Entitled XYZ" according to this definition.

The Document may include Warranty Disclaimers next to the notice which
states that this License applies to the Document.  These Warranty
Disclaimers are considered to be included by reference in this
License, but only as regards disclaiming warranties: any other
implication that these Warranty Disclaimers may have is void and has
no effect on the meaning of this License.


\begin{center}
{\bf 2. VERBATIM COPYING}
%\addcontentsline{toc}{section}{2. VERBATIM COPYING}
\end{center}

You may copy and distribute the Document in any medium, either
commercially or noncommercially, provided that this License, the
copyright notices, and the license notice saying this License applies
to the Document are reproduced in all copies, and that you add no other
conditions whatsoever to those of this License.  You may not use
technical measures to obstruct or control the reading or further
copying of the copies you make or distribute.  However, you may accept
compensation in exchange for copies.  If you distribute a large enough
number of copies you must also follow the conditions in section 3.

You may also lend copies, under the same conditions stated above, and
you may publicly display copies.


\begin{center}
{\bf 3. COPYING IN QUANTITY}
%\addcontentsline{toc}{section}{3. COPYING IN QUANTITY}
\end{center}


If you publish printed copies (or copies in media that commonly have
printed covers) of the Document, numbering more than 100, and the
Document's license notice requires Cover Texts, you must enclose the
copies in covers that carry, clearly and legibly, all these Cover
Texts: Front-Cover Texts on the front cover, and Back-Cover Texts on
the back cover.  Both covers must also clearly and legibly identify
you as the publisher of these copies.  The front cover must present
the full title with all words of the title equally prominent and
visible.  You may add other material on the covers in addition.
Copying with changes limited to the covers, as long as they preserve
the title of the Document and satisfy these conditions, can be treated
as verbatim copying in other respects.

If the required texts for either cover are too voluminous to fit
legibly, you should put the first ones listed (as many as fit
reasonably) on the actual cover, and continue the rest onto adjacent
pages.

If you publish or distribute Opaque copies of the Document numbering
more than 100, you must either include a machine-readable Transparent
copy along with each Opaque copy, or state in or with each Opaque copy
a computer-network location from which the general network-using
public has access to download using public-standard network protocols
a complete Transparent copy of the Document, free of added material.
If you use the latter option, you must take reasonably prudent steps,
when you begin distribution of Opaque copies in quantity, to ensure
that this Transparent copy will remain thus accessible at the stated
location until at least one year after the last time you distribute an
Opaque copy (directly or through your agents or retailers) of that
edition to the public.

It is requested, but not required, that you contact the authors of the
Document well before redistributing any large number of copies, to give
them a chance to provide you with an updated version of the Document.


\begin{center}
{\bf 4. MODIFICATIONS}
%\addcontentsline{toc}{section}{4. MODIFICATIONS}
\end{center}

You may copy and distribute a Modified Version of the Document under
the conditions of sections 2 and 3 above, provided that you release
the Modified Version under precisely this License, with the Modified
Version filling the role of the Document, thus licensing distribution
and modification of the Modified Version to whoever possesses a copy
of it.  In addition, you must do these things in the Modified Version:

\begin{itemize}
\item[A.] 
   Use in the Title Page (and on the covers, if any) a title distinct
   from that of the Document, and from those of previous versions
   (which should, if there were any, be listed in the History section
   of the Document).  You may use the same title as a previous version
   if the original publisher of that version gives permission.
   
\item[B.]
   List on the Title Page, as authors, one or more persons or entities
   responsible for authorship of the modifications in the Modified
   Version, together with at least five of the principal authors of the
   Document (all of its principal authors, if it has fewer than five),
   unless they release you from this requirement.
   
\item[C.]
   State on the Title page the name of the publisher of the
   Modified Version, as the publisher.
   
\item[D.]
   Preserve all the copyright notices of the Document.
   
\item[E.]
   Add an appropriate copyright notice for your modifications
   adjacent to the other copyright notices.
   
\item[F.]
   Include, immediately after the copyright notices, a license notice
   giving the public permission to use the Modified Version under the
   terms of this License, in the form shown in the Addendum below.
   
\item[G.]
   Preserve in that license notice the full lists of Invariant Sections
   and required Cover Texts given in the Document's license notice.
   
\item[H.]
   Include an unaltered copy of this License.
   
\item[I.]
   Preserve the section Entitled "History", Preserve its Title, and add
   to it an item stating at least the title, year, new authors, and
   publisher of the Modified Version as given on the Title Page.  If
   there is no section Entitled "History" in the Document, create one
   stating the title, year, authors, and publisher of the Document as
   given on its Title Page, then add an item describing the Modified
   Version as stated in the previous sentence.
   
\item[J.]
   Preserve the network location, if any, given in the Document for
   public access to a Transparent copy of the Document, and likewise
   the network locations given in the Document for previous versions
   it was based on.  These may be placed in the "History" section.
   You may omit a network location for a work that was published at
   least four years before the Document itself, or if the original
   publisher of the version it refers to gives permission.
   
\item[K.]
   For any section Entitled "Acknowledgements" or "Dedications",
   Preserve the Title of the section, and preserve in the section all
   the substance and tone of each of the contributor acknowledgements
   and/or dedications given therein.
   
\item[L.]
   Preserve all the Invariant Sections of the Document,
   unaltered in their text and in their titles.  Section numbers
   or the equivalent are not considered part of the section titles.
   
\item[M.]
   Delete any section Entitled "Endorsements".  Such a section
   may not be included in the Modified Version.
   
\item[N.]
   Do not retitle any existing section to be Entitled "Endorsements"
   or to conflict in title with any Invariant Section.
   
\item[O.]
   Preserve any Warranty Disclaimers.
\end{itemize}

If the Modified Version includes new front-matter sections or
appendices that qualify as Secondary Sections and contain no material
copied from the Document, you may at your option designate some or all
of these sections as invariant.  To do this, add their titles to the
list of Invariant Sections in the Modified Version's license notice.
These titles must be distinct from any other section titles.

You may add a section Entitled "Endorsements", provided it contains
nothing but endorsements of your Modified Version by various
parties--for example, statements of peer review or that the text has
been approved by an organization as the authoritative definition of a
standard.

You may add a passage of up to five words as a Front-Cover Text, and a
passage of up to 25 words as a Back-Cover Text, to the end of the list
of Cover Texts in the Modified Version.  Only one passage of
Front-Cover Text and one of Back-Cover Text may be added by (or
through arrangements made by) any one entity.  If the Document already
includes a cover text for the same cover, previously added by you or
by arrangement made by the same entity you are acting on behalf of,
you may not add another; but you may replace the old one, on explicit
permission from the previous publisher that added the old one.

The author(s) and publisher(s) of the Document do not by this License
give permission to use their names for publicity for or to assert or
imply endorsement of any Modified Version.


\begin{center}
{\bf 5. COMBINING DOCUMENTS}
%\addcontentsline{toc}{section}{5. COMBINING DOCUMENTS}
\end{center}


You may combine the Document with other documents released under this
License, under the terms defined in section 4 above for modified
versions, provided that you include in the combination all of the
Invariant Sections of all of the original documents, unmodified, and
list them all as Invariant Sections of your combined work in its
license notice, and that you preserve all their Warranty Disclaimers.

The combined work need only contain one copy of this License, and
multiple identical Invariant Sections may be replaced with a single
copy.  If there are multiple Invariant Sections with the same name but
different contents, make the title of each such section unique by
adding at the end of it, in parentheses, the name of the original
author or publisher of that section if known, or else a unique number.
Make the same adjustment to the section titles in the list of
Invariant Sections in the license notice of the combined work.

In the combination, you must combine any sections Entitled "History"
in the various original documents, forming one section Entitled
"History"; likewise combine any sections Entitled "Acknowledgements",
and any sections Entitled "Dedications".  You must delete all sections
Entitled "Endorsements".

\begin{center}
{\bf 6. COLLECTIONS OF DOCUMENTS}
%\addcontentsline{toc}{section}{6. COLLECTIONS OF DOCUMENTS}
\end{center}

You may make a collection consisting of the Document and other documents
released under this License, and replace the individual copies of this
License in the various documents with a single copy that is included in
the collection, provided that you follow the rules of this License for
verbatim copying of each of the documents in all other respects.

You may extract a single document from such a collection, and distribute
it individually under this License, provided you insert a copy of this
License into the extracted document, and follow this License in all
other respects regarding verbatim copying of that document.


\begin{center}
{\bf 7. AGGREGATION WITH INDEPENDENT WORKS}
%\addcontentsline{toc}{section}{7. AGGREGATION WITH INDEPENDENT WORKS}
\end{center}


A compilation of the Document or its derivatives with other separate
and independent documents or works, in or on a volume of a storage or
distribution medium, is called an "aggregate" if the copyright
resulting from the compilation is not used to limit the legal rights
of the compilation's users beyond what the individual works permit.
When the Document is included in an aggregate, this License does not
apply to the other works in the aggregate which are not themselves
derivative works of the Document.

If the Cover Text requirement of section 3 is applicable to these
copies of the Document, then if the Document is less than one half of
the entire aggregate, the Document's Cover Texts may be placed on
covers that bracket the Document within the aggregate, or the
electronic equivalent of covers if the Document is in electronic form.
Otherwise they must appear on printed covers that bracket the whole
aggregate.


\begin{center}
{\bf 8. TRANSLATION}
%\addcontentsline{toc}{section}{8. TRANSLATION}
\end{center}


Translation is considered a kind of modification, so you may
distribute translations of the Document under the terms of section 4.
Replacing Invariant Sections with translations requires special
permission from their copyright holders, but you may include
translations of some or all Invariant Sections in addition to the
original versions of these Invariant Sections.  You may include a
translation of this License, and all the license notices in the
Document, and any Warranty Disclaimers, provided that you also include
the original English version of this License and the original versions
of those notices and disclaimers.  In case of a disagreement between
the translation and the original version of this License or a notice
or disclaimer, the original version will prevail.

If a section in the Document is Entitled "Acknowledgements",
"Dedications", or "History", the requirement (section 4) to Preserve
its Title (section 1) will typically require changing the actual
title.


\begin{center}
{\bf 9. TERMINATION}
%\addcontentsline{toc}{section}{9. TERMINATION}
\end{center}


You may not copy, modify, sublicense, or distribute the Document except
as expressly provided for under this License.  Any other attempt to
copy, modify, sublicense or distribute the Document is void, and will
automatically terminate your rights under this License.  However,
parties who have received copies, or rights, from you under this
License will not have their licenses terminated so long as such
parties remain in full compliance.


\begin{center}
{\bf 10. FUTURE REVISIONS OF THIS LICENSE}
%\addcontentsline{toc}{section}{10. FUTURE REVISIONS OF THIS LICENSE}
\end{center}


The Free Software Foundation may publish new, revised versions
of the GNU Free Documentation License from time to time.  Such new
versions will be similar in spirit to the present version, but may
differ in detail to address new problems or concerns.  See
http://www.gnu.org/copyleft/.

Each version of the License is given a distinguishing version number.
If the Document specifies that a particular numbered version of this
License "or any later version" applies to it, you have the option of
following the terms and conditions either of that specified version or
of any later version that has been published (not as a draft) by the
Free Software Foundation.  If the Document does not specify a version
number of this License, you may choose any version ever published (not
as a draft) by the Free Software Foundation.


\begin{center}
{\bf ADDENDUM: How to use this License for your documents}
%\addcontentsline{toc}{section}{ADDENDUM: How to use this License for your documents}
\end{center}

To use this License in a document you have written, include a copy of
the License in the document and put the following copyright and
license notices just after the title page:

\bigskip
\begin{quote}
    Copyright \copyright  YEAR  YOUR NAME.
    Permission is granted to copy, distribute and/or modify this document
    under the terms of the GNU Free Documentation License, Version 1.2
    or any later version published by the Free Software Foundation;
    with no Invariant Sections, no Front-Cover Texts, and no Back-Cover Texts.
    A copy of the license is included in the section entitled "GNU
    Free Documentation License".
\end{quote}
\bigskip
    
If you have Invariant Sections, Front-Cover Texts and Back-Cover Texts,
replace the "with...Texts." line with this:

\bigskip
\begin{quote}
    with the Invariant Sections being LIST THEIR TITLES, with the
    Front-Cover Texts being LIST, and with the Back-Cover Texts being LIST.
\end{quote}
\bigskip
    
If you have Invariant Sections without Cover Texts, or some other
combination of the three, merge those two alternatives to suit the
situation.
 
If your document contains nontrivial examples of program code, we
recommend releasing these examples in parallel under your choice of
free software license, such as the GNU General Public License,
to permit their use in free software.


\end{document}
