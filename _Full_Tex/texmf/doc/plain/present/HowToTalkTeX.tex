%
% HOW TO DO A TALK IN TeX
%
% Author: Matthias Meister
% Version 1.0, August 2010
%
% This file is part of a work distributed under the LaTeX Project Public License
%

\input present % Load presentation specific macros

\vskip20mm plus 5mm minus 5mm
\line{\pagelink{NextPage}{\hfil\titlefont How To Do A Talk In \TeX\hfil}}
\vskip 5mm plus 2mm minus 1mm
\line{\hfil\font\subtitlefont=cmssbx10 at 15pt\subtitlefont One Of Many Solutions\hfil}
\vskip 30mm plus 2mm minus 1mm
\line{\hfil\normalfont Matthias Meister\hfil}\target{title}
\vskip 5mm plus 2mm minus 1mm
\line{\hfil\normalfont Regensburg, August 2010\hfil}
\NewSlide
\pageno=1% Don't want to count the titlepage
\normalfont
\def\LinkBar{} 	% LinkBar is left part of the footline, and currently is empty
%%
%%
%%
\SlideTitle{\hfill Motivation}%
\SlideFoot 	% "Activates" the footline, titlepage had an empty footline
\vglue7mm
\leftskip5mm
\item{$\bullet$}\textcolor{text}{Make it possible to prepare presentations in \TeX}
\vskip3mm
\item{$\bullet$}\textcolor{text}{Without having to learn lots of extra syntax}
\vskip3mm
\item{$\bullet$}\textcolor{text}{By using simple macros that can easily be adapted to one's needs, maybe for each presentation prepared}
\vskip3mm
\item{$\bullet$}\textcolor{text}{Without restricting the possibilities that \TeX\ offers}
\NewSlide
\SlideTitle{\hfil Requirements}
\definecolor{code}{rgb}{0.8,0.0,0.8}
\font\cdfo=cmtt12
\def\cf{\color{code}\cdfo}
\item{+} {\cf miniltx.tex}
\item{+} {\cf color.tex}
\item{+} {\cf color.sty}

\leftskip=0pt
\vskip3mm
These files are available from \weblink{www.ctan.org}{www.ctan.org}. You might already have (some of) them.
\vskip3mm
You need the program {\cf pdftex}, which probably is included in your \TeX-distribution.
\vskip3mm
You also need a pdf-viewer with fullscreen display capabilities, e.g. {\cf xpdf}.
\vskip5pt
\att{Some pdf-viewers do not handle links in a document properly.}
%%
%%
%%
\NewSlide
\SlideTitle{\hfil Basic Examples}
You can type text as usual, inline equations $a^2+b^2=c^2$, displayed equations
$$
\exp(z)=\sum_{n=0}^\infty {z^n\over n!}=\lim_{n\to\infty}\Big(1+{z\over n}\Big)^n
$$
and tables
\vskip0.5\baselineskip
\moveright2cm\vbox{\offinterlineskip
\halign{\vrule width 0pt height12pt depth3pt\kern5pt # &\vrule\hfil\kern5pt #\kern5pt\hfil 
&\vrule\hfil\kern5pt #\kern5pt\hfil\cr
& $x$ & $\Delta x$\cr
\noalign{\hrule}
A & 1.03 & 0.07 \cr
\noalign{\hrule}
B & 2.05 & 0.06 \cr
}}\vskip5pt
So you probably can use \TeX\ in the ordinary fashion; just run {\cf pdftex} on your source instead of {\cf tex}.
\NewSlide
You can include images easily:\vskip20pt plus 3pt minus 10pt
\hbox{\image[width3cm]{Stern.png}\hskip5mm plus 2mm minus 2mm A pixel image (png)}
\vskip10pt\hrule
\hbox{A vector image (pdf)\hskip5mm plus 2mm minus 2mm\image[width3cm]{Sagnac.pdf}}
\NewSlide
%%%
%%%
%%%
\SlideTitle{\hfil Ordinary stuff}
You type text, equations, and so on as usual in \TeX. Keep in mind, though, that the effective paper size is 
rather small (12cm wide, 9cm high in standard configuration).
\vskip20pt plus 5pt minus 10pt
The paper size is small, because you rely on the fullscreen mode of your pdf-viewer to blow the slide up to full 
screen size, thus also automatically enlarging the fonts.
\vskip20pt plus 5pt minus 10pt
Your macros should work (unless there is a collision of names), because what you are doing is creating an ordinary
pdf-file with {\cf pdftex}, for viewing it with a pdf-viewer.
\NewSlide
%%%
%%%
%%%
\SlideTitle{\hfil Fonts}
{\cf present.tex} defines the following fonts:\vskip5pt
{\openup2pt
\halign{#\hfil\kern10pt&#\hfil\kern10pt&#\hfil\cr
Font Command & Purpose & Default Value\cr
{\cf\char'134 titlefont} & presentation title & {\cf cmssbx10 at 20pt}\cr
{\cf\char'134 slidetitlefont} & title of a slide & {\cf cmssbx10}\cr
{\cf\char'134 normalfont} & ordinary text & {\cf cmss12}\cr
{\cf\char'134 linkbarfont} & text in {\cf\char'134 LinkBar} & {\cf cmss8}\cr
{\cf\char'134 it} & italic text & {\cf cmti12}\cr
}}
\vskip10pt
Of course you can define further font commands or redefine the existing ones.
The default fonts are used in this presentation, so you see what they look like.
\NewSlide
%%%
%%%
%%%
\SlideTitle{\hfil Colours}
Colour effects are handled by the {\cf color.sty}-package included via {\cf color.tex}.

The following colours are
defined in {\cf present.tex}:
\vskip5pt
{\cf background}, for the slide background 
\vskip5pt
{\cf text}, for the text, and
\vskip5pt
{\cf attention}, an \att{attention} colour.
\vskip5pt
Colour names can be defined and redefined by
\vskip5pt
{\cf\char'134 definecolor\char'173}{\it NAME\/}{\cf\char'175\char'173 rgb\char'175\char'173}{\it R,G,B\/}%
{\cf\char'175}
\vskip5pt
where {\it NAME} is a colour name to be defined, like {\cf background} or {\cf mycolour}, and {\it R}, 
{\it G}, and {\it B} are
the red, green, and blue values describing the colour; $0\leq\hbox{ {\it R}, {\it G}, {\it B} }\leq 1$.
\NewSlide
%%%
%%%
%%%
Once defined, the colours are employed as follows:
\vskip3pt
{\cf\char'134 pagecolor\char'173 background\char'175} sets the colour of the slide background to {\cf background}.
\vskip3pt
{\cf\char'134 color\char'173 text\char'175} sets the colour of subsequent text to {\cf text}.
\vskip3pt
{\cf\char'134 textcolor\char'173 text\char'175\char'173}{\it Stuff\/}{\cf\char'175} prints {\it Stuff} 
in colour {\cf text}.
\vskip5pt
The colour {\cf attention} is mainly used in the macro {\cf\char'134 att}, which prints 
\att{Stuff} in attention colour by saying
{\cf\char'134 att\char'173}{\it Stuff\/}{\cf \char'175}.
\NewSlide
%%%
%%%
%%%
\SlideTitle{\hfil Paper Size}
The paper size is determined by the parameters {\cf\char'134 pdfpagewidth} and
{\cf\char'134 pdfpageheight}. 
\vskip10pt
If you say {\cf\char'134 StandardAspect}, you get a page 120mm wide and 90mm high.
\vskip10pt
If you say {\cf\char'134 WideAspect}, you get a page 144mm wide and 90mm high.
\NewSlide
%%%
%%%
%%%
\SlideTitle{\hfil Images}
Images can be included with
\vskip5pt
{\cf\char'134 image[}{\it dimensions}{\cf]\char'173}{\it filename}{\cf\char'175}
\vskip5pt
where {\it dimensions} are {\cf height}, {\cf depth}, and {\cf width}, familiar from \TeX.
\vskip5pt
\hbox{%
\image[width3mm]{Stern.png}
\image[width6mm]{Stern.png}
\image[width9mm]{Stern.png}
\image[height9mm]{Stern.png}
\image[height3mm depth6mm]{Stern.png}
\image[height9mm width4mm]{Stern.png}
\image[width5mm]{Sagnac.pdf}\image[width10mm height4mm]{Sagnac.pdf}
\image[width15mm depth 5mm]{Sagnac.pdf}
}
\vskip5pt
If only {\cf width} is given, the image is scaled with the aspect ratio preserved.\target{Images}

Image files need to be in a format that can be handled by the pdf-viewer and by {\cf pdftex}.
E.g. pdf, png, jpg should work.
\NewSlide
%%%
%%%
%%%
\pagecolor{background}
\SlideTitle{\hfil Slide Structure}
Each slide has a headline, a body, and a footline.
\vskip5pt
The headline holds the slide title, which is set by 
\vskip5pt
{\cf\char'134 SlideTitle\char'173}{\it TITLE\/}{\cf\char'175}
\vskip5pt
The footline is defined by 
{\cf\char'134 SlideFoot}, of which several versions are contained in {\cf present.tex}. Uncomment the one you want,
or define further ones.
\vskip5pt
The version used here shows the number of the current slide and the total number of slides in the centre (the latter is
provided by the {\cf\char'134 LP} macro), and the {\cf\char'134 PageBar} on the right.

The {\cf\char'134 PageBar}-symbols \PageBar move to the previous or following page, or move back and forth in the 
page history.
\NewSlide
%%%
%%%
%%%
{The page history is relevant if cross-references are used in the presentation. These will be discussed subsequently.}
\vskip5pt
\phantom{With cross-references or links, also the {\cf\char'134 LinkBar} can be useful. In the {\cf\char'134 SlideFoot}-version
used, it is shown on the left side of the footline, but currently is defined to be empty.}
\vskip5pt
\phantom{The {\cf\char'134 LinkBar} is intended to hold links to various places in the presentation you might want to jump to.}
\vskip5pt
\phantom{Before we turn to cross-references or links: You start a new slide by saying {\cf\char'134 NewSlide}. If instead you
say {\cf\char'134 NewFrame}, it has almost the same effect, only the slide number doesn't get increased. This is useful, 
if a slide is to be shown incrementally.}
\NewFrame
{The page history is relevant if cross-references are used in the presentation. These will be discussed subsequently.}
\vskip5pt
{With cross-references or links, also the {\cf\char'134 LinkBar} can be useful. In the {\cf\char'134 SlideFoot}-version
used, it is shown on the left side of the footline, but currently is defined to be empty.}
\vskip5pt
\phantom{The {\cf\char'134 LinkBar} is intended to hold links to various places in the presentation you might want to jump to.}
\vskip5pt
\phantom{Before we turn to cross-references or links: You start a new slide by saying {\cf\char'134 NewSlide}. If instead you
say {\cf\char'134 NewFrame}, it has almost the same effect, only the slide number doesn't get increased. This is useful, 
if a slide is to be shown incrementally.}
\NewFrame
{The page history is relevant if cross-references are used in the presentation. These will be discussed subsequently.}
\vskip5pt
{With cross-references or links, also the {\cf\char'134 LinkBar} can be useful. In the {\cf\char'134 SlideFoot}-version
used, it is shown on the left side of the footline, but currently is defined to be empty.}
\vskip5pt
{The {\cf\char'134 LinkBar} is intended to hold links to various places in the presentation you might want to jump to.}
\vskip5pt
\phantom{Before we turn to cross-references or links: You start a new slide by saying {\cf\char'134 NewSlide}. If instead you
say {\cf\char'134 NewFrame}, it has almost the same effect, only the slide number doesn't get increased. This is useful, 
if a slide is to be shown incrementally.}
\NewFrame
{The page history is relevant if cross-references are used in the presentation. These will be discussed subsequently.}
\vskip5pt
{With cross-references or links, also the {\cf\char'134 LinkBar} can be useful. In the {\cf\char'134 SlideFoot}-version
used, it is shown on the left side of the footline, but currently is defined to be empty.}
\vskip5pt
{The {\cf\char'134 LinkBar} is intended to hold links to various places in the presentation you might want to jump to.}
\vskip5pt
{Before we turn to cross-references or links: You start a new slide by saying {\cf\char'134 NewSlide}. If instead you
say {\cf\char'134 NewFrame}, it has almost the same effect, only the slide number doesn't get increased. This is useful, 
if a slide is to be shown incrementally.}
\NewSlide
%%%
%%%
%%%
\SlideTitle{\hfil Links}
{\cf\char'134 target\char'173}{\it NAME\/}{\cf\char'175} creates a target named {\it NAME} for a link at the position in 
the presentation where it is used.
\vskip5pt
{\cf\char'134 link\char'173}{\it NAME\/}{\cf\char'175\char'173}{\it Stuff\/}{\cf\char'175}
makes {\it Stuff} a link to the target named {\it NAME}.
\vskip5pt
{\cf\char'134 weblink\char'173}{\it URI\/}{\cf\char'175\char'173}{\it Stuff\/}{\cf\char'175}
makes {\it Stuff} a link to the specified {\it URI}.
\vskip5pt
{\cf\char'134 filelink\char'173}{\it file}{\cf\char'175\char'173}{\it filedest}{\cf\char'175\char'173}{\it Stuff\/}{\cf\char'175}
makes {\it Stuff} a link to destination {\it filedest} in {\it file}.

{\it filedest} for example can be {\cf [}{\it page} {\cf/Fit]}, with {\it page} the page number (starting at 0).
\vskip10pt
A further possibility can be found \link{Further}{here (click)}.
\def\LinkBar{%
\link{Images}{\linkbarfont Images}
\link{Further}{\linkbarfont Area}
\link{title}{\linkbarfont Title}
}
\NewSlide
%%%
%%%
%%%
{\cf\char'134 linkarea\char'173}{\it dest}{\cf\char'175\char'173}{\it rect}{\cf\char'175\char'173}{\it border}{\cf
\char'175\char'173}{\it color}{\cf\char'175}\target{Further}
\vskip5pt
Creates a rectangular area which is a link to target {\it dest}; {\it rect} consists of 
four space-separated numbers for lower left and upper right corner, {\it border} is the border width, and {\it color} is the 
border colour, specified as three space-separated values for red, green, and blue, all between 0 and 1.
\vskip5pt
\linkarea{title}{150 30 180 60}{2}{0.8 0 0}
\image[width 2cm]{Stern.png}
\vskip5pt
Notice also that {\cf\char'134 LinkBar} has been redefined (on the previous slide already)
\linkarea{Images}{10 100 50 120}{2}{0.8 0 1}
\NewSlide
%%%
%%%
%%%
\SlideTitle{\hfil Concluding Remarks}
PDF provides the possibility to include movies, and to launch applications (which could be a movie player) via clickable
elements. The usability for a presentation depends on how to manage the player together with a full-screen display of the
slides. Movie support is not officially included in {\cf present.tex}.
\vskip5pt
If you prepare a pdf-image for a presentation which is to be shown on some different computer, it would be best to
embed fonts used in the image into the image file. For example, if you have a file {\cf image.eps}, you need to
convert it to pdf, in order to use it with {\cf pdftex}. You can embed the fonts by
\vskip5pt
{\cf  ps2pdf -dEPSCrop=true -dPDFA image.eps} 
\vskip5pt
A look at {\cf present.tex} is recommended.
\bye
