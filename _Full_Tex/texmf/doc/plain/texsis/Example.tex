%% file: Example.tex  (TeXsis version 2.17)
% $Revision: 17.0 $  :  $Date: 1994/11/30 21:53:40 $  :  $Author: myers $
%======================================================================*
% This is a sample paper typeset with TeXsis, to give you a quick idea
% of how it's done.  Note: this is just hacked together from an old
% conference proceedings, so it's not a real paper.  -EAM


% This lets your manuscript run under either mTeXsis (from
% mtexsis.tex) or the full TeXsis distribution

\ifx\texsis\undefined \input mtexsis.tex \fi

\texsis         % this turns on TeXsis 


% Saying \draft puts a time-stamp, page number, etc. on the page,
% but you don't want it for the final version of the paper.
%%\draft  


% Document Format:  uncomment one of these lines to select the style
% in which the paper is printed:
%
%\preprint                       % Preprint style
%\nuclproc                       % Nuclear Physics Proceedings style
%\PhysRev                        % Physical Review style


% ---

% some macros used in this paper:

\def\Kb{{\bar K^2 \over \beta_R}}

% ---
% BEGIN:

\titlepage                              % begin title page material
\title
Noncompact nonlinear sigma models 
and numerical quantum gravity
\endtitle
\author
Eric Myers, Bryce DeWitt, Rob Harrington, and Arie Kapulkin
Center for Relativity, Department of Physics\\ 
University of Texas, Austin, Texas 78705 USA
\endauthor
\abstract
Studying the $O(2,1)$ nonlinear sigma model is a useful step toward
determining whether or not a consistent quantum theory of gravity (based
on the Einstein-Hilbert action) exists.  Like gravity, the sigma model
is not perturbatively renormalizable, and corresponding Feynman graphs
in the two theories have the same na\"{\i}ve degrees of divergence.  Both
theories also have a single overall dimensionful coupling constant, and
both have a configuration space which is noncompact and curved. The
sigma model allows one to study the renormalizability properties of such
theories without the added complications of local symmetries.  
\endabstract
\bigskip
\endtitlepage   % will start \doublecolumns for \nuclproc
                                        
% ---
% Now start in on the text of the paper...
% ---

Quantum Field Theory and the theory of General Relativity are,
separately, probably the two most successful physical theories of this
century.  This notwithstanding, nobody has yet been able to bring the
two together into one complete and consistent quantum theory of gravity.
One major impediment to such a theory is that, unlike gauge field
theories, gravity with the Einstein-Hilbert action
$$
S = {1 \over 16\pi G_N} \int d^4x\,  \sqrt{g} R
\EQN 1$$
is not renormalizable, at least not by the usual methods of perturbation
theory. This has lead a number of physicists to adopt the position that
General Relativity is only the low energy limit of some other quantum
theory, such as superstring theory.  An alternative view which one
can adopt, however, is that the failure of perturbation theory in the
case of gravity is not an indication that the theory is inconsistent,
but only that the mathematical tools one has used are inadequate.  To
pursue a quantum theory of gravity in this direction one needs a
nonperturbative method of calculation: the methods of lattice field
theory, which have already been applied to gauge theories, are
immediately suggested.  One also needs a simple model with which to test
the ideas of nonperturbative renormalizability without the complicated
structure of the full theory of General Relativity.  This paper
describes our work with such a model, the $O(2,1)$ noncompact nonlinear
$\sigma$-model.\reference{DeWitt, 1989}
B.S.~DeWitt, ``Nonlinear sigma models in 4 dimensions: a lattice definition,''
lectures given at the International School of Cosmology and Gravitation,
``Ettore Majorana'' Centre for Scientific Culture, Erice, Sicily, May 1989
\endreference\relax

The model we consider consists of three scalar fields $\varphi_a$
described by the action
$$
S = \half \mu^2 \int d^4 x \, \eta^{ab}
     \del_\mu \varphi_a \del^\mu \varphi_b  \,,
\EQN 2$$
with $\eta_{ab}={\rm diag}(-1,+1,+1)$ and with the fields obeying
the constraint
$$
-\varphi_0^2 +  \varphi_1^2 +  \varphi_2^2 = -1
        \qquad (\varphi_0 > 0)          \,.
\EQN 3$$
The manifold of constraint is the two dimensional surface of constant
negative curvature represented schematically 
in \Fig{1}.  It is the coset space
$O(2,1)/O(2)\times Z_2$, but for simplicity we refer to \Eqs{2} and
\Ep{3} as the $O(2,1)$ nonlinear $\sigma$-model.  There are several
reasons this model is of interest:

\item{1)}
For dimensionless fields $\varphi_a$ the coupling constant $\mu^2$ has
units of $(length)^2$, the same as $1/G_N$ in the Einstein-Hilbert
action.  Thus $\mu$ plays the role of the Planck mass in the theory.
Furthermore this means that Feynman graphs in the $\sigma$-model have
the same na\"{\i}ve degree of divergence as similar graphs in gravity,
so that the model has the same renormalizability structure (actually the
same perturbative non-renormalizability structure) as the theory of
gravity. 

\item{2)}
As in gravity, the fields of the $\sigma$-model obey a constraint,
and the configuration space defined by the constraint is both {\it
curved} and {\it noncompact}.

\item{3)}
The surface of constraint is invariant under global $O(2,1)$
transformations, but 
% unlike gravity 
there is no local symmetry in the
model.  This is a great simplification which lets us study just the
renormalizability properties of the model without the added
complications introduced by local symmetries.

\item{4)}
Unlike gravity, the Euclidean action of the $\sigma$-model is bounded
from below.  The unboundedness of the gravitational action is a serious
problem which must be dealt with at some point, but one which we want to
avoid entirely for now.

\figure{1}
\forceleft
\vskip\colwidth         % just leave some space to glue in figure
%%\epsfbox{o21.ps}        % or include with epsf
\caption{The constraint surface of the $O(2,1)$ noncompact nonlinear
$\sigma$-model.} 
\endfigure

\medskip

The transcription of the $\sigma$-model to the lattice is more or less
standard with one exception, our definition of the lattice derivative.
Rather than using the simple difference between field values at
neighboring lattice sites we use the geodesic distance between two
points on the constraint surface.  The lattice action is thus
$$
S = \half \mu^2 \sum_x a^4 \sum_{\hat\mu}
        [{ \Delta(\varphi(x+\hat\mu a), \varphi(x)) \over a}]^2 \,,
\EQN 4$$
where $\Delta(\varphi,\varphi^\prime)$ is the arc length between
$\varphi$ and $\varphi^\prime$ on the manifold,
$$
\Delta(\varphi,\varphi^\prime) = 
        \cosh^{-1}(-\eta^{ab} \varphi_a\varphi^\prime_b) \,.
\EQN 5$$
Our reason for this choice is that it is consistent with the idea that
the fields be restricted only to the constraint surface of the
$\sigma$-model.
In contrast, simply taking the na\"{\i}ve difference between fields, as
is usually done for compact $\sigma$-models, produces a difference
vector which does not lie in the manifold of constraint.  While both
methods lead to the same classical continuum limit there is nothing that
guarantees that the quantum theories obtained from the two lattice
definitions will be the same.

   It is convenient to factor the dependence on the lattice spacing $a$
out to the front of \Eq{4} and to define the dimensionless coupling
constant $\beta = \mu^2 a^2$.  If the theory is nonperturbatively
renormalizable the Planck mass $\mu$ will be renormalized to $\mu_R$,
which results in a renormalized dimensionless coupling constant
$$
\beta_R = \mu_R^2 a^2   \,.
\EQN 6$$
The renormalized Planck mass defines a characteristic length scale
$1/\mu_R$ for the interactions of the theory.  The lattice approximation
to the continuum theory will be reliable when $a \ll 1/\mu_R \ll L=Na$.
Considering each inequality separately, this requires
$$
\mu_R a = \sqrt{\beta_R}  \ll 1 
\qquad \hbox{\rm and} \qquad
 N   \gg {1\over \sqrt{\beta_R}}
\EQN 7$$
In the continuum limit $a \to 0$, hence $\beta \to 0$, and for $\mu_R$
to remain finite this requires $\beta_R \to 0$.  If this condition is
not fulfilled then it would appear to be impossible to define a
consistent quantum field theory from the $\sigma$-model, even
nonperturbatively. 

%----------------------------
\figure{2}
\forceright             % force this to the righthand column
\vskip\colwidth         % leave this much space 
%%\epsfbox{beta.ps}       % or include with EPSF
\caption{The renormalized dimensionless coupling constant $\beta_R$
plotted against the bare coupling constant $\beta$ for an $N=10$
lattice.} 
\endfigure
%----------------------------

In \Fig{2} we show $\beta_R$ plotted as a function of $\beta$ as
obtained from Monte Carlo simulations on an $N=10$ lattice.  As can
clearly be seen, $\beta_R$ vanishes nowhere.  We therefore conclude that
the $O(2,1)$ nonlinear sigma model does not have an interacting
continuum limit.  One may view $1/\mu_R$ as the renormalized coupling
constant in the theory, in which case our result implies that the model
is ``trivial'' (in the technical sense) in that the continuum limit is a
free field theory.

This work was supported by NSF grants PHY\-8617103 and PHY\-8919177.

\smallskip
%\nosechead{References}         % header for references
%\nobreak
\ListReferences

\bye

%>>> EOF Example.tex <<<
