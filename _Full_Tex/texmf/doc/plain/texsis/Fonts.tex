%% File: Fonts.tex  (TeXsis version 2.16)
%  $Revision: 16.2 $  :  $Date: 1994/09/14 14:57:08 $  :  $Author: myers $
%=======================================================================*
% TeXsis Manual -- TYPE SIZES AND STYLES  : This file is a part of TeXsis
% (C) Copyright 1989, 1992 by Eric Myers and Frank E. Paige
%======================================================================*
%
\texsis\book
\singlespaced
\TeXquoteon                             % | is a TeX quote (printed tt)
\input TXSdocM.doc                      % macros for manual

\titlepage     
\line{\revdate \hfill \TeXsis\ \fmtversion}     % a banner of sorts
\vbox{\vskip.75in}                              % some whitespace
\title
\TeXsis        
\TeX\ Macros for Physicists
\bigskip
-- Font Tables --
\endtitle      
%
\author
Eric Myers     
Department of Physics and Astronomy
Vassar College
Poughkeepsie, New York 12601~USA
\endauthor          
\and  
\author
Frank E. Paige
Physics Department        
Brookhaven National Laboratory 
Upton, New York 11973~USA
\endauthor
%
\abstract
This supplement to the \TeXsis\ manual contains tables of the characters
in all of the available fonts.
\endabstract
\endtitlepage

\parskip=\medskipamount         % some extra space between paragraphs
      
\appendix{B}{Font Tables}

\midtable{Fonts}
\Caption
Fonts used in \TeXsis, with the design size names and the required
sizes in points.
\endCaption
\ruledtable
 Base Name | Description      | Sizes (pt) \crthick
 cmr10     | Roman            | 5~~7~~9~~10~~11~~12~~14~~16~~20~~24 \cr
 cmti10    | Text italic      | 5~~7~~9~~10~~11~~12~~14~~16~~20~~24 \cr
 cmsl10    | Slanted          | 5~~7~~9~~10~~11~~12~~14~~16~~20~~24 \cr
 cmbx10    | Roman bold       | 5~~7~~9~~10~~11~~12~~14~~16~~20~~24 \cr
 cmmi10    | Math italic      | 5~~7~~9~~10~~11~~12~~14~~16~~20~~24 \cr
 cmsy10    | Math symbol      | 5~~7~~9~~10~~11~~12~~14~~16~~20~~24 \cr
 cmex10    | Math extended    | ~~~~~~~~~10~~11~~12~~14~~16~~20~~24 \cr
 cmtt10    | Typewriter       | ~~~~~~~~~10~~11~~12~~~~~~~~~~~~~~~~ \cr
 cmss10    | Sans serif       | ~~~~~~~~~10~~11~~12~~14~~~~~~~~~~~~ \crthick
 cmmib10   | Math italic bold | ~~~~~~~~~10~~11~~12~~14~~16~~20~~24 \cr
 cmbsy10   | Bold symbol      | ~~~~~~~~~10~~11~~12~~14~~16~~20~~24 
\endruledtable
\endtable
     
      The Computer Modern fonts that are used either by Plain \TeX\ or by
\TeXsis\ and the required sizes for each are listed in \Tbl{Fonts}. The
|cmmib10| and |cmbsy10| fonts may not exist in all installations, but
they will not be loaded unless |\mib| is used, so they are not required.
The command |\printfont{|\meta{font}|}| can be used to print a table of
the characters in any \meta{font}, e.g.,
\TeXexample
\printfont{cmr10}
|endTeXexample
Such tables are given on the following pages.

      Most versions of \TeX\ now use Computer Modern fonts, but some
still use the older American Modern fonts. If you have problems with the
fonts, try editing |TXSfonts.tex|, replacing all occurrences of |=cm|
with |=am| and then recompiling \TeXsis. Mixing the two types of fonts
will produce errors.

\vfil\eject

% \Printfont uses \printfont from printfont.txs, but puts the font
% table in a nice table with a caption.

\def\Printfont#1#2{% 	Print a nice font table, with caption
   \table{#2}%
     \printfont{#1}%
     \caption{Font {\tt #1} -- #2}%
   \endtable}


\Printfont{cmr10}{Roman}
\Printfont{cmti10}{Text Italic}
\Printfont{cmsl10}{Slanted}
\Printfont{cmbx10}{Roman bold}
\Printfont{cmmi10}{Math italic}
\Printfont{cmsy10}{Math symbol}
\Printfont{cmex10}{Math extended}
\Printfont{cmtt10}{Typewriter}
\Printfont{cmss10}{Sans serif}
\Printfont{cmmib10}{Math italic bold}
\Printfont{cmbsy10}{Math bold symbol}

\bye
     
%>>> EOF Fonts.tex <<<

