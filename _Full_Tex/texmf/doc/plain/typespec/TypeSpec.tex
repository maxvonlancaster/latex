\hsize4.5in 
\hoffset6.5in
\advance\hoffset-\hsize
\divide\hoffset2

\def\ob{\tt\string{}
\def\cb{\tt\string}}

\footline={\ifnum\pageno=1 \else \hss\folio\hss\fi}

\chardef\other=12
\def\deactivate{%
\catcode`\\=\other \catcode`\{=\other
\catcode`\}=\other \catcode`\$=\other
\catcode`\&=\other \catcode`\#=\other
\catcode`\%=\other \catcode`\~=\other
\catcode`\^=\other \catcode`\_=\other}

\def\makeactive#1{\catcode`#1=\active \ignorespaces}

\def\LaTeX{L\kern-.26em \raise.6ex\hbox{\fivei A}%
\kern-.15em TeX}%

{\makeactive\^^M %
\gdef\obeywhitespace{\makeactive\^^M %
\let^^M=\newline %
\aftergroup\removebox %
\obeyspaces}}

\def\newline{\par\indent}
\def\removebox{\setbox0=\lastbox}

{\catcode`\|=\active
\gdef\verbatim{%
		\par\begingroup\deactivate\obeywhitespace\tt
		\catcode`\|=\active
		\def|{\endgroup\par}}}

\clubpenalty10000
\widowpenalty10000
\raggedbottom
\interlinepenalty10000

\def\section#1\par{%
\medskip
\noindent\llap{\raise1.25pt\hbox{$\diamond$\enskip}}{\bf#1}\par\nobreak\noindent}

\section Introduction

I use a lot of type. I also use \TeX~--~what better medium can there be for showing off type?~--~in
its Macintosh incarnation as Textures. I need samples that I can show clients, and that I can
refer to for the technical details of the typeface. The results are the type specimens you see
here. The code itself has some annotations that you might want to read. The macros are not
complex~--~I don't have time to decode clever macros, and I may need to alter them for unusual
situations as they arise. You are encouraged to do the same. The current distribution of these
specimens includes: This introduction; eight type specimens; an Acrobat PDF file with typeset 
examples of the specimens and this introduction.

Some of the fonts in these examples (particularly {\it Centaur} in {\tt TypeSpc8.tex}) were made \TeX-ready 
by using Alan Jeffrey's {\tt fontinst}. The {\tt fontinst} package is discussed in considerable detail in Alan
Hoenig's new book {\it\TeX\  Unbound:\ \LaTeX\ and \TeX\ Strategies for Fonts, Graphics \& More} (Oxford 1998), which
discusses {\tt fontinst} over the course of several chapters and works through numerous examples. If you deal at all with
fonts and
\TeX\ you will need this book.

Clearly, these specimens will be most easily produced on systems that use outline fonts
as opposed to bitmaps. I have used these specimens on a Mac with Textures using both PostScript and TrueType, and on a PC
with PC\TeX using TrueType. Textures (now at version 2.0) is interesting in that it can create font metrics
``on the fly'' from the Mac's bitmap files. For more demanding work, you will, however, want to stick with {\tt fontinst}.

\section A word about color

Blue Sky Research {\it Textures} is a wonderful implementation of \TeX\ that has a very convenient
way to use color through the {\tt\string\special} mechanism. I'm going to describe it so that if
you use a different implementation that can take advantage of color, you will understand the
macros here. Color is invoked by: 

\verbatim\special{color rgb 1.0 0.0 0.0}|

\noindent This tells \TeX\ to trun everything red from this point on. This specification could
also have been given by:

\verbatim\special{color cmyk 0.0 1.0 0.1 0.0}|

\noindent as red in the CMYK model is 100\% of magenta and yellow. To jump back and forth between
colors, Textures has a color stack and you can push colors onto the stack and pop them off. So, a
typical use of color might be something like this:

\verbatim\special{color push}% Puts current color (black) on stack|
\verbatim%%% Calls red and puts it on stack:|
\verbatim\special{color rgb 1.0 0.0 0.0}|
\verbatim Some text in red.|
\verbatim\special{color pop}% Pops red off stack, restores black|
\verbatim More text in black.|

\noindent Simple. You can also define a color for later use:

\verbatim\special{color define red rgb 1.0 0.0 0.0}|

\noindent and summon it by calling:

\verbatim\special{color red}|

\noindent The advantage of this is that you can call colors that downstream processors (separation
software, or a printer) can use to identify color use. You can even screen the color for tints:

\verbatim\special{color red .5}|

\noindent which would give a 50\% tint of red, or pink.

\section TypeSpc1.tex, TypeSpc2.tex, TypeSpc3.tex

These are essentially variations on a theme, differing only in the display type at the top of the
page. TypeSpc1 has a selection of telltale characters. The top of the {\it A\/} often has a distinctive
treatment; the lower half of the {\it G\/} and the tail of the {\it Q\/} also tend to
be distinctive; the shape and size of the enclosed
areas of the lowercase {\it a\/} and {\it e\/} are also
telltale, and important in that very small such spaces
are likely to close up under less than ideal printing
conditions; the ampersand frequently has a good deal of
creativity lavished upon it. TypeSpc2 sets the
uppercase and lowercase of each letter next to each other. TypeSpc3 has a line of uppercase
characters (not good design, but some people like or expect it) followed by a line of lowercase characters.

\section Selecting the type

At the beginning you can insert three items of information: the \TeX\ name of the font; the name
of the font that will appear as the `title' of the specimen; and any comments that you wish to
have at the end of the chart. I like this specimen because the chosen typeface is used for the
whole specimen, including annotations. Clearly, these specimens and macros are more suited to
PostScript and TrueType fonts than bitmaps -- proceed accordingly.

\section Column width

Under normal circumstances, a column of type should be about 70 lowercase characters (about 2.5
lowercase alphabets) in width, or about 4.5 inches in width, which ever comes first. Much longer, and the eye tires, and
loses its place when moving to the next line; much shorter and you increase the level of visual
noise such as hyphenation and highly variable word spacing. This sample shows 10pt type at the
width of 2.5 lowercase alphabets, and calculates the size of type necessay to set 70 characters
to a measure of 4.5 inches. Interestingly, 10pt type
usually comes very close to fitting 4.5 inches with 70
characters. One of the most common signs of uninformed typography is
column width that pays no attention to the realities of
type and reading.

\section Leading

The calculation of leading~--~{\tt\string\baselineskip} in \TeX~--~is based purely on the
physical realities of the type being set. The maximum height and depth of the typeface's characters is found, and then a
percentage of that amount is added to itself to arrive at the leading. The x-height of the font
is also given some influence over the leading.

\section Choice of text

I have provided three sample texts in token registers. The first is a simple piece of English
prose. The second is a collection of {\it pangrams}~--~sentences using all twenty-six letters of the alphabet~--~from the large
collection of David Lemon of Adobe Systems, Inc.; so you get to see all of the letters of the alphabet in
use. The third is an amalgam of Latin and semi-Latin texts. In the first place, Latin~--~as
Beatrice Warde pointed out~--~has fewer descenders so the text sits more cleanly on the baseline;
second, as W.\ A.\ Dwiggins pointed out to Beatrice Warde, it is good to have a sample of text in
a language the reader does not read~--~in this way the reader looks at the type itself, and not
through the type to the information it conveys.

\section Type factor

The type factor is basically an extension of the the characters per pica concept, but scaled to
the size of the typeface. It is based on a sample of type in which $f=(c\,t)/m$ where $f$ is the
typefactor, $c$ is the number of characters in the sample, $t$ is the type size in points, and $m$
is the total horizontal width of the sample. The formula can be put to some good use. For
instance, characters per pica, $p$, can be obtained from $p=h/f$ where $h$ is the size of the type
in points; the size of the type, $h$, necessary to place $n$ characters on a column width of $w$
(in picas) is: $h=(fw)/n$. I have used a sample based on the relative frequency of letters in
English. This gives excellent agreement with published figures, particularly those in {\it Brown's
Index}. It is best not to take figures like characters-per-pica or type factors too seriously. They are very useful for
judging the relative characteristics of typefaces. Before computer typesetting they were far more useful when the actual
typesetting of material was costly and time-consuming. Computers allow us to be able to change typesetting parameters
interactively and quickly, relieving us from having to know precisely ahead of time how the material is going to layout.

\section The ratio of the x-height to the ascender height

This is an important characheristic of type. As x-height increases, type tends to look darker, or
more crowded on the page. The antidote to this is to increase the leading slightly. {\it See above.} 

\section TypeSpc4.tex -- a mini-specimen

The principal attraction of TypeSpc4 is the
{\tt\string\makefontbox} macro which gives a kind miniature specimen that can be as little or as
large as you like. In the examples I give, the grid is certainly optional as is the use of color:
They are merely suggestions to get you started. The five arguments to {\tt\string\makefontbox} are
as follows:

\medskip

\vbox{\raggedright\parskip3pt

\itemitem{\#1}\TeX's name for the desired font

\itemitem{\#2}The name font name as you want it to print

\itemitem{\#3}The name of the type's designer, or other salient
          piece of information that lends itself to display

\itemitem{\#4}The size of the type inside the box

\itemitem{\#5}The overall desired width of the box}

\section TypeSpc5.tex -- by the numbers

It is very often useful to know where the characters of a typeface are, and to gain an overall
impression of all of the characters of a font. This specimen is simply an enumeration of the
characters with respect to their character numbers. The macro to notice is
{\tt\string\lineofchars}. You can specify how may characters are to be in the line, and what the
numerical starting point will be. Numerous other design variables can be specified, and they are
discussed in TypeSpc5.tex.

\section TypeSpc6.tex -- for florid fonts

There are times when you simply have to see a font in all its splendor. You don't need it at every
imaginable size. And in the case of ornate calligraphic fonts, it does not make any sense at all
to shove uppercase letters together for any reason, not even for display.

\section TypeSpc7.tex -- homage \`a Caslon

This ornate example is meant to be both fun and useful. It was inspired by the specimen sheets of
William Caslon. The names of the type sizes~--~Ruby, Minion, Brevier, etc.~--~are traditional and
could be replaced with sizes in points. This is a useful sample for PostScript typefaces which are
often packaged as a family of four fonts: roman, italic, bold and bold-italic. The principle fonts
are the roman and italic, however, and these are the fonts primarily featured in this specimen. The
sample text is in a token list and can be changed at will~--~just make sure that there is enough
of it.

\section TypeSpc8.tex

This specimen uses techniques that we have established in earlier samples. Again, this is suitable
for typefaces that exist in pairs (or pairs of pairs) of fonts: roman and italic, bold and
bold-italic. There is also a macro for a single, centered paragraph that could be used, say, for
small caps. The type size, leading and type factor are given for each font.

\section TypeSpc9.tex

I combined two formats here to show off the principal characteristics of the roman along with the
fonts of the family as used for text.

\section TypeSp10.tex

Another variation using one sample per line. Also notice the very beautiful typeface.

\bigskip
\hrule

\section Stephen Moye, 23 IV 1998

Stephen\hbox to .5em{\hrulefill}Moye@brown.edu\hfill\break Brown University Graphic Services 

\bye