\documentclass[12pt,twoside]{xepersian-magazine}
\usepackage{graphicx}
\usepackage{xltxtra}
\usepackage{amsmath}
\usepackage{xepersian}
\settextfont[Scale=1]{XB Zar}
\setlatintextfont[Scale=1]{Junicode}
\setdigitfont{XB Zar}
\pagestyle{fancy}
\title{مجلهٔ زی‌پرشین}
\author{وفا خلیقی}
\edition{جلد اول}
\customlogo{مجلهٔ زی‌پرشین}
\customminilogo{مجلهٔ زی‌پرشین}
\custommagazinename{مجلهٔ زی‌پرشین}
\customwwwTxt{http://google.com}
\begin{document}
\begin{frontpage}
\firstimage{img/ireland.jpg}{این زیرنویس تصویر اصلی در صفحهٔ اول است.}
\firstarticle{این تیتر مقالهٔ اول است.}
{خوب این قسمت کوچکی از مقالهٔ اول است که ما در حال نوشتن آن هستم. باید یک مقدار بنویسیم تا مقداری این قسمت پر شود تا بتوانیم چیز قشنگی داشته باشیم. دقت کنیم که بصورت انتخابی حتی می‌توانیم زمان را هم درج کنیم که در سمت راست قرار می‌گیرد.}%
{۱۲:۳۴}
\secondarticle{این هم سر تیتر مقالهٔ دوم است.}%
{این هم زیر تیتر مقالهٔ دوم است که آن را در اینجا می‌نویسیم.}%
{خوب این قسمت کوچکی از مقالهٔ اول است که ما در حال نوشتن آن هستم. باید یک مقدار بنویسیم تا مقداری این قسمت پر شود تا بتوانیم چیز قشنگی داشته باشیم. دقت کنیم که بصورت انتخابی حتی می‌توانیم زمان را هم درج کنیم که در سمت راست قرار می‌گیرد.}%
{قسمت الف}%
{۱۰:۲۳}

\thirdarticle{این سرتیتر مقالهٔ سوم است.}%
{این هم زیرتیتر مقالهٔ سوم است که ما آن را در اینجا قرار می‌دهیم.}%
{خوب این قسمت کوچکی از مقالهٔ اول است که ما در حال نوشتن آن هستم. باید یک مقدار بنویسیم تا مقداری این قسمت پر شود تا بتوانیم چیز قشنگی داشته باشیم. دقت کنیم که بصورت انتخابی حتی می‌توانیم زمان را هم درج کنیم که در سمت راست قرار می‌گیرد. و همانطور که می‌بینید من مطلبی برای گفتن ندارم فقط متن علکی می‌نویسم تا کمی صفحه را پر کرده باشم. اما در قسمتهای بعدی مقداری از سهراب سپهری خواهم نوشت.}%

{قسمت ب}%
{۱۰:۰۲}

\begin{indexblock}{نمایه (فهرست مطالب) اصلی}
\indexitem{۱- مقاله اول}{1}

\indexitem{۲- مقاله دوم}{3}

\indexitem{۳- مقاله سوم}{3}

\indexitem{۴- مقاله چهارم}{5}
\end{indexblock}

\begin{weatherblock}{وضع آب و هوا}
\weatheritem{img/weather/rain.jpg}{امروز}{13}{9}{}
\weatheritem{img/weather/sun.jpg}{فردا}{15}{1}{}
\weatheritem{img/weather/clouds.jpg}{جمعه}{12}{6}{}
\end{weatherblock}

\begin{authorblock}
\textbf{ویرایشگران}

وفا خلیقی، مهدی امیدعلی و مصطفی واحدی

\texttt{vafakh84@gmail.com\\[5pt]
http://google.com}\\
\end{authorblock}
\end{frontpage}
\newsection{قسمت الف}
\begin{article}{2}
{این تیتر این مقاله است.}
{این هم زیرتیتر این مقاله هست.}
{قسمت الف}
{1}
\authorandplace{نام نویسنده}{مکان}

\noindent\timestamp{۸:۲۵}
ويژگی اصلی که اين معماری را متمايز کرده و در دنيای اينترنت آن‌ها در مقابل معماری قبلی شبكه‌ها قرار داده است، امكان ايجاد ارتباط مستقيم  بين كامپيوترهای مختلف بدون نياز به دخالت سرورهای قدرتمند در بين راه است.  نام‌ اين نوع معماری هم در واقع بر گرفته شده از همین  ارتباط مستقیم بين گره‌ها است.  در واقع در اين نوع شبكه‌ها اثری از سرورها نيست و تمامی گره‌های معمولی موجود در شبكه، بايد وظايفی را که قبلا بر عهده‌ی سرورها بود، خود انجام دهند. بنابراين در اين نوع معماری گره‌های معمولی در ضمن اين‌که از خدمات شبکه بهره‌مند می‌شود بايد نقش خدمت‌گزار را هم ايفا کنند . در اين نوع شبکه‌ها گره‌های معمولی به كمك روش‌ها و پروتكل‌های توزيع شده، تمامی وظايف  مسير يابی در شبكه، جستجوی منابع، امنيت شبكه و شناسايی هويت استفاده كننده‌ها و هم‌چنين مقابله با حملات احتمالی مهاجمان را بر عهده دارند.
\footnote{این یک زیرنویس فارسی است.}\LTRfootnote{This is an English footnote.}
\begin{equation}
(a+b)^3=a^3+3a^2b+3ab^2+b^3\label{eq-1}
\end{equation}
این معادلهٔ \eqref{eq-1} است.
\columntitle{lines}{این را برای مهم یا نشان دادن حرفی مهم در این مجله انجام می‌دهیم.}

ويژگی اصلی که اين معماری را متمايز کرده و در دنيای اينترنت آن‌ها در مقابل معماری قبلی شبكه‌ها قرار داده است، امكان ايجاد ارتباط مستقيم  بين كامپيوترهای مختلف بدون نياز به دخالت سرورهای قدرتمند در بين راه است.  نام‌ اين نوع معماری هم در واقع بر گرفته شده از همین  ارتباط مستقیم بين گره‌ها است.  در واقع در اين نوع شبكه‌ها اثری از سرورها نيست و تمامی گره‌های معمولی موجود در شبكه، بايد وظايفی را که قبلا بر عهده‌ی سرورها بود، خود انجام دهند. بنابراين در اين نوع معماری گره‌های معمولی در ضمن اين‌که از خدمات شبکه بهره‌مند می‌شود بايد نقش خدمت‌گزار را هم ايفا کنند . در اين نوع شبکه‌ها گره‌های معمولی به كمك روش‌ها و پروتكل‌های توزيع شده، تمامی وظايف  مسير يابی در شبكه، جستجوی منابع، امنيت شبكه و شناسايی هويت استفاده كننده‌ها و هم‌چنين مقابله با حملات احتمالی مهاجمان را بر عهده دارند.

اما معماری همتابه‌همتا ويژگی‌های ديگری نيز دارد که آن را هم برای فراهم‌کنندگان کاربردها و هم برای استفاده‌کنندگان جذاب‌تر می‌کند.  از آن‌جا که شبکه‌های همتابه‌همتا از همان زيرساخت‌های اينترنت استفاده می‌کنند ونيازی به راه‌اندازی سرورها ندارند، ساخت اين شبكه‌ها بسيار ارزان‌تر از ايجاد زير ساخت‌های لازم برای راه‌اندازی شبكه‌های مشتری/خدمت‌گزار است.  هم‌چنين با زياد شدن تعداد کاربران چون درعمل تعداد گره‌های ارائه کننده‌ی خدمات هم بالا می‌رود، نه تنها عملكرد شبكه افت پيدا نمی‌كند بلكه انتظار بهبود عملکرد نيز می‌رود. گذشته از اين موارد، مالكيت اين شبكه‌ها به صورت اشتراكی بين تمام کاربران پخش شده و هيچ شخص يا شركتی نمی‌تواند سياست‌های دلخواه خود را در اين نوع شبكه‌ها اعمال کند.

اماهيچ چيزی بی‌ بها به دست نمی‌آید. نبود سرور مرکزی اگر چه  ويژگی‌های جذابی به شبکه‌های همتابه‌همتا می‌بخشد اما از طرف ديگر آن‌ها را با دشواری‌هايی نيز روبه‌رو می‌کند.  عدم وجود يك هماهنگ كننده مركزی در شبكه، انجام بسياری از امور و ارائه خدمات را  دچار مشكل می‌کند.  از يک طرف، تغيير و رفت‌وآمد زیاد کاربران از ويژگی‌های ذاتی اين شبکه‌ها است و از طرف ديگر در اين شبكه‌ها، ديگر اين گره‌های معمولی  هستند كه  عهده‌دار تمامی وظايف هستند. به همين دلیل يکی از مشکلات اصلی فراروی اين شبكه‌ها، مقابله با  تغييرات لحظه‌ای و فراهم آوردن ثبات در ارائه  خدمات در بستری از بی‌ثباتی است.
\end{article}

\articlesep

\begin{article}{2}
{این تیتر این مقاله است.}
{این هم زیرتیتر این مقاله هست.}
{قسمت الف}
{1}
\authorandplace{نام نویسنده}{مکان}

\noindent\timestamp{08:25}
ويژگی اصلی که اين معماری را متمايز کرده و در دنيای اينترنت آن‌ها در مقابل معماری قبلی شبكه‌ها قرار داده است، امكان ايجاد ارتباط مستقيم  بين كامپيوترهای مختلف بدون نياز به دخالت سرورهای قدرتمند در بين راه است.  نام‌ اين نوع معماری هم در واقع بر گرفته شده از همین  ارتباط مستقیم بين گره‌ها است.  در واقع در اين نوع شبكه‌ها اثری از سرورها نيست و تمامی گره‌های معمولی موجود در شبكه، بايد وظايفی را که قبلا بر عهده‌ی سرورها بود، خود انجام دهند. بنابراين در اين نوع معماری گره‌های معمولی در ضمن اين‌که از خدمات شبکه بهره‌مند می‌شود بايد نقش خدمت‌گزار را هم ايفا کنند . در اين نوع شبکه‌ها گره‌های معمولی به كمك روش‌ها و پروتكل‌های توزيع شده، تمامی وظايف  مسير يابی در شبكه، جستجوی منابع، امنيت شبكه و شناسايی هويت استفاده كننده‌ها و هم‌چنين مقابله با حملات احتمالی مهاجمان را بر عهده دارند.
\LTRfootnote{This is an English footnote.}\footnote{این یک زیرنویس فارسی است.}
اما معماری همتابه‌همتا ويژگی‌های ديگری نيز دارد که آن را هم برای فراهم‌کنندگان کاربردها و هم برای استفاده‌کنندگان جذاب‌تر می‌کند.  از آن‌جا که شبکه‌های همتابه‌همتا از همان زيرساخت‌های اينترنت استفاده می‌کنند ونيازی به راه‌اندازی سرورها ندارند، ساخت اين شبكه‌ها بسيار ارزان‌تر از ايجاد زير ساخت‌های لازم برای راه‌اندازی شبكه‌های مشتری/خدمت‌گزار است.  هم‌چنين با زياد شدن تعداد کاربران چون درعمل تعداد گره‌های ارائه کننده‌ی خدمات هم بالا می‌رود، نه تنها عملكرد شبكه افت پيدا نمی‌كند بلكه انتظار بهبود عملکرد نيز می‌رود. گذشته از اين موارد، مالكيت اين شبكه‌ها به صورت اشتراكی بين تمام کاربران پخش شده و هيچ شخص يا شركتی نمی‌تواند سياست‌های دلخواه خود را در اين نوع شبكه‌ها اعمال کند.

اماهيچ چيزی بی‌ بها به دست نمی‌آید. نبود سرور مرکزی اگر چه  ويژگی‌های جذابی به شبکه‌های همتابه‌همتا می‌بخشد اما از طرف ديگر آن‌ها را با دشواری‌هايی نيز روبه‌رو می‌کند.  عدم وجود يك هماهنگ كننده مركزی در شبكه، انجام بسياری از امور و ارائه خدمات را  دچار مشكل می‌کند.  از يک طرف، تغيير و رفت‌وآمد زیاد کاربران از ويژگی‌های ذاتی اين شبکه‌ها است و از طرف ديگر در اين شبكه‌ها، ديگر اين گره‌های معمولی  هستند كه  عهده‌دار تمامی وظايف هستند. به همين دلیل يکی از مشکلات اصلی فراروی اين شبكه‌ها، مقابله با  تغييرات لحظه‌ای و فراهم آوردن ثبات در ارائه  خدمات در بستری از بی‌ثباتی است.

ويژگی اصلی که اين معماری را متمايز کرده و در دنيای اينترنت آن‌ها در مقابل معماری قبلی شبكه‌ها قرار داده است، امكان ايجاد ارتباط مستقيم  بين كامپيوترهای مختلف بدون نياز به دخالت سرورهای قدرتمند در بين راه است.  نام‌ اين نوع معماری هم در واقع بر گرفته شده از همین  ارتباط مستقیم بين گره‌ها است.  در واقع در اين نوع شبكه‌ها اثری از سرورها نيست و تمامی گره‌های معمولی موجود در شبكه، بايد وظايفی را که قبلا بر عهده‌ی سرورها بود، خود انجام دهند. بنابراين در اين نوع معماری گره‌های معمولی در ضمن اين‌که از خدمات شبکه بهره‌مند می‌شود بايد نقش خدمت‌گزار را هم ايفا کنند . در اين نوع شبکه‌ها گره‌های معمولی به كمك روش‌ها و پروتكل‌های توزيع شده، تمامی وظايف  مسير يابی در شبكه، جستجوی منابع، امنيت شبكه و شناسايی هويت استفاده كننده‌ها و هم‌چنين مقابله با حملات احتمالی مهاجمان را بر عهده دارند.

اما معماری همتابه‌همتا ويژگی‌های ديگری نيز دارد که آن را هم برای فراهم‌کنندگان کاربردها و هم برای استفاده‌کنندگان جذاب‌تر می‌کند.  از آن‌جا که شبکه‌های همتابه‌همتا از همان زيرساخت‌های اينترنت استفاده می‌کنند ونيازی به راه‌اندازی سرورها ندارند، ساخت اين شبكه‌ها بسيار ارزان‌تر از ايجاد زير ساخت‌های لازم برای راه‌اندازی شبكه‌های مشتری/خدمت‌گزار است.  هم‌چنين با زياد شدن تعداد کاربران چون درعمل تعداد گره‌های ارائه کننده‌ی خدمات هم بالا می‌رود، نه تنها عملكرد شبكه افت پيدا نمی‌كند بلكه انتظار بهبود عملکرد نيز می‌رود. گذشته از اين موارد، مالكيت اين شبكه‌ها به صورت اشتراكی بين تمام کاربران پخش شده و هيچ شخص يا شركتی نمی‌تواند سياست‌های دلخواه خود را در اين نوع شبكه‌ها اعمال کند.

اماهيچ چيزی بی‌ بها به دست نمی‌آید. نبود سرور مرکزی اگر چه  ويژگی‌های جذابی به شبکه‌های همتابه‌همتا می‌بخشد اما از طرف ديگر آن‌ها را با دشواری‌هايی نيز روبه‌رو می‌کند.  عدم وجود يك هماهنگ كننده مركزی در شبكه، انجام بسياری از امور و ارائه خدمات را  دچار مشكل می‌کند.  از يک طرف، تغيير و رفت‌وآمد زیاد کاربران از ويژگی‌های ذاتی اين شبکه‌ها است و از طرف ديگر در اين شبكه‌ها، ديگر اين گره‌های معمولی  هستند كه  عهده‌دار تمامی وظايف هستند. به همين دلیل يکی از مشکلات اصلی فراروی اين شبكه‌ها، مقابله با  تغييرات لحظه‌ای و فراهم آوردن ثبات در ارائه  خدمات در بستری از بی‌ثباتی است.

\expandedtitle{doublebox}{این هم مطلی است مهم یا چیزی که از خلاصهٔ این مقاله ما متوجه شده‌ایم و این برای ما و خوانندگان خیلی مهم است.}

ويژگی اصلی که اين معماری را متمايز کرده و در دنيای اينترنت آن‌ها در مقابل معماری قبلی شبكه‌ها قرار داده است، امكان ايجاد ارتباط مستقيم  بين كامپيوترهای مختلف بدون نياز به دخالت سرورهای قدرتمند در بين راه است.  نام‌ اين نوع معماری هم در واقع بر گرفته شده از همین  ارتباط مستقیم بين گره‌ها است.  در واقع در اين نوع شبكه‌ها اثری از سرورها نيست و تمامی گره‌های معمولی موجود در شبكه، بايد وظايفی را که قبلا بر عهده‌ی سرورها بود، خود انجام دهند. بنابراين در اين نوع معماری گره‌های معمولی در ضمن اين‌که از خدمات شبکه بهره‌مند می‌شود بايد نقش خدمت‌گزار را هم ايفا کنند . در اين نوع شبکه‌ها گره‌های معمولی به كمك روش‌ها و پروتكل‌های توزيع شده، تمامی وظايف  مسير يابی در شبكه، جستجوی منابع، امنيت شبكه و شناسايی هويت استفاده كننده‌ها و هم‌چنين مقابله با حملات احتمالی مهاجمان را بر عهده دارند.

اما معماری همتابه‌همتا ويژگی‌های ديگری نيز دارد که آن را هم برای فراهم‌کنندگان کاربردها و هم برای استفاده‌کنندگان جذاب‌تر می‌کند.  از آن‌جا که شبکه‌های همتابه‌همتا از همان زيرساخت‌های اينترنت استفاده می‌کنند ونيازی به راه‌اندازی سرورها ندارند، ساخت اين شبكه‌ها بسيار ارزان‌تر از ايجاد زير ساخت‌های لازم برای راه‌اندازی شبكه‌های مشتری/خدمت‌گزار است.  هم‌چنين با زياد شدن تعداد کاربران چون درعمل تعداد گره‌های ارائه کننده‌ی خدمات هم بالا می‌رود، نه تنها عملكرد شبكه افت پيدا نمی‌كند بلكه انتظار بهبود عملکرد نيز می‌رود. گذشته از اين موارد، مالكيت اين شبكه‌ها به صورت اشتراكی بين تمام کاربران پخش شده و هيچ شخص يا شركتی نمی‌تواند سياست‌های دلخواه خود را در اين نوع شبكه‌ها اعمال کند.

اماهيچ چيزی بی‌ بها به دست نمی‌آید. نبود سرور مرکزی اگر چه  ويژگی‌های جذابی به شبکه‌های همتابه‌همتا می‌بخشد اما از طرف ديگر آن‌ها را با دشواری‌هايی نيز روبه‌رو می‌کند.  عدم وجود يك هماهنگ كننده مركزی در شبكه، انجام بسياری از امور و ارائه خدمات را  دچار مشكل می‌کند.  از يک طرف، تغيير و رفت‌وآمد زیاد کاربران از ويژگی‌های ذاتی اين شبکه‌ها است و از طرف ديگر در اين شبكه‌ها، ديگر اين گره‌های معمولی  هستند كه  عهده‌دار تمامی وظايف هستند. به همين دلیل يکی از مشکلات اصلی فراروی اين شبكه‌ها، مقابله با  تغييرات لحظه‌ای و فراهم آوردن ثبات در ارائه  خدمات در بستری از بی‌ثباتی است.
\end{article}

\articlesep

\newsection{قسمت ب}

\begin{article}{2}
{این یک تیتر کوتاه است.وفا خلیقی}
{این هم مثل همیشه زیرتیتر است که ما آن را در اینجا قرار می‌دهیم.}
{قسمت ب}
{3}

\authorandplace{نام نویسنده}{مکان}

\noindent\timestamp{08:25}  et ويژگی اصلی که اين معماری را متمايز کرده و در دنيای اينترنت آن‌ها در مقابل معماری قبلی شبكه‌ها قرار داده است، امكان ايجاد ارتباط مستقيم  بين كامپيوترهای مختلف بدون نياز به دخالت سرورهای قدرتمند در بين راه است.  نام‌ اين نوع معماری هم در واقع بر گرفته شده از همین  ارتباط مستقیم بين گره‌ها است.  در واقع در اين نوع شبكه‌ها اثری از سرورها نيست و تمامی گره‌های معمولی موجود در شبكه، بايد وظايفی را که قبلا بر عهده‌ی سرورها بود، خود انجام دهند. بنابراين در اين نوع معماری گره‌های معمولی در ضمن اين‌که از خدمات شبکه بهره‌مند می‌شود بايد نقش خدمت‌گزار را هم ايفا کنند . در اين نوع شبکه‌ها گره‌های معمولی به كمك روش‌ها و پروتكل‌های توزيع شده، تمامی وظايف  مسير يابی در شبكه، جستجوی منابع، امنيت شبكه و شناسايی هويت استفاده كننده‌ها و هم‌چنين مقابله با حملات احتمالی مهاجمان را بر عهده دارند.

اما معماری همتابه‌همتا ويژگی‌های ديگری نيز دارد که آن را هم برای فراهم‌کنندگان کاربردها و هم برای استفاده‌کنندگان جذاب‌تر می‌کند.  از آن‌جا که شبکه‌های همتابه‌همتا از همان زيرساخت‌های اينترنت استفاده می‌کنند ونيازی به راه‌اندازی سرورها ندارند، ساخت اين شبكه‌ها بسيار ارزان‌تر از ايجاد زير ساخت‌های لازم برای راه‌اندازی شبكه‌های مشتری/خدمت‌گزار است.  هم‌چنين با زياد شدن تعداد کاربران چون درعمل تعداد گره‌های ارائه کننده‌ی خدمات هم بالا می‌رود، نه تنها عملكرد شبكه افت پيدا نمی‌كند بلكه انتظار بهبود عملکرد نيز می‌رود. گذشته از اين موارد، مالكيت اين شبكه‌ها به صورت اشتراكی بين تمام کاربران پخش شده و هيچ شخص يا شركتی نمی‌تواند سياست‌های دلخواه خود را در اين نوع شبكه‌ها اعمال کند.

اماهيچ چيزی بی‌ بها به دست نمی‌آید. نبود سرور مرکزی اگر چه  ويژگی‌های جذابی به شبکه‌های همتابه‌همتا می‌بخشد اما از طرف ديگر آن‌ها را با دشواری‌هايی نيز روبه‌رو می‌کند.  عدم وجود يك هماهنگ كننده مركزی در شبكه، انجام بسياری از امور و ارائه خدمات را  دچار مشكل می‌کند.  از يک طرف، تغيير و رفت‌وآمد زیاد کاربران از ويژگی‌های ذاتی اين شبکه‌ها است و از طرف ديگر در اين شبكه‌ها، ديگر اين گره‌های معمولی  هستند كه  عهده‌دار تمامی وظايف هستند. به همين دلیل يکی از مشکلات اصلی فراروی اين شبكه‌ها، مقابله با  تغييرات لحظه‌ای و فراهم آوردن ثبات در ارائه  خدمات در بستری از بی‌ثباتی است.

ويژگی اصلی که اين معماری را متمايز کرده و در دنيای اينترنت آن‌ها در مقابل معماری قبلی شبكه‌ها قرار داده است، امكان ايجاد ارتباط مستقيم  بين كامپيوترهای مختلف بدون نياز به دخالت سرورهای قدرتمند در بين راه است.  نام‌ اين نوع معماری هم در واقع بر گرفته شده از همین  ارتباط مستقیم بين گره‌ها است.  در واقع در اين نوع شبكه‌ها اثری از سرورها نيست و تمامی گره‌های معمولی موجود در شبكه، بايد وظايفی را که قبلا بر عهده‌ی سرورها بود، خود انجام دهند. بنابراين در اين نوع معماری گره‌های معمولی در ضمن اين‌که از خدمات شبکه بهره‌مند می‌شود بايد نقش خدمت‌گزار را هم ايفا کنند . در اين نوع شبکه‌ها گره‌های معمولی به كمك روش‌ها و پروتكل‌های توزيع شده، تمامی وظايف  مسير يابی در شبكه، جستجوی منابع، امنيت شبكه و شناسايی هويت استفاده كننده‌ها و هم‌چنين مقابله با حملات احتمالی مهاجمان را بر عهده دارند.

اما معماری همتابه‌همتا ويژگی‌های ديگری نيز دارد که آن را هم برای فراهم‌کنندگان کاربردها و هم برای استفاده‌کنندگان جذاب‌تر می‌کند.  از آن‌جا که شبکه‌های همتابه‌همتا از همان زيرساخت‌های اينترنت استفاده می‌کنند ونيازی به راه‌اندازی سرورها ندارند، ساخت اين شبكه‌ها بسيار ارزان‌تر از ايجاد زير ساخت‌های لازم برای راه‌اندازی شبكه‌های مشتری/خدمت‌گزار است.  هم‌چنين با زياد شدن تعداد کاربران چون درعمل تعداد گره‌های ارائه کننده‌ی خدمات هم بالا می‌رود، نه تنها عملكرد شبكه افت پيدا نمی‌كند بلكه انتظار بهبود عملکرد نيز می‌رود. گذشته از اين موارد، مالكيت اين شبكه‌ها به صورت اشتراكی بين تمام کاربران پخش شده و هيچ شخص يا شركتی نمی‌تواند سياست‌های دلخواه خود را در اين نوع شبكه‌ها اعمال کند.

اماهيچ چيزی بی‌ بها به دست نمی‌آید. نبود سرور مرکزی اگر چه  ويژگی‌های جذابی به شبکه‌های همتابه‌همتا می‌بخشد اما از طرف ديگر آن‌ها را با دشواری‌هايی نيز روبه‌رو می‌کند.  عدم وجود يك هماهنگ كننده مركزی در شبكه، انجام بسياری از امور و ارائه خدمات را  دچار مشكل می‌کند.  از يک طرف، تغيير و رفت‌وآمد زیاد کاربران از ويژگی‌های ذاتی اين شبکه‌ها است و از طرف ديگر در اين شبكه‌ها، ديگر اين گره‌های معمولی  هستند كه  عهده‌دار تمامی وظايف هستند. به همين دلیل يکی از مشکلات اصلی فراروی اين شبكه‌ها، مقابله با  تغييرات لحظه‌ای و فراهم آوردن ثبات در ارائه  خدمات در بستری از بی‌ثباتی است.

\expandedtitle{lines}{این هم دوباره مطلب مهمی است که ما آن را از لابلای این مقاله برای خواننده درست کرده‌ایم.}

ويژگی اصلی که اين معماری را متمايز کرده و در دنيای اينترنت آن‌ها در مقابل معماری قبلی شبكه‌ها قرار داده است، امكان ايجاد ارتباط مستقيم  بين كامپيوترهای مختلف بدون نياز به دخالت سرورهای قدرتمند در بين راه است.  نام‌ اين نوع معماری هم در واقع بر گرفته شده از همین  ارتباط مستقیم بين گره‌ها است.  در واقع در اين نوع شبكه‌ها اثری از سرورها نيست و تمامی گره‌های معمولی موجود در شبكه، بايد وظايفی را که قبلا بر عهده‌ی سرورها بود، خود انجام دهند. بنابراين در اين نوع معماری گره‌های معمولی در ضمن اين‌که از خدمات شبکه بهره‌مند می‌شود بايد نقش خدمت‌گزار را هم ايفا کنند . در اين نوع شبکه‌ها گره‌های معمولی به كمك روش‌ها و پروتكل‌های توزيع شده، تمامی وظايف  مسير يابی در شبكه، جستجوی منابع، امنيت شبكه و شناسايی هويت استفاده كننده‌ها و هم‌چنين مقابله با حملات احتمالی مهاجمان را بر عهده دارند.

اما معماری همتابه‌همتا ويژگی‌های ديگری نيز دارد که آن را هم برای فراهم‌کنندگان کاربردها و هم برای استفاده‌کنندگان جذاب‌تر می‌کند.  از آن‌جا که شبکه‌های همتابه‌همتا از همان زيرساخت‌های اينترنت استفاده می‌کنند ونيازی به راه‌اندازی سرورها ندارند، ساخت اين شبكه‌ها بسيار ارزان‌تر از ايجاد زير ساخت‌های لازم برای راه‌اندازی شبكه‌های مشتری/خدمت‌گزار است.  هم‌چنين با زياد شدن تعداد کاربران چون درعمل تعداد گره‌های ارائه کننده‌ی خدمات هم بالا می‌رود، نه تنها عملكرد شبكه افت پيدا نمی‌كند بلكه انتظار بهبود عملکرد نيز می‌رود. گذشته از اين موارد، مالكيت اين شبكه‌ها به صورت اشتراكی بين تمام کاربران پخش شده و هيچ شخص يا شركتی نمی‌تواند سياست‌های دلخواه خود را در اين نوع شبكه‌ها اعمال کند.

اماهيچ چيزی بی‌ بها به دست نمی‌آید. نبود سرور مرکزی اگر چه  ويژگی‌های جذابی به شبکه‌های همتابه‌همتا می‌بخشد اما از طرف ديگر آن‌ها را با دشواری‌هايی نيز روبه‌رو می‌کند.  عدم وجود يك هماهنگ كننده مركزی در شبكه، انجام بسياری از امور و ارائه خدمات را  دچار مشكل می‌کند.  از يک طرف، تغيير و رفت‌وآمد زیاد کاربران از ويژگی‌های ذاتی اين شبکه‌ها است و از طرف ديگر در اين شبكه‌ها، ديگر اين گره‌های معمولی  هستند كه  عهده‌دار تمامی وظايف هستند. به همين دلیل يکی از مشکلات اصلی فراروی اين شبكه‌ها، مقابله با  تغييرات لحظه‌ای و فراهم آوردن ثبات در ارائه  خدمات در بستری از بی‌ثباتی است.
\end{article}

\articlesep

\begin{editorial}{1}{این یک مثال از مقاله‌ای از طرف ویرایشگر است.}{نام و نام خانوادگی}{4}
يژگی اصلی که اين معماری را متمايز کرده و در دنيای اينترنت آن‌ها در مقابل معماری قبلی شبكه‌ها قرار داده است، امكان ايجاد ارتباط مستقيم  بين كامپيوترهای مختلف بدون نياز به دخالت سرورهای قدرتمند در بين راه است.  نام‌ اين نوع معماری هم در واقع بر گرفته شده از همین  ارتباط مستقیم بين گره‌ها است.  در واقع در اين نوع شبكه‌ها اثری از سرورها نيست و تمامی گره‌های معمولی موجود در شبكه، بايد وظايفی را که قبلا بر عهده‌ی سرورها بود، خود انجام دهند. بنابراين در اين نوع معماری گره‌های معمولی در ضمن اين‌که از خدمات شبکه بهره‌مند می‌شود بايد نقش خدمت‌گزار را هم ايفا کنند . در اين نوع شبکه‌ها گره‌های معمولی به كمك روش‌ها و پروتكل‌های توزيع شده، تمامی وظايف  مسير يابی در شبكه، جستجوی منابع، امنيت شبكه و شناسايی هويت استفاده كننده‌ها و هم‌چنين مقابله با حملات احتمالی مهاجمان را بر عهده دارند.

اما معماری همتابه‌همتا ويژگی‌های ديگری نيز دارد که آن را هم برای فراهم‌کنندگان کاربردها و هم برای استفاده‌کنندگان جذاب‌تر می‌کند.  از آن‌جا که شبکه‌های همتابه‌همتا از همان زيرساخت‌های اينترنت استفاده می‌کنند ونيازی به راه‌اندازی سرورها ندارند، ساخت اين شبكه‌ها بسيار ارزان‌تر از ايجاد زير ساخت‌های لازم برای راه‌اندازی شبكه‌های مشتری/خدمت‌گزار است.  هم‌چنين با زياد شدن تعداد کاربران چون درعمل تعداد گره‌های ارائه کننده‌ی خدمات هم بالا می‌رود، نه تنها عملكرد شبكه افت پيدا نمی‌كند بلكه انتظار بهبود عملکرد نيز می‌رود. گذشته از اين موارد، مالكيت اين شبكه‌ها به صورت اشتراكی بين تمام کاربران پخش شده و هيچ شخص يا شركتی نمی‌تواند سياست‌های دلخواه خود را در اين نوع شبكه‌ها اعمال کند.

اماهيچ چيزی بی‌ بها به دست نمی‌آید. نبود سرور مرکزی اگر چه  ويژگی‌های جذابی به شبکه‌های همتابه‌همتا می‌بخشد اما از طرف ديگر آن‌ها را با دشواری‌هايی نيز روبه‌رو می‌کند.  عدم وجود يك هماهنگ كننده مركزی در شبكه، انجام بسياری از امور و ارائه خدمات را  دچار مشكل می‌کند.  از يک طرف، تغيير و رفت‌وآمد زیاد کاربران از ويژگی‌های ذاتی اين شبکه‌ها است و از طرف ديگر در اين شبكه‌ها، ديگر اين گره‌های معمولی  هستند كه  عهده‌دار تمامی وظايف هستند. به همين دلیل يکی از مشکلات اصلی فراروی اين شبكه‌ها، مقابله با  تغييرات لحظه‌ای و فراهم آوردن ثبات در ارائه  خدمات در بستری از بی‌ثباتی است.
\end{editorial}

\articlesep

\begin{shortarticle}{4}{محیط مقالهٔ کوتاه}{محیط مقالهٔ کوتاه داخل مجلهٔ زی‌پرشین}{5}
\shortarticleitem{این یک تیتر کوتاه است}{ويژگی اصلی که اين معماری را متمايز کرده و در دنيای اينترنت آن‌ها در مقابل معماری قبلی شبكه‌ها قرار داده است، امكان ايجاد ارتباط مستقيم  بين كامپيوترهای مختلف بدون نياز به دخالت سرورهای قدرتمند در بين راه است.  نام‌ اين نوع معماری هم در واقع بر گرفته شده از همین  ارتباط مستقیم بين گره‌ها است.  در واقع در اين نوع شبكه‌ها اثری از سرورها نيست و تمامی گره‌های معمولی موجود در شبكه، بايد وظايفی را که قبلا بر عهده‌ی سرورها بود، خود انجام دهند. بنابراين در اين نوع معماری گره‌های معمولی در ضمن اين‌که از خدمات شبکه بهره‌مند می‌شود بايد نقش خدمت‌گزار را هم ايفا کنند . در اين نوع شبکه‌ها گره‌های معمولی به كمك روش‌ها و پروتكل‌های توزيع شده، تمامی وظايف  مسير يابی در شبكه، جستجوی منابع، امنيت شبكه و شناسايی هويت استفاده كننده‌ها و هم‌چنين مقابله با حملات احتمالی مهاجمان را بر عهده دارند.}
\shortarticleitem{یک تیتر کوتاه دیگر}{ويژگی اصلی که اين معماری را متمايز کرده و در دنيای اينترنت آن‌ها در مقابل معماری قبلی شبكه‌ها قرار داده است، امكان ايجاد ارتباط مستقيم  بين كامپيوترهای مختلف بدون نياز به دخالت سرورهای قدرتمند در بين راه است.  نام‌ اين نوع معماری هم در واقع بر گرفته شده از همین  ارتباط مستقیم بين گره‌ها است.  در واقع در اين نوع شبكه‌ها اثری از سرورها نيست و تمامی گره‌های معمولی موجود در شبكه، بايد وظايفی را که قبلا بر عهده‌ی سرورها بود، خود انجام دهند. بنابراين در اين نوع معماری گره‌های معمولی در ضمن اين‌که از خدمات شبکه بهره‌مند می‌شود بايد نقش خدمت‌گزار را هم ايفا کنند . در اين نوع شبکه‌ها گره‌های معمولی به كمك روش‌ها و پروتكل‌های توزيع شده، تمامی وظايف  مسير يابی در شبكه، جستجوی منابع، امنيت شبكه و شناسايی هويت استفاده كننده‌ها و هم‌چنين مقابله با حملات احتمالی مهاجمان را بر عهده دارند.}
\shortarticleitem{یک تیتر کوتاه دیگر}{ويژگی اصلی که اين معماری را متمايز کرده و در دنيای اينترنت آن‌ها در مقابل معماری قبلی شبكه‌ها قرار داده است، امكان ايجاد ارتباط مستقيم  بين كامپيوترهای مختلف بدون نياز به دخالت سرورهای قدرتمند در بين راه است.  نام‌ اين نوع معماری هم در واقع بر گرفته شده از همین  ارتباط مستقیم بين گره‌ها است.  در واقع در اين نوع شبكه‌ها اثری از سرورها نيست و تمامی گره‌های معمولی موجود در شبكه، بايد وظايفی را که قبلا بر عهده‌ی سرورها بود، خود انجام دهند. بنابراين در اين نوع معماری گره‌های معمولی در ضمن اين‌که از خدمات شبکه بهره‌مند می‌شود بايد نقش خدمت‌گزار را هم ايفا کنند . در اين نوع شبکه‌ها گره‌های معمولی به كمك روش‌ها و پروتكل‌های توزيع شده، تمامی وظايف  مسير يابی در شبكه، جستجوی منابع، امنيت شبكه و شناسايی هويت استفاده كننده‌ها و هم‌چنين مقابله با حملات احتمالی مهاجمان را بر عهده دارند.}
\shortarticleitem{یک تیتر کوتاه دیگر}{ويژگی اصلی که اين معماری را متمايز کرده و در دنيای اينترنت آن‌ها در مقابل معماری قبلی شبكه‌ها قرار داده است، امكان ايجاد ارتباط مستقيم  بين كامپيوترهای مختلف بدون نياز به دخالت سرورهای قدرتمند در بين راه است.  نام‌ اين نوع معماری هم در واقع بر گرفته شده از همین  ارتباط مستقیم بين گره‌ها است.  در واقع در اين نوع شبكه‌ها اثری از سرورها نيست و تمامی گره‌های معمولی موجود در شبكه، بايد وظايفی را که قبلا بر عهده‌ی سرورها بود، خود انجام دهند. بنابراين در اين نوع معماری گره‌های معمولی در ضمن اين‌که از خدمات شبکه بهره‌مند می‌شود بايد نقش خدمت‌گزار را هم ايفا کنند . در اين نوع شبکه‌ها گره‌های معمولی به كمك روش‌ها و پروتكل‌های توزيع شده، تمامی وظايف  مسير يابی در شبكه، جستجوی منابع، امنيت شبكه و شناسايی هويت استفاده كننده‌ها و هم‌چنين مقابله با حملات احتمالی مهاجمان را بر عهده دارند.}
\shortarticleitem{یک تیتر کوتاه دیگر}{ويژگی اصلی که اين معماری را متمايز کرده و در دنيای اينترنت آن‌ها در مقابل معماری قبلی شبكه‌ها قرار داده است، امكان ايجاد ارتباط مستقيم  بين كامپيوترهای مختلف بدون نياز به دخالت سرورهای قدرتمند در بين راه است.  نام‌ اين نوع معماری هم در واقع بر گرفته شده از همین  ارتباط مستقیم بين گره‌ها است.  در واقع در اين نوع شبكه‌ها اثری از سرورها نيست و تمامی گره‌های معمولی موجود در شبكه، بايد وظايفی را که قبلا بر عهده‌ی سرورها بود، خود انجام دهند. بنابراين در اين نوع معماری گره‌های معمولی در ضمن اين‌که از خدمات شبکه بهره‌مند می‌شود بايد نقش خدمت‌گزار را هم ايفا کنند . در اين نوع شبکه‌ها گره‌های معمولی به كمك روش‌ها و پروتكل‌های توزيع شده، تمامی وظايف  مسير يابی در شبكه، جستجوی منابع، امنيت شبكه و شناسايی هويت استفاده كننده‌ها و هم‌چنين مقابله با حملات احتمالی مهاجمان را بر عهده دارند.}
\shortarticleitem{یک تیتر کوتاه دیگر}{ويژگی اصلی که اين معماری را متمايز کرده و در دنيای اينترنت آن‌ها در مقابل معماری قبلی شبكه‌ها قرار داده است، امكان ايجاد ارتباط مستقيم  بين كامپيوترهای مختلف بدون نياز به دخالت سرورهای قدرتمند در بين راه است.  نام‌ اين نوع معماری هم در واقع بر گرفته شده از همین  ارتباط مستقیم بين گره‌ها است.  در واقع در اين نوع شبكه‌ها اثری از سرورها نيست و تمامی گره‌های معمولی موجود در شبكه، بايد وظايفی را که قبلا بر عهده‌ی سرورها بود، خود انجام دهند. بنابراين در اين نوع معماری گره‌های معمولی در ضمن اين‌که از خدمات شبکه بهره‌مند می‌شود بايد نقش خدمت‌گزار را هم ايفا کنند . در اين نوع شبکه‌ها گره‌های معمولی به كمك روش‌ها و پروتكل‌های توزيع شده، تمامی وظايف  مسير يابی در شبكه، جستجوی منابع، امنيت شبكه و شناسايی هويت استفاده كننده‌ها و هم‌چنين مقابله با حملات احتمالی مهاجمان را بر عهده دارند.}
\end{shortarticle}

\articlesep

\end{document}
