% A testbed for font evaluation (see The METAFONTbook, Appendix H)

%  Modified by pdc to add uppercase text.  29-Nov-89
%  Modified some more by me again for testing Malvern.

% ************************************************************************
%
%  This file should NOT, repeat  * N * O * T *  be installed in the
%  standard TeX macro file area.  Instead you should leave it in
%  the same directory as the Malvern program files.  It customizes
%  testfont for the Malvern encoding conventions.
%
%  ************************************************************************

\input malvern
\errorcontextlines=20

\tracinglostchars=0
\tolerance=1000
\raggedbottom
\nopagenumbers
\parindent=0pt
\newlinechar=`@
\hyphenpenalty=200
\doublehyphendemerits=30000

\newcount\m \newcount\n \newcount\p \newdimen\dim
\chardef\other=12

\def\today{\ifcase\month\or
  January\or February\or March\or April\or May\or June\or
  July\or August\or September\or October\or November\or December\fi
  \space\number\day, \number\year}
\def\hours{\n=\time \divide\n 60
  \m=-\n \multiply\m 60 \advance\m \time
  \twodigits\n\twodigits\m}
\def\twodigits#1{\ifnum #1<10 0\fi \number#1}

\def\init{\message{@Name of the font to test = }
  \read-1 to\fontname \startfont
  \message{Now type a test command (\string\help\space for help):}}
\def\startfont{\font\testfont=\fontname
  \leftline{\sevenrm Test of \fontname\unskip\ on \today\ at \hours}
  \medskip
  \testfont \setbaselineskip
  \ifdim\fontdimen6\testfont<10pt \rightskip=0pt plus 20pt
  \else\rightskip=0pt plus 2em \fi
  \spaceskip=\fontdimen2\testfont % space between words (\raggedright)
  \xspaceskip=\fontdimen2\testfont \advance\xspaceskip by\fontdimen7\testfont}

{\catcode`\|=0 \catcode`\\=\other
|gdef|help{|message{%
\init switches to another font;@%
\end or \bye finishes the run;@%
\text prints a sample text, assuming TeX text font conventions;@%
\TEXT prints the sample text in uppercase;@%
\display prints all the characters in the font in a nice order;@%
\sample combines \display and \text;@%
\table prints the font layout in tabular format;@%
\mixture mixes a background character with a series of others;@%
\alternation interleaves a background character with a series;@%
\alphabet prints all lowercase letters within a given background;@%
\ALPHABET prints all uppercase letters within a given background;@%
\series prints a series of letters within a given background;@%
\lowers prints a comprehensive test of lowercase;@%
\uppers prints a comprehensive test of uppercase;@%
\digits prints a comprehensive test of numerals;@%
\math prints a comprehensive test of TeX math italic;@%
\names prints a text that mixes upper and lower case;@%
\punct prints a punctuation test;@%
\bigtest combines many of the above routines;@%
\help repeats this message;@%
and you can use ordinary TeX commands (e.g., to \input a file).}}}

\def\setbaselineskip{\setbox0=\hbox{\n=0
\loop\char\n \ifnum \n<255 \advance\n 1 \repeat}
\baselineskip=6pt \advance\baselineskip\ht0 \advance\baselineskip\dp0 }

\def\setchar#1{{\escapechar-1\message{\string#1 character = }%
  \def\do##1{\catcode`##1=\other}\dospecials
  \read-1 to\next
  \expandafter\finsetchar\next\next#1}}
\def\finsetchar#1#2\next#3{\global\chardef#3=`#1
  \ifnum #3=`\# \global\chardef#3=#2 \fi}
\def\promptthree{\setchar\background
  \setchar\starting \setchar\ending}

\def\mixture{\promptthree \domix\mixpattern}
\def\alternation{\promptthree \domix\altpattern}
\def\mixpattern{\0\1\0\0\1\1\0\0\0\1\1\1\0\1}
\def\altpattern{\0\1\0\1\0\1\0\1\0\1\0\1\0\1\0\1\0}
\def\domix#1{\par\chardef\0=\background \n=\starting
  \loop \chardef\1=\n #1\endgraf
  \ifnum \n<\ending \advance\n 1 \repeat}

\def\!{\discretionary{\background}{\background}{\background}}
\def\series{\promptthree \!\doseries\starting\ending\par}
\def\doseries#1#2{\n=#1\loop\char\n\!\ifnum\n<#2\advance\n 1 \repeat}
\def\complower{\!\doseries{`a}{`z}\doseries{'31}{'34}\par}
\def\compupper{\!\doseries{`A}{`Z}\doseries{'35}{'37}\par}
\def\compdigs{\!\doseries{`0}{`9}\par}
\def\alphabet{\setchar\background\complower}
\def\ALPHABET{\setchar\background\compupper}

\def\lowers{\docomprehensive\complower{`a}{`z}{'31}{'34}}
\def\uppers{\docomprehensive\compupper{`A}{`Z}{'35}{'37}}
\def\digits{\docomprehensive\compdigs{`0}{`4}{`5}{`9}}
\def\docomprehensive#1#2#3#4#5{\par\chardef\background=#2
  \loop{#1} \ifnum\background<#3\m=\background\advance\m 1
  \chardef\background=\m \repeat \chardef\background=#4
  \loop{#1} \ifnum\background<#5\m=\background\advance\m 1
  \chardef\background=\m \repeat}

\def\names{ {\AA}ngel\aa\ Beatrice Claire
  Diana \'Erica Fran\c{c}oise Ginette H\'el\`ene Iris
  Jackie K\=aren {\L}a\u{u}\.ra Mar{\'\i}a N\H{a}ta{\l}{\u\i}e {\O}ctave
  O\^ctavia\~n
  Pauline Qu\^eneau Roxanne Sabine T\~a{\'\j}a Ur\v{s}ula
  Vivian Wendy Xanthippe Yv{\o}nne Z\"azilie\par}
\def\punct{\par\dopunct{min}\dopunct{pig}\dopunct{hid}
  \dopunct{HIE}\dopunct{TIP}\dopunct{fluff}
  \$1,234\cdot56 + 7/8 = 9\% = 90\permille\ @ \#0 2\times4 \S3 \P4 
	57\degrees\ = 57\thinspace\degrees C\par}
\def\dopunct#1%
{
    #1,\ #1:\ #1;\ `#1'\ ?`#1?\ !`#1!\ (#1)\ [#1]\ \langle#1\rangle\ 
    <#1>\ <<#1>>\ <<\thinspace#1\thinspace>>\
    #1*\ #1\dag\ #1\ddag\ #1.\par
}

\def\bigtest{\sample
  hamburgefonstiv HAMBURGEFONSTIV\par
  \names \punct \lowers \uppers \digits \table}

\def\math{\textfont1=\testfont \skewchar\testfont=\skewtrial
 \mathchardef\Gamma="100 \mathchardef\Delta="101
 \mathchardef\Theta="102 \mathchardef\Lambda="103 \mathchardef\Xi="104
 \mathchardef\Pi="105 \mathchardef\Sigma="106 \mathchardef\Upsilon="107
 \mathchardef\Phi="108 \mathchardef\Psi="109 \mathchardef\Omega="10A
 \def\ii{i} \def\jj{j}
 \def\\##1{|##1|+}\mathtrial
 \def\\##1{##1_2+}\mathtrial
 \def\\##1{##1^2+}\mathtrial
 \def\\##1{##1/2+}\mathtrial
 \def\\##1{2/##1+}\mathtrial
 \def\\##1{##1,{}+}\mathtrial
 \def\\##1{d##1+}\mathtrial
 \let\ii=\imath \let\jj=\jmath \def\\##1{\hat##1+}\mathtrial}
\newcount\skewtrial \skewtrial='177
\def\mathtrial{$\\A \\B \\C \\D \\E \\F \\G \\H \\I \\J \\K \\L \\M \\N \\O
 \\P \\Q \\R \\S \\T \\U \\V \\W \\X \\Y \\Z \\a \\b \\c \\d \\e \\f \\g
 \\h \\\ii \\\jj \\k \\l \\m \\n \\o \\p \\q \\r \\s \\t \\u \\v \\w \\x \\y
 \\z \\\alpha \\\beta \\\gamma \\\delta \\\epsilon \\\zeta \\\eta \\\theta
 \\\iota \\\kappa \\\lambda \\\mu \\\nu \\\xi \\\pi \\\rho \\\sigma \\\tau
 \\\upsilon \\\phi \\\chi \\\psi \\\omega \\\vartheta \\\varpi \\\varphi
 \\\Gamma \\\Delta \\\Theta \\\Lambda \\\Xi \\\Pi \\\Sigma \\\Upsilon
 \\\Phi \\\Psi \\\Omega \\\partial \\\ell \\\wp$\par}
\def\mathsy{\begingroup\skewtrial='060 % for math symbol font tests
 \def\mathtrial{$\\A \\B \\C \\D \\E \\F \\G \\H \\I \\J \\K \\L
  \\M \\N \\O \\P \\Q \\R \\S \\T \\U \\V \\W \\X \\Y \\Z$\par}
 \math\endgroup}


\newcount\whichtext
\newif\ifuppercase

\def\text
{{\advance\baselineskip-4pt
  \setbox0=\hbox{abcdefghijklmnopqrstuvwxyz}
  \ifdim\hsize>2\wd0 \ifdim 15pc>2\wd0 \hsize=15pc \else \hsize=2\wd0
  \fi\fi
  \parindent = 0.1\hsize \parskip = 1ex plus 2pt
  \ifuppercase \uppercase \fi {
  \ifcase\whichtext
	It isn't easy to explain what attracts some people to letterforms
	with such joy and fury.  Most people can't tell typing from printing
	and the latest figures on illiteracy suggest that we should be
	grateful, while we can, that they can read at all.

	    In comparison to the details of such other callings as, say, keeping
	rabbits, small particulars of the alphabet can easily become an
	obsession.  I often think of Thomas Codben-Sanderson, a remarkable
	man and a bookbinder of distinction, as he stood on
	Hammersmith Bridge in the dark, throwing the printing type of the
	Doves Press into the Thames.  This was a crime of passion and has a
	comical side that the decades often give to such things.  But I can
	see him, heart pounding, full of earnest purpose, as he puts his
	beloved type to rest.  He was not about to wonder, as the song
	wonders about the lady, who might be kissing her now.

	    I could name a few things I'd like to throw into the Thames
	myself in the shadow of the night.  But my reasons would not be as
	pure has his.  My motive wouldn't be love.

    \or 
	Lettering people disagree over everything.  They argue about the
	very purpose of the alphabet, which is as good a point for a fight
	as any.

	    Some people say that letters exist to be read and therefore the
	things that interfere with legibility should be discouraged.  (This
	is a bit like saying that the purpose of liquor is to get you drunk.
	If they were right, people would swig metal polish, and the farmers
	of Cognac, Burgundy and Champagne grow potatoes.)

	    The alphabet, the argument goes, should be a well-mannered
	servant who quietly carries messages from author to reader.  This is
	sometimes true: it is just what we need in a telephone directory.
	Fortunately, the world still has some other uses for letters as
	well.  We celebrate the alphabet for other reasons than legibility.
	We enjoy all the passions and the fury and could hardly have more
	fun without breaking the furniture.

    \or
	The alphabet as a style for every need.  For dignity, the style of
	Roman inscriptions is unsurpassed.  To attempt the limits of human
	skill, people have decorated letters almost beyond recognition in
	some of the world's greatest manuscript books.  More than a code for
	text, the alphabet has developed throught the centuries into a
	splendid medium of expression.  Even the greeting card with pink
	pigs from your friend Henry had a printed message that was made to
	look silly expressly to match his sense of humour.  I think it's
	marvellous and I don't think any permanent damage was done to our
	heritage.

	    As long as people seem to like letters made of soap bubbles, the
	world can't be all bad.  A few colleagues of mine have been driven
	to despair over some lovely letters with snow on top that are used
	in winter advertising.  These things disd not come about by
	accident.  No designer consciously decided to forfeit his reputation
	and go down in history wearing a fool's cap.  They were made because
	people wanted them and it is because people still do that they are
	used every year as the waether gets cold.

    \or
	Even if we agree that expression is a part of the alphabet, and agree
	again that it sometimes comes before legibility, we still have many
	things to disagree over.  One thing that divides people is apprach.
	They ask what kind of lettering is permissible in polite society.  What
	kind of expression can the common man be trusted with?

	One view, especially liked by amateurs, is that the only legitimate way
	of using the alphabet for expression is to imitate medieval methods.
	This means that you should write with a feather.  You should work on
	vellum if you can afford it, on hand-made paper if not (even if it only
	differs comparably from machine-made paper in the direction of the
	fibres).  You should approach the job with reverence and possibly make
	your own ink.  All this is said to be justified by regarding writing,
	illuminating, and lettering as medieval crafts.

	\null\hfill Gunnlaugur S. E. Briem\par

	\whichtext = -1
    \fi
    }
\global\advance\whichtext by 1
\moretext
(!`THE DAZED BROWN FOX QUICKLY GAVE 12345--67890 JUMPS!)\par}}
\def\moretext{?`But are{\ng}'t Kafka's Schlo{\ss} 
and {\AE}sop's {\OE}uvres
often na{\"\i}ve vis-\`a-vis the {\dh}{\ae}monic {\th}h{\oe}nix's official
r\^ole in fluffy souffl\'es? }
\def\omitaccents{\let\moretext=\relax}

\def\TEXT{{\uppercasetrue\text}}

\def\,{\kern 0.1667em }

\def\display
{{\par \rightskip = 0pt plus 0.25\hsize
  \displayloop{65}{91}
  \displayloop{97}{123}
  \displayloop{48}{58}
  ,.;: ` ' `` '' ()[]*+=/?!@\$\%\& - -- ---
  \displayloop{0}{16}
  \displayloop{16}{32}
  \char32\char`\" \char`\<\char`\>
  \char`\\\char`\^\char`\_
  \char`\|\char`\~\char127
  \par{\sevenrm 128+:}
  \displayloop{128}{144}
  \displayloop{144}{160}
  \displayloop{160}{176}
  \displayloop{176}{192}
  \displayloop{192}{208}
  \displayloop{208}{224}
  \displayloop{224}{240}
  \displayloop{240}{256}
  \par
}}

\def\displayloop#1#2%
{%
    \n=#1
    \loop
    \ifnum\n<#2
	\char\n
	\advance\n+1
    \repeat
}

\def\grecoroman
{{\rightskip = 0pt plus 0.25\hsize
  AB%
  \char0 C%	  Gamma
  \char1 D%	  Delta
  EFZGH\char2	% Theta
  IJK%
  \char3 L%	  Lambda
  MN\char4	% Xi
  O\char5 P%	  Pi
  QR%	
  \char6 S% 	  Sigma
  T\char7 	% Upsilon
  UVW%
  \char8 X\char9% Phi, Psi
  Y\char10
  \par
}}

\def\liggy
{{\def\\{\kern0pt}%
  ff, f\\f; fi, f\\i; fl, f\\l; ffi, f\\f\\i; ffl, f\\f\\l.
  \par Kafka often fluffy souffle official fife.
  \par Kaf\\ka of\\ten f\\luf\\fy souf\\f\\le of\\f\\icial f\\if\\e.
  \par
}}  

\def\sample{\display\TEXT\text\names}

\newcount\hexcount
\def\hexdigit#1{\ifcase#1\relax 0\or 1\or 2\or 3\or 4\or 5\or 6\or 7\or
	8\or 9\or A\or B\or C\or D\or E\or F\fi}

\def\tablecr
{%
    \cr
    \noalign{\nointerlineskip}
    \multispan2\hfill &\multispan{33}\hrulefill
    \cr
    \noalign{\nointerlineskip}
}
\def\table
{
    \def\\{\char\n \global\advance\n 1}
    \def\0##1{&\omit&\sevenrm##1}
    \halign to \hsize
      {%
	\chartstrut\hss##\tabskip=0pt plus 10pt &
	&\hss##\hss&##\vrule\cr
	\lower 6.5pt\null
        &\00\01\02\03\04\05\06\07\08\09\0A\0B\0C\0D\0E\0F
	\tablecr 
	\global\n=0
	\tablelines
      }
    \bigbreak
}

\def\tablelines
{
    \ifnum\n<256
	& \hexcount=\n \divide\hexcount16 \sevenrm\hexdigit\hexcount
	&&\\&&\\&&\\&&\\&&\\&&\\&&\\&&\\&&\\&&\\&&\\&&\\&&\\&&\\&&\\&&\\&
	\tablecr \noalign{\penalty5000 }
        \tablelines
    \fi
}
\def\chartstrut{\lower 0.25\baselineskip \vbox to \baselineskip{}}

\def\oneaccent#1%
{
  O \'#1 \`#1 \^#1 \~#1 \"#1 \v #1 \u #1 \H #1 \.#1 \=#1 \accent9 #1 
  O \par
}

\def\currencies
{
 \pounds19\cdot66 = 99\cents\ =  \$1\cdot00 = \currency66\cdot66 =
	123456\yen = \florin358 (or is that 358\florin\ or what?\par
 \pounds\char"B1\char"B9\cdot\char"B6\char"B6\ = \char"B9\char"B9\cents\
	= \$\char"B1\cdot\char"B0\char"B0\ =
	\currency\char"B6\char"B6\cdot\char"B6\char"B6\ =
	\char"B1\char"B2\char"B3\char"B4\char"B5\char"B6\yen 
	= \florin\char"B3\char"B5\char"B8\ (or is that 
	\char"B3\char"B5\char"B8\florin\ or what?\par
}

\ifx\noinit!\else\init\fi

%  insane smallcaps hacks 

\newcount\alphaone \newcount\alphatwo \newcount\alphathr
\def\doalphabet#1#2#3%
{%
    \alphaone=#2\relax \alphatwo=#3\relax \alphathr=26
    \loop
	\advance\alphathr-1
    \ifnum\alphathr>0
	#1\alphaone\alphatwo
	\advance\alphaone+1
	\advance\alphatwo+1
    \repeat
}

\def\sctext{{\doalphabet\uccode{`a}{"E1}\uc\text}}
\def\lc#1{\lowercase{#1}}
\def\mctext{{\doalphabet\lccode{`A}{"C1}\lc\text}}

\def\scnames{{\doalphabet\uccode{`a}{"E1}\uppercase\expandafter{\names}}}
\def\mcnames{{\doalphabet\lccode{`A}{"C1}\lowercase\expandafter{\names}}}

\def\accents
{{
    \leftskip=1em \parindent=-1em \def\\##1:{\par{\tenrm ##1:}}
    {\tenrm Accents according to HART'S:}
    \\Anglo-Saxon: \TH\th\ \DH\dh (not quite right) \ \AE\ae\ \OE\oe\
	\=A\=a \=E\=e \=I \=\i\ \=O\=o \=U\=u \AAEE\aaee\ \OOEE\ooee\
	{\tenrm wyn, yogh}
	{\tenrm and in textbooks:} \dotG\.g \dotC\dotc 
    \\Arabic: {\arabic \`alim, mu\`allim, \.d^a\`. \'am^ir, mu\'allim,
	\.d^a\'.  taw^ar^i_{kh} ma\`l^um} 
    \\Cyrllic languages: \'C\'c \'E\'e \"E\"e \'{} \u{I}o\u{\i} \=Y\=y 
	\v{Z}\v{z}
    \\Czech: \'A\'a \hacekC\hacekc\ {\hacekD}d' \'E\'e \v{E}\v{e} \'I\'\i\
	\v{N}\v{n} \'O\'o \v{R}\v{r} \v{S}\v{s} \v{T}t' \'U\'u \@U\@u
	\'Y\'y \v{Z}\v{z}
    \\Danish: \AE\ae\ \O\o\ \AA\aa
    \\English: \'E\'e \"I \"\i\ 
    \\Esperanto: {\esperanto ^Cu ^ci tia ^Genevulo ^guas ^Hano^h ^Jones
	kaj la ^jipo de ^Sim^son kiu ^Uaas kaj ba^umas?}
    \\Finnish: \"A\"a \"O\"o \AA\aa
    \\French: \OE\oe\ \`A\`a \^A\^a \c{C}\c{c} \'E\'e \`E\`e \^E\^e
	\"E\"e \^I \^\i\ \"I \"\i\ \^O\^o \`U\`u \^U\^u \"U\"u
    \\Gaelic (Irish): \'A\'a \'E\'e \'I\'\i\ \'O\'o \'U\'u
    \\Gaelic (Scots): \`A\`a \`E\`e \'E\'e \'I\'\i\ \`O\`o \'O\'o \'U\'u
    \\German: {\german "S"s "A"a "O"o "U"u}
    \\Hungarian: \'A\'a \'E\'e \'I\'\i\ \'O\'o \"O\"o \H{O}\H{o} \'U\'u
	\"U\"u \H{U}\H{u}
    \\Icelandic: \TH\th\ \DH\dh\ \AE\ae\ \OE\oe\ \'A\'a \'E\'e
	\'I\'\i\ \'O\'o \"O\"o \'U\'u \'Y\'y
    \\Norwegian: \AE\ae\ \O\o\ \AA\aa
    \\Polish: \ogonek{A}\ogonek{a} \'C\'c \ogonek{E}\ogonek{e} 
	\L\l\ \'N\'n \'O\'o \'S\'s \'Z\'z \.Z\.z
    \\Portuguese: \`A\`a \'A\'a \^A\^a \~A\~a \c{C}\c{c} \`E\`e \'E\'e
	\^E\^e \`Io\`\i\ \'Io\'\i\ \`O\`o \'O\'o \^O\^o \~O\~o \'U\'u
	\`U\`u
    \\Romanian: \`A\`a \^A\^a \u{A}\u{a} \`E\`e \`Io\`\i\ \^Io\^\i\
	\c{S}\c{s} \c{T}\c{t} \`U\`u
    \\Spanish: \'A\'a \'E\'e \'Io\'\i\ \~N\~n \'O\'o \'U\'u
    \\Swedish: \AA\aa\ \"A\"a \"O\"o
    \\Turkish: \^A\^a \c{C}\c{c} \u{G}\u{g} \.Ii I\i\ \^Io\^\i\ \"O\"o
	\c{S}\c{s} \"U\"u \^U\^u
    \\Welsh: \^A\^a \'A\'a \^E\^e \"E\"e \^Io\^\i\ \^O\^o \"O\"o \^U\^u
	\^W\^w \^Y\^y
    \par
}}


\def\greekaccents
{
    \n=12
    \loop \ifnum \n<32
	\accent\n a\accent\n e\accent\n i%
	\char\n A\char\n E\char\n I%
	\space
	\advance\n1
    \repeat
}


\def\corkquote#1#2{#1min#2 #1pig#2 #1hid#2 #1HIE#2 #1TIP#2 #1fluff#2 \par}
\def\corkpunct
{
   	\corkquote{\char34 }{\char34 }
	\corkquote{`}{'} \corkquote{``}{''}
	\corkquote{,}{`} \corkquote{,,}{``}
	\corkquote{\char14 }{\char15 } \corkquote{<<}{>>}
	\corkquote{\char15 }{\char14 } \corkquote{>>}{<<}
	\corkquote{\char123 }{\char125 }
}
\def\corkascii	% various computer language symbols using ASCII chars
{{
    \let\\=\corkbinary f(x) a[i] p.q !x \ \char`\~a \ a++ \ b--\/-- \
	\&a \ \\* \\/ \\\% \\< \\{<\/<} \\{<=} \\> \\{>\/>} \\{>=}
	\\{==} \\{!=} \\{\char`\~=} \\{/=} \\{<>} \\{\#} \\{@} \\{+} 
	\\{\char`\\}
	\\{--} \\{=} \\{\&} \\{\&\&} \\| \\{||} \\{\char`\^} \\{\char`\^.}
	\\{\char`\~} \\{+=} \\{--=} \\{*=} \\{/=} \\{\%=} \\{\&=} \\{:=}
	\\{::=} \\{:==} \\{<--} \\{-->} \\{\char`\^=} \\{|=} \\{<\/<=}
	\\{>\/>=} \par
}}
\def\corkbinary#1{x #1\ y \ x#1y \ }

\def\corkstuff{\corkpunct \corkascii}
