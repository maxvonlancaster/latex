\documentclass[a4paper]{article}
%
% This is a font sampler for the LaTeX distribution. Most of the fonts
% shown here are either installed by default or can be downloaded from
% the Comprehensive TeX Archive Network (CTAN: www.ctan.org). It omits
% a few of the experimental CM-based fonts like cmff.
%
% Some of the fonts needed considerable massaging to make them work
% with LaTeX. This was surprising and disappointing, especially when
% the authors of a font package claimed it already worked with LaTeX,
% but the materials as distributed from CTAN did not perform as
% expected. In particular, Metafont fonts which were originally done
% for use with plain TeX needed font_size 10pt#; adding to the
% mode_setup, for example with the Backgammon font; and numerous .fd 
% files needed rewriting or creating, especially for the free fonts 
% from the X Consortium members. Some of these problems have now been
% fixed: see how you get on.
%
% This sampler is distributed under the terms of the GNU GPL.
% Copyleft 2000 by Silmaril Consultants. Code by Peter Flynn.
% See http://www.fsf.org/licenses/gpl.html for details. Briefly:
% (1) you may freely use, copy, distribute and change this file for
% any purpose; (2) you may not prevent others from doing likewise; 
% (3) a similar condition, including this condition, is imposed 
% on all subsequent users.
%
%%%%%%%%%%%%%%%%%%%%%%%%%%%%%%%%%%%%%%%%%%%%%%%%%%%%%%%%%%%%%%%%%%%%%%%
%
% Smaller margins all round: this is graphics, not text.
%
\usepackage[top=15mm,bottom=10mm,left=15mm,right=10mm,nohead,nofoot]{geometry}
%
% We need graphicx to handle the two EPS math figures (concrete and
% euler) as I haven't figured out how to have more than one set of
% math encodings in a single file (nor am I sure one would want to
% try). Color because it looks prettier and might impress the suits.
% Array so we can do some stuff in tabular mode. The remainder are
% font packages included because they either define a bunch of useful
% names, or because they have macros which do essential things.
%
\usepackage{graphicx,color,array,tengwar,tipa,bbding,textcomp}
%
% Some of my input uses ISO 8859-1 characters
\usepackage[latin1]{inputenc}
%
% Fix some nice shades, but use CMYK so pdflatex can grok them
%
\definecolor{Headings}{cmyk}{0,1,0.13,0}%{named}{RubineRed}
\definecolor{Subheadings}{cmyk}{0.64,0,0.95,0.40}%{named}{OliveGreen}
\definecolor{Fonts}{cmyk}{0.50,1,0,0}%{named}{Plum}
%
% Use a homebrew zmf.fd which allows \mf\ at standard TeX mags
% Someone else has probably done an ``official'' one for CTAN by now.
%
\newcommand{\mf}{{\fontencoding{U}\fontfamily{zmf}\selectfont METAFONT}}
%
% A strut to fend off stuff in the previous line
%
\newcommand{\pcstrut}{\vrule height11pt width0pt}
%
% The default sample text. Trite but useful. Most of the non-Antiqua
% fonts use a motto chosen for the style where possible...
%
\newcommand{\sample}{Typographia Ars Artium Omnium Conservatrix}
%
% Two macros to do the formatting. \thefont specifies a line in the
% table with a default encoding, using a font family name following
% the Karl Berry fontname conventions (breached only for some older
% well-established Metafont fonts). \fonttitle just spans both columns
% with a title and some vertical spacing.
%
\newcommand{\thefont}[4][OT1]{%
	\textcolor{Fonts}{#2}&%
	\pcstrut\fontencoding{#1}\fontfamily{#3}\selectfont#4\\}
\newcommand{\fonttitle}[1]{%
	\multicolumn2{p{\columnwidth}}{\vrule height1.5pc width0pt
	\fontseries{b}\selectfont\textcolor{Subheadings}{#1}}\\[3pt]}
\newcommand{\normalencoding}{\fontencoding{OT1}}
%
%%%%%%%%%%%%%%%%%%%%%%%%%%%%%%%%%%%%%%%%%%%%%%%%%%%%%%%%%%%%%%%%%%%%%%%%%%%%%%%%
%
\begin{document}
\pagestyle{empty}
\subsection*{\textsf{\color{Headings}Typefaces that come with the
standard \LaTeX\ distribution on the \TeX\ Live CD-ROM}}
\raggedright
\begin{tabular}{@{}>{\sffamily\bfseries}rl}
%
% CM gets first place as the default font
%
\fonttitle{Computer Modern (CM), \LaTeX's default typeface}
%
\thefont{CM Roman}{cmr}{\sample}
\thefont{CM Italic}{cmr}{\itshape\sample}
\thefont{CM Slanted (Oblique)}{cmr}{\slshape\sample}
\thefont{CM Bold}{cmr}{\fontseries{b}\selectfont\sample}
\thefont{CM Bold Extended}{cmr}{\bfseries\sample}
\thefont{CM Bold Italic}{cmr}{\itshape\bfseries\sample}
\thefont{CM Bold Slanted}{cmr}{\slshape\bfseries\sample}
\thefont{CM Caps \& Small Caps}{cmr}{\scshape\sample}
\thefont{CM Sans-Serif}{cmss}{\sample}
\thefont{CM Sans-Serif Oblique}{cmss}{\itshape\sample}
\thefont{CM Sans-Serif Bold}{cmss}{\bfseries\sample}
\thefont{CM Typewriter}{cmtt}{\sample}
\thefont{CM Typewriter Italic}{cmtt}{\itshape\sample}
\thefont{CM Typewriter Bold}{cmtt}{\bfseries\sample}
\thefont{CM Typewriter C\&SC}{cmtt}{\scshape\sample}
\thefont[OMS]{CM Mathematics}{cmsy}{$E=mc^2$\qquad
$b_4i\sqrt{u}{ru\over{14}}qt\pi$\qquad
$\bar\Phi\subset NL_1^*/N=\bar L_1^*\subseteq\cdots\subseteq NL_n^8/N=\bar L_n^*$}
\thefont{CM `Dunhill'}{cmdh}{\sample}
\thefont{CM `Fibonacci'}{cmfib}{\sample}
%
% Next the PostScript fonts
%
\fonttitle{The Adobe `LaserWriter 35', 10 typefaces in a total of 35
different styles, standard on all PostScript printers}
% Drop Avant Garde down a bit, 'cos it's a big x-height
\thefont{Avant Garde Book}{pag}{\fontsize{9}{9}\selectfont\sample}
\thefont{Bookman Light}{pbk}{\sample}
\thefont{Courier}{pcr}{\sample}
\thefont{Helvetica}{phv}{\sample}
\thefont{New Century Schoolbook}{pnc}{\sample}
\thefont{Palatino}{ppl}{\sample}
\thefont[U]{Symbol}{psy}{\sample}
\thefont{Times New Roman}{ptm}{\sample}
% Bump the size up a bit for Chancery, it's so small
\thefont{Zapf Chancery Medium Italic}{pzc}{\fontsize{12}{12}\selectfont\itshape\sample}
\thefont[U]{Zapf Dingbats}{pzd}{\sample}
%
% Other free PS fonts. Most of these were broken. Either the .fd file
% was incomplete or simply junk, or the psfonts.map entry referred to
% a non-existent (obsolete?) .pfa file, or the .pfb filename was
% wrong, or the Berry fontname in psfonts.map was incorrect for the
% virtual font used. Can someone please fix these in the TL distro?
%
\fonttitle{Fonts contributed by members of the X Consortium}
%
% Recent distributions seem to have messed around withe the .fd file
% for put, with the result that the defs for putr7t are not supplied,
% so I have included my .tfm and .vf files in this distro in the hope
% they will work.
\thefont{Utopia Regular}{put}{\sample}
\thefont{Charter BT}{bch}{\sample}
\thefont{Nimbus Roman}{unm}{\sample}
\thefont{Nimbus Sans}{unms}{\sample}
% uaq and ugq appear to have been updated recently, and the
% definitions for {m}{n} replaced by {m}{ui}, which looks entirely
% spurious to me, so I have included my (older) .fd files in this
% distro. If it fails for you, try adding \fontshape{ui}\selectfont 
% immediately before \sample and using your distro's .fd files.
\thefont{URW Antiqua}{uaq}{\sample}
\thefont{URW Grotesk}{ugq}{\sample}
%
% Some of the MF text extras. There must be more. Universal looks like
% something out of a 1950s mainframe manual.
%
\fonttitle{Additional \mf\ fonts}
%
\thefont{Pandora}{panr}{\sample}
\thefont{Universal}{uni}{\sample}
\thefont{Punk}{punk}{\sample}
\thefont{Concrete}{ccr}{\sample}
%
% The additional math was set from the same formulae as for CM, but
% done in a separate file, and the bounding box edited manually in the
% PS file.
%
\thefont{Concrete Math}{ccr}{\raisebox{-8pt}{\includegraphics{concmath}}}[-4pt]
\thefont[U]{Euler Math}{eus}{\raisebox{-8pt}{\includegraphics{euler}}}
\end{tabular}
%
% Space for a small comment
%
\par\vfill\begin{footnotesize}
GNU Copyleft \textcopyleft\ 2003 Silmaril Consultants. You may freely
copy this document but you may not prevent others from doing
likewise. It is not possible, in a sampler this size, to show every
possible variant of every typeface. In addition to those shown here
there are all the standard italic and bold forms for most of the
fonts, and a number of experimental \mf\ fonts as well, whose
discovery is left as an exercise to the reader. Hebrew, Greek,
Cyrillic, Arabic, Coptic, 
Devanagari, Japanese, and many others have been omitted because I
lack sufficient information on how to include them in a document
which uses principally Latin character encodings. I would be
interested to hear from experts in these areas about how best to
include them. See the source code for further comments. \textsl{PF}.
\par\end{footnotesize}
%
\newpage
%
% Now the rest. TeX Live 4 installed most of these if you picked the
% `recommended' option, but the whole things took nearly 500Mb... 
%
\subsubsection*{\textsf{Additional free fonts for use with \LaTeX}}
%
\begin{tabular}{@{}>{\sffamily\bfseries}rl}
%
% The oldies first because they look impressive
%
\fonttitle{\textit{The Historical Collection}}
%
\thefont{Cypriot}{cypr}{\fontsize{7}{7}\selectfont\sample}
% eiad needs a little more size
\thefont{�\i reannach}{eiad}{\fontsize{12}{12}\selectfont 
	N�l aon tinte�n mar do thinte�n f�in}
\thefont{Etruscan}{etr}{\sample}
\thefont{Linear `B'}{linb}{\fontsize{8}{8}\selectfont\sample}
\thefont{Phoenician}{phnc}{\sample}
\thefont{Runic}{fut}{TYPOGRAPHIA ARS ARTIUM OMNIUM CONSERVATRIX}
\thefont{Rustic}{rust}{\sample}
% dvips says bard ``at 600 dpi has most likely been made improperly;''
\thefont[U]{Bard}{zba}{\sample}
\thefont{Uncial}{uncl}{\sample}[-3pt]
\thefont{D�rer}{zdu}{ABCDEFGHIJKLMNOPQRSTUVWXYZ}[3pt]
%
% The more complex examples are taken from elsewhere...
%
% I need a good pair of Ogham inscriptions here
%
\thefont[U]{Ogham}{ogham}{\raisebox{-3pt}{\small 
ABHMOLDGUVTJESCZINQR}\hspace*{5pc}\rlap{\smash{\begin{tabular}{c}%
a\\o\\u\\e\\i\\b\\l\\v\\s\\n\\h\\d\\t\\c\\q\\m\\g\\j\\z\\r\end{tabular}}}}
%
% And a good piece of cuneiform...
%
\thefont[U]{Ugaritic Cuneiform}{ugar}{\sample}
%
% Symbols are a little harder. These are mostly selections designed to
% show the nice bits :-)
%
\fonttitle{\textit{The Symbol Collection}}
%
% Chessboard, split horizontally, just showing the pieces. A real
% chess problem would be nice, but would take up too much vertical
% space. 
%
\thefont[U]{Chess}{zch}{\fontsize{48}{12}\selectfont\setlength{\tabcolsep}{0pt}%
\begin{tabular}{|cccccccc|c|cccccccc|}\cline{1-8}\cline{10-17}
s&n&a&k&l&b&m&r&\hspace*{2pt}\raisebox{-6pt}{Z}\hspace*{2pt}&O&P&O&P&O&P&O&P\\[-3pt]
p&o&p&o&p&o&p&o&&R&M&B&J&Q&A&N&S\\[-3pt]\cline{1-8}\cline{10-17}\end{tabular}% cont'd...
%
% Backgammon symbols, running on from the Chess example
%
% Instead of starting a new line, fake it by hand
% \thefont[U]{Backgammon}{zbg}
\hspace*{2pc}%
{\normalencoding\normalfont\normalsize\sffamily\bfseries\textcolor{Fonts}{Backgammon}}%
\hspace*{1pc}%
% Have to do this as a table as well, to show an example layout
% Don't need to repeat the \tabcolsep
\qquad%\setlength{\tabcolsep}{0pt}
% Do need to establish the font
\fontencoding{U}\fontfamily{zbg}
% The MF file needed font_size 10pt#; adding, and it's a set of BIG
% glyphs, so make them really small for this demo
\fontsize{2}{2}\selectfont
% Six columns, one LH quarter of a Backgammon board
\raisebox{3pt}{\begin{tabular}{cccccc}
% I don't know why: it needs the interline gap suppressing
% Numerals are in the font and can be superimposed with \llap
e&j&e&j&4&j\\[-.2ex]
d&i&d&D&d&i\\[-.2ex]
c&h&2&h&c&h\\[-.2ex]
b&B&b&g&b&g\\[-.2ex]
0&f\llap{\char'212}&%
a\llap{\char'213}&%
f\llap{\char'214}&%
a\llap{\char'215}&%
f\llap{\char'216}\end{tabular}}}
% The astrosym font has a checksum mismatch in the distro .tfm file.
\thefont[U]{Astronomical}{zas}{\fontsize{48}{48}\selectfont defghijklmnopqrstuvwxyz}
% The karta font gives a Strange Turning Path when Metafont runs.
% Dunno why but it happens sometimes with symbols.
\thefont[U]{Cartography}{zka}{A\thinspace B\thinspace C\thinspace
D\thinspace E\thinspace F\thinspace G\thinspace H\thinspace
I\thinspace J\thinspace K\thinspace L\thinspace M\thinspace
N\thinspace O\thinspace P\thinspace Q\thinspace R\thinspace
S\thinspace T\thinspace U\thinspace V\thinspace W\thinspace
X\thinspace Y\thinspace Z\thinspace a\thinspace b\thinspace
c\thinspace d\thinspace e\thinspace f\thinspace g\thinspace
h\thinspace i\thinspace j\thinspace k\thinspace
m\thinspace n\thinspace q\thinspace s}
%
\thefont[U]{BB Dingbats}{ding}{\ScissorRight\ \ScissorHollowRight\
\HandRight\ \HandPencilLeft\ \PencilRight\ \NibRight\ \XSolidBrush\
\CrossMaltese\ \DavidStar\ \JackStar\ \FiveStarLines\ A B C D E F G 
J K L M \FiveFlowerOpen\ \Snowflake\ \CircleShadow\ \SquareShadowBottomRight\
\SquareCastShadowTopLeft\ \TriangleDown\ \Phone\ \PhoneHandset\ \Tape\
\Plane\ \Envelope\ \Peace}
%
\thefont[U]{Barcodes}{zbc}{\fontsize{10}{10}\selectfont\sample}
%
\thefont[U]{Logic symbols}{zms}{\fontsize{3}{3}\selectfont \char1
\kern6pt\char2 \kern6pt\char3 \kern6pt\char4 \kern6pt\char5
\kern6pt\char6 \kern6pt\char7 \kern6pt\char8 \kern6pt\char9
\kern6pt\char10 \kern6pt\char11 \kern6pt\char12 \kern6pt\char13
\kern6pt\char14 \kern6pt\char15 \kern6pt\char16 \kern6pt\char17
\kern6pt\char18 \kern6pt\char19 \kern6pt\char20 \kern6pt\char21
\kern6pt\char22 \kern6pt\char23 \kern6pt\char24}
%
\thefont[U]{Genealogy}{zgn}{\char'031\ \char'032\ \char'061\ \char'142\
\char'144\ \char'154\ \char'155}
%
\fonttitle{\textit{The Non-Latin Collection}}
%
\thefont[T3]{IPA}{cmr}{\textipa{ABCDEFGHIJKLMNOPQRSTUVWXYZg}\ 
\textsf{\textipa{ABCDEFGHIJKLMNOPQRSTUVWXYZg}}}
%
%\thefont{Greek}{}{}
% 
\thefont[OT1]{Quenya}{ztg}{\tengwar % Inscription on The One Ring
\a\sh\ n\a\z g d\u Rb\a t\u l\uu k,\
\a\sh\ n\a\z g g\i{\m b}\a t\u l;\
\a\sh\ n\a\z g Tr\a k\a t\u l\uu k,\
% TeX 3.14159 (Web2C 7.3.7x) with LaTeX2e <2001/06/01> complains
% that You can't use a prefix with `end-group character }' in the
% next line. I don't know why: my older TeX 3.14159 (Web2C 7.3.1)
% with LaTeX2e <2000/06/01> works just fine here. Perhaps someone
% can find the bug (it's in tengwar.sty in the \def of \i or \m)
\a g b\u Rz\u m\i\sh\'i kr\i{\m p}\a t\u l}
%
\thefont[U]{Cree/Inuktitut}{zci}{\raisebox{-8pt}{\fontsize{20}{20}\selectfont\sample}}
%
% Yannis's wonderful selection.
%
\fonttitle{\textit{The Other Scripts Collection}}
%
\thefont{Calligraphic}{zca}{\fontsize{15}{15}\selectfont\sample}
%
% The quote is the last four lines of Goethe's _Faust_ (Part II)
%
\thefont[U]{Fraktur}{yfrak}{%
	Alle\char'215\ Verg\"angliche ist nur ein Gleichni\char'032~/ 
	Da\char'215\ Unzul\"angliche hier wird'\char'215\
	Ereigni\char'215;}
\thefont[U]{Schwabacher}{yswab}{%
	Da\char'215\ Unbeschreibliche hier wird'\char'215\ getan~/ 
	Da\char'215\ Ewig-Weibliche zieht un\char'215\ hinan!}
%
% The quote is the start of Caxton's first printed advertisement
%
\thefont[U]{`Gothic'}{ygoth}{If it plese ony man spirituel or temporel
to bye any pye\char'140\ of two and thre comemoraci\~o\char'140}[6pt]
%
% 6pt extra space to make it look right
%
\thefont[U]{Decorative Initials}{yinit}{\fontsize{8}{8}\selectfont
\raisebox{-12pt}{HARALAMBOUS}}
%
\end{tabular}
\end{document}




