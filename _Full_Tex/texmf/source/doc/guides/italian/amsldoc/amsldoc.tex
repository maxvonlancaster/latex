%%% ====================================================================
%%% amsldoc.tex 2.07
%%% ====================================================================
\documentclass[a4paper,leqno,titlepage,openany]{amsldoc}[1999/12/13]
\usepackage[italian]{babel}
\renewcommand{\errexa}{\par\noindent\textit{Esempio}:\ }
\renewcommand{\errexpl}{\par\noindent\textit{Spiegazione}:\ }
\DeclareRobustCommand{\cls}[1]{{\ntt#1}%
  \autoindex{#1@\string\cls{#1}, classe}}
\DeclareRobustCommand{\pkg}[1]{{\ntt#1}%
  \autoindex{#1@\string\pkg{#1}, pacchetto}}
\DeclareRobustCommand{\opt}[1]{{\ntt#1}%
  \autoindex{#1@\string\opt{#1}, opzione}}
\DeclareRobustCommand{\env}[1]{{\ntt#1}%
  \autoindex{#1@\string\env{#1}, ambiente}}
\DeclareRobustCommand{\fn}[1]{{\ntt#1}%
   \autoindex{#1@\string\fn{#1}}}
\DeclareRobustCommand{\bst}[1]{{\ntt#1}\autoindex{#1@{\string\ntt{}#1,
  stile bibliografico}}}

\ifx\UndEfiNed\url
  \ClassError{amsldoc}{%
    This version of amsldoc.tex must be processed\MessageBreak
    with a newer version of amsldoc.cls (2.02 or later)}{}
\fi

\title{Manuale utente per il pacchetto \pkg{amsmath} (versione~2.0)}
\author{American Mathematical Society}
\date{13/12/1999}

%    Use the amsmath package and amscd package in order to print
%    examples.
\usepackage{amsmath}
\usepackage{amscd}
% Inserito il pacchetto makeidx - GD
\usepackage{makeidx}

\makeindex % generate index data
\providecommand{\see}[2]{\textit{vedi} #1}

%    The amsldoc class includes a number of features useful for
%    documentation about TeX, including:
%
%    ---Commands \tex/, \amstex/, \latex/, ... for uniform treatment
%    of the various logos and easy handling of following spaces.
%
%    ---Commands for printing various common elements: \cn for command
%    names, \fn for file names (including font-file names), \env for
%    environments, \pkg and \cls for packages and classes, etc.

%    Many of the command names used here are rather long and will
%    contribute to poor linebreaking if we follow the \latex/ practice
%    of not hyphenating anything set in tt font; instead we selectively
%    allow some hyphenation.
\allowtthyphens % defined in amsldoc.cls

\hyphenation{ac-cent-ed-sym-bol add-to-counter add-to-length align-at
  aligned-at allow-dis-play-breaks ams-art ams-cd ams-la-tex amsl-doc
  ams-symb ams-tex ams-text ams-xtra bmatrix bold-sym-bol cen-ter-tags
  eqn-ar-ray idots-int int-lim-its latex med-space neg-med-space
  neg-thick-space neg-thin-space no-int-lim-its no-name-lim-its
  over-left-arrow over-left-right-arrow over-right-arrow pmatrix
  qed-sym-bol set-length side-set small-er tbinom the-equa-tion
  thick-space thin-space un-der-left-arrow un-der-left-right-arrow
  un-der-right-arrow use-pack-age var-inj-lim var-proj-lim vmatrix
  xalign-at xx-align-at}

%    Prepare for illustrating the \vec example
\newcommand{\vect}[1]{\mathbf{#1}}

\newcommand{\booktitle}[1]{\textit{#1}}
\newcommand{\journalname}[1]{\textit{#1}}
\newcommand{\seriesname}[1]{\textit{#1}}

%    Command to insert and index a particular phrase. Doesn't work for
%    certain kinds of special characters in the argument.
\newcommand{\ii}[1]{#1\index{#1}}

\newcommand{\vstrut}[1]{\vrule width0pt height#1\relax}

%    An environment for presenting comprehensive address information:
\newenvironment{infoaddress}{%
  \par\topsep\medskipamount
  \trivlist\centering
  \item[]%
  \begin{minipage}{.7\columnwidth}%
  \raggedright
}{%
  \end{minipage}%
  \endtrivlist
}

\newenvironment{eqxample}{%
  \par\addvspace\medskipamount
  \noindent\begin{minipage}{.5\columnwidth}%
  \def\producing{\end{minipage}\begin{minipage}{.5\columnwidth}%
    \hbox\bgroup\kern-.2pt\vrule width.2pt%
      \vbox\bgroup\parindent0pt\relax
%    The 3pt is to cancel the -\lineskip from \displ@y
    \abovedisplayskip3pt \abovedisplayshortskip\abovedisplayskip
    \belowdisplayskip0pt \belowdisplayshortskip\belowdisplayskip
    \noindent}
}{%
  \par
%    Ensure that a lonely \[\] structure doesn't take up width less than
%    \hsize.
  \hrule height0pt width\hsize
  \egroup\vrule width.2pt\kern-.2pt\egroup
  \end{minipage}%
  \par\addvspace\medskipamount
}

%    The chapters are so short, perhaps we shouldn't call them by the
%    name `Chapter'. We make \chaptername read an argument in order to
%    remove a following \space or "{} " (both possibilities are present
%    in book.cls).

\renewcommand{\chaptername}[1]{}
\newcommand{\chapnum}[1]{\mdash #1\mdash }
\makeatletter
\def\@makechapterhead#1{%
  \vspace{1.5\baselineskip}%
  {\parindent \z@ \raggedright \reset@font
    \ifnum \c@secnumdepth >\m@ne
      \large\bfseries \chapnum{\thechapter}%
      \par\nobreak
      \vskip.5\baselineskip\relax
    \fi
    #1\par\nobreak
    \vskip\baselineskip
  }}
\makeatother

%    A command for ragged-right parbox in a tabular.
\newcommand{\rp}{\let\PBS\\\raggedright\let\\\PBS}

%    Non-indexed file name
\newcommand{\nfn}[1]{\texttt{#1}}

%    For the examples in the math spacing table.
%%\newcommand{\lspx}{\mbox{\rule{5pt}{.6pt}\rule{.6pt}{6pt}}}
%%\newcommand{\rspx}{\mbox{\rule[-1pt]{.6pt}{7pt}%
%%  \rule[-1pt]{5pt}{.6pt}}}
\newcommand{\lspx}{\mathord{\Rightarrow\mkern-1mu}}
\newcommand{\rspx}{\mathord{\mkern-1mu\Leftarrow}}
\newcommand{\spx}[1]{$\lspx #1\rspx$}

%    For a list of characters representing document input.
\newcommand{\clist}[1]{%
  \mbox{\ntt\spaceskip.2em plus.1em \xspaceskip\spaceskip#1}}

%    Fix weird \latex/ definition of rightmark.
\makeatletter
\def\rightmark{\expandafter\@rightmark\botmark{}{}}
%    Also turn off section marks.
\let\sectionmark\@gobble
\renewcommand{\chaptermark}[1]{%
  \uppercase{\markboth{\rhcn#1}{\rhcn#1}}}
\newcommand{\rhcn}{\thechapter. }
\makeatother

%    Include down to \section but not \subsection, in toc:
\setcounter{tocdepth}{1}

\DeclareMathOperator{\ix}{ix}
\DeclareMathOperator{\nul}{nul}
\DeclareMathOperator{\End}{End}
\DeclareMathOperator{\xxx}{xxx}

\begin{document}

%%%%%%%%%%%%%%%%%%%%%%%%%%%%%%%%%%%%%%%%%%%%%%%%%%%%%%%%%%%%%%%%%%%%%%%%
\frontmatter

\maketitle
%
%
%
\pagebreak
\begin{small} 
 \noindent Titolo originale: \emph{User manual for the \pkg{amsmath} package (version~2.0)}

 \smallskip
 \noindent Traduzione:

 \begin{quote}
 \flushleft %  \footnotesize
 Giulio Agostini, % <giulio.agostini@bigfoot.com>
 Giuseppe Bilotta, % <bourbaki@bigfoot.com>
 Flavio Casadei Della Chiesa, % <flavio_c@libero.it>
 Onofrio de Bari, % <thufir@tin.it>
 Giacomo Delre, % <giader@penguinpowered.com>
 Luca Ferrante, % <ironluke@split.it>
 Tommaso Pecorella, % <t.pecorella@inwind.it>
 Mileto Rigido, % <m.rigido@flashnet.it>
 Roberto Zanasi. % <roberto.zanasi@libero.it>
 \end{quote}

\end{small}
%
%
%
\pagestyle{headings}
\tableofcontents
\cleardoublepage % for better page number placement

%%%%%%%%%%%%%%%%%%%%%%%%%%%%%%%%%%%%%%%%%%%%%%%%%%%%%%%%%%%%%%%%%%%%%%%%
\mainmatter
%%%%%%%%%%%%%%%%%%%%%%%%%%%%%%%%%%%%%%%%%%%%%%%%%%%%%%%%%%%%%%%%%%%%%%%%

%%% Nota dei traduttori
\subsubsection*{Nota alla traduzione italiana}
Una copia di questo documento e altre traduzioni in italiano di
manuali su \LaTeX\ sono reperibili presso
\begin{itemize}
\item\url{http://guild.prato.linux.it}
\item\url{ftp://lorien.prato.linux.it/pub/guild}
\item\url{ftp://ftp.unina.it/pub/TeX/info/italian}
\end{itemize}
e su ogni sito CTAN, per esempio \url{ftp://ftp.tex.ac.uk/tex-archive/info/italian}.
%%%%%%%%%%%%%%%%%%%%%%%%%%%%%%%%%%%%%%%%%%%%%%%%%%%%%%%%%%%%%%%%%%%%%%%%


% by GA
\chapter{Introduzione}

Il pacchetto \pkg{amsmath} \`e un pacchetto \LaTeX{} che fornisce
svariate estensioni per il miglioramento della struttura informativa e
della stampa di documenti che contengono formule matematiche. I lettori
che non conoscono \LaTeX{} sono invitati a consultare \cite{lamport}.
Se si possiede una versione aggiornata di \LaTeX{}, il pacchetto \pkg{amsmath}
\`e normalmente incluso. Quando viene pubblicata una nuova versione del
pacchetto \pkg{amsmath}, \`e possibile effettuare un aggiornamento attraverso
\url{http://www.ams.org/tex/amsmath.html} o
\url{ftp://ftp.ams.org/pub/tex/}.

Questo documento descrive le funzionalit\`a del paccheto \pkg{amsmath}
e spiega come dovrebbero essere usate. Esso copre inoltre alcuni pacchetti
ausiliari:
\begin{ctab}{ll}
\pkg{amsbsy}& \pkg{amstext}\\
\pkg{amscd}& \pkg{amsxtra}\\
\pkg{amsopn}
\end{ctab}
Tutti questi hanno a che vedere con il contenuto di formule
matematiche. Per informazioni su ulteriori simboli e \emph{font} matematici,
si veda \cite{amsfonts} e \url{http://www.ams.org/tex/amsfonts.html}.
Per la documentazione del pacchetto \pkg{amsthm} o delle classi AMS
(\cls{amsart}, \cls{amsbook}, etc.\@) si veda \cite{amsthdoc} o
\cite{instr-l} e \url{http://www.ams.org/tex/author-info.html}.

Se siete utenti di \latex/ da molto tempo e avete molta matematica nei
vostri scritti, potreste trovare soluzioni a problemi familiari in
questo elenco di funzionalit\`a di \pkg{amsmath}:
\begin{itemize}

\item Un modo comodo per definire un nuovo comando `nome di operatore', come
\cn{sin} e \cn{lim}, con spazi appropriati ai lati e selezione automatica
di stile e dimensioni corrette del \emph{font} (anche quando usato in esponenti
o deponenti).

\item Diversi alternative all'ambiente \env{eqnarray} per rendere le
diverse disposizioni delle equazioni pi\`u facili da scrivere.

\item I numeri delle equazioni si spostano automaticamente in alto o in
basso per evitare di sovrapporsi con l'equazione stessa (al contrario
di \env{eqnarray}).

\item Gli spazi attorno ai segni di uguaglianza sono gli stessi della
normale spaziatura nell'ambiente \env{equation} (al contrario di
\env{eqnarray}).

\item Un modo per produrre deponenti a pi\`u linee come spesso \`e
richiesto dai simboli di sommatoria e produttoria.

\item Un modo semplice di numerare una determinata equazione con un
riferimento diverso da quello fornito dalla numerazione automatica.

\item Un modo semplice di produrre numerazioni subordinate per le
equazioni, nella forma (1.3a) (1.3b) (1.3c), per un determinato insieme
di equazioni.

\end{itemize}

Il pacchetto \pkg{amsmath} \`e distribuito insieme ad alcuni piccoli
pacchetti ausiliari:
\begin{description}
\item[\pkg{amsmath}] Il pacchetto principale, fornisce diverse funzionalit\`a
  per equazioni in \emph{display} e altri costrutti matematici.

\item[\pkg{amstext}] Fornisce il comando \cn{text} per
  sistemare un frammento di testo in un \emph{display}.

\item[\pkg{amsopn}] Fornisce il comando \cn{DeclareMathOperator} per definire
  nuovi `nomi di operatori' come \cn{sin} e \cn{lim}.

\item[\pkg{amsbsy}] Per compatibilit\`a all'indietro questo pacchetto
  continua a esistere, ma in alternativa ad esso si consiglia l'uso del
  pi\`u recente pacchetto \pkg{bm} fornito a corredo di \LaTeX{}.

\item[\pkg{amscd}] Fornisce un ambiente \env{CD} per semplici diagrammi
  commutativi (privi di frecce diagonali).

\item[\pkg{amsxtra}] Fornisce alcune cianfrusaglie come \cn{fracwithdelims}
  e \cn{accentedsymbol}, per compatibilit\`a con documenti creati usando
  la versione~1.1.

\end{description}

Il pacchetto \pkg{amsmath} incorpora \pkg{amstext}, \pkg{amsopn}, e
\pkg{amsbsy}. Le funzionalit\`a di \pkg{amscd} e \pkg{amsxtra}, invece,
sono disponibili solo invocando separatamente questi pacchetti.

%%%%%%%%%%%%%%%%%%%%%%%%%%%%%%%%%%%%%%%%%%%%%%%%%%%%%%%%%%%%%%%%%%%%%%%%
% by GA

\chapter{Opzioni per il pacchetto \pkg{amsmath}}\label{options}

Il pacchetto \pkg{amsmath} ha le seguenti opzioni:
\begin{description}

\item[\opt{centertags}] (\emph{default}) Centra verticalmente\index{equazioni,
numeri delle!posizionamento verticale}, rispetto all'altezza totale
dell'equazione, la numerazione delle equazioni spezzate su pi\`u linee.

\item[\opt{tbtags}] `Top-or-bottom tags' (Etichette in cima o in fondo):
Allinea la numerazione\index{equazioni, numeri delle!posizionamento
verticale} delle equazioni spezzate su pi\`u linee all'ultima
(rispettivamente alla prima) linea, se i numeri stanno sulla destra
(rispettivamente sulla sinistra).

\item[\opt{sumlimits}] (\emph{default}) Posiziona esponenti e
deponenti\index{esponenti e deponenti!posizionamento}\relax
\index{limiti|see{esponenti e deponenti}} dei simboli di sommatoria
sopra e sotto, nelle equazioni in \emph{display}. Questa opzione
influenza anche altri simboli dello stesso tipo\mdash $\prod$,
$\coprod$, $\bigotimes$, $\bigoplus$, e cos\`\i\ via\mdash eccetto gli
integrali (vedi sotto).

\item[\opt{nosumlimits}] Posiziona gli esponenti e deponenti dei
simboli simil-sommatoria sempre a fianco, anche nelle equazioni in
\emph{display}.

\item[\opt{intlimits}] Come \opt{sumlimits}, ma per i simboli di
integrale\index{integrali!posizionamento dei limiti}.

\item[\opt{nointlimits}] (\emph{default}) Il contrario di \opt{intlimits}.

\item[\opt{namelimits}] (\emph{default}) Come \opt{sumlimits}, ma per certi
`nomi di operatori' come $\det$, $\inf$, $\lim$, $\max$, $\min$, che
tradizionalmente hanno deponenti \index{esponenti e
deponenti!posizionamento} posizionati sotto di essi all'interno di
equazioni \emph{display}.

\item[\opt{nonamelimits}] Il contrario di \opt{namelimits}.

\end{description}

Per usare una di queste opzioni del pacchetto bisogna mettere il nome
dell'opzione nell'argomento opzionale del comando \cn{usepackage}\mdash
ad esempio, \verb"\usepackage[intlimits]{amsmath}".

Il pacchetto \pkg{amsmath} inoltre riconosce le seguenti opzioni che
sono normalmente selezionate (implicitamente o esplicitamente)
attraverso il comando \cn{documentclass}, e che pertanto non hanno
bisogno di essere ripetute nell'elenco di opzioni del comando
\cn{usepackage}|{amsmath}|.
\begin{description}

\item[\opt{leqno}] Posiziona i numeri di equazione sulla
sinistra.\index{equazioni, numeri delle!posizionamento a destra o a
sinistra}

\item[\opt{reqno}] Posiziona i numeri di equazione sulla destra.

\item[\opt{fleqn}] Posiziona i numeri di equazione a una distanza prefissata
dal margine sinistro piuttosto che centrata nella colonna di
testo.\index{equazioni in \emph{display}!centratura}

\end{description}

%%%%%%%%%%%%%%%%%%%%%%%%%%%%%%%%%%%%%%%%%%%%%%%%%%%%%%%%%%%%%%%%%%%%%%%%
% by GB

\chapter{Equazioni in \emph{display}}

\section{Introduzione}

Il pacchetto \pkg{amsmath} fornisce un certo numero di nuove strutture
per le equazioni in \emph{display}\index{equazioni in
\emph{display}}\index{equazioni|see{equazioni in \emph{display}}},
oltre a quelle fornite dal \latex/ di base; fra queste:
\begin{verbatim}
  equation     equation*     align       align*
  gather       gather*       flalign     flalign*
  multline     multline*     alignat     alignat*
  split
\end{verbatim}
(Sebbene l'ambiente standard \env{eqnarray} rimanga disponibile, \`e
opportuno usare \env{align} o \env{equation}+\env{split}, invece.)

Con l'eccezione di \env{split}, ogni ambiente ha sia una versione
stellata sia una non stellata, dove la versione non stellata permette
la numerazione automatica usando il contatore \latex/ \env{equation}.
Si pu\`o sopprimere il numero in ogni singola linea premettendo un
\cn{notag} al codice \cn{\\}; lo si pu\`o anche
scavalcare\index{equazioni, numeri delle!scavalcare} con un valore di
propria scelta, usando il comando \cn{tag}|{|\<etich>|}|, dove \<etich>
\`e un testo arbitrario, come |$*$| o |ii|, usato  per \qq{numerare}
l'equazione. Si pu\`o anche usare il comando \cn{tag*}, che fa in modo
che il testo fornito venga scritto letteralmente, senza aggiunta di
parentesi. \cn{tag} e \cn{tag*} possono anche essere usati nelle
versioni non numerate di tutte le strutture di allineamento di
\pkg{amsmath}. Alcuni esempi dell'uso di \cn{tag} possono essere
trovati nei file di esempio \fn{testmath.tex} e \fn{subeqn.tex}
forniti con il pacchetto \pkg{amsmath}.

L'ambiente \env{split} \`e una speciale forma subordinata, da usare
solo \emph{all'interno} di altre strutture; non pu\`o essere usato in
una \env{multline}.

Nelle strutture d'allineamento (\env{split}, \env{align} e varianti),
i simboli di relazione hanno un \verb'&' prima, ma non dopo\mdash a
differenza di \env{eqnarray}. Mettere un \verb'&' dopo il simbolo di
relazione interferirebbe con la spaziatura: \`e necessario metterlo
prima.

\begin{table}[p]
\caption[]{Confronto degli ambienti per le equazioni in \emph{display}
(le linee verticali indicano i margini nominali)}\label{displays}
\renewcommand{\theequation}{\arabic{equation}}
\begin{eqxample}
\begin{verbatim}
\begin{equation*}
a=b
\end{equation*}
\end{verbatim}
\producing
\begin{equation*}
a=b
\end{equation*}
\end{eqxample}

\begin{eqxample}
\begin{verbatim}
\begin{equation}
a=b
\end{equation}
\end{verbatim}
\producing
\begin{equation}
a=b
\end{equation}
\end{eqxample}

\begin{eqxample}
\begin{verbatim}
\begin{equation}\label{xx}
\begin{split}
a& =b+c-d\\
 & \quad +e-f\\
 & =g+h\\
 & =i
\end{split}
\end{equation}
\end{verbatim}
\producing
\begin{equation}\label{xx}
\begin{split}
a& =b+c-d\\
 & \quad +e-f\\
 & =g+h\\
 & =i
\end{split}
\end{equation}
\end{eqxample}

\begin{eqxample}
\begin{verbatim}
\begin{multline}
a+b+c+d+e+f\\
+i+j+k+l+m+n
\end{multline}
\end{verbatim}
\producing
\begin{multline}
a+b+c+d+e+f\\
+i+j+k+l+m+n
\end{multline}
\end{eqxample}

\begin{eqxample}
\begin{verbatim}
\begin{gather}
a_1=b_1+c_1\\
a_2=b_2+c_2-d_2+e_2
\end{gather}
\end{verbatim}
\producing
\begin{gather}
a_1=b_1+c_1\\
a_2=b_2+c_2-d_2+e_2
\end{gather}
\end{eqxample}

\begin{eqxample}
\begin{verbatim}
\begin{align}
a_1& =b_1+c_1\\
a_2& =b_2+c_2-d_2+e_2
\end{align}
\end{verbatim}
\producing
\begin{align}
a_1& =b_1+c_1\\
a_2& =b_2+c_2-d_2+e_2
\end{align}
\end{eqxample}

\begin{eqxample}
\begin{verbatim}
\begin{align}
a_{11}& =b_{11}&
  a_{12}& =b_{12}\\
a_{21}& =b_{21}&
  a_{22}& =b_{22}+c_{22}
\end{align}
\end{verbatim}
\producing
\begin{align}
a_{11}& =b_{11}&
  a_{12}& =b_{12}\\
a_{21}& =b_{21}&
  a_{22}& =b_{22}+c_{22}
\end{align}
\end{eqxample}

\begin{eqxample}
\begin{verbatim}
\begin{flalign*}
a_{11}& =b_{11}&
  a_{12}& =b_{12}\\
a_{21}& =b_{21}&
  a_{22}& =b_{22}+c_{22}
\end{flalign*}
\end{verbatim}
\producing
\begin{flalign*}
a_{11}& =b_{11}&
  a_{12}& =b_{12}\\
a_{21}& =b_{21}&
  a_{22}& =b_{22}+c_{22}
\end{flalign*}
\end{eqxample}
\end{table}

\section{Singole equazioni}

L'ambiente \env{equation} viene usato per singole equazioni con
numerazione automatica; l'ambiente \env{equation*} ha la stessa
funzione, senza numerazione.%
%%%%%%%%%%%%%%%%%%%%%%%%%%%%%%%%%%%%%%%%%%%%%%%%%%%%%%%%%%%%%%%%%%%%%%%%
\footnote{\latex/ non fornisce un ambiente \env{equation*}, ma un
ambiente con funzioni analoghe: \env{displaymath}.}

\section{Equazioni spezzate senza allineamento}

L'ambiente \env{multline} \`e una variante di \env{equation}, usata
per le equazioni che non entrano in un'unica riga. La prima riga di una
\env{multline} sar\`a al margine sinistro, e l'ultima al margine
destro, tranne per un rientro ambo i lati, di lunghezza
\cn{multlinegap}; tutte le altre linee verranno centrate
indipendentemente considerando la larghezza del \emph{display} (a meno
che non sia in funzione l'opzione \opt{fleqn}).

Come \env{equation}, \env{multline} fornisce un'unico numero
d'equazione (quindi, nessuna delle singole linee dovrebbe essere
segnata con \cn{notag}). Il numero dell'equazione \`e posto all'ultima
riga (opzione \opt{reqno}) o sulla prima linea (opzione \opt{leqno});
il centramento verticale (come per \env{split}) non \`e supportato in
\env{multline}.

\`E possibile forzare una delle righe di centro a sinistra o a destra
con i comandi \cn{shoveleft}, \cn{shoveright}; questi comandi prendono
l'intera linea come argomento, fino al segno \cn{\\} escluso; ad
esempio
\begin{multline}
\framebox[.65\columnwidth]{A}\\
\framebox[.5\columnwidth]{B}\\
\shoveright{\framebox[.55\columnwidth]{C}}\\
\framebox[.65\columnwidth]{D}
\end{multline}
\begin{verbatim}
\begin{multline}
\framebox[.65\columnwidth]{A}\\
\framebox[.5\columnwidth]{B}\\
\shoveright{\framebox[.55\columnwidth]{C}}\\
\framebox[.65\columnwidth]{D}
\end{multline}
\end{verbatim}

Il valore di \cn{multlinegap} pu\`o essere cambiato con i soliti
comandi \latex/ \cn{setlength} or \cn{addtolength}.

\section{Equazioni spezzate con allineamento}

Come \env{multline}, l'ambiente \env{split} \`e per \emph{singole}
equazioni troppo lunghe per entrare in una riga e che pertanto devono
essere spezzate. A differenza di \env{multline}, per\`o, l'ambiente
\env{split} permette allineamento tra le linee, con l'uso di simboli
|&| per segnare i punti di allineamento. A differenza di altre
strutture di equazioni \pkg{amsmath}, l'ambiente \env{split} non
produce numeri, poich\'e \`e progettato per essere usato
\emph{esclusivamente all'interno di qualche altra struttura per
equazioni in \emph{display}}, solitamente un ambiente \env{equation},
\env{align}, o \env{gather}, che fornisce la numerazione; ad esempio:
\begin{equation}\label{e:barwq}\begin{split}
H_c&=\frac{1}{2n} \sum^n_{l=0}(-1)^{l}(n-{l})^{p-2}
\sum_{l _1+\dots+ l _p=l}\prod^p_{i=1} \binom{n_i}{l _i}\\
&\quad\cdot[(n-l )-(n_i-l _i)]^{n_i-l _i}\cdot
\Bigl[(n-l )^2-\sum^p_{j=1}(n_i-l _i)^2\Bigr].
\kern-2em % adjust equation body to the right [mjd,13-Nov-1994]
\end{split}\end{equation}

\begin{verbatim}
\begin{equation}\label{e:barwq}\begin{split}
H_c&=\frac{1}{2n} \sum^n_{l=0}(-1)^{l}(n-{l})^{p-2}
\sum_{l _1+\dots+ l _p=l}\prod^p_{i=1} \binom{n_i}{l _i}\\
&\quad\cdot[(n-l )-(n_i-l _i)]^{n_i-l _i}\cdot
\Bigl[(n-l )^2-\sum^p_{j=1}(n_i-l _i)^2\Bigr].
\end{split}\end{equation}
\end{verbatim}

La struttura \env{split} dovrebbe costituire l'intero corpo della
struttura racchiudente, tranne per comandi come \cn{label} che non
producono testo visibile.

\section{Gruppi di equazioni senza allineamento}

L'ambiente \env{gather} viene usato per ragguppare equazioni
consecutive quando non vi \`e necessit\`a di allineamento; ogni
equazione \`e centrata separatamente entro i margini (come in
Tabella~\ref{displays}). Le equazioni in un ambiente \env{gather} sono
separati da comandi \cn{\bslash}. Ogni equazione \env{gather} pu\`o
essere un blocco \verb'\begin{split}'
  \dots\ \verb'\end{split}' \mdash ad esempio:
\begin{verbatim}
\begin{gather}
  prima equazione\\
  \begin{split}
    seconda & equazione\\
           & su due linee
  \end{split}
  \\
  terza equazione
\end{gather}
\end{verbatim}

\section{Gruppi di equazioni con allineamento reciproco}

L'ambiente \env{align} \`e usato per gruppi di due o pi\`u  equazioni
quando \`e richiesto allineamento reciproco; di solito vengono scelti
i simboli di relazione per gli allineamenti (come in
Tabella~\ref{displays}).

Per avere pi\`u colonne di equazioni affiancate, si possono usare
simboli di ``e'' commerciale aggiuntivi per separare le colonne:
\begin{align}
x&=y       & X&=Y       & a&=b+c\\
x'&=y'     & X'&=Y'     & a'&=b\\
x+x'&=y+y' & X+X'&=Y+Y' & a'b&=c'b
\end{align}
%
\begin{verbatim}
\begin{align}
x&=y       & X&=Y       & a&=b+c\\
x'&=y'     & X'&=Y'     & a'&=b\\
x+x'&=y+y' & X+X'&=Y+Y' & a'b&=c'b
\end{align}
\end{verbatim}
Annotazioni linea-per-linea sulle equazioni possono essere ottenute
con un opportuno uso di \cn{text} in un ambiente \env{align}:
\begin{align}
x& = y_1-y_2+y_3-y_5+y_8-\dots
                    && \text{per \eqref{eq:C}}\\
 & = y'\circ y^*    && \text{per \eqref{eq:D}}\\
 & = y(0) y'        && \text {per l'Assioma 1.}
\end{align}
%
\begin{verbatim}
\begin{align}
x& = y_1-y_2+y_3-y_5+y_8-\dots
                    && \text{per \eqref{eq:C}}\\
 & = y'\circ y^*    && \text{per \eqref{eq:D}}\\
 & = y(0) y'        && \text {per l'Assioma 1.}
\end{align}
\end{verbatim}
Una variante, l'ambiente \env{alignat}, permette di specificare
manualmente lo spazio orizzontale fra le equazioni; questo ambiente ha
un argomento obbligatorio, il numero di \qq{colonne di equazioni}: si
contano il numero di \verb'&' in una riga, si aggiunge 1 e si divide
per 2.
\begin{alignat}{2}
x& = y_1-y_2+y_3-y_5+y_8-\dots
                  &\quad& \text{per \eqref{eq:C}}\\
 & = y'\circ y^*  && \text{per \eqref{eq:D}}\\
 & = y(0) y'      && \text {per l'Assioma 1.}
\end{alignat}
%
\begin{verbatim}
\begin{alignat}{2}
x& = y_1-y_2+y_3-y_5+y_8-\dots
                  &\quad& \text{per \eqref{eq:C}}\\
 & = y'\circ y^*  && \text{per \eqref{eq:D}}\\
 & = y(0) y'      && \text {per l'Assioma 1.}
\end{alignat}
\end{verbatim}

\section{Blocchi per costrutti allineati}

Come \env{equation}, gli ambienti a equazioni multiple \env{gather},
\env{align} e \env{alignat} sono progettati per produrre strutture
aventi lunghezza complessiva pari alla lunghezza di una riga; questo
implica, ad esempio, che non \`e facile aggiungere parentesi attorno
alle strutture; vengono quindi fornite le varianti \env{gathered},
\env{aligned} e \env{alignedat}, la cui lunghezza totale \`e pari alla
reale lunghezza dei contenuti; possono quindi essere usate come
componenti di un'espressione pi\`u complessa; ad esempio,
\begin{equation*}
\left.\begin{aligned}
  B'&=-\partial\times E,\\
  E'&=\partial\times B - 4\pi j,
\end{aligned}
\right\}
\qquad \text{equazioni di Maxwell}
\end{equation*}
\begin{verbatim}
\begin{equation*}
\left.\begin{aligned}
  B'&=-\partial\times E,\\
  E'&=\partial\times B - 4\pi j,
\end{aligned}
\right\}
\qquad \text{equazioni di Maxwell}
\end{equation*}
\end{verbatim}
Come l'ambiente \env{array}, le varianti \texttt{-ed} possono
accettare un argomento facoltativo \verb'[t]' o \verb'[b]' per
specificare il posizionamento verticale.

Costrutti di tipo \qq{casi} come il seguente sono comuni in matematica:
\begin{equation}\label{eq:C}
P_{r-j}=
  \begin{cases}
    0&  \text{se $r-j$ \`e dispari},\\
    r!\,(-1)^{(r-j)/2}&  \text{se $r-j$ \`e pari}.
  \end{cases}
\end{equation}
e nel pacchetto \pkg{amsmath} c'\`e un ambiente \env{cases} per
facilitarne la scrittura:
\begin{verbatim}
P_{r-j}=
  \begin{cases}
    0&  \text{se $r-j$ \`e dispari},\\
    r!\,(-1)^{(r-j)/2}&  \text{se $r-j$ \`e pari}.
  \end{cases}
\end{verbatim}
Osservare l'uso di \cn{text} (cfr.~\secref{text}) e della matematica
annidata nella precedente formula.

\section{Correggere il posizionamento dei tag}

Posizionare i numeri delle equazioni in blocchi multilinea pu\`o essere
un problema piuttosto complesso; gli ambienti del pacchetto
\pkg{amsmath} fanno il possibile per evitare di sovrascrivere le
equazioni con il numero, eventualmente spostando il numero pi\`u in
alto o pi\`u in basso su una riga diversa; le difficolt\`a nel calcolo
preciso del profilo di un'equazione possono talvolta risultare in
spostamenti inopportuni dei numeri: si pu\`o allora usare il comando
\cn{raisetag}, fornito proprio per regolare manualmente la posizione
verticale del numero dell'equazione attiva, se \`e stato spostato
dalla sua posizione normale: per spostare ad esempio un particolare
numero in alto di sei punti, si scrive |\raisetag{6pt}|; questo tipo
di correzione \`e un lavoro di precisione come le interruzioni di riga
o di pagina, e andrebbe quindi lasciato fino a quando il documento
non sia ormai quasi completo, poich\'e si rischierebbe altrimenti di
dover disfare e rifare una correzione pi\`u volte, per tenersi al passo
con i cambiamenti del contenuto del documento.

\section{Spaziatura verticale e interruzioni di pagina in
\emph{display} su pi\`u linee}

Come nel \latex/, si pu\`o usare il comando \cn{\\}|[|\<dimensione>|]|
per ottenere spazi verticale aggiuntivi in tutti gli ambienti di
equazioni a blocchi del pacchetto \pkg{amsmath}. Quando si usa il
pacchetto \pkg{amsmath}, le \ii{interruzioni di pagina} tra le righe
delle equazioni sono normalmente impedite; la filosofia di ci\`o \`e
che le interruzioni di pagina in questo tipo di materiale dovrebbero
essere scelto dall'autore nei vari casi; per ottenere un'interruzione
di pagina in una particolare equazione in \emph{display}, si pu\`o
usare il comando \cn{displaybreak}; il luogo migliore dove posizionare
un \cn{displaybreak} \`e immediatamente prima del \cn{\\} dove si vuole
che abbia effetto; come il comando \latex/ \cn{pagebreak},
\cn{displaybreak} accetta un argomento opzionale, tra 0 e 4, per
indicare la opportunit\`a dell'interruzione. |\displaybreak[0]|
significa \qq{\`e possibile interrompere qui}, senza incoraggiare
l'interruzione; \cn{displaybreak} senza argomento equivale a
|\displaybreak[4]| e forza l'interruzione.

Se si preferisce permettere le interruzioni di pagina dove capita,
anche in mezzo a una equazione su pi\`u linee, si pu\`o usare
\cn{allowdisplaybreaks}\texttt{[1]} nel preambolo del documento. Un
argomento 1\ndash 4 pu\`o essere usato per un controllo pi\`u fine:
|[1]| permette le interruzioni, evitandole tuttavia il pi\`u
possibile; valori 2,3,4 indicano una permissivit\`a maggiore. Quando
le interruzioni sono abilitate con \cn{allowdisplaybreaks}, il comando
\cn{\\*} pu\`o essere usato, come al solito, per impedire
un'interruzione di pagina a una ben precisa riga.

\begin{bfseries}
Nota: alcuni ambienti di equazioni racchiudono il loro contenuto in
una scatola indistruttibile, con la conseguenza che n\'e
\cn{displaybreak}, n\'e \cn{allowdisplaybreaks} avranno effetto su di
loro; tra questi ambienti vi sono \env{split}, \env{aligned},
\env{gathered} e \env{alignedat}.
\end{bfseries}

\section{Interrompere i \emph{display}}

Il comando \cn{intertext} pu\`o essere usato per una breve inserzione
di una o due righe di testo\index{frammenti di testo in matematica} in
un \emph{display} su pi\`u righe (cfr. il comando \cn{text} in
\secref{text}): la sua caratteristica principale \`e il mantenimento
dell'allineamento, cosa che non avverrebbe se si terminasse il blocco
per ricominciarlo pi\`u avanti. \cn{intertext} pu\`o comparire solo
dopo un comando \cn{\\} o \cn{\\*}. Notare la posizione della parola
\qq{e} in questo esempio.
\begin{align}
A_1&=N_0(\lambda;\Omega')-\phi(\lambda;\Omega'),\\
A_2&=\phi(\lambda;\Omega')-\phi(\lambda;\Omega),\\
\intertext{e}
A_3&=\mathcal{N}(\lambda;\omega).
\end{align}
\begin{verbatim}
\begin{align}
A_1&=N_0(\lambda;\Omega')-\phi(\lambda;\Omega'),\\
A_2&=\phi(\lambda;\Omega')-\phi(\lambda;\Omega),\\
\intertext{e}
A_3&=\mathcal{N}(\lambda;\omega).
\end{align}
\end{verbatim}

\section{Numerazione delle equazioni}

\subsection{Gerarchia della numerazione}
Con il \latex/ se si vogliono numerare le equazioni secondo le
sezioni\mdash cio\`e, con numeri di equazione tipo (1.1), (1.2), \dots,
(2.1), (2.2), \dots, nelle sezioni 1, 2, e cos\`{\i} via\mdash
bisognava ridefinire \cn{theequation} come suggerito nel manuale del
\latex/ \cite[\S6.3, \S C.8.4]{lamport}:
\begin{verbatim}
\renewcommand{\theequation}{\thesection.\arabic{equation}}
\end{verbatim}

Ci\`o funziona piuttosto bene, tranne per il fatto che il contatore
delle equazioni non viene reimpostato a zero all'inizio di un nuovo
capitolo o sezione, a meno di non farlo manualmente con
\cn{setcounter}; per facilitare il procedimento, il pacchetto
\pkg{amsmath} fornisce il comando\index{equazioni, numeri
delle!gerarchia} \cn{numberwithin}. Per legare la numerazione delle
equazioni alla numerazione delle sezioni, con reimpostazione
automatica dei contatori, si pu\`o usare
\begin{verbatim}
\numberwithin{equation}{section}
\end{verbatim}
Come suggerito dal nome, il comando \cn{numberwithin} pu\`o essere
applicato a qualunque contatore, non solo al contatore
\texttt{equation}.

\subsection{Riferimenti incrociati ai numeri delle equazioni}

Per facilitare i riferimenti incrociati alle equazioni, \`e stato
creato il comando \cn{eqref}\index{equazioni, numeri delle!riferimenti
incrociati}, che fornisce automaticamente le parentesi attorno al
numero: cos\`{\i}, mentre \verb'\ref{abc}' produce 3.2,
\verb'\eqref{abc}' produce (3.2).

\subsection{Numerazione subordinata}

Il pacchetto \pkg{amsmath} fornisce anche un ambiente
\env{subequations}\index{equazioni, numeri delle!numerazione delle
subordinate} per facilitare la numerazione delle equazioni di un
gruppo con uno schema subordinato; ad esempio,
\begin{verbatim}
\begin{subequations}
...
\end{subequations}
\end{verbatim}
fa in modo che tutte le equazioni numerate in quella parte del
documento vengano numerate con (4.9a) (4.9b) (4.9c) \dots, se la
precedente equazione aveva numero (4.8). Un comando \cn{label} subito
dopo \verb/\begin{subequations}/ produrr\`a un \cn{ref} al numero
genitore 4.9, non a 4.9a; i contatori usati dall'ambiente
\env{subequations} sono \verb/parentequation/ e \verb/equation/;
\cn{addtocounter}, \cn{setcounter}, \cn{value} etc.\ possono essere
applicati come al solito ai nomi di questi contatori; per ottenere
qualcosa di diverso dalle lettere minuscole per i numeri delle
subordinate, si usa il metodo standard \latex/ per cambiare lo stile
di numerazione \cite[\S6.3, \S C.8.4]{lamport}. Ad esempio, ridefinendo
\cn{theequation} come segue fornisce numeri romani.
\begin{verbatim}
\begin{subequations}
\renewcommand{\theequation}{\theparentequation \roman{equation}}
...
\end{verbatim}

%%%%%%%%%%%%%%%%%%%%%%%%%%%%%%%%%%%%%%%%%%%%%%%%%%%%%%%%%%%%%%%%%%%%%%%%
%% FcDC
\chapter{Varie funzionalit\`{a} matematiche}

\section{Matrici}\label{ss:matrix}

Il pacchetto \pkg{amsmath} fornisce qualche ambiente per le
matrici\index{matrici} oltre al fondamentale  ambiente \env{array} del
\latex/. Gli ambienti \env{pmatrix}, \env{bmatrix}, \env{Bmatrix},
\env{vmatrix} e \env{Vmatrix} hanno come delimitatori rispettivamente
$(\,)$, $[\,]$, $\lbrace\,\rbrace$, $\lvert\,\rvert$, $\lVert\,\rVert$;
per coerenza con la nomenclatura viene fornito anche un ambiente
\env{matrix} senza delimitatori. Questo pu\`o sembrare superfluo,
vista la presenza dell'ambiente \env{array}, ma ci\`o non \`e vero;
infatti tutti gli ambienti per matrici utilizzano una spaziatura
orizzontale pi\`u economica di quella generosa messa a disposizione
dall'ambiente \env{array}. Inoltre, diversamente dall'ambiente
\env{array}, non si devono specificare i parametri relativi alle
colonne in nessuno degli ambienti per matrici; di \emph{default} si possono
avere fino a 10 colonne centrate.%
\footnote{%%%%%%%%%%%%%%%%%%%%%%%%%%%%%%%%%%%%%%%%%%%%%%%%%%%%%%%%%%%%%%
In dettaglio: Il massimo numero di colonne in una matrice \`e indicato
dal contatore |MaxMatrixCols| (valore normale=10), che si pu\`o
cambiare con i comandi \latex/ \cn{setcounter} o \cn{addcounter}.
}\space%%%%%%%%%%%%%%%%%%%%%%%%%%%%%%%%%%%%%%%%%%%%%%%%%%%%%%%%%%%%%%%%%
(Per ottenere l'allineamento a destra o a sinistra in una colonna, oppure
per qualsiasi altro formato speciale, \`e necessario utilizzare
\env{array})

Per ottenere una piccola matrice adatta al testo, \`e disponibile
l'ambiente \env{smallmatrix} (es:
\begin{math}
\bigl( \begin{smallmatrix}
  a&b\\ c&d
\end{smallmatrix} \bigr)
\end{math})
che \`e pi\`u adatta di qualsiasi altra matrice a entrare in una riga
di testo. Devono essere comunque forniti i delimitatori: non ci sono le
versioni |p|,|b|,|B|,|v|,|V| di \env{smallmatrix}. L'esempio qua sopra
\`e stato prodotto da
\begin{verbatim}
\bigl( \begin{smallmatrix}
  a&b\\ c&d
\end{smallmatrix} \bigr)
\end{verbatim}

\cn{hdotsfor}|{|\<numero>|}| produce una riga di punti in una matrice
\index{matrici!puntini}\index{puntini!nelle matrici}\index{punti|see{puntini}}%
larga tante colonne quanto il numero passato come argomento. Per
esempio,
\begin{center}
\begin{minipage}{.3\columnwidth}
\noindent$\begin{matrix} a&b&c&d\\
e&\hdotsfor{3} \end{matrix}$
\end{minipage}%
\qquad
\begin{minipage}{.45\columnwidth}
\begin{verbatim}
\begin{matrix} a&b&c&d\\
e&\hdotsfor{3} \end{matrix}
\end{verbatim}
\end{minipage}%
\end{center}

La spaziatura dei punti pu\`o essere variata con l'utilizzo di un
opzione tra parentesi quadre, ad esempio, |\hdotsfor[1.5]{3}|. Il
numero racchiuso dalle parentesi funge da moltiplicatore (il valore
normale \`e 1.0)

\begin{equation}\label{eq:D}
\begin{pmatrix} D_1t&-a_{12}t_2&\dots&-a_{1n}t_n\\
-a_{21}t_1&D_2t&\dots&-a_{2n}t_n\\
\hdotsfor[2]{4}\\
-a_{n1}t_1&-a_{n2}t_2&\dots&D_nt\end{pmatrix},
\end{equation}
\begin{verbatim}
\begin{pmatrix} D_1t&-a_{12}t_2&\dots&-a_{1n}t_n\\
-a_{21}t_1&D_2t&\dots&-a_{2n}t_n\\
\hdotsfor[2]{4}\\
-a_{n1}t_1&-a_{n2}t_2&\dots&D_nt\end{pmatrix}
\end{verbatim}


\section{Comandi per la spaziatura matematica}

Il pacchetto \pkg{amsmath} estende l'insieme dei comandi di spaziatura
\index{spaziatura orizzontale!in matematica} come mostrato sotto. Sia
la forma intera che quella contratta di questi comandi sono robuste e
possono essere utilizzate anche al di fuori dell'ambiente matematico.

\begin{ctab}{lll|lll}
Abbrev.&Forma intera& Esempio &Abbrev.&Forma intera&Esempio\\
\hline
\vstrut{2.5ex}
& no space& \spx{}& & no space & \spx{}\\
\cn{\,}& \cn{thinspace}& \spx{\,}&
  \cnbang& \cn{negthinspace}& \spx{\!}\\
\cn{\:}& \cn{medspace}& \spx{\:}&
  & \cn{negmedspace}& \spx{\negmedspace}\\
\cn{\;}& \cn{thickspace}& \spx{\;}&
  & \cn{negthickspace}& \spx{\negthickspace}\\
& \cn{quad}& \spx{\quad}\\
& \cn{qquad}& \spx{\qquad}
\end{ctab}
Per il maggior controllo possibile sulla spaziatura matematica \`e
possibile utilizzare \cn{mspace} e le `unit\`{a} matematiche';
un'unit\`{a} matematica o |mu| \`e uguale a 1/18esimo. Per avere un
\cn{quad} negativo si deve scrivere |\mspace{-18.0mu}|.


\section{Punti}
Non esiste un consenso generale per quanto riguarda il piazzamento dei
punti ellittici (a mezza riga o in fondo della riga) in vari contesti.
La cosa pu\`o quindi essere considerata una questione di gusto.
Utilizzando i comandi orientati verso la semantica
\begin{itemize}
\item \cn{dotsc} per \qq{punti con virgole}
\item \cn{dotsb} per \qq{punti con operazioni/relazioni binarie}
\item \cn{dotsm} per \qq{punti con moltiplicazioni}
\item \cn{dotsi} per \qq{punti con integrali}
\item \cn{dotso} per \qq{altri tipi} (nessuno dei precedenti)
\end{itemize}
invece di \cn{ldots} e \cn{cdots}, \`e possibile adattare a varie
convenzioni un documento ``al volo'', nel caso che (per esempio)
dovendo pubblicare tale documento, l'editore insista nel seguire le
tradizioni della casa. Il trattamento predefinito a seconda delle
situazioni segue le convenzioni dell'American Mathematical Society:
\begin{center}
\begin{tabular}{@{}l@{}l@{}}
\begin{minipage}[t]{.54\textwidth}
\begin{verbatim}
Abbiamo quindi la serie $A_1, A_2,
\dotsc$, la somma di regioni $A_1
+A_2 +\dotsb $, il prodotto
ortogonale $A_1 A_2 \dotsm $, e
l'integrale infinito
\[\int_{A_1}\int_{A_2}\dotsi.\]
\end{verbatim}
\end{minipage}
&
\begin{minipage}[t]{.45\textwidth}
\noindent
Abbiamo quindi la serie $A_1, A_2,
\dotsc$, la somma di regioni $A_1
+A_2 +\dotsb $, il prodotto
ortogonale $A_1 A_2 \dotsm $, e
l'integrale infinito
\[\int_{A_1}\int_{A_2}\dotsi.\]
\end{minipage}
\end{tabular}
\end{center}

\section{Trattini senza interruzioni}
Viene fornito il comando \cn{nobreakdash} per eliminare la
possibilit\`{a} che avvenga un'interruzione di linea dopo un trattino.
Ad esempio scrivendo `pagine 1\ndash 9'  come |pagine 1\nobreakdash 9|
non occorrer\`{a} mai un'interruzione di linea tra il trattino e il 9.
\`E possibile utilizzare \cn{nobreakdash} anche per prevenire
sillabazioni indesiderate in combinazioni tipo |$p$-adico|. Per un
utilizzo frequente \`e consigliato fare delle abbreviazioni; ad
esempio

\begin{verbatim}
\newcommand{\p}{$p$\nobreakdash}% per "\p-adico"
\newcommand{\Ndash}{\nobreakdash--}% per "pagine 1\Ndash 9"
%    Per "\n dimensionale" ("n-dimensionale"):
\newcommand{\n}[1]{$n$\nobreakdash-\hspace{0pt}}
\end{verbatim}
L'ultimo esempio mostra come impedire un'interruzione di linea dopo il
trattino ma permette la corretta sillabazione delle parole
seguenti.(Basta aggiungere un spazio di dimensione zero dopo il
trattino.)


\section{Accenti in matematica}

Nel \latex/ ordinario, il piazzamento del secondo accento negli accenti
matematici doppi \`e spesso mediocre; con il pacchetto \pkg{amsmath}
si migliora notevolmente il piazzamento del secondo accento:
$\hat{\hat{A}}$ (\cn{hat}|{\hat{A}}|).

Sono disponibili i  comandi \cn{dddot} e \cn{dddddot} per produrre
accenti tripli e quadrupli in aggiunta a \cn{dot} e \cn{ddot} presenti
nel \latex/.

Per ottenere un carattere di tilde o di cappello come apice, si deve
caricare il pacchetto \pkg{amsxtra} e utilizzare i comandi \cn{sphat}
o \cn{sptilde}, l'utilizzo \`e \verb'A\sphat' (notare l'assenza del
carattere \verb'^'). Per piazzare un simbolo arbitrario in posizione
di accento matematico o per ottenere accenti come pedici, consultare
il pacchetto \pkg{accents} di Javier Bezos.

\section{Radici}

Nel \latex/ ordinario il piazzamento degli indici delle radici a volte
non \`e buono:  $\sqrt[\beta]{k}$ (|\sqrt||[\beta]{k}|), nel pacchetto
 \pkg{amsmath} i comandi  \cn{leftroot} e \cn{uproot} permettono di aggiustare
la posizione della radice:

\begin{verbatim}
  \sqrt[\leftroot{-2}\uproot{2}\beta]{k}
\end{verbatim}
muove la beta in alto e verso destra:
$\sqrt[\leftroot{-2}\uproot{2}\beta]{k}$. L'argomento negativo di
\cn{leftroot} muove $\beta$ verso destra; le unit\`{a} sono piccole, e
quindi adatte per questo tipo di aggiustamenti.

\section{Formule in riquadro}

Il comando \cn{boxed} costruisce un riquadro attorno al suo argomento,
come \cn{fbox}, eccetto che i contenuti dei riquadri sono in modo
matematico:

\begin{equation}
\boxed{\eta \leq C(\delta(\eta) +\Lambda_M(0,\delta))}
\end{equation}
\begin{verbatim}
  \boxed{\eta \leq C(\delta(\eta) +\Lambda_M(0,\delta))}
\end{verbatim}

\section{Frecce in alto e in basso}
Il \latex/ di base fornisce i comandi \cn{overrightarrow} e
\cn{overleftarrow}; il paccheto \pkg{amsmath} fornisce altri comandi
per frecce in alto e in basso per estendere l'insieme di base:

\begin{tabbing}
\qquad\=\ncn{overleftrightarrow}\qquad\=\kill
\> \cn{overleftarrow}    \> \cn{underleftarrow} \+\\
   \cn{overrightarrow}    \> \cn{underrightarrow} \\
   \cn{overleftrightarrow}\> \cn{underleftrightarrow}
\end{tabbing}

\section{Frecce estendibili}
\cn{xleftarrow} e \cn{xrightarrow} producono frecce
\index{frecce!estendibili} che si estendono automaticamente per accomodare
grandezze inusuali di apici e pedici. Questi comandi prendono un argomento
facoltativo (il pedice) e un argomento obbligatorio (l'apice, possibilmente
anche vuoto):

\begin{equation}
A\xleftarrow{n+\mu-1}B \xrightarrow[T]{n\pm i-1}C
\end{equation}
\begin{verbatim}
  \xleftarrow{n+\mu-1}\quad \xrightarrow[T]{n\pm i-1}
\end{verbatim}

\section{Attaccare simboli ad altri simboli}

\latex/ fornisce  \cn{stackrel} per piazzare un apice
\index{esponenti e deponenti} sopra una relazione binaria.
Nel pacchetto \pkg{amsmath} ci sono comandi pi\`u generali,
\cn{overset} e \cn{underset} che possono essere utilizzati per
piazzare un simbolo sopra o sotto un altro simbolo, ogni qualvolta che
si trova una relazione binaria  o qualcos'altro.
L'input |\overset{*}{X}| piazza un $*$ della dimensione
di un apice
sopra la $X$: $\overset{*}{X}$; \cn{underset} \`e l'analogo
per aggiungere un simbolo in basso.
Controllare anche la descrizione di \cn{sideset} in \secref{sideset}.

\section{Frazioni e costrutti correlati}

\subsection{I comandi \cn{frac}, \cn{dfrac}, e \cn{tfrac}}

Il comando \cn{frac}, che fa parte dell'insieme dei comandi dei base
del \latex/,\index{frazioni} prende due argomenti\mdash numeratore
e denominatore\mdash e compone questi nella classica forma di una frazione.
Il pacchetto \pkg{amsmath} fornisce anche \cn{dfrac} e \cn{tfrac} come
convenienti abbreviazioni per |{\displaystyle\frac| |...| |}|
e\index{textstyle@\cn{textstyle}}\relax
\index{displaystyle@\cn{displaystyle}} |{\textstyle\frac| |...| |}|.

\begin{equation}
\frac{1}{k}\log_2 c(f)\quad\tfrac{1}{k}\log_2 c(f)\quad
\sqrt{\frac{1}{k}\log_2 c(f)}\quad\sqrt{\dfrac{1}{k}\log_2 c(f)}
\end{equation}
\begin{verbatim}
\begin{equation}
\frac{1}{k}\log_2 c(f)\;\tfrac{1}{k}\log_2 c(f)\;
\sqrt{\frac{1}{k}\log_2 c(f)}\;\sqrt{\dfrac{1}{k}\log_2 c(f)}
\end{equation}
\end{verbatim}

\subsection{I comandi \cn{binom}, \cn{dbinom}, e \cn{tbinom}}

Per espressioni binomiali\index{binomiali} tipo $\binom{n}{k}$
\pkg{amsmath} fornisce \cn{binom}, \cn{dbinom} e \cn{tbinom}:
\begin{equation}
2^k-\binom{k}{1}2^{k-1}+\binom{k}{2}2^{k-2}
\end{equation}
\begin{verbatim}
2^k-\binom{k}{1}2^{k-1}+\binom{k}{2}2^{k-2}
\end{verbatim}

\subsection{Il comando \cn{genfrac}}

Le capacit\`{a} di \cn{frac}, \cn{binom}, e delle loro varianti sono
sintetizzate dal comando generale \cn{genfrac}, che richiede sei
argomenti. Gli ultimi due corrispondono al numeratore e denominatore di
\cn{frac}, i primi due sono delimitatori opzionali (come visto in
\cn{binom}); il terzo riguarda lo spessore della linea (\cn{binom}
utilizza questo per impostare lo spessore della linea di frazione a 0
\mdash cio\`e invisibile) e il quarto argomento cambia lo stile
matematico: valori interi tra 0 e 3 selezionano rispettivamente
\cn{displaystyle}, \cn{textstyle}, \cn{scriptstyle} e
\cn{scriptscriptstyle}. Se il terzo argomento viene lasciato vuoto, lo
spessore della linea viene impostato per convenzione a `normale'.

\begin{center}\begin{minipage}{.85\columnwidth}
\raggedright \normalfont\ttfamily \exhyphenpenalty10000
\newcommand{\ma}[1]{%
  \string{{\normalfont\itshape#1}\string}\penalty9999 \ignorespaces}
\string\genfrac \ma{delim-sx} \ma{delim-dx} \ma{spessore}
\ma{stile} \ma{numeratore} \ma{denominatore}
\end{minipage}\end{center}
Per completezza viene mostrato come \cn{frac}, \cn{tfrac} e \cn{binom}
potrebbero essere definiti.
\begin{verbatim}
\newcommand{\frac}[2]{\genfrac{}{}{}{}{#1}{#2}}
\newcommand{\tfrac}[2]{\genfrac{}{}{}{1}{#1}{#2}}
\newcommand{\binom}[2]{\genfrac{(}{)}{0pt}{}{#1}{#2}}
\end{verbatim}
Se si utilizza ripetutamente \cn{genfrac} in un documento per una
particolare notazione, sarebbe di grande comodit\`a per lo scrittore
(e l'editore) definire un'abbreviazione significativa per questa
notazione, come \cn{frac} e \cn{binom} illustrate sopra. I comandi
primitivi generali per le frazioni \cs{over}, \cs{overwithdelims},
\cs{atop}, \cs{atopwithdelims}, \cs{above} e \cs{abovewithdelims}
producono messaggi di avvertimento se utilizzati in congiunzione con
\pkg{amsmath}, per ragioni discusse in \fn{technote.tex}.

\section{Frazioni continue}

La frazione continua\index{frazioni continue}
\begin{equation}
\cfrac{1}{\sqrt{2}+
 \cfrac{1}{\sqrt{2}+
  \cfrac{1}{\sqrt{2}+\cdots
}}}
\end{equation}
si ottiene digitando
{\samepage
\begin{verbatim}
\cfrac{1}{\sqrt{2}+
 \cfrac{1}{\sqrt{2}+
  \cfrac{1}{\sqrt{2}+\dotsb
}}}
\end{verbatim}
}% End of \samepage
Questo produce un  risultato visivamente migliore di quello ottenuto
con l'utilizzo prolungato di \cn{frac}. Il piazzamento a destra o a sinistra
di qualsiasi dei numeratori \`e ottenuto utilizzando \cn{cfrac}|[l]| o
\cn{cfrac}|[r]| invece di \cn{cfrac}.

\section{Opzioni smash}

Il comando \cn{smash} viene utilizzato per comporre una sottoformula
con effettiva larghezza e profondit\`{a} zero; questo a volte rimane
utile dovendo aggiustare la posizione della sottoformula rispetto ai
simboli adiacenti. Con il pacchetto \pkg{amsmath}, \cn{smash} ha
argomenti opzionali |t| e |b|, perch\'e occasionalmente \`e
vantaggioso essere capaci di \qq{appiattire} solo l'altezza o la
profondit\`a, conservando l'altra. Ad esempio, quando simboli di
radicali sono posizionati o dimensionati in modo diverso a causa delle
differenze di altezza e larghezza dei loro contenuti, \cn{smash}
pu\`o essere applicato per rendere il tutto pi\`u consistente.
Confrontare $\sqrt{x}+\sqrt{y}+\sqrt{z}$ con
$\sqrt{x}+\sqrt{\smash[b]{y}}+\sqrt{z}$, dove l'ultimo \`e stato
prodotto con \verb"$\sqrt{x}" \verb"+"
\verb"\sqrt{"\5\verb"\smash[b]{y}}" \verb"+" \verb"\sqrt{z}$".

\section{Delimitatori}

\subsection{Dimensione dei delimitatori}\label{bigdel}

Il dimensionamento automatico dei delimitatori fatto da \cn{left} e
\cn{right} ha due limitazioni: innanzi tutto, viene applicato
meccanicamente per produrre delimitatori abbastanza grandi da
ricoprire il pi\`u grande oggetto contenuto in essi, e inoltre,
l'intervallo delle dimensioni non \`e neanche approssimativamente
continuo, ma ha dei salti abbastanza grandi. Questo significa che un
frammemto matematico infinitesimamente troppo grande per una data
grandezza del delimitatore prender\`{a} la misura successiva, un salto
di 3pt o simile in un testo a grandezza normale. Ci sono due o tre
situazioni dove la grandezza del delimitatore viene comunemente
aggiustata, utilizzando un insieme di comandi che contengono `big' nei
loro nomi.

\begin{ctab}{l|llllll}
Dim. del&
  dim. del& \ncn{left}& \ncn{bigl}& \ncn{Bigl}& \ncn{biggl}& \ncn{Biggl}\\
delimitatore&
  testo& \ncn{right}& \ncn{bigr}& \ncn{Bigr}& \ncn{biggr}& \ncn{Biggr}\\
\hline
Risultato\vstrut{5ex}&
  $\displaystyle(b)(\frac{c}{d})$&
  $\displaystyle\left(b\right)\left(\frac{c}{d}\right)$&
  $\displaystyle\bigl(b\bigr)\bigl(\frac{c}{d}\bigr)$&
  $\displaystyle\Bigl(b\Bigr)\Bigl(\frac{c}{d}\Bigr)$&
  $\displaystyle\biggl(b\biggr)\biggl(\frac{c}{d}\biggr)$&
  $\displaystyle\Biggl(b\Biggr)\Biggl(\frac{c}{d}\Biggr)$
\end{ctab}
Il primo tipo di situazione \`e un operatore cumulativo con limiti
sopra e sotto. Con \cn{left} e \cn{right} i delimitatori di solito
diventano pi\`u larghi del necessario, e utilizzando invece
le dimensioni |Big| o |bigg| si ottengono risultati migliori.
\begin{equation*}
\left[\sum_i a_i\left\lvert\sum_j x_{ij}\right\rvert^p\right]^{1/p}
\quad\text{contro}\quad
\biggl[\sum_i a_i\Bigl\lvert\sum_j x_{ij}\Bigr\rvert^p\biggr]^{1/p}
\end{equation*}
\begin{verbatim}
\biggl[\sum_i a_i\Bigl\lvert\sum_j x_{ij}\Bigr\rvert^p\biggr]^{1/p}
\end{verbatim}
Il secondo tipo di situazione \`e un ammasso di coppie  di delimitatori
dove \cn{left} e \cn{right} rendono le loro grandezze uguali
(dato che questo risulta adeguato per racchiudere tutto il materiale)
ma l'effetto desiderato \`e quello di avere alcuni delimitatori
con grandezza maggiore, per rendere l'annidamento pi\`u semplice da
vedere.
\begin{equation*}
\left((a_1 b_1) - (a_2 b_2)\right)
\left((a_2 b_1) + (a_1 b_2)\right)
\quad\text{contro}\quad
\bigl((a_1 b_1) - (a_2 b_2)\bigr)
\bigl((a_2 b_1) + (a_1 b_2)\bigr)
\end{equation*}
\begin{verbatim}
\left((a_1 b_1) - (a_2 b_2)\right)
\left((a_2 b_1) + (a_1 b_2)\right)
\quad\text{versus}\quad
\bigl((a_1 b_1) - (a_2 b_2)\bigr)
\bigl((a_2 b_1) + (a_1 b_2)\bigr)
\end{verbatim}
Il terzo tipo di situazione \`e un oggetto di dimensione leggermente
elevata nel testo libero, %running rext??
come $\left\lvert\frac{b'}{d'}\right\rvert$, dove i delimitatori
prodotti da \cn{left} e \cn{right} causano un'eccessiva altezza
della linea. In questo caso \ncn{bigl} e \ncn{bigr}\index{big@\cn{big},
\cn{Big}, \cn{bigg}, \dots\ delimiters} possono essere utilizzati
per produrre delimitatori che sono leggermente pi\`u grandi della
dimensione di base, ma che comunque rientrano all'interno della normale
spaziatura della linea: $\bigl\lvert\frac{b'}{d'}\bigr\rvert$.
Nel \latex/ ordinario i delimitatori \ncn{big}, \ncn{bigg}, \ncn{Big},
e \ncn{Bigg} non sono scalati in modo opportuno per tutto il ``range''
delle dimensioni dei \emph{font} \latex/, con il pacchetto \pkg{amsmath}
invece lo sono.

\subsection{Notazioni per la barra verticale}

Il pacchetto \pkg{amsmath} fornisce i comandi \cn{vert}, \cn{rvert},
\cn{lVert}, \cn{rVert} (confrontare \cn{langle} e \cn{rangle}) per
indirizzare il problema del sovraccarico per il carattere di barra
verticale \qc{\|}. Questo carattere viene utilizato nei documenti
\latex/ per una grande variet\`{a} di oggetti matematici: la relazione
`divide' in un' espressione della teoria dei numeri tipo $p\vert q$,
oppure l'operazione di  valore assoluto $\lvert z\rvert$, oppure la
condizione `tale che' nella notazione insiemistica, oppure la
notazione `valutato in' $f_\zeta(t)\bigr\rvert_{t=0}$. La
molteplicit\`{a} degli utilizzi non \`e essa stessa un male, ci\`o
che non va bene comunque \`e il fatto che non tutti questi vari
oggetti ottengono lo stesso trattamento tipografico e che le complesse
capcit\`a discriminatorie di un lettore colto non possono essere
replicate in un computer che deve elaborare documenti matematici. Si
raccomanda quindi che ci sia una corrispondenza uno-a-uno in ogni
documento tra il carattere di barra verticale \qc{\|} e una scelta
notazione matematica, analogamente per il comando di doppia barra
\cn{\|}. Questo immediatamente esclude l'utilizzo di \qc{|} e
\ncn{\|}\index{"|@\cn{"\"|}} come delimitatori, dato che i delimitatori
destri e sinistri %
hanno usi distinti, non correlati allo stesso modo con simboli
adiacenti
%delimiters are distinct usages that do not relate in the same way to
%adjacent symbols;
si raccomanda la pratica di definire nel preambolo del documento
comandi adatti a ogni utilizzo di coppie di delimitatori con simboli
di barre verticali:

\begin{verbatim}
\providecommand{\abs}[1]{\lvert#1\rvert}
\providecommand{\norm}[1]{\lVert#1\rVert}
\end{verbatim}
al che il documento dovrebbe contenere |\abs{z}| per produrre
 $\lvert z\rvert$ e |\norm{v}| per produrre $\lVert v\rVert$.
%%%%% fine FCdC

%%%%% OdB
\chapter{Nomi per gli operatori}

\section{Come definire nuovi nomi di operatori}\label{s:opname}

Le funzioni matematiche\index{nomi di operatori}\index{nomi di
funzioni|see{nomi di operatori}} come $\log$, $\sin$, e $\lim$ sono
per tradizione stampate in tondo per renderne pi\`u immediata la
visibilit\`a rispetto alle variabili matematiche di un carattere, che
sono stampate in stile matematico corsivo. Le pi\`u comuni hanno nomi
predefiniti, \cn{log}, \cn{sin}, \cn{lim}, e cos\`\i{} via, ma se ne
introducono continuamente di nuovi nelle pubblicazioni relative alla
matematica, pertanto il pacchetto \pkg{amsmath} fornisce un metodo
generale per definire nuovi `nomi di operatori'. Per definire una
funzione matematica \ncn{xxx} che si presenti come \cn{sin}, si
scriver\`a
\begin{verbatim}
\DeclareMathOperator{\xxx}{xxx}
\end{verbatim}
Come conseguenza, l'utilizzo di \ncn{xxx} produrr\`a {\upshape xxx}
nel corrispondente \emph{font} e automaticamente aggiunger\`a l'adeguata
spaziatura\index{spaziatura orizzontale!attorno ai nomi di operatori}
su entrambi i lati quando necessario, in maniera tale da ottenere
$A\xxx B$ invece di $A\mathrm{xxx}B$. Nel secondo argomento di
\cn{DeclareMathOperator} (il testo con il nome), \`e prevalente una
modalit\`a pseudo-testuale: il carattere di sillabazione \qc{\-}
verr\`a stampato come un trattino di sillabazione piuttosto che come
un segno meno e un asterisco \qc{\*} risulter\`a stampato come un
asterisco in alto piuttosto che come un asterisco centrato di tipo
matematico (confrontare \textit{a}-\textit{b}*\textit{c} e $a-b*c$.);
d'altra parte il testo contenente il nome \`e stampato in modalit\`a
matematica, ad es. in modo tale da poter ivi usare pedici e apici.

Se il nuovo operatore dovesse esser dotato di pedici e apici
posizionati alla maniera dei
`limiti', al di sopra e al di sotto come per $\lim$, $\sup$, o $\max$,
si user\`a
la forma \qc{\*} del comando \cn{DeclareMathOperator}:
\begin{verbatim}
\DeclareMathOperator*{\Lim}{Lim}
\end{verbatim}
Fare inoltre riferimento alla trattazione del posizionamento dell'indice
nel paragrafo~\ref{subplace}.

I seguenti nomi di operatori sono predefiniti:
\begin{ctab}{rlrlrlrl}
\cn{arccos}& $\arccos$ &\cn{deg}& $\deg$ &      \cn{lg}& $\lg$ &        \cn{projlim}& $\projlim$\\
\cn{arcsin}& $\arcsin$ &\cn{det}& $\det$ &      \cn{lim}& $\lim$ &      \cn{sec}& $\sec$\\
\cn{arctan}& $\arctan$ &\cn{dim}& $\dim$ &      \cn{liminf}& $\liminf$ &\cn{sin}& $\sin$\\
\cn{arg}& $\arg$ &      \cn{exp}& $\exp$ &      \cn{limsup}& $\limsup$ &\cn{sinh}& $\sinh$\\
\cn{cos}& $\cos$ &      \cn{gcd}& $\gcd$ &      \cn{ln}& $\ln$ &        \cn{sup}& $\sup$\\
\cn{cosh}& $\cosh$ &    \cn{hom}& $\hom$ &      \cn{log}& $\log$ &      \cn{tan}& $\tan$\\
\cn{cot}& $\cot$ &      \cn{inf}& $\inf$ &      \cn{max}& $\max$ &      \cn{tanh}& $\tanh$\\
\cn{coth}& $\coth$ &    \cn{injlim}& $\injlim$ &\cn{min}& $\min$\\
\cn{csc}& $\csc$ &      \cn{ker}& $\ker$ &      \cn{Pr}& $\Pr$
\end{ctab}
\begin{ctab}{rlrl}
\cn{varlimsup}&  $\displaystyle\varlimsup$&
  \cn{varinjlim}&  $\displaystyle\varinjlim$\\
\cn{varliminf}&  $\displaystyle\varliminf$&
  \cn{varprojlim}& $\displaystyle\varprojlim$
\end{ctab}

\`E inoltre disponibile un comando \cn{operatorname}, in modo tale che
l'uso di
\begin{verbatim}
\operatorname{abc}
\end{verbatim}
in una formula matematica equivalga all'uso di \ncn{abc} definito da
\cn{DeclareMathOperator}; questo pu\`o in certi casi essere utile per
realizzare notazioni pi\`u complesse o per altri scopi. (Usare la variante
\cn{operatorname*} per ottenere i limiti.)

\section{\cn{mod} e i suoi affini}

I comandi \cn{mod}, \cn{bmod}, \cn{pmod}, \cn{pod} sono forniti per
affrontare le particolari convenzioni di spaziatura della notazione
\qq{mod}. In  \latex/ sono disponibili \cn{bmod} e
\cn{pmod}, ma con il pacchetto \pkg{amsmath}
la spaziatura di \cn{pmod} sar\`a regolata a un valore inferiore se
viene usato in una formula in modalit\`a \emph{non-display}.
 \cn{mod} e \cn{pod} sono varianti di
\cn{pmod} preferite da alcuni autori; \cn{mod} omette le parentesi,
mentre \cn{pod} omette il \qq{mod} e mantiene le parentesi.
\begin{equation}
\gcd(n,m\bmod n);\quad x\equiv y\pmod b;
\quad x\equiv y\mod c;\quad x\equiv y\pod d
\end{equation}
\begin{verbatim}
\gcd(n,m\bmod n);\quad x\equiv y\pmod b;
\quad x\equiv y\mod c;\quad x\equiv y\pod d
\end{verbatim}
%%%%% Fine OdB

%%%%%%%%%%%%%%%%%%%%%%%%%%%%%%%%%%%%%%%%%%%%%%%%%%%%%%%%%%%%%%%%%%%%%%%%
% by GD

\chapter{Il comando \cn{text}}\label{text}

Il principale utilizzo del comando \cn{text} consiste nello scrivere
parole o frasi\index{testo!frammenti di testo in matematica} in un
\emph{display}. Il suo comportamento \`e molto simile al comando
\latex/  \cn{mbox}, ma presenta un paio di vantaggi. Se si desidera
inserire una parola o una frase in un deponente \`e leggermente pi\`u
semplice digitare |..._{\text{parola o frase}}| piuttosto che il
comando equivalente \cn{mbox}: |..._{\mbox{\scriptsize| |parola| |o|
|frase}}|. L'altro vantaggio \`e nel suo nome, pi\`u descrittivo.
\begin{equation}
f_{[x_{i-1},x_i]} \text{ \`e monotona,}
\quad i = 1,\dots,c+1
\end{equation}
\begin{verbatim}
f_{[x_{i-1},x_i]} \text{ \`e monotona,}
\quad i = 1,\dots,c+1
\end{verbatim}

% by GD - end

% by LF
\chapter{Integrali e sommatorie}

\section{Deponenti ed esponenti su pi\`u righe}

Il comando \cn{substack} pu\`o essere usato per produrre un deponente o un
esponente su pi\`u righe:\index{deponenti ed esponenti!su pi\`u righe}\relax
\index{esponenti|see{deponenti ed esponenti}} per esempio
\begin{ctab}{ll}
\begin{minipage}[t]{.6\columnwidth}
\begin{verbatim}
\sum_{\substack{
         0\le i\le m\\
         0<j<n}}
  P(i,j)
\end{verbatim}
\end{minipage}
&
$\displaystyle
\sum_{\substack{0\le i\le m\\ 0<j<n}} P(i,j)$
\end{ctab}
Una forma un po' pi\`u generalizzata \`e l'ambiente \env{subarray} che
consente di specificare che ogni riga deve essere allineata a sinistra invece che
centrata, come in questo caso:
\begin{ctab}{ll}
\begin{minipage}[t]{.6\columnwidth}
\begin{verbatim}
\sum_{\begin{subarray}{l}
        i\in\Lambda\\ 0<j<n
      \end{subarray}}
 P(i,j)
\end{verbatim}
\end{minipage}
&
$\displaystyle
  \sum_{\begin{subarray}{l}
        i\in \Lambda\\ 0<j<n
      \end{subarray}}
 P(i,j)$
\end{ctab}

\section{Il comando \cn{sideset}}\label{sideset}

C'\`e anche un comando chiamato \cn{sideset}, per uno scopo abbastanza
particolare: porre dei simboli agli angoli di deponente ed
esponente\index{deponenti ed esponenti!su sommatorie} di un simbolo
operatorio di grandi dimensioni come $\sum$ o $\prod$. \emph{Nota:
questo comando non \`e pensato per essere applicato ad altro che a
simboli tipo sommatoria.} L'esempio principale \`e il caso in cui si
voglia porre un simbolo di primo su un simbolo di sommatoria. Se non
ci sono estremi sopra o sotto la sommatoria, si pu\`o semplicemente
usare \cn{nolimits}: ecco come appare
%%%%%%%%%%%%%%%%%%%%%%%%%%%%%%%%%%%%%%%%%%%%%%%%%%%%%%%%%%%%%%%%%%%%%%%%
|\sum\nolimits' E_n| in modo \emph{display}:
\begin{equation}
\sum\nolimits' E_n
\end{equation}
Se tuttavia si desidera non solo il segno di primo ma anche qualcosa sopra o
sotto il simbolo di sommatoria, non \`e cos\`\i{} facile\mdash invero, senza
\cn{sideset}, sarebbe proprio difficile. Con \cn{sideset}, si
pu\`o scrivere
\begin{ctab}{ll}
\begin{minipage}[t]{.6\columnwidth}
\begin{verbatim}
\sideset{}{'}
  \sum_{n<k,\;\text{$n$ odd}} nE_n
\end{verbatim}
\end{minipage}
&$\displaystyle
\sideset{}{'}\sum_{n<k,\;\text{$n$ odd}} nE_n
$
\end{ctab}
La coppia di parentesi graffe vuote si spiega con il fatto che
\cn{sideset} ha la possibilit\`a di porre uno o pi\`u simboli aggiuntivi a
ogni angolo di un simbolo operatorio di grandi dimensioni; per porre un asterisco in ciascun angolo di un
simbolo di prodotto, si potrebbe scrivere
\begin{ctab}{ll}
\begin{minipage}[t]{.6\columnwidth}
\begin{verbatim}
\sideset{_*^*}{_*^*}\prod
\end{verbatim}
\end{minipage}
&$\displaystyle
\sideset{_*^*}{_*^*}\prod
$
\end{ctab}

\section{Posizionamento di deponenti ed estremi}\label{subplace}

Il tipo di posizionamento predefinito per i deponenti dipende dal
simbolo base considerato. Per i simboli tipo sommatoria \`e usato il
posizionamento `displaylimits': quando un simbolo tipo sommatoria appare
in una formula in \emph{display}, deponente ed esponente sono posti nella posizione
`limits' sopra e sotto, ma in una formula nel testo sono invece posti
a lato, per evitare l'antiestetico e sprecato allargamento della distanza dalle
righe di testo adiacenti.
L'impostazione predefinita per i simboli tipo integrale \`e avere deponenti
ed esponenti sempre a lato, anche nelle formule in \emph{display}.
(Si veda la discussione su \opt{intlimits} e opzioni correlate nella
Sec.~\ref{options}.)

I nomi di operatore, come $\sin$ o $\lim$, possono avere il posizionamento
`displaylimits' o quello `limits' a seconda di come sono stati definiti. Gli operatori
pi\`u comuni sono definiti in base all'uso consueto in matematica.

I comandi \cn{limits} e \cn{nolimits} possono essere usati per modificare
il normale comportamento di un simbolo base:
\begin{equation*}
\sum\nolimits_X,\qquad \iint\limits_{A},
\qquad\varliminf\nolimits_{n\to \infty}
\end{equation*}
Per definire un comando i cui deponenti seguono lo
stesso comportamento `displaylimits' di \cn{sum}, si pu\`o aggiungere
\cn{displaylimits} in coda alla definizione. Quando ci sono
pi\`u istanze consecutive di \cn{limits}, \cn{nolimits}, o \cn{displaylimits},
l'ultima ha la priorit\`a.

\section{Simboli di integrale multiplo}

\cn{iint}, \cn{iiint}, e \cn{iiiint} producono pi\`u simboli di integrale
\index{integrali!multipli} con la spaziatura tra di loro opportunamente
corretta, sia in stile testo che \emph{display}. \cn{idotsint} estende
la stessa idea producendo due segni di integrale separati da tre punti.
\begin{gather}
\iint\limits_A f(x,y)\,dx\,dy\qquad\iiint\limits_A
f(x,y,z)\,dx\,dy\,dz\\
\iiiint\limits_A
f(w,x,y,z)\,dw\,dx\,dy\,dz\qquad\idotsint\limits_A f(x_1,\dots,x_k)
\end{gather}

% by LF - end
%%%%%%%%%%%%%%%%%%%%%%%%%%%%%%%%%%%%%%%%%%%%%%%%%%%%%%%%%%%%%%%%%%%%%%%%
% by GD

\chapter{Diagrammi commutativi}\label{s:commdiag}

Vari comandi, come quelli in \amstex/, per disegnare i diagrammi
commutativi sono disponibili separatamente nel pacchetto \pkg{amscd}.
Per i diagrammi commutativi di una certa complessit\`a, gli autori
dovranno necessariamente considerare pacchetti pi\`u estesi come
\pkg{kuvio} o \xypic/, ma per diagrammi semplici privi di
frecce\index{frecce!nei diagrammi commutativi} diagonali, i
comandi dell'\pkg{amscd} potrebbero rivelarsi pi\`u convenienti.
Di seguito vi \`e un esempio.
\begin{equation*}
\begin{CD}
S^{{\mathcal{W}}_\Lambda}\otimes T   @>j>>   T\\
@VVV                                    @VV{\End P}V\\
(S\otimes T)/I                  @=      (Z\otimes T)/J
\end{CD}
\end{equation*}
\begin{verbatim}
\begin{CD}
S^{{\mathcal{W}}_\Lambda}\otimes T   @>j>>   T\\
@VVV                                    @VV{\End P}V\\
(S\otimes T)/I                  @=      (Z\otimes T)/J
\end{CD}
\end{verbatim}
Nell'ambiente \env{CD}, i comandi |@>>>|, |@<<<|, |@VVV| e |@AAA|
disegnano, rispettivamente, le frecce a destra, a sinistra, verso il
basso e verso l'alto.
Per quanto riguarda le frecce orizzontali, il contenuto tra il primo
e il secondo simbolo |>| oppure |<| sar\`a inserito a esponente sulla
freccia, e il contenuto tra il secondo e il terzo simbolo sar\`a inserito
a deponente sotto la freccia.
Analogamente per le frecce verticali, il contenuto tra il primo e il
secondo oppure tra il secondo e il terzo dei simboli |A| o |V| sar\`a
inserito a \qq{margine} sinistro o destro della freccia.
I comandi |@=| e \verb'@|' tracciano rispettivamente una doppia linea
orizzontale e una verticale.
Il comando |@.| equivale a una \qq{freccia nulla} e pu\`o essere usato
al posto di una freccia visibile per espandere, dove sia necessario, una
matrice.

% by GD - end

%%%%%%%%%%%%%%%%%%%%%%%%%%%%%%%%%%%%%%%%%%%%%%%%%%%%%%%%%%%%%%%%%%%%%%%%
\chapter{Usare \emph{font} matematici}

\section{Introduzione}

Per informazioni pi\`u complete riguardo l'uso dei \emph{font} in \latex/,
consultate la guida ai \emph{font} del \latex/ (\fn{fntguide.tex}) o
il libro \booktitle{The \latex/ Companion} \cite{tlc}.  L'insieme di base
dei comandi per usare \emph{font} matematici\index{\emph{font} matematici}\relax
\index{simboli matematici|see{\emph{font} matematici}} nel \latex/ \`e
costituito da \cn{mathbf}, \cn{mathrm}, \cn{mathcal}, \cn{mathsf},
\cn{mathtt} e \cn{mathit}. Comandi aggiuntivi per \emph{font} matematici
come \cn{mathbb} per il \emph{blackboard-bold}, \cn{mathfrak} per il Fraktur
e \cn{mathscr} per l'Euler script sono disponibili attraverso i
pacchetti \pkg{amsfonts} e \pkg{euscript} (distribuiti separatamente).

\section{Uso consigliato per i comandi dei \emph{font} matematici}

Se ci si trova a usare frequentemente comandi per \emph{font} matematici nei
propri documenti, si potrebbero voler usare nomi pi\`u brevi, come
\ncn{mb} al posto di \cn{mathbf}.  Ovviamente non c'\`e nulla che
impedisca di farsi da soli tali abbreviazioni, usando i comandi
\cn{newcommand} appropriati.  D'altro canto, per il \latex/, offrire
comandi pi\`u brevi sarebbe addirittura un disservizio per gli autori,
poich\'e renderebbe meno ovvia un'alternativa molto migliore:
definire nomi di comandi personalizzati che si riferiscano ai nomi degli
oggetti matematici che a loro competono, piuttosto che ai nomi dei
\emph{font} che sono usati per distinguere tali oggetti.  Per esempio, se
si usa il grassetto per indicare i vettori, alla lunga sarebbe meglio definire
un comando `vector' al posto di un `math-bold':
\begin{verbatim}
  \newcommand{\vect}[1]{\mathbf{#1}}
\end{verbatim}
si pu\`o scrivere |\vect{a} + \vect{b}| per avere $\vect{a} +
\vect{b}$.
Se, mesi dopo aver cominciato il lavoro, si decide di usare il
grassetto per qualche altro scopo e di indicare i vettori con una
freccina sopra, si pu\`o fare il tutto semplicemente cambiando la
definizione di \ncn{vect}; altrimenti si sarebbero dovute rimpiazzare
tutte le occorrenze di \cn{mathbf} nel documento, eventualmente
persino controllandole una a una per vedere se si riferivano
effettivamente a un vettore.

Pu\`o essere utile anche assegnare nomi di comandi distinti per
differenti lettere dell'alfabeto di un particolare \emph{font}:
\begin{verbatim}
\DeclareSymbolFont{AMSb}{U}{msb}{m}{n}% oppure si usi il pacchetto amsfonts
\DeclareMathSymbol{\C}{\mathalpha}{AMSb}{"43}
\DeclareMathSymbol{\R}{\mathalpha}{AMSb}{"52}
\end{verbatim}
Queste righe definirebbero i comandi \cn{C} e \cn{R} in modo che
producano le lettere \emph{blackboard-bold} del \emph{font} di simboli matematici
`AMSb'.  Se nel proprio documento si fa spesso riferimento ai numeri
reali o a quelli complessi, si pu\`o preferire questo metodo a
quello di definire, per esempio, un comando \ncn{field} e scrivere
|\field{C}| e |\field{R}|, ma per ottenere la massima flessibilit\`a
e il massimo controllo sarebbe opportuno definire tale comando e poi
definire \ncn{C} e \ncn{R} in funzione di quello:\index{mathbb@\cn{mathbb}}
\begin{verbatim}
\usepackage{amsfonts}% per disporre dell'alfabeto \mathbb
\newcommand{\field}[1]{\mathbb{#1}}
\newcommand{\C}{\field{C}}
\newcommand{\R}{\field{R}}
\end{verbatim}

\section{Simboli matematici in grassetto}

Il comando \cn{mathbf} \`e usato comunemente per ottenere lettere
latine grassette in modo matematico, ma per la maggior parte degli
altri tipi di simboli matematici non ha effetto, o i suoi effetti
dipendono in maniera non prevedibile dalla serie di \emph{font} matematici in
uso.  Per esempio, scrivendo
\begin{verbatim}
\Delta \mathbf{\Delta}\mathbf{+}\delta \mathbf{\delta}
\end{verbatim}
si ottiene $\Delta \mathbf{\Delta}\mathbf{+}\delta \mathbf{\delta}$;
il comando \cn{mathbf} non ha cambiato il segno pi\`u e il delta
minuscolo.

Per questo motivo il pacchetto \pkg{amsmath} fornisce altri due
comandi, \cn{boldsymbol} e \cn{pmb}, che possono essere usati con gli
altri tipi di simboli matematici.  \cn{boldsymbol} pu\`o essere usato
per i simboli matematici sui quali non ha effetto il comando \cn{mathbf}
se (e solo se) il \emph{font} matematico in uso in quel momento dispone
di una versione in grassetto di quel simbolo.  \cn{pmb} pu\`o essere usato
come ultima risorsa per qualsiasi simbolo matematico che non abbia una
vera versione in grassetto all'interno del \emph{font} matematico; \qq{pmb}
\`e l'abbreviazione di \qq{poor man's bold} (grassetto dei poveri) e
funziona stampando copie pi\`u copie dello stesso simbolo leggermente spostate
le une dalle altre.  Il risultato \`e di qualit\`a
inferiore, specialmente per quei simboli che contengono linee sottili.
Quando si usa la famiglia standard di \emph{font} matematici del \latex/ (il
Computer Modern), gli unici simboli che potrebbero richiedere il
\cn{pmb} sono quelli dei simboli operatori di grandi dimensioni, come \cn{sum}, i simboli
di delimitatori estesi, o i simboli addizionali forniti dal pacchetto
\pkg{amssymb}~\cite{amsfonts}.

La formula seguente mostra alcuni dei possibili risultati:
\begin{verbatim}
A_\infty + \pi A_0
\sim \mathbf{A}_{\boldsymbol{\infty}} \boldsymbol{+}
  \boldsymbol{\pi} \mathbf{A}_{\boldsymbol{0}}
\sim\pmb{A}_{\pmb{\infty}} \pmb{+}\pmb{\pi} \pmb{A}_{\pmb{0}}
\end{verbatim}
\begin{equation*}
A_\infty + \pi A_0
\sim \mathbf{A}_{\boldsymbol{\infty}} \boldsymbol{+}
  \boldsymbol{\pi} \mathbf{A}_{\boldsymbol{0}}
\sim\pmb{A}_{\pmb{\infty}} \pmb{+}\pmb{\pi} \pmb{A}_{\pmb{0}}
\end{equation*}
Se si vuole usare solo il comando \cn{boldsymbol} senza caricare tutto
il pacchetto \pkg{amsmath}, si pu\`o usare il pacchetto \pkg{bm} (questo
\`e un pacchetto standard del \latex/, non fa parte di quelli AMS; se
si ha una versione del \latex/ del 1997 o posteriore, probabilmente lo
si ha gi\`a).

\section{Lettere greche corsive}

Per ottenere una versione corsiva delle lettere greche maiuscole, si
possono usare i seguenti comandi:
\begin{ctab}{rlrl}
\cn{varGamma}& $\varGamma$& \cn{varSigma}& $\varSigma$\\
\cn{varDelta}& $\varDelta$& \cn{varUpsilon}& $\varUpsilon$\\
\cn{varTheta}& $\varTheta$& \cn{varPhi}& $\varPhi$\\
\cn{varLambda}& $\varLambda$& \cn{varPsi}& $\varPsi$\\
\cn{varXi}& $\varXi$& \cn{varOmega}& $\varOmega$\\
\cn{varPi}& $\varPi$
\end{ctab}

%%%%%%%%%%%%%%%%%%%%%%%%%%%%%%%%%%%%%%%%%%%%%%%%%%%%%%%%%%%%%%%%%%%%%%%%
% by GA

\chapter{Messaggi di errore e problemi di output}

\section{Osservazioni di carattere generale}

Questo \`e un supplemento al capitolo~8 del manuale del \latex/
\cite{lamport} (prima edizione: capitolo~6). Per comodit\`a del
lettore, l'insieme dei messaggi d'errore discussi qui si sovrappone
parzialmente con quello di \cite{lamport}, ma sia chiaro che qui
non si intende dare una copertura esaustiva. I messaggi
d'errore sono disposti in ordine alfabetico, senza badare a testo
irrilevante quale |! LaTeX Error:| all'inizio del messaggio, e
caratteri non alfabetici quali \qc{\\}. Dove vengono forniti esempi,
vengono anche mostrati i messaggi d'aiuto che appaiono sullo schermo
quando si risponde a un messagio d'errore digitando |h|.

C'\`e anche una sezione che discute qualche errore di output, per esempio
in casi in cui il documento stampato ha qualcosa che non va, ma \latex/
non ha rilevato alcun errore.

\section{Messaggi di errore}

\begin{error}{\begin{split} won't work here.}
\errexa
\begin{verbatim}
! Package amsmath Error: \begin{split} won't work here.
 ...

l.8 \begin{split}

? h
\Did you forget a preceding \begin{equation}?
If not, perhaps the `aligned' environment is what you want.
?
\end{verbatim}
\errexpl L'ambiente \env{split} non costruisce un'equazione in
\emph{display} a s\'e stante; deve essere usato all'interno di qualche
altro ambiente quali \env{equation} o \env{gather}.

\end{error}

\begin{error}{Extra & on this line}
\errexa
\begin{verbatim}
! Package amsmath Error: Extra & on this line.

See the amsmath package documentation for explanation.
Type  H <return>  for immediate help.
 ...

l.9 \end{alignat}

? h
\An extra & here is so disastrous that you should probably exit
 and fix things up.
?
\end{verbatim}
\errexpl
In una struttura \env{alignat} il numero di punti di allineamento su una linea
\`e determinato dall'argomento numerico fornito dopo |\begin{alignat}|.
Se in una linea si usano pi\`u punti di allineamento rispetto a quelli consentiti,
\latex/ assume che sia stato dimenticato accidentalmente un comando di
interruzione di riga \cn{\\} e produce questo errore.
\end{error}

\begin{error}{Improper argument for math accent}
\errexa
\begin{verbatim}
! Package amsmath Error: Improper argument for math accent:
(amsmath)                Extra braces must be added to
(amsmath)                prevent wrong output.

See the amsmath package documentation for explanation.
Type  H <return>  for immediate help.
 ...

l.415 \tilde k_{\lambda_j} = P_{\tilde \mathcal
                                               {M}}
?
\end{verbatim}
\errexpl
Argomenti complessi per tutti i comandi \latex/ dovrebbero venire
racchiusi tra parentesi graffe. In questo esempio le graffe sono
necessarie come mostrato:
\begin{verbatim}
... P_{\tilde{\mathcal{M}}}
\end{verbatim}
\end{error}

\begin{error}{Font OMX/cmex/m/n/7=cmex7 not loadable ...}
\errexa
\begin{verbatim}
! Font OMX/cmex/m/n/7=cmex7 not loadable: Metric (TFM) file not found.
<to be read again>
                   relax
l.8 $a
      b+b^2$
? h
I wasn't able to read the size data for this font,
so I will ignore the font specification.
[Wizards can fix TFM files using TFtoPL/PLtoTF.]
You might try inserting a different font spec;
e.g., type `I\font<same font id>=<substitute font name>'.
?
\end{verbatim}
\errexpl Certe dimensioni di alcuni \emph{font} del Computer Modern che erano
un tempo disponibili principalmente attraverso la raccolta
AMSFonts\index{AMSFonts, raccolta} sono considerate parte del \latex/
standard (giugno 1994): \fn{cmex7}\ndash \texttt{9},
\fn{cmmib5}\ndash \texttt{9}, e \fn{cmbsy5}\ndash \texttt{9}. Se
queste dimensioni straordinarie mancano nel proprio sistema,
bisognerebbe prima recuperarle dalla sogrente che ha fornito \latex/.
Altrimenti, si potrebbe provare a reperirle dalla CTAN (per esempio,
sotto forma di sorgenti Metafont\index{Sorgenti Metafont}, directory
\nfn{/tex-archive/fonts/latex/mf}, o in formato PostScript Type 1,
directory \nfn{/tex-archive/fonts/cm/ps-type1/bakoma}\index{\emph{font}
BaKoMa}\index{\emph{font} PostScript}).

Se il nome del \emph{font} comincia per \fn{cmex}, c'\`e un'opzione speciale
\fn{cmex10} per il pacchetto \pkg{amsmath} che fornisce una scappatoia
temporanea. In altre parole, si cambi il comando \cn{usepackage} in
\begin{verbatim}
\usepackage[cmex10]{amsmath}
\end{verbatim}
Questo forzer\`a l'uso della dimensione di 10 punti del \emph{font} \fn{cmex}
in ogni caso. A seconda del contenuto del documento, questo potrebbe
essere adeguato.
\end{error}

\begin{error}{Math formula deleted: Insufficient extension fonts}
\errexa
\begin{verbatim}
! Math formula deleted: Insufficient extension fonts.
l.8 $ab+b^2$

?
\end{verbatim}
\errexpl
Solitamente questo \`e preceduto da un errore del tipo |Font ... not loadable|;
si veda (sopra) la descrizione di quell'errore per risolvere il problema.
\end{error}

\begin{error}{Missing number, treated as zero}
\errexa
\begin{verbatim}
! Missing number, treated as zero.
<to be read again>
                   a
l.100 \end{alignat}

? h
A number should have been here; I inserted `0'.
(If you can't figure out why I needed to see a number,
look up `weird error' in the index to The TeXbook.)

?
\end{verbatim}
\errexpl
Ci sono parecchie cause che possono provocare questo errore. Comunque, una
possibilit\`a che \`e rilevante per il pacchetto \pkg{amsmath} \`e che si
\`e dimenticato di specificare l'argomento numerico di un ambiente \env{alignat},
come illustrato in questo esempio:
\begin{verbatim}
\begin{alignat}
 a&  =b&    c& =d\\
a'& =b'&   c'& =d'
\end{alignat}
\end{verbatim}
dove la prima linea dovrebbe invece essere
\begin{verbatim}
\begin{alignat}{2}
\end{verbatim}

Un'altra possibilit\`a \`e che una aperta parentesi quadra |[|
segua un comando di interruzione di linea \cn{\\} in un costrutto multilinea
come \env{array}, \env{tabular}, o \env{eqnarray}. Questo sar\`a
interpretato da \latex/ come l'inizio di una richiesta di `spazio verticale
aggiuntivo' \cite[\S C.1.6]{lamport}, anche se capita nella linea successiva
con l'intenzione di renderlo parte del contenuto. Per esempio
\begin{verbatim}
\begin{array}
a+b\\
[f,g]\\
m+n
\end{array}
\end{verbatim}
Per evitare il messaggio d'errore in casi di questo tipo, si possono
aggiungere parentesi graffe come suggerito nel manuale di \latex/
\cite[\S C.1.1]{lamport}:
\begin{verbatim}
\begin{array}
a+b\\
{[f,g]}\\
m+n
\end{array}
\end{verbatim}

\end{error}

\begin{error}{Missing \right. inserted}
\errexa
\begin{verbatim}
! Missing \right. inserted.
<inserted text>
                \right .
l.10 \end{multline}

? h
I've inserted something that you may have forgotten.
(See the <inserted text> above.)
With luck, this will get me unwedged. But if you
really didn't forget anything, try typing `2' now; then
my insertion and my current dilemma will both disappear.
\end{verbatim}
\errexpl
Questo errore si verifica tipicamente quando si cerca di inserire una
interruzione di linea all'interno di una coppia di delimitatori
\cn{left}-\cn{right} in un ambiente \env{multline} o \env{split}:
\begin{verbatim}
\begin{multline}
AAA\left(BBB\\
  CCC\right)
\end{multline}
\end{verbatim}
Ci sono due possibili soluzioni: (1)~invece di usare \cn{left} e
\cn{right}, si usino delimitatori `big' di grandezza fissa (\cn{bigl}
\cn{bigr} \cn{biggl} \cn{biggr} \dots; si veda \secref{bigdel}); oppure
(2)~si usino delimitatori nulli per spezzare la copia \cn{left}-\cn{right}
in due parti, una per ogni linea:
\begin{verbatim}
AAA\left(BBB\right.\\
  \left.CCC\right)
\end{verbatim}
La seconda soluzione potrebbe dar luogo a dimensioni incoerenti dei
delimitatori; ci si pu\`o assicurare che coincidono usando \cn{vphantom}
nella linea in cui compare il delimitatore pi\`u piccolo (o magari \cn{smash}
nella linea in cui compare il delimitatore pi\`u grande). Nell'argomento di
\cn{vphantom} bisogna mettere una copia dell'elemento pi\`u alto che compare
nell'altra linea, per esempio
\begin{verbatim}
xxx \left(\int_t yyy\right.\\
  \left.\vphantom{\int_t} zzz ... \right)
\end{verbatim}
\end{error}

\begin{error}{Paragraph ended before \xxx was complete}
\errexa
\begin{verbatim}
Runaway argument?

! Paragraph ended before \multline was complete.
<to be read again>
                   \par
l.100

? h
I suspect you've forgotten a `}', causing me to apply this
control sequence to too much text. How can we recover?
My plan is to forget the whole thing and hope for the best.
?
\end{verbatim}
\errexpl
Questo potrebbe dipendere da un errore di battitura nel comando
|\end{multline}|, per esempio
\begin{verbatim}
\begin{multline}
...
\end{multiline}
\end{verbatim}
o dall'uso di abbreviazioni di alcuni ambienti, come |\bal| e
|\eal| invece di |\begin{align}| e |\end{align}|:
\begin{verbatim}
\bal
...
\eal
\end{verbatim}
Per motivi tecnici quel tipo di abbreviazioni non funzionano con gli
ambienti pi\`u complesi per equazioni in \emph{display} del pacchetto
\pkg{amsmath} (\env{gather}, \env{align}, \env{split}, etc.; cfr.\@ \fn{technote.tex}).
\end{error}

\begin{error}{Runaway argument?}
Si veda la descrizione del messaggio di errore
\texttt{Paragraph ended before \ncn{xxx} was complete}.
\end{error}

\begin{error}{Unknown option `xxx' for package `yyy'}
\errexa
\begin{verbatim}
! LaTeX Error: Unknown option `intlim' for package `amsmath'.
...
? h
The option `intlim' was not declared in package `amsmath', perhaps you
misspelled its name. Try typing  <return>  to proceed.
?
\end{verbatim}
\errexpl
Questo significa che il nome dell'opzione \`e stato scritto male, o
semplicemente che il pacchetto, al  contrario di quanto ci si aspettava,
non ha quell'opzione. Si consulti la documentazione di quel pacchetto.
\end{error}

\begin{error}{Old form `\pmatrix' should be \begin{pmatrix}.}
\errexa
\begin{verbatim}
! Package amsmath Error: Old form `\pmatrix' should be
                         \begin{pmatrix}.

See the amsmath package documentation for explanation.
Type  H <return>  for immediate help.
 ...

\pmatrix ->\left (\matrix@check \pmatrix
                                         \env@matrix
l.16 \pmatrix
             {a&b\cr c&d\cr}
? h
`\pmatrix{...}' is old Plain-TeX syntax whose use is
ill-advised in LaTeX.
?
\end{verbatim}
\errexpl
Quando si usa il paccheto \pkg{amsmath}, le vecchie forme \cn{pmatrix},
\cn{matrix}, e \cn{cases} non posso pi\`u essere usate, a causa di conflitti
di nome. Ad ogni modo, la loro sintassi non era conforme alla sintassi
standard \LaTeX{}.
\end{error}

\begin{error}{Erroneous nesting of equation structures}
\errexa
\begin{verbatim}
! Package amsmath Error: Erroneous nesting of equation structures;
(amsmath)                trying to recover with `aligned'.

See the amsmath package documentation for explanation.
Type  H <return>  for immediate help.
 ...

l.260 \end{alignat*}
                    \end{equation*}
\end{verbatim}
\errexpl
Le strutture \env{align}, \env{alignat}, etc., sono progettate per essere
usate nel livello pi\`u alto, e perlopi\`u non possono essere annidate in
alcune altre strutture di equazioni in \emph{display}. Una eccezione notevole
\`e data dal fatto che \env{align} e molte sue varianti possono essere usate
nell'ambiente \env{gather}.
\end{error}

\section{Messaggi di warning}

\begin{error}{Foreign command \over [or \atop or \above]}
\errexa
\begin{verbatim}
Package amsmath Warning: Foreign command \over; \frac or \genfrac
(amsmath)                should be used instead.
\end{verbatim}
\errexpl L'utilizzo dei comandi di frazione originali del \tex/\mdash
\cs{over}, \cs{atop}, \cs{above}\mdash \`e deprecato quando si usa il
pacchetto \pkg{amsmath}, perch\`e la loro sintassi \`e estranea a \latex/,
e \pkg{amsmath} fornisce comandi equivalenti nativi di \latex/. Si veda
\fn{technote.tex} per ulteriori informazioni.
\end{error}

\begin{error}{Cannot use `split' here}
\errexa
\begin{verbatim}
Package amsmath Warning: Cannot use `split' here;
(amsmath)                trying to recover with `aligned'
\end{verbatim}
\errexpl L'ambiente \env{split} \`e studiato per essere usato con l'intero
corpo di un'equazione, o una intera linea di un ambiente \env{align} o
\env{gather}. Non ci pu\`o essere alcun tipo di materiale prima o
dopo di esso all'interno della stessa struttura contenente:
\begin{verbatim}
\begin{equation}
\left\{ % <-- Proibito
\begin{split}
...
\end{split}
\right. % <-- Proibito
\end{equation}
\end{verbatim}
\end{error}

\section{Output sbagliato}

\subsection{Sezioni numerate 0.1, 5.1, 8.1 invece che 1, 2, 3}
\label{numinverse}

Questo molto probabilmente significa che gli argomenti di \cn{numberwithin}
sono stati inseriti alla rovescia:
\begin{verbatim}
\numberwithin{section}{equation}
\end{verbatim}
Questo significa ``stampa il numero di sezione nella forma
\textit{numero-equazione}.\textit{numero-sezione} e ricomincia da
1 ogni volta che incontri
un'equazione'' mentre probabilmente si voleva ottenere l'effetto opposto
\begin{verbatim}
\numberwithin{equation}{section}
\end{verbatim}

\subsection{Il comando \cn{numberwithin} non ha avuto effetto sui numeri
di equazione}

State guardando la prima sezione del vostro documento? Controllate la
numerazione delle equazioni in altre parti del documento per vedere se
il problema \`e quello descritto in \secref{numinverse}.

%%%%%%%%%%%%%%%%%%%%%%%%%%%%%%%%%%%%%%%%%%%%%%%%%%%%%%%%%%%%%%%%%%%%%%%%
% by RZ

\chapter{Ulteriori informazioni}

\section{Convertire documenti gi\`a scritti}

\subsection{Convertire da \LaTeX{} ``puro''} %%%% plain

Sotto molti aspetti, un documento \LaTeX{} continua a funzionare
allo stesso modo quando al preambolo del documento si aggiunge
\verb'\usepackage{amsmath}'. Il pacchetto \pkg{amsmath} sopprime
per\`o, salvo diversa indicazione, le interruzioni di pagina all'interno di
strutture che contengono equazioni in \emph{display} come \env{eqnarray},
\env{align} e \env{gather}. Per continuare a permettere le
interruzioni di pagina all'interno di \env{eqnarray} dopo essere
passati al pacchetto \pkg{amsmath}, \`e necessario aggiungere la
seguente riga nel preambolo del documento:
\begin{verbatim}
\allowdisplaybreaks[1]
\end{verbatim}
Per assicurare una spaziatura normale attorno ai simboli di relazione,
si dove sostituire \env{eqnarray} con \env{align}, \env{multline} o
\env{equation}\slash\env{split}, in maniera appropriata.

La maggior parte delle altre differenze d'uso del pacchetto
\pkg{amsmath} possono essere considerate raffinatezze facoltative,
come per esempio l'uso di
\begin{verbatim}
\DeclareMathOperator{\Hom}{Hom}
\end{verbatim}
invece di \verb'\newcommand{\Hom}{\mbox{Hom}}'.

\subsection{Convertire da \amslatex/ 1.1}
Si veda \fn{diffs-m.txt}.

\section{Note tecniche}
Il file \fn{technote.tex} contiene alcuni commenti su diverse questioni
che difficilmente possono essere di interesse generale.

\section{Ottenere aiuto}

Domande o commenti riguardanti \pkg{amsmath} e pacchetti correlati
dovrebbero essere inviati a:
\begin{infoaddress}
American Mathematical Society\\
Technical Support\\
Electronic Products and Services\\
P. O. Box 6248\\
Providence, RI 02940\\[3pt]
Phone: 800-321-4AMS (321-4267) \quad or \quad 401-455-4080\\
Internet: \mail{tech-support@ams.org}
\end{infoaddress}
Quando si riporta un problema occorre includere, per consentire
un'indagine adeguata, le seguenti informazioni:

\begin{enumerate}
\item Il file sorgente in cui \`e sorto il problema, preferibilmente
  ridotto alle minime dimensioni rimuovendo tutto il materiale che pu\`o
  essere rimosso senza interferire sul problema in questione.
\item Un file di log di \latex/ che mostri il messaggio di errore (se
  presente) e i numeri di versione delle classi di documento e file di
  opzioni in uso.
\end{enumerate}

\section{Di possibile interesse}\label{a:possible-interest}
\`E possibile avere informazioni su come ottenere i \emph{font} AMS o
altro materiale relativo a \tex/ dall'archivio Internet AMS
\fn{e-math.ams.org} inviando una richiesta attraverso la posta
elettronica a: \mail{webmaster@ams.org}.

Si possono avere informazioni su come ottenere dall'AMS la distribuzione
\pkg{amsmath} su dischetti da:
\begin{infoaddress}
American Mathematical Society\\
Customer Services\\
P. O. Box 6248\\
Providence, RI 02940\\[3pt]
Phone: 800-321-4AMS (321-4267) \quad or \quad 401-455-4000\\
Internet: \mail{cust-serv@ams.org}
\end{infoaddress}

Il ``\tex/ Users Group\index{TeX Users@\tex/ Users Group}''
\`e una organizzazione senza scopo di lucro che pubblica una
rivista (\journalname{TUGboat}\index{TUGboat@\journalname{TUGboat}}),
organizza meeting, e serve da punto di smistamento per informazioni
su \tex/ e software relativo ad esso.
\begin{infoaddress}
\tex/ Users Group\\
PO Box 2311\\
Portland, OR 97208-2311\\
USA\\[3pt]
Phone: +1-503-223-9994\\
Email: \mail{office@tug.org}
\end{infoaddress}
Iscriversi al ``\tex/ Users Group'' \`e un buon modo per sostenere il
continuo sviluppo di software libero relativo a \tex/.
Esistono inoltre molti ``\tex/ users group'' locali in altri stati;
si possono ottenere informazioni su come contattare un gruppo locale dal
``\tex/ Users Group''.

Esiste un gruppo di discussione Usenet chiamato \fn{comp.text.tex},
che \`e una buona fonte di informazioni su \latex/ e
\tex/ in generale. Se non si sa come leggere un gruppo di discussione,
occorre chiedere all'amministratore di sistema locale se \`e
disponibile un servizio di lettura di \emph{newsgroup}.


\begin{thebibliography}{9}
\addcontentsline{toc}{chapter}{Bibliografia}

\bibitem{amsfonts}\booktitle{AMSFonts version \textup{2.2}\mdash user's guide},
Amer. Math. Soc., Providence, RI, 1994; distribuito
con il pacchetto AMSFonts.

\bibitem{instr-l}\booktitle{Instructions for preparation of
papers and monographs\mdash \amslatex/},
Amer. Math. Soc., Providence, RI, 1996, 1999.

\bibitem{amsthdoc}\booktitle{Using the \pkg{amsthm} Package},
Amer. Math. Soc., Providence, RI, 1999.

\bibitem{tlc} Michel Goossens, Frank Mittelbach e Alexander Samarin,
\booktitle{The \latex/ companion}, Addison-Wesley, Reading, MA, 1994.
  [\emph{Note: L'edizione del 1994 non \`e una guida affidabile per il
    pacchetto \pkg{amsmath} a meno che non ci si riferisca al file
    \fn{compan.err}, distribuito con \LaTeX{}, che contiene una errata
    corrige per il Capitolo 8\mdash.}]

% Deal with a line breaking problem
\begin{raggedright}
\bibitem{mil} G. Gr\"{a}tzer,
\emph{Math into \LaTeX{}: An Introduction to \LaTeX{} and AMS-\LaTeX{}}
  \url{http://www.ams.org/cgi-bin/bookstore/bookpromo?fn=91&arg1=bookvideo&itmc=MLTEX},
Birkh\"{a}user, Boston, 1995.\par
\end{raggedright}

\bibitem{kn} Donald E. Knuth, \booktitle{The \tex/book},
Addison-Wesley, Reading, MA, 1984.

\bibitem{lamport} Leslie Lamport, \booktitle{\latex/: A document preparation
system}, 2nd revised ed., Addison-Wesley, Reading, MA, 1994.

\bibitem{msf} Frank Mittelbach and Rainer Sch\"opf,
\textit{The new font family selection\mdash user
interface to standard \latex/}, \journalname{TUGboat} \textbf{11},
no.~2 (June 1990), pp.~297\ndash 305.

\bibitem{jt} Michael Spivak, \booktitle{The joy of \tex/}, 2nd revised ed.,
Amer. Math. Soc., Providence, RI, 1990.

\end{thebibliography}
% by RZ - end

%%%% ====================================================================
%%% amsldoc.tex 2.07
%%% ====================================================================
\documentclass[a4paper,leqno,titlepage,openany]{amsldoc}[1999/12/13]
\usepackage[italian]{babel}
\renewcommand{\errexa}{\par\noindent\textit{Esempio}:\ }
\renewcommand{\errexpl}{\par\noindent\textit{Spiegazione}:\ }
\DeclareRobustCommand{\cls}[1]{{\ntt#1}%
  \autoindex{#1@\string\cls{#1}, classe}}
\DeclareRobustCommand{\pkg}[1]{{\ntt#1}%
  \autoindex{#1@\string\pkg{#1}, pacchetto}}
\DeclareRobustCommand{\opt}[1]{{\ntt#1}%
  \autoindex{#1@\string\opt{#1}, opzione}}
\DeclareRobustCommand{\env}[1]{{\ntt#1}%
  \autoindex{#1@\string\env{#1}, ambiente}}
\DeclareRobustCommand{\fn}[1]{{\ntt#1}%
   \autoindex{#1@\string\fn{#1}}}
\DeclareRobustCommand{\bst}[1]{{\ntt#1}\autoindex{#1@{\string\ntt{}#1,
  stile bibliografico}}}

\ifx\UndEfiNed\url
  \ClassError{amsldoc}{%
    This version of amsldoc.tex must be processed\MessageBreak
    with a newer version of amsldoc.cls (2.02 or later)}{}
\fi

\title{Manuale utente per il pacchetto \pkg{amsmath} (versione~2.0)}
\author{American Mathematical Society}
\date{13/12/1999}

%    Use the amsmath package and amscd package in order to print
%    examples.
\usepackage{amsmath}
\usepackage{amscd}
% Inserito il pacchetto makeidx - GD
\usepackage{makeidx}

\makeindex % generate index data
\providecommand{\see}[2]{\textit{vedi} #1}

%    The amsldoc class includes a number of features useful for
%    documentation about TeX, including:
%
%    ---Commands \tex/, \amstex/, \latex/, ... for uniform treatment
%    of the various logos and easy handling of following spaces.
%
%    ---Commands for printing various common elements: \cn for command
%    names, \fn for file names (including font-file names), \env for
%    environments, \pkg and \cls for packages and classes, etc.

%    Many of the command names used here are rather long and will
%    contribute to poor linebreaking if we follow the \latex/ practice
%    of not hyphenating anything set in tt font; instead we selectively
%    allow some hyphenation.
\allowtthyphens % defined in amsldoc.cls

\hyphenation{ac-cent-ed-sym-bol add-to-counter add-to-length align-at
  aligned-at allow-dis-play-breaks ams-art ams-cd ams-la-tex amsl-doc
  ams-symb ams-tex ams-text ams-xtra bmatrix bold-sym-bol cen-ter-tags
  eqn-ar-ray idots-int int-lim-its latex med-space neg-med-space
  neg-thick-space neg-thin-space no-int-lim-its no-name-lim-its
  over-left-arrow over-left-right-arrow over-right-arrow pmatrix
  qed-sym-bol set-length side-set small-er tbinom the-equa-tion
  thick-space thin-space un-der-left-arrow un-der-left-right-arrow
  un-der-right-arrow use-pack-age var-inj-lim var-proj-lim vmatrix
  xalign-at xx-align-at}

%    Prepare for illustrating the \vec example
\newcommand{\vect}[1]{\mathbf{#1}}

\newcommand{\booktitle}[1]{\textit{#1}}
\newcommand{\journalname}[1]{\textit{#1}}
\newcommand{\seriesname}[1]{\textit{#1}}

%    Command to insert and index a particular phrase. Doesn't work for
%    certain kinds of special characters in the argument.
\newcommand{\ii}[1]{#1\index{#1}}

\newcommand{\vstrut}[1]{\vrule width0pt height#1\relax}

%    An environment for presenting comprehensive address information:
\newenvironment{infoaddress}{%
  \par\topsep\medskipamount
  \trivlist\centering
  \item[]%
  \begin{minipage}{.7\columnwidth}%
  \raggedright
}{%
  \end{minipage}%
  \endtrivlist
}

\newenvironment{eqxample}{%
  \par\addvspace\medskipamount
  \noindent\begin{minipage}{.5\columnwidth}%
  \def\producing{\end{minipage}\begin{minipage}{.5\columnwidth}%
    \hbox\bgroup\kern-.2pt\vrule width.2pt%
      \vbox\bgroup\parindent0pt\relax
%    The 3pt is to cancel the -\lineskip from \displ@y
    \abovedisplayskip3pt \abovedisplayshortskip\abovedisplayskip
    \belowdisplayskip0pt \belowdisplayshortskip\belowdisplayskip
    \noindent}
}{%
  \par
%    Ensure that a lonely \[\] structure doesn't take up width less than
%    \hsize.
  \hrule height0pt width\hsize
  \egroup\vrule width.2pt\kern-.2pt\egroup
  \end{minipage}%
  \par\addvspace\medskipamount
}

%    The chapters are so short, perhaps we shouldn't call them by the
%    name `Chapter'. We make \chaptername read an argument in order to
%    remove a following \space or "{} " (both possibilities are present
%    in book.cls).

\renewcommand{\chaptername}[1]{}
\newcommand{\chapnum}[1]{\mdash #1\mdash }
\makeatletter
\def\@makechapterhead#1{%
  \vspace{1.5\baselineskip}%
  {\parindent \z@ \raggedright \reset@font
    \ifnum \c@secnumdepth >\m@ne
      \large\bfseries \chapnum{\thechapter}%
      \par\nobreak
      \vskip.5\baselineskip\relax
    \fi
    #1\par\nobreak
    \vskip\baselineskip
  }}
\makeatother

%    A command for ragged-right parbox in a tabular.
\newcommand{\rp}{\let\PBS\\\raggedright\let\\\PBS}

%    Non-indexed file name
\newcommand{\nfn}[1]{\texttt{#1}}

%    For the examples in the math spacing table.
%%\newcommand{\lspx}{\mbox{\rule{5pt}{.6pt}\rule{.6pt}{6pt}}}
%%\newcommand{\rspx}{\mbox{\rule[-1pt]{.6pt}{7pt}%
%%  \rule[-1pt]{5pt}{.6pt}}}
\newcommand{\lspx}{\mathord{\Rightarrow\mkern-1mu}}
\newcommand{\rspx}{\mathord{\mkern-1mu\Leftarrow}}
\newcommand{\spx}[1]{$\lspx #1\rspx$}

%    For a list of characters representing document input.
\newcommand{\clist}[1]{%
  \mbox{\ntt\spaceskip.2em plus.1em \xspaceskip\spaceskip#1}}

%    Fix weird \latex/ definition of rightmark.
\makeatletter
\def\rightmark{\expandafter\@rightmark\botmark{}{}}
%    Also turn off section marks.
\let\sectionmark\@gobble
\renewcommand{\chaptermark}[1]{%
  \uppercase{\markboth{\rhcn#1}{\rhcn#1}}}
\newcommand{\rhcn}{\thechapter. }
\makeatother

%    Include down to \section but not \subsection, in toc:
\setcounter{tocdepth}{1}

\DeclareMathOperator{\ix}{ix}
\DeclareMathOperator{\nul}{nul}
\DeclareMathOperator{\End}{End}
\DeclareMathOperator{\xxx}{xxx}

\begin{document}

%%%%%%%%%%%%%%%%%%%%%%%%%%%%%%%%%%%%%%%%%%%%%%%%%%%%%%%%%%%%%%%%%%%%%%%%
\frontmatter

\maketitle
%
%
%
\pagebreak
\begin{small} 
 \noindent Titolo originale: \emph{User manual for the \pkg{amsmath} package (version~2.0)}

 \smallskip
 \noindent Traduzione:

 \begin{quote}
 \flushleft %  \footnotesize
 Giulio Agostini, % <giulio.agostini@bigfoot.com>
 Giuseppe Bilotta, % <bourbaki@bigfoot.com>
 Flavio Casadei Della Chiesa, % <flavio_c@libero.it>
 Onofrio de Bari, % <thufir@tin.it>
 Giacomo Delre, % <giader@penguinpowered.com>
 Luca Ferrante, % <ironluke@split.it>
 Tommaso Pecorella, % <t.pecorella@inwind.it>
 Mileto Rigido, % <m.rigido@flashnet.it>
 Roberto Zanasi. % <roberto.zanasi@libero.it>
 \end{quote}

\end{small}
%
%
%
\pagestyle{headings}
\tableofcontents
\cleardoublepage % for better page number placement

%%%%%%%%%%%%%%%%%%%%%%%%%%%%%%%%%%%%%%%%%%%%%%%%%%%%%%%%%%%%%%%%%%%%%%%%
\mainmatter
%%%%%%%%%%%%%%%%%%%%%%%%%%%%%%%%%%%%%%%%%%%%%%%%%%%%%%%%%%%%%%%%%%%%%%%%

%%% Nota dei traduttori
\subsubsection*{Nota alla traduzione italiana}
Una copia di questo documento e altre traduzioni in italiano di
manuali su \LaTeX\ sono reperibili presso
\begin{itemize}
\item\url{http://guild.prato.linux.it}
\item\url{ftp://lorien.prato.linux.it/pub/guild}
\item\url{ftp://ftp.unina.it/pub/TeX/info/italian}
\end{itemize}
e su ogni sito CTAN, per esempio \url{ftp://ftp.tex.ac.uk/tex-archive/info/italian}.
%%%%%%%%%%%%%%%%%%%%%%%%%%%%%%%%%%%%%%%%%%%%%%%%%%%%%%%%%%%%%%%%%%%%%%%%


% by GA
\chapter{Introduzione}

Il pacchetto \pkg{amsmath} \`e un pacchetto \LaTeX{} che fornisce
svariate estensioni per il miglioramento della struttura informativa e
della stampa di documenti che contengono formule matematiche. I lettori
che non conoscono \LaTeX{} sono invitati a consultare \cite{lamport}.
Se si possiede una versione aggiornata di \LaTeX{}, il pacchetto \pkg{amsmath}
\`e normalmente incluso. Quando viene pubblicata una nuova versione del
pacchetto \pkg{amsmath}, \`e possibile effettuare un aggiornamento attraverso
\url{http://www.ams.org/tex/amsmath.html} o
\url{ftp://ftp.ams.org/pub/tex/}.

Questo documento descrive le funzionalit\`a del paccheto \pkg{amsmath}
e spiega come dovrebbero essere usate. Esso copre inoltre alcuni pacchetti
ausiliari:
\begin{ctab}{ll}
\pkg{amsbsy}& \pkg{amstext}\\
\pkg{amscd}& \pkg{amsxtra}\\
\pkg{amsopn}
\end{ctab}
Tutti questi hanno a che vedere con il contenuto di formule
matematiche. Per informazioni su ulteriori simboli e \emph{font} matematici,
si veda \cite{amsfonts} e \url{http://www.ams.org/tex/amsfonts.html}.
Per la documentazione del pacchetto \pkg{amsthm} o delle classi AMS
(\cls{amsart}, \cls{amsbook}, etc.\@) si veda \cite{amsthdoc} o
\cite{instr-l} e \url{http://www.ams.org/tex/author-info.html}.

Se siete utenti di \latex/ da molto tempo e avete molta matematica nei
vostri scritti, potreste trovare soluzioni a problemi familiari in
questo elenco di funzionalit\`a di \pkg{amsmath}:
\begin{itemize}

\item Un modo comodo per definire un nuovo comando `nome di operatore', come
\cn{sin} e \cn{lim}, con spazi appropriati ai lati e selezione automatica
di stile e dimensioni corrette del \emph{font} (anche quando usato in esponenti
o deponenti).

\item Diversi alternative all'ambiente \env{eqnarray} per rendere le
diverse disposizioni delle equazioni pi\`u facili da scrivere.

\item I numeri delle equazioni si spostano automaticamente in alto o in
basso per evitare di sovrapporsi con l'equazione stessa (al contrario
di \env{eqnarray}).

\item Gli spazi attorno ai segni di uguaglianza sono gli stessi della
normale spaziatura nell'ambiente \env{equation} (al contrario di
\env{eqnarray}).

\item Un modo per produrre deponenti a pi\`u linee come spesso \`e
richiesto dai simboli di sommatoria e produttoria.

\item Un modo semplice di numerare una determinata equazione con un
riferimento diverso da quello fornito dalla numerazione automatica.

\item Un modo semplice di produrre numerazioni subordinate per le
equazioni, nella forma (1.3a) (1.3b) (1.3c), per un determinato insieme
di equazioni.

\end{itemize}

Il pacchetto \pkg{amsmath} \`e distribuito insieme ad alcuni piccoli
pacchetti ausiliari:
\begin{description}
\item[\pkg{amsmath}] Il pacchetto principale, fornisce diverse funzionalit\`a
  per equazioni in \emph{display} e altri costrutti matematici.

\item[\pkg{amstext}] Fornisce il comando \cn{text} per
  sistemare un frammento di testo in un \emph{display}.

\item[\pkg{amsopn}] Fornisce il comando \cn{DeclareMathOperator} per definire
  nuovi `nomi di operatori' come \cn{sin} e \cn{lim}.

\item[\pkg{amsbsy}] Per compatibilit\`a all'indietro questo pacchetto
  continua a esistere, ma in alternativa ad esso si consiglia l'uso del
  pi\`u recente pacchetto \pkg{bm} fornito a corredo di \LaTeX{}.

\item[\pkg{amscd}] Fornisce un ambiente \env{CD} per semplici diagrammi
  commutativi (privi di frecce diagonali).

\item[\pkg{amsxtra}] Fornisce alcune cianfrusaglie come \cn{fracwithdelims}
  e \cn{accentedsymbol}, per compatibilit\`a con documenti creati usando
  la versione~1.1.

\end{description}

Il pacchetto \pkg{amsmath} incorpora \pkg{amstext}, \pkg{amsopn}, e
\pkg{amsbsy}. Le funzionalit\`a di \pkg{amscd} e \pkg{amsxtra}, invece,
sono disponibili solo invocando separatamente questi pacchetti.

%%%%%%%%%%%%%%%%%%%%%%%%%%%%%%%%%%%%%%%%%%%%%%%%%%%%%%%%%%%%%%%%%%%%%%%%
% by GA

\chapter{Opzioni per il pacchetto \pkg{amsmath}}\label{options}

Il pacchetto \pkg{amsmath} ha le seguenti opzioni:
\begin{description}

\item[\opt{centertags}] (\emph{default}) Centra verticalmente\index{equazioni,
numeri delle!posizionamento verticale}, rispetto all'altezza totale
dell'equazione, la numerazione delle equazioni spezzate su pi\`u linee.

\item[\opt{tbtags}] `Top-or-bottom tags' (Etichette in cima o in fondo):
Allinea la numerazione\index{equazioni, numeri delle!posizionamento
verticale} delle equazioni spezzate su pi\`u linee all'ultima
(rispettivamente alla prima) linea, se i numeri stanno sulla destra
(rispettivamente sulla sinistra).

\item[\opt{sumlimits}] (\emph{default}) Posiziona esponenti e
deponenti\index{esponenti e deponenti!posizionamento}\relax
\index{limiti|see{esponenti e deponenti}} dei simboli di sommatoria
sopra e sotto, nelle equazioni in \emph{display}. Questa opzione
influenza anche altri simboli dello stesso tipo\mdash $\prod$,
$\coprod$, $\bigotimes$, $\bigoplus$, e cos\`\i\ via\mdash eccetto gli
integrali (vedi sotto).

\item[\opt{nosumlimits}] Posiziona gli esponenti e deponenti dei
simboli simil-sommatoria sempre a fianco, anche nelle equazioni in
\emph{display}.

\item[\opt{intlimits}] Come \opt{sumlimits}, ma per i simboli di
integrale\index{integrali!posizionamento dei limiti}.

\item[\opt{nointlimits}] (\emph{default}) Il contrario di \opt{intlimits}.

\item[\opt{namelimits}] (\emph{default}) Come \opt{sumlimits}, ma per certi
`nomi di operatori' come $\det$, $\inf$, $\lim$, $\max$, $\min$, che
tradizionalmente hanno deponenti \index{esponenti e
deponenti!posizionamento} posizionati sotto di essi all'interno di
equazioni \emph{display}.

\item[\opt{nonamelimits}] Il contrario di \opt{namelimits}.

\end{description}

Per usare una di queste opzioni del pacchetto bisogna mettere il nome
dell'opzione nell'argomento opzionale del comando \cn{usepackage}\mdash
ad esempio, \verb"\usepackage[intlimits]{amsmath}".

Il pacchetto \pkg{amsmath} inoltre riconosce le seguenti opzioni che
sono normalmente selezionate (implicitamente o esplicitamente)
attraverso il comando \cn{documentclass}, e che pertanto non hanno
bisogno di essere ripetute nell'elenco di opzioni del comando
\cn{usepackage}|{amsmath}|.
\begin{description}

\item[\opt{leqno}] Posiziona i numeri di equazione sulla
sinistra.\index{equazioni, numeri delle!posizionamento a destra o a
sinistra}

\item[\opt{reqno}] Posiziona i numeri di equazione sulla destra.

\item[\opt{fleqn}] Posiziona i numeri di equazione a una distanza prefissata
dal margine sinistro piuttosto che centrata nella colonna di
testo.\index{equazioni in \emph{display}!centratura}

\end{description}

%%%%%%%%%%%%%%%%%%%%%%%%%%%%%%%%%%%%%%%%%%%%%%%%%%%%%%%%%%%%%%%%%%%%%%%%
% by GB

\chapter{Equazioni in \emph{display}}

\section{Introduzione}

Il pacchetto \pkg{amsmath} fornisce un certo numero di nuove strutture
per le equazioni in \emph{display}\index{equazioni in
\emph{display}}\index{equazioni|see{equazioni in \emph{display}}},
oltre a quelle fornite dal \latex/ di base; fra queste:
\begin{verbatim}
  equation     equation*     align       align*
  gather       gather*       flalign     flalign*
  multline     multline*     alignat     alignat*
  split
\end{verbatim}
(Sebbene l'ambiente standard \env{eqnarray} rimanga disponibile, \`e
opportuno usare \env{align} o \env{equation}+\env{split}, invece.)

Con l'eccezione di \env{split}, ogni ambiente ha sia una versione
stellata sia una non stellata, dove la versione non stellata permette
la numerazione automatica usando il contatore \latex/ \env{equation}.
Si pu\`o sopprimere il numero in ogni singola linea premettendo un
\cn{notag} al codice \cn{\\}; lo si pu\`o anche
scavalcare\index{equazioni, numeri delle!scavalcare} con un valore di
propria scelta, usando il comando \cn{tag}|{|\<etich>|}|, dove \<etich>
\`e un testo arbitrario, come |$*$| o |ii|, usato  per \qq{numerare}
l'equazione. Si pu\`o anche usare il comando \cn{tag*}, che fa in modo
che il testo fornito venga scritto letteralmente, senza aggiunta di
parentesi. \cn{tag} e \cn{tag*} possono anche essere usati nelle
versioni non numerate di tutte le strutture di allineamento di
\pkg{amsmath}. Alcuni esempi dell'uso di \cn{tag} possono essere
trovati nei file di esempio \fn{testmath.tex} e \fn{subeqn.tex}
forniti con il pacchetto \pkg{amsmath}.

L'ambiente \env{split} \`e una speciale forma subordinata, da usare
solo \emph{all'interno} di altre strutture; non pu\`o essere usato in
una \env{multline}.

Nelle strutture d'allineamento (\env{split}, \env{align} e varianti),
i simboli di relazione hanno un \verb'&' prima, ma non dopo\mdash a
differenza di \env{eqnarray}. Mettere un \verb'&' dopo il simbolo di
relazione interferirebbe con la spaziatura: \`e necessario metterlo
prima.

\begin{table}[p]
\caption[]{Confronto degli ambienti per le equazioni in \emph{display}
(le linee verticali indicano i margini nominali)}\label{displays}
\renewcommand{\theequation}{\arabic{equation}}
\begin{eqxample}
\begin{verbatim}
\begin{equation*}
a=b
\end{equation*}
\end{verbatim}
\producing
\begin{equation*}
a=b
\end{equation*}
\end{eqxample}

\begin{eqxample}
\begin{verbatim}
\begin{equation}
a=b
\end{equation}
\end{verbatim}
\producing
\begin{equation}
a=b
\end{equation}
\end{eqxample}

\begin{eqxample}
\begin{verbatim}
\begin{equation}\label{xx}
\begin{split}
a& =b+c-d\\
 & \quad +e-f\\
 & =g+h\\
 & =i
\end{split}
\end{equation}
\end{verbatim}
\producing
\begin{equation}\label{xx}
\begin{split}
a& =b+c-d\\
 & \quad +e-f\\
 & =g+h\\
 & =i
\end{split}
\end{equation}
\end{eqxample}

\begin{eqxample}
\begin{verbatim}
\begin{multline}
a+b+c+d+e+f\\
+i+j+k+l+m+n
\end{multline}
\end{verbatim}
\producing
\begin{multline}
a+b+c+d+e+f\\
+i+j+k+l+m+n
\end{multline}
\end{eqxample}

\begin{eqxample}
\begin{verbatim}
\begin{gather}
a_1=b_1+c_1\\
a_2=b_2+c_2-d_2+e_2
\end{gather}
\end{verbatim}
\producing
\begin{gather}
a_1=b_1+c_1\\
a_2=b_2+c_2-d_2+e_2
\end{gather}
\end{eqxample}

\begin{eqxample}
\begin{verbatim}
\begin{align}
a_1& =b_1+c_1\\
a_2& =b_2+c_2-d_2+e_2
\end{align}
\end{verbatim}
\producing
\begin{align}
a_1& =b_1+c_1\\
a_2& =b_2+c_2-d_2+e_2
\end{align}
\end{eqxample}

\begin{eqxample}
\begin{verbatim}
\begin{align}
a_{11}& =b_{11}&
  a_{12}& =b_{12}\\
a_{21}& =b_{21}&
  a_{22}& =b_{22}+c_{22}
\end{align}
\end{verbatim}
\producing
\begin{align}
a_{11}& =b_{11}&
  a_{12}& =b_{12}\\
a_{21}& =b_{21}&
  a_{22}& =b_{22}+c_{22}
\end{align}
\end{eqxample}

\begin{eqxample}
\begin{verbatim}
\begin{flalign*}
a_{11}& =b_{11}&
  a_{12}& =b_{12}\\
a_{21}& =b_{21}&
  a_{22}& =b_{22}+c_{22}
\end{flalign*}
\end{verbatim}
\producing
\begin{flalign*}
a_{11}& =b_{11}&
  a_{12}& =b_{12}\\
a_{21}& =b_{21}&
  a_{22}& =b_{22}+c_{22}
\end{flalign*}
\end{eqxample}
\end{table}

\section{Singole equazioni}

L'ambiente \env{equation} viene usato per singole equazioni con
numerazione automatica; l'ambiente \env{equation*} ha la stessa
funzione, senza numerazione.%
%%%%%%%%%%%%%%%%%%%%%%%%%%%%%%%%%%%%%%%%%%%%%%%%%%%%%%%%%%%%%%%%%%%%%%%%
\footnote{\latex/ non fornisce un ambiente \env{equation*}, ma un
ambiente con funzioni analoghe: \env{displaymath}.}

\section{Equazioni spezzate senza allineamento}

L'ambiente \env{multline} \`e una variante di \env{equation}, usata
per le equazioni che non entrano in un'unica riga. La prima riga di una
\env{multline} sar\`a al margine sinistro, e l'ultima al margine
destro, tranne per un rientro ambo i lati, di lunghezza
\cn{multlinegap}; tutte le altre linee verranno centrate
indipendentemente considerando la larghezza del \emph{display} (a meno
che non sia in funzione l'opzione \opt{fleqn}).

Come \env{equation}, \env{multline} fornisce un'unico numero
d'equazione (quindi, nessuna delle singole linee dovrebbe essere
segnata con \cn{notag}). Il numero dell'equazione \`e posto all'ultima
riga (opzione \opt{reqno}) o sulla prima linea (opzione \opt{leqno});
il centramento verticale (come per \env{split}) non \`e supportato in
\env{multline}.

\`E possibile forzare una delle righe di centro a sinistra o a destra
con i comandi \cn{shoveleft}, \cn{shoveright}; questi comandi prendono
l'intera linea come argomento, fino al segno \cn{\\} escluso; ad
esempio
\begin{multline}
\framebox[.65\columnwidth]{A}\\
\framebox[.5\columnwidth]{B}\\
\shoveright{\framebox[.55\columnwidth]{C}}\\
\framebox[.65\columnwidth]{D}
\end{multline}
\begin{verbatim}
\begin{multline}
\framebox[.65\columnwidth]{A}\\
\framebox[.5\columnwidth]{B}\\
\shoveright{\framebox[.55\columnwidth]{C}}\\
\framebox[.65\columnwidth]{D}
\end{multline}
\end{verbatim}

Il valore di \cn{multlinegap} pu\`o essere cambiato con i soliti
comandi \latex/ \cn{setlength} or \cn{addtolength}.

\section{Equazioni spezzate con allineamento}

Come \env{multline}, l'ambiente \env{split} \`e per \emph{singole}
equazioni troppo lunghe per entrare in una riga e che pertanto devono
essere spezzate. A differenza di \env{multline}, per\`o, l'ambiente
\env{split} permette allineamento tra le linee, con l'uso di simboli
|&| per segnare i punti di allineamento. A differenza di altre
strutture di equazioni \pkg{amsmath}, l'ambiente \env{split} non
produce numeri, poich\'e \`e progettato per essere usato
\emph{esclusivamente all'interno di qualche altra struttura per
equazioni in \emph{display}}, solitamente un ambiente \env{equation},
\env{align}, o \env{gather}, che fornisce la numerazione; ad esempio:
\begin{equation}\label{e:barwq}\begin{split}
H_c&=\frac{1}{2n} \sum^n_{l=0}(-1)^{l}(n-{l})^{p-2}
\sum_{l _1+\dots+ l _p=l}\prod^p_{i=1} \binom{n_i}{l _i}\\
&\quad\cdot[(n-l )-(n_i-l _i)]^{n_i-l _i}\cdot
\Bigl[(n-l )^2-\sum^p_{j=1}(n_i-l _i)^2\Bigr].
\kern-2em % adjust equation body to the right [mjd,13-Nov-1994]
\end{split}\end{equation}

\begin{verbatim}
\begin{equation}\label{e:barwq}\begin{split}
H_c&=\frac{1}{2n} \sum^n_{l=0}(-1)^{l}(n-{l})^{p-2}
\sum_{l _1+\dots+ l _p=l}\prod^p_{i=1} \binom{n_i}{l _i}\\
&\quad\cdot[(n-l )-(n_i-l _i)]^{n_i-l _i}\cdot
\Bigl[(n-l )^2-\sum^p_{j=1}(n_i-l _i)^2\Bigr].
\end{split}\end{equation}
\end{verbatim}

La struttura \env{split} dovrebbe costituire l'intero corpo della
struttura racchiudente, tranne per comandi come \cn{label} che non
producono testo visibile.

\section{Gruppi di equazioni senza allineamento}

L'ambiente \env{gather} viene usato per ragguppare equazioni
consecutive quando non vi \`e necessit\`a di allineamento; ogni
equazione \`e centrata separatamente entro i margini (come in
Tabella~\ref{displays}). Le equazioni in un ambiente \env{gather} sono
separati da comandi \cn{\bslash}. Ogni equazione \env{gather} pu\`o
essere un blocco \verb'\begin{split}'
  \dots\ \verb'\end{split}' \mdash ad esempio:
\begin{verbatim}
\begin{gather}
  prima equazione\\
  \begin{split}
    seconda & equazione\\
           & su due linee
  \end{split}
  \\
  terza equazione
\end{gather}
\end{verbatim}

\section{Gruppi di equazioni con allineamento reciproco}

L'ambiente \env{align} \`e usato per gruppi di due o pi\`u  equazioni
quando \`e richiesto allineamento reciproco; di solito vengono scelti
i simboli di relazione per gli allineamenti (come in
Tabella~\ref{displays}).

Per avere pi\`u colonne di equazioni affiancate, si possono usare
simboli di ``e'' commerciale aggiuntivi per separare le colonne:
\begin{align}
x&=y       & X&=Y       & a&=b+c\\
x'&=y'     & X'&=Y'     & a'&=b\\
x+x'&=y+y' & X+X'&=Y+Y' & a'b&=c'b
\end{align}
%
\begin{verbatim}
\begin{align}
x&=y       & X&=Y       & a&=b+c\\
x'&=y'     & X'&=Y'     & a'&=b\\
x+x'&=y+y' & X+X'&=Y+Y' & a'b&=c'b
\end{align}
\end{verbatim}
Annotazioni linea-per-linea sulle equazioni possono essere ottenute
con un opportuno uso di \cn{text} in un ambiente \env{align}:
\begin{align}
x& = y_1-y_2+y_3-y_5+y_8-\dots
                    && \text{per \eqref{eq:C}}\\
 & = y'\circ y^*    && \text{per \eqref{eq:D}}\\
 & = y(0) y'        && \text {per l'Assioma 1.}
\end{align}
%
\begin{verbatim}
\begin{align}
x& = y_1-y_2+y_3-y_5+y_8-\dots
                    && \text{per \eqref{eq:C}}\\
 & = y'\circ y^*    && \text{per \eqref{eq:D}}\\
 & = y(0) y'        && \text {per l'Assioma 1.}
\end{align}
\end{verbatim}
Una variante, l'ambiente \env{alignat}, permette di specificare
manualmente lo spazio orizzontale fra le equazioni; questo ambiente ha
un argomento obbligatorio, il numero di \qq{colonne di equazioni}: si
contano il numero di \verb'&' in una riga, si aggiunge 1 e si divide
per 2.
\begin{alignat}{2}
x& = y_1-y_2+y_3-y_5+y_8-\dots
                  &\quad& \text{per \eqref{eq:C}}\\
 & = y'\circ y^*  && \text{per \eqref{eq:D}}\\
 & = y(0) y'      && \text {per l'Assioma 1.}
\end{alignat}
%
\begin{verbatim}
\begin{alignat}{2}
x& = y_1-y_2+y_3-y_5+y_8-\dots
                  &\quad& \text{per \eqref{eq:C}}\\
 & = y'\circ y^*  && \text{per \eqref{eq:D}}\\
 & = y(0) y'      && \text {per l'Assioma 1.}
\end{alignat}
\end{verbatim}

\section{Blocchi per costrutti allineati}

Come \env{equation}, gli ambienti a equazioni multiple \env{gather},
\env{align} e \env{alignat} sono progettati per produrre strutture
aventi lunghezza complessiva pari alla lunghezza di una riga; questo
implica, ad esempio, che non \`e facile aggiungere parentesi attorno
alle strutture; vengono quindi fornite le varianti \env{gathered},
\env{aligned} e \env{alignedat}, la cui lunghezza totale \`e pari alla
reale lunghezza dei contenuti; possono quindi essere usate come
componenti di un'espressione pi\`u complessa; ad esempio,
\begin{equation*}
\left.\begin{aligned}
  B'&=-\partial\times E,\\
  E'&=\partial\times B - 4\pi j,
\end{aligned}
\right\}
\qquad \text{equazioni di Maxwell}
\end{equation*}
\begin{verbatim}
\begin{equation*}
\left.\begin{aligned}
  B'&=-\partial\times E,\\
  E'&=\partial\times B - 4\pi j,
\end{aligned}
\right\}
\qquad \text{equazioni di Maxwell}
\end{equation*}
\end{verbatim}
Come l'ambiente \env{array}, le varianti \texttt{-ed} possono
accettare un argomento facoltativo \verb'[t]' o \verb'[b]' per
specificare il posizionamento verticale.

Costrutti di tipo \qq{casi} come il seguente sono comuni in matematica:
\begin{equation}\label{eq:C}
P_{r-j}=
  \begin{cases}
    0&  \text{se $r-j$ \`e dispari},\\
    r!\,(-1)^{(r-j)/2}&  \text{se $r-j$ \`e pari}.
  \end{cases}
\end{equation}
e nel pacchetto \pkg{amsmath} c'\`e un ambiente \env{cases} per
facilitarne la scrittura:
\begin{verbatim}
P_{r-j}=
  \begin{cases}
    0&  \text{se $r-j$ \`e dispari},\\
    r!\,(-1)^{(r-j)/2}&  \text{se $r-j$ \`e pari}.
  \end{cases}
\end{verbatim}
Osservare l'uso di \cn{text} (cfr.~\secref{text}) e della matematica
annidata nella precedente formula.

\section{Correggere il posizionamento dei tag}

Posizionare i numeri delle equazioni in blocchi multilinea pu\`o essere
un problema piuttosto complesso; gli ambienti del pacchetto
\pkg{amsmath} fanno il possibile per evitare di sovrascrivere le
equazioni con il numero, eventualmente spostando il numero pi\`u in
alto o pi\`u in basso su una riga diversa; le difficolt\`a nel calcolo
preciso del profilo di un'equazione possono talvolta risultare in
spostamenti inopportuni dei numeri: si pu\`o allora usare il comando
\cn{raisetag}, fornito proprio per regolare manualmente la posizione
verticale del numero dell'equazione attiva, se \`e stato spostato
dalla sua posizione normale: per spostare ad esempio un particolare
numero in alto di sei punti, si scrive |\raisetag{6pt}|; questo tipo
di correzione \`e un lavoro di precisione come le interruzioni di riga
o di pagina, e andrebbe quindi lasciato fino a quando il documento
non sia ormai quasi completo, poich\'e si rischierebbe altrimenti di
dover disfare e rifare una correzione pi\`u volte, per tenersi al passo
con i cambiamenti del contenuto del documento.

\section{Spaziatura verticale e interruzioni di pagina in
\emph{display} su pi\`u linee}

Come nel \latex/, si pu\`o usare il comando \cn{\\}|[|\<dimensione>|]|
per ottenere spazi verticale aggiuntivi in tutti gli ambienti di
equazioni a blocchi del pacchetto \pkg{amsmath}. Quando si usa il
pacchetto \pkg{amsmath}, le \ii{interruzioni di pagina} tra le righe
delle equazioni sono normalmente impedite; la filosofia di ci\`o \`e
che le interruzioni di pagina in questo tipo di materiale dovrebbero
essere scelto dall'autore nei vari casi; per ottenere un'interruzione
di pagina in una particolare equazione in \emph{display}, si pu\`o
usare il comando \cn{displaybreak}; il luogo migliore dove posizionare
un \cn{displaybreak} \`e immediatamente prima del \cn{\\} dove si vuole
che abbia effetto; come il comando \latex/ \cn{pagebreak},
\cn{displaybreak} accetta un argomento opzionale, tra 0 e 4, per
indicare la opportunit\`a dell'interruzione. |\displaybreak[0]|
significa \qq{\`e possibile interrompere qui}, senza incoraggiare
l'interruzione; \cn{displaybreak} senza argomento equivale a
|\displaybreak[4]| e forza l'interruzione.

Se si preferisce permettere le interruzioni di pagina dove capita,
anche in mezzo a una equazione su pi\`u linee, si pu\`o usare
\cn{allowdisplaybreaks}\texttt{[1]} nel preambolo del documento. Un
argomento 1\ndash 4 pu\`o essere usato per un controllo pi\`u fine:
|[1]| permette le interruzioni, evitandole tuttavia il pi\`u
possibile; valori 2,3,4 indicano una permissivit\`a maggiore. Quando
le interruzioni sono abilitate con \cn{allowdisplaybreaks}, il comando
\cn{\\*} pu\`o essere usato, come al solito, per impedire
un'interruzione di pagina a una ben precisa riga.

\begin{bfseries}
Nota: alcuni ambienti di equazioni racchiudono il loro contenuto in
una scatola indistruttibile, con la conseguenza che n\'e
\cn{displaybreak}, n\'e \cn{allowdisplaybreaks} avranno effetto su di
loro; tra questi ambienti vi sono \env{split}, \env{aligned},
\env{gathered} e \env{alignedat}.
\end{bfseries}

\section{Interrompere i \emph{display}}

Il comando \cn{intertext} pu\`o essere usato per una breve inserzione
di una o due righe di testo\index{frammenti di testo in matematica} in
un \emph{display} su pi\`u righe (cfr. il comando \cn{text} in
\secref{text}): la sua caratteristica principale \`e il mantenimento
dell'allineamento, cosa che non avverrebbe se si terminasse il blocco
per ricominciarlo pi\`u avanti. \cn{intertext} pu\`o comparire solo
dopo un comando \cn{\\} o \cn{\\*}. Notare la posizione della parola
\qq{e} in questo esempio.
\begin{align}
A_1&=N_0(\lambda;\Omega')-\phi(\lambda;\Omega'),\\
A_2&=\phi(\lambda;\Omega')-\phi(\lambda;\Omega),\\
\intertext{e}
A_3&=\mathcal{N}(\lambda;\omega).
\end{align}
\begin{verbatim}
\begin{align}
A_1&=N_0(\lambda;\Omega')-\phi(\lambda;\Omega'),\\
A_2&=\phi(\lambda;\Omega')-\phi(\lambda;\Omega),\\
\intertext{e}
A_3&=\mathcal{N}(\lambda;\omega).
\end{align}
\end{verbatim}

\section{Numerazione delle equazioni}

\subsection{Gerarchia della numerazione}
Con il \latex/ se si vogliono numerare le equazioni secondo le
sezioni\mdash cio\`e, con numeri di equazione tipo (1.1), (1.2), \dots,
(2.1), (2.2), \dots, nelle sezioni 1, 2, e cos\`{\i} via\mdash
bisognava ridefinire \cn{theequation} come suggerito nel manuale del
\latex/ \cite[\S6.3, \S C.8.4]{lamport}:
\begin{verbatim}
\renewcommand{\theequation}{\thesection.\arabic{equation}}
\end{verbatim}

Ci\`o funziona piuttosto bene, tranne per il fatto che il contatore
delle equazioni non viene reimpostato a zero all'inizio di un nuovo
capitolo o sezione, a meno di non farlo manualmente con
\cn{setcounter}; per facilitare il procedimento, il pacchetto
\pkg{amsmath} fornisce il comando\index{equazioni, numeri
delle!gerarchia} \cn{numberwithin}. Per legare la numerazione delle
equazioni alla numerazione delle sezioni, con reimpostazione
automatica dei contatori, si pu\`o usare
\begin{verbatim}
\numberwithin{equation}{section}
\end{verbatim}
Come suggerito dal nome, il comando \cn{numberwithin} pu\`o essere
applicato a qualunque contatore, non solo al contatore
\texttt{equation}.

\subsection{Riferimenti incrociati ai numeri delle equazioni}

Per facilitare i riferimenti incrociati alle equazioni, \`e stato
creato il comando \cn{eqref}\index{equazioni, numeri delle!riferimenti
incrociati}, che fornisce automaticamente le parentesi attorno al
numero: cos\`{\i}, mentre \verb'\ref{abc}' produce 3.2,
\verb'\eqref{abc}' produce (3.2).

\subsection{Numerazione subordinata}

Il pacchetto \pkg{amsmath} fornisce anche un ambiente
\env{subequations}\index{equazioni, numeri delle!numerazione delle
subordinate} per facilitare la numerazione delle equazioni di un
gruppo con uno schema subordinato; ad esempio,
\begin{verbatim}
\begin{subequations}
...
\end{subequations}
\end{verbatim}
fa in modo che tutte le equazioni numerate in quella parte del
documento vengano numerate con (4.9a) (4.9b) (4.9c) \dots, se la
precedente equazione aveva numero (4.8). Un comando \cn{label} subito
dopo \verb/\begin{subequations}/ produrr\`a un \cn{ref} al numero
genitore 4.9, non a 4.9a; i contatori usati dall'ambiente
\env{subequations} sono \verb/parentequation/ e \verb/equation/;
\cn{addtocounter}, \cn{setcounter}, \cn{value} etc.\ possono essere
applicati come al solito ai nomi di questi contatori; per ottenere
qualcosa di diverso dalle lettere minuscole per i numeri delle
subordinate, si usa il metodo standard \latex/ per cambiare lo stile
di numerazione \cite[\S6.3, \S C.8.4]{lamport}. Ad esempio, ridefinendo
\cn{theequation} come segue fornisce numeri romani.
\begin{verbatim}
\begin{subequations}
\renewcommand{\theequation}{\theparentequation \roman{equation}}
...
\end{verbatim}

%%%%%%%%%%%%%%%%%%%%%%%%%%%%%%%%%%%%%%%%%%%%%%%%%%%%%%%%%%%%%%%%%%%%%%%%
%% FcDC
\chapter{Varie funzionalit\`{a} matematiche}

\section{Matrici}\label{ss:matrix}

Il pacchetto \pkg{amsmath} fornisce qualche ambiente per le
matrici\index{matrici} oltre al fondamentale  ambiente \env{array} del
\latex/. Gli ambienti \env{pmatrix}, \env{bmatrix}, \env{Bmatrix},
\env{vmatrix} e \env{Vmatrix} hanno come delimitatori rispettivamente
$(\,)$, $[\,]$, $\lbrace\,\rbrace$, $\lvert\,\rvert$, $\lVert\,\rVert$;
per coerenza con la nomenclatura viene fornito anche un ambiente
\env{matrix} senza delimitatori. Questo pu\`o sembrare superfluo,
vista la presenza dell'ambiente \env{array}, ma ci\`o non \`e vero;
infatti tutti gli ambienti per matrici utilizzano una spaziatura
orizzontale pi\`u economica di quella generosa messa a disposizione
dall'ambiente \env{array}. Inoltre, diversamente dall'ambiente
\env{array}, non si devono specificare i parametri relativi alle
colonne in nessuno degli ambienti per matrici; di \emph{default} si possono
avere fino a 10 colonne centrate.%
\footnote{%%%%%%%%%%%%%%%%%%%%%%%%%%%%%%%%%%%%%%%%%%%%%%%%%%%%%%%%%%%%%%
In dettaglio: Il massimo numero di colonne in una matrice \`e indicato
dal contatore |MaxMatrixCols| (valore normale=10), che si pu\`o
cambiare con i comandi \latex/ \cn{setcounter} o \cn{addcounter}.
}\space%%%%%%%%%%%%%%%%%%%%%%%%%%%%%%%%%%%%%%%%%%%%%%%%%%%%%%%%%%%%%%%%%
(Per ottenere l'allineamento a destra o a sinistra in una colonna, oppure
per qualsiasi altro formato speciale, \`e necessario utilizzare
\env{array})

Per ottenere una piccola matrice adatta al testo, \`e disponibile
l'ambiente \env{smallmatrix} (es:
\begin{math}
\bigl( \begin{smallmatrix}
  a&b\\ c&d
\end{smallmatrix} \bigr)
\end{math})
che \`e pi\`u adatta di qualsiasi altra matrice a entrare in una riga
di testo. Devono essere comunque forniti i delimitatori: non ci sono le
versioni |p|,|b|,|B|,|v|,|V| di \env{smallmatrix}. L'esempio qua sopra
\`e stato prodotto da
\begin{verbatim}
\bigl( \begin{smallmatrix}
  a&b\\ c&d
\end{smallmatrix} \bigr)
\end{verbatim}

\cn{hdotsfor}|{|\<numero>|}| produce una riga di punti in una matrice
\index{matrici!puntini}\index{puntini!nelle matrici}\index{punti|see{puntini}}%
larga tante colonne quanto il numero passato come argomento. Per
esempio,
\begin{center}
\begin{minipage}{.3\columnwidth}
\noindent$\begin{matrix} a&b&c&d\\
e&\hdotsfor{3} \end{matrix}$
\end{minipage}%
\qquad
\begin{minipage}{.45\columnwidth}
\begin{verbatim}
\begin{matrix} a&b&c&d\\
e&\hdotsfor{3} \end{matrix}
\end{verbatim}
\end{minipage}%
\end{center}

La spaziatura dei punti pu\`o essere variata con l'utilizzo di un
opzione tra parentesi quadre, ad esempio, |\hdotsfor[1.5]{3}|. Il
numero racchiuso dalle parentesi funge da moltiplicatore (il valore
normale \`e 1.0)

\begin{equation}\label{eq:D}
\begin{pmatrix} D_1t&-a_{12}t_2&\dots&-a_{1n}t_n\\
-a_{21}t_1&D_2t&\dots&-a_{2n}t_n\\
\hdotsfor[2]{4}\\
-a_{n1}t_1&-a_{n2}t_2&\dots&D_nt\end{pmatrix},
\end{equation}
\begin{verbatim}
\begin{pmatrix} D_1t&-a_{12}t_2&\dots&-a_{1n}t_n\\
-a_{21}t_1&D_2t&\dots&-a_{2n}t_n\\
\hdotsfor[2]{4}\\
-a_{n1}t_1&-a_{n2}t_2&\dots&D_nt\end{pmatrix}
\end{verbatim}


\section{Comandi per la spaziatura matematica}

Il pacchetto \pkg{amsmath} estende l'insieme dei comandi di spaziatura
\index{spaziatura orizzontale!in matematica} come mostrato sotto. Sia
la forma intera che quella contratta di questi comandi sono robuste e
possono essere utilizzate anche al di fuori dell'ambiente matematico.

\begin{ctab}{lll|lll}
Abbrev.&Forma intera& Esempio &Abbrev.&Forma intera&Esempio\\
\hline
\vstrut{2.5ex}
& no space& \spx{}& & no space & \spx{}\\
\cn{\,}& \cn{thinspace}& \spx{\,}&
  \cnbang& \cn{negthinspace}& \spx{\!}\\
\cn{\:}& \cn{medspace}& \spx{\:}&
  & \cn{negmedspace}& \spx{\negmedspace}\\
\cn{\;}& \cn{thickspace}& \spx{\;}&
  & \cn{negthickspace}& \spx{\negthickspace}\\
& \cn{quad}& \spx{\quad}\\
& \cn{qquad}& \spx{\qquad}
\end{ctab}
Per il maggior controllo possibile sulla spaziatura matematica \`e
possibile utilizzare \cn{mspace} e le `unit\`{a} matematiche';
un'unit\`{a} matematica o |mu| \`e uguale a 1/18esimo. Per avere un
\cn{quad} negativo si deve scrivere |\mspace{-18.0mu}|.


\section{Punti}
Non esiste un consenso generale per quanto riguarda il piazzamento dei
punti ellittici (a mezza riga o in fondo della riga) in vari contesti.
La cosa pu\`o quindi essere considerata una questione di gusto.
Utilizzando i comandi orientati verso la semantica
\begin{itemize}
\item \cn{dotsc} per \qq{punti con virgole}
\item \cn{dotsb} per \qq{punti con operazioni/relazioni binarie}
\item \cn{dotsm} per \qq{punti con moltiplicazioni}
\item \cn{dotsi} per \qq{punti con integrali}
\item \cn{dotso} per \qq{altri tipi} (nessuno dei precedenti)
\end{itemize}
invece di \cn{ldots} e \cn{cdots}, \`e possibile adattare a varie
convenzioni un documento ``al volo'', nel caso che (per esempio)
dovendo pubblicare tale documento, l'editore insista nel seguire le
tradizioni della casa. Il trattamento predefinito a seconda delle
situazioni segue le convenzioni dell'American Mathematical Society:
\begin{center}
\begin{tabular}{@{}l@{}l@{}}
\begin{minipage}[t]{.54\textwidth}
\begin{verbatim}
Abbiamo quindi la serie $A_1, A_2,
\dotsc$, la somma di regioni $A_1
+A_2 +\dotsb $, il prodotto
ortogonale $A_1 A_2 \dotsm $, e
l'integrale infinito
\[\int_{A_1}\int_{A_2}\dotsi.\]
\end{verbatim}
\end{minipage}
&
\begin{minipage}[t]{.45\textwidth}
\noindent
Abbiamo quindi la serie $A_1, A_2,
\dotsc$, la somma di regioni $A_1
+A_2 +\dotsb $, il prodotto
ortogonale $A_1 A_2 \dotsm $, e
l'integrale infinito
\[\int_{A_1}\int_{A_2}\dotsi.\]
\end{minipage}
\end{tabular}
\end{center}

\section{Trattini senza interruzioni}
Viene fornito il comando \cn{nobreakdash} per eliminare la
possibilit\`{a} che avvenga un'interruzione di linea dopo un trattino.
Ad esempio scrivendo `pagine 1\ndash 9'  come |pagine 1\nobreakdash 9|
non occorrer\`{a} mai un'interruzione di linea tra il trattino e il 9.
\`E possibile utilizzare \cn{nobreakdash} anche per prevenire
sillabazioni indesiderate in combinazioni tipo |$p$-adico|. Per un
utilizzo frequente \`e consigliato fare delle abbreviazioni; ad
esempio

\begin{verbatim}
\newcommand{\p}{$p$\nobreakdash}% per "\p-adico"
\newcommand{\Ndash}{\nobreakdash--}% per "pagine 1\Ndash 9"
%    Per "\n dimensionale" ("n-dimensionale"):
\newcommand{\n}[1]{$n$\nobreakdash-\hspace{0pt}}
\end{verbatim}
L'ultimo esempio mostra come impedire un'interruzione di linea dopo il
trattino ma permette la corretta sillabazione delle parole
seguenti.(Basta aggiungere un spazio di dimensione zero dopo il
trattino.)


\section{Accenti in matematica}

Nel \latex/ ordinario, il piazzamento del secondo accento negli accenti
matematici doppi \`e spesso mediocre; con il pacchetto \pkg{amsmath}
si migliora notevolmente il piazzamento del secondo accento:
$\hat{\hat{A}}$ (\cn{hat}|{\hat{A}}|).

Sono disponibili i  comandi \cn{dddot} e \cn{dddddot} per produrre
accenti tripli e quadrupli in aggiunta a \cn{dot} e \cn{ddot} presenti
nel \latex/.

Per ottenere un carattere di tilde o di cappello come apice, si deve
caricare il pacchetto \pkg{amsxtra} e utilizzare i comandi \cn{sphat}
o \cn{sptilde}, l'utilizzo \`e \verb'A\sphat' (notare l'assenza del
carattere \verb'^'). Per piazzare un simbolo arbitrario in posizione
di accento matematico o per ottenere accenti come pedici, consultare
il pacchetto \pkg{accents} di Javier Bezos.

\section{Radici}

Nel \latex/ ordinario il piazzamento degli indici delle radici a volte
non \`e buono:  $\sqrt[\beta]{k}$ (|\sqrt||[\beta]{k}|), nel pacchetto
 \pkg{amsmath} i comandi  \cn{leftroot} e \cn{uproot} permettono di aggiustare
la posizione della radice:

\begin{verbatim}
  \sqrt[\leftroot{-2}\uproot{2}\beta]{k}
\end{verbatim}
muove la beta in alto e verso destra:
$\sqrt[\leftroot{-2}\uproot{2}\beta]{k}$. L'argomento negativo di
\cn{leftroot} muove $\beta$ verso destra; le unit\`{a} sono piccole, e
quindi adatte per questo tipo di aggiustamenti.

\section{Formule in riquadro}

Il comando \cn{boxed} costruisce un riquadro attorno al suo argomento,
come \cn{fbox}, eccetto che i contenuti dei riquadri sono in modo
matematico:

\begin{equation}
\boxed{\eta \leq C(\delta(\eta) +\Lambda_M(0,\delta))}
\end{equation}
\begin{verbatim}
  \boxed{\eta \leq C(\delta(\eta) +\Lambda_M(0,\delta))}
\end{verbatim}

\section{Frecce in alto e in basso}
Il \latex/ di base fornisce i comandi \cn{overrightarrow} e
\cn{overleftarrow}; il paccheto \pkg{amsmath} fornisce altri comandi
per frecce in alto e in basso per estendere l'insieme di base:

\begin{tabbing}
\qquad\=\ncn{overleftrightarrow}\qquad\=\kill
\> \cn{overleftarrow}    \> \cn{underleftarrow} \+\\
   \cn{overrightarrow}    \> \cn{underrightarrow} \\
   \cn{overleftrightarrow}\> \cn{underleftrightarrow}
\end{tabbing}

\section{Frecce estendibili}
\cn{xleftarrow} e \cn{xrightarrow} producono frecce
\index{frecce!estendibili} che si estendono automaticamente per accomodare
grandezze inusuali di apici e pedici. Questi comandi prendono un argomento
facoltativo (il pedice) e un argomento obbligatorio (l'apice, possibilmente
anche vuoto):

\begin{equation}
A\xleftarrow{n+\mu-1}B \xrightarrow[T]{n\pm i-1}C
\end{equation}
\begin{verbatim}
  \xleftarrow{n+\mu-1}\quad \xrightarrow[T]{n\pm i-1}
\end{verbatim}

\section{Attaccare simboli ad altri simboli}

\latex/ fornisce  \cn{stackrel} per piazzare un apice
\index{esponenti e deponenti} sopra una relazione binaria.
Nel pacchetto \pkg{amsmath} ci sono comandi pi\`u generali,
\cn{overset} e \cn{underset} che possono essere utilizzati per
piazzare un simbolo sopra o sotto un altro simbolo, ogni qualvolta che
si trova una relazione binaria  o qualcos'altro.
L'input |\overset{*}{X}| piazza un $*$ della dimensione
di un apice
sopra la $X$: $\overset{*}{X}$; \cn{underset} \`e l'analogo
per aggiungere un simbolo in basso.
Controllare anche la descrizione di \cn{sideset} in \secref{sideset}.

\section{Frazioni e costrutti correlati}

\subsection{I comandi \cn{frac}, \cn{dfrac}, e \cn{tfrac}}

Il comando \cn{frac}, che fa parte dell'insieme dei comandi dei base
del \latex/,\index{frazioni} prende due argomenti\mdash numeratore
e denominatore\mdash e compone questi nella classica forma di una frazione.
Il pacchetto \pkg{amsmath} fornisce anche \cn{dfrac} e \cn{tfrac} come
convenienti abbreviazioni per |{\displaystyle\frac| |...| |}|
e\index{textstyle@\cn{textstyle}}\relax
\index{displaystyle@\cn{displaystyle}} |{\textstyle\frac| |...| |}|.

\begin{equation}
\frac{1}{k}\log_2 c(f)\quad\tfrac{1}{k}\log_2 c(f)\quad
\sqrt{\frac{1}{k}\log_2 c(f)}\quad\sqrt{\dfrac{1}{k}\log_2 c(f)}
\end{equation}
\begin{verbatim}
\begin{equation}
\frac{1}{k}\log_2 c(f)\;\tfrac{1}{k}\log_2 c(f)\;
\sqrt{\frac{1}{k}\log_2 c(f)}\;\sqrt{\dfrac{1}{k}\log_2 c(f)}
\end{equation}
\end{verbatim}

\subsection{I comandi \cn{binom}, \cn{dbinom}, e \cn{tbinom}}

Per espressioni binomiali\index{binomiali} tipo $\binom{n}{k}$
\pkg{amsmath} fornisce \cn{binom}, \cn{dbinom} e \cn{tbinom}:
\begin{equation}
2^k-\binom{k}{1}2^{k-1}+\binom{k}{2}2^{k-2}
\end{equation}
\begin{verbatim}
2^k-\binom{k}{1}2^{k-1}+\binom{k}{2}2^{k-2}
\end{verbatim}

\subsection{Il comando \cn{genfrac}}

Le capacit\`{a} di \cn{frac}, \cn{binom}, e delle loro varianti sono
sintetizzate dal comando generale \cn{genfrac}, che richiede sei
argomenti. Gli ultimi due corrispondono al numeratore e denominatore di
\cn{frac}, i primi due sono delimitatori opzionali (come visto in
\cn{binom}); il terzo riguarda lo spessore della linea (\cn{binom}
utilizza questo per impostare lo spessore della linea di frazione a 0
\mdash cio\`e invisibile) e il quarto argomento cambia lo stile
matematico: valori interi tra 0 e 3 selezionano rispettivamente
\cn{displaystyle}, \cn{textstyle}, \cn{scriptstyle} e
\cn{scriptscriptstyle}. Se il terzo argomento viene lasciato vuoto, lo
spessore della linea viene impostato per convenzione a `normale'.

\begin{center}\begin{minipage}{.85\columnwidth}
\raggedright \normalfont\ttfamily \exhyphenpenalty10000
\newcommand{\ma}[1]{%
  \string{{\normalfont\itshape#1}\string}\penalty9999 \ignorespaces}
\string\genfrac \ma{delim-sx} \ma{delim-dx} \ma{spessore}
\ma{stile} \ma{numeratore} \ma{denominatore}
\end{minipage}\end{center}
Per completezza viene mostrato come \cn{frac}, \cn{tfrac} e \cn{binom}
potrebbero essere definiti.
\begin{verbatim}
\newcommand{\frac}[2]{\genfrac{}{}{}{}{#1}{#2}}
\newcommand{\tfrac}[2]{\genfrac{}{}{}{1}{#1}{#2}}
\newcommand{\binom}[2]{\genfrac{(}{)}{0pt}{}{#1}{#2}}
\end{verbatim}
Se si utilizza ripetutamente \cn{genfrac} in un documento per una
particolare notazione, sarebbe di grande comodit\`a per lo scrittore
(e l'editore) definire un'abbreviazione significativa per questa
notazione, come \cn{frac} e \cn{binom} illustrate sopra. I comandi
primitivi generali per le frazioni \cs{over}, \cs{overwithdelims},
\cs{atop}, \cs{atopwithdelims}, \cs{above} e \cs{abovewithdelims}
producono messaggi di avvertimento se utilizzati in congiunzione con
\pkg{amsmath}, per ragioni discusse in \fn{technote.tex}.

\section{Frazioni continue}

La frazione continua\index{frazioni continue}
\begin{equation}
\cfrac{1}{\sqrt{2}+
 \cfrac{1}{\sqrt{2}+
  \cfrac{1}{\sqrt{2}+\cdots
}}}
\end{equation}
si ottiene digitando
{\samepage
\begin{verbatim}
\cfrac{1}{\sqrt{2}+
 \cfrac{1}{\sqrt{2}+
  \cfrac{1}{\sqrt{2}+\dotsb
}}}
\end{verbatim}
}% End of \samepage
Questo produce un  risultato visivamente migliore di quello ottenuto
con l'utilizzo prolungato di \cn{frac}. Il piazzamento a destra o a sinistra
di qualsiasi dei numeratori \`e ottenuto utilizzando \cn{cfrac}|[l]| o
\cn{cfrac}|[r]| invece di \cn{cfrac}.

\section{Opzioni smash}

Il comando \cn{smash} viene utilizzato per comporre una sottoformula
con effettiva larghezza e profondit\`{a} zero; questo a volte rimane
utile dovendo aggiustare la posizione della sottoformula rispetto ai
simboli adiacenti. Con il pacchetto \pkg{amsmath}, \cn{smash} ha
argomenti opzionali |t| e |b|, perch\'e occasionalmente \`e
vantaggioso essere capaci di \qq{appiattire} solo l'altezza o la
profondit\`a, conservando l'altra. Ad esempio, quando simboli di
radicali sono posizionati o dimensionati in modo diverso a causa delle
differenze di altezza e larghezza dei loro contenuti, \cn{smash}
pu\`o essere applicato per rendere il tutto pi\`u consistente.
Confrontare $\sqrt{x}+\sqrt{y}+\sqrt{z}$ con
$\sqrt{x}+\sqrt{\smash[b]{y}}+\sqrt{z}$, dove l'ultimo \`e stato
prodotto con \verb"$\sqrt{x}" \verb"+"
\verb"\sqrt{"\5\verb"\smash[b]{y}}" \verb"+" \verb"\sqrt{z}$".

\section{Delimitatori}

\subsection{Dimensione dei delimitatori}\label{bigdel}

Il dimensionamento automatico dei delimitatori fatto da \cn{left} e
\cn{right} ha due limitazioni: innanzi tutto, viene applicato
meccanicamente per produrre delimitatori abbastanza grandi da
ricoprire il pi\`u grande oggetto contenuto in essi, e inoltre,
l'intervallo delle dimensioni non \`e neanche approssimativamente
continuo, ma ha dei salti abbastanza grandi. Questo significa che un
frammemto matematico infinitesimamente troppo grande per una data
grandezza del delimitatore prender\`{a} la misura successiva, un salto
di 3pt o simile in un testo a grandezza normale. Ci sono due o tre
situazioni dove la grandezza del delimitatore viene comunemente
aggiustata, utilizzando un insieme di comandi che contengono `big' nei
loro nomi.

\begin{ctab}{l|llllll}
Dim. del&
  dim. del& \ncn{left}& \ncn{bigl}& \ncn{Bigl}& \ncn{biggl}& \ncn{Biggl}\\
delimitatore&
  testo& \ncn{right}& \ncn{bigr}& \ncn{Bigr}& \ncn{biggr}& \ncn{Biggr}\\
\hline
Risultato\vstrut{5ex}&
  $\displaystyle(b)(\frac{c}{d})$&
  $\displaystyle\left(b\right)\left(\frac{c}{d}\right)$&
  $\displaystyle\bigl(b\bigr)\bigl(\frac{c}{d}\bigr)$&
  $\displaystyle\Bigl(b\Bigr)\Bigl(\frac{c}{d}\Bigr)$&
  $\displaystyle\biggl(b\biggr)\biggl(\frac{c}{d}\biggr)$&
  $\displaystyle\Biggl(b\Biggr)\Biggl(\frac{c}{d}\Biggr)$
\end{ctab}
Il primo tipo di situazione \`e un operatore cumulativo con limiti
sopra e sotto. Con \cn{left} e \cn{right} i delimitatori di solito
diventano pi\`u larghi del necessario, e utilizzando invece
le dimensioni |Big| o |bigg| si ottengono risultati migliori.
\begin{equation*}
\left[\sum_i a_i\left\lvert\sum_j x_{ij}\right\rvert^p\right]^{1/p}
\quad\text{contro}\quad
\biggl[\sum_i a_i\Bigl\lvert\sum_j x_{ij}\Bigr\rvert^p\biggr]^{1/p}
\end{equation*}
\begin{verbatim}
\biggl[\sum_i a_i\Bigl\lvert\sum_j x_{ij}\Bigr\rvert^p\biggr]^{1/p}
\end{verbatim}
Il secondo tipo di situazione \`e un ammasso di coppie  di delimitatori
dove \cn{left} e \cn{right} rendono le loro grandezze uguali
(dato che questo risulta adeguato per racchiudere tutto il materiale)
ma l'effetto desiderato \`e quello di avere alcuni delimitatori
con grandezza maggiore, per rendere l'annidamento pi\`u semplice da
vedere.
\begin{equation*}
\left((a_1 b_1) - (a_2 b_2)\right)
\left((a_2 b_1) + (a_1 b_2)\right)
\quad\text{contro}\quad
\bigl((a_1 b_1) - (a_2 b_2)\bigr)
\bigl((a_2 b_1) + (a_1 b_2)\bigr)
\end{equation*}
\begin{verbatim}
\left((a_1 b_1) - (a_2 b_2)\right)
\left((a_2 b_1) + (a_1 b_2)\right)
\quad\text{versus}\quad
\bigl((a_1 b_1) - (a_2 b_2)\bigr)
\bigl((a_2 b_1) + (a_1 b_2)\bigr)
\end{verbatim}
Il terzo tipo di situazione \`e un oggetto di dimensione leggermente
elevata nel testo libero, %running rext??
come $\left\lvert\frac{b'}{d'}\right\rvert$, dove i delimitatori
prodotti da \cn{left} e \cn{right} causano un'eccessiva altezza
della linea. In questo caso \ncn{bigl} e \ncn{bigr}\index{big@\cn{big},
\cn{Big}, \cn{bigg}, \dots\ delimiters} possono essere utilizzati
per produrre delimitatori che sono leggermente pi\`u grandi della
dimensione di base, ma che comunque rientrano all'interno della normale
spaziatura della linea: $\bigl\lvert\frac{b'}{d'}\bigr\rvert$.
Nel \latex/ ordinario i delimitatori \ncn{big}, \ncn{bigg}, \ncn{Big},
e \ncn{Bigg} non sono scalati in modo opportuno per tutto il ``range''
delle dimensioni dei \emph{font} \latex/, con il pacchetto \pkg{amsmath}
invece lo sono.

\subsection{Notazioni per la barra verticale}

Il pacchetto \pkg{amsmath} fornisce i comandi \cn{vert}, \cn{rvert},
\cn{lVert}, \cn{rVert} (confrontare \cn{langle} e \cn{rangle}) per
indirizzare il problema del sovraccarico per il carattere di barra
verticale \qc{\|}. Questo carattere viene utilizato nei documenti
\latex/ per una grande variet\`{a} di oggetti matematici: la relazione
`divide' in un' espressione della teoria dei numeri tipo $p\vert q$,
oppure l'operazione di  valore assoluto $\lvert z\rvert$, oppure la
condizione `tale che' nella notazione insiemistica, oppure la
notazione `valutato in' $f_\zeta(t)\bigr\rvert_{t=0}$. La
molteplicit\`{a} degli utilizzi non \`e essa stessa un male, ci\`o
che non va bene comunque \`e il fatto che non tutti questi vari
oggetti ottengono lo stesso trattamento tipografico e che le complesse
capcit\`a discriminatorie di un lettore colto non possono essere
replicate in un computer che deve elaborare documenti matematici. Si
raccomanda quindi che ci sia una corrispondenza uno-a-uno in ogni
documento tra il carattere di barra verticale \qc{\|} e una scelta
notazione matematica, analogamente per il comando di doppia barra
\cn{\|}. Questo immediatamente esclude l'utilizzo di \qc{|} e
\ncn{\|}\index{"|@\cn{"\"|}} come delimitatori, dato che i delimitatori
destri e sinistri %
hanno usi distinti, non correlati allo stesso modo con simboli
adiacenti
%delimiters are distinct usages that do not relate in the same way to
%adjacent symbols;
si raccomanda la pratica di definire nel preambolo del documento
comandi adatti a ogni utilizzo di coppie di delimitatori con simboli
di barre verticali:

\begin{verbatim}
\providecommand{\abs}[1]{\lvert#1\rvert}
\providecommand{\norm}[1]{\lVert#1\rVert}
\end{verbatim}
al che il documento dovrebbe contenere |\abs{z}| per produrre
 $\lvert z\rvert$ e |\norm{v}| per produrre $\lVert v\rVert$.
%%%%% fine FCdC

%%%%% OdB
\chapter{Nomi per gli operatori}

\section{Come definire nuovi nomi di operatori}\label{s:opname}

Le funzioni matematiche\index{nomi di operatori}\index{nomi di
funzioni|see{nomi di operatori}} come $\log$, $\sin$, e $\lim$ sono
per tradizione stampate in tondo per renderne pi\`u immediata la
visibilit\`a rispetto alle variabili matematiche di un carattere, che
sono stampate in stile matematico corsivo. Le pi\`u comuni hanno nomi
predefiniti, \cn{log}, \cn{sin}, \cn{lim}, e cos\`\i{} via, ma se ne
introducono continuamente di nuovi nelle pubblicazioni relative alla
matematica, pertanto il pacchetto \pkg{amsmath} fornisce un metodo
generale per definire nuovi `nomi di operatori'. Per definire una
funzione matematica \ncn{xxx} che si presenti come \cn{sin}, si
scriver\`a
\begin{verbatim}
\DeclareMathOperator{\xxx}{xxx}
\end{verbatim}
Come conseguenza, l'utilizzo di \ncn{xxx} produrr\`a {\upshape xxx}
nel corrispondente \emph{font} e automaticamente aggiunger\`a l'adeguata
spaziatura\index{spaziatura orizzontale!attorno ai nomi di operatori}
su entrambi i lati quando necessario, in maniera tale da ottenere
$A\xxx B$ invece di $A\mathrm{xxx}B$. Nel secondo argomento di
\cn{DeclareMathOperator} (il testo con il nome), \`e prevalente una
modalit\`a pseudo-testuale: il carattere di sillabazione \qc{\-}
verr\`a stampato come un trattino di sillabazione piuttosto che come
un segno meno e un asterisco \qc{\*} risulter\`a stampato come un
asterisco in alto piuttosto che come un asterisco centrato di tipo
matematico (confrontare \textit{a}-\textit{b}*\textit{c} e $a-b*c$.);
d'altra parte il testo contenente il nome \`e stampato in modalit\`a
matematica, ad es. in modo tale da poter ivi usare pedici e apici.

Se il nuovo operatore dovesse esser dotato di pedici e apici
posizionati alla maniera dei
`limiti', al di sopra e al di sotto come per $\lim$, $\sup$, o $\max$,
si user\`a
la forma \qc{\*} del comando \cn{DeclareMathOperator}:
\begin{verbatim}
\DeclareMathOperator*{\Lim}{Lim}
\end{verbatim}
Fare inoltre riferimento alla trattazione del posizionamento dell'indice
nel paragrafo~\ref{subplace}.

I seguenti nomi di operatori sono predefiniti:
\begin{ctab}{rlrlrlrl}
\cn{arccos}& $\arccos$ &\cn{deg}& $\deg$ &      \cn{lg}& $\lg$ &        \cn{projlim}& $\projlim$\\
\cn{arcsin}& $\arcsin$ &\cn{det}& $\det$ &      \cn{lim}& $\lim$ &      \cn{sec}& $\sec$\\
\cn{arctan}& $\arctan$ &\cn{dim}& $\dim$ &      \cn{liminf}& $\liminf$ &\cn{sin}& $\sin$\\
\cn{arg}& $\arg$ &      \cn{exp}& $\exp$ &      \cn{limsup}& $\limsup$ &\cn{sinh}& $\sinh$\\
\cn{cos}& $\cos$ &      \cn{gcd}& $\gcd$ &      \cn{ln}& $\ln$ &        \cn{sup}& $\sup$\\
\cn{cosh}& $\cosh$ &    \cn{hom}& $\hom$ &      \cn{log}& $\log$ &      \cn{tan}& $\tan$\\
\cn{cot}& $\cot$ &      \cn{inf}& $\inf$ &      \cn{max}& $\max$ &      \cn{tanh}& $\tanh$\\
\cn{coth}& $\coth$ &    \cn{injlim}& $\injlim$ &\cn{min}& $\min$\\
\cn{csc}& $\csc$ &      \cn{ker}& $\ker$ &      \cn{Pr}& $\Pr$
\end{ctab}
\begin{ctab}{rlrl}
\cn{varlimsup}&  $\displaystyle\varlimsup$&
  \cn{varinjlim}&  $\displaystyle\varinjlim$\\
\cn{varliminf}&  $\displaystyle\varliminf$&
  \cn{varprojlim}& $\displaystyle\varprojlim$
\end{ctab}

\`E inoltre disponibile un comando \cn{operatorname}, in modo tale che
l'uso di
\begin{verbatim}
\operatorname{abc}
\end{verbatim}
in una formula matematica equivalga all'uso di \ncn{abc} definito da
\cn{DeclareMathOperator}; questo pu\`o in certi casi essere utile per
realizzare notazioni pi\`u complesse o per altri scopi. (Usare la variante
\cn{operatorname*} per ottenere i limiti.)

\section{\cn{mod} e i suoi affini}

I comandi \cn{mod}, \cn{bmod}, \cn{pmod}, \cn{pod} sono forniti per
affrontare le particolari convenzioni di spaziatura della notazione
\qq{mod}. In  \latex/ sono disponibili \cn{bmod} e
\cn{pmod}, ma con il pacchetto \pkg{amsmath}
la spaziatura di \cn{pmod} sar\`a regolata a un valore inferiore se
viene usato in una formula in modalit\`a \emph{non-display}.
 \cn{mod} e \cn{pod} sono varianti di
\cn{pmod} preferite da alcuni autori; \cn{mod} omette le parentesi,
mentre \cn{pod} omette il \qq{mod} e mantiene le parentesi.
\begin{equation}
\gcd(n,m\bmod n);\quad x\equiv y\pmod b;
\quad x\equiv y\mod c;\quad x\equiv y\pod d
\end{equation}
\begin{verbatim}
\gcd(n,m\bmod n);\quad x\equiv y\pmod b;
\quad x\equiv y\mod c;\quad x\equiv y\pod d
\end{verbatim}
%%%%% Fine OdB

%%%%%%%%%%%%%%%%%%%%%%%%%%%%%%%%%%%%%%%%%%%%%%%%%%%%%%%%%%%%%%%%%%%%%%%%
% by GD

\chapter{Il comando \cn{text}}\label{text}

Il principale utilizzo del comando \cn{text} consiste nello scrivere
parole o frasi\index{testo!frammenti di testo in matematica} in un
\emph{display}. Il suo comportamento \`e molto simile al comando
\latex/  \cn{mbox}, ma presenta un paio di vantaggi. Se si desidera
inserire una parola o una frase in un deponente \`e leggermente pi\`u
semplice digitare |..._{\text{parola o frase}}| piuttosto che il
comando equivalente \cn{mbox}: |..._{\mbox{\scriptsize| |parola| |o|
|frase}}|. L'altro vantaggio \`e nel suo nome, pi\`u descrittivo.
\begin{equation}
f_{[x_{i-1},x_i]} \text{ \`e monotona,}
\quad i = 1,\dots,c+1
\end{equation}
\begin{verbatim}
f_{[x_{i-1},x_i]} \text{ \`e monotona,}
\quad i = 1,\dots,c+1
\end{verbatim}

% by GD - end

% by LF
\chapter{Integrali e sommatorie}

\section{Deponenti ed esponenti su pi\`u righe}

Il comando \cn{substack} pu\`o essere usato per produrre un deponente o un
esponente su pi\`u righe:\index{deponenti ed esponenti!su pi\`u righe}\relax
\index{esponenti|see{deponenti ed esponenti}} per esempio
\begin{ctab}{ll}
\begin{minipage}[t]{.6\columnwidth}
\begin{verbatim}
\sum_{\substack{
         0\le i\le m\\
         0<j<n}}
  P(i,j)
\end{verbatim}
\end{minipage}
&
$\displaystyle
\sum_{\substack{0\le i\le m\\ 0<j<n}} P(i,j)$
\end{ctab}
Una forma un po' pi\`u generalizzata \`e l'ambiente \env{subarray} che
consente di specificare che ogni riga deve essere allineata a sinistra invece che
centrata, come in questo caso:
\begin{ctab}{ll}
\begin{minipage}[t]{.6\columnwidth}
\begin{verbatim}
\sum_{\begin{subarray}{l}
        i\in\Lambda\\ 0<j<n
      \end{subarray}}
 P(i,j)
\end{verbatim}
\end{minipage}
&
$\displaystyle
  \sum_{\begin{subarray}{l}
        i\in \Lambda\\ 0<j<n
      \end{subarray}}
 P(i,j)$
\end{ctab}

\section{Il comando \cn{sideset}}\label{sideset}

C'\`e anche un comando chiamato \cn{sideset}, per uno scopo abbastanza
particolare: porre dei simboli agli angoli di deponente ed
esponente\index{deponenti ed esponenti!su sommatorie} di un simbolo
operatorio di grandi dimensioni come $\sum$ o $\prod$. \emph{Nota:
questo comando non \`e pensato per essere applicato ad altro che a
simboli tipo sommatoria.} L'esempio principale \`e il caso in cui si
voglia porre un simbolo di primo su un simbolo di sommatoria. Se non
ci sono estremi sopra o sotto la sommatoria, si pu\`o semplicemente
usare \cn{nolimits}: ecco come appare
%%%%%%%%%%%%%%%%%%%%%%%%%%%%%%%%%%%%%%%%%%%%%%%%%%%%%%%%%%%%%%%%%%%%%%%%
|\sum\nolimits' E_n| in modo \emph{display}:
\begin{equation}
\sum\nolimits' E_n
\end{equation}
Se tuttavia si desidera non solo il segno di primo ma anche qualcosa sopra o
sotto il simbolo di sommatoria, non \`e cos\`\i{} facile\mdash invero, senza
\cn{sideset}, sarebbe proprio difficile. Con \cn{sideset}, si
pu\`o scrivere
\begin{ctab}{ll}
\begin{minipage}[t]{.6\columnwidth}
\begin{verbatim}
\sideset{}{'}
  \sum_{n<k,\;\text{$n$ odd}} nE_n
\end{verbatim}
\end{minipage}
&$\displaystyle
\sideset{}{'}\sum_{n<k,\;\text{$n$ odd}} nE_n
$
\end{ctab}
La coppia di parentesi graffe vuote si spiega con il fatto che
\cn{sideset} ha la possibilit\`a di porre uno o pi\`u simboli aggiuntivi a
ogni angolo di un simbolo operatorio di grandi dimensioni; per porre un asterisco in ciascun angolo di un
simbolo di prodotto, si potrebbe scrivere
\begin{ctab}{ll}
\begin{minipage}[t]{.6\columnwidth}
\begin{verbatim}
\sideset{_*^*}{_*^*}\prod
\end{verbatim}
\end{minipage}
&$\displaystyle
\sideset{_*^*}{_*^*}\prod
$
\end{ctab}

\section{Posizionamento di deponenti ed estremi}\label{subplace}

Il tipo di posizionamento predefinito per i deponenti dipende dal
simbolo base considerato. Per i simboli tipo sommatoria \`e usato il
posizionamento `displaylimits': quando un simbolo tipo sommatoria appare
in una formula in \emph{display}, deponente ed esponente sono posti nella posizione
`limits' sopra e sotto, ma in una formula nel testo sono invece posti
a lato, per evitare l'antiestetico e sprecato allargamento della distanza dalle
righe di testo adiacenti.
L'impostazione predefinita per i simboli tipo integrale \`e avere deponenti
ed esponenti sempre a lato, anche nelle formule in \emph{display}.
(Si veda la discussione su \opt{intlimits} e opzioni correlate nella
Sec.~\ref{options}.)

I nomi di operatore, come $\sin$ o $\lim$, possono avere il posizionamento
`displaylimits' o quello `limits' a seconda di come sono stati definiti. Gli operatori
pi\`u comuni sono definiti in base all'uso consueto in matematica.

I comandi \cn{limits} e \cn{nolimits} possono essere usati per modificare
il normale comportamento di un simbolo base:
\begin{equation*}
\sum\nolimits_X,\qquad \iint\limits_{A},
\qquad\varliminf\nolimits_{n\to \infty}
\end{equation*}
Per definire un comando i cui deponenti seguono lo
stesso comportamento `displaylimits' di \cn{sum}, si pu\`o aggiungere
\cn{displaylimits} in coda alla definizione. Quando ci sono
pi\`u istanze consecutive di \cn{limits}, \cn{nolimits}, o \cn{displaylimits},
l'ultima ha la priorit\`a.

\section{Simboli di integrale multiplo}

\cn{iint}, \cn{iiint}, e \cn{iiiint} producono pi\`u simboli di integrale
\index{integrali!multipli} con la spaziatura tra di loro opportunamente
corretta, sia in stile testo che \emph{display}. \cn{idotsint} estende
la stessa idea producendo due segni di integrale separati da tre punti.
\begin{gather}
\iint\limits_A f(x,y)\,dx\,dy\qquad\iiint\limits_A
f(x,y,z)\,dx\,dy\,dz\\
\iiiint\limits_A
f(w,x,y,z)\,dw\,dx\,dy\,dz\qquad\idotsint\limits_A f(x_1,\dots,x_k)
\end{gather}

% by LF - end
%%%%%%%%%%%%%%%%%%%%%%%%%%%%%%%%%%%%%%%%%%%%%%%%%%%%%%%%%%%%%%%%%%%%%%%%
% by GD

\chapter{Diagrammi commutativi}\label{s:commdiag}

Vari comandi, come quelli in \amstex/, per disegnare i diagrammi
commutativi sono disponibili separatamente nel pacchetto \pkg{amscd}.
Per i diagrammi commutativi di una certa complessit\`a, gli autori
dovranno necessariamente considerare pacchetti pi\`u estesi come
\pkg{kuvio} o \xypic/, ma per diagrammi semplici privi di
frecce\index{frecce!nei diagrammi commutativi} diagonali, i
comandi dell'\pkg{amscd} potrebbero rivelarsi pi\`u convenienti.
Di seguito vi \`e un esempio.
\begin{equation*}
\begin{CD}
S^{{\mathcal{W}}_\Lambda}\otimes T   @>j>>   T\\
@VVV                                    @VV{\End P}V\\
(S\otimes T)/I                  @=      (Z\otimes T)/J
\end{CD}
\end{equation*}
\begin{verbatim}
\begin{CD}
S^{{\mathcal{W}}_\Lambda}\otimes T   @>j>>   T\\
@VVV                                    @VV{\End P}V\\
(S\otimes T)/I                  @=      (Z\otimes T)/J
\end{CD}
\end{verbatim}
Nell'ambiente \env{CD}, i comandi |@>>>|, |@<<<|, |@VVV| e |@AAA|
disegnano, rispettivamente, le frecce a destra, a sinistra, verso il
basso e verso l'alto.
Per quanto riguarda le frecce orizzontali, il contenuto tra il primo
e il secondo simbolo |>| oppure |<| sar\`a inserito a esponente sulla
freccia, e il contenuto tra il secondo e il terzo simbolo sar\`a inserito
a deponente sotto la freccia.
Analogamente per le frecce verticali, il contenuto tra il primo e il
secondo oppure tra il secondo e il terzo dei simboli |A| o |V| sar\`a
inserito a \qq{margine} sinistro o destro della freccia.
I comandi |@=| e \verb'@|' tracciano rispettivamente una doppia linea
orizzontale e una verticale.
Il comando |@.| equivale a una \qq{freccia nulla} e pu\`o essere usato
al posto di una freccia visibile per espandere, dove sia necessario, una
matrice.

% by GD - end

%%%%%%%%%%%%%%%%%%%%%%%%%%%%%%%%%%%%%%%%%%%%%%%%%%%%%%%%%%%%%%%%%%%%%%%%
\chapter{Usare \emph{font} matematici}

\section{Introduzione}

Per informazioni pi\`u complete riguardo l'uso dei \emph{font} in \latex/,
consultate la guida ai \emph{font} del \latex/ (\fn{fntguide.tex}) o
il libro \booktitle{The \latex/ Companion} \cite{tlc}.  L'insieme di base
dei comandi per usare \emph{font} matematici\index{\emph{font} matematici}\relax
\index{simboli matematici|see{\emph{font} matematici}} nel \latex/ \`e
costituito da \cn{mathbf}, \cn{mathrm}, \cn{mathcal}, \cn{mathsf},
\cn{mathtt} e \cn{mathit}. Comandi aggiuntivi per \emph{font} matematici
come \cn{mathbb} per il \emph{blackboard-bold}, \cn{mathfrak} per il Fraktur
e \cn{mathscr} per l'Euler script sono disponibili attraverso i
pacchetti \pkg{amsfonts} e \pkg{euscript} (distribuiti separatamente).

\section{Uso consigliato per i comandi dei \emph{font} matematici}

Se ci si trova a usare frequentemente comandi per \emph{font} matematici nei
propri documenti, si potrebbero voler usare nomi pi\`u brevi, come
\ncn{mb} al posto di \cn{mathbf}.  Ovviamente non c'\`e nulla che
impedisca di farsi da soli tali abbreviazioni, usando i comandi
\cn{newcommand} appropriati.  D'altro canto, per il \latex/, offrire
comandi pi\`u brevi sarebbe addirittura un disservizio per gli autori,
poich\'e renderebbe meno ovvia un'alternativa molto migliore:
definire nomi di comandi personalizzati che si riferiscano ai nomi degli
oggetti matematici che a loro competono, piuttosto che ai nomi dei
\emph{font} che sono usati per distinguere tali oggetti.  Per esempio, se
si usa il grassetto per indicare i vettori, alla lunga sarebbe meglio definire
un comando `vector' al posto di un `math-bold':
\begin{verbatim}
  \newcommand{\vect}[1]{\mathbf{#1}}
\end{verbatim}
si pu\`o scrivere |\vect{a} + \vect{b}| per avere $\vect{a} +
\vect{b}$.
Se, mesi dopo aver cominciato il lavoro, si decide di usare il
grassetto per qualche altro scopo e di indicare i vettori con una
freccina sopra, si pu\`o fare il tutto semplicemente cambiando la
definizione di \ncn{vect}; altrimenti si sarebbero dovute rimpiazzare
tutte le occorrenze di \cn{mathbf} nel documento, eventualmente
persino controllandole una a una per vedere se si riferivano
effettivamente a un vettore.

Pu\`o essere utile anche assegnare nomi di comandi distinti per
differenti lettere dell'alfabeto di un particolare \emph{font}:
\begin{verbatim}
\DeclareSymbolFont{AMSb}{U}{msb}{m}{n}% oppure si usi il pacchetto amsfonts
\DeclareMathSymbol{\C}{\mathalpha}{AMSb}{"43}
\DeclareMathSymbol{\R}{\mathalpha}{AMSb}{"52}
\end{verbatim}
Queste righe definirebbero i comandi \cn{C} e \cn{R} in modo che
producano le lettere \emph{blackboard-bold} del \emph{font} di simboli matematici
`AMSb'.  Se nel proprio documento si fa spesso riferimento ai numeri
reali o a quelli complessi, si pu\`o preferire questo metodo a
quello di definire, per esempio, un comando \ncn{field} e scrivere
|\field{C}| e |\field{R}|, ma per ottenere la massima flessibilit\`a
e il massimo controllo sarebbe opportuno definire tale comando e poi
definire \ncn{C} e \ncn{R} in funzione di quello:\index{mathbb@\cn{mathbb}}
\begin{verbatim}
\usepackage{amsfonts}% per disporre dell'alfabeto \mathbb
\newcommand{\field}[1]{\mathbb{#1}}
\newcommand{\C}{\field{C}}
\newcommand{\R}{\field{R}}
\end{verbatim}

\section{Simboli matematici in grassetto}

Il comando \cn{mathbf} \`e usato comunemente per ottenere lettere
latine grassette in modo matematico, ma per la maggior parte degli
altri tipi di simboli matematici non ha effetto, o i suoi effetti
dipendono in maniera non prevedibile dalla serie di \emph{font} matematici in
uso.  Per esempio, scrivendo
\begin{verbatim}
\Delta \mathbf{\Delta}\mathbf{+}\delta \mathbf{\delta}
\end{verbatim}
si ottiene $\Delta \mathbf{\Delta}\mathbf{+}\delta \mathbf{\delta}$;
il comando \cn{mathbf} non ha cambiato il segno pi\`u e il delta
minuscolo.

Per questo motivo il pacchetto \pkg{amsmath} fornisce altri due
comandi, \cn{boldsymbol} e \cn{pmb}, che possono essere usati con gli
altri tipi di simboli matematici.  \cn{boldsymbol} pu\`o essere usato
per i simboli matematici sui quali non ha effetto il comando \cn{mathbf}
se (e solo se) il \emph{font} matematico in uso in quel momento dispone
di una versione in grassetto di quel simbolo.  \cn{pmb} pu\`o essere usato
come ultima risorsa per qualsiasi simbolo matematico che non abbia una
vera versione in grassetto all'interno del \emph{font} matematico; \qq{pmb}
\`e l'abbreviazione di \qq{poor man's bold} (grassetto dei poveri) e
funziona stampando copie pi\`u copie dello stesso simbolo leggermente spostate
le une dalle altre.  Il risultato \`e di qualit\`a
inferiore, specialmente per quei simboli che contengono linee sottili.
Quando si usa la famiglia standard di \emph{font} matematici del \latex/ (il
Computer Modern), gli unici simboli che potrebbero richiedere il
\cn{pmb} sono quelli dei simboli operatori di grandi dimensioni, come \cn{sum}, i simboli
di delimitatori estesi, o i simboli addizionali forniti dal pacchetto
\pkg{amssymb}~\cite{amsfonts}.

La formula seguente mostra alcuni dei possibili risultati:
\begin{verbatim}
A_\infty + \pi A_0
\sim \mathbf{A}_{\boldsymbol{\infty}} \boldsymbol{+}
  \boldsymbol{\pi} \mathbf{A}_{\boldsymbol{0}}
\sim\pmb{A}_{\pmb{\infty}} \pmb{+}\pmb{\pi} \pmb{A}_{\pmb{0}}
\end{verbatim}
\begin{equation*}
A_\infty + \pi A_0
\sim \mathbf{A}_{\boldsymbol{\infty}} \boldsymbol{+}
  \boldsymbol{\pi} \mathbf{A}_{\boldsymbol{0}}
\sim\pmb{A}_{\pmb{\infty}} \pmb{+}\pmb{\pi} \pmb{A}_{\pmb{0}}
\end{equation*}
Se si vuole usare solo il comando \cn{boldsymbol} senza caricare tutto
il pacchetto \pkg{amsmath}, si pu\`o usare il pacchetto \pkg{bm} (questo
\`e un pacchetto standard del \latex/, non fa parte di quelli AMS; se
si ha una versione del \latex/ del 1997 o posteriore, probabilmente lo
si ha gi\`a).

\section{Lettere greche corsive}

Per ottenere una versione corsiva delle lettere greche maiuscole, si
possono usare i seguenti comandi:
\begin{ctab}{rlrl}
\cn{varGamma}& $\varGamma$& \cn{varSigma}& $\varSigma$\\
\cn{varDelta}& $\varDelta$& \cn{varUpsilon}& $\varUpsilon$\\
\cn{varTheta}& $\varTheta$& \cn{varPhi}& $\varPhi$\\
\cn{varLambda}& $\varLambda$& \cn{varPsi}& $\varPsi$\\
\cn{varXi}& $\varXi$& \cn{varOmega}& $\varOmega$\\
\cn{varPi}& $\varPi$
\end{ctab}

%%%%%%%%%%%%%%%%%%%%%%%%%%%%%%%%%%%%%%%%%%%%%%%%%%%%%%%%%%%%%%%%%%%%%%%%
% by GA

\chapter{Messaggi di errore e problemi di output}

\section{Osservazioni di carattere generale}

Questo \`e un supplemento al capitolo~8 del manuale del \latex/
\cite{lamport} (prima edizione: capitolo~6). Per comodit\`a del
lettore, l'insieme dei messaggi d'errore discussi qui si sovrappone
parzialmente con quello di \cite{lamport}, ma sia chiaro che qui
non si intende dare una copertura esaustiva. I messaggi
d'errore sono disposti in ordine alfabetico, senza badare a testo
irrilevante quale |! LaTeX Error:| all'inizio del messaggio, e
caratteri non alfabetici quali \qc{\\}. Dove vengono forniti esempi,
vengono anche mostrati i messaggi d'aiuto che appaiono sullo schermo
quando si risponde a un messagio d'errore digitando |h|.

C'\`e anche una sezione che discute qualche errore di output, per esempio
in casi in cui il documento stampato ha qualcosa che non va, ma \latex/
non ha rilevato alcun errore.

\section{Messaggi di errore}

\begin{error}{\begin{split} won't work here.}
\errexa
\begin{verbatim}
! Package amsmath Error: \begin{split} won't work here.
 ...

l.8 \begin{split}

? h
\Did you forget a preceding \begin{equation}?
If not, perhaps the `aligned' environment is what you want.
?
\end{verbatim}
\errexpl L'ambiente \env{split} non costruisce un'equazione in
\emph{display} a s\'e stante; deve essere usato all'interno di qualche
altro ambiente quali \env{equation} o \env{gather}.

\end{error}

\begin{error}{Extra & on this line}
\errexa
\begin{verbatim}
! Package amsmath Error: Extra & on this line.

See the amsmath package documentation for explanation.
Type  H <return>  for immediate help.
 ...

l.9 \end{alignat}

? h
\An extra & here is so disastrous that you should probably exit
 and fix things up.
?
\end{verbatim}
\errexpl
In una struttura \env{alignat} il numero di punti di allineamento su una linea
\`e determinato dall'argomento numerico fornito dopo |\begin{alignat}|.
Se in una linea si usano pi\`u punti di allineamento rispetto a quelli consentiti,
\latex/ assume che sia stato dimenticato accidentalmente un comando di
interruzione di riga \cn{\\} e produce questo errore.
\end{error}

\begin{error}{Improper argument for math accent}
\errexa
\begin{verbatim}
! Package amsmath Error: Improper argument for math accent:
(amsmath)                Extra braces must be added to
(amsmath)                prevent wrong output.

See the amsmath package documentation for explanation.
Type  H <return>  for immediate help.
 ...

l.415 \tilde k_{\lambda_j} = P_{\tilde \mathcal
                                               {M}}
?
\end{verbatim}
\errexpl
Argomenti complessi per tutti i comandi \latex/ dovrebbero venire
racchiusi tra parentesi graffe. In questo esempio le graffe sono
necessarie come mostrato:
\begin{verbatim}
... P_{\tilde{\mathcal{M}}}
\end{verbatim}
\end{error}

\begin{error}{Font OMX/cmex/m/n/7=cmex7 not loadable ...}
\errexa
\begin{verbatim}
! Font OMX/cmex/m/n/7=cmex7 not loadable: Metric (TFM) file not found.
<to be read again>
                   relax
l.8 $a
      b+b^2$
? h
I wasn't able to read the size data for this font,
so I will ignore the font specification.
[Wizards can fix TFM files using TFtoPL/PLtoTF.]
You might try inserting a different font spec;
e.g., type `I\font<same font id>=<substitute font name>'.
?
\end{verbatim}
\errexpl Certe dimensioni di alcuni \emph{font} del Computer Modern che erano
un tempo disponibili principalmente attraverso la raccolta
AMSFonts\index{AMSFonts, raccolta} sono considerate parte del \latex/
standard (giugno 1994): \fn{cmex7}\ndash \texttt{9},
\fn{cmmib5}\ndash \texttt{9}, e \fn{cmbsy5}\ndash \texttt{9}. Se
queste dimensioni straordinarie mancano nel proprio sistema,
bisognerebbe prima recuperarle dalla sogrente che ha fornito \latex/.
Altrimenti, si potrebbe provare a reperirle dalla CTAN (per esempio,
sotto forma di sorgenti Metafont\index{Sorgenti Metafont}, directory
\nfn{/tex-archive/fonts/latex/mf}, o in formato PostScript Type 1,
directory \nfn{/tex-archive/fonts/cm/ps-type1/bakoma}\index{\emph{font}
BaKoMa}\index{\emph{font} PostScript}).

Se il nome del \emph{font} comincia per \fn{cmex}, c'\`e un'opzione speciale
\fn{cmex10} per il pacchetto \pkg{amsmath} che fornisce una scappatoia
temporanea. In altre parole, si cambi il comando \cn{usepackage} in
\begin{verbatim}
\usepackage[cmex10]{amsmath}
\end{verbatim}
Questo forzer\`a l'uso della dimensione di 10 punti del \emph{font} \fn{cmex}
in ogni caso. A seconda del contenuto del documento, questo potrebbe
essere adeguato.
\end{error}

\begin{error}{Math formula deleted: Insufficient extension fonts}
\errexa
\begin{verbatim}
! Math formula deleted: Insufficient extension fonts.
l.8 $ab+b^2$

?
\end{verbatim}
\errexpl
Solitamente questo \`e preceduto da un errore del tipo |Font ... not loadable|;
si veda (sopra) la descrizione di quell'errore per risolvere il problema.
\end{error}

\begin{error}{Missing number, treated as zero}
\errexa
\begin{verbatim}
! Missing number, treated as zero.
<to be read again>
                   a
l.100 \end{alignat}

? h
A number should have been here; I inserted `0'.
(If you can't figure out why I needed to see a number,
look up `weird error' in the index to The TeXbook.)

?
\end{verbatim}
\errexpl
Ci sono parecchie cause che possono provocare questo errore. Comunque, una
possibilit\`a che \`e rilevante per il pacchetto \pkg{amsmath} \`e che si
\`e dimenticato di specificare l'argomento numerico di un ambiente \env{alignat},
come illustrato in questo esempio:
\begin{verbatim}
\begin{alignat}
 a&  =b&    c& =d\\
a'& =b'&   c'& =d'
\end{alignat}
\end{verbatim}
dove la prima linea dovrebbe invece essere
\begin{verbatim}
\begin{alignat}{2}
\end{verbatim}

Un'altra possibilit\`a \`e che una aperta parentesi quadra |[|
segua un comando di interruzione di linea \cn{\\} in un costrutto multilinea
come \env{array}, \env{tabular}, o \env{eqnarray}. Questo sar\`a
interpretato da \latex/ come l'inizio di una richiesta di `spazio verticale
aggiuntivo' \cite[\S C.1.6]{lamport}, anche se capita nella linea successiva
con l'intenzione di renderlo parte del contenuto. Per esempio
\begin{verbatim}
\begin{array}
a+b\\
[f,g]\\
m+n
\end{array}
\end{verbatim}
Per evitare il messaggio d'errore in casi di questo tipo, si possono
aggiungere parentesi graffe come suggerito nel manuale di \latex/
\cite[\S C.1.1]{lamport}:
\begin{verbatim}
\begin{array}
a+b\\
{[f,g]}\\
m+n
\end{array}
\end{verbatim}

\end{error}

\begin{error}{Missing \right. inserted}
\errexa
\begin{verbatim}
! Missing \right. inserted.
<inserted text>
                \right .
l.10 \end{multline}

? h
I've inserted something that you may have forgotten.
(See the <inserted text> above.)
With luck, this will get me unwedged. But if you
really didn't forget anything, try typing `2' now; then
my insertion and my current dilemma will both disappear.
\end{verbatim}
\errexpl
Questo errore si verifica tipicamente quando si cerca di inserire una
interruzione di linea all'interno di una coppia di delimitatori
\cn{left}-\cn{right} in un ambiente \env{multline} o \env{split}:
\begin{verbatim}
\begin{multline}
AAA\left(BBB\\
  CCC\right)
\end{multline}
\end{verbatim}
Ci sono due possibili soluzioni: (1)~invece di usare \cn{left} e
\cn{right}, si usino delimitatori `big' di grandezza fissa (\cn{bigl}
\cn{bigr} \cn{biggl} \cn{biggr} \dots; si veda \secref{bigdel}); oppure
(2)~si usino delimitatori nulli per spezzare la copia \cn{left}-\cn{right}
in due parti, una per ogni linea:
\begin{verbatim}
AAA\left(BBB\right.\\
  \left.CCC\right)
\end{verbatim}
La seconda soluzione potrebbe dar luogo a dimensioni incoerenti dei
delimitatori; ci si pu\`o assicurare che coincidono usando \cn{vphantom}
nella linea in cui compare il delimitatore pi\`u piccolo (o magari \cn{smash}
nella linea in cui compare il delimitatore pi\`u grande). Nell'argomento di
\cn{vphantom} bisogna mettere una copia dell'elemento pi\`u alto che compare
nell'altra linea, per esempio
\begin{verbatim}
xxx \left(\int_t yyy\right.\\
  \left.\vphantom{\int_t} zzz ... \right)
\end{verbatim}
\end{error}

\begin{error}{Paragraph ended before \xxx was complete}
\errexa
\begin{verbatim}
Runaway argument?

! Paragraph ended before \multline was complete.
<to be read again>
                   \par
l.100

? h
I suspect you've forgotten a `}', causing me to apply this
control sequence to too much text. How can we recover?
My plan is to forget the whole thing and hope for the best.
?
\end{verbatim}
\errexpl
Questo potrebbe dipendere da un errore di battitura nel comando
|\end{multline}|, per esempio
\begin{verbatim}
\begin{multline}
...
\end{multiline}
\end{verbatim}
o dall'uso di abbreviazioni di alcuni ambienti, come |\bal| e
|\eal| invece di |\begin{align}| e |\end{align}|:
\begin{verbatim}
\bal
...
\eal
\end{verbatim}
Per motivi tecnici quel tipo di abbreviazioni non funzionano con gli
ambienti pi\`u complesi per equazioni in \emph{display} del pacchetto
\pkg{amsmath} (\env{gather}, \env{align}, \env{split}, etc.; cfr.\@ \fn{technote.tex}).
\end{error}

\begin{error}{Runaway argument?}
Si veda la descrizione del messaggio di errore
\texttt{Paragraph ended before \ncn{xxx} was complete}.
\end{error}

\begin{error}{Unknown option `xxx' for package `yyy'}
\errexa
\begin{verbatim}
! LaTeX Error: Unknown option `intlim' for package `amsmath'.
...
? h
The option `intlim' was not declared in package `amsmath', perhaps you
misspelled its name. Try typing  <return>  to proceed.
?
\end{verbatim}
\errexpl
Questo significa che il nome dell'opzione \`e stato scritto male, o
semplicemente che il pacchetto, al  contrario di quanto ci si aspettava,
non ha quell'opzione. Si consulti la documentazione di quel pacchetto.
\end{error}

\begin{error}{Old form `\pmatrix' should be \begin{pmatrix}.}
\errexa
\begin{verbatim}
! Package amsmath Error: Old form `\pmatrix' should be
                         \begin{pmatrix}.

See the amsmath package documentation for explanation.
Type  H <return>  for immediate help.
 ...

\pmatrix ->\left (\matrix@check \pmatrix
                                         \env@matrix
l.16 \pmatrix
             {a&b\cr c&d\cr}
? h
`\pmatrix{...}' is old Plain-TeX syntax whose use is
ill-advised in LaTeX.
?
\end{verbatim}
\errexpl
Quando si usa il paccheto \pkg{amsmath}, le vecchie forme \cn{pmatrix},
\cn{matrix}, e \cn{cases} non posso pi\`u essere usate, a causa di conflitti
di nome. Ad ogni modo, la loro sintassi non era conforme alla sintassi
standard \LaTeX{}.
\end{error}

\begin{error}{Erroneous nesting of equation structures}
\errexa
\begin{verbatim}
! Package amsmath Error: Erroneous nesting of equation structures;
(amsmath)                trying to recover with `aligned'.

See the amsmath package documentation for explanation.
Type  H <return>  for immediate help.
 ...

l.260 \end{alignat*}
                    \end{equation*}
\end{verbatim}
\errexpl
Le strutture \env{align}, \env{alignat}, etc., sono progettate per essere
usate nel livello pi\`u alto, e perlopi\`u non possono essere annidate in
alcune altre strutture di equazioni in \emph{display}. Una eccezione notevole
\`e data dal fatto che \env{align} e molte sue varianti possono essere usate
nell'ambiente \env{gather}.
\end{error}

\section{Messaggi di warning}

\begin{error}{Foreign command \over [or \atop or \above]}
\errexa
\begin{verbatim}
Package amsmath Warning: Foreign command \over; \frac or \genfrac
(amsmath)                should be used instead.
\end{verbatim}
\errexpl L'utilizzo dei comandi di frazione originali del \tex/\mdash
\cs{over}, \cs{atop}, \cs{above}\mdash \`e deprecato quando si usa il
pacchetto \pkg{amsmath}, perch\`e la loro sintassi \`e estranea a \latex/,
e \pkg{amsmath} fornisce comandi equivalenti nativi di \latex/. Si veda
\fn{technote.tex} per ulteriori informazioni.
\end{error}

\begin{error}{Cannot use `split' here}
\errexa
\begin{verbatim}
Package amsmath Warning: Cannot use `split' here;
(amsmath)                trying to recover with `aligned'
\end{verbatim}
\errexpl L'ambiente \env{split} \`e studiato per essere usato con l'intero
corpo di un'equazione, o una intera linea di un ambiente \env{align} o
\env{gather}. Non ci pu\`o essere alcun tipo di materiale prima o
dopo di esso all'interno della stessa struttura contenente:
\begin{verbatim}
\begin{equation}
\left\{ % <-- Proibito
\begin{split}
...
\end{split}
\right. % <-- Proibito
\end{equation}
\end{verbatim}
\end{error}

\section{Output sbagliato}

\subsection{Sezioni numerate 0.1, 5.1, 8.1 invece che 1, 2, 3}
\label{numinverse}

Questo molto probabilmente significa che gli argomenti di \cn{numberwithin}
sono stati inseriti alla rovescia:
\begin{verbatim}
\numberwithin{section}{equation}
\end{verbatim}
Questo significa ``stampa il numero di sezione nella forma
\textit{numero-equazione}.\textit{numero-sezione} e ricomincia da
1 ogni volta che incontri
un'equazione'' mentre probabilmente si voleva ottenere l'effetto opposto
\begin{verbatim}
\numberwithin{equation}{section}
\end{verbatim}

\subsection{Il comando \cn{numberwithin} non ha avuto effetto sui numeri
di equazione}

State guardando la prima sezione del vostro documento? Controllate la
numerazione delle equazioni in altre parti del documento per vedere se
il problema \`e quello descritto in \secref{numinverse}.

%%%%%%%%%%%%%%%%%%%%%%%%%%%%%%%%%%%%%%%%%%%%%%%%%%%%%%%%%%%%%%%%%%%%%%%%
% by RZ

\chapter{Ulteriori informazioni}

\section{Convertire documenti gi\`a scritti}

\subsection{Convertire da \LaTeX{} ``puro''} %%%% plain

Sotto molti aspetti, un documento \LaTeX{} continua a funzionare
allo stesso modo quando al preambolo del documento si aggiunge
\verb'\usepackage{amsmath}'. Il pacchetto \pkg{amsmath} sopprime
per\`o, salvo diversa indicazione, le interruzioni di pagina all'interno di
strutture che contengono equazioni in \emph{display} come \env{eqnarray},
\env{align} e \env{gather}. Per continuare a permettere le
interruzioni di pagina all'interno di \env{eqnarray} dopo essere
passati al pacchetto \pkg{amsmath}, \`e necessario aggiungere la
seguente riga nel preambolo del documento:
\begin{verbatim}
\allowdisplaybreaks[1]
\end{verbatim}
Per assicurare una spaziatura normale attorno ai simboli di relazione,
si dove sostituire \env{eqnarray} con \env{align}, \env{multline} o
\env{equation}\slash\env{split}, in maniera appropriata.

La maggior parte delle altre differenze d'uso del pacchetto
\pkg{amsmath} possono essere considerate raffinatezze facoltative,
come per esempio l'uso di
\begin{verbatim}
\DeclareMathOperator{\Hom}{Hom}
\end{verbatim}
invece di \verb'\newcommand{\Hom}{\mbox{Hom}}'.

\subsection{Convertire da \amslatex/ 1.1}
Si veda \fn{diffs-m.txt}.

\section{Note tecniche}
Il file \fn{technote.tex} contiene alcuni commenti su diverse questioni
che difficilmente possono essere di interesse generale.

\section{Ottenere aiuto}

Domande o commenti riguardanti \pkg{amsmath} e pacchetti correlati
dovrebbero essere inviati a:
\begin{infoaddress}
American Mathematical Society\\
Technical Support\\
Electronic Products and Services\\
P. O. Box 6248\\
Providence, RI 02940\\[3pt]
Phone: 800-321-4AMS (321-4267) \quad or \quad 401-455-4080\\
Internet: \mail{tech-support@ams.org}
\end{infoaddress}
Quando si riporta un problema occorre includere, per consentire
un'indagine adeguata, le seguenti informazioni:

\begin{enumerate}
\item Il file sorgente in cui \`e sorto il problema, preferibilmente
  ridotto alle minime dimensioni rimuovendo tutto il materiale che pu\`o
  essere rimosso senza interferire sul problema in questione.
\item Un file di log di \latex/ che mostri il messaggio di errore (se
  presente) e i numeri di versione delle classi di documento e file di
  opzioni in uso.
\end{enumerate}

\section{Di possibile interesse}\label{a:possible-interest}
\`E possibile avere informazioni su come ottenere i \emph{font} AMS o
altro materiale relativo a \tex/ dall'archivio Internet AMS
\fn{e-math.ams.org} inviando una richiesta attraverso la posta
elettronica a: \mail{webmaster@ams.org}.

Si possono avere informazioni su come ottenere dall'AMS la distribuzione
\pkg{amsmath} su dischetti da:
\begin{infoaddress}
American Mathematical Society\\
Customer Services\\
P. O. Box 6248\\
Providence, RI 02940\\[3pt]
Phone: 800-321-4AMS (321-4267) \quad or \quad 401-455-4000\\
Internet: \mail{cust-serv@ams.org}
\end{infoaddress}

Il ``\tex/ Users Group\index{TeX Users@\tex/ Users Group}''
\`e una organizzazione senza scopo di lucro che pubblica una
rivista (\journalname{TUGboat}\index{TUGboat@\journalname{TUGboat}}),
organizza meeting, e serve da punto di smistamento per informazioni
su \tex/ e software relativo ad esso.
\begin{infoaddress}
\tex/ Users Group\\
PO Box 2311\\
Portland, OR 97208-2311\\
USA\\[3pt]
Phone: +1-503-223-9994\\
Email: \mail{office@tug.org}
\end{infoaddress}
Iscriversi al ``\tex/ Users Group'' \`e un buon modo per sostenere il
continuo sviluppo di software libero relativo a \tex/.
Esistono inoltre molti ``\tex/ users group'' locali in altri stati;
si possono ottenere informazioni su come contattare un gruppo locale dal
``\tex/ Users Group''.

Esiste un gruppo di discussione Usenet chiamato \fn{comp.text.tex},
che \`e una buona fonte di informazioni su \latex/ e
\tex/ in generale. Se non si sa come leggere un gruppo di discussione,
occorre chiedere all'amministratore di sistema locale se \`e
disponibile un servizio di lettura di \emph{newsgroup}.


\begin{thebibliography}{9}
\addcontentsline{toc}{chapter}{Bibliografia}

\bibitem{amsfonts}\booktitle{AMSFonts version \textup{2.2}\mdash user's guide},
Amer. Math. Soc., Providence, RI, 1994; distribuito
con il pacchetto AMSFonts.

\bibitem{instr-l}\booktitle{Instructions for preparation of
papers and monographs\mdash \amslatex/},
Amer. Math. Soc., Providence, RI, 1996, 1999.

\bibitem{amsthdoc}\booktitle{Using the \pkg{amsthm} Package},
Amer. Math. Soc., Providence, RI, 1999.

\bibitem{tlc} Michel Goossens, Frank Mittelbach e Alexander Samarin,
\booktitle{The \latex/ companion}, Addison-Wesley, Reading, MA, 1994.
  [\emph{Note: L'edizione del 1994 non \`e una guida affidabile per il
    pacchetto \pkg{amsmath} a meno che non ci si riferisca al file
    \fn{compan.err}, distribuito con \LaTeX{}, che contiene una errata
    corrige per il Capitolo 8\mdash.}]

% Deal with a line breaking problem
\begin{raggedright}
\bibitem{mil} G. Gr\"{a}tzer,
\emph{Math into \LaTeX{}: An Introduction to \LaTeX{} and AMS-\LaTeX{}}
  \url{http://www.ams.org/cgi-bin/bookstore/bookpromo?fn=91&arg1=bookvideo&itmc=MLTEX},
Birkh\"{a}user, Boston, 1995.\par
\end{raggedright}

\bibitem{kn} Donald E. Knuth, \booktitle{The \tex/book},
Addison-Wesley, Reading, MA, 1984.

\bibitem{lamport} Leslie Lamport, \booktitle{\latex/: A document preparation
system}, 2nd revised ed., Addison-Wesley, Reading, MA, 1994.

\bibitem{msf} Frank Mittelbach and Rainer Sch\"opf,
\textit{The new font family selection\mdash user
interface to standard \latex/}, \journalname{TUGboat} \textbf{11},
no.~2 (June 1990), pp.~297\ndash 305.

\bibitem{jt} Michael Spivak, \booktitle{The joy of \tex/}, 2nd revised ed.,
Amer. Math. Soc., Providence, RI, 1990.

\end{thebibliography}
% by RZ - end

%%%% ====================================================================
%%% amsldoc.tex 2.07
%%% ====================================================================
\documentclass[a4paper,leqno,titlepage,openany]{amsldoc}[1999/12/13]
\usepackage[italian]{babel}
\renewcommand{\errexa}{\par\noindent\textit{Esempio}:\ }
\renewcommand{\errexpl}{\par\noindent\textit{Spiegazione}:\ }
\DeclareRobustCommand{\cls}[1]{{\ntt#1}%
  \autoindex{#1@\string\cls{#1}, classe}}
\DeclareRobustCommand{\pkg}[1]{{\ntt#1}%
  \autoindex{#1@\string\pkg{#1}, pacchetto}}
\DeclareRobustCommand{\opt}[1]{{\ntt#1}%
  \autoindex{#1@\string\opt{#1}, opzione}}
\DeclareRobustCommand{\env}[1]{{\ntt#1}%
  \autoindex{#1@\string\env{#1}, ambiente}}
\DeclareRobustCommand{\fn}[1]{{\ntt#1}%
   \autoindex{#1@\string\fn{#1}}}
\DeclareRobustCommand{\bst}[1]{{\ntt#1}\autoindex{#1@{\string\ntt{}#1,
  stile bibliografico}}}

\ifx\UndEfiNed\url
  \ClassError{amsldoc}{%
    This version of amsldoc.tex must be processed\MessageBreak
    with a newer version of amsldoc.cls (2.02 or later)}{}
\fi

\title{Manuale utente per il pacchetto \pkg{amsmath} (versione~2.0)}
\author{American Mathematical Society}
\date{13/12/1999}

%    Use the amsmath package and amscd package in order to print
%    examples.
\usepackage{amsmath}
\usepackage{amscd}
% Inserito il pacchetto makeidx - GD
\usepackage{makeidx}

\makeindex % generate index data
\providecommand{\see}[2]{\textit{vedi} #1}

%    The amsldoc class includes a number of features useful for
%    documentation about TeX, including:
%
%    ---Commands \tex/, \amstex/, \latex/, ... for uniform treatment
%    of the various logos and easy handling of following spaces.
%
%    ---Commands for printing various common elements: \cn for command
%    names, \fn for file names (including font-file names), \env for
%    environments, \pkg and \cls for packages and classes, etc.

%    Many of the command names used here are rather long and will
%    contribute to poor linebreaking if we follow the \latex/ practice
%    of not hyphenating anything set in tt font; instead we selectively
%    allow some hyphenation.
\allowtthyphens % defined in amsldoc.cls

\hyphenation{ac-cent-ed-sym-bol add-to-counter add-to-length align-at
  aligned-at allow-dis-play-breaks ams-art ams-cd ams-la-tex amsl-doc
  ams-symb ams-tex ams-text ams-xtra bmatrix bold-sym-bol cen-ter-tags
  eqn-ar-ray idots-int int-lim-its latex med-space neg-med-space
  neg-thick-space neg-thin-space no-int-lim-its no-name-lim-its
  over-left-arrow over-left-right-arrow over-right-arrow pmatrix
  qed-sym-bol set-length side-set small-er tbinom the-equa-tion
  thick-space thin-space un-der-left-arrow un-der-left-right-arrow
  un-der-right-arrow use-pack-age var-inj-lim var-proj-lim vmatrix
  xalign-at xx-align-at}

%    Prepare for illustrating the \vec example
\newcommand{\vect}[1]{\mathbf{#1}}

\newcommand{\booktitle}[1]{\textit{#1}}
\newcommand{\journalname}[1]{\textit{#1}}
\newcommand{\seriesname}[1]{\textit{#1}}

%    Command to insert and index a particular phrase. Doesn't work for
%    certain kinds of special characters in the argument.
\newcommand{\ii}[1]{#1\index{#1}}

\newcommand{\vstrut}[1]{\vrule width0pt height#1\relax}

%    An environment for presenting comprehensive address information:
\newenvironment{infoaddress}{%
  \par\topsep\medskipamount
  \trivlist\centering
  \item[]%
  \begin{minipage}{.7\columnwidth}%
  \raggedright
}{%
  \end{minipage}%
  \endtrivlist
}

\newenvironment{eqxample}{%
  \par\addvspace\medskipamount
  \noindent\begin{minipage}{.5\columnwidth}%
  \def\producing{\end{minipage}\begin{minipage}{.5\columnwidth}%
    \hbox\bgroup\kern-.2pt\vrule width.2pt%
      \vbox\bgroup\parindent0pt\relax
%    The 3pt is to cancel the -\lineskip from \displ@y
    \abovedisplayskip3pt \abovedisplayshortskip\abovedisplayskip
    \belowdisplayskip0pt \belowdisplayshortskip\belowdisplayskip
    \noindent}
}{%
  \par
%    Ensure that a lonely \[\] structure doesn't take up width less than
%    \hsize.
  \hrule height0pt width\hsize
  \egroup\vrule width.2pt\kern-.2pt\egroup
  \end{minipage}%
  \par\addvspace\medskipamount
}

%    The chapters are so short, perhaps we shouldn't call them by the
%    name `Chapter'. We make \chaptername read an argument in order to
%    remove a following \space or "{} " (both possibilities are present
%    in book.cls).

\renewcommand{\chaptername}[1]{}
\newcommand{\chapnum}[1]{\mdash #1\mdash }
\makeatletter
\def\@makechapterhead#1{%
  \vspace{1.5\baselineskip}%
  {\parindent \z@ \raggedright \reset@font
    \ifnum \c@secnumdepth >\m@ne
      \large\bfseries \chapnum{\thechapter}%
      \par\nobreak
      \vskip.5\baselineskip\relax
    \fi
    #1\par\nobreak
    \vskip\baselineskip
  }}
\makeatother

%    A command for ragged-right parbox in a tabular.
\newcommand{\rp}{\let\PBS\\\raggedright\let\\\PBS}

%    Non-indexed file name
\newcommand{\nfn}[1]{\texttt{#1}}

%    For the examples in the math spacing table.
%%\newcommand{\lspx}{\mbox{\rule{5pt}{.6pt}\rule{.6pt}{6pt}}}
%%\newcommand{\rspx}{\mbox{\rule[-1pt]{.6pt}{7pt}%
%%  \rule[-1pt]{5pt}{.6pt}}}
\newcommand{\lspx}{\mathord{\Rightarrow\mkern-1mu}}
\newcommand{\rspx}{\mathord{\mkern-1mu\Leftarrow}}
\newcommand{\spx}[1]{$\lspx #1\rspx$}

%    For a list of characters representing document input.
\newcommand{\clist}[1]{%
  \mbox{\ntt\spaceskip.2em plus.1em \xspaceskip\spaceskip#1}}

%    Fix weird \latex/ definition of rightmark.
\makeatletter
\def\rightmark{\expandafter\@rightmark\botmark{}{}}
%    Also turn off section marks.
\let\sectionmark\@gobble
\renewcommand{\chaptermark}[1]{%
  \uppercase{\markboth{\rhcn#1}{\rhcn#1}}}
\newcommand{\rhcn}{\thechapter. }
\makeatother

%    Include down to \section but not \subsection, in toc:
\setcounter{tocdepth}{1}

\DeclareMathOperator{\ix}{ix}
\DeclareMathOperator{\nul}{nul}
\DeclareMathOperator{\End}{End}
\DeclareMathOperator{\xxx}{xxx}

\begin{document}

%%%%%%%%%%%%%%%%%%%%%%%%%%%%%%%%%%%%%%%%%%%%%%%%%%%%%%%%%%%%%%%%%%%%%%%%
\frontmatter

\maketitle
%
%
%
\pagebreak
\begin{small} 
 \noindent Titolo originale: \emph{User manual for the \pkg{amsmath} package (version~2.0)}

 \smallskip
 \noindent Traduzione:

 \begin{quote}
 \flushleft %  \footnotesize
 Giulio Agostini, % <giulio.agostini@bigfoot.com>
 Giuseppe Bilotta, % <bourbaki@bigfoot.com>
 Flavio Casadei Della Chiesa, % <flavio_c@libero.it>
 Onofrio de Bari, % <thufir@tin.it>
 Giacomo Delre, % <giader@penguinpowered.com>
 Luca Ferrante, % <ironluke@split.it>
 Tommaso Pecorella, % <t.pecorella@inwind.it>
 Mileto Rigido, % <m.rigido@flashnet.it>
 Roberto Zanasi. % <roberto.zanasi@libero.it>
 \end{quote}

\end{small}
%
%
%
\pagestyle{headings}
\tableofcontents
\cleardoublepage % for better page number placement

%%%%%%%%%%%%%%%%%%%%%%%%%%%%%%%%%%%%%%%%%%%%%%%%%%%%%%%%%%%%%%%%%%%%%%%%
\mainmatter
%%%%%%%%%%%%%%%%%%%%%%%%%%%%%%%%%%%%%%%%%%%%%%%%%%%%%%%%%%%%%%%%%%%%%%%%

%%% Nota dei traduttori
\subsubsection*{Nota alla traduzione italiana}
Una copia di questo documento e altre traduzioni in italiano di
manuali su \LaTeX\ sono reperibili presso
\begin{itemize}
\item\url{http://guild.prato.linux.it}
\item\url{ftp://lorien.prato.linux.it/pub/guild}
\item\url{ftp://ftp.unina.it/pub/TeX/info/italian}
\end{itemize}
e su ogni sito CTAN, per esempio \url{ftp://ftp.tex.ac.uk/tex-archive/info/italian}.
%%%%%%%%%%%%%%%%%%%%%%%%%%%%%%%%%%%%%%%%%%%%%%%%%%%%%%%%%%%%%%%%%%%%%%%%


% by GA
\chapter{Introduzione}

Il pacchetto \pkg{amsmath} \`e un pacchetto \LaTeX{} che fornisce
svariate estensioni per il miglioramento della struttura informativa e
della stampa di documenti che contengono formule matematiche. I lettori
che non conoscono \LaTeX{} sono invitati a consultare \cite{lamport}.
Se si possiede una versione aggiornata di \LaTeX{}, il pacchetto \pkg{amsmath}
\`e normalmente incluso. Quando viene pubblicata una nuova versione del
pacchetto \pkg{amsmath}, \`e possibile effettuare un aggiornamento attraverso
\url{http://www.ams.org/tex/amsmath.html} o
\url{ftp://ftp.ams.org/pub/tex/}.

Questo documento descrive le funzionalit\`a del paccheto \pkg{amsmath}
e spiega come dovrebbero essere usate. Esso copre inoltre alcuni pacchetti
ausiliari:
\begin{ctab}{ll}
\pkg{amsbsy}& \pkg{amstext}\\
\pkg{amscd}& \pkg{amsxtra}\\
\pkg{amsopn}
\end{ctab}
Tutti questi hanno a che vedere con il contenuto di formule
matematiche. Per informazioni su ulteriori simboli e \emph{font} matematici,
si veda \cite{amsfonts} e \url{http://www.ams.org/tex/amsfonts.html}.
Per la documentazione del pacchetto \pkg{amsthm} o delle classi AMS
(\cls{amsart}, \cls{amsbook}, etc.\@) si veda \cite{amsthdoc} o
\cite{instr-l} e \url{http://www.ams.org/tex/author-info.html}.

Se siete utenti di \latex/ da molto tempo e avete molta matematica nei
vostri scritti, potreste trovare soluzioni a problemi familiari in
questo elenco di funzionalit\`a di \pkg{amsmath}:
\begin{itemize}

\item Un modo comodo per definire un nuovo comando `nome di operatore', come
\cn{sin} e \cn{lim}, con spazi appropriati ai lati e selezione automatica
di stile e dimensioni corrette del \emph{font} (anche quando usato in esponenti
o deponenti).

\item Diversi alternative all'ambiente \env{eqnarray} per rendere le
diverse disposizioni delle equazioni pi\`u facili da scrivere.

\item I numeri delle equazioni si spostano automaticamente in alto o in
basso per evitare di sovrapporsi con l'equazione stessa (al contrario
di \env{eqnarray}).

\item Gli spazi attorno ai segni di uguaglianza sono gli stessi della
normale spaziatura nell'ambiente \env{equation} (al contrario di
\env{eqnarray}).

\item Un modo per produrre deponenti a pi\`u linee come spesso \`e
richiesto dai simboli di sommatoria e produttoria.

\item Un modo semplice di numerare una determinata equazione con un
riferimento diverso da quello fornito dalla numerazione automatica.

\item Un modo semplice di produrre numerazioni subordinate per le
equazioni, nella forma (1.3a) (1.3b) (1.3c), per un determinato insieme
di equazioni.

\end{itemize}

Il pacchetto \pkg{amsmath} \`e distribuito insieme ad alcuni piccoli
pacchetti ausiliari:
\begin{description}
\item[\pkg{amsmath}] Il pacchetto principale, fornisce diverse funzionalit\`a
  per equazioni in \emph{display} e altri costrutti matematici.

\item[\pkg{amstext}] Fornisce il comando \cn{text} per
  sistemare un frammento di testo in un \emph{display}.

\item[\pkg{amsopn}] Fornisce il comando \cn{DeclareMathOperator} per definire
  nuovi `nomi di operatori' come \cn{sin} e \cn{lim}.

\item[\pkg{amsbsy}] Per compatibilit\`a all'indietro questo pacchetto
  continua a esistere, ma in alternativa ad esso si consiglia l'uso del
  pi\`u recente pacchetto \pkg{bm} fornito a corredo di \LaTeX{}.

\item[\pkg{amscd}] Fornisce un ambiente \env{CD} per semplici diagrammi
  commutativi (privi di frecce diagonali).

\item[\pkg{amsxtra}] Fornisce alcune cianfrusaglie come \cn{fracwithdelims}
  e \cn{accentedsymbol}, per compatibilit\`a con documenti creati usando
  la versione~1.1.

\end{description}

Il pacchetto \pkg{amsmath} incorpora \pkg{amstext}, \pkg{amsopn}, e
\pkg{amsbsy}. Le funzionalit\`a di \pkg{amscd} e \pkg{amsxtra}, invece,
sono disponibili solo invocando separatamente questi pacchetti.

%%%%%%%%%%%%%%%%%%%%%%%%%%%%%%%%%%%%%%%%%%%%%%%%%%%%%%%%%%%%%%%%%%%%%%%%
% by GA

\chapter{Opzioni per il pacchetto \pkg{amsmath}}\label{options}

Il pacchetto \pkg{amsmath} ha le seguenti opzioni:
\begin{description}

\item[\opt{centertags}] (\emph{default}) Centra verticalmente\index{equazioni,
numeri delle!posizionamento verticale}, rispetto all'altezza totale
dell'equazione, la numerazione delle equazioni spezzate su pi\`u linee.

\item[\opt{tbtags}] `Top-or-bottom tags' (Etichette in cima o in fondo):
Allinea la numerazione\index{equazioni, numeri delle!posizionamento
verticale} delle equazioni spezzate su pi\`u linee all'ultima
(rispettivamente alla prima) linea, se i numeri stanno sulla destra
(rispettivamente sulla sinistra).

\item[\opt{sumlimits}] (\emph{default}) Posiziona esponenti e
deponenti\index{esponenti e deponenti!posizionamento}\relax
\index{limiti|see{esponenti e deponenti}} dei simboli di sommatoria
sopra e sotto, nelle equazioni in \emph{display}. Questa opzione
influenza anche altri simboli dello stesso tipo\mdash $\prod$,
$\coprod$, $\bigotimes$, $\bigoplus$, e cos\`\i\ via\mdash eccetto gli
integrali (vedi sotto).

\item[\opt{nosumlimits}] Posiziona gli esponenti e deponenti dei
simboli simil-sommatoria sempre a fianco, anche nelle equazioni in
\emph{display}.

\item[\opt{intlimits}] Come \opt{sumlimits}, ma per i simboli di
integrale\index{integrali!posizionamento dei limiti}.

\item[\opt{nointlimits}] (\emph{default}) Il contrario di \opt{intlimits}.

\item[\opt{namelimits}] (\emph{default}) Come \opt{sumlimits}, ma per certi
`nomi di operatori' come $\det$, $\inf$, $\lim$, $\max$, $\min$, che
tradizionalmente hanno deponenti \index{esponenti e
deponenti!posizionamento} posizionati sotto di essi all'interno di
equazioni \emph{display}.

\item[\opt{nonamelimits}] Il contrario di \opt{namelimits}.

\end{description}

Per usare una di queste opzioni del pacchetto bisogna mettere il nome
dell'opzione nell'argomento opzionale del comando \cn{usepackage}\mdash
ad esempio, \verb"\usepackage[intlimits]{amsmath}".

Il pacchetto \pkg{amsmath} inoltre riconosce le seguenti opzioni che
sono normalmente selezionate (implicitamente o esplicitamente)
attraverso il comando \cn{documentclass}, e che pertanto non hanno
bisogno di essere ripetute nell'elenco di opzioni del comando
\cn{usepackage}|{amsmath}|.
\begin{description}

\item[\opt{leqno}] Posiziona i numeri di equazione sulla
sinistra.\index{equazioni, numeri delle!posizionamento a destra o a
sinistra}

\item[\opt{reqno}] Posiziona i numeri di equazione sulla destra.

\item[\opt{fleqn}] Posiziona i numeri di equazione a una distanza prefissata
dal margine sinistro piuttosto che centrata nella colonna di
testo.\index{equazioni in \emph{display}!centratura}

\end{description}

%%%%%%%%%%%%%%%%%%%%%%%%%%%%%%%%%%%%%%%%%%%%%%%%%%%%%%%%%%%%%%%%%%%%%%%%
% by GB

\chapter{Equazioni in \emph{display}}

\section{Introduzione}

Il pacchetto \pkg{amsmath} fornisce un certo numero di nuove strutture
per le equazioni in \emph{display}\index{equazioni in
\emph{display}}\index{equazioni|see{equazioni in \emph{display}}},
oltre a quelle fornite dal \latex/ di base; fra queste:
\begin{verbatim}
  equation     equation*     align       align*
  gather       gather*       flalign     flalign*
  multline     multline*     alignat     alignat*
  split
\end{verbatim}
(Sebbene l'ambiente standard \env{eqnarray} rimanga disponibile, \`e
opportuno usare \env{align} o \env{equation}+\env{split}, invece.)

Con l'eccezione di \env{split}, ogni ambiente ha sia una versione
stellata sia una non stellata, dove la versione non stellata permette
la numerazione automatica usando il contatore \latex/ \env{equation}.
Si pu\`o sopprimere il numero in ogni singola linea premettendo un
\cn{notag} al codice \cn{\\}; lo si pu\`o anche
scavalcare\index{equazioni, numeri delle!scavalcare} con un valore di
propria scelta, usando il comando \cn{tag}|{|\<etich>|}|, dove \<etich>
\`e un testo arbitrario, come |$*$| o |ii|, usato  per \qq{numerare}
l'equazione. Si pu\`o anche usare il comando \cn{tag*}, che fa in modo
che il testo fornito venga scritto letteralmente, senza aggiunta di
parentesi. \cn{tag} e \cn{tag*} possono anche essere usati nelle
versioni non numerate di tutte le strutture di allineamento di
\pkg{amsmath}. Alcuni esempi dell'uso di \cn{tag} possono essere
trovati nei file di esempio \fn{testmath.tex} e \fn{subeqn.tex}
forniti con il pacchetto \pkg{amsmath}.

L'ambiente \env{split} \`e una speciale forma subordinata, da usare
solo \emph{all'interno} di altre strutture; non pu\`o essere usato in
una \env{multline}.

Nelle strutture d'allineamento (\env{split}, \env{align} e varianti),
i simboli di relazione hanno un \verb'&' prima, ma non dopo\mdash a
differenza di \env{eqnarray}. Mettere un \verb'&' dopo il simbolo di
relazione interferirebbe con la spaziatura: \`e necessario metterlo
prima.

\begin{table}[p]
\caption[]{Confronto degli ambienti per le equazioni in \emph{display}
(le linee verticali indicano i margini nominali)}\label{displays}
\renewcommand{\theequation}{\arabic{equation}}
\begin{eqxample}
\begin{verbatim}
\begin{equation*}
a=b
\end{equation*}
\end{verbatim}
\producing
\begin{equation*}
a=b
\end{equation*}
\end{eqxample}

\begin{eqxample}
\begin{verbatim}
\begin{equation}
a=b
\end{equation}
\end{verbatim}
\producing
\begin{equation}
a=b
\end{equation}
\end{eqxample}

\begin{eqxample}
\begin{verbatim}
\begin{equation}\label{xx}
\begin{split}
a& =b+c-d\\
 & \quad +e-f\\
 & =g+h\\
 & =i
\end{split}
\end{equation}
\end{verbatim}
\producing
\begin{equation}\label{xx}
\begin{split}
a& =b+c-d\\
 & \quad +e-f\\
 & =g+h\\
 & =i
\end{split}
\end{equation}
\end{eqxample}

\begin{eqxample}
\begin{verbatim}
\begin{multline}
a+b+c+d+e+f\\
+i+j+k+l+m+n
\end{multline}
\end{verbatim}
\producing
\begin{multline}
a+b+c+d+e+f\\
+i+j+k+l+m+n
\end{multline}
\end{eqxample}

\begin{eqxample}
\begin{verbatim}
\begin{gather}
a_1=b_1+c_1\\
a_2=b_2+c_2-d_2+e_2
\end{gather}
\end{verbatim}
\producing
\begin{gather}
a_1=b_1+c_1\\
a_2=b_2+c_2-d_2+e_2
\end{gather}
\end{eqxample}

\begin{eqxample}
\begin{verbatim}
\begin{align}
a_1& =b_1+c_1\\
a_2& =b_2+c_2-d_2+e_2
\end{align}
\end{verbatim}
\producing
\begin{align}
a_1& =b_1+c_1\\
a_2& =b_2+c_2-d_2+e_2
\end{align}
\end{eqxample}

\begin{eqxample}
\begin{verbatim}
\begin{align}
a_{11}& =b_{11}&
  a_{12}& =b_{12}\\
a_{21}& =b_{21}&
  a_{22}& =b_{22}+c_{22}
\end{align}
\end{verbatim}
\producing
\begin{align}
a_{11}& =b_{11}&
  a_{12}& =b_{12}\\
a_{21}& =b_{21}&
  a_{22}& =b_{22}+c_{22}
\end{align}
\end{eqxample}

\begin{eqxample}
\begin{verbatim}
\begin{flalign*}
a_{11}& =b_{11}&
  a_{12}& =b_{12}\\
a_{21}& =b_{21}&
  a_{22}& =b_{22}+c_{22}
\end{flalign*}
\end{verbatim}
\producing
\begin{flalign*}
a_{11}& =b_{11}&
  a_{12}& =b_{12}\\
a_{21}& =b_{21}&
  a_{22}& =b_{22}+c_{22}
\end{flalign*}
\end{eqxample}
\end{table}

\section{Singole equazioni}

L'ambiente \env{equation} viene usato per singole equazioni con
numerazione automatica; l'ambiente \env{equation*} ha la stessa
funzione, senza numerazione.%
%%%%%%%%%%%%%%%%%%%%%%%%%%%%%%%%%%%%%%%%%%%%%%%%%%%%%%%%%%%%%%%%%%%%%%%%
\footnote{\latex/ non fornisce un ambiente \env{equation*}, ma un
ambiente con funzioni analoghe: \env{displaymath}.}

\section{Equazioni spezzate senza allineamento}

L'ambiente \env{multline} \`e una variante di \env{equation}, usata
per le equazioni che non entrano in un'unica riga. La prima riga di una
\env{multline} sar\`a al margine sinistro, e l'ultima al margine
destro, tranne per un rientro ambo i lati, di lunghezza
\cn{multlinegap}; tutte le altre linee verranno centrate
indipendentemente considerando la larghezza del \emph{display} (a meno
che non sia in funzione l'opzione \opt{fleqn}).

Come \env{equation}, \env{multline} fornisce un'unico numero
d'equazione (quindi, nessuna delle singole linee dovrebbe essere
segnata con \cn{notag}). Il numero dell'equazione \`e posto all'ultima
riga (opzione \opt{reqno}) o sulla prima linea (opzione \opt{leqno});
il centramento verticale (come per \env{split}) non \`e supportato in
\env{multline}.

\`E possibile forzare una delle righe di centro a sinistra o a destra
con i comandi \cn{shoveleft}, \cn{shoveright}; questi comandi prendono
l'intera linea come argomento, fino al segno \cn{\\} escluso; ad
esempio
\begin{multline}
\framebox[.65\columnwidth]{A}\\
\framebox[.5\columnwidth]{B}\\
\shoveright{\framebox[.55\columnwidth]{C}}\\
\framebox[.65\columnwidth]{D}
\end{multline}
\begin{verbatim}
\begin{multline}
\framebox[.65\columnwidth]{A}\\
\framebox[.5\columnwidth]{B}\\
\shoveright{\framebox[.55\columnwidth]{C}}\\
\framebox[.65\columnwidth]{D}
\end{multline}
\end{verbatim}

Il valore di \cn{multlinegap} pu\`o essere cambiato con i soliti
comandi \latex/ \cn{setlength} or \cn{addtolength}.

\section{Equazioni spezzate con allineamento}

Come \env{multline}, l'ambiente \env{split} \`e per \emph{singole}
equazioni troppo lunghe per entrare in una riga e che pertanto devono
essere spezzate. A differenza di \env{multline}, per\`o, l'ambiente
\env{split} permette allineamento tra le linee, con l'uso di simboli
|&| per segnare i punti di allineamento. A differenza di altre
strutture di equazioni \pkg{amsmath}, l'ambiente \env{split} non
produce numeri, poich\'e \`e progettato per essere usato
\emph{esclusivamente all'interno di qualche altra struttura per
equazioni in \emph{display}}, solitamente un ambiente \env{equation},
\env{align}, o \env{gather}, che fornisce la numerazione; ad esempio:
\begin{equation}\label{e:barwq}\begin{split}
H_c&=\frac{1}{2n} \sum^n_{l=0}(-1)^{l}(n-{l})^{p-2}
\sum_{l _1+\dots+ l _p=l}\prod^p_{i=1} \binom{n_i}{l _i}\\
&\quad\cdot[(n-l )-(n_i-l _i)]^{n_i-l _i}\cdot
\Bigl[(n-l )^2-\sum^p_{j=1}(n_i-l _i)^2\Bigr].
\kern-2em % adjust equation body to the right [mjd,13-Nov-1994]
\end{split}\end{equation}

\begin{verbatim}
\begin{equation}\label{e:barwq}\begin{split}
H_c&=\frac{1}{2n} \sum^n_{l=0}(-1)^{l}(n-{l})^{p-2}
\sum_{l _1+\dots+ l _p=l}\prod^p_{i=1} \binom{n_i}{l _i}\\
&\quad\cdot[(n-l )-(n_i-l _i)]^{n_i-l _i}\cdot
\Bigl[(n-l )^2-\sum^p_{j=1}(n_i-l _i)^2\Bigr].
\end{split}\end{equation}
\end{verbatim}

La struttura \env{split} dovrebbe costituire l'intero corpo della
struttura racchiudente, tranne per comandi come \cn{label} che non
producono testo visibile.

\section{Gruppi di equazioni senza allineamento}

L'ambiente \env{gather} viene usato per ragguppare equazioni
consecutive quando non vi \`e necessit\`a di allineamento; ogni
equazione \`e centrata separatamente entro i margini (come in
Tabella~\ref{displays}). Le equazioni in un ambiente \env{gather} sono
separati da comandi \cn{\bslash}. Ogni equazione \env{gather} pu\`o
essere un blocco \verb'\begin{split}'
  \dots\ \verb'\end{split}' \mdash ad esempio:
\begin{verbatim}
\begin{gather}
  prima equazione\\
  \begin{split}
    seconda & equazione\\
           & su due linee
  \end{split}
  \\
  terza equazione
\end{gather}
\end{verbatim}

\section{Gruppi di equazioni con allineamento reciproco}

L'ambiente \env{align} \`e usato per gruppi di due o pi\`u  equazioni
quando \`e richiesto allineamento reciproco; di solito vengono scelti
i simboli di relazione per gli allineamenti (come in
Tabella~\ref{displays}).

Per avere pi\`u colonne di equazioni affiancate, si possono usare
simboli di ``e'' commerciale aggiuntivi per separare le colonne:
\begin{align}
x&=y       & X&=Y       & a&=b+c\\
x'&=y'     & X'&=Y'     & a'&=b\\
x+x'&=y+y' & X+X'&=Y+Y' & a'b&=c'b
\end{align}
%
\begin{verbatim}
\begin{align}
x&=y       & X&=Y       & a&=b+c\\
x'&=y'     & X'&=Y'     & a'&=b\\
x+x'&=y+y' & X+X'&=Y+Y' & a'b&=c'b
\end{align}
\end{verbatim}
Annotazioni linea-per-linea sulle equazioni possono essere ottenute
con un opportuno uso di \cn{text} in un ambiente \env{align}:
\begin{align}
x& = y_1-y_2+y_3-y_5+y_8-\dots
                    && \text{per \eqref{eq:C}}\\
 & = y'\circ y^*    && \text{per \eqref{eq:D}}\\
 & = y(0) y'        && \text {per l'Assioma 1.}
\end{align}
%
\begin{verbatim}
\begin{align}
x& = y_1-y_2+y_3-y_5+y_8-\dots
                    && \text{per \eqref{eq:C}}\\
 & = y'\circ y^*    && \text{per \eqref{eq:D}}\\
 & = y(0) y'        && \text {per l'Assioma 1.}
\end{align}
\end{verbatim}
Una variante, l'ambiente \env{alignat}, permette di specificare
manualmente lo spazio orizzontale fra le equazioni; questo ambiente ha
un argomento obbligatorio, il numero di \qq{colonne di equazioni}: si
contano il numero di \verb'&' in una riga, si aggiunge 1 e si divide
per 2.
\begin{alignat}{2}
x& = y_1-y_2+y_3-y_5+y_8-\dots
                  &\quad& \text{per \eqref{eq:C}}\\
 & = y'\circ y^*  && \text{per \eqref{eq:D}}\\
 & = y(0) y'      && \text {per l'Assioma 1.}
\end{alignat}
%
\begin{verbatim}
\begin{alignat}{2}
x& = y_1-y_2+y_3-y_5+y_8-\dots
                  &\quad& \text{per \eqref{eq:C}}\\
 & = y'\circ y^*  && \text{per \eqref{eq:D}}\\
 & = y(0) y'      && \text {per l'Assioma 1.}
\end{alignat}
\end{verbatim}

\section{Blocchi per costrutti allineati}

Come \env{equation}, gli ambienti a equazioni multiple \env{gather},
\env{align} e \env{alignat} sono progettati per produrre strutture
aventi lunghezza complessiva pari alla lunghezza di una riga; questo
implica, ad esempio, che non \`e facile aggiungere parentesi attorno
alle strutture; vengono quindi fornite le varianti \env{gathered},
\env{aligned} e \env{alignedat}, la cui lunghezza totale \`e pari alla
reale lunghezza dei contenuti; possono quindi essere usate come
componenti di un'espressione pi\`u complessa; ad esempio,
\begin{equation*}
\left.\begin{aligned}
  B'&=-\partial\times E,\\
  E'&=\partial\times B - 4\pi j,
\end{aligned}
\right\}
\qquad \text{equazioni di Maxwell}
\end{equation*}
\begin{verbatim}
\begin{equation*}
\left.\begin{aligned}
  B'&=-\partial\times E,\\
  E'&=\partial\times B - 4\pi j,
\end{aligned}
\right\}
\qquad \text{equazioni di Maxwell}
\end{equation*}
\end{verbatim}
Come l'ambiente \env{array}, le varianti \texttt{-ed} possono
accettare un argomento facoltativo \verb'[t]' o \verb'[b]' per
specificare il posizionamento verticale.

Costrutti di tipo \qq{casi} come il seguente sono comuni in matematica:
\begin{equation}\label{eq:C}
P_{r-j}=
  \begin{cases}
    0&  \text{se $r-j$ \`e dispari},\\
    r!\,(-1)^{(r-j)/2}&  \text{se $r-j$ \`e pari}.
  \end{cases}
\end{equation}
e nel pacchetto \pkg{amsmath} c'\`e un ambiente \env{cases} per
facilitarne la scrittura:
\begin{verbatim}
P_{r-j}=
  \begin{cases}
    0&  \text{se $r-j$ \`e dispari},\\
    r!\,(-1)^{(r-j)/2}&  \text{se $r-j$ \`e pari}.
  \end{cases}
\end{verbatim}
Osservare l'uso di \cn{text} (cfr.~\secref{text}) e della matematica
annidata nella precedente formula.

\section{Correggere il posizionamento dei tag}

Posizionare i numeri delle equazioni in blocchi multilinea pu\`o essere
un problema piuttosto complesso; gli ambienti del pacchetto
\pkg{amsmath} fanno il possibile per evitare di sovrascrivere le
equazioni con il numero, eventualmente spostando il numero pi\`u in
alto o pi\`u in basso su una riga diversa; le difficolt\`a nel calcolo
preciso del profilo di un'equazione possono talvolta risultare in
spostamenti inopportuni dei numeri: si pu\`o allora usare il comando
\cn{raisetag}, fornito proprio per regolare manualmente la posizione
verticale del numero dell'equazione attiva, se \`e stato spostato
dalla sua posizione normale: per spostare ad esempio un particolare
numero in alto di sei punti, si scrive |\raisetag{6pt}|; questo tipo
di correzione \`e un lavoro di precisione come le interruzioni di riga
o di pagina, e andrebbe quindi lasciato fino a quando il documento
non sia ormai quasi completo, poich\'e si rischierebbe altrimenti di
dover disfare e rifare una correzione pi\`u volte, per tenersi al passo
con i cambiamenti del contenuto del documento.

\section{Spaziatura verticale e interruzioni di pagina in
\emph{display} su pi\`u linee}

Come nel \latex/, si pu\`o usare il comando \cn{\\}|[|\<dimensione>|]|
per ottenere spazi verticale aggiuntivi in tutti gli ambienti di
equazioni a blocchi del pacchetto \pkg{amsmath}. Quando si usa il
pacchetto \pkg{amsmath}, le \ii{interruzioni di pagina} tra le righe
delle equazioni sono normalmente impedite; la filosofia di ci\`o \`e
che le interruzioni di pagina in questo tipo di materiale dovrebbero
essere scelto dall'autore nei vari casi; per ottenere un'interruzione
di pagina in una particolare equazione in \emph{display}, si pu\`o
usare il comando \cn{displaybreak}; il luogo migliore dove posizionare
un \cn{displaybreak} \`e immediatamente prima del \cn{\\} dove si vuole
che abbia effetto; come il comando \latex/ \cn{pagebreak},
\cn{displaybreak} accetta un argomento opzionale, tra 0 e 4, per
indicare la opportunit\`a dell'interruzione. |\displaybreak[0]|
significa \qq{\`e possibile interrompere qui}, senza incoraggiare
l'interruzione; \cn{displaybreak} senza argomento equivale a
|\displaybreak[4]| e forza l'interruzione.

Se si preferisce permettere le interruzioni di pagina dove capita,
anche in mezzo a una equazione su pi\`u linee, si pu\`o usare
\cn{allowdisplaybreaks}\texttt{[1]} nel preambolo del documento. Un
argomento 1\ndash 4 pu\`o essere usato per un controllo pi\`u fine:
|[1]| permette le interruzioni, evitandole tuttavia il pi\`u
possibile; valori 2,3,4 indicano una permissivit\`a maggiore. Quando
le interruzioni sono abilitate con \cn{allowdisplaybreaks}, il comando
\cn{\\*} pu\`o essere usato, come al solito, per impedire
un'interruzione di pagina a una ben precisa riga.

\begin{bfseries}
Nota: alcuni ambienti di equazioni racchiudono il loro contenuto in
una scatola indistruttibile, con la conseguenza che n\'e
\cn{displaybreak}, n\'e \cn{allowdisplaybreaks} avranno effetto su di
loro; tra questi ambienti vi sono \env{split}, \env{aligned},
\env{gathered} e \env{alignedat}.
\end{bfseries}

\section{Interrompere i \emph{display}}

Il comando \cn{intertext} pu\`o essere usato per una breve inserzione
di una o due righe di testo\index{frammenti di testo in matematica} in
un \emph{display} su pi\`u righe (cfr. il comando \cn{text} in
\secref{text}): la sua caratteristica principale \`e il mantenimento
dell'allineamento, cosa che non avverrebbe se si terminasse il blocco
per ricominciarlo pi\`u avanti. \cn{intertext} pu\`o comparire solo
dopo un comando \cn{\\} o \cn{\\*}. Notare la posizione della parola
\qq{e} in questo esempio.
\begin{align}
A_1&=N_0(\lambda;\Omega')-\phi(\lambda;\Omega'),\\
A_2&=\phi(\lambda;\Omega')-\phi(\lambda;\Omega),\\
\intertext{e}
A_3&=\mathcal{N}(\lambda;\omega).
\end{align}
\begin{verbatim}
\begin{align}
A_1&=N_0(\lambda;\Omega')-\phi(\lambda;\Omega'),\\
A_2&=\phi(\lambda;\Omega')-\phi(\lambda;\Omega),\\
\intertext{e}
A_3&=\mathcal{N}(\lambda;\omega).
\end{align}
\end{verbatim}

\section{Numerazione delle equazioni}

\subsection{Gerarchia della numerazione}
Con il \latex/ se si vogliono numerare le equazioni secondo le
sezioni\mdash cio\`e, con numeri di equazione tipo (1.1), (1.2), \dots,
(2.1), (2.2), \dots, nelle sezioni 1, 2, e cos\`{\i} via\mdash
bisognava ridefinire \cn{theequation} come suggerito nel manuale del
\latex/ \cite[\S6.3, \S C.8.4]{lamport}:
\begin{verbatim}
\renewcommand{\theequation}{\thesection.\arabic{equation}}
\end{verbatim}

Ci\`o funziona piuttosto bene, tranne per il fatto che il contatore
delle equazioni non viene reimpostato a zero all'inizio di un nuovo
capitolo o sezione, a meno di non farlo manualmente con
\cn{setcounter}; per facilitare il procedimento, il pacchetto
\pkg{amsmath} fornisce il comando\index{equazioni, numeri
delle!gerarchia} \cn{numberwithin}. Per legare la numerazione delle
equazioni alla numerazione delle sezioni, con reimpostazione
automatica dei contatori, si pu\`o usare
\begin{verbatim}
\numberwithin{equation}{section}
\end{verbatim}
Come suggerito dal nome, il comando \cn{numberwithin} pu\`o essere
applicato a qualunque contatore, non solo al contatore
\texttt{equation}.

\subsection{Riferimenti incrociati ai numeri delle equazioni}

Per facilitare i riferimenti incrociati alle equazioni, \`e stato
creato il comando \cn{eqref}\index{equazioni, numeri delle!riferimenti
incrociati}, che fornisce automaticamente le parentesi attorno al
numero: cos\`{\i}, mentre \verb'\ref{abc}' produce 3.2,
\verb'\eqref{abc}' produce (3.2).

\subsection{Numerazione subordinata}

Il pacchetto \pkg{amsmath} fornisce anche un ambiente
\env{subequations}\index{equazioni, numeri delle!numerazione delle
subordinate} per facilitare la numerazione delle equazioni di un
gruppo con uno schema subordinato; ad esempio,
\begin{verbatim}
\begin{subequations}
...
\end{subequations}
\end{verbatim}
fa in modo che tutte le equazioni numerate in quella parte del
documento vengano numerate con (4.9a) (4.9b) (4.9c) \dots, se la
precedente equazione aveva numero (4.8). Un comando \cn{label} subito
dopo \verb/\begin{subequations}/ produrr\`a un \cn{ref} al numero
genitore 4.9, non a 4.9a; i contatori usati dall'ambiente
\env{subequations} sono \verb/parentequation/ e \verb/equation/;
\cn{addtocounter}, \cn{setcounter}, \cn{value} etc.\ possono essere
applicati come al solito ai nomi di questi contatori; per ottenere
qualcosa di diverso dalle lettere minuscole per i numeri delle
subordinate, si usa il metodo standard \latex/ per cambiare lo stile
di numerazione \cite[\S6.3, \S C.8.4]{lamport}. Ad esempio, ridefinendo
\cn{theequation} come segue fornisce numeri romani.
\begin{verbatim}
\begin{subequations}
\renewcommand{\theequation}{\theparentequation \roman{equation}}
...
\end{verbatim}

%%%%%%%%%%%%%%%%%%%%%%%%%%%%%%%%%%%%%%%%%%%%%%%%%%%%%%%%%%%%%%%%%%%%%%%%
%% FcDC
\chapter{Varie funzionalit\`{a} matematiche}

\section{Matrici}\label{ss:matrix}

Il pacchetto \pkg{amsmath} fornisce qualche ambiente per le
matrici\index{matrici} oltre al fondamentale  ambiente \env{array} del
\latex/. Gli ambienti \env{pmatrix}, \env{bmatrix}, \env{Bmatrix},
\env{vmatrix} e \env{Vmatrix} hanno come delimitatori rispettivamente
$(\,)$, $[\,]$, $\lbrace\,\rbrace$, $\lvert\,\rvert$, $\lVert\,\rVert$;
per coerenza con la nomenclatura viene fornito anche un ambiente
\env{matrix} senza delimitatori. Questo pu\`o sembrare superfluo,
vista la presenza dell'ambiente \env{array}, ma ci\`o non \`e vero;
infatti tutti gli ambienti per matrici utilizzano una spaziatura
orizzontale pi\`u economica di quella generosa messa a disposizione
dall'ambiente \env{array}. Inoltre, diversamente dall'ambiente
\env{array}, non si devono specificare i parametri relativi alle
colonne in nessuno degli ambienti per matrici; di \emph{default} si possono
avere fino a 10 colonne centrate.%
\footnote{%%%%%%%%%%%%%%%%%%%%%%%%%%%%%%%%%%%%%%%%%%%%%%%%%%%%%%%%%%%%%%
In dettaglio: Il massimo numero di colonne in una matrice \`e indicato
dal contatore |MaxMatrixCols| (valore normale=10), che si pu\`o
cambiare con i comandi \latex/ \cn{setcounter} o \cn{addcounter}.
}\space%%%%%%%%%%%%%%%%%%%%%%%%%%%%%%%%%%%%%%%%%%%%%%%%%%%%%%%%%%%%%%%%%
(Per ottenere l'allineamento a destra o a sinistra in una colonna, oppure
per qualsiasi altro formato speciale, \`e necessario utilizzare
\env{array})

Per ottenere una piccola matrice adatta al testo, \`e disponibile
l'ambiente \env{smallmatrix} (es:
\begin{math}
\bigl( \begin{smallmatrix}
  a&b\\ c&d
\end{smallmatrix} \bigr)
\end{math})
che \`e pi\`u adatta di qualsiasi altra matrice a entrare in una riga
di testo. Devono essere comunque forniti i delimitatori: non ci sono le
versioni |p|,|b|,|B|,|v|,|V| di \env{smallmatrix}. L'esempio qua sopra
\`e stato prodotto da
\begin{verbatim}
\bigl( \begin{smallmatrix}
  a&b\\ c&d
\end{smallmatrix} \bigr)
\end{verbatim}

\cn{hdotsfor}|{|\<numero>|}| produce una riga di punti in una matrice
\index{matrici!puntini}\index{puntini!nelle matrici}\index{punti|see{puntini}}%
larga tante colonne quanto il numero passato come argomento. Per
esempio,
\begin{center}
\begin{minipage}{.3\columnwidth}
\noindent$\begin{matrix} a&b&c&d\\
e&\hdotsfor{3} \end{matrix}$
\end{minipage}%
\qquad
\begin{minipage}{.45\columnwidth}
\begin{verbatim}
\begin{matrix} a&b&c&d\\
e&\hdotsfor{3} \end{matrix}
\end{verbatim}
\end{minipage}%
\end{center}

La spaziatura dei punti pu\`o essere variata con l'utilizzo di un
opzione tra parentesi quadre, ad esempio, |\hdotsfor[1.5]{3}|. Il
numero racchiuso dalle parentesi funge da moltiplicatore (il valore
normale \`e 1.0)

\begin{equation}\label{eq:D}
\begin{pmatrix} D_1t&-a_{12}t_2&\dots&-a_{1n}t_n\\
-a_{21}t_1&D_2t&\dots&-a_{2n}t_n\\
\hdotsfor[2]{4}\\
-a_{n1}t_1&-a_{n2}t_2&\dots&D_nt\end{pmatrix},
\end{equation}
\begin{verbatim}
\begin{pmatrix} D_1t&-a_{12}t_2&\dots&-a_{1n}t_n\\
-a_{21}t_1&D_2t&\dots&-a_{2n}t_n\\
\hdotsfor[2]{4}\\
-a_{n1}t_1&-a_{n2}t_2&\dots&D_nt\end{pmatrix}
\end{verbatim}


\section{Comandi per la spaziatura matematica}

Il pacchetto \pkg{amsmath} estende l'insieme dei comandi di spaziatura
\index{spaziatura orizzontale!in matematica} come mostrato sotto. Sia
la forma intera che quella contratta di questi comandi sono robuste e
possono essere utilizzate anche al di fuori dell'ambiente matematico.

\begin{ctab}{lll|lll}
Abbrev.&Forma intera& Esempio &Abbrev.&Forma intera&Esempio\\
\hline
\vstrut{2.5ex}
& no space& \spx{}& & no space & \spx{}\\
\cn{\,}& \cn{thinspace}& \spx{\,}&
  \cnbang& \cn{negthinspace}& \spx{\!}\\
\cn{\:}& \cn{medspace}& \spx{\:}&
  & \cn{negmedspace}& \spx{\negmedspace}\\
\cn{\;}& \cn{thickspace}& \spx{\;}&
  & \cn{negthickspace}& \spx{\negthickspace}\\
& \cn{quad}& \spx{\quad}\\
& \cn{qquad}& \spx{\qquad}
\end{ctab}
Per il maggior controllo possibile sulla spaziatura matematica \`e
possibile utilizzare \cn{mspace} e le `unit\`{a} matematiche';
un'unit\`{a} matematica o |mu| \`e uguale a 1/18esimo. Per avere un
\cn{quad} negativo si deve scrivere |\mspace{-18.0mu}|.


\section{Punti}
Non esiste un consenso generale per quanto riguarda il piazzamento dei
punti ellittici (a mezza riga o in fondo della riga) in vari contesti.
La cosa pu\`o quindi essere considerata una questione di gusto.
Utilizzando i comandi orientati verso la semantica
\begin{itemize}
\item \cn{dotsc} per \qq{punti con virgole}
\item \cn{dotsb} per \qq{punti con operazioni/relazioni binarie}
\item \cn{dotsm} per \qq{punti con moltiplicazioni}
\item \cn{dotsi} per \qq{punti con integrali}
\item \cn{dotso} per \qq{altri tipi} (nessuno dei precedenti)
\end{itemize}
invece di \cn{ldots} e \cn{cdots}, \`e possibile adattare a varie
convenzioni un documento ``al volo'', nel caso che (per esempio)
dovendo pubblicare tale documento, l'editore insista nel seguire le
tradizioni della casa. Il trattamento predefinito a seconda delle
situazioni segue le convenzioni dell'American Mathematical Society:
\begin{center}
\begin{tabular}{@{}l@{}l@{}}
\begin{minipage}[t]{.54\textwidth}
\begin{verbatim}
Abbiamo quindi la serie $A_1, A_2,
\dotsc$, la somma di regioni $A_1
+A_2 +\dotsb $, il prodotto
ortogonale $A_1 A_2 \dotsm $, e
l'integrale infinito
\[\int_{A_1}\int_{A_2}\dotsi.\]
\end{verbatim}
\end{minipage}
&
\begin{minipage}[t]{.45\textwidth}
\noindent
Abbiamo quindi la serie $A_1, A_2,
\dotsc$, la somma di regioni $A_1
+A_2 +\dotsb $, il prodotto
ortogonale $A_1 A_2 \dotsm $, e
l'integrale infinito
\[\int_{A_1}\int_{A_2}\dotsi.\]
\end{minipage}
\end{tabular}
\end{center}

\section{Trattini senza interruzioni}
Viene fornito il comando \cn{nobreakdash} per eliminare la
possibilit\`{a} che avvenga un'interruzione di linea dopo un trattino.
Ad esempio scrivendo `pagine 1\ndash 9'  come |pagine 1\nobreakdash 9|
non occorrer\`{a} mai un'interruzione di linea tra il trattino e il 9.
\`E possibile utilizzare \cn{nobreakdash} anche per prevenire
sillabazioni indesiderate in combinazioni tipo |$p$-adico|. Per un
utilizzo frequente \`e consigliato fare delle abbreviazioni; ad
esempio

\begin{verbatim}
\newcommand{\p}{$p$\nobreakdash}% per "\p-adico"
\newcommand{\Ndash}{\nobreakdash--}% per "pagine 1\Ndash 9"
%    Per "\n dimensionale" ("n-dimensionale"):
\newcommand{\n}[1]{$n$\nobreakdash-\hspace{0pt}}
\end{verbatim}
L'ultimo esempio mostra come impedire un'interruzione di linea dopo il
trattino ma permette la corretta sillabazione delle parole
seguenti.(Basta aggiungere un spazio di dimensione zero dopo il
trattino.)


\section{Accenti in matematica}

Nel \latex/ ordinario, il piazzamento del secondo accento negli accenti
matematici doppi \`e spesso mediocre; con il pacchetto \pkg{amsmath}
si migliora notevolmente il piazzamento del secondo accento:
$\hat{\hat{A}}$ (\cn{hat}|{\hat{A}}|).

Sono disponibili i  comandi \cn{dddot} e \cn{dddddot} per produrre
accenti tripli e quadrupli in aggiunta a \cn{dot} e \cn{ddot} presenti
nel \latex/.

Per ottenere un carattere di tilde o di cappello come apice, si deve
caricare il pacchetto \pkg{amsxtra} e utilizzare i comandi \cn{sphat}
o \cn{sptilde}, l'utilizzo \`e \verb'A\sphat' (notare l'assenza del
carattere \verb'^'). Per piazzare un simbolo arbitrario in posizione
di accento matematico o per ottenere accenti come pedici, consultare
il pacchetto \pkg{accents} di Javier Bezos.

\section{Radici}

Nel \latex/ ordinario il piazzamento degli indici delle radici a volte
non \`e buono:  $\sqrt[\beta]{k}$ (|\sqrt||[\beta]{k}|), nel pacchetto
 \pkg{amsmath} i comandi  \cn{leftroot} e \cn{uproot} permettono di aggiustare
la posizione della radice:

\begin{verbatim}
  \sqrt[\leftroot{-2}\uproot{2}\beta]{k}
\end{verbatim}
muove la beta in alto e verso destra:
$\sqrt[\leftroot{-2}\uproot{2}\beta]{k}$. L'argomento negativo di
\cn{leftroot} muove $\beta$ verso destra; le unit\`{a} sono piccole, e
quindi adatte per questo tipo di aggiustamenti.

\section{Formule in riquadro}

Il comando \cn{boxed} costruisce un riquadro attorno al suo argomento,
come \cn{fbox}, eccetto che i contenuti dei riquadri sono in modo
matematico:

\begin{equation}
\boxed{\eta \leq C(\delta(\eta) +\Lambda_M(0,\delta))}
\end{equation}
\begin{verbatim}
  \boxed{\eta \leq C(\delta(\eta) +\Lambda_M(0,\delta))}
\end{verbatim}

\section{Frecce in alto e in basso}
Il \latex/ di base fornisce i comandi \cn{overrightarrow} e
\cn{overleftarrow}; il paccheto \pkg{amsmath} fornisce altri comandi
per frecce in alto e in basso per estendere l'insieme di base:

\begin{tabbing}
\qquad\=\ncn{overleftrightarrow}\qquad\=\kill
\> \cn{overleftarrow}    \> \cn{underleftarrow} \+\\
   \cn{overrightarrow}    \> \cn{underrightarrow} \\
   \cn{overleftrightarrow}\> \cn{underleftrightarrow}
\end{tabbing}

\section{Frecce estendibili}
\cn{xleftarrow} e \cn{xrightarrow} producono frecce
\index{frecce!estendibili} che si estendono automaticamente per accomodare
grandezze inusuali di apici e pedici. Questi comandi prendono un argomento
facoltativo (il pedice) e un argomento obbligatorio (l'apice, possibilmente
anche vuoto):

\begin{equation}
A\xleftarrow{n+\mu-1}B \xrightarrow[T]{n\pm i-1}C
\end{equation}
\begin{verbatim}
  \xleftarrow{n+\mu-1}\quad \xrightarrow[T]{n\pm i-1}
\end{verbatim}

\section{Attaccare simboli ad altri simboli}

\latex/ fornisce  \cn{stackrel} per piazzare un apice
\index{esponenti e deponenti} sopra una relazione binaria.
Nel pacchetto \pkg{amsmath} ci sono comandi pi\`u generali,
\cn{overset} e \cn{underset} che possono essere utilizzati per
piazzare un simbolo sopra o sotto un altro simbolo, ogni qualvolta che
si trova una relazione binaria  o qualcos'altro.
L'input |\overset{*}{X}| piazza un $*$ della dimensione
di un apice
sopra la $X$: $\overset{*}{X}$; \cn{underset} \`e l'analogo
per aggiungere un simbolo in basso.
Controllare anche la descrizione di \cn{sideset} in \secref{sideset}.

\section{Frazioni e costrutti correlati}

\subsection{I comandi \cn{frac}, \cn{dfrac}, e \cn{tfrac}}

Il comando \cn{frac}, che fa parte dell'insieme dei comandi dei base
del \latex/,\index{frazioni} prende due argomenti\mdash numeratore
e denominatore\mdash e compone questi nella classica forma di una frazione.
Il pacchetto \pkg{amsmath} fornisce anche \cn{dfrac} e \cn{tfrac} come
convenienti abbreviazioni per |{\displaystyle\frac| |...| |}|
e\index{textstyle@\cn{textstyle}}\relax
\index{displaystyle@\cn{displaystyle}} |{\textstyle\frac| |...| |}|.

\begin{equation}
\frac{1}{k}\log_2 c(f)\quad\tfrac{1}{k}\log_2 c(f)\quad
\sqrt{\frac{1}{k}\log_2 c(f)}\quad\sqrt{\dfrac{1}{k}\log_2 c(f)}
\end{equation}
\begin{verbatim}
\begin{equation}
\frac{1}{k}\log_2 c(f)\;\tfrac{1}{k}\log_2 c(f)\;
\sqrt{\frac{1}{k}\log_2 c(f)}\;\sqrt{\dfrac{1}{k}\log_2 c(f)}
\end{equation}
\end{verbatim}

\subsection{I comandi \cn{binom}, \cn{dbinom}, e \cn{tbinom}}

Per espressioni binomiali\index{binomiali} tipo $\binom{n}{k}$
\pkg{amsmath} fornisce \cn{binom}, \cn{dbinom} e \cn{tbinom}:
\begin{equation}
2^k-\binom{k}{1}2^{k-1}+\binom{k}{2}2^{k-2}
\end{equation}
\begin{verbatim}
2^k-\binom{k}{1}2^{k-1}+\binom{k}{2}2^{k-2}
\end{verbatim}

\subsection{Il comando \cn{genfrac}}

Le capacit\`{a} di \cn{frac}, \cn{binom}, e delle loro varianti sono
sintetizzate dal comando generale \cn{genfrac}, che richiede sei
argomenti. Gli ultimi due corrispondono al numeratore e denominatore di
\cn{frac}, i primi due sono delimitatori opzionali (come visto in
\cn{binom}); il terzo riguarda lo spessore della linea (\cn{binom}
utilizza questo per impostare lo spessore della linea di frazione a 0
\mdash cio\`e invisibile) e il quarto argomento cambia lo stile
matematico: valori interi tra 0 e 3 selezionano rispettivamente
\cn{displaystyle}, \cn{textstyle}, \cn{scriptstyle} e
\cn{scriptscriptstyle}. Se il terzo argomento viene lasciato vuoto, lo
spessore della linea viene impostato per convenzione a `normale'.

\begin{center}\begin{minipage}{.85\columnwidth}
\raggedright \normalfont\ttfamily \exhyphenpenalty10000
\newcommand{\ma}[1]{%
  \string{{\normalfont\itshape#1}\string}\penalty9999 \ignorespaces}
\string\genfrac \ma{delim-sx} \ma{delim-dx} \ma{spessore}
\ma{stile} \ma{numeratore} \ma{denominatore}
\end{minipage}\end{center}
Per completezza viene mostrato come \cn{frac}, \cn{tfrac} e \cn{binom}
potrebbero essere definiti.
\begin{verbatim}
\newcommand{\frac}[2]{\genfrac{}{}{}{}{#1}{#2}}
\newcommand{\tfrac}[2]{\genfrac{}{}{}{1}{#1}{#2}}
\newcommand{\binom}[2]{\genfrac{(}{)}{0pt}{}{#1}{#2}}
\end{verbatim}
Se si utilizza ripetutamente \cn{genfrac} in un documento per una
particolare notazione, sarebbe di grande comodit\`a per lo scrittore
(e l'editore) definire un'abbreviazione significativa per questa
notazione, come \cn{frac} e \cn{binom} illustrate sopra. I comandi
primitivi generali per le frazioni \cs{over}, \cs{overwithdelims},
\cs{atop}, \cs{atopwithdelims}, \cs{above} e \cs{abovewithdelims}
producono messaggi di avvertimento se utilizzati in congiunzione con
\pkg{amsmath}, per ragioni discusse in \fn{technote.tex}.

\section{Frazioni continue}

La frazione continua\index{frazioni continue}
\begin{equation}
\cfrac{1}{\sqrt{2}+
 \cfrac{1}{\sqrt{2}+
  \cfrac{1}{\sqrt{2}+\cdots
}}}
\end{equation}
si ottiene digitando
{\samepage
\begin{verbatim}
\cfrac{1}{\sqrt{2}+
 \cfrac{1}{\sqrt{2}+
  \cfrac{1}{\sqrt{2}+\dotsb
}}}
\end{verbatim}
}% End of \samepage
Questo produce un  risultato visivamente migliore di quello ottenuto
con l'utilizzo prolungato di \cn{frac}. Il piazzamento a destra o a sinistra
di qualsiasi dei numeratori \`e ottenuto utilizzando \cn{cfrac}|[l]| o
\cn{cfrac}|[r]| invece di \cn{cfrac}.

\section{Opzioni smash}

Il comando \cn{smash} viene utilizzato per comporre una sottoformula
con effettiva larghezza e profondit\`{a} zero; questo a volte rimane
utile dovendo aggiustare la posizione della sottoformula rispetto ai
simboli adiacenti. Con il pacchetto \pkg{amsmath}, \cn{smash} ha
argomenti opzionali |t| e |b|, perch\'e occasionalmente \`e
vantaggioso essere capaci di \qq{appiattire} solo l'altezza o la
profondit\`a, conservando l'altra. Ad esempio, quando simboli di
radicali sono posizionati o dimensionati in modo diverso a causa delle
differenze di altezza e larghezza dei loro contenuti, \cn{smash}
pu\`o essere applicato per rendere il tutto pi\`u consistente.
Confrontare $\sqrt{x}+\sqrt{y}+\sqrt{z}$ con
$\sqrt{x}+\sqrt{\smash[b]{y}}+\sqrt{z}$, dove l'ultimo \`e stato
prodotto con \verb"$\sqrt{x}" \verb"+"
\verb"\sqrt{"\5\verb"\smash[b]{y}}" \verb"+" \verb"\sqrt{z}$".

\section{Delimitatori}

\subsection{Dimensione dei delimitatori}\label{bigdel}

Il dimensionamento automatico dei delimitatori fatto da \cn{left} e
\cn{right} ha due limitazioni: innanzi tutto, viene applicato
meccanicamente per produrre delimitatori abbastanza grandi da
ricoprire il pi\`u grande oggetto contenuto in essi, e inoltre,
l'intervallo delle dimensioni non \`e neanche approssimativamente
continuo, ma ha dei salti abbastanza grandi. Questo significa che un
frammemto matematico infinitesimamente troppo grande per una data
grandezza del delimitatore prender\`{a} la misura successiva, un salto
di 3pt o simile in un testo a grandezza normale. Ci sono due o tre
situazioni dove la grandezza del delimitatore viene comunemente
aggiustata, utilizzando un insieme di comandi che contengono `big' nei
loro nomi.

\begin{ctab}{l|llllll}
Dim. del&
  dim. del& \ncn{left}& \ncn{bigl}& \ncn{Bigl}& \ncn{biggl}& \ncn{Biggl}\\
delimitatore&
  testo& \ncn{right}& \ncn{bigr}& \ncn{Bigr}& \ncn{biggr}& \ncn{Biggr}\\
\hline
Risultato\vstrut{5ex}&
  $\displaystyle(b)(\frac{c}{d})$&
  $\displaystyle\left(b\right)\left(\frac{c}{d}\right)$&
  $\displaystyle\bigl(b\bigr)\bigl(\frac{c}{d}\bigr)$&
  $\displaystyle\Bigl(b\Bigr)\Bigl(\frac{c}{d}\Bigr)$&
  $\displaystyle\biggl(b\biggr)\biggl(\frac{c}{d}\biggr)$&
  $\displaystyle\Biggl(b\Biggr)\Biggl(\frac{c}{d}\Biggr)$
\end{ctab}
Il primo tipo di situazione \`e un operatore cumulativo con limiti
sopra e sotto. Con \cn{left} e \cn{right} i delimitatori di solito
diventano pi\`u larghi del necessario, e utilizzando invece
le dimensioni |Big| o |bigg| si ottengono risultati migliori.
\begin{equation*}
\left[\sum_i a_i\left\lvert\sum_j x_{ij}\right\rvert^p\right]^{1/p}
\quad\text{contro}\quad
\biggl[\sum_i a_i\Bigl\lvert\sum_j x_{ij}\Bigr\rvert^p\biggr]^{1/p}
\end{equation*}
\begin{verbatim}
\biggl[\sum_i a_i\Bigl\lvert\sum_j x_{ij}\Bigr\rvert^p\biggr]^{1/p}
\end{verbatim}
Il secondo tipo di situazione \`e un ammasso di coppie  di delimitatori
dove \cn{left} e \cn{right} rendono le loro grandezze uguali
(dato che questo risulta adeguato per racchiudere tutto il materiale)
ma l'effetto desiderato \`e quello di avere alcuni delimitatori
con grandezza maggiore, per rendere l'annidamento pi\`u semplice da
vedere.
\begin{equation*}
\left((a_1 b_1) - (a_2 b_2)\right)
\left((a_2 b_1) + (a_1 b_2)\right)
\quad\text{contro}\quad
\bigl((a_1 b_1) - (a_2 b_2)\bigr)
\bigl((a_2 b_1) + (a_1 b_2)\bigr)
\end{equation*}
\begin{verbatim}
\left((a_1 b_1) - (a_2 b_2)\right)
\left((a_2 b_1) + (a_1 b_2)\right)
\quad\text{versus}\quad
\bigl((a_1 b_1) - (a_2 b_2)\bigr)
\bigl((a_2 b_1) + (a_1 b_2)\bigr)
\end{verbatim}
Il terzo tipo di situazione \`e un oggetto di dimensione leggermente
elevata nel testo libero, %running rext??
come $\left\lvert\frac{b'}{d'}\right\rvert$, dove i delimitatori
prodotti da \cn{left} e \cn{right} causano un'eccessiva altezza
della linea. In questo caso \ncn{bigl} e \ncn{bigr}\index{big@\cn{big},
\cn{Big}, \cn{bigg}, \dots\ delimiters} possono essere utilizzati
per produrre delimitatori che sono leggermente pi\`u grandi della
dimensione di base, ma che comunque rientrano all'interno della normale
spaziatura della linea: $\bigl\lvert\frac{b'}{d'}\bigr\rvert$.
Nel \latex/ ordinario i delimitatori \ncn{big}, \ncn{bigg}, \ncn{Big},
e \ncn{Bigg} non sono scalati in modo opportuno per tutto il ``range''
delle dimensioni dei \emph{font} \latex/, con il pacchetto \pkg{amsmath}
invece lo sono.

\subsection{Notazioni per la barra verticale}

Il pacchetto \pkg{amsmath} fornisce i comandi \cn{vert}, \cn{rvert},
\cn{lVert}, \cn{rVert} (confrontare \cn{langle} e \cn{rangle}) per
indirizzare il problema del sovraccarico per il carattere di barra
verticale \qc{\|}. Questo carattere viene utilizato nei documenti
\latex/ per una grande variet\`{a} di oggetti matematici: la relazione
`divide' in un' espressione della teoria dei numeri tipo $p\vert q$,
oppure l'operazione di  valore assoluto $\lvert z\rvert$, oppure la
condizione `tale che' nella notazione insiemistica, oppure la
notazione `valutato in' $f_\zeta(t)\bigr\rvert_{t=0}$. La
molteplicit\`{a} degli utilizzi non \`e essa stessa un male, ci\`o
che non va bene comunque \`e il fatto che non tutti questi vari
oggetti ottengono lo stesso trattamento tipografico e che le complesse
capcit\`a discriminatorie di un lettore colto non possono essere
replicate in un computer che deve elaborare documenti matematici. Si
raccomanda quindi che ci sia una corrispondenza uno-a-uno in ogni
documento tra il carattere di barra verticale \qc{\|} e una scelta
notazione matematica, analogamente per il comando di doppia barra
\cn{\|}. Questo immediatamente esclude l'utilizzo di \qc{|} e
\ncn{\|}\index{"|@\cn{"\"|}} come delimitatori, dato che i delimitatori
destri e sinistri %
hanno usi distinti, non correlati allo stesso modo con simboli
adiacenti
%delimiters are distinct usages that do not relate in the same way to
%adjacent symbols;
si raccomanda la pratica di definire nel preambolo del documento
comandi adatti a ogni utilizzo di coppie di delimitatori con simboli
di barre verticali:

\begin{verbatim}
\providecommand{\abs}[1]{\lvert#1\rvert}
\providecommand{\norm}[1]{\lVert#1\rVert}
\end{verbatim}
al che il documento dovrebbe contenere |\abs{z}| per produrre
 $\lvert z\rvert$ e |\norm{v}| per produrre $\lVert v\rVert$.
%%%%% fine FCdC

%%%%% OdB
\chapter{Nomi per gli operatori}

\section{Come definire nuovi nomi di operatori}\label{s:opname}

Le funzioni matematiche\index{nomi di operatori}\index{nomi di
funzioni|see{nomi di operatori}} come $\log$, $\sin$, e $\lim$ sono
per tradizione stampate in tondo per renderne pi\`u immediata la
visibilit\`a rispetto alle variabili matematiche di un carattere, che
sono stampate in stile matematico corsivo. Le pi\`u comuni hanno nomi
predefiniti, \cn{log}, \cn{sin}, \cn{lim}, e cos\`\i{} via, ma se ne
introducono continuamente di nuovi nelle pubblicazioni relative alla
matematica, pertanto il pacchetto \pkg{amsmath} fornisce un metodo
generale per definire nuovi `nomi di operatori'. Per definire una
funzione matematica \ncn{xxx} che si presenti come \cn{sin}, si
scriver\`a
\begin{verbatim}
\DeclareMathOperator{\xxx}{xxx}
\end{verbatim}
Come conseguenza, l'utilizzo di \ncn{xxx} produrr\`a {\upshape xxx}
nel corrispondente \emph{font} e automaticamente aggiunger\`a l'adeguata
spaziatura\index{spaziatura orizzontale!attorno ai nomi di operatori}
su entrambi i lati quando necessario, in maniera tale da ottenere
$A\xxx B$ invece di $A\mathrm{xxx}B$. Nel secondo argomento di
\cn{DeclareMathOperator} (il testo con il nome), \`e prevalente una
modalit\`a pseudo-testuale: il carattere di sillabazione \qc{\-}
verr\`a stampato come un trattino di sillabazione piuttosto che come
un segno meno e un asterisco \qc{\*} risulter\`a stampato come un
asterisco in alto piuttosto che come un asterisco centrato di tipo
matematico (confrontare \textit{a}-\textit{b}*\textit{c} e $a-b*c$.);
d'altra parte il testo contenente il nome \`e stampato in modalit\`a
matematica, ad es. in modo tale da poter ivi usare pedici e apici.

Se il nuovo operatore dovesse esser dotato di pedici e apici
posizionati alla maniera dei
`limiti', al di sopra e al di sotto come per $\lim$, $\sup$, o $\max$,
si user\`a
la forma \qc{\*} del comando \cn{DeclareMathOperator}:
\begin{verbatim}
\DeclareMathOperator*{\Lim}{Lim}
\end{verbatim}
Fare inoltre riferimento alla trattazione del posizionamento dell'indice
nel paragrafo~\ref{subplace}.

I seguenti nomi di operatori sono predefiniti:
\begin{ctab}{rlrlrlrl}
\cn{arccos}& $\arccos$ &\cn{deg}& $\deg$ &      \cn{lg}& $\lg$ &        \cn{projlim}& $\projlim$\\
\cn{arcsin}& $\arcsin$ &\cn{det}& $\det$ &      \cn{lim}& $\lim$ &      \cn{sec}& $\sec$\\
\cn{arctan}& $\arctan$ &\cn{dim}& $\dim$ &      \cn{liminf}& $\liminf$ &\cn{sin}& $\sin$\\
\cn{arg}& $\arg$ &      \cn{exp}& $\exp$ &      \cn{limsup}& $\limsup$ &\cn{sinh}& $\sinh$\\
\cn{cos}& $\cos$ &      \cn{gcd}& $\gcd$ &      \cn{ln}& $\ln$ &        \cn{sup}& $\sup$\\
\cn{cosh}& $\cosh$ &    \cn{hom}& $\hom$ &      \cn{log}& $\log$ &      \cn{tan}& $\tan$\\
\cn{cot}& $\cot$ &      \cn{inf}& $\inf$ &      \cn{max}& $\max$ &      \cn{tanh}& $\tanh$\\
\cn{coth}& $\coth$ &    \cn{injlim}& $\injlim$ &\cn{min}& $\min$\\
\cn{csc}& $\csc$ &      \cn{ker}& $\ker$ &      \cn{Pr}& $\Pr$
\end{ctab}
\begin{ctab}{rlrl}
\cn{varlimsup}&  $\displaystyle\varlimsup$&
  \cn{varinjlim}&  $\displaystyle\varinjlim$\\
\cn{varliminf}&  $\displaystyle\varliminf$&
  \cn{varprojlim}& $\displaystyle\varprojlim$
\end{ctab}

\`E inoltre disponibile un comando \cn{operatorname}, in modo tale che
l'uso di
\begin{verbatim}
\operatorname{abc}
\end{verbatim}
in una formula matematica equivalga all'uso di \ncn{abc} definito da
\cn{DeclareMathOperator}; questo pu\`o in certi casi essere utile per
realizzare notazioni pi\`u complesse o per altri scopi. (Usare la variante
\cn{operatorname*} per ottenere i limiti.)

\section{\cn{mod} e i suoi affini}

I comandi \cn{mod}, \cn{bmod}, \cn{pmod}, \cn{pod} sono forniti per
affrontare le particolari convenzioni di spaziatura della notazione
\qq{mod}. In  \latex/ sono disponibili \cn{bmod} e
\cn{pmod}, ma con il pacchetto \pkg{amsmath}
la spaziatura di \cn{pmod} sar\`a regolata a un valore inferiore se
viene usato in una formula in modalit\`a \emph{non-display}.
 \cn{mod} e \cn{pod} sono varianti di
\cn{pmod} preferite da alcuni autori; \cn{mod} omette le parentesi,
mentre \cn{pod} omette il \qq{mod} e mantiene le parentesi.
\begin{equation}
\gcd(n,m\bmod n);\quad x\equiv y\pmod b;
\quad x\equiv y\mod c;\quad x\equiv y\pod d
\end{equation}
\begin{verbatim}
\gcd(n,m\bmod n);\quad x\equiv y\pmod b;
\quad x\equiv y\mod c;\quad x\equiv y\pod d
\end{verbatim}
%%%%% Fine OdB

%%%%%%%%%%%%%%%%%%%%%%%%%%%%%%%%%%%%%%%%%%%%%%%%%%%%%%%%%%%%%%%%%%%%%%%%
% by GD

\chapter{Il comando \cn{text}}\label{text}

Il principale utilizzo del comando \cn{text} consiste nello scrivere
parole o frasi\index{testo!frammenti di testo in matematica} in un
\emph{display}. Il suo comportamento \`e molto simile al comando
\latex/  \cn{mbox}, ma presenta un paio di vantaggi. Se si desidera
inserire una parola o una frase in un deponente \`e leggermente pi\`u
semplice digitare |..._{\text{parola o frase}}| piuttosto che il
comando equivalente \cn{mbox}: |..._{\mbox{\scriptsize| |parola| |o|
|frase}}|. L'altro vantaggio \`e nel suo nome, pi\`u descrittivo.
\begin{equation}
f_{[x_{i-1},x_i]} \text{ \`e monotona,}
\quad i = 1,\dots,c+1
\end{equation}
\begin{verbatim}
f_{[x_{i-1},x_i]} \text{ \`e monotona,}
\quad i = 1,\dots,c+1
\end{verbatim}

% by GD - end

% by LF
\chapter{Integrali e sommatorie}

\section{Deponenti ed esponenti su pi\`u righe}

Il comando \cn{substack} pu\`o essere usato per produrre un deponente o un
esponente su pi\`u righe:\index{deponenti ed esponenti!su pi\`u righe}\relax
\index{esponenti|see{deponenti ed esponenti}} per esempio
\begin{ctab}{ll}
\begin{minipage}[t]{.6\columnwidth}
\begin{verbatim}
\sum_{\substack{
         0\le i\le m\\
         0<j<n}}
  P(i,j)
\end{verbatim}
\end{minipage}
&
$\displaystyle
\sum_{\substack{0\le i\le m\\ 0<j<n}} P(i,j)$
\end{ctab}
Una forma un po' pi\`u generalizzata \`e l'ambiente \env{subarray} che
consente di specificare che ogni riga deve essere allineata a sinistra invece che
centrata, come in questo caso:
\begin{ctab}{ll}
\begin{minipage}[t]{.6\columnwidth}
\begin{verbatim}
\sum_{\begin{subarray}{l}
        i\in\Lambda\\ 0<j<n
      \end{subarray}}
 P(i,j)
\end{verbatim}
\end{minipage}
&
$\displaystyle
  \sum_{\begin{subarray}{l}
        i\in \Lambda\\ 0<j<n
      \end{subarray}}
 P(i,j)$
\end{ctab}

\section{Il comando \cn{sideset}}\label{sideset}

C'\`e anche un comando chiamato \cn{sideset}, per uno scopo abbastanza
particolare: porre dei simboli agli angoli di deponente ed
esponente\index{deponenti ed esponenti!su sommatorie} di un simbolo
operatorio di grandi dimensioni come $\sum$ o $\prod$. \emph{Nota:
questo comando non \`e pensato per essere applicato ad altro che a
simboli tipo sommatoria.} L'esempio principale \`e il caso in cui si
voglia porre un simbolo di primo su un simbolo di sommatoria. Se non
ci sono estremi sopra o sotto la sommatoria, si pu\`o semplicemente
usare \cn{nolimits}: ecco come appare
%%%%%%%%%%%%%%%%%%%%%%%%%%%%%%%%%%%%%%%%%%%%%%%%%%%%%%%%%%%%%%%%%%%%%%%%
|\sum\nolimits' E_n| in modo \emph{display}:
\begin{equation}
\sum\nolimits' E_n
\end{equation}
Se tuttavia si desidera non solo il segno di primo ma anche qualcosa sopra o
sotto il simbolo di sommatoria, non \`e cos\`\i{} facile\mdash invero, senza
\cn{sideset}, sarebbe proprio difficile. Con \cn{sideset}, si
pu\`o scrivere
\begin{ctab}{ll}
\begin{minipage}[t]{.6\columnwidth}
\begin{verbatim}
\sideset{}{'}
  \sum_{n<k,\;\text{$n$ odd}} nE_n
\end{verbatim}
\end{minipage}
&$\displaystyle
\sideset{}{'}\sum_{n<k,\;\text{$n$ odd}} nE_n
$
\end{ctab}
La coppia di parentesi graffe vuote si spiega con il fatto che
\cn{sideset} ha la possibilit\`a di porre uno o pi\`u simboli aggiuntivi a
ogni angolo di un simbolo operatorio di grandi dimensioni; per porre un asterisco in ciascun angolo di un
simbolo di prodotto, si potrebbe scrivere
\begin{ctab}{ll}
\begin{minipage}[t]{.6\columnwidth}
\begin{verbatim}
\sideset{_*^*}{_*^*}\prod
\end{verbatim}
\end{minipage}
&$\displaystyle
\sideset{_*^*}{_*^*}\prod
$
\end{ctab}

\section{Posizionamento di deponenti ed estremi}\label{subplace}

Il tipo di posizionamento predefinito per i deponenti dipende dal
simbolo base considerato. Per i simboli tipo sommatoria \`e usato il
posizionamento `displaylimits': quando un simbolo tipo sommatoria appare
in una formula in \emph{display}, deponente ed esponente sono posti nella posizione
`limits' sopra e sotto, ma in una formula nel testo sono invece posti
a lato, per evitare l'antiestetico e sprecato allargamento della distanza dalle
righe di testo adiacenti.
L'impostazione predefinita per i simboli tipo integrale \`e avere deponenti
ed esponenti sempre a lato, anche nelle formule in \emph{display}.
(Si veda la discussione su \opt{intlimits} e opzioni correlate nella
Sec.~\ref{options}.)

I nomi di operatore, come $\sin$ o $\lim$, possono avere il posizionamento
`displaylimits' o quello `limits' a seconda di come sono stati definiti. Gli operatori
pi\`u comuni sono definiti in base all'uso consueto in matematica.

I comandi \cn{limits} e \cn{nolimits} possono essere usati per modificare
il normale comportamento di un simbolo base:
\begin{equation*}
\sum\nolimits_X,\qquad \iint\limits_{A},
\qquad\varliminf\nolimits_{n\to \infty}
\end{equation*}
Per definire un comando i cui deponenti seguono lo
stesso comportamento `displaylimits' di \cn{sum}, si pu\`o aggiungere
\cn{displaylimits} in coda alla definizione. Quando ci sono
pi\`u istanze consecutive di \cn{limits}, \cn{nolimits}, o \cn{displaylimits},
l'ultima ha la priorit\`a.

\section{Simboli di integrale multiplo}

\cn{iint}, \cn{iiint}, e \cn{iiiint} producono pi\`u simboli di integrale
\index{integrali!multipli} con la spaziatura tra di loro opportunamente
corretta, sia in stile testo che \emph{display}. \cn{idotsint} estende
la stessa idea producendo due segni di integrale separati da tre punti.
\begin{gather}
\iint\limits_A f(x,y)\,dx\,dy\qquad\iiint\limits_A
f(x,y,z)\,dx\,dy\,dz\\
\iiiint\limits_A
f(w,x,y,z)\,dw\,dx\,dy\,dz\qquad\idotsint\limits_A f(x_1,\dots,x_k)
\end{gather}

% by LF - end
%%%%%%%%%%%%%%%%%%%%%%%%%%%%%%%%%%%%%%%%%%%%%%%%%%%%%%%%%%%%%%%%%%%%%%%%
% by GD

\chapter{Diagrammi commutativi}\label{s:commdiag}

Vari comandi, come quelli in \amstex/, per disegnare i diagrammi
commutativi sono disponibili separatamente nel pacchetto \pkg{amscd}.
Per i diagrammi commutativi di una certa complessit\`a, gli autori
dovranno necessariamente considerare pacchetti pi\`u estesi come
\pkg{kuvio} o \xypic/, ma per diagrammi semplici privi di
frecce\index{frecce!nei diagrammi commutativi} diagonali, i
comandi dell'\pkg{amscd} potrebbero rivelarsi pi\`u convenienti.
Di seguito vi \`e un esempio.
\begin{equation*}
\begin{CD}
S^{{\mathcal{W}}_\Lambda}\otimes T   @>j>>   T\\
@VVV                                    @VV{\End P}V\\
(S\otimes T)/I                  @=      (Z\otimes T)/J
\end{CD}
\end{equation*}
\begin{verbatim}
\begin{CD}
S^{{\mathcal{W}}_\Lambda}\otimes T   @>j>>   T\\
@VVV                                    @VV{\End P}V\\
(S\otimes T)/I                  @=      (Z\otimes T)/J
\end{CD}
\end{verbatim}
Nell'ambiente \env{CD}, i comandi |@>>>|, |@<<<|, |@VVV| e |@AAA|
disegnano, rispettivamente, le frecce a destra, a sinistra, verso il
basso e verso l'alto.
Per quanto riguarda le frecce orizzontali, il contenuto tra il primo
e il secondo simbolo |>| oppure |<| sar\`a inserito a esponente sulla
freccia, e il contenuto tra il secondo e il terzo simbolo sar\`a inserito
a deponente sotto la freccia.
Analogamente per le frecce verticali, il contenuto tra il primo e il
secondo oppure tra il secondo e il terzo dei simboli |A| o |V| sar\`a
inserito a \qq{margine} sinistro o destro della freccia.
I comandi |@=| e \verb'@|' tracciano rispettivamente una doppia linea
orizzontale e una verticale.
Il comando |@.| equivale a una \qq{freccia nulla} e pu\`o essere usato
al posto di una freccia visibile per espandere, dove sia necessario, una
matrice.

% by GD - end

%%%%%%%%%%%%%%%%%%%%%%%%%%%%%%%%%%%%%%%%%%%%%%%%%%%%%%%%%%%%%%%%%%%%%%%%
\chapter{Usare \emph{font} matematici}

\section{Introduzione}

Per informazioni pi\`u complete riguardo l'uso dei \emph{font} in \latex/,
consultate la guida ai \emph{font} del \latex/ (\fn{fntguide.tex}) o
il libro \booktitle{The \latex/ Companion} \cite{tlc}.  L'insieme di base
dei comandi per usare \emph{font} matematici\index{\emph{font} matematici}\relax
\index{simboli matematici|see{\emph{font} matematici}} nel \latex/ \`e
costituito da \cn{mathbf}, \cn{mathrm}, \cn{mathcal}, \cn{mathsf},
\cn{mathtt} e \cn{mathit}. Comandi aggiuntivi per \emph{font} matematici
come \cn{mathbb} per il \emph{blackboard-bold}, \cn{mathfrak} per il Fraktur
e \cn{mathscr} per l'Euler script sono disponibili attraverso i
pacchetti \pkg{amsfonts} e \pkg{euscript} (distribuiti separatamente).

\section{Uso consigliato per i comandi dei \emph{font} matematici}

Se ci si trova a usare frequentemente comandi per \emph{font} matematici nei
propri documenti, si potrebbero voler usare nomi pi\`u brevi, come
\ncn{mb} al posto di \cn{mathbf}.  Ovviamente non c'\`e nulla che
impedisca di farsi da soli tali abbreviazioni, usando i comandi
\cn{newcommand} appropriati.  D'altro canto, per il \latex/, offrire
comandi pi\`u brevi sarebbe addirittura un disservizio per gli autori,
poich\'e renderebbe meno ovvia un'alternativa molto migliore:
definire nomi di comandi personalizzati che si riferiscano ai nomi degli
oggetti matematici che a loro competono, piuttosto che ai nomi dei
\emph{font} che sono usati per distinguere tali oggetti.  Per esempio, se
si usa il grassetto per indicare i vettori, alla lunga sarebbe meglio definire
un comando `vector' al posto di un `math-bold':
\begin{verbatim}
  \newcommand{\vect}[1]{\mathbf{#1}}
\end{verbatim}
si pu\`o scrivere |\vect{a} + \vect{b}| per avere $\vect{a} +
\vect{b}$.
Se, mesi dopo aver cominciato il lavoro, si decide di usare il
grassetto per qualche altro scopo e di indicare i vettori con una
freccina sopra, si pu\`o fare il tutto semplicemente cambiando la
definizione di \ncn{vect}; altrimenti si sarebbero dovute rimpiazzare
tutte le occorrenze di \cn{mathbf} nel documento, eventualmente
persino controllandole una a una per vedere se si riferivano
effettivamente a un vettore.

Pu\`o essere utile anche assegnare nomi di comandi distinti per
differenti lettere dell'alfabeto di un particolare \emph{font}:
\begin{verbatim}
\DeclareSymbolFont{AMSb}{U}{msb}{m}{n}% oppure si usi il pacchetto amsfonts
\DeclareMathSymbol{\C}{\mathalpha}{AMSb}{"43}
\DeclareMathSymbol{\R}{\mathalpha}{AMSb}{"52}
\end{verbatim}
Queste righe definirebbero i comandi \cn{C} e \cn{R} in modo che
producano le lettere \emph{blackboard-bold} del \emph{font} di simboli matematici
`AMSb'.  Se nel proprio documento si fa spesso riferimento ai numeri
reali o a quelli complessi, si pu\`o preferire questo metodo a
quello di definire, per esempio, un comando \ncn{field} e scrivere
|\field{C}| e |\field{R}|, ma per ottenere la massima flessibilit\`a
e il massimo controllo sarebbe opportuno definire tale comando e poi
definire \ncn{C} e \ncn{R} in funzione di quello:\index{mathbb@\cn{mathbb}}
\begin{verbatim}
\usepackage{amsfonts}% per disporre dell'alfabeto \mathbb
\newcommand{\field}[1]{\mathbb{#1}}
\newcommand{\C}{\field{C}}
\newcommand{\R}{\field{R}}
\end{verbatim}

\section{Simboli matematici in grassetto}

Il comando \cn{mathbf} \`e usato comunemente per ottenere lettere
latine grassette in modo matematico, ma per la maggior parte degli
altri tipi di simboli matematici non ha effetto, o i suoi effetti
dipendono in maniera non prevedibile dalla serie di \emph{font} matematici in
uso.  Per esempio, scrivendo
\begin{verbatim}
\Delta \mathbf{\Delta}\mathbf{+}\delta \mathbf{\delta}
\end{verbatim}
si ottiene $\Delta \mathbf{\Delta}\mathbf{+}\delta \mathbf{\delta}$;
il comando \cn{mathbf} non ha cambiato il segno pi\`u e il delta
minuscolo.

Per questo motivo il pacchetto \pkg{amsmath} fornisce altri due
comandi, \cn{boldsymbol} e \cn{pmb}, che possono essere usati con gli
altri tipi di simboli matematici.  \cn{boldsymbol} pu\`o essere usato
per i simboli matematici sui quali non ha effetto il comando \cn{mathbf}
se (e solo se) il \emph{font} matematico in uso in quel momento dispone
di una versione in grassetto di quel simbolo.  \cn{pmb} pu\`o essere usato
come ultima risorsa per qualsiasi simbolo matematico che non abbia una
vera versione in grassetto all'interno del \emph{font} matematico; \qq{pmb}
\`e l'abbreviazione di \qq{poor man's bold} (grassetto dei poveri) e
funziona stampando copie pi\`u copie dello stesso simbolo leggermente spostate
le une dalle altre.  Il risultato \`e di qualit\`a
inferiore, specialmente per quei simboli che contengono linee sottili.
Quando si usa la famiglia standard di \emph{font} matematici del \latex/ (il
Computer Modern), gli unici simboli che potrebbero richiedere il
\cn{pmb} sono quelli dei simboli operatori di grandi dimensioni, come \cn{sum}, i simboli
di delimitatori estesi, o i simboli addizionali forniti dal pacchetto
\pkg{amssymb}~\cite{amsfonts}.

La formula seguente mostra alcuni dei possibili risultati:
\begin{verbatim}
A_\infty + \pi A_0
\sim \mathbf{A}_{\boldsymbol{\infty}} \boldsymbol{+}
  \boldsymbol{\pi} \mathbf{A}_{\boldsymbol{0}}
\sim\pmb{A}_{\pmb{\infty}} \pmb{+}\pmb{\pi} \pmb{A}_{\pmb{0}}
\end{verbatim}
\begin{equation*}
A_\infty + \pi A_0
\sim \mathbf{A}_{\boldsymbol{\infty}} \boldsymbol{+}
  \boldsymbol{\pi} \mathbf{A}_{\boldsymbol{0}}
\sim\pmb{A}_{\pmb{\infty}} \pmb{+}\pmb{\pi} \pmb{A}_{\pmb{0}}
\end{equation*}
Se si vuole usare solo il comando \cn{boldsymbol} senza caricare tutto
il pacchetto \pkg{amsmath}, si pu\`o usare il pacchetto \pkg{bm} (questo
\`e un pacchetto standard del \latex/, non fa parte di quelli AMS; se
si ha una versione del \latex/ del 1997 o posteriore, probabilmente lo
si ha gi\`a).

\section{Lettere greche corsive}

Per ottenere una versione corsiva delle lettere greche maiuscole, si
possono usare i seguenti comandi:
\begin{ctab}{rlrl}
\cn{varGamma}& $\varGamma$& \cn{varSigma}& $\varSigma$\\
\cn{varDelta}& $\varDelta$& \cn{varUpsilon}& $\varUpsilon$\\
\cn{varTheta}& $\varTheta$& \cn{varPhi}& $\varPhi$\\
\cn{varLambda}& $\varLambda$& \cn{varPsi}& $\varPsi$\\
\cn{varXi}& $\varXi$& \cn{varOmega}& $\varOmega$\\
\cn{varPi}& $\varPi$
\end{ctab}

%%%%%%%%%%%%%%%%%%%%%%%%%%%%%%%%%%%%%%%%%%%%%%%%%%%%%%%%%%%%%%%%%%%%%%%%
% by GA

\chapter{Messaggi di errore e problemi di output}

\section{Osservazioni di carattere generale}

Questo \`e un supplemento al capitolo~8 del manuale del \latex/
\cite{lamport} (prima edizione: capitolo~6). Per comodit\`a del
lettore, l'insieme dei messaggi d'errore discussi qui si sovrappone
parzialmente con quello di \cite{lamport}, ma sia chiaro che qui
non si intende dare una copertura esaustiva. I messaggi
d'errore sono disposti in ordine alfabetico, senza badare a testo
irrilevante quale |! LaTeX Error:| all'inizio del messaggio, e
caratteri non alfabetici quali \qc{\\}. Dove vengono forniti esempi,
vengono anche mostrati i messaggi d'aiuto che appaiono sullo schermo
quando si risponde a un messagio d'errore digitando |h|.

C'\`e anche una sezione che discute qualche errore di output, per esempio
in casi in cui il documento stampato ha qualcosa che non va, ma \latex/
non ha rilevato alcun errore.

\section{Messaggi di errore}

\begin{error}{\begin{split} won't work here.}
\errexa
\begin{verbatim}
! Package amsmath Error: \begin{split} won't work here.
 ...

l.8 \begin{split}

? h
\Did you forget a preceding \begin{equation}?
If not, perhaps the `aligned' environment is what you want.
?
\end{verbatim}
\errexpl L'ambiente \env{split} non costruisce un'equazione in
\emph{display} a s\'e stante; deve essere usato all'interno di qualche
altro ambiente quali \env{equation} o \env{gather}.

\end{error}

\begin{error}{Extra & on this line}
\errexa
\begin{verbatim}
! Package amsmath Error: Extra & on this line.

See the amsmath package documentation for explanation.
Type  H <return>  for immediate help.
 ...

l.9 \end{alignat}

? h
\An extra & here is so disastrous that you should probably exit
 and fix things up.
?
\end{verbatim}
\errexpl
In una struttura \env{alignat} il numero di punti di allineamento su una linea
\`e determinato dall'argomento numerico fornito dopo |\begin{alignat}|.
Se in una linea si usano pi\`u punti di allineamento rispetto a quelli consentiti,
\latex/ assume che sia stato dimenticato accidentalmente un comando di
interruzione di riga \cn{\\} e produce questo errore.
\end{error}

\begin{error}{Improper argument for math accent}
\errexa
\begin{verbatim}
! Package amsmath Error: Improper argument for math accent:
(amsmath)                Extra braces must be added to
(amsmath)                prevent wrong output.

See the amsmath package documentation for explanation.
Type  H <return>  for immediate help.
 ...

l.415 \tilde k_{\lambda_j} = P_{\tilde \mathcal
                                               {M}}
?
\end{verbatim}
\errexpl
Argomenti complessi per tutti i comandi \latex/ dovrebbero venire
racchiusi tra parentesi graffe. In questo esempio le graffe sono
necessarie come mostrato:
\begin{verbatim}
... P_{\tilde{\mathcal{M}}}
\end{verbatim}
\end{error}

\begin{error}{Font OMX/cmex/m/n/7=cmex7 not loadable ...}
\errexa
\begin{verbatim}
! Font OMX/cmex/m/n/7=cmex7 not loadable: Metric (TFM) file not found.
<to be read again>
                   relax
l.8 $a
      b+b^2$
? h
I wasn't able to read the size data for this font,
so I will ignore the font specification.
[Wizards can fix TFM files using TFtoPL/PLtoTF.]
You might try inserting a different font spec;
e.g., type `I\font<same font id>=<substitute font name>'.
?
\end{verbatim}
\errexpl Certe dimensioni di alcuni \emph{font} del Computer Modern che erano
un tempo disponibili principalmente attraverso la raccolta
AMSFonts\index{AMSFonts, raccolta} sono considerate parte del \latex/
standard (giugno 1994): \fn{cmex7}\ndash \texttt{9},
\fn{cmmib5}\ndash \texttt{9}, e \fn{cmbsy5}\ndash \texttt{9}. Se
queste dimensioni straordinarie mancano nel proprio sistema,
bisognerebbe prima recuperarle dalla sogrente che ha fornito \latex/.
Altrimenti, si potrebbe provare a reperirle dalla CTAN (per esempio,
sotto forma di sorgenti Metafont\index{Sorgenti Metafont}, directory
\nfn{/tex-archive/fonts/latex/mf}, o in formato PostScript Type 1,
directory \nfn{/tex-archive/fonts/cm/ps-type1/bakoma}\index{\emph{font}
BaKoMa}\index{\emph{font} PostScript}).

Se il nome del \emph{font} comincia per \fn{cmex}, c'\`e un'opzione speciale
\fn{cmex10} per il pacchetto \pkg{amsmath} che fornisce una scappatoia
temporanea. In altre parole, si cambi il comando \cn{usepackage} in
\begin{verbatim}
\usepackage[cmex10]{amsmath}
\end{verbatim}
Questo forzer\`a l'uso della dimensione di 10 punti del \emph{font} \fn{cmex}
in ogni caso. A seconda del contenuto del documento, questo potrebbe
essere adeguato.
\end{error}

\begin{error}{Math formula deleted: Insufficient extension fonts}
\errexa
\begin{verbatim}
! Math formula deleted: Insufficient extension fonts.
l.8 $ab+b^2$

?
\end{verbatim}
\errexpl
Solitamente questo \`e preceduto da un errore del tipo |Font ... not loadable|;
si veda (sopra) la descrizione di quell'errore per risolvere il problema.
\end{error}

\begin{error}{Missing number, treated as zero}
\errexa
\begin{verbatim}
! Missing number, treated as zero.
<to be read again>
                   a
l.100 \end{alignat}

? h
A number should have been here; I inserted `0'.
(If you can't figure out why I needed to see a number,
look up `weird error' in the index to The TeXbook.)

?
\end{verbatim}
\errexpl
Ci sono parecchie cause che possono provocare questo errore. Comunque, una
possibilit\`a che \`e rilevante per il pacchetto \pkg{amsmath} \`e che si
\`e dimenticato di specificare l'argomento numerico di un ambiente \env{alignat},
come illustrato in questo esempio:
\begin{verbatim}
\begin{alignat}
 a&  =b&    c& =d\\
a'& =b'&   c'& =d'
\end{alignat}
\end{verbatim}
dove la prima linea dovrebbe invece essere
\begin{verbatim}
\begin{alignat}{2}
\end{verbatim}

Un'altra possibilit\`a \`e che una aperta parentesi quadra |[|
segua un comando di interruzione di linea \cn{\\} in un costrutto multilinea
come \env{array}, \env{tabular}, o \env{eqnarray}. Questo sar\`a
interpretato da \latex/ come l'inizio di una richiesta di `spazio verticale
aggiuntivo' \cite[\S C.1.6]{lamport}, anche se capita nella linea successiva
con l'intenzione di renderlo parte del contenuto. Per esempio
\begin{verbatim}
\begin{array}
a+b\\
[f,g]\\
m+n
\end{array}
\end{verbatim}
Per evitare il messaggio d'errore in casi di questo tipo, si possono
aggiungere parentesi graffe come suggerito nel manuale di \latex/
\cite[\S C.1.1]{lamport}:
\begin{verbatim}
\begin{array}
a+b\\
{[f,g]}\\
m+n
\end{array}
\end{verbatim}

\end{error}

\begin{error}{Missing \right. inserted}
\errexa
\begin{verbatim}
! Missing \right. inserted.
<inserted text>
                \right .
l.10 \end{multline}

? h
I've inserted something that you may have forgotten.
(See the <inserted text> above.)
With luck, this will get me unwedged. But if you
really didn't forget anything, try typing `2' now; then
my insertion and my current dilemma will both disappear.
\end{verbatim}
\errexpl
Questo errore si verifica tipicamente quando si cerca di inserire una
interruzione di linea all'interno di una coppia di delimitatori
\cn{left}-\cn{right} in un ambiente \env{multline} o \env{split}:
\begin{verbatim}
\begin{multline}
AAA\left(BBB\\
  CCC\right)
\end{multline}
\end{verbatim}
Ci sono due possibili soluzioni: (1)~invece di usare \cn{left} e
\cn{right}, si usino delimitatori `big' di grandezza fissa (\cn{bigl}
\cn{bigr} \cn{biggl} \cn{biggr} \dots; si veda \secref{bigdel}); oppure
(2)~si usino delimitatori nulli per spezzare la copia \cn{left}-\cn{right}
in due parti, una per ogni linea:
\begin{verbatim}
AAA\left(BBB\right.\\
  \left.CCC\right)
\end{verbatim}
La seconda soluzione potrebbe dar luogo a dimensioni incoerenti dei
delimitatori; ci si pu\`o assicurare che coincidono usando \cn{vphantom}
nella linea in cui compare il delimitatore pi\`u piccolo (o magari \cn{smash}
nella linea in cui compare il delimitatore pi\`u grande). Nell'argomento di
\cn{vphantom} bisogna mettere una copia dell'elemento pi\`u alto che compare
nell'altra linea, per esempio
\begin{verbatim}
xxx \left(\int_t yyy\right.\\
  \left.\vphantom{\int_t} zzz ... \right)
\end{verbatim}
\end{error}

\begin{error}{Paragraph ended before \xxx was complete}
\errexa
\begin{verbatim}
Runaway argument?

! Paragraph ended before \multline was complete.
<to be read again>
                   \par
l.100

? h
I suspect you've forgotten a `}', causing me to apply this
control sequence to too much text. How can we recover?
My plan is to forget the whole thing and hope for the best.
?
\end{verbatim}
\errexpl
Questo potrebbe dipendere da un errore di battitura nel comando
|\end{multline}|, per esempio
\begin{verbatim}
\begin{multline}
...
\end{multiline}
\end{verbatim}
o dall'uso di abbreviazioni di alcuni ambienti, come |\bal| e
|\eal| invece di |\begin{align}| e |\end{align}|:
\begin{verbatim}
\bal
...
\eal
\end{verbatim}
Per motivi tecnici quel tipo di abbreviazioni non funzionano con gli
ambienti pi\`u complesi per equazioni in \emph{display} del pacchetto
\pkg{amsmath} (\env{gather}, \env{align}, \env{split}, etc.; cfr.\@ \fn{technote.tex}).
\end{error}

\begin{error}{Runaway argument?}
Si veda la descrizione del messaggio di errore
\texttt{Paragraph ended before \ncn{xxx} was complete}.
\end{error}

\begin{error}{Unknown option `xxx' for package `yyy'}
\errexa
\begin{verbatim}
! LaTeX Error: Unknown option `intlim' for package `amsmath'.
...
? h
The option `intlim' was not declared in package `amsmath', perhaps you
misspelled its name. Try typing  <return>  to proceed.
?
\end{verbatim}
\errexpl
Questo significa che il nome dell'opzione \`e stato scritto male, o
semplicemente che il pacchetto, al  contrario di quanto ci si aspettava,
non ha quell'opzione. Si consulti la documentazione di quel pacchetto.
\end{error}

\begin{error}{Old form `\pmatrix' should be \begin{pmatrix}.}
\errexa
\begin{verbatim}
! Package amsmath Error: Old form `\pmatrix' should be
                         \begin{pmatrix}.

See the amsmath package documentation for explanation.
Type  H <return>  for immediate help.
 ...

\pmatrix ->\left (\matrix@check \pmatrix
                                         \env@matrix
l.16 \pmatrix
             {a&b\cr c&d\cr}
? h
`\pmatrix{...}' is old Plain-TeX syntax whose use is
ill-advised in LaTeX.
?
\end{verbatim}
\errexpl
Quando si usa il paccheto \pkg{amsmath}, le vecchie forme \cn{pmatrix},
\cn{matrix}, e \cn{cases} non posso pi\`u essere usate, a causa di conflitti
di nome. Ad ogni modo, la loro sintassi non era conforme alla sintassi
standard \LaTeX{}.
\end{error}

\begin{error}{Erroneous nesting of equation structures}
\errexa
\begin{verbatim}
! Package amsmath Error: Erroneous nesting of equation structures;
(amsmath)                trying to recover with `aligned'.

See the amsmath package documentation for explanation.
Type  H <return>  for immediate help.
 ...

l.260 \end{alignat*}
                    \end{equation*}
\end{verbatim}
\errexpl
Le strutture \env{align}, \env{alignat}, etc., sono progettate per essere
usate nel livello pi\`u alto, e perlopi\`u non possono essere annidate in
alcune altre strutture di equazioni in \emph{display}. Una eccezione notevole
\`e data dal fatto che \env{align} e molte sue varianti possono essere usate
nell'ambiente \env{gather}.
\end{error}

\section{Messaggi di warning}

\begin{error}{Foreign command \over [or \atop or \above]}
\errexa
\begin{verbatim}
Package amsmath Warning: Foreign command \over; \frac or \genfrac
(amsmath)                should be used instead.
\end{verbatim}
\errexpl L'utilizzo dei comandi di frazione originali del \tex/\mdash
\cs{over}, \cs{atop}, \cs{above}\mdash \`e deprecato quando si usa il
pacchetto \pkg{amsmath}, perch\`e la loro sintassi \`e estranea a \latex/,
e \pkg{amsmath} fornisce comandi equivalenti nativi di \latex/. Si veda
\fn{technote.tex} per ulteriori informazioni.
\end{error}

\begin{error}{Cannot use `split' here}
\errexa
\begin{verbatim}
Package amsmath Warning: Cannot use `split' here;
(amsmath)                trying to recover with `aligned'
\end{verbatim}
\errexpl L'ambiente \env{split} \`e studiato per essere usato con l'intero
corpo di un'equazione, o una intera linea di un ambiente \env{align} o
\env{gather}. Non ci pu\`o essere alcun tipo di materiale prima o
dopo di esso all'interno della stessa struttura contenente:
\begin{verbatim}
\begin{equation}
\left\{ % <-- Proibito
\begin{split}
...
\end{split}
\right. % <-- Proibito
\end{equation}
\end{verbatim}
\end{error}

\section{Output sbagliato}

\subsection{Sezioni numerate 0.1, 5.1, 8.1 invece che 1, 2, 3}
\label{numinverse}

Questo molto probabilmente significa che gli argomenti di \cn{numberwithin}
sono stati inseriti alla rovescia:
\begin{verbatim}
\numberwithin{section}{equation}
\end{verbatim}
Questo significa ``stampa il numero di sezione nella forma
\textit{numero-equazione}.\textit{numero-sezione} e ricomincia da
1 ogni volta che incontri
un'equazione'' mentre probabilmente si voleva ottenere l'effetto opposto
\begin{verbatim}
\numberwithin{equation}{section}
\end{verbatim}

\subsection{Il comando \cn{numberwithin} non ha avuto effetto sui numeri
di equazione}

State guardando la prima sezione del vostro documento? Controllate la
numerazione delle equazioni in altre parti del documento per vedere se
il problema \`e quello descritto in \secref{numinverse}.

%%%%%%%%%%%%%%%%%%%%%%%%%%%%%%%%%%%%%%%%%%%%%%%%%%%%%%%%%%%%%%%%%%%%%%%%
% by RZ

\chapter{Ulteriori informazioni}

\section{Convertire documenti gi\`a scritti}

\subsection{Convertire da \LaTeX{} ``puro''} %%%% plain

Sotto molti aspetti, un documento \LaTeX{} continua a funzionare
allo stesso modo quando al preambolo del documento si aggiunge
\verb'\usepackage{amsmath}'. Il pacchetto \pkg{amsmath} sopprime
per\`o, salvo diversa indicazione, le interruzioni di pagina all'interno di
strutture che contengono equazioni in \emph{display} come \env{eqnarray},
\env{align} e \env{gather}. Per continuare a permettere le
interruzioni di pagina all'interno di \env{eqnarray} dopo essere
passati al pacchetto \pkg{amsmath}, \`e necessario aggiungere la
seguente riga nel preambolo del documento:
\begin{verbatim}
\allowdisplaybreaks[1]
\end{verbatim}
Per assicurare una spaziatura normale attorno ai simboli di relazione,
si dove sostituire \env{eqnarray} con \env{align}, \env{multline} o
\env{equation}\slash\env{split}, in maniera appropriata.

La maggior parte delle altre differenze d'uso del pacchetto
\pkg{amsmath} possono essere considerate raffinatezze facoltative,
come per esempio l'uso di
\begin{verbatim}
\DeclareMathOperator{\Hom}{Hom}
\end{verbatim}
invece di \verb'\newcommand{\Hom}{\mbox{Hom}}'.

\subsection{Convertire da \amslatex/ 1.1}
Si veda \fn{diffs-m.txt}.

\section{Note tecniche}
Il file \fn{technote.tex} contiene alcuni commenti su diverse questioni
che difficilmente possono essere di interesse generale.

\section{Ottenere aiuto}

Domande o commenti riguardanti \pkg{amsmath} e pacchetti correlati
dovrebbero essere inviati a:
\begin{infoaddress}
American Mathematical Society\\
Technical Support\\
Electronic Products and Services\\
P. O. Box 6248\\
Providence, RI 02940\\[3pt]
Phone: 800-321-4AMS (321-4267) \quad or \quad 401-455-4080\\
Internet: \mail{tech-support@ams.org}
\end{infoaddress}
Quando si riporta un problema occorre includere, per consentire
un'indagine adeguata, le seguenti informazioni:

\begin{enumerate}
\item Il file sorgente in cui \`e sorto il problema, preferibilmente
  ridotto alle minime dimensioni rimuovendo tutto il materiale che pu\`o
  essere rimosso senza interferire sul problema in questione.
\item Un file di log di \latex/ che mostri il messaggio di errore (se
  presente) e i numeri di versione delle classi di documento e file di
  opzioni in uso.
\end{enumerate}

\section{Di possibile interesse}\label{a:possible-interest}
\`E possibile avere informazioni su come ottenere i \emph{font} AMS o
altro materiale relativo a \tex/ dall'archivio Internet AMS
\fn{e-math.ams.org} inviando una richiesta attraverso la posta
elettronica a: \mail{webmaster@ams.org}.

Si possono avere informazioni su come ottenere dall'AMS la distribuzione
\pkg{amsmath} su dischetti da:
\begin{infoaddress}
American Mathematical Society\\
Customer Services\\
P. O. Box 6248\\
Providence, RI 02940\\[3pt]
Phone: 800-321-4AMS (321-4267) \quad or \quad 401-455-4000\\
Internet: \mail{cust-serv@ams.org}
\end{infoaddress}

Il ``\tex/ Users Group\index{TeX Users@\tex/ Users Group}''
\`e una organizzazione senza scopo di lucro che pubblica una
rivista (\journalname{TUGboat}\index{TUGboat@\journalname{TUGboat}}),
organizza meeting, e serve da punto di smistamento per informazioni
su \tex/ e software relativo ad esso.
\begin{infoaddress}
\tex/ Users Group\\
PO Box 2311\\
Portland, OR 97208-2311\\
USA\\[3pt]
Phone: +1-503-223-9994\\
Email: \mail{office@tug.org}
\end{infoaddress}
Iscriversi al ``\tex/ Users Group'' \`e un buon modo per sostenere il
continuo sviluppo di software libero relativo a \tex/.
Esistono inoltre molti ``\tex/ users group'' locali in altri stati;
si possono ottenere informazioni su come contattare un gruppo locale dal
``\tex/ Users Group''.

Esiste un gruppo di discussione Usenet chiamato \fn{comp.text.tex},
che \`e una buona fonte di informazioni su \latex/ e
\tex/ in generale. Se non si sa come leggere un gruppo di discussione,
occorre chiedere all'amministratore di sistema locale se \`e
disponibile un servizio di lettura di \emph{newsgroup}.


\begin{thebibliography}{9}
\addcontentsline{toc}{chapter}{Bibliografia}

\bibitem{amsfonts}\booktitle{AMSFonts version \textup{2.2}\mdash user's guide},
Amer. Math. Soc., Providence, RI, 1994; distribuito
con il pacchetto AMSFonts.

\bibitem{instr-l}\booktitle{Instructions for preparation of
papers and monographs\mdash \amslatex/},
Amer. Math. Soc., Providence, RI, 1996, 1999.

\bibitem{amsthdoc}\booktitle{Using the \pkg{amsthm} Package},
Amer. Math. Soc., Providence, RI, 1999.

\bibitem{tlc} Michel Goossens, Frank Mittelbach e Alexander Samarin,
\booktitle{The \latex/ companion}, Addison-Wesley, Reading, MA, 1994.
  [\emph{Note: L'edizione del 1994 non \`e una guida affidabile per il
    pacchetto \pkg{amsmath} a meno che non ci si riferisca al file
    \fn{compan.err}, distribuito con \LaTeX{}, che contiene una errata
    corrige per il Capitolo 8\mdash.}]

% Deal with a line breaking problem
\begin{raggedright}
\bibitem{mil} G. Gr\"{a}tzer,
\emph{Math into \LaTeX{}: An Introduction to \LaTeX{} and AMS-\LaTeX{}}
  \url{http://www.ams.org/cgi-bin/bookstore/bookpromo?fn=91&arg1=bookvideo&itmc=MLTEX},
Birkh\"{a}user, Boston, 1995.\par
\end{raggedright}

\bibitem{kn} Donald E. Knuth, \booktitle{The \tex/book},
Addison-Wesley, Reading, MA, 1984.

\bibitem{lamport} Leslie Lamport, \booktitle{\latex/: A document preparation
system}, 2nd revised ed., Addison-Wesley, Reading, MA, 1994.

\bibitem{msf} Frank Mittelbach and Rainer Sch\"opf,
\textit{The new font family selection\mdash user
interface to standard \latex/}, \journalname{TUGboat} \textbf{11},
no.~2 (June 1990), pp.~297\ndash 305.

\bibitem{jt} Michael Spivak, \booktitle{The joy of \tex/}, 2nd revised ed.,
Amer. Math. Soc., Providence, RI, 1990.

\end{thebibliography}
% by RZ - end

%%%% ====================================================================
%%% amsldoc.tex 2.07
%%% ====================================================================
\documentclass[a4paper,leqno,titlepage,openany]{amsldoc}[1999/12/13]
\usepackage[italian]{babel}
\renewcommand{\errexa}{\par\noindent\textit{Esempio}:\ }
\renewcommand{\errexpl}{\par\noindent\textit{Spiegazione}:\ }
\DeclareRobustCommand{\cls}[1]{{\ntt#1}%
  \autoindex{#1@\string\cls{#1}, classe}}
\DeclareRobustCommand{\pkg}[1]{{\ntt#1}%
  \autoindex{#1@\string\pkg{#1}, pacchetto}}
\DeclareRobustCommand{\opt}[1]{{\ntt#1}%
  \autoindex{#1@\string\opt{#1}, opzione}}
\DeclareRobustCommand{\env}[1]{{\ntt#1}%
  \autoindex{#1@\string\env{#1}, ambiente}}
\DeclareRobustCommand{\fn}[1]{{\ntt#1}%
   \autoindex{#1@\string\fn{#1}}}
\DeclareRobustCommand{\bst}[1]{{\ntt#1}\autoindex{#1@{\string\ntt{}#1,
  stile bibliografico}}}

\ifx\UndEfiNed\url
  \ClassError{amsldoc}{%
    This version of amsldoc.tex must be processed\MessageBreak
    with a newer version of amsldoc.cls (2.02 or later)}{}
\fi

\title{Manuale utente per il pacchetto \pkg{amsmath} (versione~2.0)}
\author{American Mathematical Society}
\date{13/12/1999}

%    Use the amsmath package and amscd package in order to print
%    examples.
\usepackage{amsmath}
\usepackage{amscd}
% Inserito il pacchetto makeidx - GD
\usepackage{makeidx}

\makeindex % generate index data
\providecommand{\see}[2]{\textit{vedi} #1}

%    The amsldoc class includes a number of features useful for
%    documentation about TeX, including:
%
%    ---Commands \tex/, \amstex/, \latex/, ... for uniform treatment
%    of the various logos and easy handling of following spaces.
%
%    ---Commands for printing various common elements: \cn for command
%    names, \fn for file names (including font-file names), \env for
%    environments, \pkg and \cls for packages and classes, etc.

%    Many of the command names used here are rather long and will
%    contribute to poor linebreaking if we follow the \latex/ practice
%    of not hyphenating anything set in tt font; instead we selectively
%    allow some hyphenation.
\allowtthyphens % defined in amsldoc.cls

\hyphenation{ac-cent-ed-sym-bol add-to-counter add-to-length align-at
  aligned-at allow-dis-play-breaks ams-art ams-cd ams-la-tex amsl-doc
  ams-symb ams-tex ams-text ams-xtra bmatrix bold-sym-bol cen-ter-tags
  eqn-ar-ray idots-int int-lim-its latex med-space neg-med-space
  neg-thick-space neg-thin-space no-int-lim-its no-name-lim-its
  over-left-arrow over-left-right-arrow over-right-arrow pmatrix
  qed-sym-bol set-length side-set small-er tbinom the-equa-tion
  thick-space thin-space un-der-left-arrow un-der-left-right-arrow
  un-der-right-arrow use-pack-age var-inj-lim var-proj-lim vmatrix
  xalign-at xx-align-at}

%    Prepare for illustrating the \vec example
\newcommand{\vect}[1]{\mathbf{#1}}

\newcommand{\booktitle}[1]{\textit{#1}}
\newcommand{\journalname}[1]{\textit{#1}}
\newcommand{\seriesname}[1]{\textit{#1}}

%    Command to insert and index a particular phrase. Doesn't work for
%    certain kinds of special characters in the argument.
\newcommand{\ii}[1]{#1\index{#1}}

\newcommand{\vstrut}[1]{\vrule width0pt height#1\relax}

%    An environment for presenting comprehensive address information:
\newenvironment{infoaddress}{%
  \par\topsep\medskipamount
  \trivlist\centering
  \item[]%
  \begin{minipage}{.7\columnwidth}%
  \raggedright
}{%
  \end{minipage}%
  \endtrivlist
}

\newenvironment{eqxample}{%
  \par\addvspace\medskipamount
  \noindent\begin{minipage}{.5\columnwidth}%
  \def\producing{\end{minipage}\begin{minipage}{.5\columnwidth}%
    \hbox\bgroup\kern-.2pt\vrule width.2pt%
      \vbox\bgroup\parindent0pt\relax
%    The 3pt is to cancel the -\lineskip from \displ@y
    \abovedisplayskip3pt \abovedisplayshortskip\abovedisplayskip
    \belowdisplayskip0pt \belowdisplayshortskip\belowdisplayskip
    \noindent}
}{%
  \par
%    Ensure that a lonely \[\] structure doesn't take up width less than
%    \hsize.
  \hrule height0pt width\hsize
  \egroup\vrule width.2pt\kern-.2pt\egroup
  \end{minipage}%
  \par\addvspace\medskipamount
}

%    The chapters are so short, perhaps we shouldn't call them by the
%    name `Chapter'. We make \chaptername read an argument in order to
%    remove a following \space or "{} " (both possibilities are present
%    in book.cls).

\renewcommand{\chaptername}[1]{}
\newcommand{\chapnum}[1]{\mdash #1\mdash }
\makeatletter
\def\@makechapterhead#1{%
  \vspace{1.5\baselineskip}%
  {\parindent \z@ \raggedright \reset@font
    \ifnum \c@secnumdepth >\m@ne
      \large\bfseries \chapnum{\thechapter}%
      \par\nobreak
      \vskip.5\baselineskip\relax
    \fi
    #1\par\nobreak
    \vskip\baselineskip
  }}
\makeatother

%    A command for ragged-right parbox in a tabular.
\newcommand{\rp}{\let\PBS\\\raggedright\let\\\PBS}

%    Non-indexed file name
\newcommand{\nfn}[1]{\texttt{#1}}

%    For the examples in the math spacing table.
%%\newcommand{\lspx}{\mbox{\rule{5pt}{.6pt}\rule{.6pt}{6pt}}}
%%\newcommand{\rspx}{\mbox{\rule[-1pt]{.6pt}{7pt}%
%%  \rule[-1pt]{5pt}{.6pt}}}
\newcommand{\lspx}{\mathord{\Rightarrow\mkern-1mu}}
\newcommand{\rspx}{\mathord{\mkern-1mu\Leftarrow}}
\newcommand{\spx}[1]{$\lspx #1\rspx$}

%    For a list of characters representing document input.
\newcommand{\clist}[1]{%
  \mbox{\ntt\spaceskip.2em plus.1em \xspaceskip\spaceskip#1}}

%    Fix weird \latex/ definition of rightmark.
\makeatletter
\def\rightmark{\expandafter\@rightmark\botmark{}{}}
%    Also turn off section marks.
\let\sectionmark\@gobble
\renewcommand{\chaptermark}[1]{%
  \uppercase{\markboth{\rhcn#1}{\rhcn#1}}}
\newcommand{\rhcn}{\thechapter. }
\makeatother

%    Include down to \section but not \subsection, in toc:
\setcounter{tocdepth}{1}

\DeclareMathOperator{\ix}{ix}
\DeclareMathOperator{\nul}{nul}
\DeclareMathOperator{\End}{End}
\DeclareMathOperator{\xxx}{xxx}

\begin{document}

%%%%%%%%%%%%%%%%%%%%%%%%%%%%%%%%%%%%%%%%%%%%%%%%%%%%%%%%%%%%%%%%%%%%%%%%
\frontmatter

\maketitle
%
%
%
\pagebreak
\begin{small} 
 \noindent Titolo originale: \emph{User manual for the \pkg{amsmath} package (version~2.0)}

 \smallskip
 \noindent Traduzione:

 \begin{quote}
 \flushleft %  \footnotesize
 Giulio Agostini, % <giulio.agostini@bigfoot.com>
 Giuseppe Bilotta, % <bourbaki@bigfoot.com>
 Flavio Casadei Della Chiesa, % <flavio_c@libero.it>
 Onofrio de Bari, % <thufir@tin.it>
 Giacomo Delre, % <giader@penguinpowered.com>
 Luca Ferrante, % <ironluke@split.it>
 Tommaso Pecorella, % <t.pecorella@inwind.it>
 Mileto Rigido, % <m.rigido@flashnet.it>
 Roberto Zanasi. % <roberto.zanasi@libero.it>
 \end{quote}

\end{small}
%
%
%
\pagestyle{headings}
\tableofcontents
\cleardoublepage % for better page number placement

%%%%%%%%%%%%%%%%%%%%%%%%%%%%%%%%%%%%%%%%%%%%%%%%%%%%%%%%%%%%%%%%%%%%%%%%
\mainmatter
%%%%%%%%%%%%%%%%%%%%%%%%%%%%%%%%%%%%%%%%%%%%%%%%%%%%%%%%%%%%%%%%%%%%%%%%

%%% Nota dei traduttori
\subsubsection*{Nota alla traduzione italiana}
Una copia di questo documento e altre traduzioni in italiano di
manuali su \LaTeX\ sono reperibili presso
\begin{itemize}
\item\url{http://guild.prato.linux.it}
\item\url{ftp://lorien.prato.linux.it/pub/guild}
\item\url{ftp://ftp.unina.it/pub/TeX/info/italian}
\end{itemize}
e su ogni sito CTAN, per esempio \url{ftp://ftp.tex.ac.uk/tex-archive/info/italian}.
%%%%%%%%%%%%%%%%%%%%%%%%%%%%%%%%%%%%%%%%%%%%%%%%%%%%%%%%%%%%%%%%%%%%%%%%


% by GA
\chapter{Introduzione}

Il pacchetto \pkg{amsmath} \`e un pacchetto \LaTeX{} che fornisce
svariate estensioni per il miglioramento della struttura informativa e
della stampa di documenti che contengono formule matematiche. I lettori
che non conoscono \LaTeX{} sono invitati a consultare \cite{lamport}.
Se si possiede una versione aggiornata di \LaTeX{}, il pacchetto \pkg{amsmath}
\`e normalmente incluso. Quando viene pubblicata una nuova versione del
pacchetto \pkg{amsmath}, \`e possibile effettuare un aggiornamento attraverso
\url{http://www.ams.org/tex/amsmath.html} o
\url{ftp://ftp.ams.org/pub/tex/}.

Questo documento descrive le funzionalit\`a del paccheto \pkg{amsmath}
e spiega come dovrebbero essere usate. Esso copre inoltre alcuni pacchetti
ausiliari:
\begin{ctab}{ll}
\pkg{amsbsy}& \pkg{amstext}\\
\pkg{amscd}& \pkg{amsxtra}\\
\pkg{amsopn}
\end{ctab}
Tutti questi hanno a che vedere con il contenuto di formule
matematiche. Per informazioni su ulteriori simboli e \emph{font} matematici,
si veda \cite{amsfonts} e \url{http://www.ams.org/tex/amsfonts.html}.
Per la documentazione del pacchetto \pkg{amsthm} o delle classi AMS
(\cls{amsart}, \cls{amsbook}, etc.\@) si veda \cite{amsthdoc} o
\cite{instr-l} e \url{http://www.ams.org/tex/author-info.html}.

Se siete utenti di \latex/ da molto tempo e avete molta matematica nei
vostri scritti, potreste trovare soluzioni a problemi familiari in
questo elenco di funzionalit\`a di \pkg{amsmath}:
\begin{itemize}

\item Un modo comodo per definire un nuovo comando `nome di operatore', come
\cn{sin} e \cn{lim}, con spazi appropriati ai lati e selezione automatica
di stile e dimensioni corrette del \emph{font} (anche quando usato in esponenti
o deponenti).

\item Diversi alternative all'ambiente \env{eqnarray} per rendere le
diverse disposizioni delle equazioni pi\`u facili da scrivere.

\item I numeri delle equazioni si spostano automaticamente in alto o in
basso per evitare di sovrapporsi con l'equazione stessa (al contrario
di \env{eqnarray}).

\item Gli spazi attorno ai segni di uguaglianza sono gli stessi della
normale spaziatura nell'ambiente \env{equation} (al contrario di
\env{eqnarray}).

\item Un modo per produrre deponenti a pi\`u linee come spesso \`e
richiesto dai simboli di sommatoria e produttoria.

\item Un modo semplice di numerare una determinata equazione con un
riferimento diverso da quello fornito dalla numerazione automatica.

\item Un modo semplice di produrre numerazioni subordinate per le
equazioni, nella forma (1.3a) (1.3b) (1.3c), per un determinato insieme
di equazioni.

\end{itemize}

Il pacchetto \pkg{amsmath} \`e distribuito insieme ad alcuni piccoli
pacchetti ausiliari:
\begin{description}
\item[\pkg{amsmath}] Il pacchetto principale, fornisce diverse funzionalit\`a
  per equazioni in \emph{display} e altri costrutti matematici.

\item[\pkg{amstext}] Fornisce il comando \cn{text} per
  sistemare un frammento di testo in un \emph{display}.

\item[\pkg{amsopn}] Fornisce il comando \cn{DeclareMathOperator} per definire
  nuovi `nomi di operatori' come \cn{sin} e \cn{lim}.

\item[\pkg{amsbsy}] Per compatibilit\`a all'indietro questo pacchetto
  continua a esistere, ma in alternativa ad esso si consiglia l'uso del
  pi\`u recente pacchetto \pkg{bm} fornito a corredo di \LaTeX{}.

\item[\pkg{amscd}] Fornisce un ambiente \env{CD} per semplici diagrammi
  commutativi (privi di frecce diagonali).

\item[\pkg{amsxtra}] Fornisce alcune cianfrusaglie come \cn{fracwithdelims}
  e \cn{accentedsymbol}, per compatibilit\`a con documenti creati usando
  la versione~1.1.

\end{description}

Il pacchetto \pkg{amsmath} incorpora \pkg{amstext}, \pkg{amsopn}, e
\pkg{amsbsy}. Le funzionalit\`a di \pkg{amscd} e \pkg{amsxtra}, invece,
sono disponibili solo invocando separatamente questi pacchetti.

%%%%%%%%%%%%%%%%%%%%%%%%%%%%%%%%%%%%%%%%%%%%%%%%%%%%%%%%%%%%%%%%%%%%%%%%
% by GA

\chapter{Opzioni per il pacchetto \pkg{amsmath}}\label{options}

Il pacchetto \pkg{amsmath} ha le seguenti opzioni:
\begin{description}

\item[\opt{centertags}] (\emph{default}) Centra verticalmente\index{equazioni,
numeri delle!posizionamento verticale}, rispetto all'altezza totale
dell'equazione, la numerazione delle equazioni spezzate su pi\`u linee.

\item[\opt{tbtags}] `Top-or-bottom tags' (Etichette in cima o in fondo):
Allinea la numerazione\index{equazioni, numeri delle!posizionamento
verticale} delle equazioni spezzate su pi\`u linee all'ultima
(rispettivamente alla prima) linea, se i numeri stanno sulla destra
(rispettivamente sulla sinistra).

\item[\opt{sumlimits}] (\emph{default}) Posiziona esponenti e
deponenti\index{esponenti e deponenti!posizionamento}\relax
\index{limiti|see{esponenti e deponenti}} dei simboli di sommatoria
sopra e sotto, nelle equazioni in \emph{display}. Questa opzione
influenza anche altri simboli dello stesso tipo\mdash $\prod$,
$\coprod$, $\bigotimes$, $\bigoplus$, e cos\`\i\ via\mdash eccetto gli
integrali (vedi sotto).

\item[\opt{nosumlimits}] Posiziona gli esponenti e deponenti dei
simboli simil-sommatoria sempre a fianco, anche nelle equazioni in
\emph{display}.

\item[\opt{intlimits}] Come \opt{sumlimits}, ma per i simboli di
integrale\index{integrali!posizionamento dei limiti}.

\item[\opt{nointlimits}] (\emph{default}) Il contrario di \opt{intlimits}.

\item[\opt{namelimits}] (\emph{default}) Come \opt{sumlimits}, ma per certi
`nomi di operatori' come $\det$, $\inf$, $\lim$, $\max$, $\min$, che
tradizionalmente hanno deponenti \index{esponenti e
deponenti!posizionamento} posizionati sotto di essi all'interno di
equazioni \emph{display}.

\item[\opt{nonamelimits}] Il contrario di \opt{namelimits}.

\end{description}

Per usare una di queste opzioni del pacchetto bisogna mettere il nome
dell'opzione nell'argomento opzionale del comando \cn{usepackage}\mdash
ad esempio, \verb"\usepackage[intlimits]{amsmath}".

Il pacchetto \pkg{amsmath} inoltre riconosce le seguenti opzioni che
sono normalmente selezionate (implicitamente o esplicitamente)
attraverso il comando \cn{documentclass}, e che pertanto non hanno
bisogno di essere ripetute nell'elenco di opzioni del comando
\cn{usepackage}|{amsmath}|.
\begin{description}

\item[\opt{leqno}] Posiziona i numeri di equazione sulla
sinistra.\index{equazioni, numeri delle!posizionamento a destra o a
sinistra}

\item[\opt{reqno}] Posiziona i numeri di equazione sulla destra.

\item[\opt{fleqn}] Posiziona i numeri di equazione a una distanza prefissata
dal margine sinistro piuttosto che centrata nella colonna di
testo.\index{equazioni in \emph{display}!centratura}

\end{description}

%%%%%%%%%%%%%%%%%%%%%%%%%%%%%%%%%%%%%%%%%%%%%%%%%%%%%%%%%%%%%%%%%%%%%%%%
% by GB

\chapter{Equazioni in \emph{display}}

\section{Introduzione}

Il pacchetto \pkg{amsmath} fornisce un certo numero di nuove strutture
per le equazioni in \emph{display}\index{equazioni in
\emph{display}}\index{equazioni|see{equazioni in \emph{display}}},
oltre a quelle fornite dal \latex/ di base; fra queste:
\begin{verbatim}
  equation     equation*     align       align*
  gather       gather*       flalign     flalign*
  multline     multline*     alignat     alignat*
  split
\end{verbatim}
(Sebbene l'ambiente standard \env{eqnarray} rimanga disponibile, \`e
opportuno usare \env{align} o \env{equation}+\env{split}, invece.)

Con l'eccezione di \env{split}, ogni ambiente ha sia una versione
stellata sia una non stellata, dove la versione non stellata permette
la numerazione automatica usando il contatore \latex/ \env{equation}.
Si pu\`o sopprimere il numero in ogni singola linea premettendo un
\cn{notag} al codice \cn{\\}; lo si pu\`o anche
scavalcare\index{equazioni, numeri delle!scavalcare} con un valore di
propria scelta, usando il comando \cn{tag}|{|\<etich>|}|, dove \<etich>
\`e un testo arbitrario, come |$*$| o |ii|, usato  per \qq{numerare}
l'equazione. Si pu\`o anche usare il comando \cn{tag*}, che fa in modo
che il testo fornito venga scritto letteralmente, senza aggiunta di
parentesi. \cn{tag} e \cn{tag*} possono anche essere usati nelle
versioni non numerate di tutte le strutture di allineamento di
\pkg{amsmath}. Alcuni esempi dell'uso di \cn{tag} possono essere
trovati nei file di esempio \fn{testmath.tex} e \fn{subeqn.tex}
forniti con il pacchetto \pkg{amsmath}.

L'ambiente \env{split} \`e una speciale forma subordinata, da usare
solo \emph{all'interno} di altre strutture; non pu\`o essere usato in
una \env{multline}.

Nelle strutture d'allineamento (\env{split}, \env{align} e varianti),
i simboli di relazione hanno un \verb'&' prima, ma non dopo\mdash a
differenza di \env{eqnarray}. Mettere un \verb'&' dopo il simbolo di
relazione interferirebbe con la spaziatura: \`e necessario metterlo
prima.

\begin{table}[p]
\caption[]{Confronto degli ambienti per le equazioni in \emph{display}
(le linee verticali indicano i margini nominali)}\label{displays}
\renewcommand{\theequation}{\arabic{equation}}
\begin{eqxample}
\begin{verbatim}
\begin{equation*}
a=b
\end{equation*}
\end{verbatim}
\producing
\begin{equation*}
a=b
\end{equation*}
\end{eqxample}

\begin{eqxample}
\begin{verbatim}
\begin{equation}
a=b
\end{equation}
\end{verbatim}
\producing
\begin{equation}
a=b
\end{equation}
\end{eqxample}

\begin{eqxample}
\begin{verbatim}
\begin{equation}\label{xx}
\begin{split}
a& =b+c-d\\
 & \quad +e-f\\
 & =g+h\\
 & =i
\end{split}
\end{equation}
\end{verbatim}
\producing
\begin{equation}\label{xx}
\begin{split}
a& =b+c-d\\
 & \quad +e-f\\
 & =g+h\\
 & =i
\end{split}
\end{equation}
\end{eqxample}

\begin{eqxample}
\begin{verbatim}
\begin{multline}
a+b+c+d+e+f\\
+i+j+k+l+m+n
\end{multline}
\end{verbatim}
\producing
\begin{multline}
a+b+c+d+e+f\\
+i+j+k+l+m+n
\end{multline}
\end{eqxample}

\begin{eqxample}
\begin{verbatim}
\begin{gather}
a_1=b_1+c_1\\
a_2=b_2+c_2-d_2+e_2
\end{gather}
\end{verbatim}
\producing
\begin{gather}
a_1=b_1+c_1\\
a_2=b_2+c_2-d_2+e_2
\end{gather}
\end{eqxample}

\begin{eqxample}
\begin{verbatim}
\begin{align}
a_1& =b_1+c_1\\
a_2& =b_2+c_2-d_2+e_2
\end{align}
\end{verbatim}
\producing
\begin{align}
a_1& =b_1+c_1\\
a_2& =b_2+c_2-d_2+e_2
\end{align}
\end{eqxample}

\begin{eqxample}
\begin{verbatim}
\begin{align}
a_{11}& =b_{11}&
  a_{12}& =b_{12}\\
a_{21}& =b_{21}&
  a_{22}& =b_{22}+c_{22}
\end{align}
\end{verbatim}
\producing
\begin{align}
a_{11}& =b_{11}&
  a_{12}& =b_{12}\\
a_{21}& =b_{21}&
  a_{22}& =b_{22}+c_{22}
\end{align}
\end{eqxample}

\begin{eqxample}
\begin{verbatim}
\begin{flalign*}
a_{11}& =b_{11}&
  a_{12}& =b_{12}\\
a_{21}& =b_{21}&
  a_{22}& =b_{22}+c_{22}
\end{flalign*}
\end{verbatim}
\producing
\begin{flalign*}
a_{11}& =b_{11}&
  a_{12}& =b_{12}\\
a_{21}& =b_{21}&
  a_{22}& =b_{22}+c_{22}
\end{flalign*}
\end{eqxample}
\end{table}

\section{Singole equazioni}

L'ambiente \env{equation} viene usato per singole equazioni con
numerazione automatica; l'ambiente \env{equation*} ha la stessa
funzione, senza numerazione.%
%%%%%%%%%%%%%%%%%%%%%%%%%%%%%%%%%%%%%%%%%%%%%%%%%%%%%%%%%%%%%%%%%%%%%%%%
\footnote{\latex/ non fornisce un ambiente \env{equation*}, ma un
ambiente con funzioni analoghe: \env{displaymath}.}

\section{Equazioni spezzate senza allineamento}

L'ambiente \env{multline} \`e una variante di \env{equation}, usata
per le equazioni che non entrano in un'unica riga. La prima riga di una
\env{multline} sar\`a al margine sinistro, e l'ultima al margine
destro, tranne per un rientro ambo i lati, di lunghezza
\cn{multlinegap}; tutte le altre linee verranno centrate
indipendentemente considerando la larghezza del \emph{display} (a meno
che non sia in funzione l'opzione \opt{fleqn}).

Come \env{equation}, \env{multline} fornisce un'unico numero
d'equazione (quindi, nessuna delle singole linee dovrebbe essere
segnata con \cn{notag}). Il numero dell'equazione \`e posto all'ultima
riga (opzione \opt{reqno}) o sulla prima linea (opzione \opt{leqno});
il centramento verticale (come per \env{split}) non \`e supportato in
\env{multline}.

\`E possibile forzare una delle righe di centro a sinistra o a destra
con i comandi \cn{shoveleft}, \cn{shoveright}; questi comandi prendono
l'intera linea come argomento, fino al segno \cn{\\} escluso; ad
esempio
\begin{multline}
\framebox[.65\columnwidth]{A}\\
\framebox[.5\columnwidth]{B}\\
\shoveright{\framebox[.55\columnwidth]{C}}\\
\framebox[.65\columnwidth]{D}
\end{multline}
\begin{verbatim}
\begin{multline}
\framebox[.65\columnwidth]{A}\\
\framebox[.5\columnwidth]{B}\\
\shoveright{\framebox[.55\columnwidth]{C}}\\
\framebox[.65\columnwidth]{D}
\end{multline}
\end{verbatim}

Il valore di \cn{multlinegap} pu\`o essere cambiato con i soliti
comandi \latex/ \cn{setlength} or \cn{addtolength}.

\section{Equazioni spezzate con allineamento}

Come \env{multline}, l'ambiente \env{split} \`e per \emph{singole}
equazioni troppo lunghe per entrare in una riga e che pertanto devono
essere spezzate. A differenza di \env{multline}, per\`o, l'ambiente
\env{split} permette allineamento tra le linee, con l'uso di simboli
|&| per segnare i punti di allineamento. A differenza di altre
strutture di equazioni \pkg{amsmath}, l'ambiente \env{split} non
produce numeri, poich\'e \`e progettato per essere usato
\emph{esclusivamente all'interno di qualche altra struttura per
equazioni in \emph{display}}, solitamente un ambiente \env{equation},
\env{align}, o \env{gather}, che fornisce la numerazione; ad esempio:
\begin{equation}\label{e:barwq}\begin{split}
H_c&=\frac{1}{2n} \sum^n_{l=0}(-1)^{l}(n-{l})^{p-2}
\sum_{l _1+\dots+ l _p=l}\prod^p_{i=1} \binom{n_i}{l _i}\\
&\quad\cdot[(n-l )-(n_i-l _i)]^{n_i-l _i}\cdot
\Bigl[(n-l )^2-\sum^p_{j=1}(n_i-l _i)^2\Bigr].
\kern-2em % adjust equation body to the right [mjd,13-Nov-1994]
\end{split}\end{equation}

\begin{verbatim}
\begin{equation}\label{e:barwq}\begin{split}
H_c&=\frac{1}{2n} \sum^n_{l=0}(-1)^{l}(n-{l})^{p-2}
\sum_{l _1+\dots+ l _p=l}\prod^p_{i=1} \binom{n_i}{l _i}\\
&\quad\cdot[(n-l )-(n_i-l _i)]^{n_i-l _i}\cdot
\Bigl[(n-l )^2-\sum^p_{j=1}(n_i-l _i)^2\Bigr].
\end{split}\end{equation}
\end{verbatim}

La struttura \env{split} dovrebbe costituire l'intero corpo della
struttura racchiudente, tranne per comandi come \cn{label} che non
producono testo visibile.

\section{Gruppi di equazioni senza allineamento}

L'ambiente \env{gather} viene usato per ragguppare equazioni
consecutive quando non vi \`e necessit\`a di allineamento; ogni
equazione \`e centrata separatamente entro i margini (come in
Tabella~\ref{displays}). Le equazioni in un ambiente \env{gather} sono
separati da comandi \cn{\bslash}. Ogni equazione \env{gather} pu\`o
essere un blocco \verb'\begin{split}'
  \dots\ \verb'\end{split}' \mdash ad esempio:
\begin{verbatim}
\begin{gather}
  prima equazione\\
  \begin{split}
    seconda & equazione\\
           & su due linee
  \end{split}
  \\
  terza equazione
\end{gather}
\end{verbatim}

\section{Gruppi di equazioni con allineamento reciproco}

L'ambiente \env{align} \`e usato per gruppi di due o pi\`u  equazioni
quando \`e richiesto allineamento reciproco; di solito vengono scelti
i simboli di relazione per gli allineamenti (come in
Tabella~\ref{displays}).

Per avere pi\`u colonne di equazioni affiancate, si possono usare
simboli di ``e'' commerciale aggiuntivi per separare le colonne:
\begin{align}
x&=y       & X&=Y       & a&=b+c\\
x'&=y'     & X'&=Y'     & a'&=b\\
x+x'&=y+y' & X+X'&=Y+Y' & a'b&=c'b
\end{align}
%
\begin{verbatim}
\begin{align}
x&=y       & X&=Y       & a&=b+c\\
x'&=y'     & X'&=Y'     & a'&=b\\
x+x'&=y+y' & X+X'&=Y+Y' & a'b&=c'b
\end{align}
\end{verbatim}
Annotazioni linea-per-linea sulle equazioni possono essere ottenute
con un opportuno uso di \cn{text} in un ambiente \env{align}:
\begin{align}
x& = y_1-y_2+y_3-y_5+y_8-\dots
                    && \text{per \eqref{eq:C}}\\
 & = y'\circ y^*    && \text{per \eqref{eq:D}}\\
 & = y(0) y'        && \text {per l'Assioma 1.}
\end{align}
%
\begin{verbatim}
\begin{align}
x& = y_1-y_2+y_3-y_5+y_8-\dots
                    && \text{per \eqref{eq:C}}\\
 & = y'\circ y^*    && \text{per \eqref{eq:D}}\\
 & = y(0) y'        && \text {per l'Assioma 1.}
\end{align}
\end{verbatim}
Una variante, l'ambiente \env{alignat}, permette di specificare
manualmente lo spazio orizzontale fra le equazioni; questo ambiente ha
un argomento obbligatorio, il numero di \qq{colonne di equazioni}: si
contano il numero di \verb'&' in una riga, si aggiunge 1 e si divide
per 2.
\begin{alignat}{2}
x& = y_1-y_2+y_3-y_5+y_8-\dots
                  &\quad& \text{per \eqref{eq:C}}\\
 & = y'\circ y^*  && \text{per \eqref{eq:D}}\\
 & = y(0) y'      && \text {per l'Assioma 1.}
\end{alignat}
%
\begin{verbatim}
\begin{alignat}{2}
x& = y_1-y_2+y_3-y_5+y_8-\dots
                  &\quad& \text{per \eqref{eq:C}}\\
 & = y'\circ y^*  && \text{per \eqref{eq:D}}\\
 & = y(0) y'      && \text {per l'Assioma 1.}
\end{alignat}
\end{verbatim}

\section{Blocchi per costrutti allineati}

Come \env{equation}, gli ambienti a equazioni multiple \env{gather},
\env{align} e \env{alignat} sono progettati per produrre strutture
aventi lunghezza complessiva pari alla lunghezza di una riga; questo
implica, ad esempio, che non \`e facile aggiungere parentesi attorno
alle strutture; vengono quindi fornite le varianti \env{gathered},
\env{aligned} e \env{alignedat}, la cui lunghezza totale \`e pari alla
reale lunghezza dei contenuti; possono quindi essere usate come
componenti di un'espressione pi\`u complessa; ad esempio,
\begin{equation*}
\left.\begin{aligned}
  B'&=-\partial\times E,\\
  E'&=\partial\times B - 4\pi j,
\end{aligned}
\right\}
\qquad \text{equazioni di Maxwell}
\end{equation*}
\begin{verbatim}
\begin{equation*}
\left.\begin{aligned}
  B'&=-\partial\times E,\\
  E'&=\partial\times B - 4\pi j,
\end{aligned}
\right\}
\qquad \text{equazioni di Maxwell}
\end{equation*}
\end{verbatim}
Come l'ambiente \env{array}, le varianti \texttt{-ed} possono
accettare un argomento facoltativo \verb'[t]' o \verb'[b]' per
specificare il posizionamento verticale.

Costrutti di tipo \qq{casi} come il seguente sono comuni in matematica:
\begin{equation}\label{eq:C}
P_{r-j}=
  \begin{cases}
    0&  \text{se $r-j$ \`e dispari},\\
    r!\,(-1)^{(r-j)/2}&  \text{se $r-j$ \`e pari}.
  \end{cases}
\end{equation}
e nel pacchetto \pkg{amsmath} c'\`e un ambiente \env{cases} per
facilitarne la scrittura:
\begin{verbatim}
P_{r-j}=
  \begin{cases}
    0&  \text{se $r-j$ \`e dispari},\\
    r!\,(-1)^{(r-j)/2}&  \text{se $r-j$ \`e pari}.
  \end{cases}
\end{verbatim}
Osservare l'uso di \cn{text} (cfr.~\secref{text}) e della matematica
annidata nella precedente formula.

\section{Correggere il posizionamento dei tag}

Posizionare i numeri delle equazioni in blocchi multilinea pu\`o essere
un problema piuttosto complesso; gli ambienti del pacchetto
\pkg{amsmath} fanno il possibile per evitare di sovrascrivere le
equazioni con il numero, eventualmente spostando il numero pi\`u in
alto o pi\`u in basso su una riga diversa; le difficolt\`a nel calcolo
preciso del profilo di un'equazione possono talvolta risultare in
spostamenti inopportuni dei numeri: si pu\`o allora usare il comando
\cn{raisetag}, fornito proprio per regolare manualmente la posizione
verticale del numero dell'equazione attiva, se \`e stato spostato
dalla sua posizione normale: per spostare ad esempio un particolare
numero in alto di sei punti, si scrive |\raisetag{6pt}|; questo tipo
di correzione \`e un lavoro di precisione come le interruzioni di riga
o di pagina, e andrebbe quindi lasciato fino a quando il documento
non sia ormai quasi completo, poich\'e si rischierebbe altrimenti di
dover disfare e rifare una correzione pi\`u volte, per tenersi al passo
con i cambiamenti del contenuto del documento.

\section{Spaziatura verticale e interruzioni di pagina in
\emph{display} su pi\`u linee}

Come nel \latex/, si pu\`o usare il comando \cn{\\}|[|\<dimensione>|]|
per ottenere spazi verticale aggiuntivi in tutti gli ambienti di
equazioni a blocchi del pacchetto \pkg{amsmath}. Quando si usa il
pacchetto \pkg{amsmath}, le \ii{interruzioni di pagina} tra le righe
delle equazioni sono normalmente impedite; la filosofia di ci\`o \`e
che le interruzioni di pagina in questo tipo di materiale dovrebbero
essere scelto dall'autore nei vari casi; per ottenere un'interruzione
di pagina in una particolare equazione in \emph{display}, si pu\`o
usare il comando \cn{displaybreak}; il luogo migliore dove posizionare
un \cn{displaybreak} \`e immediatamente prima del \cn{\\} dove si vuole
che abbia effetto; come il comando \latex/ \cn{pagebreak},
\cn{displaybreak} accetta un argomento opzionale, tra 0 e 4, per
indicare la opportunit\`a dell'interruzione. |\displaybreak[0]|
significa \qq{\`e possibile interrompere qui}, senza incoraggiare
l'interruzione; \cn{displaybreak} senza argomento equivale a
|\displaybreak[4]| e forza l'interruzione.

Se si preferisce permettere le interruzioni di pagina dove capita,
anche in mezzo a una equazione su pi\`u linee, si pu\`o usare
\cn{allowdisplaybreaks}\texttt{[1]} nel preambolo del documento. Un
argomento 1\ndash 4 pu\`o essere usato per un controllo pi\`u fine:
|[1]| permette le interruzioni, evitandole tuttavia il pi\`u
possibile; valori 2,3,4 indicano una permissivit\`a maggiore. Quando
le interruzioni sono abilitate con \cn{allowdisplaybreaks}, il comando
\cn{\\*} pu\`o essere usato, come al solito, per impedire
un'interruzione di pagina a una ben precisa riga.

\begin{bfseries}
Nota: alcuni ambienti di equazioni racchiudono il loro contenuto in
una scatola indistruttibile, con la conseguenza che n\'e
\cn{displaybreak}, n\'e \cn{allowdisplaybreaks} avranno effetto su di
loro; tra questi ambienti vi sono \env{split}, \env{aligned},
\env{gathered} e \env{alignedat}.
\end{bfseries}

\section{Interrompere i \emph{display}}

Il comando \cn{intertext} pu\`o essere usato per una breve inserzione
di una o due righe di testo\index{frammenti di testo in matematica} in
un \emph{display} su pi\`u righe (cfr. il comando \cn{text} in
\secref{text}): la sua caratteristica principale \`e il mantenimento
dell'allineamento, cosa che non avverrebbe se si terminasse il blocco
per ricominciarlo pi\`u avanti. \cn{intertext} pu\`o comparire solo
dopo un comando \cn{\\} o \cn{\\*}. Notare la posizione della parola
\qq{e} in questo esempio.
\begin{align}
A_1&=N_0(\lambda;\Omega')-\phi(\lambda;\Omega'),\\
A_2&=\phi(\lambda;\Omega')-\phi(\lambda;\Omega),\\
\intertext{e}
A_3&=\mathcal{N}(\lambda;\omega).
\end{align}
\begin{verbatim}
\begin{align}
A_1&=N_0(\lambda;\Omega')-\phi(\lambda;\Omega'),\\
A_2&=\phi(\lambda;\Omega')-\phi(\lambda;\Omega),\\
\intertext{e}
A_3&=\mathcal{N}(\lambda;\omega).
\end{align}
\end{verbatim}

\section{Numerazione delle equazioni}

\subsection{Gerarchia della numerazione}
Con il \latex/ se si vogliono numerare le equazioni secondo le
sezioni\mdash cio\`e, con numeri di equazione tipo (1.1), (1.2), \dots,
(2.1), (2.2), \dots, nelle sezioni 1, 2, e cos\`{\i} via\mdash
bisognava ridefinire \cn{theequation} come suggerito nel manuale del
\latex/ \cite[\S6.3, \S C.8.4]{lamport}:
\begin{verbatim}
\renewcommand{\theequation}{\thesection.\arabic{equation}}
\end{verbatim}

Ci\`o funziona piuttosto bene, tranne per il fatto che il contatore
delle equazioni non viene reimpostato a zero all'inizio di un nuovo
capitolo o sezione, a meno di non farlo manualmente con
\cn{setcounter}; per facilitare il procedimento, il pacchetto
\pkg{amsmath} fornisce il comando\index{equazioni, numeri
delle!gerarchia} \cn{numberwithin}. Per legare la numerazione delle
equazioni alla numerazione delle sezioni, con reimpostazione
automatica dei contatori, si pu\`o usare
\begin{verbatim}
\numberwithin{equation}{section}
\end{verbatim}
Come suggerito dal nome, il comando \cn{numberwithin} pu\`o essere
applicato a qualunque contatore, non solo al contatore
\texttt{equation}.

\subsection{Riferimenti incrociati ai numeri delle equazioni}

Per facilitare i riferimenti incrociati alle equazioni, \`e stato
creato il comando \cn{eqref}\index{equazioni, numeri delle!riferimenti
incrociati}, che fornisce automaticamente le parentesi attorno al
numero: cos\`{\i}, mentre \verb'\ref{abc}' produce 3.2,
\verb'\eqref{abc}' produce (3.2).

\subsection{Numerazione subordinata}

Il pacchetto \pkg{amsmath} fornisce anche un ambiente
\env{subequations}\index{equazioni, numeri delle!numerazione delle
subordinate} per facilitare la numerazione delle equazioni di un
gruppo con uno schema subordinato; ad esempio,
\begin{verbatim}
\begin{subequations}
...
\end{subequations}
\end{verbatim}
fa in modo che tutte le equazioni numerate in quella parte del
documento vengano numerate con (4.9a) (4.9b) (4.9c) \dots, se la
precedente equazione aveva numero (4.8). Un comando \cn{label} subito
dopo \verb/\begin{subequations}/ produrr\`a un \cn{ref} al numero
genitore 4.9, non a 4.9a; i contatori usati dall'ambiente
\env{subequations} sono \verb/parentequation/ e \verb/equation/;
\cn{addtocounter}, \cn{setcounter}, \cn{value} etc.\ possono essere
applicati come al solito ai nomi di questi contatori; per ottenere
qualcosa di diverso dalle lettere minuscole per i numeri delle
subordinate, si usa il metodo standard \latex/ per cambiare lo stile
di numerazione \cite[\S6.3, \S C.8.4]{lamport}. Ad esempio, ridefinendo
\cn{theequation} come segue fornisce numeri romani.
\begin{verbatim}
\begin{subequations}
\renewcommand{\theequation}{\theparentequation \roman{equation}}
...
\end{verbatim}

%%%%%%%%%%%%%%%%%%%%%%%%%%%%%%%%%%%%%%%%%%%%%%%%%%%%%%%%%%%%%%%%%%%%%%%%
%% FcDC
\chapter{Varie funzionalit\`{a} matematiche}

\section{Matrici}\label{ss:matrix}

Il pacchetto \pkg{amsmath} fornisce qualche ambiente per le
matrici\index{matrici} oltre al fondamentale  ambiente \env{array} del
\latex/. Gli ambienti \env{pmatrix}, \env{bmatrix}, \env{Bmatrix},
\env{vmatrix} e \env{Vmatrix} hanno come delimitatori rispettivamente
$(\,)$, $[\,]$, $\lbrace\,\rbrace$, $\lvert\,\rvert$, $\lVert\,\rVert$;
per coerenza con la nomenclatura viene fornito anche un ambiente
\env{matrix} senza delimitatori. Questo pu\`o sembrare superfluo,
vista la presenza dell'ambiente \env{array}, ma ci\`o non \`e vero;
infatti tutti gli ambienti per matrici utilizzano una spaziatura
orizzontale pi\`u economica di quella generosa messa a disposizione
dall'ambiente \env{array}. Inoltre, diversamente dall'ambiente
\env{array}, non si devono specificare i parametri relativi alle
colonne in nessuno degli ambienti per matrici; di \emph{default} si possono
avere fino a 10 colonne centrate.%
\footnote{%%%%%%%%%%%%%%%%%%%%%%%%%%%%%%%%%%%%%%%%%%%%%%%%%%%%%%%%%%%%%%
In dettaglio: Il massimo numero di colonne in una matrice \`e indicato
dal contatore |MaxMatrixCols| (valore normale=10), che si pu\`o
cambiare con i comandi \latex/ \cn{setcounter} o \cn{addcounter}.
}\space%%%%%%%%%%%%%%%%%%%%%%%%%%%%%%%%%%%%%%%%%%%%%%%%%%%%%%%%%%%%%%%%%
(Per ottenere l'allineamento a destra o a sinistra in una colonna, oppure
per qualsiasi altro formato speciale, \`e necessario utilizzare
\env{array})

Per ottenere una piccola matrice adatta al testo, \`e disponibile
l'ambiente \env{smallmatrix} (es:
\begin{math}
\bigl( \begin{smallmatrix}
  a&b\\ c&d
\end{smallmatrix} \bigr)
\end{math})
che \`e pi\`u adatta di qualsiasi altra matrice a entrare in una riga
di testo. Devono essere comunque forniti i delimitatori: non ci sono le
versioni |p|,|b|,|B|,|v|,|V| di \env{smallmatrix}. L'esempio qua sopra
\`e stato prodotto da
\begin{verbatim}
\bigl( \begin{smallmatrix}
  a&b\\ c&d
\end{smallmatrix} \bigr)
\end{verbatim}

\cn{hdotsfor}|{|\<numero>|}| produce una riga di punti in una matrice
\index{matrici!puntini}\index{puntini!nelle matrici}\index{punti|see{puntini}}%
larga tante colonne quanto il numero passato come argomento. Per
esempio,
\begin{center}
\begin{minipage}{.3\columnwidth}
\noindent$\begin{matrix} a&b&c&d\\
e&\hdotsfor{3} \end{matrix}$
\end{minipage}%
\qquad
\begin{minipage}{.45\columnwidth}
\begin{verbatim}
\begin{matrix} a&b&c&d\\
e&\hdotsfor{3} \end{matrix}
\end{verbatim}
\end{minipage}%
\end{center}

La spaziatura dei punti pu\`o essere variata con l'utilizzo di un
opzione tra parentesi quadre, ad esempio, |\hdotsfor[1.5]{3}|. Il
numero racchiuso dalle parentesi funge da moltiplicatore (il valore
normale \`e 1.0)

\begin{equation}\label{eq:D}
\begin{pmatrix} D_1t&-a_{12}t_2&\dots&-a_{1n}t_n\\
-a_{21}t_1&D_2t&\dots&-a_{2n}t_n\\
\hdotsfor[2]{4}\\
-a_{n1}t_1&-a_{n2}t_2&\dots&D_nt\end{pmatrix},
\end{equation}
\begin{verbatim}
\begin{pmatrix} D_1t&-a_{12}t_2&\dots&-a_{1n}t_n\\
-a_{21}t_1&D_2t&\dots&-a_{2n}t_n\\
\hdotsfor[2]{4}\\
-a_{n1}t_1&-a_{n2}t_2&\dots&D_nt\end{pmatrix}
\end{verbatim}


\section{Comandi per la spaziatura matematica}

Il pacchetto \pkg{amsmath} estende l'insieme dei comandi di spaziatura
\index{spaziatura orizzontale!in matematica} come mostrato sotto. Sia
la forma intera che quella contratta di questi comandi sono robuste e
possono essere utilizzate anche al di fuori dell'ambiente matematico.

\begin{ctab}{lll|lll}
Abbrev.&Forma intera& Esempio &Abbrev.&Forma intera&Esempio\\
\hline
\vstrut{2.5ex}
& no space& \spx{}& & no space & \spx{}\\
\cn{\,}& \cn{thinspace}& \spx{\,}&
  \cnbang& \cn{negthinspace}& \spx{\!}\\
\cn{\:}& \cn{medspace}& \spx{\:}&
  & \cn{negmedspace}& \spx{\negmedspace}\\
\cn{\;}& \cn{thickspace}& \spx{\;}&
  & \cn{negthickspace}& \spx{\negthickspace}\\
& \cn{quad}& \spx{\quad}\\
& \cn{qquad}& \spx{\qquad}
\end{ctab}
Per il maggior controllo possibile sulla spaziatura matematica \`e
possibile utilizzare \cn{mspace} e le `unit\`{a} matematiche';
un'unit\`{a} matematica o |mu| \`e uguale a 1/18esimo. Per avere un
\cn{quad} negativo si deve scrivere |\mspace{-18.0mu}|.


\section{Punti}
Non esiste un consenso generale per quanto riguarda il piazzamento dei
punti ellittici (a mezza riga o in fondo della riga) in vari contesti.
La cosa pu\`o quindi essere considerata una questione di gusto.
Utilizzando i comandi orientati verso la semantica
\begin{itemize}
\item \cn{dotsc} per \qq{punti con virgole}
\item \cn{dotsb} per \qq{punti con operazioni/relazioni binarie}
\item \cn{dotsm} per \qq{punti con moltiplicazioni}
\item \cn{dotsi} per \qq{punti con integrali}
\item \cn{dotso} per \qq{altri tipi} (nessuno dei precedenti)
\end{itemize}
invece di \cn{ldots} e \cn{cdots}, \`e possibile adattare a varie
convenzioni un documento ``al volo'', nel caso che (per esempio)
dovendo pubblicare tale documento, l'editore insista nel seguire le
tradizioni della casa. Il trattamento predefinito a seconda delle
situazioni segue le convenzioni dell'American Mathematical Society:
\begin{center}
\begin{tabular}{@{}l@{}l@{}}
\begin{minipage}[t]{.54\textwidth}
\begin{verbatim}
Abbiamo quindi la serie $A_1, A_2,
\dotsc$, la somma di regioni $A_1
+A_2 +\dotsb $, il prodotto
ortogonale $A_1 A_2 \dotsm $, e
l'integrale infinito
\[\int_{A_1}\int_{A_2}\dotsi.\]
\end{verbatim}
\end{minipage}
&
\begin{minipage}[t]{.45\textwidth}
\noindent
Abbiamo quindi la serie $A_1, A_2,
\dotsc$, la somma di regioni $A_1
+A_2 +\dotsb $, il prodotto
ortogonale $A_1 A_2 \dotsm $, e
l'integrale infinito
\[\int_{A_1}\int_{A_2}\dotsi.\]
\end{minipage}
\end{tabular}
\end{center}

\section{Trattini senza interruzioni}
Viene fornito il comando \cn{nobreakdash} per eliminare la
possibilit\`{a} che avvenga un'interruzione di linea dopo un trattino.
Ad esempio scrivendo `pagine 1\ndash 9'  come |pagine 1\nobreakdash 9|
non occorrer\`{a} mai un'interruzione di linea tra il trattino e il 9.
\`E possibile utilizzare \cn{nobreakdash} anche per prevenire
sillabazioni indesiderate in combinazioni tipo |$p$-adico|. Per un
utilizzo frequente \`e consigliato fare delle abbreviazioni; ad
esempio

\begin{verbatim}
\newcommand{\p}{$p$\nobreakdash}% per "\p-adico"
\newcommand{\Ndash}{\nobreakdash--}% per "pagine 1\Ndash 9"
%    Per "\n dimensionale" ("n-dimensionale"):
\newcommand{\n}[1]{$n$\nobreakdash-\hspace{0pt}}
\end{verbatim}
L'ultimo esempio mostra come impedire un'interruzione di linea dopo il
trattino ma permette la corretta sillabazione delle parole
seguenti.(Basta aggiungere un spazio di dimensione zero dopo il
trattino.)


\section{Accenti in matematica}

Nel \latex/ ordinario, il piazzamento del secondo accento negli accenti
matematici doppi \`e spesso mediocre; con il pacchetto \pkg{amsmath}
si migliora notevolmente il piazzamento del secondo accento:
$\hat{\hat{A}}$ (\cn{hat}|{\hat{A}}|).

Sono disponibili i  comandi \cn{dddot} e \cn{dddddot} per produrre
accenti tripli e quadrupli in aggiunta a \cn{dot} e \cn{ddot} presenti
nel \latex/.

Per ottenere un carattere di tilde o di cappello come apice, si deve
caricare il pacchetto \pkg{amsxtra} e utilizzare i comandi \cn{sphat}
o \cn{sptilde}, l'utilizzo \`e \verb'A\sphat' (notare l'assenza del
carattere \verb'^'). Per piazzare un simbolo arbitrario in posizione
di accento matematico o per ottenere accenti come pedici, consultare
il pacchetto \pkg{accents} di Javier Bezos.

\section{Radici}

Nel \latex/ ordinario il piazzamento degli indici delle radici a volte
non \`e buono:  $\sqrt[\beta]{k}$ (|\sqrt||[\beta]{k}|), nel pacchetto
 \pkg{amsmath} i comandi  \cn{leftroot} e \cn{uproot} permettono di aggiustare
la posizione della radice:

\begin{verbatim}
  \sqrt[\leftroot{-2}\uproot{2}\beta]{k}
\end{verbatim}
muove la beta in alto e verso destra:
$\sqrt[\leftroot{-2}\uproot{2}\beta]{k}$. L'argomento negativo di
\cn{leftroot} muove $\beta$ verso destra; le unit\`{a} sono piccole, e
quindi adatte per questo tipo di aggiustamenti.

\section{Formule in riquadro}

Il comando \cn{boxed} costruisce un riquadro attorno al suo argomento,
come \cn{fbox}, eccetto che i contenuti dei riquadri sono in modo
matematico:

\begin{equation}
\boxed{\eta \leq C(\delta(\eta) +\Lambda_M(0,\delta))}
\end{equation}
\begin{verbatim}
  \boxed{\eta \leq C(\delta(\eta) +\Lambda_M(0,\delta))}
\end{verbatim}

\section{Frecce in alto e in basso}
Il \latex/ di base fornisce i comandi \cn{overrightarrow} e
\cn{overleftarrow}; il paccheto \pkg{amsmath} fornisce altri comandi
per frecce in alto e in basso per estendere l'insieme di base:

\begin{tabbing}
\qquad\=\ncn{overleftrightarrow}\qquad\=\kill
\> \cn{overleftarrow}    \> \cn{underleftarrow} \+\\
   \cn{overrightarrow}    \> \cn{underrightarrow} \\
   \cn{overleftrightarrow}\> \cn{underleftrightarrow}
\end{tabbing}

\section{Frecce estendibili}
\cn{xleftarrow} e \cn{xrightarrow} producono frecce
\index{frecce!estendibili} che si estendono automaticamente per accomodare
grandezze inusuali di apici e pedici. Questi comandi prendono un argomento
facoltativo (il pedice) e un argomento obbligatorio (l'apice, possibilmente
anche vuoto):

\begin{equation}
A\xleftarrow{n+\mu-1}B \xrightarrow[T]{n\pm i-1}C
\end{equation}
\begin{verbatim}
  \xleftarrow{n+\mu-1}\quad \xrightarrow[T]{n\pm i-1}
\end{verbatim}

\section{Attaccare simboli ad altri simboli}

\latex/ fornisce  \cn{stackrel} per piazzare un apice
\index{esponenti e deponenti} sopra una relazione binaria.
Nel pacchetto \pkg{amsmath} ci sono comandi pi\`u generali,
\cn{overset} e \cn{underset} che possono essere utilizzati per
piazzare un simbolo sopra o sotto un altro simbolo, ogni qualvolta che
si trova una relazione binaria  o qualcos'altro.
L'input |\overset{*}{X}| piazza un $*$ della dimensione
di un apice
sopra la $X$: $\overset{*}{X}$; \cn{underset} \`e l'analogo
per aggiungere un simbolo in basso.
Controllare anche la descrizione di \cn{sideset} in \secref{sideset}.

\section{Frazioni e costrutti correlati}

\subsection{I comandi \cn{frac}, \cn{dfrac}, e \cn{tfrac}}

Il comando \cn{frac}, che fa parte dell'insieme dei comandi dei base
del \latex/,\index{frazioni} prende due argomenti\mdash numeratore
e denominatore\mdash e compone questi nella classica forma di una frazione.
Il pacchetto \pkg{amsmath} fornisce anche \cn{dfrac} e \cn{tfrac} come
convenienti abbreviazioni per |{\displaystyle\frac| |...| |}|
e\index{textstyle@\cn{textstyle}}\relax
\index{displaystyle@\cn{displaystyle}} |{\textstyle\frac| |...| |}|.

\begin{equation}
\frac{1}{k}\log_2 c(f)\quad\tfrac{1}{k}\log_2 c(f)\quad
\sqrt{\frac{1}{k}\log_2 c(f)}\quad\sqrt{\dfrac{1}{k}\log_2 c(f)}
\end{equation}
\begin{verbatim}
\begin{equation}
\frac{1}{k}\log_2 c(f)\;\tfrac{1}{k}\log_2 c(f)\;
\sqrt{\frac{1}{k}\log_2 c(f)}\;\sqrt{\dfrac{1}{k}\log_2 c(f)}
\end{equation}
\end{verbatim}

\subsection{I comandi \cn{binom}, \cn{dbinom}, e \cn{tbinom}}

Per espressioni binomiali\index{binomiali} tipo $\binom{n}{k}$
\pkg{amsmath} fornisce \cn{binom}, \cn{dbinom} e \cn{tbinom}:
\begin{equation}
2^k-\binom{k}{1}2^{k-1}+\binom{k}{2}2^{k-2}
\end{equation}
\begin{verbatim}
2^k-\binom{k}{1}2^{k-1}+\binom{k}{2}2^{k-2}
\end{verbatim}

\subsection{Il comando \cn{genfrac}}

Le capacit\`{a} di \cn{frac}, \cn{binom}, e delle loro varianti sono
sintetizzate dal comando generale \cn{genfrac}, che richiede sei
argomenti. Gli ultimi due corrispondono al numeratore e denominatore di
\cn{frac}, i primi due sono delimitatori opzionali (come visto in
\cn{binom}); il terzo riguarda lo spessore della linea (\cn{binom}
utilizza questo per impostare lo spessore della linea di frazione a 0
\mdash cio\`e invisibile) e il quarto argomento cambia lo stile
matematico: valori interi tra 0 e 3 selezionano rispettivamente
\cn{displaystyle}, \cn{textstyle}, \cn{scriptstyle} e
\cn{scriptscriptstyle}. Se il terzo argomento viene lasciato vuoto, lo
spessore della linea viene impostato per convenzione a `normale'.

\begin{center}\begin{minipage}{.85\columnwidth}
\raggedright \normalfont\ttfamily \exhyphenpenalty10000
\newcommand{\ma}[1]{%
  \string{{\normalfont\itshape#1}\string}\penalty9999 \ignorespaces}
\string\genfrac \ma{delim-sx} \ma{delim-dx} \ma{spessore}
\ma{stile} \ma{numeratore} \ma{denominatore}
\end{minipage}\end{center}
Per completezza viene mostrato come \cn{frac}, \cn{tfrac} e \cn{binom}
potrebbero essere definiti.
\begin{verbatim}
\newcommand{\frac}[2]{\genfrac{}{}{}{}{#1}{#2}}
\newcommand{\tfrac}[2]{\genfrac{}{}{}{1}{#1}{#2}}
\newcommand{\binom}[2]{\genfrac{(}{)}{0pt}{}{#1}{#2}}
\end{verbatim}
Se si utilizza ripetutamente \cn{genfrac} in un documento per una
particolare notazione, sarebbe di grande comodit\`a per lo scrittore
(e l'editore) definire un'abbreviazione significativa per questa
notazione, come \cn{frac} e \cn{binom} illustrate sopra. I comandi
primitivi generali per le frazioni \cs{over}, \cs{overwithdelims},
\cs{atop}, \cs{atopwithdelims}, \cs{above} e \cs{abovewithdelims}
producono messaggi di avvertimento se utilizzati in congiunzione con
\pkg{amsmath}, per ragioni discusse in \fn{technote.tex}.

\section{Frazioni continue}

La frazione continua\index{frazioni continue}
\begin{equation}
\cfrac{1}{\sqrt{2}+
 \cfrac{1}{\sqrt{2}+
  \cfrac{1}{\sqrt{2}+\cdots
}}}
\end{equation}
si ottiene digitando
{\samepage
\begin{verbatim}
\cfrac{1}{\sqrt{2}+
 \cfrac{1}{\sqrt{2}+
  \cfrac{1}{\sqrt{2}+\dotsb
}}}
\end{verbatim}
}% End of \samepage
Questo produce un  risultato visivamente migliore di quello ottenuto
con l'utilizzo prolungato di \cn{frac}. Il piazzamento a destra o a sinistra
di qualsiasi dei numeratori \`e ottenuto utilizzando \cn{cfrac}|[l]| o
\cn{cfrac}|[r]| invece di \cn{cfrac}.

\section{Opzioni smash}

Il comando \cn{smash} viene utilizzato per comporre una sottoformula
con effettiva larghezza e profondit\`{a} zero; questo a volte rimane
utile dovendo aggiustare la posizione della sottoformula rispetto ai
simboli adiacenti. Con il pacchetto \pkg{amsmath}, \cn{smash} ha
argomenti opzionali |t| e |b|, perch\'e occasionalmente \`e
vantaggioso essere capaci di \qq{appiattire} solo l'altezza o la
profondit\`a, conservando l'altra. Ad esempio, quando simboli di
radicali sono posizionati o dimensionati in modo diverso a causa delle
differenze di altezza e larghezza dei loro contenuti, \cn{smash}
pu\`o essere applicato per rendere il tutto pi\`u consistente.
Confrontare $\sqrt{x}+\sqrt{y}+\sqrt{z}$ con
$\sqrt{x}+\sqrt{\smash[b]{y}}+\sqrt{z}$, dove l'ultimo \`e stato
prodotto con \verb"$\sqrt{x}" \verb"+"
\verb"\sqrt{"\5\verb"\smash[b]{y}}" \verb"+" \verb"\sqrt{z}$".

\section{Delimitatori}

\subsection{Dimensione dei delimitatori}\label{bigdel}

Il dimensionamento automatico dei delimitatori fatto da \cn{left} e
\cn{right} ha due limitazioni: innanzi tutto, viene applicato
meccanicamente per produrre delimitatori abbastanza grandi da
ricoprire il pi\`u grande oggetto contenuto in essi, e inoltre,
l'intervallo delle dimensioni non \`e neanche approssimativamente
continuo, ma ha dei salti abbastanza grandi. Questo significa che un
frammemto matematico infinitesimamente troppo grande per una data
grandezza del delimitatore prender\`{a} la misura successiva, un salto
di 3pt o simile in un testo a grandezza normale. Ci sono due o tre
situazioni dove la grandezza del delimitatore viene comunemente
aggiustata, utilizzando un insieme di comandi che contengono `big' nei
loro nomi.

\begin{ctab}{l|llllll}
Dim. del&
  dim. del& \ncn{left}& \ncn{bigl}& \ncn{Bigl}& \ncn{biggl}& \ncn{Biggl}\\
delimitatore&
  testo& \ncn{right}& \ncn{bigr}& \ncn{Bigr}& \ncn{biggr}& \ncn{Biggr}\\
\hline
Risultato\vstrut{5ex}&
  $\displaystyle(b)(\frac{c}{d})$&
  $\displaystyle\left(b\right)\left(\frac{c}{d}\right)$&
  $\displaystyle\bigl(b\bigr)\bigl(\frac{c}{d}\bigr)$&
  $\displaystyle\Bigl(b\Bigr)\Bigl(\frac{c}{d}\Bigr)$&
  $\displaystyle\biggl(b\biggr)\biggl(\frac{c}{d}\biggr)$&
  $\displaystyle\Biggl(b\Biggr)\Biggl(\frac{c}{d}\Biggr)$
\end{ctab}
Il primo tipo di situazione \`e un operatore cumulativo con limiti
sopra e sotto. Con \cn{left} e \cn{right} i delimitatori di solito
diventano pi\`u larghi del necessario, e utilizzando invece
le dimensioni |Big| o |bigg| si ottengono risultati migliori.
\begin{equation*}
\left[\sum_i a_i\left\lvert\sum_j x_{ij}\right\rvert^p\right]^{1/p}
\quad\text{contro}\quad
\biggl[\sum_i a_i\Bigl\lvert\sum_j x_{ij}\Bigr\rvert^p\biggr]^{1/p}
\end{equation*}
\begin{verbatim}
\biggl[\sum_i a_i\Bigl\lvert\sum_j x_{ij}\Bigr\rvert^p\biggr]^{1/p}
\end{verbatim}
Il secondo tipo di situazione \`e un ammasso di coppie  di delimitatori
dove \cn{left} e \cn{right} rendono le loro grandezze uguali
(dato che questo risulta adeguato per racchiudere tutto il materiale)
ma l'effetto desiderato \`e quello di avere alcuni delimitatori
con grandezza maggiore, per rendere l'annidamento pi\`u semplice da
vedere.
\begin{equation*}
\left((a_1 b_1) - (a_2 b_2)\right)
\left((a_2 b_1) + (a_1 b_2)\right)
\quad\text{contro}\quad
\bigl((a_1 b_1) - (a_2 b_2)\bigr)
\bigl((a_2 b_1) + (a_1 b_2)\bigr)
\end{equation*}
\begin{verbatim}
\left((a_1 b_1) - (a_2 b_2)\right)
\left((a_2 b_1) + (a_1 b_2)\right)
\quad\text{versus}\quad
\bigl((a_1 b_1) - (a_2 b_2)\bigr)
\bigl((a_2 b_1) + (a_1 b_2)\bigr)
\end{verbatim}
Il terzo tipo di situazione \`e un oggetto di dimensione leggermente
elevata nel testo libero, %running rext??
come $\left\lvert\frac{b'}{d'}\right\rvert$, dove i delimitatori
prodotti da \cn{left} e \cn{right} causano un'eccessiva altezza
della linea. In questo caso \ncn{bigl} e \ncn{bigr}\index{big@\cn{big},
\cn{Big}, \cn{bigg}, \dots\ delimiters} possono essere utilizzati
per produrre delimitatori che sono leggermente pi\`u grandi della
dimensione di base, ma che comunque rientrano all'interno della normale
spaziatura della linea: $\bigl\lvert\frac{b'}{d'}\bigr\rvert$.
Nel \latex/ ordinario i delimitatori \ncn{big}, \ncn{bigg}, \ncn{Big},
e \ncn{Bigg} non sono scalati in modo opportuno per tutto il ``range''
delle dimensioni dei \emph{font} \latex/, con il pacchetto \pkg{amsmath}
invece lo sono.

\subsection{Notazioni per la barra verticale}

Il pacchetto \pkg{amsmath} fornisce i comandi \cn{vert}, \cn{rvert},
\cn{lVert}, \cn{rVert} (confrontare \cn{langle} e \cn{rangle}) per
indirizzare il problema del sovraccarico per il carattere di barra
verticale \qc{\|}. Questo carattere viene utilizato nei documenti
\latex/ per una grande variet\`{a} di oggetti matematici: la relazione
`divide' in un' espressione della teoria dei numeri tipo $p\vert q$,
oppure l'operazione di  valore assoluto $\lvert z\rvert$, oppure la
condizione `tale che' nella notazione insiemistica, oppure la
notazione `valutato in' $f_\zeta(t)\bigr\rvert_{t=0}$. La
molteplicit\`{a} degli utilizzi non \`e essa stessa un male, ci\`o
che non va bene comunque \`e il fatto che non tutti questi vari
oggetti ottengono lo stesso trattamento tipografico e che le complesse
capcit\`a discriminatorie di un lettore colto non possono essere
replicate in un computer che deve elaborare documenti matematici. Si
raccomanda quindi che ci sia una corrispondenza uno-a-uno in ogni
documento tra il carattere di barra verticale \qc{\|} e una scelta
notazione matematica, analogamente per il comando di doppia barra
\cn{\|}. Questo immediatamente esclude l'utilizzo di \qc{|} e
\ncn{\|}\index{"|@\cn{"\"|}} come delimitatori, dato che i delimitatori
destri e sinistri %
hanno usi distinti, non correlati allo stesso modo con simboli
adiacenti
%delimiters are distinct usages that do not relate in the same way to
%adjacent symbols;
si raccomanda la pratica di definire nel preambolo del documento
comandi adatti a ogni utilizzo di coppie di delimitatori con simboli
di barre verticali:

\begin{verbatim}
\providecommand{\abs}[1]{\lvert#1\rvert}
\providecommand{\norm}[1]{\lVert#1\rVert}
\end{verbatim}
al che il documento dovrebbe contenere |\abs{z}| per produrre
 $\lvert z\rvert$ e |\norm{v}| per produrre $\lVert v\rVert$.
%%%%% fine FCdC

%%%%% OdB
\chapter{Nomi per gli operatori}

\section{Come definire nuovi nomi di operatori}\label{s:opname}

Le funzioni matematiche\index{nomi di operatori}\index{nomi di
funzioni|see{nomi di operatori}} come $\log$, $\sin$, e $\lim$ sono
per tradizione stampate in tondo per renderne pi\`u immediata la
visibilit\`a rispetto alle variabili matematiche di un carattere, che
sono stampate in stile matematico corsivo. Le pi\`u comuni hanno nomi
predefiniti, \cn{log}, \cn{sin}, \cn{lim}, e cos\`\i{} via, ma se ne
introducono continuamente di nuovi nelle pubblicazioni relative alla
matematica, pertanto il pacchetto \pkg{amsmath} fornisce un metodo
generale per definire nuovi `nomi di operatori'. Per definire una
funzione matematica \ncn{xxx} che si presenti come \cn{sin}, si
scriver\`a
\begin{verbatim}
\DeclareMathOperator{\xxx}{xxx}
\end{verbatim}
Come conseguenza, l'utilizzo di \ncn{xxx} produrr\`a {\upshape xxx}
nel corrispondente \emph{font} e automaticamente aggiunger\`a l'adeguata
spaziatura\index{spaziatura orizzontale!attorno ai nomi di operatori}
su entrambi i lati quando necessario, in maniera tale da ottenere
$A\xxx B$ invece di $A\mathrm{xxx}B$. Nel secondo argomento di
\cn{DeclareMathOperator} (il testo con il nome), \`e prevalente una
modalit\`a pseudo-testuale: il carattere di sillabazione \qc{\-}
verr\`a stampato come un trattino di sillabazione piuttosto che come
un segno meno e un asterisco \qc{\*} risulter\`a stampato come un
asterisco in alto piuttosto che come un asterisco centrato di tipo
matematico (confrontare \textit{a}-\textit{b}*\textit{c} e $a-b*c$.);
d'altra parte il testo contenente il nome \`e stampato in modalit\`a
matematica, ad es. in modo tale da poter ivi usare pedici e apici.

Se il nuovo operatore dovesse esser dotato di pedici e apici
posizionati alla maniera dei
`limiti', al di sopra e al di sotto come per $\lim$, $\sup$, o $\max$,
si user\`a
la forma \qc{\*} del comando \cn{DeclareMathOperator}:
\begin{verbatim}
\DeclareMathOperator*{\Lim}{Lim}
\end{verbatim}
Fare inoltre riferimento alla trattazione del posizionamento dell'indice
nel paragrafo~\ref{subplace}.

I seguenti nomi di operatori sono predefiniti:
\begin{ctab}{rlrlrlrl}
\cn{arccos}& $\arccos$ &\cn{deg}& $\deg$ &      \cn{lg}& $\lg$ &        \cn{projlim}& $\projlim$\\
\cn{arcsin}& $\arcsin$ &\cn{det}& $\det$ &      \cn{lim}& $\lim$ &      \cn{sec}& $\sec$\\
\cn{arctan}& $\arctan$ &\cn{dim}& $\dim$ &      \cn{liminf}& $\liminf$ &\cn{sin}& $\sin$\\
\cn{arg}& $\arg$ &      \cn{exp}& $\exp$ &      \cn{limsup}& $\limsup$ &\cn{sinh}& $\sinh$\\
\cn{cos}& $\cos$ &      \cn{gcd}& $\gcd$ &      \cn{ln}& $\ln$ &        \cn{sup}& $\sup$\\
\cn{cosh}& $\cosh$ &    \cn{hom}& $\hom$ &      \cn{log}& $\log$ &      \cn{tan}& $\tan$\\
\cn{cot}& $\cot$ &      \cn{inf}& $\inf$ &      \cn{max}& $\max$ &      \cn{tanh}& $\tanh$\\
\cn{coth}& $\coth$ &    \cn{injlim}& $\injlim$ &\cn{min}& $\min$\\
\cn{csc}& $\csc$ &      \cn{ker}& $\ker$ &      \cn{Pr}& $\Pr$
\end{ctab}
\begin{ctab}{rlrl}
\cn{varlimsup}&  $\displaystyle\varlimsup$&
  \cn{varinjlim}&  $\displaystyle\varinjlim$\\
\cn{varliminf}&  $\displaystyle\varliminf$&
  \cn{varprojlim}& $\displaystyle\varprojlim$
\end{ctab}

\`E inoltre disponibile un comando \cn{operatorname}, in modo tale che
l'uso di
\begin{verbatim}
\operatorname{abc}
\end{verbatim}
in una formula matematica equivalga all'uso di \ncn{abc} definito da
\cn{DeclareMathOperator}; questo pu\`o in certi casi essere utile per
realizzare notazioni pi\`u complesse o per altri scopi. (Usare la variante
\cn{operatorname*} per ottenere i limiti.)

\section{\cn{mod} e i suoi affini}

I comandi \cn{mod}, \cn{bmod}, \cn{pmod}, \cn{pod} sono forniti per
affrontare le particolari convenzioni di spaziatura della notazione
\qq{mod}. In  \latex/ sono disponibili \cn{bmod} e
\cn{pmod}, ma con il pacchetto \pkg{amsmath}
la spaziatura di \cn{pmod} sar\`a regolata a un valore inferiore se
viene usato in una formula in modalit\`a \emph{non-display}.
 \cn{mod} e \cn{pod} sono varianti di
\cn{pmod} preferite da alcuni autori; \cn{mod} omette le parentesi,
mentre \cn{pod} omette il \qq{mod} e mantiene le parentesi.
\begin{equation}
\gcd(n,m\bmod n);\quad x\equiv y\pmod b;
\quad x\equiv y\mod c;\quad x\equiv y\pod d
\end{equation}
\begin{verbatim}
\gcd(n,m\bmod n);\quad x\equiv y\pmod b;
\quad x\equiv y\mod c;\quad x\equiv y\pod d
\end{verbatim}
%%%%% Fine OdB

%%%%%%%%%%%%%%%%%%%%%%%%%%%%%%%%%%%%%%%%%%%%%%%%%%%%%%%%%%%%%%%%%%%%%%%%
% by GD

\chapter{Il comando \cn{text}}\label{text}

Il principale utilizzo del comando \cn{text} consiste nello scrivere
parole o frasi\index{testo!frammenti di testo in matematica} in un
\emph{display}. Il suo comportamento \`e molto simile al comando
\latex/  \cn{mbox}, ma presenta un paio di vantaggi. Se si desidera
inserire una parola o una frase in un deponente \`e leggermente pi\`u
semplice digitare |..._{\text{parola o frase}}| piuttosto che il
comando equivalente \cn{mbox}: |..._{\mbox{\scriptsize| |parola| |o|
|frase}}|. L'altro vantaggio \`e nel suo nome, pi\`u descrittivo.
\begin{equation}
f_{[x_{i-1},x_i]} \text{ \`e monotona,}
\quad i = 1,\dots,c+1
\end{equation}
\begin{verbatim}
f_{[x_{i-1},x_i]} \text{ \`e monotona,}
\quad i = 1,\dots,c+1
\end{verbatim}

% by GD - end

% by LF
\chapter{Integrali e sommatorie}

\section{Deponenti ed esponenti su pi\`u righe}

Il comando \cn{substack} pu\`o essere usato per produrre un deponente o un
esponente su pi\`u righe:\index{deponenti ed esponenti!su pi\`u righe}\relax
\index{esponenti|see{deponenti ed esponenti}} per esempio
\begin{ctab}{ll}
\begin{minipage}[t]{.6\columnwidth}
\begin{verbatim}
\sum_{\substack{
         0\le i\le m\\
         0<j<n}}
  P(i,j)
\end{verbatim}
\end{minipage}
&
$\displaystyle
\sum_{\substack{0\le i\le m\\ 0<j<n}} P(i,j)$
\end{ctab}
Una forma un po' pi\`u generalizzata \`e l'ambiente \env{subarray} che
consente di specificare che ogni riga deve essere allineata a sinistra invece che
centrata, come in questo caso:
\begin{ctab}{ll}
\begin{minipage}[t]{.6\columnwidth}
\begin{verbatim}
\sum_{\begin{subarray}{l}
        i\in\Lambda\\ 0<j<n
      \end{subarray}}
 P(i,j)
\end{verbatim}
\end{minipage}
&
$\displaystyle
  \sum_{\begin{subarray}{l}
        i\in \Lambda\\ 0<j<n
      \end{subarray}}
 P(i,j)$
\end{ctab}

\section{Il comando \cn{sideset}}\label{sideset}

C'\`e anche un comando chiamato \cn{sideset}, per uno scopo abbastanza
particolare: porre dei simboli agli angoli di deponente ed
esponente\index{deponenti ed esponenti!su sommatorie} di un simbolo
operatorio di grandi dimensioni come $\sum$ o $\prod$. \emph{Nota:
questo comando non \`e pensato per essere applicato ad altro che a
simboli tipo sommatoria.} L'esempio principale \`e il caso in cui si
voglia porre un simbolo di primo su un simbolo di sommatoria. Se non
ci sono estremi sopra o sotto la sommatoria, si pu\`o semplicemente
usare \cn{nolimits}: ecco come appare
%%%%%%%%%%%%%%%%%%%%%%%%%%%%%%%%%%%%%%%%%%%%%%%%%%%%%%%%%%%%%%%%%%%%%%%%
|\sum\nolimits' E_n| in modo \emph{display}:
\begin{equation}
\sum\nolimits' E_n
\end{equation}
Se tuttavia si desidera non solo il segno di primo ma anche qualcosa sopra o
sotto il simbolo di sommatoria, non \`e cos\`\i{} facile\mdash invero, senza
\cn{sideset}, sarebbe proprio difficile. Con \cn{sideset}, si
pu\`o scrivere
\begin{ctab}{ll}
\begin{minipage}[t]{.6\columnwidth}
\begin{verbatim}
\sideset{}{'}
  \sum_{n<k,\;\text{$n$ odd}} nE_n
\end{verbatim}
\end{minipage}
&$\displaystyle
\sideset{}{'}\sum_{n<k,\;\text{$n$ odd}} nE_n
$
\end{ctab}
La coppia di parentesi graffe vuote si spiega con il fatto che
\cn{sideset} ha la possibilit\`a di porre uno o pi\`u simboli aggiuntivi a
ogni angolo di un simbolo operatorio di grandi dimensioni; per porre un asterisco in ciascun angolo di un
simbolo di prodotto, si potrebbe scrivere
\begin{ctab}{ll}
\begin{minipage}[t]{.6\columnwidth}
\begin{verbatim}
\sideset{_*^*}{_*^*}\prod
\end{verbatim}
\end{minipage}
&$\displaystyle
\sideset{_*^*}{_*^*}\prod
$
\end{ctab}

\section{Posizionamento di deponenti ed estremi}\label{subplace}

Il tipo di posizionamento predefinito per i deponenti dipende dal
simbolo base considerato. Per i simboli tipo sommatoria \`e usato il
posizionamento `displaylimits': quando un simbolo tipo sommatoria appare
in una formula in \emph{display}, deponente ed esponente sono posti nella posizione
`limits' sopra e sotto, ma in una formula nel testo sono invece posti
a lato, per evitare l'antiestetico e sprecato allargamento della distanza dalle
righe di testo adiacenti.
L'impostazione predefinita per i simboli tipo integrale \`e avere deponenti
ed esponenti sempre a lato, anche nelle formule in \emph{display}.
(Si veda la discussione su \opt{intlimits} e opzioni correlate nella
Sec.~\ref{options}.)

I nomi di operatore, come $\sin$ o $\lim$, possono avere il posizionamento
`displaylimits' o quello `limits' a seconda di come sono stati definiti. Gli operatori
pi\`u comuni sono definiti in base all'uso consueto in matematica.

I comandi \cn{limits} e \cn{nolimits} possono essere usati per modificare
il normale comportamento di un simbolo base:
\begin{equation*}
\sum\nolimits_X,\qquad \iint\limits_{A},
\qquad\varliminf\nolimits_{n\to \infty}
\end{equation*}
Per definire un comando i cui deponenti seguono lo
stesso comportamento `displaylimits' di \cn{sum}, si pu\`o aggiungere
\cn{displaylimits} in coda alla definizione. Quando ci sono
pi\`u istanze consecutive di \cn{limits}, \cn{nolimits}, o \cn{displaylimits},
l'ultima ha la priorit\`a.

\section{Simboli di integrale multiplo}

\cn{iint}, \cn{iiint}, e \cn{iiiint} producono pi\`u simboli di integrale
\index{integrali!multipli} con la spaziatura tra di loro opportunamente
corretta, sia in stile testo che \emph{display}. \cn{idotsint} estende
la stessa idea producendo due segni di integrale separati da tre punti.
\begin{gather}
\iint\limits_A f(x,y)\,dx\,dy\qquad\iiint\limits_A
f(x,y,z)\,dx\,dy\,dz\\
\iiiint\limits_A
f(w,x,y,z)\,dw\,dx\,dy\,dz\qquad\idotsint\limits_A f(x_1,\dots,x_k)
\end{gather}

% by LF - end
%%%%%%%%%%%%%%%%%%%%%%%%%%%%%%%%%%%%%%%%%%%%%%%%%%%%%%%%%%%%%%%%%%%%%%%%
% by GD

\chapter{Diagrammi commutativi}\label{s:commdiag}

Vari comandi, come quelli in \amstex/, per disegnare i diagrammi
commutativi sono disponibili separatamente nel pacchetto \pkg{amscd}.
Per i diagrammi commutativi di una certa complessit\`a, gli autori
dovranno necessariamente considerare pacchetti pi\`u estesi come
\pkg{kuvio} o \xypic/, ma per diagrammi semplici privi di
frecce\index{frecce!nei diagrammi commutativi} diagonali, i
comandi dell'\pkg{amscd} potrebbero rivelarsi pi\`u convenienti.
Di seguito vi \`e un esempio.
\begin{equation*}
\begin{CD}
S^{{\mathcal{W}}_\Lambda}\otimes T   @>j>>   T\\
@VVV                                    @VV{\End P}V\\
(S\otimes T)/I                  @=      (Z\otimes T)/J
\end{CD}
\end{equation*}
\begin{verbatim}
\begin{CD}
S^{{\mathcal{W}}_\Lambda}\otimes T   @>j>>   T\\
@VVV                                    @VV{\End P}V\\
(S\otimes T)/I                  @=      (Z\otimes T)/J
\end{CD}
\end{verbatim}
Nell'ambiente \env{CD}, i comandi |@>>>|, |@<<<|, |@VVV| e |@AAA|
disegnano, rispettivamente, le frecce a destra, a sinistra, verso il
basso e verso l'alto.
Per quanto riguarda le frecce orizzontali, il contenuto tra il primo
e il secondo simbolo |>| oppure |<| sar\`a inserito a esponente sulla
freccia, e il contenuto tra il secondo e il terzo simbolo sar\`a inserito
a deponente sotto la freccia.
Analogamente per le frecce verticali, il contenuto tra il primo e il
secondo oppure tra il secondo e il terzo dei simboli |A| o |V| sar\`a
inserito a \qq{margine} sinistro o destro della freccia.
I comandi |@=| e \verb'@|' tracciano rispettivamente una doppia linea
orizzontale e una verticale.
Il comando |@.| equivale a una \qq{freccia nulla} e pu\`o essere usato
al posto di una freccia visibile per espandere, dove sia necessario, una
matrice.

% by GD - end

%%%%%%%%%%%%%%%%%%%%%%%%%%%%%%%%%%%%%%%%%%%%%%%%%%%%%%%%%%%%%%%%%%%%%%%%
\chapter{Usare \emph{font} matematici}

\section{Introduzione}

Per informazioni pi\`u complete riguardo l'uso dei \emph{font} in \latex/,
consultate la guida ai \emph{font} del \latex/ (\fn{fntguide.tex}) o
il libro \booktitle{The \latex/ Companion} \cite{tlc}.  L'insieme di base
dei comandi per usare \emph{font} matematici\index{\emph{font} matematici}\relax
\index{simboli matematici|see{\emph{font} matematici}} nel \latex/ \`e
costituito da \cn{mathbf}, \cn{mathrm}, \cn{mathcal}, \cn{mathsf},
\cn{mathtt} e \cn{mathit}. Comandi aggiuntivi per \emph{font} matematici
come \cn{mathbb} per il \emph{blackboard-bold}, \cn{mathfrak} per il Fraktur
e \cn{mathscr} per l'Euler script sono disponibili attraverso i
pacchetti \pkg{amsfonts} e \pkg{euscript} (distribuiti separatamente).

\section{Uso consigliato per i comandi dei \emph{font} matematici}

Se ci si trova a usare frequentemente comandi per \emph{font} matematici nei
propri documenti, si potrebbero voler usare nomi pi\`u brevi, come
\ncn{mb} al posto di \cn{mathbf}.  Ovviamente non c'\`e nulla che
impedisca di farsi da soli tali abbreviazioni, usando i comandi
\cn{newcommand} appropriati.  D'altro canto, per il \latex/, offrire
comandi pi\`u brevi sarebbe addirittura un disservizio per gli autori,
poich\'e renderebbe meno ovvia un'alternativa molto migliore:
definire nomi di comandi personalizzati che si riferiscano ai nomi degli
oggetti matematici che a loro competono, piuttosto che ai nomi dei
\emph{font} che sono usati per distinguere tali oggetti.  Per esempio, se
si usa il grassetto per indicare i vettori, alla lunga sarebbe meglio definire
un comando `vector' al posto di un `math-bold':
\begin{verbatim}
  \newcommand{\vect}[1]{\mathbf{#1}}
\end{verbatim}
si pu\`o scrivere |\vect{a} + \vect{b}| per avere $\vect{a} +
\vect{b}$.
Se, mesi dopo aver cominciato il lavoro, si decide di usare il
grassetto per qualche altro scopo e di indicare i vettori con una
freccina sopra, si pu\`o fare il tutto semplicemente cambiando la
definizione di \ncn{vect}; altrimenti si sarebbero dovute rimpiazzare
tutte le occorrenze di \cn{mathbf} nel documento, eventualmente
persino controllandole una a una per vedere se si riferivano
effettivamente a un vettore.

Pu\`o essere utile anche assegnare nomi di comandi distinti per
differenti lettere dell'alfabeto di un particolare \emph{font}:
\begin{verbatim}
\DeclareSymbolFont{AMSb}{U}{msb}{m}{n}% oppure si usi il pacchetto amsfonts
\DeclareMathSymbol{\C}{\mathalpha}{AMSb}{"43}
\DeclareMathSymbol{\R}{\mathalpha}{AMSb}{"52}
\end{verbatim}
Queste righe definirebbero i comandi \cn{C} e \cn{R} in modo che
producano le lettere \emph{blackboard-bold} del \emph{font} di simboli matematici
`AMSb'.  Se nel proprio documento si fa spesso riferimento ai numeri
reali o a quelli complessi, si pu\`o preferire questo metodo a
quello di definire, per esempio, un comando \ncn{field} e scrivere
|\field{C}| e |\field{R}|, ma per ottenere la massima flessibilit\`a
e il massimo controllo sarebbe opportuno definire tale comando e poi
definire \ncn{C} e \ncn{R} in funzione di quello:\index{mathbb@\cn{mathbb}}
\begin{verbatim}
\usepackage{amsfonts}% per disporre dell'alfabeto \mathbb
\newcommand{\field}[1]{\mathbb{#1}}
\newcommand{\C}{\field{C}}
\newcommand{\R}{\field{R}}
\end{verbatim}

\section{Simboli matematici in grassetto}

Il comando \cn{mathbf} \`e usato comunemente per ottenere lettere
latine grassette in modo matematico, ma per la maggior parte degli
altri tipi di simboli matematici non ha effetto, o i suoi effetti
dipendono in maniera non prevedibile dalla serie di \emph{font} matematici in
uso.  Per esempio, scrivendo
\begin{verbatim}
\Delta \mathbf{\Delta}\mathbf{+}\delta \mathbf{\delta}
\end{verbatim}
si ottiene $\Delta \mathbf{\Delta}\mathbf{+}\delta \mathbf{\delta}$;
il comando \cn{mathbf} non ha cambiato il segno pi\`u e il delta
minuscolo.

Per questo motivo il pacchetto \pkg{amsmath} fornisce altri due
comandi, \cn{boldsymbol} e \cn{pmb}, che possono essere usati con gli
altri tipi di simboli matematici.  \cn{boldsymbol} pu\`o essere usato
per i simboli matematici sui quali non ha effetto il comando \cn{mathbf}
se (e solo se) il \emph{font} matematico in uso in quel momento dispone
di una versione in grassetto di quel simbolo.  \cn{pmb} pu\`o essere usato
come ultima risorsa per qualsiasi simbolo matematico che non abbia una
vera versione in grassetto all'interno del \emph{font} matematico; \qq{pmb}
\`e l'abbreviazione di \qq{poor man's bold} (grassetto dei poveri) e
funziona stampando copie pi\`u copie dello stesso simbolo leggermente spostate
le une dalle altre.  Il risultato \`e di qualit\`a
inferiore, specialmente per quei simboli che contengono linee sottili.
Quando si usa la famiglia standard di \emph{font} matematici del \latex/ (il
Computer Modern), gli unici simboli che potrebbero richiedere il
\cn{pmb} sono quelli dei simboli operatori di grandi dimensioni, come \cn{sum}, i simboli
di delimitatori estesi, o i simboli addizionali forniti dal pacchetto
\pkg{amssymb}~\cite{amsfonts}.

La formula seguente mostra alcuni dei possibili risultati:
\begin{verbatim}
A_\infty + \pi A_0
\sim \mathbf{A}_{\boldsymbol{\infty}} \boldsymbol{+}
  \boldsymbol{\pi} \mathbf{A}_{\boldsymbol{0}}
\sim\pmb{A}_{\pmb{\infty}} \pmb{+}\pmb{\pi} \pmb{A}_{\pmb{0}}
\end{verbatim}
\begin{equation*}
A_\infty + \pi A_0
\sim \mathbf{A}_{\boldsymbol{\infty}} \boldsymbol{+}
  \boldsymbol{\pi} \mathbf{A}_{\boldsymbol{0}}
\sim\pmb{A}_{\pmb{\infty}} \pmb{+}\pmb{\pi} \pmb{A}_{\pmb{0}}
\end{equation*}
Se si vuole usare solo il comando \cn{boldsymbol} senza caricare tutto
il pacchetto \pkg{amsmath}, si pu\`o usare il pacchetto \pkg{bm} (questo
\`e un pacchetto standard del \latex/, non fa parte di quelli AMS; se
si ha una versione del \latex/ del 1997 o posteriore, probabilmente lo
si ha gi\`a).

\section{Lettere greche corsive}

Per ottenere una versione corsiva delle lettere greche maiuscole, si
possono usare i seguenti comandi:
\begin{ctab}{rlrl}
\cn{varGamma}& $\varGamma$& \cn{varSigma}& $\varSigma$\\
\cn{varDelta}& $\varDelta$& \cn{varUpsilon}& $\varUpsilon$\\
\cn{varTheta}& $\varTheta$& \cn{varPhi}& $\varPhi$\\
\cn{varLambda}& $\varLambda$& \cn{varPsi}& $\varPsi$\\
\cn{varXi}& $\varXi$& \cn{varOmega}& $\varOmega$\\
\cn{varPi}& $\varPi$
\end{ctab}

%%%%%%%%%%%%%%%%%%%%%%%%%%%%%%%%%%%%%%%%%%%%%%%%%%%%%%%%%%%%%%%%%%%%%%%%
% by GA

\chapter{Messaggi di errore e problemi di output}

\section{Osservazioni di carattere generale}

Questo \`e un supplemento al capitolo~8 del manuale del \latex/
\cite{lamport} (prima edizione: capitolo~6). Per comodit\`a del
lettore, l'insieme dei messaggi d'errore discussi qui si sovrappone
parzialmente con quello di \cite{lamport}, ma sia chiaro che qui
non si intende dare una copertura esaustiva. I messaggi
d'errore sono disposti in ordine alfabetico, senza badare a testo
irrilevante quale |! LaTeX Error:| all'inizio del messaggio, e
caratteri non alfabetici quali \qc{\\}. Dove vengono forniti esempi,
vengono anche mostrati i messaggi d'aiuto che appaiono sullo schermo
quando si risponde a un messagio d'errore digitando |h|.

C'\`e anche una sezione che discute qualche errore di output, per esempio
in casi in cui il documento stampato ha qualcosa che non va, ma \latex/
non ha rilevato alcun errore.

\section{Messaggi di errore}

\begin{error}{\begin{split} won't work here.}
\errexa
\begin{verbatim}
! Package amsmath Error: \begin{split} won't work here.
 ...

l.8 \begin{split}

? h
\Did you forget a preceding \begin{equation}?
If not, perhaps the `aligned' environment is what you want.
?
\end{verbatim}
\errexpl L'ambiente \env{split} non costruisce un'equazione in
\emph{display} a s\'e stante; deve essere usato all'interno di qualche
altro ambiente quali \env{equation} o \env{gather}.

\end{error}

\begin{error}{Extra & on this line}
\errexa
\begin{verbatim}
! Package amsmath Error: Extra & on this line.

See the amsmath package documentation for explanation.
Type  H <return>  for immediate help.
 ...

l.9 \end{alignat}

? h
\An extra & here is so disastrous that you should probably exit
 and fix things up.
?
\end{verbatim}
\errexpl
In una struttura \env{alignat} il numero di punti di allineamento su una linea
\`e determinato dall'argomento numerico fornito dopo |\begin{alignat}|.
Se in una linea si usano pi\`u punti di allineamento rispetto a quelli consentiti,
\latex/ assume che sia stato dimenticato accidentalmente un comando di
interruzione di riga \cn{\\} e produce questo errore.
\end{error}

\begin{error}{Improper argument for math accent}
\errexa
\begin{verbatim}
! Package amsmath Error: Improper argument for math accent:
(amsmath)                Extra braces must be added to
(amsmath)                prevent wrong output.

See the amsmath package documentation for explanation.
Type  H <return>  for immediate help.
 ...

l.415 \tilde k_{\lambda_j} = P_{\tilde \mathcal
                                               {M}}
?
\end{verbatim}
\errexpl
Argomenti complessi per tutti i comandi \latex/ dovrebbero venire
racchiusi tra parentesi graffe. In questo esempio le graffe sono
necessarie come mostrato:
\begin{verbatim}
... P_{\tilde{\mathcal{M}}}
\end{verbatim}
\end{error}

\begin{error}{Font OMX/cmex/m/n/7=cmex7 not loadable ...}
\errexa
\begin{verbatim}
! Font OMX/cmex/m/n/7=cmex7 not loadable: Metric (TFM) file not found.
<to be read again>
                   relax
l.8 $a
      b+b^2$
? h
I wasn't able to read the size data for this font,
so I will ignore the font specification.
[Wizards can fix TFM files using TFtoPL/PLtoTF.]
You might try inserting a different font spec;
e.g., type `I\font<same font id>=<substitute font name>'.
?
\end{verbatim}
\errexpl Certe dimensioni di alcuni \emph{font} del Computer Modern che erano
un tempo disponibili principalmente attraverso la raccolta
AMSFonts\index{AMSFonts, raccolta} sono considerate parte del \latex/
standard (giugno 1994): \fn{cmex7}\ndash \texttt{9},
\fn{cmmib5}\ndash \texttt{9}, e \fn{cmbsy5}\ndash \texttt{9}. Se
queste dimensioni straordinarie mancano nel proprio sistema,
bisognerebbe prima recuperarle dalla sogrente che ha fornito \latex/.
Altrimenti, si potrebbe provare a reperirle dalla CTAN (per esempio,
sotto forma di sorgenti Metafont\index{Sorgenti Metafont}, directory
\nfn{/tex-archive/fonts/latex/mf}, o in formato PostScript Type 1,
directory \nfn{/tex-archive/fonts/cm/ps-type1/bakoma}\index{\emph{font}
BaKoMa}\index{\emph{font} PostScript}).

Se il nome del \emph{font} comincia per \fn{cmex}, c'\`e un'opzione speciale
\fn{cmex10} per il pacchetto \pkg{amsmath} che fornisce una scappatoia
temporanea. In altre parole, si cambi il comando \cn{usepackage} in
\begin{verbatim}
\usepackage[cmex10]{amsmath}
\end{verbatim}
Questo forzer\`a l'uso della dimensione di 10 punti del \emph{font} \fn{cmex}
in ogni caso. A seconda del contenuto del documento, questo potrebbe
essere adeguato.
\end{error}

\begin{error}{Math formula deleted: Insufficient extension fonts}
\errexa
\begin{verbatim}
! Math formula deleted: Insufficient extension fonts.
l.8 $ab+b^2$

?
\end{verbatim}
\errexpl
Solitamente questo \`e preceduto da un errore del tipo |Font ... not loadable|;
si veda (sopra) la descrizione di quell'errore per risolvere il problema.
\end{error}

\begin{error}{Missing number, treated as zero}
\errexa
\begin{verbatim}
! Missing number, treated as zero.
<to be read again>
                   a
l.100 \end{alignat}

? h
A number should have been here; I inserted `0'.
(If you can't figure out why I needed to see a number,
look up `weird error' in the index to The TeXbook.)

?
\end{verbatim}
\errexpl
Ci sono parecchie cause che possono provocare questo errore. Comunque, una
possibilit\`a che \`e rilevante per il pacchetto \pkg{amsmath} \`e che si
\`e dimenticato di specificare l'argomento numerico di un ambiente \env{alignat},
come illustrato in questo esempio:
\begin{verbatim}
\begin{alignat}
 a&  =b&    c& =d\\
a'& =b'&   c'& =d'
\end{alignat}
\end{verbatim}
dove la prima linea dovrebbe invece essere
\begin{verbatim}
\begin{alignat}{2}
\end{verbatim}

Un'altra possibilit\`a \`e che una aperta parentesi quadra |[|
segua un comando di interruzione di linea \cn{\\} in un costrutto multilinea
come \env{array}, \env{tabular}, o \env{eqnarray}. Questo sar\`a
interpretato da \latex/ come l'inizio di una richiesta di `spazio verticale
aggiuntivo' \cite[\S C.1.6]{lamport}, anche se capita nella linea successiva
con l'intenzione di renderlo parte del contenuto. Per esempio
\begin{verbatim}
\begin{array}
a+b\\
[f,g]\\
m+n
\end{array}
\end{verbatim}
Per evitare il messaggio d'errore in casi di questo tipo, si possono
aggiungere parentesi graffe come suggerito nel manuale di \latex/
\cite[\S C.1.1]{lamport}:
\begin{verbatim}
\begin{array}
a+b\\
{[f,g]}\\
m+n
\end{array}
\end{verbatim}

\end{error}

\begin{error}{Missing \right. inserted}
\errexa
\begin{verbatim}
! Missing \right. inserted.
<inserted text>
                \right .
l.10 \end{multline}

? h
I've inserted something that you may have forgotten.
(See the <inserted text> above.)
With luck, this will get me unwedged. But if you
really didn't forget anything, try typing `2' now; then
my insertion and my current dilemma will both disappear.
\end{verbatim}
\errexpl
Questo errore si verifica tipicamente quando si cerca di inserire una
interruzione di linea all'interno di una coppia di delimitatori
\cn{left}-\cn{right} in un ambiente \env{multline} o \env{split}:
\begin{verbatim}
\begin{multline}
AAA\left(BBB\\
  CCC\right)
\end{multline}
\end{verbatim}
Ci sono due possibili soluzioni: (1)~invece di usare \cn{left} e
\cn{right}, si usino delimitatori `big' di grandezza fissa (\cn{bigl}
\cn{bigr} \cn{biggl} \cn{biggr} \dots; si veda \secref{bigdel}); oppure
(2)~si usino delimitatori nulli per spezzare la copia \cn{left}-\cn{right}
in due parti, una per ogni linea:
\begin{verbatim}
AAA\left(BBB\right.\\
  \left.CCC\right)
\end{verbatim}
La seconda soluzione potrebbe dar luogo a dimensioni incoerenti dei
delimitatori; ci si pu\`o assicurare che coincidono usando \cn{vphantom}
nella linea in cui compare il delimitatore pi\`u piccolo (o magari \cn{smash}
nella linea in cui compare il delimitatore pi\`u grande). Nell'argomento di
\cn{vphantom} bisogna mettere una copia dell'elemento pi\`u alto che compare
nell'altra linea, per esempio
\begin{verbatim}
xxx \left(\int_t yyy\right.\\
  \left.\vphantom{\int_t} zzz ... \right)
\end{verbatim}
\end{error}

\begin{error}{Paragraph ended before \xxx was complete}
\errexa
\begin{verbatim}
Runaway argument?

! Paragraph ended before \multline was complete.
<to be read again>
                   \par
l.100

? h
I suspect you've forgotten a `}', causing me to apply this
control sequence to too much text. How can we recover?
My plan is to forget the whole thing and hope for the best.
?
\end{verbatim}
\errexpl
Questo potrebbe dipendere da un errore di battitura nel comando
|\end{multline}|, per esempio
\begin{verbatim}
\begin{multline}
...
\end{multiline}
\end{verbatim}
o dall'uso di abbreviazioni di alcuni ambienti, come |\bal| e
|\eal| invece di |\begin{align}| e |\end{align}|:
\begin{verbatim}
\bal
...
\eal
\end{verbatim}
Per motivi tecnici quel tipo di abbreviazioni non funzionano con gli
ambienti pi\`u complesi per equazioni in \emph{display} del pacchetto
\pkg{amsmath} (\env{gather}, \env{align}, \env{split}, etc.; cfr.\@ \fn{technote.tex}).
\end{error}

\begin{error}{Runaway argument?}
Si veda la descrizione del messaggio di errore
\texttt{Paragraph ended before \ncn{xxx} was complete}.
\end{error}

\begin{error}{Unknown option `xxx' for package `yyy'}
\errexa
\begin{verbatim}
! LaTeX Error: Unknown option `intlim' for package `amsmath'.
...
? h
The option `intlim' was not declared in package `amsmath', perhaps you
misspelled its name. Try typing  <return>  to proceed.
?
\end{verbatim}
\errexpl
Questo significa che il nome dell'opzione \`e stato scritto male, o
semplicemente che il pacchetto, al  contrario di quanto ci si aspettava,
non ha quell'opzione. Si consulti la documentazione di quel pacchetto.
\end{error}

\begin{error}{Old form `\pmatrix' should be \begin{pmatrix}.}
\errexa
\begin{verbatim}
! Package amsmath Error: Old form `\pmatrix' should be
                         \begin{pmatrix}.

See the amsmath package documentation for explanation.
Type  H <return>  for immediate help.
 ...

\pmatrix ->\left (\matrix@check \pmatrix
                                         \env@matrix
l.16 \pmatrix
             {a&b\cr c&d\cr}
? h
`\pmatrix{...}' is old Plain-TeX syntax whose use is
ill-advised in LaTeX.
?
\end{verbatim}
\errexpl
Quando si usa il paccheto \pkg{amsmath}, le vecchie forme \cn{pmatrix},
\cn{matrix}, e \cn{cases} non posso pi\`u essere usate, a causa di conflitti
di nome. Ad ogni modo, la loro sintassi non era conforme alla sintassi
standard \LaTeX{}.
\end{error}

\begin{error}{Erroneous nesting of equation structures}
\errexa
\begin{verbatim}
! Package amsmath Error: Erroneous nesting of equation structures;
(amsmath)                trying to recover with `aligned'.

See the amsmath package documentation for explanation.
Type  H <return>  for immediate help.
 ...

l.260 \end{alignat*}
                    \end{equation*}
\end{verbatim}
\errexpl
Le strutture \env{align}, \env{alignat}, etc., sono progettate per essere
usate nel livello pi\`u alto, e perlopi\`u non possono essere annidate in
alcune altre strutture di equazioni in \emph{display}. Una eccezione notevole
\`e data dal fatto che \env{align} e molte sue varianti possono essere usate
nell'ambiente \env{gather}.
\end{error}

\section{Messaggi di warning}

\begin{error}{Foreign command \over [or \atop or \above]}
\errexa
\begin{verbatim}
Package amsmath Warning: Foreign command \over; \frac or \genfrac
(amsmath)                should be used instead.
\end{verbatim}
\errexpl L'utilizzo dei comandi di frazione originali del \tex/\mdash
\cs{over}, \cs{atop}, \cs{above}\mdash \`e deprecato quando si usa il
pacchetto \pkg{amsmath}, perch\`e la loro sintassi \`e estranea a \latex/,
e \pkg{amsmath} fornisce comandi equivalenti nativi di \latex/. Si veda
\fn{technote.tex} per ulteriori informazioni.
\end{error}

\begin{error}{Cannot use `split' here}
\errexa
\begin{verbatim}
Package amsmath Warning: Cannot use `split' here;
(amsmath)                trying to recover with `aligned'
\end{verbatim}
\errexpl L'ambiente \env{split} \`e studiato per essere usato con l'intero
corpo di un'equazione, o una intera linea di un ambiente \env{align} o
\env{gather}. Non ci pu\`o essere alcun tipo di materiale prima o
dopo di esso all'interno della stessa struttura contenente:
\begin{verbatim}
\begin{equation}
\left\{ % <-- Proibito
\begin{split}
...
\end{split}
\right. % <-- Proibito
\end{equation}
\end{verbatim}
\end{error}

\section{Output sbagliato}

\subsection{Sezioni numerate 0.1, 5.1, 8.1 invece che 1, 2, 3}
\label{numinverse}

Questo molto probabilmente significa che gli argomenti di \cn{numberwithin}
sono stati inseriti alla rovescia:
\begin{verbatim}
\numberwithin{section}{equation}
\end{verbatim}
Questo significa ``stampa il numero di sezione nella forma
\textit{numero-equazione}.\textit{numero-sezione} e ricomincia da
1 ogni volta che incontri
un'equazione'' mentre probabilmente si voleva ottenere l'effetto opposto
\begin{verbatim}
\numberwithin{equation}{section}
\end{verbatim}

\subsection{Il comando \cn{numberwithin} non ha avuto effetto sui numeri
di equazione}

State guardando la prima sezione del vostro documento? Controllate la
numerazione delle equazioni in altre parti del documento per vedere se
il problema \`e quello descritto in \secref{numinverse}.

%%%%%%%%%%%%%%%%%%%%%%%%%%%%%%%%%%%%%%%%%%%%%%%%%%%%%%%%%%%%%%%%%%%%%%%%
% by RZ

\chapter{Ulteriori informazioni}

\section{Convertire documenti gi\`a scritti}

\subsection{Convertire da \LaTeX{} ``puro''} %%%% plain

Sotto molti aspetti, un documento \LaTeX{} continua a funzionare
allo stesso modo quando al preambolo del documento si aggiunge
\verb'\usepackage{amsmath}'. Il pacchetto \pkg{amsmath} sopprime
per\`o, salvo diversa indicazione, le interruzioni di pagina all'interno di
strutture che contengono equazioni in \emph{display} come \env{eqnarray},
\env{align} e \env{gather}. Per continuare a permettere le
interruzioni di pagina all'interno di \env{eqnarray} dopo essere
passati al pacchetto \pkg{amsmath}, \`e necessario aggiungere la
seguente riga nel preambolo del documento:
\begin{verbatim}
\allowdisplaybreaks[1]
\end{verbatim}
Per assicurare una spaziatura normale attorno ai simboli di relazione,
si dove sostituire \env{eqnarray} con \env{align}, \env{multline} o
\env{equation}\slash\env{split}, in maniera appropriata.

La maggior parte delle altre differenze d'uso del pacchetto
\pkg{amsmath} possono essere considerate raffinatezze facoltative,
come per esempio l'uso di
\begin{verbatim}
\DeclareMathOperator{\Hom}{Hom}
\end{verbatim}
invece di \verb'\newcommand{\Hom}{\mbox{Hom}}'.

\subsection{Convertire da \amslatex/ 1.1}
Si veda \fn{diffs-m.txt}.

\section{Note tecniche}
Il file \fn{technote.tex} contiene alcuni commenti su diverse questioni
che difficilmente possono essere di interesse generale.

\section{Ottenere aiuto}

Domande o commenti riguardanti \pkg{amsmath} e pacchetti correlati
dovrebbero essere inviati a:
\begin{infoaddress}
American Mathematical Society\\
Technical Support\\
Electronic Products and Services\\
P. O. Box 6248\\
Providence, RI 02940\\[3pt]
Phone: 800-321-4AMS (321-4267) \quad or \quad 401-455-4080\\
Internet: \mail{tech-support@ams.org}
\end{infoaddress}
Quando si riporta un problema occorre includere, per consentire
un'indagine adeguata, le seguenti informazioni:

\begin{enumerate}
\item Il file sorgente in cui \`e sorto il problema, preferibilmente
  ridotto alle minime dimensioni rimuovendo tutto il materiale che pu\`o
  essere rimosso senza interferire sul problema in questione.
\item Un file di log di \latex/ che mostri il messaggio di errore (se
  presente) e i numeri di versione delle classi di documento e file di
  opzioni in uso.
\end{enumerate}

\section{Di possibile interesse}\label{a:possible-interest}
\`E possibile avere informazioni su come ottenere i \emph{font} AMS o
altro materiale relativo a \tex/ dall'archivio Internet AMS
\fn{e-math.ams.org} inviando una richiesta attraverso la posta
elettronica a: \mail{webmaster@ams.org}.

Si possono avere informazioni su come ottenere dall'AMS la distribuzione
\pkg{amsmath} su dischetti da:
\begin{infoaddress}
American Mathematical Society\\
Customer Services\\
P. O. Box 6248\\
Providence, RI 02940\\[3pt]
Phone: 800-321-4AMS (321-4267) \quad or \quad 401-455-4000\\
Internet: \mail{cust-serv@ams.org}
\end{infoaddress}

Il ``\tex/ Users Group\index{TeX Users@\tex/ Users Group}''
\`e una organizzazione senza scopo di lucro che pubblica una
rivista (\journalname{TUGboat}\index{TUGboat@\journalname{TUGboat}}),
organizza meeting, e serve da punto di smistamento per informazioni
su \tex/ e software relativo ad esso.
\begin{infoaddress}
\tex/ Users Group\\
PO Box 2311\\
Portland, OR 97208-2311\\
USA\\[3pt]
Phone: +1-503-223-9994\\
Email: \mail{office@tug.org}
\end{infoaddress}
Iscriversi al ``\tex/ Users Group'' \`e un buon modo per sostenere il
continuo sviluppo di software libero relativo a \tex/.
Esistono inoltre molti ``\tex/ users group'' locali in altri stati;
si possono ottenere informazioni su come contattare un gruppo locale dal
``\tex/ Users Group''.

Esiste un gruppo di discussione Usenet chiamato \fn{comp.text.tex},
che \`e una buona fonte di informazioni su \latex/ e
\tex/ in generale. Se non si sa come leggere un gruppo di discussione,
occorre chiedere all'amministratore di sistema locale se \`e
disponibile un servizio di lettura di \emph{newsgroup}.


\begin{thebibliography}{9}
\addcontentsline{toc}{chapter}{Bibliografia}

\bibitem{amsfonts}\booktitle{AMSFonts version \textup{2.2}\mdash user's guide},
Amer. Math. Soc., Providence, RI, 1994; distribuito
con il pacchetto AMSFonts.

\bibitem{instr-l}\booktitle{Instructions for preparation of
papers and monographs\mdash \amslatex/},
Amer. Math. Soc., Providence, RI, 1996, 1999.

\bibitem{amsthdoc}\booktitle{Using the \pkg{amsthm} Package},
Amer. Math. Soc., Providence, RI, 1999.

\bibitem{tlc} Michel Goossens, Frank Mittelbach e Alexander Samarin,
\booktitle{The \latex/ companion}, Addison-Wesley, Reading, MA, 1994.
  [\emph{Note: L'edizione del 1994 non \`e una guida affidabile per il
    pacchetto \pkg{amsmath} a meno che non ci si riferisca al file
    \fn{compan.err}, distribuito con \LaTeX{}, che contiene una errata
    corrige per il Capitolo 8\mdash.}]

% Deal with a line breaking problem
\begin{raggedright}
\bibitem{mil} G. Gr\"{a}tzer,
\emph{Math into \LaTeX{}: An Introduction to \LaTeX{} and AMS-\LaTeX{}}
  \url{http://www.ams.org/cgi-bin/bookstore/bookpromo?fn=91&arg1=bookvideo&itmc=MLTEX},
Birkh\"{a}user, Boston, 1995.\par
\end{raggedright}

\bibitem{kn} Donald E. Knuth, \booktitle{The \tex/book},
Addison-Wesley, Reading, MA, 1984.

\bibitem{lamport} Leslie Lamport, \booktitle{\latex/: A document preparation
system}, 2nd revised ed., Addison-Wesley, Reading, MA, 1994.

\bibitem{msf} Frank Mittelbach and Rainer Sch\"opf,
\textit{The new font family selection\mdash user
interface to standard \latex/}, \journalname{TUGboat} \textbf{11},
no.~2 (June 1990), pp.~297\ndash 305.

\bibitem{jt} Michael Spivak, \booktitle{The joy of \tex/}, 2nd revised ed.,
Amer. Math. Soc., Providence, RI, 1990.

\end{thebibliography}
% by RZ - end

%\input{amsldoc.ind}
% Sostituita la riga di sopra con la seguente - GD
\printindex

\end{document}

% Sostituita la riga di sopra con la seguente - GD
\printindex

\end{document}

% Sostituita la riga di sopra con la seguente - GD
\printindex

\end{document}

% Sostituita la riga di sopra con la seguente - GD
\printindex

\end{document}
