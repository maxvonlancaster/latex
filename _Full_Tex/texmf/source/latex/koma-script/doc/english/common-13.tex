% ======================================================================
% common-13.tex
% Copyright (c) Markus Kohm, 2001-2010
%
% This file is part of the LaTeX2e KOMA-Script bundle.
%
% This work may be distributed and/or modified under the conditions of
% the LaTeX Project Public License, version 1.3c of the license.
% The latest version of this license is in
%   http://www.latex-project.org/lppl.txt
% and version 1.3c or later is part of all distributions of LaTeX 
% version 2005/12/01 or later and of this work.
%
% This work has the LPPL maintenance status "author-maintained".
%
% The Current Maintainer and author of this work is Markus Kohm.
%
% This work consists of all files listed in manifest.txt.
% ----------------------------------------------------------------------
% common-13.tex
% Copyright (c) Markus Kohm, 2001-2010
%
% Dieses Werk darf nach den Bedingungen der LaTeX Project Public Lizenz,
% Version 1.3c, verteilt und/oder veraendert werden.
% Die neuste Version dieser Lizenz ist
%   http://www.latex-project.org/lppl.txt
% und Version 1.3c ist Teil aller Verteilungen von LaTeX
% Version 2005/12/01 oder spaeter und dieses Werks.
%
% Dieses Werk hat den LPPL-Verwaltungs-Status "author-maintained"
% (allein durch den Autor verwaltet).
%
% Der Aktuelle Verwalter und Autor dieses Werkes ist Markus Kohm.
% 
% Dieses Werk besteht aus den in manifest.txt aufgefuehrten Dateien.
% ======================================================================
%
% Paragraphs that are common for several chapters of the KOMA-Script guide
% Maintained by Markus Kohm
%
% ----------------------------------------------------------------------
%
% Abs�tze, die mehreren Kapiteln der KOMA-Script-Anleitung gemeinsam sind
% Verwaltet von Markus Kohm
%
% ======================================================================

\ProvidesFile{common-13.tex}[2009/01/04 KOMA-Script guide (common paragraphs)]
\translator{Gernot Hassenpflug}

\makeatletter
\@ifundefined{ifCommonmaincls}{\newif\ifCommonmaincls}{}%
\@ifundefined{ifCommonscrextend}{\newif\ifCommonscrextend}{}%
\@ifundefined{ifCommonscrlttr}{\newif\ifCommonscrlttr}{}%
\@ifundefined{ifIgnoreThis}{\newif\ifIgnoreThis}{}%
\makeatother


\section{Randnotizen}
\label{sec:\csname label@base\endcsname.marginNotes}%
\ifshortversion\IgnoreThisfalse\IfNotCommon{maincls}{\IgnoreThistrue}\fi%
\ifIgnoreThis
Es gilt sinngem��, was in \autoref{sec:maincls.marginNotes} geschrieben
wurde.
\else
\BeginIndex{}{Randnotizen}%

Au�er dem eigentlichen Textbereich, der normalerweise den Satzspiegel
ausf�llt, existiert in Dokumenten noch die so genannte Marginalienspalte. In
dieser k�nnen Randnotizen gesetzt werden. % Umbruchkorrekturtext
\IfNotCommon{scrlttr2}{In diesem \iffree{Dokument}{Buch} wird davon ebenfalls
  Gebrauch gemacht.}%
\fi
\IfCommon{scrlttr2}{Bei Briefen sind Randnotizen allerdings eher un�blich und
  sollten �u�erst sparsam eingesetzt werden.}%
\ifIgnoreThis\else


\begin{Declaration}
  \Macro{marginpar}\OParameter{Randnotiz links}\Parameter{Randnotiz}\\
  \Macro{marginline}\Parameter{Randnotiz}
\end{Declaration}%
\BeginIndex{Cmd}{marginpar}%
\BeginIndex{Cmd}{marginline}%
F�r Randnotizen\Index[indexmain]{Randnotizen} ist bei {\LaTeX} normalerweise die
Anweisung \Macro{marginpar} vorgesehen. Die Randnotizen werden dabei im
�u�eren Rand gesetzt. Bei einseitigen Dokumenten wird der rechte Rand
verwendet. Zwar kann bei \Macro{marginpar} optional eine abweichende Randnotiz
angegeben werden, falls die Randnotiz im linken Rand landet, jedoch werden
Randnotizen immer im Blocksatz ausgegeben. Die Erfahrung zeigt, dass bei
Randnotizen statt des Blocksatzes oft je nach Rand linksb�ndiger oder
rechtsb�ndiger Flattersatz zu bevorzugen ist. {\KOMAScript} bietet hierf�r
die Anweisung \Macro{marginline}.

\ifCommonscrlttr\else
\begin{Example}
  \phantomsection\label{desc:\csname
    label@base\endcsname.cmd.marginline.example}%
  In diesem Kapitel ist an einigen Stellen die %
  \IfCommon{maincls}{Klassenangabe \Class{scrartcl} }%
  \IfCommon{scrextend}{Paketangabe \Package{scrextend} }%
  im Rand zu finden. Diese kann mit:
\ifCommonmaincls
  \begin{lstcode}
  \marginline{\texttt{scrartcl}}
  \end{lstcode}
\else
  \begin{lstcode}
  \marginline{\texttt{scrextend}}
  \end{lstcode}
\fi
  erreicht werden.%
\iftrue % Umbruchkorrekturtext
  \footnote{Tats�chlich wurde nicht \Macro{texttt},
    sondern eine semantische Auszeichnung verwendet. Um nicht unn�tig
    zu verwirren, wurde diese im Beispiel ersetzt.}%
\fi

  Statt \Macro{marginline} w�re auch die Verwendung von \Macro{marginpar}
  m�glich gewesen. Tats�chlich wird bei obiger Anweisung intern nichts anders
  gemacht als:
\ifCommonmaincls
  \begin{lstcode}
  \marginpar[\raggedleft\texttt{scrartcl}]
    {\raggedright\texttt{scrartcl}}
  \end{lstcode}
\else
  \begin{lstcode}
  \marginpar[\raggedleft\texttt{scrextend}]
    {\raggedright\texttt{scrextend}}
  \end{lstcode}
\fi
  Letztlich ist \Macro{marginline} also nur eine abk�rzende Schreibweise.
\end{Example}

F�r Experten sind in \autoref{sec:maincls-experts.addInfos},
\autopageref{desc:maincls-experts.cmd.marginpar} Probleme bei der Verwendung
von \Macro{marginpar} dokumentiert. Diese gelten ebenso f�r
\Macro{marginline}.%
\fi%
\IfCommon{scrlttr2}{Ein ausf�hrliches Beispiel hierzu finden Sie in
  \autoref{sec:maincls.marginNotes},
  \autopageref{desc:maincls.cmd.marginline.example}.}%
%
\EndIndex{Cmd}{marginpar}%
\EndIndex{Cmd}{marginline}%
%
\EndIndex{}{Randnotizen}%
\fi% IgnoreThis


%%% Local Variables:
%%% mode: latex
%%% coding: iso-latin-1
%%% TeX-master: "../guide"
%%% End:
