% master: main
% format: latex

\LaTeX\ is a special version of \TeX, a state-of-art command-driven document
production system.  \TeX\ was created by \mbox{Donald Knuth}, a professor in
the Stanford University, and its domain is public, but \TeX\ logo is the
trade mark of the American Mathematical Society, and \TeX\ AMSfonts belong to
the same society.  This also implies, which is indeed the fact, that \TeX\ is
powerful at typesetting formulae, a headache of all typists.

Leslie Lamport from DEC programmed a collection of macros in \TeX.
These macros together with some things contributed by other authors make up a
\TeX\ database package which is known as \LaTeX.  \LaTeX\ is actually designed
for those who want to typeset documents with the high typographic quality of
\TeX\ but do not want to learn the sophisticated \TeX\ programming.  For a
non-computer-science graduate student or faculty member, who may need to
typeset a paper, resume, thesis, book, manual or a letter with high quality and
who does not mind learning a command-driven text formatter, \LaTeX\ is probably
an ideal package.  The \LaTeX\ had the major update from \LaTeX\ 2.09 to
\LaTeXe\ in 1994.

This sample \LaTeXe\ file will serve as a user guide to the \LaTeXe\ document
class for theses at Univ.  of Pittsburgh.  As a modified version of Leslie
Lamport's \texttt{sample2e.tex} file is included, this file will also serve to
help beginners understand how some of the important \LaTeXe\ commands work.


\vspace{1em}
\section*{DESCRIPTORS}
\vspace{1em}
\begin{center}
\renewcommand{\arraystretch}{1.5}
\begin{tabular*}{\textwidth}{p{0.47\textwidth}p{0.47\textwidth}}
  \LaTeXe\ document class &
  Pitt engineering thesis \\
  Pitt standard thesis &
  Thesis document sample file \\
  University of Pittsburgh
\end{tabular*}
\end{center}
