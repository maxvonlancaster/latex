% mangletex (11 May 1992) run at 22:09 GMT Monday 08 November 1993
%%\title		{\TeX\ macros for proof boxes}
%%\author		{Paul Taylor\\
%%			Department of Computing,\\
%%			Imperial College,\\
%%			London SW7 2BZ\\
%%			+44 71 589 5111 {\em ext.} 5057\\
%%			{\tt<pt@doc.ic.ac.uk>}}
%%\date{8 November 1993}
%% % to get the provisional user documentation do
%% % grep ^%% boxproof.tex | sed 's/^%*//' > boxproof-manual.tex
%%\documentstyle{article}
%%\input boxproof.tex
%%\def\meta#1{\mbox{$\langle\hbox{#1}\rangle$}}
%%\def\macrowitharg#1#2{{\tt\string#1\bra\meta{#2}\ket}}
%%{\escapechar-1 \xdef\bra{\string\{}\xdef\ket{\string\}}}
%%\let\subsection\section
%%\begin{document}
%%\maketitle
%%
%% \subsection{Introduction}
%% The proof
%%\begin{proofbox}
%%  \(\"1"\:\alpha\leftrightarrow\psi(x,\top)\=\\
%%        \:\Some\beta.\psi(x,\beta)\=\mathsf{total}\\
%%    \[\exists\beta\kern-1em\:\psi(x,\beta)\\
%%      \(\:\beta\=\\
%%        \:\beta=\top\=(*)\\
%%        \:\psi(x,\!\top)\=\mathsf{subs}\\
%%        \:\alpha\=\elim\leftrightarrow(\ref{1})\\
%%      \*\:\alpha\=\\
%%        \:\psi(x,\!\top)\=\elim\leftrightarrow(\ref{1})\\
%%        \:\beta=\top\=\mathsf{func}\\
%%        \:\beta\=(*)\\
%%      \)\:\alpha=\beta\=\intro\leftrightarrow\\
%%        \:\psi(x,\alpha)\=\mathsf{subs}\\
%%    \]  \:\psi(x,\alpha)\=\elim\exists\\
%%    \*  \:\psi(x,\alpha)\=\\
%%      \(\:\alpha\=\\
%%        \:\alpha=\top\=(*)\\
%%        \:\phi(x,\!\top)\=\mathsf{subs}\\
%%      \*\:\psi(x,\!\top)\=\\
%%        \:\alpha=\top\=\mathsf{func}\\
%%        \:\alpha\=(*)\\
%%      \)\:\alpha\leftrightarrow\psi(x,\!\top)\=\intro\leftrightarrow\\
%%%    \)  \:\psi(x,\alpha)\leftrightarrow(\alpha\leftrightarrow\psi(x,\!\top))
%%          \=\intro\leftrightarrow\\
%%\end{proofbox}
%% is produced by
%%\begin{verbatim}
%%\begin{proofbox}
%%  \(\"1"\:\alpha\leftrightarrow\psi(x,\top)\\
%%        \:\Some\beta.\psi(x,\beta)                      \=\mathsf{total}\\
%%    \[\exists\beta\kern-1em\:\psi(x,\beta)\\
%%      \(\:\beta\\
%%        \:\beta=\top                                    \=(*)\\
%%        \:\psi(x,\!\top)                                \=\mathsf{subs}\\
%%        \:\alpha                         \=\elim\leftrightarrow(\ref{1})\\
%%      \*\:\alpha\\
%%        \:\psi(x,\!\top)                 \=\elim\leftrightarrow(\ref{1})\\
%%        \:\beta=\top                                     \=\mathsf{func}\\
%%        \:\beta\=(*)\\
%%      \)\:\alpha=\beta                           \=\intro\leftrightarrow\\
%%        \:\psi(x,\alpha)                                 \=\mathsf{subs}\\
%%    \]  \:\psi(x,\alpha)                                  \=\elim\exists\\
%%    \*  \:\psi(x,\alpha)\\
%%      \(\:\alpha\\
%%        \:\alpha=\top                                   \=(*)\\
%%        \:\phi(x,\!\top)                                \=\mathsf{subs}\\
%%      \*\:\psi(x,\!\top)\\
%%        \:\alpha=\top                                   \=\mathsf{func}\\
%%        \:\alpha                                        \=(*)\\
%%      \)\:\alpha\leftrightarrow\psi(x,\!\top)   \=\intro\leftrightarrow\\
%%    \)  \:\psi(x,\alpha)\leftrightarrow
%%          (\alpha\leftrightarrow\psi(x,\!\top)) \=\intro\leftrightarrow\\
%%\end{proofbox}
%%\end{verbatim}
%%
%%  Syntax as follows:
%%  each line is of the form
%%  \begin{center}
%%     \meta{variables}
%%     \meta{name}
%%     \verb/\:/ \meta{formula}
%%     \verb/\=/ \meta{reason}
%%     \verb/\-/ \meta{use}
%%     \verb/\\/
%%  \end{center}
%%  where
%%  \begin{itemize}
%%  \item \meta{variables} is something like ``$x,y$'' ---
%%    it's for variables declared at the beginning of $\intro\forall$- and
%%    $\elim\exists$-boxes.
%%  \item \meta{name} is a command \verb/\label{fred}/ which
%%    defines \verb/fred/ to be the label text, which may be used anywhere
%%    as \verb/\ref{fred}/ --- see {\em The \LaTeX book.}
%%    Local labels are also available, using \macrowitharg\lbl{name}
%%    or \verb/\"/\meta{name}\verb/"/; these obey the scoping rules of
%%    the boxes.
%%    You may also refer to the previous line as \verb/\ref{-}/.
%%  \item \meta{formula} is the proposition being asserted.
%%  \item \meta{reason} is \verb/\intro\land(\ref{john},\ref{mary})/
%%         or \verb/\elim\forall(\ref{jim})/.
%%  \item \meta{use} is provided for linear logic, to record the step
%%%         which uses this one. How this accords with theory I don't yet know.
%%  \end{itemize}
%%  Note that the parts are separated by \verb/\:/, \verb/\=/ and \verb/\\/;
%%  these correspond to
%%  \begin{center}
%%     {\bf let } \meta{name} = \meta{expression} : \meta{type}
%%  \end{center}
%%  in a declarative language.
%%  The \verb/\:/, \verb/\=/ and \verb/\-/ fields are optional and may occur
%%  in any order. If any of them is repeated the last is taken.
%%  If none of them is present the \meta{variables} field is also ignored.
%%
%%  Proof {\em boxes\/} are ``wrapped up'' as follows:
%%  \begin{itemize}
%%%  \item the whole proof in \verb/\begin{proofbox}/...\verb/\end{proofbox}/;
%%%  \item single-column boxes ($\intro\forall$, $\intro\imp$, $\elim\exists$),
%%      in \verb/\[/...\verb/\]/.
%%  \item multiple-column boxes are of two kinds:
%%    \begin{itemize}
%%%     \item separate ($\intro\land$) boxes: \verb/\(/...\verb/\*/...\verb/\)/.
%%     \item stuck together ($\elim\lor$) boxes:
%%        \verb/\(/...\verb/\+/...\verb/\)/.
%%    \end{itemize}
%%  \end{itemize}
%%  You may put more than two columns in \verb/\(/...\verb/\)/ and even mix
%%  the \verb/\+/ and \verb/\*/ separators.
%%
%%  The whole proof is enclosed in \verb/\proofbox/...\verb/\endproofbox/
%%  or \verb/\begin{proofbox}/...\verb/\end{proofbox}/, but the \LaTeX\
%%  environment form {\em must not\/} be used for nested boxes.
%%
%%  If the proof occurs in paragraph mode (ie in vertical or
%%  unrestricted horizontal mode) then it is set as a display, using the
%%  full width of the page. Otherwise it uses only the required width.
%%
%%  A lot of the internals are potentially configurable, but there is not
%%  yet a user interface suitable for doing this. This will be provided
%%  in the next version.
%%
%%  This is a {\em prototype\/} proof-box macro package;
%%  the syntax and implementation may change.
%%  It is {\tt boxproof.tex} in the \TeX\ system
%%  (so just \verb/\input boxproof/).
%%
%%  There are some examples in
%%  \verb+~pt/utilities/proofs/proofboxeg.tex+ and
%%  \verb+proofboxeg1.tex+.
%%
%%  The prototype version was begun on 2 August 1991, the new one on
%%  4 June 1993.
%%
%%
\message{<Paul Taylor's Proof Boxes, 8 November 1993>}\edef\proofboxrestoreat
{\noexpand\catcode`\noexpand\@\the\catcode`\@}\catcode`\@=11 \count@\year
\multiply\count@12 \advance\count@\month\ifnum\count@> 23970 \message{ ******
This is an experimental version - please get an up-to-date one ******}%
\message{ by anonymous FTP theory.doc.ic.ac.uk /tex/contrib/Taylor/tex/%
boxproof.tex}\fi\let\pb@A\+\let\+\relax\let\pb@B\(\let\pb@C\)\def\pb@D{\pb@B
\let\)\pb@C}\def\pb@E{\pb@B\displaystyle\let\]\pb@C}%%
%% The following are user-settable dimensions.
%% 
\newdimen\prooflinenowidth\prooflinenowidth1em \newdimen\proofboxfullwidth
\newdimen\proofboxleftmargin\proofboxleftmargin3em \newdimen
\proofboxrightmargin\newdimen\proofboxrulebreadth\proofboxrulebreadth.4pt
\newdimen\proofboxseparation\proofboxseparation.2em \newdimen
\proofboxbaselineskip\proofboxbaselineskip4ex %%
%%
%% The following are user-accessible numbers.
%%
\newcount\proofcolumns\newcount\prooflinetotal\newcount\prooflineinbox%%
%%
%% These boxes receive the fields of the proof steps.
%% They are used in \verb/\proofboxmakeleftcolumn/ {\em etc}.
%%
\newbox\proofboxproposition\newbox\proofboxreason\newbox\proofboxuse\newbox
\proofboxvariables%%
%% The left-most switch is used by
%%% \verb/\proofboxmakelabel/ to suppress the label in all but the first column.
%% On the outside it is true.
\newif\ifleftmostproofbox\leftmostproofboxtrue%%
%%
%% The following are obsolete: they are only included for reverse
%% compatibility with the prototype version.
\newdimen\proofboxintercol\newdimen\proofboxsurround\newdimen
\proofboxformulawidth\newdimen\proofboxmargin\newskip\proofboxlefttabskip
\newskip\proofboxrighttabskip%%
%%
%% \subsection{Redefinable macros}
%% WARNING: most of these commands will be hidden and replaced with
%% optional arguments to \verb/\proofbox/ in a future version.
%% Do not rely on them.
%%
%% We provide three different ways of numbering the lines of the proof:
%% \begin{itemize}
%% \item \verb/\runningproofline/: a global running sequence (default),
%% 
\def\runningproofline{\number\prooflinetotal}%%
%% \item \verb/\nestedproofline/: a hierarchical system with dots,
%%
\def\nestedproofline{\relax\ifx\enclosingproofline\empty\else
\enclosingproofline{.}\fi\number\prooflineinbox}\def\enclosingproofline{}%%
%% \item \verb/\nestedproofline/: a fully hierarchical system which
%% also includes the column number
%% (\verb/\proof@columns/) as a letter (ASCII quote plus number).
%%
\def\proofboxcolumn{\ifnum\proofcolumns>\z@{\count@96\advance\count@
\proofcolumns\char\count@}\fi}\def\fullynestedproofline{\relax\ifx
\enclosingproofline\empty\enclosingproofline\proofboxcolumn.\else
\proofboxcolumn\fi\number\prooflineinbox}%%
%% \end{itemize}
%% \verb/\theproofline/ is the default.
%%
\let\theproofline\runningproofline%%
%% The macro \verb/\proofboxmakelabel#1/ is used to print the line label.
%% We only put it in the leftmost box.
%% It is printed in small non-ranging Arabic numerals
%% ({\the\scriptfont1 0123456789}).
%% Right-justify it in \verb/\prooflinenowidth/ if it will fit,
%% otherwise let it stick out on the right, {\em i.e.}~left-justify it.
%%
\def\proofboxmakelabel#1{{\relax\ifleftmostproofbox\setbox\z@\hbox{\hss\the
\scriptfont\@ne#1}\ifdim\wd\z@<\prooflinenowidth\setbox\z@\hbox to%
\prooflinenowidth{\unhbox\z@}\fi\box\z@\quad\fi}}%%
%% Kill the numbers altogether with \verb/\proofboxnonumbers/.
%%
\def\proofboxnonumbers{\def\proofboxmakelabel##1{}}%%
%% How to make the left column of the proof box:
%% use the variables field, a space if necessary and the line label.
%%
\def\proofboxmakeleftcolumn{\kern\proofboxseparation\hfil\dimen@\wd
\proofboxvariables\unhbox\proofboxvariables\ifdim\dimen@>\z@\quad\fi
\proofboxmakelabel\@currentlabel}%%
%% How to make the middle column of the proof box:
%% left justify the formula field.
%%
\def\proofboxmakemiddlecolumn{\unhbox\proofboxproposition\hfil\quad}%%
%% How to make the right column of the proof box:
%% use the reason and use fields.
%%
\def\proofboxmakerightcolumn{\box\proofboxreason\box\proofboxuse\hfil\kern
\proofboxseparation}%%
%%
%% Make the four edges of a rectangular box and the separator
%% between \verb/\+/ columns.
%%
\def\proofboxleftside{\leaders\hrule width\proofboxrulebreadth\vfill}\def
\proofboxrightside{\leaders\hrule width\proofboxrulebreadth\vfill}\def
\proofboxorseparator{\leaders\hrule width\proofboxrulebreadth\vfill}\def
\proofboxtopside{\leaders\vrule height\proofboxrulebreadth depth\z@\hfill}%
\def\proofboxbottomside{\leaders\vrule height\proofboxrulebreadth depth\z@
\hfill}%%
%%
%% Use dotted lines: \verb/\dottedproofbox/.
%%
\def\dottedproofbox{\def\proofboxleftside{\leaders\pb@F\vfill} \def
\proofboxrightside{\leaders\pb@F\vfill} \def\proofboxorseparator{\leaders
\pb@F\vfill} \def\proofboxtopside{\leaders\pb@F\hfill}\def\proofboxbottomside
{\leaders\pb@F\hfill}%%
}\def\pb@F{\vbox to 5pt{\vss\hbox to 5pt{\hss.\hss}\vss}}%%
%% Leave the boxes open at the bottom: \verb/\openproofbox/
%%
\def\openproofbox{\def\proofboxbottomside{\hfill}} %%
%% \subsection{Miscellaneous logical notations}
%%
%% Print the names of the introduction and elimination rules, for example:
%%  \begin{quote}
%%     \verb/\elim\forall/ $\elim\forall$ \qquad
%%     \verb/\intro\land/ $\intro\land$ \qquad
%%  \end{quote}
\def\intro#1{{{#1}{\cal I}}}\def\elim#1{{{#1}{\cal E}}}%%
%% Recall that in \TeX\ the logical connectives and quantifiers are called
%%  \begin{center}
%%    \verb/\lor/ $\lor$\quad
%%    \verb/\land/ $\land$\quad
%%    \verb/\lnot/ $\lnot$\quad
%%    \verb/\forall/ $\forall$\quad
%%    \verb/\exists/ $\exists$
%%  \end{center}
%% The following provide macros for the \verb/\implies/ {\em relation\/}
%% and for the binary {\em operation\/} which yields the abstract
%% \verb/\implic/ation between formulae.
%% The point is that \TeX\ spaces them and breaks the lines differently:
%%  \begin{center}
%%    \verb/A\implies B/ $A\implies B$\quad{\em versus\/}\quad
%%    \verb/A\implic B/ $A\implic B$
%%  \end{center}
%% There are forward and reverse, single and Double versions.
%%
\mathchardef\implies="3221 \mathchardef\impliedby="3220 \mathchardef\Implies=%
"3229 \mathchardef\Impliedby="3228 \mathchardef\implic="2221 \mathchardef
\implicby="2220 \mathchardef\Implic="2229 \mathchardef\Implicby="2228 \let
\imp\to%%
%% Handle the spacing after a variable (and optionally its type)
%% bound by a quantifier symbol. For example
%%  \begin{quote}
%%    \verb/\All x:X. \phi(x)/\quad prints as\quad $\All x:X.\phi(x)$\quad
%%     instead of\quad $\forall x:X.\phi(x)$
%%  \end{quote}
\def\Quantifier#1#2.{\pb@G{#1}#2::.}\def\pb@G#1#2:#3:#4.{{{#1}{#2}\def\next{#%
3}\ifx\next\empty{.}\mkern1mu \else\mkern1mu{\colon}\mkern1mu{#3}\mkern.5mu{.%
}\mkern3mu \fi}}%
%% We provide some commonly used forms; \verb/\iota/ ($\iota$) is Russell's
%% description operator and should really be inverted.
\def\All{\Quantifier\forall}\def\Some{\Quantifier\exists}\def\Function{%
\Quantifier\lambda}\def\Product{\Quantifier\Pi}\def\Sum{\Quantifier\Sigma}%
\def\TheOne{\Quantifier\iota}\def\Least{\Quantifier\mu}\def\Greatest{%
\Quantifier\nu}%%
%% There are several notations for substitution.
%% After writing $a[x:=b]$ throughout my book I~thought I~might change to
%% $[b/x]^*a$.
%% This macro reads the source in the first form and prints in the second.
%% If you use it you can, like me, defer the decision about
%% which notation to use until the final stages, doing
%% \begin{quote}
%%    \verb/\renewcommand{\Subst}{\plainsubstitution}/
%% \end{quote}
%% if you finally decide on making substitution act on the right.
%% This is already an improvement on the literal text, because it
%% automatically enlarges the brackets according to the text inside.
%% \verb/\Subst/ itself is (following my book)
%% defined in terms of the action of a context morphism (\verb/\CtxtMor/)
%% on a term. Again you can do
%% \begin{quote}
%%    \verb/\renewcommand{\CtxtMor}{\plaincontextmorphism}/
%% \end{quote}
%% for something simpler.
%% This macro interprets its argument as a comma-separated list
%% of items in the form $x:=b$, which it switches to $b/x$.
\def\swappingsubstitution#1[#2]{{\CtxtMor[#2]^*{#1}}}\def
\swappingcontextmorphism[#1]{{\let\pb@J\empty\left[\pb@H#1,[],\/\right]}}\def
\pb@H#1,{\def\next{#1}\ifx\next\pb@K\else\pb@J\let\pb@J,\pb@I#1:=:=,%
\expandafter\pb@H\fi}\def\pb@I#1:=#2:=#3,{{\def\next{#2}\ifx\next\empty\else{%
#2}/\fi{#1} }}\def\pb@K{[]}\def\Subst{\swappingsubstitution}\def\CtxtMor{%
\swappingcontextmorphism}%%
%% The simple versions.
\def\plainsubstitution#1#2{{#1}{\left[{#2}\right]}}\def\plaincontextmorphism#%
1{{\left[#1\right]}}\newbox\pb@L\newdimen\pb@T\newdimen\pb@U\newdimen\pb@M
\newdimen\pb@N\newdimen\pb@O\newdimen\pb@P\newdimen\pb@Q\newdimen\pb@R
\newdimen\pb@S\def\pb@V#1#2{\relax\ifdim#1<#2\relax#1=#2\relax\fi}\count@
\year\multiply\count@12 \advance\count@\month\ifnum\count@> 23966 \message{%
because this one has expired and will no longer work!}\endinput\fi%%
\def\proofbox{\relax\ifmmode\let\next\pb@a\else\let\next\pb@b\ifhmode\ifinner
\let\next\pb@a\fi\fi\fi\next\global\prooflinetotal\@ne\boxmaxdepth\maxdimen
\def\[{\pb@w\pb@f\pb@W}\def\({\pb@w\pb@f\pb@X}\def\proofbox{\[}\let\lbl\pb@y
\def\"##1"{\pb@y{##1}}\def\:{\pb@v\proofboxproposition\displaystyle}\def\={%
\pb@v\proofboxreason}\def\-{\pb@v\proofboxuse}\def\\{\pb@w\pb@u}\let\+\pb@c
\let\*\pb@c\let\]\pb@c\pb@p\pb@s}\def\endproofbox{\pb@q\pb@IA\box\pb@L\egroup
}\def\pb@W{\let\+\pb@d\let\*\pb@d\def\){\pb@e\[\)}\let\]\pb@Y\proofcolumns\z@
}\def\pb@X{\proofcolumns\@ne\def\+{\pb@j\pb@l}\def\*{\pb@j\pb@m}\def\]{\pb@e
\(\]}\let\)\pb@Y}\def\pb@Y{\pb@o\pb@u}\def\pb@a{\hbox\bgroup}\def\pb@b{$$%
\kern-\displayindent\hbox to\hsize\bgroup\proofboxfullwidth\hsize\aftergroup$%
\aftergroup$}\def\pb@c{\errmessage{Box proofs: \string\+, \string\*, \string
\) and \string\] may not be used at the top level}}\def\pb@d{\errmessage{Box
proofs: enclose \string\+ and \string\* in \string\(\string\), not \string\[%
\string\]}}\def\pb@e#1#2{\errmessage{Box proofs: \string#1 closed by \string#%
2}\pb@Y}\def\pb@f#1{\setbox\pb@L\hbox\bgroup#1\let\pb@g\pb@n\let
\enclosingproofline\@currentlabel\edef\pb@h{\the\prooflinetotal}\let\pb@i
\pb@h\pb@O\z@\pb@M\z@\pb@N-\maxdimen\pb@P\z@\pb@p\pb@t}\def\pb@j#1{\pb@q\pb@k
\let\pb@g#1\advance\proofcolumns\@ne\global\prooflinetotal\pb@h
\leftmostproofboxfalse\pb@p\pb@t}\def\pb@k{\pb@g\pb@V\pb@M{\wd\pb@L}\advance
\pb@O\wd\pb@L\ifnum\pb@i<\prooflinetotal\edef\pb@i{\the\prooflinetotal}\fi
\ifdim\pb@N<\dp\pb@L\pb@N\dp\pb@L\pb@P\pb@Q\fi\setbox\z@\hbox{%
\proofboxtopside}\wd\z@\z@\box\z@\setbox\z@\hbox{\proofboxbottomside}\wd\z@
\z@\box\z@\kern\pb@R\box\pb@L\kern\pb@S}\def\pb@l{\vtop{\proofboxorseparator}%
\penalty\thr@@}\def\pb@m{\vtop{\proofboxrightside}\kern\proofboxseparation
\vtop{\proofboxleftside}\penalty\tw@}\def\pb@n{\pb@T\pb@R\setbox\z@\vtop{%
\proofboxleftside}\dimen@\wd\z@\advance\pb@T\dimen@\advance\pb@T
\proofboxseparation\kern-\pb@T\kern\proofboxseparation\box\z@\penalty\@ne}%
\def\pb@o{\pb@q\pb@k\setbox\z@\vtop{\proofboxrightside}\global\advance\pb@S
\wd\z@\box\z@\global\advance\pb@S\proofboxseparation\kern\proofboxseparation
\global\prooflinetotal\pb@i\global\pb@R\pb@T\global\pb@Q\pb@P\ifnum
\proofcolumns<\@ne\proofcolumns\@ne\fi\penalty\proofcolumns\kern-\pb@M\kern
\pb@M\kern-\pb@O\kern\pb@O\kern-\pb@S\penalty\@ne\egroup\pb@r}\def\pb@p#1{%
\setbox\pb@L\vtop\bgroup\baselineskip\proofboxbaselineskip\prooflineinbox\@ne
\pb@T\z@\pb@U\z@#1\pb@u}\def\pb@q{\pb@w\global\pb@R\pb@T\global\pb@S\pb@U
\egroup}\def\pb@r{\pb@V\pb@T\pb@R\pb@V\pb@U\pb@S\box\pb@L\prevdepth\pb@Q}\def
\pb@s{}\def\pb@t{}\def\pb@u{\edef\@currentlabel{\theproofline}\let\label@name
\@currentlabel\setbox\proofboxproposition\box\voidb@x\setbox\proofboxreason
\box\voidb@x\setbox\proofboxuse\box\voidb@x\setbox\proofboxvariables\hbox
\bgroup\let\pb@z\pb@AA\let\pb@w\pb@x$}\def\pb@v#1{$\egroup\setbox#1\hbox
\bgroup\let\pb@z\pb@AA$}\def\pb@w{$\egroup\global\advance\prooflinetotal\@ne
\advance\prooflineinbox\@ne\setbox\pb@L\hbox{\setbox\z@\hbox{%
\proofboxmakeleftcolumn}\global\pb@R\wd\z@\kern-\pb@R\box\z@\hbox{%
\proofboxmakemiddlecolumn}\setbox\z@\hbox{\proofboxmakerightcolumn}\global
\pb@S\wd\z@\box\z@\kern-\pb@S\penalty\tw@}\global\pb@Q\dp\pb@L\pb@r}\def\pb@x
{$\egroup}\def\pb@y#1{\aftergroup\pb@z\expandafter\aftergroup\csname r@#1%
\endcsname}\def\pb@z#1{\edef#1{{\@currentlabel}{}}\expandafter\pb@@A\string#1%
\\}\expandafter\def\expandafter\pb@@A\string\r@#1\\{\def\label@name{#1}}\def
\pb@AA#1{\aftergroup\pb@z\aftergroup#1}\expandafter\def\csname r@-\endcsname{%
{\previous@label}{}}\dimendef\pb@BA=\@ne\dimendef\pb@CA=\tw@\dimendef\pb@DA=%
\thr@@\dimendef\pb@EA=4 \dimendef\pb@FA=5 \dimendef\pb@GA=6 \chardef\pb@HA=9
\def\pb@IA{\pb@GA\proofboxfullwidth\pb@T\proofboxleftmargin\pb@V\pb@T\pb@R
\advance\pb@GA-\pb@T\pb@U\proofboxrightmargin\pb@V\pb@U\pb@S\advance\pb@GA-%
\pb@U\pb@V\pb@GA{\wd\pb@L}\pb@JA}\def\pb@JA{\setbox\pb@L\vtop{\unvbox\pb@L
\loop\skip@\lastskip\unskip\setbox\pb@HA\lastbox\ifhbox\pb@HA\pb@KA\repeat
\unvbox\pb@L\unskip}}\def\pb@KA{\pb@DA\dp\pb@HA\pb@EA\ht\pb@HA\pb@CA\wd\pb@HA
\setbox\pb@HA\hbox{\unhbox\pb@HA\count@\lastpenalty\unpenalty\unkern\ifodd
\count@\pb@LA\else\setbox\thr@@\lastbox\setbox\tw@\lastbox\setbox\@ne\lastbox
\unkern\hbox to\pb@T{\unhbox\@ne}\hbox to\pb@GA{\unhbox\tw@}\hbox to\pb@U{%
\unhbox\thr@@}\fi}\dp\pb@HA\pb@DA\ht\pb@HA\pb@EA\setbox\pb@L\vtop{\box\pb@HA
\nointerlineskip\vskip\skip@\unvbox\pb@L}}\def\pb@LA{\pb@O\lastkern\unkern
\unkern\pb@M\lastkern\unkern\unkern\proofcolumns\lastpenalty\unpenalty\pb@FA
\pb@T\ifnum\@ne=\proofcolumns\let\pb@MA\pb@NA\else\advance\pb@GA-\pb@CA\pb@BA
\pb@GA\advance\pb@GA\pb@O\ifdim\pb@GA<\proofcolumns\pb@M\divide\pb@BA
\proofcolumns\let\pb@MA\pb@PA\else\divide\pb@GA\proofcolumns\let\pb@MA\pb@OA
\fi\fi\dimen@\lastkern\unkern\setbox\@ne\lastbox\pb@RA\setbox\pb@HA\hbox{\box
\@ne\kern\dimen@}\advance\pb@U-\wd\pb@HA\unkern\loop\setbox\pb@L\lastbox\pb@T
\lastkern\unkern\setbox\tw@\lastbox\setbox\thr@@\lastbox\count@\lastpenalty
\unpenalty\pb@MA\pb@JA\setbox\@ne\vbox{\offinterlineskip\pb@QA\thr@@\vfil
\pb@QA\tw@}\pb@RA\pb@TA{\box\@ne\box\pb@L}\ifnum\count@>\@ne\pb@SA\pb@U
\lastkern\unkern\repeat\unhbox\pb@HA}\def\pb@NA{\pb@T\pb@FA\setbox\@ne
\lastbox\advance\pb@T-\wd\@ne\dimen@\lastkern\unkern\unkern\kern\dimen@\pb@RA
\box\@ne\advance\pb@T-\dimen@}\def\pb@OA{\ifnum\@ne=\count@\pb@NA\fi}\def
\pb@PA{\pb@GA\wd\pb@L\advance\pb@GA\pb@BA\pb@OA}\def\pb@QA#1{\setbox#1\hbox to%
\wd\pb@L{\unhbox#1}\wd#1\z@\box#1\relax}\def\pb@RA{\setbox\@ne\vbox to\pb@EA{%
\offinterlineskip\vskip-\proofboxseparation\unvbox\@ne\vskip-%
\proofboxseparation\setbox\z@\null\ht\z@-\pb@DA\dp\z@\pb@DA\box\z@}}\def
\pb@SA{\setbox\@ne\lastbox\pb@RA\pb@TA{\box\@ne}\ifodd\count@\else\dimen@
\lastkern\unkern\setbox\@ne\lastbox\pb@RA\pb@TA{\box\@ne\kern\dimen@}\fi}\def
\pb@TA#1{\setbox\pb@HA\hbox{#1\unhbox\pb@HA}}\let\+\pb@A\proofboxrestoreat
\endinput\def\x{}%%
%% \end{document}
% \fi
