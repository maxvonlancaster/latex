%%% SHADE.rdm
%%% ReadMe file for SHADE.mf and SHADE.tex/sty
%%% -------------------------------------------------------------------------
%%% Version 1 - March 21, 1993
%%% Peter Schmitt, Institute of Mathematics, University of Vienna
%%%                Strudlhofgasse 4, A-1090 Wien, Austria
%%%         e-mail A8131DAL@helios.edvz.univie.ac.at
%%%                schmitt@awirap.bitnet
%%% -------------------------------------------------------------------------

\magnification=\magstep1
\parindent0pt \parskip 3pt
\input shade.tex

{\tt SHADE} is a \TeX\ tool that can be used to produce shaded boxes
neither using PostScript nor Pic\TeX.

% The syntax is:

\shade{This text will appear on a shaded background!}

The shading macro first packs its argument into an hbox.

\shade{\vbox{\hsize 5cm
       \leftskip=0pt plus 1fil\rightskip=0pt plus 1fil
       \parfillskip=0pt
       Then it measures the dimensions of this box and produces a
       shaded box slightly larger then this box, using the font
       shade produced by {\tt shade.mf}. Finally it
       puts the box back and overlays it with the shaded box.
       \par
      }}

Since the best shading is achieved at the lowest possible resolution,
the font \shade{SHADE} is {\bf not device independent} --- it uses
the printer resolution. Therefore the .dvi file is
{\bf not device independent}, either.
The macro code, however, {\bf is device independent}.
This method was used because otherwise shading, and in particular
the boundaries between two shading characters, could turn out to be not
completely uniform.
For the same reason, the font should only be used at its true size.

The {\tt shade} package consists of three files:
    {\tt shade.mf,           % MetaFont file for shading font
         shade.tex,          % macro file be used with LaTeX, too
         shade.rdm.          % this file
                            % it can be processed by TeX as a demo file
       }
It was tested with a HP Laserjet II plus.

\bye
