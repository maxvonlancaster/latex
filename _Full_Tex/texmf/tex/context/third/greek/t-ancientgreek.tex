%D \module
%D   [       file=t-ancientgreek,
%D        version=2006.04.05,
%D          title=Ancient Greek,
%D       subtitle=Publications,
%D         author=Thomas A. Schmitz, Taco Hoekwater
%D           date=\currentdate,
%D      copyright=Thomas A. Schmitz]
%C
%C This program is free software; you can redistribute it and/or modify it under
%C the terms of the GNU General Public License as published by the Free Software
%C Foundation; either version 2 of the License, or (at your option) any later
%C version.

%D Module t-ancientgreek providing method for polytonic Greek input in
% \CONTEXT, based on typescripts.

\writestatus{loading}{module ancientgreek}

\startmodule[ancientgreek]

\unprotect

%D The new mechanism for modules allows setting values. We need to provide
%D default setups for fonts and scaling. The \tex{setupgreek} command is not
%D needed any longer; we now set these values when we call the module in this
%D form:
%D \starttyping
%D \usemodule[ancientgreek][font=KadmosNew,scale=1.2,altfont=BosporosNew,altscale=1]
%D \stoptyping

\setupmodule[scale=1,font=Ibycus,altfont=Leipzig,altscale=1]

%D Next, we load the appropriate unicode vector for polytonic Greek.

\useunicodevector[31]

%D We use the encoding for ancient Greek.

\useencoding[agr]

%D We load the typescript file that contains all definitions for Greek fonts.

\usetypescriptfile[type-agr.tex]

%D The next lines define a Metric font. This is just a quick and dirty hack to
%D get some metric symbols; we may some day write proper code for it. For the
%D time being, no frills are needed here since this will always be used for a
%D few characters only.

%\definefontsynonym[MFont][Anaxiphorminx]
\definefont[metre][Anaxiphorminx sa 1.2]

\define\Mf{\groupedcommand{\metre}{}}

\define\aeolbase{\getglyph{Anaxiphorminx}{\char227}\getglyph{Anaxiphorminx}{\char227}}

\define\acephal{\getglyph{Anaxiphorminx}{\char142}}

%D We now define the bodyfont for our font and our altfont:

\starttypescript[greek]
\definetypeface[greek][rm][serif][\currentmoduleparameter{font}][default][encoding=agr,rscale=\currentmoduleparameter{scale}]
\stoptypescript
\usetypescript[greek]

\starttypescript[altgreek]
\definetypeface[altgreek][rm][serif][\currentmoduleparameter{altfont}][default][encoding=agr,rscale=\currentmoduleparameter{altscale}]
\stoptypescript
\usetypescript[altgreek]

%D The following code was suggested by Hans to take care of problems with the
%D catcode changes within table environments: 

\long\def\rescanwithsetup#1#2%
   {{\directsetup{#1}\scantokens{#2\ignorespaces}}} 

%D We wrap everything that is needed for obtaining Greek output into a setup
%D buffer: change the catcodes of accents, switch to the language ancientgreek
%D so we get proper hyphenation, and switch to the proper bodyfont:
   
\startsetups[enablegreek]%
   \catcode`~=\other% 
   \catcode`|=\other%
   \catcode`'=\other% 
   \language[agr]%
   \switchtobodyfont[greek]%
\stopsetups 
   
\startsetups[enablealtgreek]%
   \catcode`~=\other% 
   \catcode`|=\other%
   \catcode`'=\other% 
   \language[ancientgreek]%
   \switchtobodyfont[altgreek]%
\stopsetups 

%D Finally, we define a startstop environment and a command to enable these
%D setups:

\definestartstop[greek][commands=\directsetup{enablegreek}]

\definestartstop[altgreek][commands=\directsetup{enablealtgreek}]

\def\localgreek{\rescanwithsetup{enablegreek}}

\def\localaltgreek{\rescanwithsetup{enablealtgreek}}

\protect
\stopmodule
\endinput
