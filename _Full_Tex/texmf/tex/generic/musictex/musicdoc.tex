\documentstyle[twoside,multicol,musictex,musictrp,musicper,a4report,11pt]{report}

\def\hboxit#1{\hbox{\vrule\vbox{\hrule\kern3pt
    \hbox{\kern3pt\small\it #1\kern3pt}\kern3pt\hrule}\vrule}}
%
\def\tty{\csname normalshape\endcsname\csname mediumseries\endcsname\tt}
%
\def\text#1{\leavevmode\hbox{\rm #1}}%
%\check
\def\ital#1{{\sl #1\/}}%
\def\mutex{M\raise 2pt\hbox{\kern -1pt u\kern -1pt}\TeX}
\def\bslash{{\tty\char'134}}%
\def\unix{{\sc unix}}%
\def\Bslash{\tty\char'134}%
\def\|{{\tty\char'174}}%
\def\#{{\tty\char'043}}%
\def\{{{\char'173}}%
\def\}{{\char'175}}%
\def\musictex{Music\TeX}%
\def\musixtex{MusiX\TeX}%
\def\musicxx{\musictwenty}
\def\keyindex#1{\leavevmode
 \hbox{\tty\bslash #1}\index{#1=\protect\tty\protect\bslash #1}}
\def\zkeyindex#1{\index{#1=\protect\tty\protect\bslash #1}}
\def\ixem#1{#1\index{#1}}
\def\itxem#1{\ital{#1}\index{#1}}
\def\aujourdhui{\today}
%
%\check
\tracingstats=1
%
%%%%   \input musicext  : extracting the wanted macro

\makeatletter

\def\tmp@dimenb{\y@ii}
\newdimen\tmp@dimen
% \overbracket{p}{l}{s}   draws a bracket over the music starting at the
% current position at pitch p, width l and slope s percent, causing no space.
\def\overbracket#1#2#3{\relax % height (note), length (dimen) slope (%)
   \tmp@dimenb #2\multiply\tmp@dimenb by#3\divide\tmp@dimenb by100\relax
   \getn@i{#1}\advance\tmp@dimenb by\n@i\internote\rlap{\relax
   \raise\n@i\internote\rlap{\vrule width\lthick height \lthick
   depth .8\Interligne}\oblique{#2}{#3}{\n@i\internote}\relax
   \advance\tmp@dimenb by.5\lthick\relax
   \raise\tmp@dimenb\hbox{\vrule width\lthick height \lthick
 depth.8\Interligne}}}

% \ovbkt{p}{n}{s} is the same as \overbracket, except that it draws the
% bracket to cover n notes (note however that glue inserted by \temps
% commands will expand the space between the notes but will not affect the
% bracket).
% I use this macro to indicate triplets, for instance.

\def\ovbkt#1#2#3{\relax % height (note), length (number of notes) slope (%)
   \tmp@dimen #2\noteskip\advance\tmp@dimen by\wd@skip\advance\tmp@dimen
   by -\noteskip\advance\tmp@dimen by\lthick \overbracket{#1}{\tmp@dimen}{#3}}

% \underbracket and \unbkt are similar to the above, but produce
% brackets under the music.
\def\underbracket#1#2#3{\relax % height (note), length (dimen) slope (%)
   \tmp@dimenb #2\multiply\tmp@dimenb by#3\divide\tmp@dimenb by100\relax
   \getn@i{#1}\advance\tmp@dimenb by\n@i\internote\rlap{\relax
   \raise\n@i\internote\rlap{\vrule width\lthick depth -\lthick
   height \Interligne}\oblique{#2}{#3}{\n@i\internote}\relax
   \advance\tmp@dimenb by.5\lthick\relax
   \raise\tmp@dimenb\hbox{\vrule width\lthick depth -\lthick
 height\Interligne}}}
\def\unbkt#1#2#3{\relax % height (note), length (number of notes) slope (%)
   \tmp@dimen #2\noteskip\advance\tmp@dimen by\wd@skip\advance\tmp@dimen
   by -\noteskip\advance\tmp@dimen by\lthick
   \kern-2\lthick\underbracket{#1}{\tmp@dimen}{#3}\kern2\lthick}


\newdimen\tmp@dimenc
\newdimen\z@iii\newdimen\z@iv\newdimen\z@v

% \oblique {l}{s}{h} draws an oblique line of length l, slope s percent, and
% height h. This will cause an unrecorded space so it should be used within
% \rlap. Note that this macro is a hack and probably gobbles up resources!
% This macro is used by some of the others which follow.

\def\oblique#1#2#3{\relax % length slope height
\ifnum #2=0\relax\raise #3\vbox{\hrule width #1 height\lthick depth\z@}\relax
\else\q@antum=25\lthick\divide\q@antum by #2\relax
\ifdim\q@antum<0pt\relax\multiply\q@antum by -1\fi
\global\z@iv=0pt\relax
\loop\ifdim\z@iv<#1\relax
 {\z@v=#1\relax\advance\z@v by -\z@iv\relax\advance\z@v by -\q@antum\relax
 \tmp@dimenc\z@iv\multiply\tmp@dimenc by#2\relax
 \z@iii=#3\relax\advance\z@iii by .01\tmp@dimenc\relax\advance\z@iii by
 -0.5\internote
 \ifnum #2<0\relax\advance\z@iii by \lthick\relax\fi
 \tmp@dimenc\z@v\multiply\tmp@dimenc by#2\relax
 \ifdim\z@v<0pt\relax\advance\z@iii by .01\tmp@dimenc\relax\hskip\z@v\fi
 \advance\z@iii by 0.6\internote\relax
 \raise\z@iii\hbox to \q@antum{\vrule width\q@antum height .5\lthick depth
 .5\lthick}\relax
 \global\advance\z@iv by \q@antum\relax
}\repeat
\fi
}

%%%%%%%%%%%%%%%%%%%%%%%%%%%%%%%%%%%%%%% end musicext excerpt
\font\musext=musicbrb
\tolerance 10000
\csname normalshape\endcsname\csname mediumseries\endcsname
\csname pnormalshape\endcsname\csname pmediumseries\endcsname
\makeindex

\long\def\theindex{\chapter{Index}
\parindent \z@
\parskip \z@ \@plus .3\p@ \relax \let \item \@idxitem \columnseprule \z@
\columnsep 35\p@ \begin{multicols}{2}
}

\def\endtheindex{\end{multicols}}

%
\begin{document}
\title{\Huge\bf \musictex\\\LARGE\bf Using
 \TeX\ to write polyphonic\\or
instrumental music\\\Large\sl Version  5.17 -- \today}
\author{\Large\rm Daniel \sc Taupin\\\large\sl
 Laboratoire de Physique des Solides\\\normalsize\sl
 (associ\'e au CNRS)\\\normalsize\sl
 b\^atiment 510, Centre Universitaire, F-91405 ORSAY Cedex}
 \date{}
\maketitle
\clearpage
\thispagestyle{empty}
\null
\setcounter{page}0
\clearpage
\tableofcontents
\setcounter{secnumdepth}3
\pagestyle{headings}
%
\chapter{What is \musictex\ ?}
 \musictex\ is a set of \TeX\ macros to typeset polyphonic, orchestral or
polyphonic music. Therefore, it is mainly supposed to be used to type wide
scores -- just because true musicians seldom like to have to frequently turn
pages -- and this is not really compatible with \LaTeX's standard page
formats, even the {\tty A4.sty} the {\Bslash textheight} and  {\Bslash
textwidth} of which are too small for musician needs.

  However, a \LaTeX\ style has been also provided (and it is used for the
typing of the present paper) but this {\tty musictex} style is fit for
musicographic books  rather than for normal scores to be actually played.

 It should be emphasized that \musictex\ is not intended to be a
compiler which would translate some \ixem{standard musical
notation}s into \TeX\, nor to decide by itself about aesthetic problems in music typing.
\musictex\ only typesets staves, notes, chords, beams, slurs and ornaments as
requested by the composer. Since it makes very few typesetting decisions,
\musictex\ appears to be a versatile and rather powerful tool. However, due
to the important amount of information to be provided to the typesetting
process, coding \musictex\ might appear to be awfully complicated, just as
the real keyboard or orchestral music. Therefore, it should be interfaced therefore by
some pre-compiler in the case of the composer/typesetter wanting aesthetic
decisions to be automatically made by somebody (or something) else.

\section{\musictex\ principal features}
\subsection{Music typesetting is two-dimensional}

 Most of the people who just learnt a bit of music at college probably think
that music is a linear sequence of symbols, just as literary texts to be
\TeX-ed. In fact, with the exception of strictly monodic instruments like
most orchestral wind instruments and solo voices, one should be aware that
reading music actually is a matricial operation: the non-soloist musician
successively reads \ital{columns} of simultaneous notes which he actually plays
if he is a pianist, clavichordist or organist, which he actually reads and
watches if he conducts an orchestra, and which he is supposed to check and
partially play when he is a soloist who wants to play in time with the
accompanying instrument or choir.

 In fact, our personal experience of playing piano and organ as well as
sometimes helping as an alternate Kapellmeister leads us to think that actual
music reading and composing is a slightly more complicated intellectual
process: music reading, music composing and music thinking seems to be a
three layer process. The musician usually reads or thinks several consecutive
notes (typically a long \ital{beat} or a group of logically connected notes), then
he goes down to the next instrument or voice and finally assembles the whole
to build a part of music lasting roughly a few seconds. Then he handles the
next \ital{beat} or \ital{bar} of his score.

  Thus, it appears that the humanly logical way of coding music consists
in horizontally accumulating a set of \ital{vertical combs} with
\ital{horizontal teeth} as described in Table \ref{readtable}.

 \begin{table}
 \begin{center}
 \begin{tabular}{|ll|ll|}\hboxit{note sequence one}
  &\hboxit{note seq.\ four}
  &\hboxit{note seq.\ seven}
  &\hboxit{note seq.\ ten}\\\hboxit{note sequence two}
  &\hboxit{note seq.\ five}
  &\hboxit{note seq.\ eight}
  &\hboxit{note seq.\ eleven}\\\hboxit{note seqence three}
  &\hboxit{note seq.\  six}
  &\hboxit{note seq.\  nine}
  &\hboxit{note seq.\  twelve}\\
 \end{tabular}
 \end{center}
 \caption{The order in which a musician reads music}\label{readtable}
 \end{table}

This is the reason
why, in {\bf \musictex} the fundamental \ital{macro} is of the form

\begin{center}
{\Bslash notes \dots\ \& \dots\ \& \dots\ \bslash enotes}
\end{center}

\noindent where the character {\tty\&} is used to separate the notes to be
typeset on respective staffs of the various instruments, starting from the
bottom.

In the case of an instrument whose score has to be written with
several staffs, these staffs are separated by the
character \|. Thus, a score written for a keyboard instrument and
a monodic instrument (for example piano and violin)
will be coded as follows:

\begin{center}
{\Bslash notes \dots\ \| \dots\ \& \dots\ \bslash enotes}
\end{center}
\noindent for each column of simultaneous \ital{groups of notes}.
It is worth emphasizing that we actually said \ital{``groups of notes''}:
this means that in each section of the previous macro, the music typesetter is
welcome to insert, not only chord notes to be played at once, but small
sequences of consecutive notes which build something he understands as a
musical phrase. This is why note typing macros are of two kinds in \musictex,
namely the note macros which are not followed by spacing afterwards, and those
which induce horizontal spacing afterwards.

\subsection{The spacing of the notes}

It seems that many books have dealt with this problem.
Although it can lead to interesting algorithms, we think it is
not in practice an important point.

In fact, each column of notes has not necessarily the same spacing
and, in principle, this \itxem{spacing} should depend on the shortest
duration of the simultaneous notes. But this cannot be established as a rule,
for at least two reasons:

\begin{enumerate}
 \item spacing does not depend only on the local notes,
but also on the context, at least in the same bar.
\item in the case of polyphonic music, exceptions can easily be found.
Here is an example:


 \begin{music}
\def\nbinstruments{1}\relax
\computewidths
\generalmeter{\meterfrac{4}{4}}\relax
\debutextrait
\normal\elemskip=0.110\hsize
\temps\Notes\qsk\qsk\rlap{\hu j}\ql h\enotes
\temps\Notes\hl g\enotes
\temps\Notes\hu k\enotes
\temps\Notes\ql f\enotes
\finextrait
 \end{music}

%\check

\noindent where it can be clearly seen that the half notes at beats 2 and 3
must be spaced as if they were quarter notes since they overlap, which is
obvious only because of the presence of the indication of the \itxem{meter} 4/4.
\end{enumerate}
%

Therefore, we preferred to provide the composer/typesetter with a
set of macros, the spacing of which increases by a factor of $\sqrt 2$
(incidentally, this can be adjusted):

\begin{center}
{\Bslash notes \dots\ \& \dots\ \& \dots\ \bslash enotes\ \ \%
\hbox to 2.5cm{\rm 1 spatial unit\hss}}
\\
{\Bslash Notes \dots\ \& \dots\ \& \dots\ \bslash enotes\ \ \%
\hbox to 2.5cm{\rm 1.4 spacial unit\hss}}
\\
{\Bslash NOtes \dots\ \& \dots\ \& \dots\ \bslash enotes\ \ \%
\hbox to 2.5cm{\rm 2 spatial units\hss}}
\\{\Bslash NOTes \dots\ \& \dots\ \& \dots\ \bslash enotes\ \ \%
\hbox to 2.5cm{\rm 2.8 spatial units\hss}}
\\{\Bslash NOTEs \dots\ \& \dots\ \& \dots\ \bslash enotes\ \ \%
\hbox to 2.5cm{\rm 4 spatial units\hss}}
\\{\Bslash NOTES \dots\ \& \dots\ \& \dots\ \bslash enotes\ \ \%
\hbox to 2.5cm{\rm 5.6 spatial units\hss}}
\end{center}

%\check
 The size of the spatial unit (\keyindex{elemskip}) can be freely adjusted.
In addition, \musictex\ provides a means of adjusting the note spacing
according to an average number of elementary spaces within a line (macro
\keyindex{autolines}).
 
\subsection{Music tokens, rather than a readymade generator}
%\check

The tokens provided by \musictex\ are:

\begin{itemize}
\item the note symbols \ital{without stem};
\item the note symbols \ital{with stems, and hooks for eighth notes and
beyond};
\item the indications of beam beginnings and beam ends;
\item the indications of beginnings and ends of ties and slurs;
\item the indications of accidentals;
\item the ornaments: arpeggios, trills, mordents, pinc\'es, turns, staccatos
and pizzicatos, fermatas;
\item the bars, the meter and signature changes, etc.
\end{itemize}

 Thus, {\Bslash wh a} produces an \ital{A (of nominal frequency 222.5~Hz,
unless transposed)} of duration being a
\ital{whole note}. In the same way, {\Bslash wh h} produces an \ital{A
(445~Hz)} of duration represented by a \ital{whole note}, {\Bslash qu~c}
produces a \ital{C (250~Hz approx.)} whose value is a \ital{quarter note with
stem up}, {\Bslash cl J} produces a \ital{C (125~Hz approx.)} of
duration equal to an \ital{eighth note with stem down}, etc.

 To generate quarter, eighth, sixteenth, etc. chords, the macro
\keyindex{zq} can be used: it produces a quarter note head the vertical
position of which is
memorized and recalled when another stemmed note (possibly with a hook) is
coded; then the stem is adjusted to link all simultaneous notes. Thus, the
perfect C-major chord, i.e.

%\check
{\def\nbinstruments{1}\relax
\cleftoksi={{0}{0}{0}{0}}\generalmeter{}\relax
\debutextrait
\normal
\NOTes\qsk\zq c\zq e\zq g\qu j\enotes
\finextrait}

%\check
\noindent is coded
 \hbox{\Bslash zq~c\bslash zq~e\bslash zq~g\bslash qu~j}
or, in a more concise way, \hbox{\Bslash zq\{ceg\}\bslash qu~j} (stem up):
in fact, single notes are treated\dots\ like one-note chords.

\subsection{Beams}
 \ital{Beams}\index{beams} are generated using macros which define their
beginning (at the current horizontal position), together with their altitude,
their direction (upper of lower), their multiplicity, their slope and their
reference number. This latter feature -- the reference number -- appears to
be necessary, since one may want to write beams whose horizontal extents
overlap: therefore, it is necessary to specify which beam the notes hang on
and which beam is terminated at a given position.

%\check
\subsection{Setting anything on the score}

A general macro (\keyindex{zcharnote}) provides a means of
putting any sequence of symbols (in fact, some {\Bslash hbox\{...\}}) at any
pitch of any staff of any instrument. Thus, any symbol defined in a font
(letters, math symbols, etc.) can be used to typeset music.

\section{A simple example}

 Before giving more details, we give below an example of the two first
bars of the sonata in C-major K545 by {\sc\ixem{Mozart}}:

\begin{music}
\parindent 1cm
\def\nbinstruments{1}\relax
\def\instrumenti{Piano}%
\nbporteesi=2\relax
\generalmeter{\meterfrac{4}{4}}\relax
\debutextrait
\normal
\temps\Notes\ibu0f0\qh0{cge}\tbu0\qh0g|\hl j\enotes
\temps\Notes\ibu0f0\qh0{cge}\tbu0\qh0g|\ql l\sk\ql n\enotes
\barre
\Notes\ibu0f0\qh0{dgf}|\qlp i\enotes
\notes\tbu0\qh0g|\ibbl1j3\qb1j\tbl1\qb1k\enotes
\temps\Notes\ibu0f0\qh0{cge}\tbu0\qh0g|\hl j\enotes
\finextrait
\end{music}
%\check

 The \ital{coding} is set as follows:

\begin{quote}\begin{verbatim}
\begin{music}
\parindent 1cm
\def\nbinstruments{1}\relax % a single instrument
\def\instrumenti{Piano}%    % whose name is Piano
\nbporteesi=2\relax         % with two staffs
\generalmeter{\meterfrac{4}{4}}\relax % 4/4 meter chosen
\debutextrait               % starting real score
\normal                     % normal 12 pt note spacing
\temps\Notes\ibu0f0\qh0{cge}\tbu0\qh0g|\hl j\enotes
\temps\Notes\ibu0f0\qh0{cge}\tbu0\qh0g|\ql l\sk\ql n\enotes
\barre                      % bar
\Notes\ibu0f0\qh0{dgf}|\qlp i\enotes
\notes\tbu0\qh0g|\ibbl1j3\qb1j\tbl1\qb1k\enotes
\temps\Notes\ibu0f0\qh0{cge}\tbu0\qh0g|\hl j\enotes
\finextrait                 % terminate excerpt
\end{music}
 \end{verbatim}\end{quote}

\begin{itemize}
\item {\Bslash ibu0f0} begins an upper beam, aligned on the
\ital{f}, reference number 0, slope 0
\item {\Bslash tbu0} terminates this beam before writing
the second \ital{g} by means of {\Bslash qh0g}
\item {\Bslash qh..} indicates a note hanging on a beam.
 \item {\Bslash sk } sets a space between the two quarters at the right
hand, so that the second is aligned with the third eighth of the left hand.
 \item{\Bslash qlp} is a  quarter with a point.
 \item{\Bslash ibbl1j3} begins
a double beam, aligned on the \ital{C} ({\tty j} at this pitch)
of slope 0.15.
\end{itemize}

\section{Some highlights}
\subsection{Signatures}

Signatures can be stated either for all instruments, for example by
\keyindex{generalsignature}{\tty\{-2\}} which sets two flats on each staff, or
separately for each instrument. Thus, the \keyindex{generalsignature}
can be partly overridden by \keyindex{signii}{\tty=1} which puts one
sharp on the staffs of \itxem{instrument number} 2 (ii). Of course, the
current signature may change at any time as well as the meters and clefs.

\subsection{Transposition}

 An important question  is: \ital{``can \musictex\ transpose a score~?''}.
The answer is now 99.5~\% \ital{yes}. If fact, there is an internal register
named \keyindex{transpose} the default value of which is zero, but it may be
set to any reasonable positive of negative value. In that case, it offsets all
symbols pitched with letter symbols by that number of pitch steps. However,
it will neither change the signature nor the local accidentals, and if -- for
example -- you transpose a piece written in $C$ by 1 pitch, \musictex\ will
not know whether you want it in $D\flat$, in $D$ or in $D\sharp$. This might
become tricky if accidentals occur within the piece, which might have to be
converted into flats, naturals, sharps or double sharps, depending on the new
chosen signature. To avoid this trouble, \ital{relative} accidentals
have been implemented, the actual output of which depends on the pitch of
this accidental and of the current signature.\index{relative accidentals}


\subsection{Selecting special instrument scores}

Another question is: \ital{``can I write an orchestral score and extract the
separate scores for individual instruments~?''} The answer is 95~\%
\ital{yes}: in fact, you can define your own macros {\Bslash
mynotes...\bslash enotes}, {\Bslash myNotes...\bslash enotes} with as many
arguments as there are in the orchestral score (hope this is less or equal to
9, but \TeX perts know how to work around) and change its definition
depending on the selected instrument (or insert a test on the value of some
selection register). But the limitation is that the numbering of instruments
may change, so that {\Bslash signiii} may have to become {\Bslash signi} if
instrument $iii$ is alone. But, in turn, this is not a serious problem for
average \TeX\ wizard apprentices.

 \subsection{Variable staff and note sizes}

 Although the staff size is 20 pt as a standard, \musictex\ allors scores of
16 pt staff sizes. In addition, any instrument may have a special staff size
(usually smaller than the overall staff size) and special commands
\verb|\smallnotesize| or \verb|\tinynotesize| enable notes (and also beams or
accidentals) to be of a smaller size, in order to quote optional notes or
\ital{cadenzas}.

 \section{How to get it}

 The whole \itxem{distribution} fits into a single 1.2Mbyte or
1.44Mbyte diskette. It can also be obtained through an \ital{anonymous ftp}
at {\tty rsovax.lps.u-psud.fr} (193.55.39.100), after selecting the
subdirectory {\tty [.musictex]}. All sources (including fonts) are provided,
either separately or ``zipped'' or as VMS ``\ixem{savesets}''.

%\check
\section{Enhancements}
\subsection{Recent easy enhancements}

Many enhancements have been asked for, and this is a proof that \musictex\ is
considered as useful by many people. Some of these enhancements which seemed
hard were in fact rather easy to implement, for example small notes to represent
grace notes and cadenzas. But others may induce heavy problems, for example
the need of having \ital{nice} slurs and ties.

In the same way, we recently introduce Andreas {\sc Egler}'s fonts to have
nicer braces at the left of the score, for example for piano music.

Besides, we recently found a hard incompatibility of \musictex\ and a genuine
product of the author's native country, namely {\tt french.sty} by Bernard
{\sc Gaulle} which is the \ital{standard} of the French \TeX\ user group,
namely the GUTenberg association. This was partly due to macro name collisions
--- easy to solve --- and to the fact that {\tt french.sty} sets a lot of
characters as \keyindex{active} characters in order to make them handle
correctly the French spacing before and after punctuation marks. This seems to
have been solved in the 4.99 version of \musictex, with the restriction that
French macros cannot be used --- at least easily --- within lyrics.

Besides, \musictex\ has been adapted --- since version 5.00 --- to score
\ital{gregorian chant} and \ital{percussion music}.

\subsection{The tie/slur problem}

While typesetting notes and even beams is a rather simple problem because it
is a \ital{local typesetting}, ties and slurs are much more difficult to handle.

Of course there is small problem in case of a typesetter wanting a slur or a
tie binding two consecutive notes, not separated by a bar. In practice this
\ital{very restricted} use of slurs or ties can easily be solved by putting
some symbols extracted from the {\tty slur16} or {\tty slurn16}/{\tty slurn20}
fonts somewhere on the staffs using the general use \keyindex{zcharnote}
macro.

 But serious music typesetters or composers know that many ties are supposed
to link notes which are on both sides of a bar, which is a likely place to
insert line breakings, so that the coding of \itxem{ties} must have various
versions and sizes to resist that possible line breaking. What has been said
about ties is still more serious in the case of \itxem{phrasing slurs} which
may extend over several bars, lines and sometimes pages. In this case, their
shape is not only a question of producing a long curved symbol of nice
looking shape, it also has to cope with \itxem{glue}. An then the worst is
that music way of typing does not accept \itxem{ragged lines} but equal
length lines, even for the last line of a music piece. Thus, long distance
slurs and ties need to be cut into separate parts (beginning, continuing(s),
endings) which \TeX\ can only link using \ital{horizontal line overlaps}
or \keyindex{leaders} to insure slur continuity over this unavoidable glue.


Therefore and up to now, ties and slurs have been implemented in a way which
may look rather ugly, but we think it is the only way of implementing
\ital{in one pass} ties and slurs which run \ital{across glue}. The principle
is to have tie/slur symbols with a rather long horizontal part.
Then, at each time a glue occurs and at each time a group of notes is coded
while a slur or tie is pending, an \keyindex{hrule} is issued which overlaps
the preceding tie/slur symbol so that the final output seems to contain a
continuous line. Unfortunately, this is possible only in the glue expansion
direction, namely in the horizontal direction.

Variable size
initial and final curved slur symbols have recently been implemented;
the user has to choose them according
to his intention to have short or long range slur symbols.

\subsection{\musixtex}

\ixem{\musixtex} is a new package, heavily derived from \musictex\ by Andreas
{\sc Egler} and Ross {\sc Mitchell}. Its fundamentals are taken from
\musictex, but it works in three passes~:\index{Egler, A.}\index{Mitchell, R.}
 \begin{enumerate}
  \item The first pass \TeX es the source with a different set of macros,
which generate a special file telling where slurs/ties start and begin.
  \item The second pass (a C program) computes the optimal length of the
slurs/recorded.
  \item The third pass generates the definitive DVI from the initial source,
owing to information produced by the second pass, name to choose the adequate
slur/tie symbols according to the chosen final note spacing.
 \end{enumerate}
 We do not decribe \musixtex\ here, but we give some suggestion to make the
same \musictex\ source runable with both \musictex\ and \musixtex.

\subsection{Enhancement limitations}
Many requested improvements have not been
\ital{yet} implemented for several reasons:

 \begin{itemize}

   \item The author's natural lazyness (!)

   \item More seriously: many of them would require using some more
registers;  unfortunately, \TeX\ registers are not numerous (256 of each
kind and the limit of \keyindex{dimen} registers is nearly reached) and we
are afraid many requested new features would make \TeX\ stupidly crash even
when typesetting reasonable scores.

  \item We do not think it is wise to introduce in \musictex\ itself a great
number of macros which would be poorly used by most users: the reason is that
\TeX\ memory is hardly limited and that unused macros may occupy some \TeX\
storage which could make things crash because of {\tty TeX capacity
exceeded}...

 \end{itemize}  


\section{Acknowledgements}

The idea of implementing the present package is due to the previous work
(\mutex)\index{mutex=\protect\mutex} of Andrea {\sc Steinbach} and Angelika
{\sc Schofer}\footnote{Steinbach A. \& Schofer A., \ital{Theses} (1987, 1988),
Rheinische Friedrich-Wilhelms Universit\"at, Bonn, Germany.}. This work
provided the basis of the Metafont codes and some line breaking procedures,
which both are still used here... with 99\% corrections and
updates.\index{Schofer, A.}\index{Steinbach, A.}

Besides, the original fonts of \musictex, named {\tt musicn20}, {\tt
musicn16},  {\tt musicn13} and  {\tt musicn11}, have been recently reviewed
by Andreas {\sc Egler} and the new release --- used in the version 5.00 and
above of \musictex\ --- is called  {\tt musikn20}, {\tt musikn16},  {\tt
musikn13}  {\tt musikn11} and {\tt musicbrb}.\index{Egler, A.}


\chapter{Practical use}
\section{Heading statements}

Before any reference to \musictex\ macros:

\medskip{\Bslash input musicnft }

{\Bslash input musictex}

 \medskip\noindent which may be followed by {\Bslash input musicadd}
in the case you have more than six instruments (voice is two instruments: one
for the music, one for the text) or more than 6 simultaneous beams or ties or
slurs.\index{musicnft.tex}\index{musictex.tex}\index{musicadd.tex}
\index{musicnft.tex}

 \medskip After that, you may write a complete book of \TeX\ provided that
you do not use {\tty \&} as a tabulation character (its \keyindex{catcode}
has been changed) inside the music score and that you do not overwrite
\musictex's definitions. This means that no special macros have been designed
to help you write titles, author names, comments, literature excerpts, etc.,
unless you use \LaTeX\ with the {\tt musictex} style.

\section{Before you begin to write notes}

You should first specify whether you want to typeset music in size
20pt\index{sizes} per staff or 16pt. This only optional, the default value
being 20pt. If you want the 16pt size, then you have to say:

\medskip \keyindex{musicsize}{\tty=16}

\medskip Then, the first compulsory declaration is:

\medskip {\Bslash def\keyindex{nbinstruments}{\tty\{$n$\}}}

\medskip \noindent where $n$ is the number of instruments, used by
\musictex\ to performs loops building staffs, setting signatures, meters,
etc. Therefore, it must be defined before any other statements. An instrument
may consist of several staffs, e.g. the piano. The difference between one
instrument of several staffs and several instruments is as follows:

\begin{itemize}
 \item distinct instruments may have distinct \itxem{signatures}, distinct
staffs of a unique instrument share the same signature.
 \item \itxem{stems} may be hung to \itxem{beams} belonging to differents
staffs of the same instrument.
 \item \itxem{chords} may extend across several staffs of the same
instrument.
 \item staffs of a unique \itxem{instrument} are tied together
with a big brace at the beginning of each line.

\end{itemize}

\medskip If the number of staffs (in French ``\itxem{port\'ees}'') is not
equal to one, this number must be specified by:

\medskip
%\check
{\tty\keyindex{nbportees$r$}=$p$\bslash relax}

 \medskip\noindent where $p$ is the number of staffs, and where $r$ is the
roman numeral of the instrument considered (e.g. \keyindex{nbporteesiii}
for the 3rd instrument, starting from the bottom). This value may be zero, in
which case the staff lines are omitted, and this instrument may be used to
code the \itxem{lyrics} of a song, below the actual ``instrument''
representing the notes of the song.

\medskip Unless all your instruments only use the \ital{violin} clef, you
have to specifiy all the clefs used for all the instruments. This is done by
coding:

\medskip{\tty\keyindex{cleftoks$r$}=\{\{$s1$\}\{$s2$\}\{$s3$\}\{$s4$\}\}\%}

\medskip\noindent where $r$ is the roman numeral of the instrument, $s1$
specifies the clef of the lower staff, $s2$ the clef of the second staff,
etc. One must always give four values with the above syntax, otherwise\dots\
$s1=0$ means the \ital{violin} clef (\ixem{clef de sol} in French), $s1=1$
through $s1=4$ mean the \ital{alto} clef (\ixem{clef d'ut} in French) set on
first (lower) through fourth (next to upper line of the staff), $s1=5$ means
the \ital{bass} clef at third (middle) line, and $s1=6$ means the usual
\ital{bass} clef (\ixem{clef de fa} in French) at the usual fourth line.
\index{violin clef}\index{bass clef}\index{alto clef} As an example, a
standard piano score should include:

\medskip{\tty\keyindex{cleftoksi}=\{\{6\}\{0\}\{0\}\{0\}\}\%} or {\Bslash
cleftoksi=\{6000\}\%}

\medskip
If the signature is not void, one should code:

\medskip{\tty\keyindex{generalsignature}{\tty\{$s$\}}\bslash relax}

\medskip\noindent where $s>0$ indicates the number of \itxem{sharps} in the
signature and $s<0$ the number of \itxem{flats}\footnote{We have seen once a
score in G-minor where the signature consisted of two flats (B and E) plus
one sharp (F). This is not supported by \musictex.}. 

\medskip If there a \itxem{meter} indication is to be posted, it should be
specified using the macro

\medskip {\tty\keyindex{generalmeter}\{$m$\}\%}

\medskip \noindent where $m$ is the description of the meter indication which should
appear on each staff. If it is a \ital{fraction} (e.g. 3/4) on should code

\medskip{\Bslash generalmeter\{\bslash meterfrac\{3\}\{4\}\}\%}

\medskip\noindent or, in a simpler way (if the numbers are less than 10):

\medskip{\Bslash generalmeter\{\bslash meterfrac 34\}\%}

\medskip\rm Special denotations can be used, such as \keyindex{allabreve} to
get \raise -6pt\hbox{\allabreve} and \keyindex{meterC} to get \raise
-6pt\hbox{\meterC}.

\medskip However, not all music scores have the same meter in each staff.
Especially, some staffs may have \ital{ternary} meters while others have
\ital{binary}. This can be specified by using the \keyindex{generalmeter} macro
to set the meter for most of the scores and overriding it by means of a more
sophisticated command:


%\check
\medskip{\Bslash metertoksii=\{\{\bslash
meterfrac\{12\}8\}\{\bslash
allabreve\}\{\}\{\}\}\%}
\zkeyindex{metertoks$i$}

\medskip \rm\noindent which sets the meter to 12/8 for the first (lower)
staff, and \ital{alla breve} for the second staff of the instrument number 2
(\ital{ii}). Note that there is room for 4 staffs and that void items must be
specified, otherwise \TeX\ weird errors occur.

\subsection{Instrument names} If you want the \itxem{name of the instrument}s
(or the \itxem{name of the voice}s) to be displayed in front of their
respective staffs at the beginning, you may code:

\medskip\Bslash def\bslash instrument$r$\{\ital{name of the instrument}\}\%
\zkeyindex{instrument$r$}

\medskip \noindent\rm where $r$ is the roman numeral of the instrument
considered. In this case, you should also adjust the \keyindex{parindent}
dimension so that the long name of an instrument does not spill too far into
the left margin.

\subsection{Polyphonic songs}
Except staffs of a unique instrument tied together with a big brace, staffs
normally begin on the left with a thin vertical rule. However, it is usual to
tie all human voices together with a left heavy and right thin vertical rule.
This can be specified (but only once per system) by specifying

\medskip{\tty\bslash def\keyindex{lowersonginstrum}\{$m$\}}

{\tty\bslash def\keyindex{uppersonginstrum}\{$n$\}}

\medskip\noindent where $m$ and $n$ are the intrument numbers of the first
and last choral voices. An example of using this feature is given in
{\tty PRAETORI} and -- more complicated -- also in {\tty ANGESCAM}
and {\tty ANGESCAO}.

%\check
\section{Starting your masterpiece}
\subsection{Typing the first system}

Just code

\medskip\keyindex{debutmorceau}

\medskip\noindent which will initiate (with indentation \keyindex{parindent})
the first set of staffs for all instruments you have previously defined. But
that is not sufficient to begin writing notes and silences. In fact, you also
must choose the spacing of the notes.

\subsection{Easy selecting note spacing}
The easiest way of getting a reasonable note spacing is done by saying

\medskip\keyindex{normal}

\medskip This defines an elementary spacing of {\tty 10pt}. If you say
\keyindex{large}\footnote{To avoid problems with the \LaTeX\
\index{LaTeX=\protect\LaTeX}
macro of the same name, this macro is only activated under \LaTeX\ when
\keyindex{begin\{music\}} is invoked.} the elementary spacing (\keyindex{elemskip}) is set to {\tty
12pt}. Once this is done, you can select $\sqrt 2$ multiples of this value to
select specific note spacing by initiating your note column with {\Bslash
notes} (spacing {\Bslash elemskip}), {\Bslash Notes} (spacing {\tty
1.4\bslash elemskip}), etc.

The is also a macro named \keyindex{etroit} which yields
narrower\footnote{\ital{Etroit} means \ital{narrow} in French.} spacings but
not increasing in the same way.

\medskip In practice, the choice of the macro {\Bslash notes}, {\Bslash
Notes}, {\Bslash NOtes}, etc., to initiate of column of notes sets an
internal dimension register, named \keyindex{noteskip} to the given multiple
of \keyindex{elemskip}. Thus, each \itxem{spacing note} (\keyindex{qu},
\keyindex{qh}, \keyindex{hl}, etc.) will be followed by a spacing of
\keyindex{noteskip}. Then, the advantage of the definition of {\Bslash
elemskip} is that, whenever it is changed, all subsequent {\Bslash noteskip}s
will be updated proportionally so that a simple change of {\Bslash elemskip}
can expand or shrink all consecutive note spacings as a whole.

If these values of {\Bslash elemskip} are not
fit to your needs, you can also say

\medskip{\tty\keyindex{normal}\keyindex{elemskip}=15pt}

\medskip\noindent or revert to the basic note introducing sequence:

\medskip{\tty\keyindex{vnotes} $q$ \bslash elemskip
\ital{note specifications} \bslash enotes}

\medskip\noindent where $q$ is a positive integer or decimal number.

\medskip Of course, if you are sufficiently skilled with \TeX, you can also
compute \keyindex{elemskip} according to the number of bars you want in a
line and the number of notes in each bar.


\medskip However, you may also prefer to ask \musictex\ to compute the size
of \keyindex{elemskip} from the current page width (\keyindex{hsize}) after
assuming a constant number of bars of a constant number of beats within each
line\footnote{This does not meet the requirements of contemporaneous music,
but fits very well to baroque and romantic music.}. In this case, you have
better use the \keyindex{autolines} macro, which is described below (see:
\ref{linepagebreak}, ``Line and page breaking'').

%\check
 \section{Note pitch specification}

 Note pitches are usually specified by letters ranging from {\tty a} to {\tty
z} for those which are usually written under the G-clef ({\tty a} corresponds
to the $A$ of nominal frequency 222.5~Hz; the \ital{G} of the G-clef is denoted
{\tty g}). Lower pitch notes are specified using upper case letters ranging
from {\tty A} to {\tty N} (the \ital{F} of the F-clef is denoted {\tty M}, and
{\tty F} is one octave below).
%\check

\medskip If necessary, a numeric symbol can be used to place a symbol
independently of the active clef.

\medskip Besides, notes below {\tty A} (i.e. the $A$ of nominal frequency
55.625~Hz), namely the lowest octave of the modern pianos, can only be coded
using the transposition features (see below: \ital{transposition} and
\ital{octaviation}) or in absolute vertical position using numbers.

\section{Writing notes}

There are two major kinds of note macros: \begin{enumerate} \item those
which terminate a note/chord stem and are followed by a horizontal spacing of
value \keyindex{noteskip},
 \item those which initiate or extend a note/chord stem and do not cause
horizontal spacing.
\end{enumerate}

\medskip The first kind is used to type a melody, the second kind is used to
type chords.
%\check

\subsection{Single (spacing) notes}

\begin{quote}\begin{description}
\item[\keyindex{wh}~$p$ : ]whole note at pitch $p$.
\item[\keyindex{hu}~$p$ : ]half note at pitch $p$ with stem up.
\item[\keyindex{hl}~$p$ : ]half note at pitch $p$ with stem down.
\item[\keyindex{qu}~$p$ : ]quarter note at pitch $p$ with stem up.
\item[\keyindex{ql}~$p$ : ]quarter note at pitch $p$ with stem down.
\item[\keyindex{cu}~$p$ : ]eighth note\footnote{The {\Bslash c} of this macro
 name is taken from the French word {``croche''} which is by the way one
 half of the english {``crotchet''}; {\Bslash cc...}, {\Bslash ccc...}
 are standing for {``double croche''}, {``triple croche''}, etc.}
 at pitch $p$ with stem up.
\item[\keyindex{cl}~$p$ : ]eighth note at pitch $p$ with stem down.
\item[\keyindex{ccu}~$p$ : ]sixteenth note at pitch $p$ with stem up.
\item[\keyindex{ccl}~$p$ : ]sixteenth note at pitch $p$ with stem down.
\item[\keyindex{cccu}~$p$ : ]32-th note at pitch $p$ with stem up.
\item[\keyindex{cccl}~$p$ : ]32-th note at pitch $p$ with stem down.
\item[\keyindex{ccccu}~$p$ : ]64-th note at pitch $p$ with stem up.
\item[\keyindex{ccccl}~$p$ : ]64-th note at pitch $p$ with stem down.
%\check
\end{description}\end{quote}

\medskip As an example, the sequence:

\begin{music}\parindent=0pt\def\nbinstruments{1}\def\instrumenti{}
 \generalsignature{0}\savesignature\cleftoksi={{0}{0}{0}{0}}
 \debutextrait\normal
\notes\cu c\temps\cl j\enotes\barre
\notes\ccu c\temps\ccl j\enotes\barre
\notes\cccu c\temps\cccl j\enotes\barre
\notes\ccccu c\temps\ccccl j\enotes\finextrait
\end{music}

\medskip \noindent was coded as:

%\check
\begin{quote}\begin{verbatim}
\notes\cu c\temps\cl j\enotes\barre
\notes\ccu c\temps\ccl j\enotes\barre
\notes\cccu c\temps\cccl j\enotes\barre
\notes\ccccu c\temps\ccccl j\enotes
\end{verbatim}\end{quote}

\medskip If these notes are preceded by \ital{non-spacing} notes (i.e.
macros \keyindex{zq} or \keyindex{zh}) their stem is extended up or down so
as to join all notes into a single chord.
 \subsection{Non-spacing (chord) notes}

\begin{quote}\begin{description}
\item[\keyindex{zq}~$p$ : ]quarter (or shorter) note head at pitch $p$ with no
spacing after.
\item[\keyindex{zh}~$p$ : ]half note head at pitch $p$ with no spacing after.
\end{description}\end{quote}

\medskip It must be pointed out that the pitch $p$ of these notes is
memorized so that the stem of the further spacing note will join them into
a chord. This stem top and bottom pitch is \ital{reset} at each spacing
note.

\medskip \noindent{\sl REMARK : Notes of duration longer than whole notes are
always non-spacing. This saves one useless definition, since these notes are
always longer than other simultaneous ones. If needed they can be followed by
\keyindex{sk} to force spacing.}
 \subsection{Shifted non-spacing (chord)
heads} These symbols are used mainly in chords where \ital{second} intervals
are present. It is the responsibility of the typist to choose which heads
should be shifted left or right.

\rm\medskip \par\keyindex{rw}~$p$ : whole note head shifted right by one note
width ($\approx$ 6pt), no spacing.
 \par\keyindex{lw}~$p$ : whole note head shifted left by one note
width ($\approx$ 6pt), no spacing.
 \par\keyindex{rh}~$p$ : half note head shifted right by one note
width ($\approx$ 6pt), no spacing.
 \par\keyindex{lh}~$p$ : half note head shifted left by one note
width ($\approx$ 6pt), no spacing.
 \par\keyindex{rq}~$p$ : quarter note head shifted right by one note
width ($\approx$ 6pt), no spacing.
 \par\keyindex{lq}~$p$ : quarter note head shifted left by
one note width ($\approx$ 6pt), no spacing.

\medskip Except that they are shifted left of right, these macros act like
{\Bslash z...} macros for stem building.

\subsection{Shifted notes}

\begin{quote}\begin{description}
 \item[\keyindex{rqu} : ]acts like {\Bslash qu} but the note head is
shifted one note width. This is used for \itxem{chords} with upper note 
on the right side of the stem.
 \item[\keyindex{rql} : ]same with stem down.
 \item[\keyindex{rhu} : ]same as above for a half note, stem up.
 \item[\keyindex{rhl} : ]same with stem down.
\end{description}\end{quote}

In addition on can use the command \keyindex{roff} to offset any note
character by one quarter note head, e.g.:

 \verb|\roff{\zqup h}|

\noindent which will post a pointed quarter note with stem up, offset by one
quarter note head with respect to its normal abscissa. \verb|\roff{...}| is
heavily recommended instead of \verb|\qsk| if the score is intended to be run
also with \musixtex.

\subsection{Non-spacing single notes}

\begin{quote}\begin{description}
 \item[\keyindex{zhu} : ]half note with stem up but no spacing. It acts like
{\Bslash hu} for stem building.
 \item[\keyindex{zhl} : ]half note with stem down but no spacing. It acts like
\keyindex{hl} for stem building.
 \item[\keyindex{zqu} : ]quarter note with stem up but no spacing. It acts like
{\Bslash qu} for stem building.
 \item[\keyindex{zql} : ]quarter note with stem down but no spacing. It acts
like {\Bslash ql} for stem building.
 \item[\keyindex{zcu} : ]eighth note with stem up but no spacing. It acts like
{\Bslash cu} for stem building.
 \item[\keyindex{zcl} : ]eighth note with stem down but no spacing. It acts
like {\Bslash cl} for stem building.


\item[\keyindex{lhu}, \keyindex{lhl}, \keyindex{lqu}, \keyindex{lql} : ]same
as above, but the whole of the note is shifted one note width on the left.
 \item[\keyindex{zw}~$p$ : ]whole note at pitch $p$ with no spacing after.
\item[\keyindex{zwq}~$p$ : ]arbitrary duration note (\raise
2.5pt\hbox{\musicxx\char 125}) at pitch $p$ with no spacing after.
\item[\keyindex{zbv}~$p$ : ]breve note (\raise
2.5pt\hbox{\musicxx\char 36}) at pitch $p$ with no spacing after.
\item[\keyindex{zsb}~$p$ : ]semi-breve note (\raise
2.5pt\hbox{\musicxx\char 32}) at pitch $p$ with no spacing after.
\end{description}\end{quote}

\subsection{Single (spacing) stemless notes}

 Although not standard in real music scores, one may need to have stemless
quarter and half noteheads posted in the same way as whole notes. This can be
done with the following commands: 

\begin{quote}\begin{description}
\item[\keyindex{nh}~$p$ : ]half notehead at pitch $p$.
\item[\keyindex{nq}~$p$ : ]quarter notehead at pitch $p$.
\end{description}\end{quote}

\medskip As an example, the sequence:

\begin{music}\parindent=0pt\def\nbinstruments{1}\def\instrumenti{}
 \generalsignature{0}\savesignature\cleftoksi={{0}{0}{0}{0}}
 \debutextrait\normal
\notes\nq c\temps\nq j\enotes\barre
\Notes\nh c\temps\nh j\enotes\barre
\notes\nq {cdef}\enotes\finextrait
\end{music}

\medskip \noindent was coded as:

%\check
\begin{quote}\begin{verbatim}
\notes\nq c\temps\nq j\enotes\barre
\Notes\nh c\temps\nh j\enotes\barre
\notes\nq {cdef}\enotes\finextrait
\end{verbatim}\end{quote}

 In case of special need, non spacing variants have been provided, namely
\keyindex{znh} and \keyindex{znq}.

\subsection{Pointed notes}
 One simple way of doing consists in putting \keyindex{pt
$p$} to get a \ital{dot} after the normal note head at pitch $p$. Thus a
quarter note with a point can be coded {\Bslash pt h\bslash qu h}.


\medskip A simpler way of doing consists in using compact macros, namely:
\keyindex{qup}, \keyindex{qupp}, \keyindex{quppp}, \keyindex{zqp},
\keyindex{zhp}, \keyindex{zwp} (these three {\Bslash z...p} are useful in
chords), \keyindex{hup}, \keyindex{whp}, \keyindex{qhp}, \keyindex{qhpp},
\keyindex{qlp}, \keyindex{qlpp}, etc.


\medskip You may also introduce pointed notes, especially in groups by coding
a \ital{period} before (not after) the letter representing the pitch:
{\Bslash qu\{.a.\^{}b.c\}} which is equivalent to:

\medskip
\begin{quote}\begin{verbatim}
\pt{a}\qu{a}\pt{b}\sh{b}\qu{b}\pt{c}\qu{c}
\end{verbatim}\end{quote}

Finally, pointed nots can also be produced without spacing aftef, using
\keyindex{zhup}, \keyindex{zhlp}, \keyindex{zqup}, \keyindex{zqlp},
\keyindex{zcup}, \keyindex{zclp}, and the same with two {\tty p}'s for
double-pointed notes.

\section{Beams}

 Beams are not automatically handled, but they must be declared explicitely,
\ital{before} the first spacing note involving them is coded. Two kinds of
macros are provided:

\begin{enumerate}

\item fixed slope beams have an arbitrary slope chosen by the user in
the range -45\% to +45\% (by multiples of 5\%);

\item semi-automatic beams have their slope computed knowing the number
of \keyindex{noteskip} over which they are supposed to extend, and knowing the
initial and final pitch of the notes they are supposed to link.

\end{enumerate}

\subsection{Fixed slope beams}

\begin{quote}\begin{description}
\item[\keyindex{ibu}~$nps$ : ]initiates an \ital{upper beam} 3 horizontal
line spacings above the pitch $p$~; $m$ is its reference number, which must be
in the range [0-5] ([0-9] if {\tty musicadd} file has been {\Bslash
input});
$s$ is the slope of the beam.

\end{description}\end{quote}

 $s$ is an integer in the range [-9,9].
$s=1$ means a slope of 5\%, $s=9$ means a slope of 45\% (the maximum with the
{\tty beamn20} or {\tty beamn16} fonts), $s=-3$ means a slope of -15\%, etc.
With usual spacings a slope of 2 or 3 is fit for ascending scales. A slope of
6 to 9 is fit for ascending arpeggios.

\begin{quote}\begin{description}

\item[\keyindex{ibl}~$nps$ : ]initiates a \ital{lower beam} 3 horizontal line
spacings below the pitch $p$. Other parameters as above.


\item[\keyindex{ibbu}~$nps$ : ]initiates a \ital{double upper beam} (same
parameter meaning).
 \item[\keyindex{ibbl}~$nps$ : ]initiates a \ital{double lower beam} (same
parameter meaning).
 \item[\keyindex{ibbbu}~$nps$ : ]initiates a \ital{triple upper beam} (same
parameter meaning).
 \item[\keyindex{ibbbl}~$nps$ : ]initiates a \ital{triple lower beam} (same
parameter meaning).
 \item[\keyindex{ibbbbu}~$nps$ : ]initiates a \ital{quadruple upper beam} (same
parameter meaning).
 \item[\keyindex{ibbbbl}~$nps$ : ]initiates a \ital{quadruple lower beam} (same
parameter meaning).
 \item[\keyindex{ibbbbbu}~$nps$ : ]initiates a \ital{quintuple upper beam} (same
parameter meaning).
 \item[\keyindex{ibbbbbl}~$nps$ : ]initiates a \ital{quintuple lower beam} (same
parameter meaning).
\end{description}\end{quote}

\medskip Beam termination is also not automatic. The termination of a given
beam must be explicitely declared \ital{before} coding the last spacing note
connected to that beam.

\begin{quote}\begin{description}
\item[\keyindex{tbu}~$n$ : ]terminates upper beam number $n$ at current
position. \item[\keyindex{tbl}~$n$ : ]terminates lower beam number $n$ at
current position.
\end{description}\end{quote}

\medskip\keyindex{tbu} and \keyindex{tbl} terminate beams of any
multiplicity. Therefore 32-th notes hanging on a triple beam are initiated by
\keyindex{ibbbu}~$nps$ and terminated by \keyindex{tbu}~$n$.

\medskip It is also possible to code beams whose multiplicity is not the same
at the beginning. The multiplicity can be increased at any position. For
instance, \keyindex{nbbu}~$n$ which sets the multiplicity of upper beam
number $n$ to 2 starting at the current position, \keyindex{nbbbu}~$n$ sets
its multiplicity to 3, \keyindex{nbbbbu}~$n$ sets its multiplicity to 4, and
\keyindex{nbbbbbu}~$n$ sets it to 5. \keyindex{nbbl}~$n$ \dots
\keyindex{nbbbbbl}~$n$ perform the same functions for lower beams.

\medskip Notes hanging or standing on beams are coded in the form
{\Bslash qh$n\,\,p$} and {\Bslash qb$n\,\,p$} where $n$ is the beam number and $p$
the pitch of the note head. \musictex\ adjusts the length of the note stem to
link the bottom of the chord to an upper beam (normally with \keyindex{qh}) and
the top of the chord to a lower beam (normally with \keyindex{qb}).

\medskip Note that the difference between upper and lower beams does not
mainly consist in the beam being above or below the note heads; rather, it
specifies whether the abscissa of the beginning and the end of this beam is
aligned on the right (upper beam) or on the left (lower) beam. Thus, the
sequence:

\begin{music}
\def\nbinstruments{1}\nbporteesi=1\relax
\cleftoksi={{0}{0}{0}{0}}\def\instrumenti{}\resetsignatures\debutextrait
\large\temps
\notes\ibu0h0\qh0e\nbbu0\qh0e\nbbbu0\qh0e\nbbbu0\relax
\qh0e\nbbbbu0\qh0e\nbbbbbu0\qh0e\tbu0\qh0e\enotes
\finextrait
\end{music}

%\check
\noindent has been coded as
\begin{quote}\begin{verbatim}
\notes\ibu0h0\qh0e\nbbu0\qh0e\nbbbu0\qh0e\nbbbu0\relax
\qh0e\nbbbbu0\qh0e\nbbbbbu0\qh0e\tbu0\qh0e\enotes
\end{verbatim}\end{quote}
%\check

 \rm It is quite possible to terminate with \keyindex{tbu} a beam initiated
with \keyindex{ibl}. This may give:


\begin{music}
\def\nbinstruments{1}\nbporteesi=1\relax
\cleftoksi={{0}{0}{0}{0}}\def\instrumenti{}\resetsignatures\debutextrait
\large\temps
\notes\ibl0p0\qb0p\nbbl0\qb0p\nbbbl0\qb0p\tbu0\qh0e\enotes
\finextrait
\end{music}

%\check
\noindent which has been coded as

\begin{quote}\begin{verbatim}
\notes\ibl0p0\qb0p\nbbl0\qb0p\nbbbl0\qb0p\tbu0\qh0e\enotes
\end{verbatim}\end{quote}
%\check
 \rm Partial termination of beams is also possible, by using \keyindex{tbbu}
or \keyindex{tbbl}~: these macros terminate the current beam except that of
order 1 (eighths). \keyindex{tbbbu} or \keyindex{tbbbl} terminate the current
beam except those of order 1 and 2, etc.


\medskip
The macros \keyindex{tbbu} and \keyindex{tbbl} may also be invoked when only
a single beam is active. Then, a second beam (upper or lower according
the initiating procedure) is opened \ital{one note width before
the current position, and closed immediately}. Thus the following sequence

\begin{music}
\def\nbinstruments{1}\nbporteesi=1\relax
\cleftoksi={{0}{0}{0}{0}}\def\instrumenti{}\resetsignatures\debutextrait
\large     \temps
\notes\ibu0e0\qh0e\tbbu0\tbu0\qh0e\enotes
\finextrait
\end{music}

\medskip\noindent is coded:

\medskip
\begin{quote}\begin{verbatim}
\notes\ibu0e0\qh0e\tbbu0\tbu0\qh0e\enotes
\end{verbatim}\end{quote}

%\check

\rm \medskip The same behaviour occurs in the case of \keyindex{tbbbu},
\keyindex{tbbbl}, \keyindex{tbbbbu}, \keyindex{tbbbbl}, \keyindex{tbbbbbu}
and \keyindex{tbbbbbl}.

\medskip The symmetrical pattern is also possible. For example:

\begin{music}\def\nbinstruments{1}\nbporteesi=1\relax
\cleftoksi={{0}{0}{0}{0}}\def\instrumenti{}\resetsignatures
\debutextrait\normal
\Notes\ibbl0j0\rlap{\qsk\tbbl0}\qb0j\tbl0\qb0j\enotes
\finextrait
\end{music}

\noindent has been coded as:

\medskip
\begin{quote}\begin{verbatim}
\Notes\ibbl0j0\rlap{\qsk\tbbl0}\qb0j\tbl0\qb0j\enotes
\end{verbatim}\end{quote}


\medskip \noindent{\sl REMARK: these codings may seem complicated. In fact,
it is the responsibility of the user to define macros which perform the most
common sequences in his masterpiece. For example, one could define sets of
four sixteenths by the macro:}

\medskip{\Bslash def\bslash qqh\#1\#2\#3\#4\#5\{\%

\bslash ibbl0\#2\#1\bslash qh \#2\bslash qh \#3\bslash tbl0\bslash qh \#4\}}

\medskip\noindent{\sl
where the first argument is the slope and the other four arguments are the
pitches of the four consecutive sixteenths wanted.}
 %\check
\subsection{Repeated pattern beams}\index{repeated patterns}
 Note heads hanging on beams are not necessarily quarter (or higher order)
note heads. It is possible to hang half note heads on beams using
\keyindex{hh} and \keyindex{hb} macros, e.g.:

\begin{music}\def\nbinstruments{1}\nbporteesi=1\relax
\cleftoksi={{0}{0}{0}{0}}\def\instrumenti{}\resetsignatures
\debutextrait\normal
\Notes\ibbl0j0\hb0j\tbl0\hb0j\enotes
\Notes\ibbu0g0\hh0g\tbu0\hh0g\enotes
\finextrait
\end{music}

\noindent has been coded as:

\medskip
\begin{quote}\begin{verbatim}
\Notes\ibbl0j0\hb0j\tbl0\hb0j\enotes
\Notes\ibbu0g0\hh0g\tbu0\hh0g\enotes
\end{verbatim}\end{quote}

It is also possible to write

\begin{music}\def\nbinstruments{1}\nbporteesi=1\relax
\cleftoksi={{0}{0}{0}{0}}\def\instrumenti{}\resetsignatures
\debutextrait\normal
\Notes\ibbl0j3\wh j\tbl0\wh l\enotes
\Notes\ibbu0g3\wh g\tbu0\wh i\enotes
\finextrait
\end{music}

\noindent which was coded as:

\medskip
\begin{quote}\begin{verbatim}
\Notes\ibbl0j3\wh j\tbl0\wh l\enotes
\Notes\ibbu0g3\wh g\tbu0\wh i\enotes
\end{verbatim}\end{quote}

\noindent However, a better look could be obtained in a more sophisticated
way\footnote{You are suggested to make your own macro if you have to type many
of these.}:

\begin{music}\def\nbinstruments{1}\nbporteesi=1\relax
\cleftoksi={{0}{0}{0}{0}}\def\instrumenti{}\resetsignatures
\debutextrait\normal
\Notes\zw j\qsk\ibbl0j3\sk\tbl0\wh l\enotes
\Notes\ibbu0g3\wh g\tbu0\qsk\wh i\enotes
\finextrait
\end{music}

\noindent which was simply coded as:

\medskip
\begin{quote}\begin{verbatim}
\Notes\zw j\qsk\ibbl0j3\sk\tbl0\wh l\enotes
\Notes\ibbu0g3\wh g\tbu0\qsk\wh i\enotes
\end{verbatim}\end{quote}


\subsection{Semi-automatic beams}
 In order to avoid tedious checks to adjust the slope (and even the starting
pitch) of beams in music with a lot of steep beams, a set of automatically
slope computing has recently been implemented (file {\tty  musicvbm.tex}). If
you say {\Bslash Ibu2gj3} \musictex\ will understand that you want to
build an upper beam (beam number 2) horizontally extending \hbox{3\Bslash
noteskip}, the first note of which is a {\tty g} and the last note is a{\tty j}.
Knowing these parameters it will choose the highest slope number which
corresponds to a slope not more than $(\hbox{\tty j}-\hbox{\tty
g})/(3\keyindex{noteskip})$. Moreover, if there is no sufficiently steep beam
slope available, then it will raise the starting point.


\medskip Ten such macros are available: \keyindex{Ibu}, \keyindex{Ibbu},
\keyindex{Ibbbu}, \keyindex{Ibbbbu}, \keyindex{Ibbbbbu}, \keyindex{Ibl},
\keyindex{Ibbl}, \keyindex{Ibbbl}, \keyindex{Ibbbbl} and \keyindex{Ibbbbbl}.
Examples of their use is given in {\tty marcello.tex}.

\par\penalty -8000
\rm
 \section{Rests}

 Except that difference that they have no specific pitch, rests are coded in a
very simple way.

\subsection{ordinary rests}

 Full bar rests (also called ``pauses'') are coded as \keyindex{pause};
smaller rests are \keyindex{hpause} (of duration equal to a \ital{half
note}), \keyindex{soupir} (duration equal to a \ital{quarter note}),  
\keyindex{dsoupir} (duration equal to an \ital{eighth note}, also
\keyindex{ds}),   \keyindex{quartsoupir} (duration equal to an
\ital{sixteenth note}, also \keyindex{qs}), \keyindex{huitsoupir} (duration
equal to an \ital{thirtysecond note}, also \keyindex{hs}),
\keyindex{seizsoupir} (duration equal to an \ital{sixty fourth note}, also
\keyindex{qqs}).

  Long rests (lasting several bars) can be coded as \keyindex{PAuse}
and \keyindex{PAUSe}, which respectively yield:

 \begin{music}
\def\nbinstruments{1}\relax
\computewidths
\generalmeter{\meterfrac{4}{4}}\relax
\debutextrait
\normal\elemskip=0.115\hsize
\temps\Notes\sk\PAuse\enotes
\barre\Notes\sk\PAUSe\enotes
\finextrait
 \end{music}

\subsection{Lifted rests}
  All the previous rests are \ital{hboxes}; this means that they
can be vertically offset to meet polyphonic music requirements using the
standard \TeX\ command \keyindex{raise}.  

 In addition, two symbols have been provided to put a \itxem{full rest} or a
\itxem{half rest} above or below the staff. Then the ordinary \verb|\pause| or
\verb|\hpause| cannot be used since there is a need for small horizontal line
to distinguish between the full and the half rest. They are~:
 \begin{itemize}
 \item \keyindex{expause}~$p$ (spacing) for \hbox to 1.5em{\hss\expause 1\hss}
at pitch position $p$,
 \item \keyindex{exhpause}~$p$ (spacing) for \hbox to 1.5em{\hss\exhpause 1\hss},
 \item \keyindex{zexpause}~$p$ (non spacing) for \hbox to 1.5em{\hss\zexpause 1\hss},
 \item \keyindex{zexhpause}~$p$ (non spacing) for \hbox to
1.5em{\hss\zexhpause 1\hss}.
 \end{itemize}

 \section{Phantom notes}

It may be interesting, when coding a sequence of notes within a unique pair
{\Bslash notes...\bslash enotes}, to skip one note place in order -- for
example -- to set the third note of one staff at the same abscissa as that of
the second note of another staff. This can be done by inserting \keyindex{sk}
which causes a spacing of one \keyindex{noteskip}\footnote{Do not use
\keyindex{kern} nor \keyindex{hskip}~: in fact \keyindex{sk} not only causes
a space but also records that space for correct handling of beams.}.

\medskip If you just want to shift a note or a symbol by one note head width,
you may write \keyindex{qsk}. However, if you want to offset a note by one
note head width in a chord, you have better use the \keyindex{roff}
macro\footnote{The reason for that pertains to \ixem{\musixtex} which might
dilate or shrink the {\bslash \qsk} spacing while the {\bslash roff} offset is
absolute.}.

\medskip Note that these two latter macros must be used inside a pair {\tty
\bslash notes...\bslash enotes}. If you want to make a spacing of one note
head width outside, write \keyindex{nspace}. Conversely, a more general
spacing is allowed within the {\Bslash notes...\bslash enotes} pairs, namely:

\medskip{\tty\keyindex{off} $d$}

\medskip\noindent where $d$ is any \TeX\ valid dimension, for example
{\Bslash noteskip} or {\tty 5\bslash Interligne}. In fact, if you look to the
\musictex\ source, you will find that {\Bslash off} is the basic control
sequence used to define \keyindex{sk}, \keyindex{qsk}, etc.


\section{Collective coding: sequences of notes}


As seen in the {\sc Mozart} example, it is not necessary to write a macro
sequence {\Bslash notes...\bslash enotes} for each column. If, on all
staffs of all instruments, spacings are equal or multiple of a unique value,
the notes may be concatenated in each staff: each note in each staff makes
the current position horizontally advance by the elementary spacing specified
by the choice of {\Bslash notes}, {\Bslash Notes}, {\Bslash
NOtes}, etc.

%\check



\medskip The major interest of this feature resides in that fact that
the note macros are able to write several items; for instance
{\Bslash qu\{cdefghij\}} writes the \ital{C-major} scale in quarters with
stem up. In the same way {\Bslash cl\{abcdef\^{ }gh\}} writes the
\ital{A-minor} scale in eighths. Not all note generating macros can be used
to perform collective coding, but most of them can. They are:

\begin{itemize} \item all the spacing notes: \keyindex{wh}, \keyindex{hu},
\keyindex{hl}... \keyindex{ccccl} and the beam hooked notes, i.e.
\keyindex{qh}$n$ and \keyindex{qb}$n$. \item all the chord notes with names
of the form {\Bslash z...}. \end{itemize}

 Conversely, shifted notes are not supposed to be used in collective coding,
mainly because they are used in very special cases which do not deserve
wasting memory space to make them collective.

\medskip If necessary a void space can be inserted in a collective coding by
using {\tty*}\index{*}.


%\check
\section{Accidentals}


Accidentals can be introduced in two ways.

\medskip The first way, the \ital{manual} way of coding them, consists for
example in coding {\tty\keyindex{fl} a} to put a \ital{flat} at the pitch $a$,
supposedly before the further note of that pitch. There is no control upon
the fact that a note will be put at this position and at this pitch.
Naturals, sharps, double flats and double sharps are coded \keyindex{na}~$p$,
\keyindex{sh}~$p$, \keyindex{dfl}~$p$ and \keyindex{dsh}~$p$ respectively.

\medskip
Alternate procedures \keyindex{lfl}, \keyindex{lna}, \keyindex{lsh},
\keyindex{ldfl} and \keyindex{ldsh}
place the same accidentals, but their abscissa is shifted one note head width
on the left. The purpose of this is to avoid collision of accidentals
in a chord with narrow intervals.

\medskip
%\check
 The second way of coding accidentals consists in putting the symbol {\tty
\^{ }} (sharp), the symbol {\tty\_} (flat), the symbol {\tty=} (natural), the
symbol {\tty>\relax} (double sharp), or the symbol {\tty<\relax} (double
flat) within the coding of the note, e.g.: {\Bslash qh\{\^{ }g\}} yields a
$G\sharp$. This may very well be combined with collective coding, e.g.:
{\Bslash qu\{ac\^{ }d\}}.


\medskip Two sizes are available for accidentals. They revert to the small
version when notes are supposed to be too close to each other. These two
sizes can be forces by coding \keyindex{bigfl}, \keyindex{bigsh}, etc., or
\keyindex{smallfl}, \keyindex{smallsh}, etc. If one does not want to have any
small accidentals, then one can declare \keyindex{bigaccid} (conversely
\keyindex{smallaccid} or \keyindex{varaccid} -- the latter restoring variable
sizes).


\medskip Small accidentals can also be put \ital{above} the note
heads. This is done using \keyindex{uppersh}~$p$, \keyindex{upperna}~$p$ or
\keyindex{upperfl}~$p$:

 \begin{music}
\def\nbinstruments{1}\relax
\computewidths
\debutextrait
\normal
\temps\NOtes\sk\uppersh l\ql l\enotes
\temps\NOtes\upperna m\ql m\enotes
\temps\NOtes\upperfl l\ql l\enotes
\finextrait
 \end{music}

 It also possible to introduce \ital{\ixem{cautionary accidental}s} on a
score, i.e.\ small size accidentals between parentheses. This done by preceding
the name of the accidental keyword by a {\tty c}, e.g.
by coding {\tty\keyindex{cfl}}$p$ to get a cautionary flat.
 Available cautionary accidentals are \keyindex{csh}, \keyindex{cfl}, \keyindex{cna},
\keyindex{cdfl} and \keyindex{cdsh}, which give:

 \begin{music}
\def\nbinstruments{1}\relax
\computewidths
\debutextrait
\normal
\temps\NOtes\sk\csh g\qu g\enotes
\temps\NOtes\cfl h\qu h\enotes
\temps\NOtes\cna i\qu i\enotes
\temps\NOtes\cdfl j\qu j\enotes
\temps\NOtes\cdsh k\qu k\enotes
\finextrait
 \end{music}

 Since cautionary accidentals are wider
that normal ones, it might be useful to insert some spacing before them, either
using \keyindex{temps} or a \keyindex{sk} within a {\Bslash notes...\bslash
enotes} group.

 
 \section{Transposition and octaviation}


An important feature is the existence of a special register
\keyindex{transpose} the normal value of which is 0. If you say

\medskip
{\Bslash transpose=3}

\medskip\rm\noindent all subsequent pitches specified by
upper or lower case letters will be transposed 3 positions. If you set
\keyindex{transpose} to 7 you may write your music one octave below its final
pitch. Thus, you can define \ital{octaviation} macros like
%\check

\medskip
\Bslash def\bslash soqu\#1\{\bslash zq\{\#1\}\{\bslash transpose=7\bslash relax
\bslash qu\{\#1\}\}\}

\medskip\rm\noindent to build quarter note octaves in a single call. Note
that the octaviated note is coded within braces so that the transposition
is only local.
%\check


\medskip
 \ital{Octaviation}\index{octaviation} can also be performed in is another
way, namely unsing special codes to transpose by multiples of 7 intervals.
For example {\Bslash qu\{'ab\}} is equivalent to {\Bslash qu\{hi\}} and
{\Bslash qu\{\`{}kl\}} is equivalent to {\Bslash qu\{de\}}. It should be
emphasized here that the {\tty'} (\itxem{acute accent}) and the {\tty\`{}}
(\itxem{grave accent}) have cumulative effects, so that {\Bslash qu\{''A'A\}}
is equivalent to {\Bslash qu\{ah\}} and that the {\Bslash transpose}
parameter is only reset to its initial value (not necessarily zero) when
changing staff or instrument (i.e. {\tty \|} or {\tty \&}) or at {\Bslash
enotes}. Since this may be confusing, it is useful to use the {\tty !} prefix
to reset the {\Bslash transpose} register explicitely to the value it had
when entering {\Bslash notes}\footnote{This value is saved in another
register named \keyindex{normaltranspose}.}. Thus {\Bslash qu\{!a'a\}} always
gives the note {\tty a} and its upper octave {\tty h} \ital{shifted by the
value of {\Bslash transpose} at the beginning of the current} \noindent
{\Bslash notes...\bslash enotes} group (or {\Bslash Notes...\bslash enotes},
etc.) whatever the number of previous grave and acute accents occurring
inbetween.

\medskip
 The above processes indeed change the vertical position of the note heads
and associated symbols (note stems, accents and beams) but they do not
take care of the necessary changes of accidentals when transposing, i.e. the
fact that an $F\sharp$ occurring with a zero signature should become a
$B\natural$ when transposing from the tonality of $C$ major to $F$ major where
the normal $B$ is the $B\flat$. Since the intent of the composer is not
obvious -- he may want to shift a group of notes within the same tonality
or conversely to transpose it in another tonality -- this is not done
automatically. Thus the \keyindex{sh}, \keyindex{fl}, \keyindex{na}, \keyindex{dsh} and
\keyindex{dfl} symbols \ital{are not affected} by a change of the \keyindex{transpose}
register.


\medskip
 But the composer/typesetter may ask \musictex\ to do that work. In this
case, he should code \keyindex{Sh}, \keyindex{Fl}, \keyindex{Na},
\keyindex{dSh} and \keyindex{dFl} (or \keyindex{bigSh}, \keyindex{bigFl},
\keyindex{bigNa}, \keyindex{bigdSh} and \keyindex{bigdFl} or
\keyindex{smallSh}, \keyindex{smallFl}, \keyindex{smallNa},
\keyindex{smalldSh}, \keyindex{smalldFl}, \keyindex{cSh}, \keyindex{cFl},
\keyindex{cNa},  \keyindex{cdSh} and \keyindex{cdFl}) instead of the usual
lower case accidental symbols. The symbol \keyindex{Sh} (resp.
\keyindex{bigSh} and \keyindex{smallSh}) means that the corresponding pitch
has to be raised by \ital{one half pitch} with respect to its normal value
\ital{according to the current signature}. Thus {\Bslash Sh b} means a
$B\sharp$ if the signature is zero or positive, and a $B\natural$ if it is
negative. The same logic applies for all accidentals having an upper case
forelast letter.


\medskip
 Obviously, the computation is done after taking account of the value of
the \keyindex{transpose} register.


\medskip The compact codes {\tty \^{}}, {\tty \_}, {\tty =} are normally not
affected by transposition and signatures, but their behaviour can be changed
by saying \keyindex{relativeaccidentals} and reset by
\keyindex{absoluteaccidentals} (the default situation).

%\check


\medskip Although \ital{relative accidental coding} is an easy and safe way
of coding \ital{transposable} scores, care should be exercised in getting rid
of the habit of saying {\Bslash na b} to rise the pitch of a $B$ when the
tonality is $F$ major (i.e. with {\tty\keyindex{sign$n$}=-1} or
{\tty\keyindex{generalsignature}\{-1\}}). An example of sophisticated
transposition is given in the score {\tty souvenir.tex} (which {\Bslash
input}s {\tty souvenix.tex}). It should be noted that \ital{relative accidental
coding} is compatible with \itxem{cautionary accidental} coding: in this
case, one should code \keyindex{cNa}, \keyindex{cSh},  \keyindex{cFl}, 
\keyindex{cdSh},  \keyindex{cdFl} respectively.

\medskip
 Besides, the typical piano octave transposition \hbox{\sl 8 - - - - - } can
be obtained by coding:

%\check
\medskip  \keyindex{octfin} $p$ $n$

\medskip\noindent which puts the 8 and dotted line symbols at the pitch $p$
(usually {\tty p} to {\tty r}). The length is $n$\keyindex{noteskip}. This
obviously fit for short octaviation denotations. To transpose a whole line,
use \keyindex{octline}~$p$. Since \keyindex{octfin} terminates with a small
hook down, to indicate clearly where octaviation stops, you may also like to
use \keyindex{octsup} which behaves like \keyindex{octfin} without the final
hook. All this supposes you have an idea of the actual line breaking of your
score in that section, and this is admittedly difficult to handle when octave
transposition is supposed to long a large number of lines. In that latter
case, you can use the {\Bslash def\keyindex{everystaff}\{...\}} to insert
whatever code you like at each new line of score. This has been used to set
octaviation in the score of the \ital{Toccata in F} by Charles-Marie
\index{Widor, C.-M.}{\sc Widor}\footnote{A French organist (1844-1937) and
composer who was in charge of the organ of  St-Sulpice in Paris, from 1864 to
1934.}: at each bar (or virtual bar, namely \keyindex{zbarre}), the {\Bslash
everystaff} procedure is updated to produce the convenient code, in case of
the line breaking happening before the next definition.
%\check

 \section{Ties and slurs} 

 They have been implemented in a way which may look
rather ugly, but we think it is the only way of implementing \ital{in one
pass} ties and slurs which run \ital{across glue}.


\medskip Slurs and ties must be initiated within the pair {\Bslash
notes...\bslash notes} before the spacing note is coded. They must be
terminated also before the last note is coded.


\medskip\keyindex{itenu}~$np$


\medskip\noindent\rm (\ital{ten} stands for the Italian word \ital{tenuto})
initiates an upper tie (convex) at pitch $p$. Just like beams, ties have a
reference number $n$, from 0 to 6 (or 9 if {\tty musicadd} is included).
\keyindex{itenl}~$np$ initiates a lower tie (concave).


\medskip The tie of reference number $n$ is terminated by
\keyindex{tten}~$n$.


\medskip Slurs are initiated with \keyindex{ilegu}~$np$ and
\keyindex{ilegl}~$np$, where \ital{leg} stands for the Italian word
\ital{legato}, and they are terminated with \keyindex{tleg}~$n$. Except that
slurs start before the note position and stop after, ties and slurs work the
same way with the same syntax. They also share the same registers so that
slurs nesting tied notes should have numbers $n$ distinct from each other.
\subsection{Standard and nicer slurs}

\medskip  Since version 4.7, beginnings and ends of slurs are produced  using
a special font ({\tty slurn16} or {\tty slurn20}) rather than using \TeX\
bracing tokens from {\tty cmex10}. Thus the bending of slurs is now nicer and,
moreover, several slur sizes are available. For example:

\begin{music}
\def\nbinstruments{1}\relax
\computewidths
\debutextrait
\normal
\temps\NOtes\Ilegu0l\qu g\enotes
\NOtes\qu h\enotes
\NOtes\ql i\enotes
\NOtes\tleg0\ql j\enotes
\finextrait
 \end{music}

\noindent has been coded using \keyindex{Ilegu} rather than \keyindex{ilegu}
(resp.\ using \keyindex{Ilegl} rather than \keyindex{ilegl} if a lower slur
is wanted), namely:

\begin{quote}\begin{verbatim}
\NOtes\Ilegu0l\qu g\enotes
\NOtes\qu h\enotes
\NOtes\ql i\enotes
\NOtes\tleg0\ql j\enotes
\end{verbatim}\end{quote}

  It should to be noted that, as a standard, \itxem{ties} still use the \TeX\
bracing symbols; this is mainly done to avoid visibility problems with ties
occurring in the middle of staffs, i.e. in a situation where tie lines could
overlap and be confused with staff lines. But the \musictex\ user can request
using special slur fonts by coding \keyindex{Itenu} or \keyindex{Itenl}
instead of \keyindex{itenu} or \keyindex{itenl}. Even more, big symbol ties
are available using \keyindex{ITenu} or \keyindex{ITenl}.

\medskip  As a standard, only two slur and tie symbol sizes are referred by
\musictex\ macros, but a quick look into {\tty musicnft.tex} show that these
definitions explicitely refer to a character number of the slur font:

\begin{quote}\begin{verbatim}
\def\ITenu#1{\selecttenue{#1}\I@tenu{70}}%
\def\Ilegu#1#2{\selecttenue{#1}{\advance\locx@skip by -1.5\qn@width
\I@tenu{70}{#2}}}%
\def\Itenu#1{\selecttenue{#1}\I@tenu{66}}%
\def\ilegu#1#2{\selecttenue{#1}{\advance\locx@skip by -1.5\qn@width
\I@tenu{66}{#2}}}%
\end{verbatim}\end{quote}

 Then the \musictex\ user can, for his own special needs, define new macros
using other character codes in the range 64--79, knowing that each character
of this range is one point longer than the previous one.

\subsection{Problem with large slur ends}

Due to spacing problems, it may happen that the end of an \keyindex{Ilegu} or
\keyindex{Ilegl} makses some ugly overlap between the currrent slur
horizontal rule and the ending symbol itself. This is due to the ending
symbol being so long that it overlaps the end of the horizontal rule. The
easiest way of removing that slight trouble consists in replacing --- for
example --- {\tty\keyindex{tleg}0} with the more sophisticated commands~:

\medskip\verb|\rlap{\qsk\tleg0}|

\medskip\noindent which delays the position of the closing slur symbol by a
space equal to the note head width.

\subsection{Weird slurs and ties}

In some special cases, usually in romantic music, one may want to invert the
sense of a slur, linking for example a keyboard phrase beginning at the left
hand and ending at the right hand. Thus, the slur of number $n$ (a numeric
vamue, not a roman numeral) can be inverted
by

\medskip\keyindex{Invertslur} $n$

\medskip Besides, for some good reason, one may want to raise a slur at the
next line. This is done by

\medskip\verb|\def\atnextline{|\keyindex{liftslur} $n$\verb|{|
\ital{dimension} \verb|}}|

\medskip\noindent which will lift the slur number $n$ by the given amount at
the next line change. Of course, \keyindex{liftslur} could be called in the middle
of a line, but the result would be rather ugly.

These two features are used at bars number 131--132 of examples {\tty
PACIFIQB} and {\tty PACIFIQN}.

\subsection{Refined slurs for clever typesetters}
  In addition to the previous extension and improvement of existing features,
long nice horizontal slurs are also available, but they can be used only with
two restrictions:

\begin{itemize}

\item They are not managed when \itxem{glue} is inserted, especially by
\keyindex{temps} and \keyindex{barre};

\item They cannot be cut at line breaking.

\end{itemize}

 Thus, their use is recommended only for nice slurs extending a few bars and
not supposed to occur across lines. With this restrictions,

\begin{music}
\def\nbinstruments{1}\relax
\computewidths
\debutextrait
\normal
\temps\NOtes\zcharnote l{\huslur{3.5\noteskip}}\qu g\enotes
\NOtes\qu h\enotes
\NOtes\ql i\enotes
\NOtes\ql j\enotes
\finextrait
\end{music}

\noindent has been obtained by coding:

\begin{quote}\begin{verbatim}
\NOtes\zcharnote l{\huslur{3.5\noteskip}}\qu g\enotes
\NOtes\qu h\enotes
\NOtes\ql i\enotes
\NOtes\ql j\enotes
\end{verbatim}\end{quote}

However, the best solution is now to use \ixem{\musixtex} which systematically
makes beautiful slurs, at the cost of a three pass system instead of one...

\section{Bars}

\subsection{Bars and spacing}


 Ordinary \ital{bars} a coded using the macro \keyindex{barre} (this is a
French word\footnote{Whose advantage is that it differs from {\Bslash
bar} which is already defined in \TeX.}). This macro provides an optional
(discretionary) line or page break\footnote{Unless it is triggered according
to bar counting when \keyindex{autolines} has been invoked.}. It also
provides some \ital{glue} in order to expand the text over an evenly filled
line.


\medskip However, since the number of bars in a score line is generally
small, it may be convenient to allow \ital{glue} not only on each sides of
\ital{bars}. This can be done using the macro \keyindex{temps} (the
French word for \ital{beat}). This macro has two effects:

\begin{enumerate}
 \item it inserts some \ital{glue} but prevents line breaking\footnote{This
could unpleasantly occur if you insert a space...},
 \item if some ties or slurs are pending it expands them across the glue by
writing an \keyindex{hrule} which overlaps the unfinished tie and makes it
look prolongated.
 \end{enumerate}


\medskip
 Whatever the care you exercize in adjusting the size of the
\keyindex{elemskip}, you are still likely to find some broken ties (which
indicate excessive glue disassembling the gliding tie \keyindex{hrule}s) or
some unexpected (and unwanted) line breaks or some Over[aw]full
\keyindex{hbox}es. A useful means of estimating the remaining space to be
filled with glue consists in declaring \keyindex{raggedlinestrue}~: after
that, an \keyindex{hfil} will be inserted by \musictex\ before each computed
(when using \keyindex{autolines}) or forced line break. Thus, all the musical
text will be packed on the left of the line and you will clearly see the
amount of remaining space. Then, it will be up to you do decide changing some
spacing parameters.


\medskip\noindent{\bf Important: \sl do not use \keyindex{temps} when
\ital{beams} are pending, otherwise their spatial synchronization would fail.
 In other words, ties and slurs can jump over glue (because horizontal rules
may overlap and thus have some elasticity) but beams (as well as any oblique
lines) cannot.}

\subsection{Bar numbering}
\rm Unless otherwise specified, \index{bar numbering}bars are
numbered. This is a good means of finding errors provided that the Music\TeX\
user has put comments in his source text recording the (expected) bar number.
However, this can look unpleasant for final outputs, since the habit is to
number bars only each other five or ten bars. This is not a serious problem
since the frequency of bar numbering is defined as:

\medskip{\Bslash def\keyindex{freqbarno}\{1\}}

\medskip
If you replace the {\tty 1} by {\tty 5}, bar numbering will occur each other five
bars. You can also inhibit any bar number printing by telling:

\medskip{\Bslash def\bslash wbarno\{\}}

\medskip\noindent or, in a more clever way:

\medskip{\Bslash def\bslash freqbarno\{9999\}}.

\medskip The bar counter is also accessible, its name is \keyindex{barno}.
This has nothing to do the the bar counting invoked by \keyindex{autolines},
so you can change it without any dramatic consequence. 

\subsection{Full and instrument divided bars}
 Normally, bars (as well as double bars, final bars and repeat bars) are
drawn as a continuous line, starting for the bottom of the lower staff of the
lower instrument, and ending at the top of the upper staff of the upper
instrument. However, one may want to have discontinuous bars, that is, one
continuous bar for all the staffs of a unique instrument. This is done by
issuing the command \keyindex{sepbarrules}. An example of this is given in the
{\tty ANGESCAO} (or {\tty ANGESCAM}) example; it has also been used in the
example of section \ref{avemaria}.

 The initial situation can be forced or restored by
\keyindex{stdbarrules}. 

\section{Line and page breaking}\label{linepagebreak}


Bars provide a line breaking mechanism which is supposed to enable \TeX\
to break a full score into lines and pages, with an optimal distribution
of the text into lines and pages. Unfortunately, this does not work
correctly for scores of more that approximately one page. The reason is
that \TeX\ must compile the entire contents of a paragraph, before it
tries to break it into lines and pages. Therefore, one cannot rely on \TeX\
to make that work which automatically results in the diagnostic {\tty TeX
capacity exceeded, memory...}.
%\check


\medskip To circumvent this dramatically restrictive capacity\footnote{\TeX\
has been designed to type text, not music.} unless you use some
Big\TeX\footnote{Whose drawback is that it is very slow on ordinary PCs.}
another mechanism must be invoked to break lines.\index{Big\protect\TeX}


\medskip The first one is the manual one: you replace some of the {\Bslash
barre} macro calls by either \keyindex{alaligne} (equivalent to
\keyindex{break} within text: in fact it contains a \keyindex{break} plus
some (many) other things). In the same way, you can code {\Bslash alapage} to
force an \keyindex{eject} with proper reinitialization of staffs, clefs and
signatures.


\medskip The second one is fit for scores with
bars of regular length: after \keyindex{debutmorceau}, you code the
following macro:


\medskip\Bslash autolines $tml$

\medskip\noindent\rm where $t$ is the number of \ital{elementary spacings}
(the length of {\Bslash notes...\bslash enotes}) in an average bar, $m$ is
the number of bars you wish in a line and $l$ is the number of lines you wish
in a page\footnote{After having coded nearly one hundred of pages of music, I
strongly recommend the use of \keyindex{autolines} except when inserting
short excerpts of less than one line, such as in musicographic books.}.

\medskip This sets some parameters, namely \keyindex{maxbarsinline} and
\keyindex{maxlinesinpage} which are simply used to count the bars, optionally
perform \keyindex{alaligne} or \keyindex{alapage} instead of the normal
\keyindex{barre}. You may freely alter the values of these parameters, once
they have been established by \keyindex{autolines}. Moreover, you can still
force line breaking of page ejection using {\Bslash alaligne} or
\keyindex{alapage} without problem since these macros actually reset the bar
counters appropriately.


\medskip On the other hand, you may want to forbid line breaking at a bar,
then you replace {\Bslash barre} by \keyindex{xbarre}.


\medskip Conversely, you may want to
break a line \ital{not at a bar}\footnote{For example, you may prefer to turn
the page at a place where the pianist has one hand free.}. This is allowed by
\keyindex{zbarre} (optional line break) or forced by {\Bslash
zalaligne} or \keyindex{zalapage}.


\medskip
The final heavy double bar of a piece is provided by \keyindex{finmorceau}.
If you just want to terminate the text with a simple bar, you say
\keyindex{suspmorceau}. If you want to terminate it without a bar, you code
\keyindex{zsuspmorceau}. Once you have stopped the score by any of these
means, you can restart it using \keyindex{reprmorceau}.

However,  using {\Bslash reprmorceau} after a {\Bslash finmorceau} may have a
little drawback: if {\Bslash finmorceau} occurs at the bottom of a page
defined by \keyindex{autolines}, it does not cause a page break which could
lead to en empty page after the end of the piece but, conversely, this might
cause another trouble if this only was the end of a scora part, not the end of
the whole score. Therefore, an alternate command has been provided, namely
\keyindex{Suspmorceau} which duly performs the page break, in order to be
ready to score another part of a complex peace. Thus, as a rule~:

\begin{itemize}
 \item use \keyindex{finmorceau} at the very end of your score,
 \item use \keyindex{Suspmorceau} at the end of the parts of a complex piece.

\end{itemize}


\medskip If you want the next vertical bar to be a double bar, you have to
declare \keyindex{setdoublebar} before the \keyindex{barre} (or the
\keyindex{suspmorceau} or \keyindex{alaligne} or \keyindex{alapage}) to be
marked with a double thin bar.


\section{Changing score attributes}


 As seen before, you can change the
signature of the whole set of instruments by \keyindex{generalsignature}~$n$
where $n>0$ means a number of sharps, $n<0$ means a number of flats. Or, you
may prefer to change the signature of only one or two instruments by the
statement:


\medskip\Bslash sign$r$=$s$

\medskip\rm\noindent
 %\check
 where $r$ is the roman numeral of the instrument considered, and $s$ its
specific signature. Since you may change simultaneously (with respect to the
score) but consecutively (with respect to your code) the signatures of
several instruments, this change takes place only when you say
\keyindex{changesignature} (within a bar) or \keyindex{changecontext} (after
a single vertical rule) or \keyindex{Changecontext} (after a double vertical
rule).
%\check
 In the same way, you may want to change the active clefs.
This is done by

\medskip\tty\keyindex{cleftoks$r$}=\{\{$s1$\}\{$s2$\}\{$s3$\}\{$s4$\}\}\%

\medskip\rm\noindent where $r$ is the roman numeral of the instrument, $s1$
specifies the clef of the lower staff, $s2$ the clef of the second staff,
etc. One must always give four values with the above syntax, otherwise\dots\
$s1=6$ means the \ital{bass} clef (\ixem{clef de fa} in French), $s1=0$ means
the \ital{violin} clef (\ixem{clef de sol} in French), $s1=1$ through $s1=5$
mean the \ital{alto} clef (\ixem{clef d'ut} in French) set on first (lower)
through fifth (upper line of the staff). \index{violin clef}\index{bass
clef}\index{alto clef}
 As seen above in the case of signatures, several clefs may be changed at the
same time; thus all the clef changes become operational only when the macro
\keyindex{changeclefs} is coded. Normal usage consists in issuing this
command before the bar, not after (this helps the music player when the
change happens across a line break).

 The \verb|\changeclefs| command normally takes some horizontal space to put
one of more clef symbols, but it may happen that you have no notes
immediately before on the staff whose clef is changed. In that case, you can
use \keyindex{zchangeclefs} which performs the same posting, overwritten left on
the last part of the score in that staff. Of course it is your responsability
to ensure that no notes will collide with the clef change symbols. 
 %\check

 Caution should be exercized changing clefs when \itxem{beams} are pending:
in fact the \verb|\changeclefs| and \verb|\zchangeclefs| perturbate the
computation of beams and this command should be invoked only
 \begin{itemize}
 \item out of {\Bslash notes}, {\Bslash Notes}... --- {\Bslash enotes} pairs;
 \item only when no beam is pending\footnote{No problem with slurs and ties.}.
\end{itemize}

If a clef change has to be typeset when one or several beams are pending, the
operation should be done in a more manual way:
\begin{itemize}
 \item say \keyindex{saveclefs} instead of \keyindex{changeclefs}~:
This actually records the \keyindex{cleftoks$r$} statement and no further
{\Bslash changeclefs} will repeat the clef change; the new clef will not
be typeset by \verb|\saveclefs| itself but rather by the next \verb|\notes| or
\verb|\Notes| command; 
 \item say \verb|\charnote0{\smallclefdesol}| (resp.
\verb|\charnote0{\smallclefdefa}| or \verb|\charnote0{\smallclefdut}|) where
you like it and at the right staff (i.e.  within  {\Bslash notes}, {\Bslash
Notes}... --- {\Bslash enotes} pairs) to type set the wanted clef symbol.
 \end{itemize}
 As an example, an excerpt of {\sc Brahms}'s Intermezzo op.~117,1 provided by
Miguel {\sc Filgueiras}:\index{Brahms}

\begin{music}
\parindent 1cm
\def\nbinstruments{1}\relax
\def\freqbarno{9999}% no bar numbers
\nbporteesi=2\relax
\cleftoksi={6000}% F- and G-clefs
\generalsignature{-3}% 3 flats
\relax
\debutextrait
\normal
\off{1em}\temps\Notes\larpeggio{E}5\zq{EI}\qu{N}\charnote0{\smallclefdesol}\relax
\nextstaff\qsk\ibl0e{-1}\zq{eg}\qb0l\zq{d}\qb0k\enotes
\cleftoksi={0000}\saveclefs\Notes\tbu0\zq{ce}\qh0j\relax
\nextstaff\zq{sn}\cl{l}\enotes
\Notes\zq{be}\qu{i}\nextstaff\zq{sn}\ql{l}\enotes
\cleftoksi={6000}\changeclefs\Notes\zq{E}\cu{I}\relax
\nextstaff\ibbu1h{-1}\zq{ae}\qh1h\tbu1\zq{N}\qh1g\enotes
\finextrait
 \end{music}
\noindent which was coded as:
 \begin{verbatim}
\begin{music}
\def\nbinstruments{1}\relax
\def\freqbarno{9999}% no bar numbers
\nbporteesi=2\relax
\cleftoksi={6000}\generalsignature{-3}% 3 flats
\debutextrait
\normal
\off{1em}\temps\Notes\larpeggio{E}5\zq{EI}\qu{N}%
\charnote0{\smallclefdesol}\relax
\nextstaff\qsk\ibl0e{-1}\zq{eg}\qb0l\zq{d}\qb0k\enotes
\cleftoksi={0000}\saveclefs\Notes\tbu0\zq{ce}\qh0j\relax
\nextstaff\zq{sn}\cl{l}\enotes
\Notes\zq{be}\qu{i}\nextstaff\zq{sn}\ql{l}\enotes
\cleftoksi={6000}\changeclefs\Notes\zq{E}\cu{I}\relax
\nextstaff\ibbu1h{-1}\zq{ae}\qh1h\tbu1\zq{N}\qh1g\enotes
\finextrait
\end{music}
 \end{verbatim} 

\medskip \ital{Meter} changes are implemented the same way:

\medskip\Bslash metertoks$r$=\{\{$m1$\}\{$m2$\}\{$m3$\}\{$m4$\}\}\%

\medskip\rm\noindent
 where $r$ is the roman numeral of the instrument, $m1$ specifies the meter
of the lower staff, $m2$ the meter of the second staff, etc. One must always
give four values with the above syntax, otherwise\dots\ Since meter changes
are meaningful only across bars, they are actually taken in account with
\keyindex{changecontext} or \keyindex{Changecontext} or \keyindex{alaligne}
or \keyindex{alapage}.


\section{Repeats}
%\check


 To insert a \ital{repeat bar} you can use several sets of procedures.

\subsection{Elementary and unsafe solutions} The
simplest consists in using the commands \keyindex{leftrepeatsymbol}, 
\keyindex{rightrepeatsymbol} and \keyindex{leftrightrepeatsymbol} -- coded
outside the {\Bslash notes...\bslash enotes} pairs -- which will simply
insert these colon adorned double bars at the requested place. For example:

\begin{music}
 \def\nbinstruments{1}
 \generalsignature{0}\savesignature
 \nbporteesi=1\cleftoksi={{0}{0}{0}{0}}\relax
 \debutextrait\normal
 \notes\hu g\enotes
 \leftrepeatsymbol
 \notes\hu h\enotes
 \leftrightrepeatsymbol
 \notes\hu i\enotes
 \rightrepeatsymbol
 \Notes\wh j\enotes
 \finextrait
\end{music} 

 \noindent has been coded as:
%\check


\begin{quote}\begin{verbatim}
\notes\hu g\enotes
\leftrepeatsymbol
\notes\hu h\enotes
\leftrightrepeatsymbol
\notes\hu i\enotes
\rightrepeatsymbol
\Notes\wh j\enotes
\end{verbatim}\end{quote}

\medskip
 However, the previous way of coding does not provide for line breaking
at repeat bars, nor does it advance the bar numbering. In fact, this way of
coding if only fit for repeats occurring in the middle of a bar.


\medskip A second way of coding consists in saying \keyindex{setleftrepeat},
\keyindex{setrightrepeat} or \keyindex{setleftrightrepeat} before a bar
(\keyindex{barre}), \keyindex{suspmorceau} or \keyindex{changecontext}). In
this case, the next single\footnote{\keyindex{setrightrepeat} does not change
the effect of \keyindex{finmorceau}; in that case use
\keyindex{finrightrepeat}.} vertical bar will be replaced with the selected
repeat bar. This meets the traditional music typesetting conventions in the
only case of the \ital{right repeat} but, unfortunately, left and left/right
repeats use to behave in a different manner when in the middle of a line and
at a line break.


\subsection{Safe and correct coding} The third coding, namely the correct coding -- i.e. transforming in
the correct manner when occurring at a line break -- is obtained by
substituting the \keyindex{barre} command with \keyindex{leftrepeat},
\keyindex{rightrepeat} and \keyindex{leftrightrepeat}.
%\check


\medskip
 Now, if you want to force a new line at a repeat, you should code
respectively:


\medskip\keyindex{setrightrepeat}\keyindex{alaligne}

 \keyindex{setrightrepeat}\keyindex{suspmorceau}

 {\Bslash alaligne}\keyindex{leftrepeatsymbol}

 \keyindex{reprmorceau}\keyindex{leftrepeatsymbol}

 \keyindex{debutmorceau}\keyindex{leftrepeatsymbol}

\medskip\noindent or the combination of two of these in the case of a
left/right repeat.

\subsection{Specific first and second pass scoring}

A frequent situation consists in a long repeated score, but the last few bars
are different at first pass and at second pass. This can be specified by
saying \keyindex{setprimavolta} or \keyindex{setsecondavolta}\footnote{From
the Italian \ital{prima volta} meaning ``first time'', and \ital{seconda
volta} meaning ``second time''.} before the next \keyindex{barre} or
\ital{repeat} (usually {\Bslash rightrepeatsymbol}). For example~:


\begin{music}
 \def\nbinstruments{1}
 \generalsignature{0}\savesignature
 \nbporteesi=1\cleftoksi={{0}{0}{0}{0}}\relax
 \debutextrait\normal
 \Notes\qu {ghij}\enotes
 \setprimavolta\barre
 \Notes\qu {hijk}\enotes
 \setsecondavolta\setrightrepeat\barre
 \Notes\qu {ijkl}\enotes
 \barre
 \NOTes\wh j\enotes
 \finextrait
\end{music} 

 \noindent has been coded as:
\begin{quote}\begin{verbatim}
 \def\nbinstruments{1}
 \generalsignature{0}\savesignature
 \nbporteesi=1\cleftoksi={{0}{0}{0}{0}}\relax
 \debutextrait\normal
 \Notes\qu {ghij}\enotes
 \setprimavolta\barre
 \Notes\qu {hijk}\enotes
 \setsecondavolta\setrightrepeat\barre
 \Notes\qu {ijkl}\enotes
 \barre
 \NOTes\wh j\enotes
 \finextrait
\end{verbatim} \end{quote}

The ``prima volta'' and ``seconda volta'' symbols are normally set at {\tty
2\bslash Interligne} above the upper line of the staff, and one centimeter
wide. If the music typesetter does not like that, he can issue for example:

\medskip\keyindex{Setprimavolta}\verb|{4\Interligne}{0.78in}|

\medskip\noindent where the first argument is the vertical offset, and the
second one is the length of the upper horizontal line. A symmetrical feature is
\keyindex{Setsecondavolta} with also two dimension arguments.

Note that these special settings are only valid for one usage; after that,
they are reset to the standard values.

%\check
\subsection{Large scope repeats} Large scope repeats have also been provided
special symbols, namely using \keyindex{coda} and \keyindex{segno}~:

 \begin{music}
\def\nbinstruments{1}\relax
\computewidths
\nbporteesi=1\relax
\debutextrait
\normal
\notes\sk\enotes
\temps\NOtes\coda l\enotes
\temps\NOtes\segno l\enotes
\finextrait
 \end{music}

\medskip\noindent which have been coded:

\begin{quote}\begin{verbatim}
\NOtes\coda l\enotes
\NOtes\segno l\enotes
\end{verbatim}\end{quote}

%\check
 \section{Miscellaneous}

\subsection{Putting anything anywhere}

Special macros are provided to help the composer to set any \TeX\
text on the staffs. The macro

\medskip\keyindex{charnote}~$p$\tty\{ {\rm text} \}

\medskip\rm\noindent
sets the given text with its base line at pitch $p$ of the current staff
(this means it must be coded inside {\Bslash notes...\bslash enotes}).
Whatever the length of the text, the spacing is \keyindex{noteskip}.
If you do not want it to cause spacing, you code \keyindex{zcharnote}.
If you want the possible spilling text to expand on the left rather than on the
right, then you can use \keyindex{lcharnote}.
%

\medskip
To place some text at the mid-position between the two staffs of
a keyboard instrument, you may code:

\medskip\tty\keyindex{midtwotext}\{ {\rm text} \}\ \ \% \rm (spacing)

\medskip\tty\keyindex{zmidtwotext}\{ {\rm text} \}\ \ \% \rm (non spacing)

\medskip\rm\noindent being however
careful, a) to put it inside {\Bslash notes...\bslash enotes}, b) to code it in the
text of the lower staff.

\medskip A text to be put above the current staff is introduced by
{\tty\keyindex{uptext}\{...\}}. This may however cause some collision with bar
numbering or notes above the staff; it is then wise to use {\tty
\keyindex{Uptext}\{...\}} which puts the text two pitches higher (recommended
to post the tempo).

\medskip
The macro \keyindex{zcharnote} is fit for coding special notations like
accents above or below the notes. It behaves like \keyindex{charnote} but
causes no spacing.
\subsection{Metronomic indications}

Metronomic indication deserves a special macro. The mention:
 \par\centerline{\def\nbinstruments{0}\computewidths
\metron{\hup}{60}}\smallskip\noindent
 is coded by {\tty\keyindex{metron}\{\bslash hup\}\{60\}} (normally embedded
in \keyindex{Uptext} which is in turn embedded within {\Bslash
notes...\bslash enotes}).



\subsection{Usual ornaments}

%\check
\ital{Arpeggios} (i.e. \raise -4pt\hbox{\musicxx\char92}\ ) can be coded with
 the macro

\medskip\keyindex{arpeggio} $pm$

\medskip\rm\noindent where $p$ is the pitch of
the base of the arpeggio symbol and $m$ is its multiplicity (one period is
equal to one space between staff lines, i.e. 5 points). This macro causes
a space of one note head width. If should be issued before the concerned
chords. Its variant \keyindex{larpeggio} sets the arpeggio symbol slightly
more on the left, in order to avoid collision with accidentals in front of
the chords.

 Note the \keyindex{arpeggio} and \keyindex{larpeggio} make an extra spacing
of width one notehead (i.e; \keyindex{qsk}) so that it is wise to insert a
{\Bslash qsk} in the other staffs before notes which are syncrhonous to the
arpeggiated note.

\medskip \ital{Trills} can be coded in several ways. \keyindex{trille}~$l$
(where $l$ is a \TeX\ dimension) yields \hbox{\trille{1cm}} while
\keyindex{Trille}~$l$ (where $l$ is a \TeX\ dimension) yields
\hbox{\Trille{2cm}}. To put these patterns at a given pitch, one may use
\keyindex{xtrille}~$pl$ or \keyindex{xTrille}~$pl$. On the other hand
\keyindex{ntrille}~$pn$ is equivalent to
\keyindex{xtrille}~$p${\tty\{$n$\bslash noteskip\}} and
\keyindex{nTrille}~$pn$ is equivalent to {\tty
\keyindex{xTrille}~$p$\{$n$\bslash noteskip\}}


\medskip Other \ital{ornaments} are available:

 \begin{itemize}
 \item \keyindex{mordant}~$p$ for \hbox to 1.5em{\mordant 0\hss},
 \item \keyindex{pince}~$p$ for \hbox to 1.5em{\pince 0\hss},
 \item \keyindex{Pince}~$p$ for \hbox to 1.5em{\Pince 0\hss},
 \item \keyindex{Lpince}~$p$ for \hbox to 2em{\hss\Lpince 0\hss} (thanks to A.
{\sc Egler})\index{Egler, A.},
 \item \keyindex{Pincesw}~$p$ for \hbox to 2em{\hss\Pincesw 0\hss},
 \item \keyindex{Pincene}~$p$ for \hbox to 2em{\hss\Pincene 0\hss},
 \item \keyindex{Pincenw}~$p$ for \hbox to 2em{\hss\Pincenw 0\hss},
 \item \keyindex{turn}~$p$ for \hbox to 2.5em{\kern 0.6em\turn 0\hss},
 \item \keyindex{backturn}~$p$ for \hbox to 2.5em{\kern 0.6em\backturn 0\hss},
 \item \keyindex{upz}~$p$ (upper \ital{pizzicato}) to put a dot above a note
head at pitch $p$,
 \item \keyindex{lpz}~$p$ (lower \ital{pizzicato}) to put a dot below a note
head at pitch $p$,
 \item \keyindex{usf}~$p$ (upper \itxem{sforzando}) to put a $>$ accent above
a note head at pitch $p$,
 \item \keyindex{lsf}~$p$ (lower \itxem{pizzicato}) to put a $>$ accent below
a note head at pitch $p$,
 \item \keyindex{ust}~$p$ (upper \itxem{staccato} or \itxem{portato}) to put a
hyphen above a note head at pitch $p$,
 \item \keyindex{lst}~$p$ (lower \ital{staccato} or \ital{portato}) to put a
hyphen below a note head at pitch $p$,
 \item \keyindex{uppz}~$p$ (upper strong \ital{pizzicato}) to put an
apostrophe above a note head at pitch $p$,
 \item \keyindex{lppz}~$p$ (lower strong \ital{pizzicato}) to put a reversed
apostrophe below a note head at pitch $p$.
 \item \keyindex{pointdorgue}~$p$ puts a \itxem{fermata} (in French ``point
d'orgue'') at pitch $p$. No spacing occurs.
 \item \keyindex{pointdurgue}~$p$ puts a reverse \ital{fermata} at the same
place.
 \item \keyindex{PED} to insert the piano pedal command below the staff;
pedal release is specified by \keyindex{DEP}; thus the following example

 \begin{music}
\def\nbinstruments{1}\relax
\computewidths
\cleftoksi={{6}{0}{0}{0}}\relax
\nbporteesi=2\relax
\debutextrait
\normal
\temps\NOtes\qsk\PED\wh J|\qsk\qu h\enotes
\temps\NOtes|\qu g\enotes
\temps\NOtes|\hu k\enotes
\temps\Notes\DEP\enotes
\finextrait
 \end{music}

 \noindent was coded as:

\begin{verbatim}
\temps\NOtes\qsk\PED\wh J|\qsk\qu h\enotes
\temps\NOtes|\qu g\enotes
\temps\NOtes|\hu k\enotes
\temps\Notes\DEP\enotes
\end{verbatim}

 \end{itemize}

%\check

\subsection{New line synchronization of coding}

 The procedure named \keyindex{everystaff} is executed each time a new system
is typed. It is normally void, but it can be defined (simply by {\Bslash
def\keyindex{everystaff}\{...\}}) to tell \musictex\ to post anything
reasonable at the beginning of each system. It was used in the example
{NWIDOR} to post octaviation dashed lines at the end of the piece.


\medskip
 The procedure named \keyindex{atnextline}, normally void, is executed at the
next computed (through \keyindex{autolines}) or forced line break (using
\keyindex{alaligne} or \keyindex{alapage}). More precisely, it is executed
after the break and before the next system is typed. Thus it is fit for
posting new definitions of layout parameters, when no system is
pending\footnote{Its logic is similar to plain \TeX's \keyindex{vadjust}
command.}. 

 \medskip
 \index{tenor violin clef}In some scores, tenor parts are not code using the
\ital{bass} clef, but using rather the \ital{violin clef} subscripted by a
{\tty 8}. This is not directly supported by the \keyindex{cleftoks$r$}{\tty=}
command, but it can be handled using \keyindex{everystaff} and \keyindex{zcharnote}.
As an example the following score

\begin{music}
\def\nbinstruments{4}
\cleftoksi={6000}
\cleftoksii={0000}
\cleftoksiii={0000}
\cleftoksiv={0000}
\font\eightfig=cmr8
\def\everystaff{\znotes&\zcharnote{-6}{\eightfig
  \kern -1.4\Interligne 8}&\zcharnote{-6}{\eightfig
  \kern -1.4\Interligne 8}\enotes}
\debutextrait
\normal
\NOtes\hu{HIJK}&\hu{efgh}&\hl{hijk}&\hl{hmlk}\enotes
\finextrait
\end{music}
\noindent was coded as:

\begin{quote}\begin{verbatim}
\def\nbinstruments{4}
\cleftoksi={6000}
\cleftoksii={0000}
\cleftoksiii={0000}
\cleftoksiv={0000}
\font\eightfig=cmr8
\def\everystaff{\znotes&\zcharnote{-6}{\eightfig
  \kern -1.4\Interligne 8}&\zcharnote{-6}{\eightfig
  \kern -1.4\Interligne 8}\enotes}
\debutextrait
\normal
\NOtes\hu{HIJK}&\hu{efgh}&\hl{hijk}&\hl{hmlk}\enotes
\finextrait
\end{verbatim}\end{quote}
 where the \verb|\eightfig| font declaration can obviously be omitted if
the \TeX\ format declares an 8 point roman font.

%\check

\subsection{Beams across bars}

The \keyindex{barre} macro inserts glue and terminates spacing account. Thus
it is not compatible with pending \itxem{beams} and this is the usual way most
composers wrote their scores, at least until the beginning of the XIX-th
century. Unfortunately, later composers ({\sc Brahms}, {\sc Scriabin}, etc.)
wanted to write beams jumping across bars.

This can be partly handled by \musictex, using the macro
\keyindex{xbeambarre}\footnote{An ugly mixture of French and English which is
therefore not likely to be redefined by somebody else.} which puts a bar line
of global width {\Bslash elemskip} with no glue around. Bar counting is done
for \keyindex{autolines} purpose but no line breaking can occur, which means
that the beams going across line or page breaks are not supported.

We give an example from {\sc Brahms}'s Intermezzo op.~118,1 provided by
Miguel {\sc Filgueiras}:\index{Brahms}

\begin{music}
\def\interfacteur{13}
\def\nbinstruments{1}
\nbporteesi=2\relax
\cleftoksi={6000}
\generalmeter{\allabreve}
\debutextrait
\normal
\off{1em}\temps\Notes\qp\nextstaff\Ilegu0r\zq{q}\ql{j}\enotes
\barre
\Notes\qsk\ibu0a1\qh0{CEJLcL}\relax
\nextstaff\qsk\rw{l}\pt{p}\zh{_p}\pt{i}\hl{_i}\enotes
\Notes\qh0J\itenl1a\qh0a\nextstaff\tleg0\zq{o}\ql{h}\enotes
\xbeambarre
\Notes\zh{.L.a}\hl{.e}\relax
\nextstaff\tten1\qb0{chj}\tbl0\qb0l\cl{q}\ds\enotes
\Notes\qp\nextstaff\zq{q}\ql{j}\enotes
\finextrait
 \end{music}
\noindent whose coding is:

\begin{verbatim}
\begin{music}
\def\interfacteur{13}
\def\nbinstruments{1}
\nbporteesi=2\relax
\cleftoksi={6000}
\generalmeter{\allabreve}
\debutextrait
\normal
\off{1em}\temps\Notes\qp\nextstaff\Ilegu0r\zq{q}\ql{j}\enotes
\barre
\Notes\qsk\ibu0a1\qh0{CEJLcL}\relax
\nextstaff\qsk\rw{l}\pt{p}\zh{_p}\pt{i}\hl{_i}\enotes
\Notes\qh0J\itenl1a\qh0a\nextstaff\tleg0\zq{o}\ql{h}\enotes
\xbeambarre
\Notes\zh{.L.a}\hl{.e}\relax
\nextstaff\tten1\qb0{chj}\tbl0\qb0l\cl{q}\ds\enotes
\Notes\qp\nextstaff\zq{q}\ql{j}\enotes
\finextrait
 \end{music}
\end{verbatim}

%\check
\section{Small and tiny notes}

Before entering details, let us point out that we are presently concerned with
typing notes of smaller size than the normal one, without attempting to change
the interval between the five lines building a single staff. Changing staff
line interval will be treated in a further section.

\subsection{Cadenzas and explicit ornaments}

 Ornaments and \itxem{cadenzas} usually need to be written using smaller
notes\footnote{This is independent of the staff size.}. This can be done
everywhere by stating \keyindex{smallnotesize} or \keyindex{tinynotesize}.
Normal note size is restored by \keyindex{normalnotesize}.\index{ornaments}

 These macros only have a local scope. Thus, if these macros are invoked
outside the {\Bslash notes...\bslash enotes} pair, the change is valid for
the rest of the piece unless explicitely modified but, if they are invoked
inside, their effect is local to the current staff of the current  {\Bslash
notes...\bslash enotes} pair. As an example, the following excerpt (beginning
of the Aria of the ``Creation'' by Joseph {\sc Haydn})\index{Haydn, J.}

\begin{music}
\def\DS{\hbox{\ds}}
\def\FS{\hbox{\kern 0.3\noteskip\soupir}\kern -0.3\noteskip}
\def\qbl#1#2#3{\ibl{#1}{#2}{#3}\qb{#1}{#2}}%
\def\qbu#1#2#3{\ibu{#1}{#2}{#3}\qh{#1}{#2}}%
\def\nbinstruments{2}%
\generalmeter{\meterfrac{4}{4}}%
\signaturegenerale{0}%
\nbporteesii=1\relax
\nbporteesii=2\relax
\cleftoksi={{6}{0}{0}{0}}
\cleftoksii={{6}{0}{0}{0}}
\etroit     
\debutextrait
\NOTes\soupir&\soupir|\qu g\enotes
% mesure 1
\advance\barno by -1\relax
\barre\NOtes\itenu2J\wh J&\zw N\ibl0c0\qb0e|\qu j\enotes
\notes&\qbl0c0|\noteskip=0.6\elemskip\tinynotesize
\Ibbu1ki2\qh1{kj}\tqh1i\qsk\enotes
\Notes&\qb0e\tbl0\qb0c|\qu j\enotes
\temps\Notes&\ibl0c0\qb0{ece}\tbl0\qb0c|\ql l\sk\ql j\enotes
% mesure 2
\barre\Notes\tten2\wh J&\ql J\sk\ql L|\ppt g\rlap{\qu g}\qbl1e0\relax
       \zq c\qb1e\zq c\qb1e\relax
       \zq c\tbl1\rlap{\qb1e}\ \ \ccu h\enotes
\temps\Notes&\ql N\sk\pt L\ibl0L{-4}\qb0L|\ibl1e0\zq c\rlap{\qb1e}\cu g\relax
       \zq c\rlap{\qb1e}\raise\Interligne\DS \rlap{\qu g}\qb1g\enotes
\notes&\sk\tbbl0\tbl0\qb0J|\tbl1\zq c\qb1e\enotes
\finextrait
\end{music}

\noindent can be coded as:

\begin{verbatim}
\def\DS{\hbox{\ds}}
\def\FS{\hbox{\kern 0.3\noteskip\soupir}\kern -0.3\noteskip}
\def\qbl#1#2#3{\ibl{#1}{#2}{#3}\qb{#1}{#2}}%
\def\qbu#1#2#3{\ibu{#1}{#2}{#3}\qh{#1}{#2}}%
\def\nbinstruments{2}%
\generalmeter{\meterfrac{4}{4}}%
\signaturegenerale{0}%
\nbporteesii=1\relax
\nbporteesii=2\relax
\cleftoksi={{6}{0}{0}{0}}
\cleftoksii={{6}{0}{0}{0}}
\etroit     
%
%  end of preliminary definitions
%
\debutextrait
\NOTes\soupir&\soupir|\qu g\enotes
% mesure 1
\advance\barno by -1\relax
\barre\NOtes\itenu2J\wh J&\zw N\ibl0c0\qb0e|\qu j\enotes
\notes&\qbl0c0|\noteskip=0.6\elemskip\tinynotesize
\Ibbu1ki2\qh1{kj}\tqh1i\qsk\enotes
\Notes&\qb0e\tbl0\qb0c|\qu j\enotes
\temps\Notes&\ibl0c0\qb0{ece}\tbl0\qb0c|\ql l\sk\ql j\enotes
% mesure 2
\barre\Notes\tten2\wh J&\ql J\sk\ql L|\ppt g\rlap{\qu g}\qbl1e0\relax
       \zq c\qb1e\zq c\qb1e\relax
       \zq c\tbl1\rlap{\qb1e}\ \ \ccu h\enotes
\temps\Notes&\ql N\sk\pt L\ibl0L{-4}\qb0L|\ibl1e0\zq c\rlap{\qb1e}\cu g\relax
       \zq c\rlap{\qb1e}\raise\Interligne\DS \rlap{\qu g}\qb1g\enotes
\notes&\sk\tbbl0\tbl0\qb0J|\tbl1\zq c\qb1e\enotes
\finextrait
\end{verbatim}

 \subsection{Grace notes}

 Grace notes are a special case of small and tiny notes: the difference is
that they are always coded as eighth notes with an oblique bar over the flag.
To perform this, special variants of \keyindex{cu} and \keyindex{cl} have been
provided, namely \keyindex{grcu} and \keyindex{grcl}, with the only difference
that the flag has been slashed. Using this together with the note reduction
macro, grace notes (optionally chord grace notes) can be easily coded:

  \begin{music}
  \def\nbinstruments{1}
  \nbporteesi=1\relax
  \cleftoksi={{0}{0}{0}{0}}\relax
  \debutextrait
  \normal
  \NOtes\qsk\hu h\enotes
  \temps\smallnotesize
  \notes\grcu j\enotes
  \temps\normalnotesize
  \NOtes\hu i\enotes
  \barre\tinynotesize
  \notes\qsk\zq h\grcl j\enotes
  \normalnotesize
  \NOTEs\wh i\enotes
  \finextrait
  \end{music}

  The previous example was coded as:

\begin{quote}\begin{verbatim}
\NOtes\qsk\hu h\enotes
\temps\smallnotesize
\notes\grcu j\enotes
\temps\normalnotesize
\NOtes\hu i\enotes
\barre\tinynotesize
\notes\qsk\zq h\grcl j\enotes
\normalnotesize
\NOTEs\wh i\enotes
\end{verbatim}\end{quote}

\subsection{Gregorian note shapes}\label{gregnotes}
 Provided that four line staffs are used, \itxem{gregorian music} was
frequently quoted using specific \itxem{neumes}, and this way of coding
has been commonly used for the coding of liturgical chant in the Catholic
Church until the middle of the twentieth century. Symbols available in
\musictex\ are~:

\begin{itemize}
 \item Diamond shaped \itxem{punctum}: \raise 2.5pt\hbox{\musicxx\char 0} =
\keyindex{diamg}~$p$~,
 \item Square \itxem{punctum}: \raise 2.5pt\hbox{\musicxx\char 1} =
\keyindex{carrg}~$p$~,
 \item Left stemmed \itxem{virga}: \raise 2.5pt\hbox{\musicxx\char 2} =
\keyindex{carpg}~$p$~,
 \item Right stemmed \itxem{virga}: \raise 2.5pt\hbox{\musicxx\char 3} =
\keyindex{carqg}~$p$~. \end{itemize}

They all have a non-spacing variant, which are~:

\begin{itemize}
 \item Non spacing diamond shaped \itxem{punctum}: \raise
2.5pt\hbox{\musicxx\char 0} = \keyindex{zdiamg}~$p$~,
 \item Non spacing square \itxem{punctum}: \raise 2.5pt\hbox{\musicxx\char 1}
= \keyindex{zcarrg}~$p$~,
 \item Non spacing left stemmed \itxem{virga}: \raise
2.5pt\hbox{\musicxx\char 2} = \keyindex{zcarpg}~$p$~,
 \item Non spacing right stemmed \itxem{virga}: \raise
2.5pt\hbox{\musicxx\char 3} = \keyindex{zcarqg}~$p$~.
 \end{itemize}

To memorize these symbols, remember that {\tt carr} comes from the French
word \ital{carr\'e} which means \ital{square}, that the letter {\tt p} has a
tail on the left side and the {\tt q} has a tail on the right side. Other
\itxem{neumes} can ne obtained by combining two or more of these symbols


\subsection{Percussion stemless note shapes}\label{percunotes}
Percussion\index{percussion music} music often uses stemless note heads
different from the usual ones. 

\begin{itemize}
 \item  White diamond or \itxem{rimshot}: \raise 2.5pt\hbox{\musicxx\char 127} =
\keyindex{diamw}~$p$~,
 \item  The \itxem{shaker}: \raise 2.5pt\hbox{\musicxx\char 79} =
\keyindex{shaker}~$p$~,
 \item  The {tremolo chord}: \hbox to 2em{\hss\musicxx\char 78\hss} =
\keyindex{tremolord}~$p$~,
\end{itemize}


They all have a non-spacing variant\footnote{Stemless, not to be confused with
{\bslash zq} qui extends stems within a chord. For that purpose, see
\ref{genpercus} and \ref{othernotes}.}, which are~:

\begin{itemize}
 \item  Non spacing white diamond or \itxem{rimshot}: \raise
2.5pt\hbox{\musicxx\char 127} = \keyindex{zdiamw}~$p$~,
 \item  Non spacing \itxem{shaker}: \raise 2.5pt\hbox{\musicxx\char 79} =
\keyindex{zshaker}~$p$~,
 \item  Non spacing {tremolo chord}: \hbox to 2em{\hss\musicxx\char 78\hss} =
\keyindex{ztremolord}~$p$~,
\end{itemize}


\subsection{Other note shapes}\label{othernotes} The classical note heads
given above --- namely \raise 0.5ex\hbox{\musicnorfont\char 33}~, \raise
0.5ex\hbox{\musicnorfont\char34} and \raise 0.5ex\hbox{\musicnorfont\char35}
--- can be replaced with less classical note heads, for example to code
special \itxem{violin harmonic notes} or \itxem{percussion music}. See an
example in \ref{abnormalscores}

 At present time, alternate available note heads are~:
 \begin{itemize}
 
 \item The \raise 0.5ex\hbox{\musicnorfont\char 0} symbol which is obtained using the \verb|\qu|,
\verb|\qh| etc. macros replacing the ``{\tt q}'' letter in the macro name
with a ``{\tt y}'' or writing \keyindex{ycu},  \keyindex{ycl},
\keyindex{yccu}, \keyindex{ycup} etc. instead of \verb|\cu|, \verb|\cl|,
\verb|\ccu|, \verb|\cup|, etc.

 \item The \raise 0.5ex\hbox{\musicnorfont\char 127} symbol which is obtained
using the \verb|\qu|, \verb|\qh| etc. macros replacing the ``{\tt q}'' letter
in the macro name with a ``{\tt d}'' (think of \ital{diamond}) or writing
\keyindex{dcu},  \keyindex{dcl}, \keyindex{dccu}, \keyindex{dcup}, etc.
instead of \verb|\cu|, \verb|\cl|, \verb|\ccu|, \verb|\cup|, etc.

 
 \item The \raise 0.5ex\hbox{\musicnorfont\char 39} symbol which is obtained
using the \verb|\qu|, \verb|\qh| etc. macros replacing the ``{\tt q}'' letter
in the macro name with a ``{\tt k}'' or writing \keyindex{kcu}, 
\keyindex{kcl}, \keyindex{kccu}, \keyindex{kcup}, etc. instead of \verb|\cu|,
\verb|\cl|, \verb|\ccu|, \verb|\cup|, etc.

 
 \item The \raise 0.5ex\hbox{\musicnorfont\char"35} symbol which is obtained
using the \verb|\qu|, \verb|\qh| etc. macros replacing the ``{\tt q}'' letter
in the macro name with a ``{\tt x}'' or writing \keyindex{xcu}, 
\keyindex{xcl}, \keyindex{xccu}, \keyindex{xcup} instead of \verb|\cu|,
\verb|\cl|, \verb|\ccu|, \verb|\cup|, etc.

 
 \item The \raise 0.5ex\hbox{\musicnorfont\char38} symbol which is obtained
using the \verb|\qu|, \verb|\qh| etc. macros replacing the ``{\tt q}'' letter
in the macro name with the pair ``{\tt ox}'' or writing \keyindex{oxcu}, 
\keyindex{oxcl}, \keyindex{oxccu}, \keyindex{oxcup} instead of \verb|\cu|,
\verb|\cl|, \verb|\ccu|, \verb|\cup|, etc.

 \end{itemize}
 %\check
\section{Staff size} \index{staff size}
\subsection{Moving from 20pt to 16pt general staff sizes and conversely}
 You also want to write some parts of your score in 20pt
staff size and in 16pt staff size, namely
for distinct parts of pieces. Changing the \ixem{general staff size}
is done by saying:

\tty\keyindex{musicsize}=16\bslash computespecifics\quad \rm or

\Bslash musicsize=20\keyindex{computespecifics}

\rm\noindent respectively.

\subsection{Changing staff size for certain instruments}\label{staffspace}
Regardless of the general choice of 16pt or 20pt staff sizes, it is now
possible to assign certain instruments -- not sperate staffs belonging to a
same instrument -- to have narrower staff size. This is done by giving a
special definition of \keyindex{staffspacing$r$}  where $r$ is the roman
numeral of the instrument considered. For example, if the second instrument
(starting from the system bottom) is required to have 25\% narrower staffs,
one only hase to declare:

\medskip{\Bslash def\bslash staffspacingii\{0.75\}}

\medskip\noindent before the starting command {\Bslash debutmorceau}.

Not only does this statement change the vertical spacing between staff lines,
but is also changes the size of the key, accidental and note symbols to fit
the modified staff line spacing. However, the existence of discrete font
scales for music symbol typing results in some reasonable restrictions of the
values of \keyindex{staffspacing$r$}~: recommended values are approximately
1, 0.8 and 0.64; if different values are chosen, then the symbol size is
taken smaller than the line spacing, which may lead to readable but rather
ugly typesetting.

As an example, we give two bars of the \ital{Ave Maria} by Charles {\sc
Gounod}\index{Gounod, C.} using the first prelude of Johann-Sebastian Bach's
\ital{Well Tempered Clavier} (transcription for organ, violin and voice,
thanks to Markus {\sc Veittes}):\label{avemaria}

 \begin{music}
\def\nbinstruments{4}
\computewidths
\sepbarrules
\generalmeter{\meterC}\relax
\def\oct{\advance\transpose by 7\relax}
\def\staffspacingii{0.64}
\def\staffspacingiv{0.64}
\cleftoksi{{6}{0}{0}{0}}
\cleftoksii{{0}{0}{0}{0}}
\cleftoksiv{{0}{0}{0}{0}}
\nbporteesi=2\relax
\nbporteesiii=0\relax
\debutextrait
\normal\elemskip=0.02\hsize
%Takt 9
\Notes\zhl c\raise27pt\qs\qupp e|\ds&\oct
  \itenu5h\hl h&gra---&\itenl4h\hu h\enotes
\Notes|\ibbl0j3\qb0h\tqb0l\enotes
\Notes|\ibbl1k0\qb1o\qb1h\qb1l\tqb1o\enotes
\temps\Notes\zhl c\raise27pt\qs\qupp e|\ds&\oct
  \tten5\ibl4c0\qb4h&&\tten4\ibu5g{-3}\qh5h\enotes
\Notes|\ibbl0j3\qb0h\tqb0l&\oct\qb4a&---&\tqh5a\enotes
\Notes|\ibbl1k0\qb1o\qb1h&\oct\qb4b&ti-&\cu b\enotes
\Notes|\qb1l\tqb1o&\oct\tqb4c&a&\cu c\enotes
\barre
%Takt 10
\Notes\zhl c\raise27pt\qs\qupp d|\ds&\oct
  \qlp d&ple---&\Ilegu4k\pt d\qu d\enotes
\Notes|\ibbu1g3\bigaccid\qh1{^f}\tqh1h\enotes
\Notes|\ibbu2i0\qh2k\qh2f\enotes
\Notes|\qh2h\tqh2k&\oct \cl e&&\tleg4\cu e\enotes
\temps\Notes\zhl c\raise27pt\qs\qupp d|\ds&\oct
  \ql d&na,&\qu d\enotes
\Notes|\ibbu1g3\qh1f\tqh1h\enotes
\Notes|\ibbu2i0\qh2k\qh2f\qh2h\tqh2k&\qp&&\qp\enotes
\finextrait
 \end{music}

   This example was coded as:

\begin{quote}\begin{verbatim}
\def\nbinstruments{4}
\sepbarrules
\generalmeter{\meterC}\relax
\def\oct{\advance\transpose by 7\relax}
\def\staffspacingii{0.64}
\def\staffspacingiv{0.64}
\cleftoksi{{6}{0}{0}{0}}
\cleftoksii{{0}{0}{0}{0}}
\cleftoksiv{{0}{0}{0}{0}}
\nbporteesi=2\relax
\nbporteesiii=0\relax
%
\debutextrait
\normal\elemskip=0.02\hsize
%Takt 9
\Notes\zhl c\raise27pt\qs\qupp e|\ds&\oct
  \itenu5h\hl h&gra---&\itenl4h\hu h\enotes
\Notes|\ibbl0j3\qb0h\tqb0l\enotes
\Notes|\ibbl1k0\qb1o\qb1h\qb1l\tqb1o\enotes
\temps\Notes\zhl c\raise27pt\qs\qupp e|\ds&\oct
  \tten5\ibl4c0\qb4h&&\tten4\ibu5g{-3}\qh5h\enotes
\Notes|\ibbl0j3\qb0h\tqb0l&\oct\qb4a&---&\tqh5a\enotes
\Notes|\ibbl1k0\qb1o\qb1h&\oct\qb4b&ti-&\cu b\enotes
\Notes|\qb1l\tqb1o&\oct\tqb4c&a&\cu c\enotes
\barre
%Takt 10
\Notes\zhl c\raise27pt\qs\qupp d|\ds&\oct
  \qlp d&ple---&\Ilegu4k\pt d\qu d\enotes
\Notes|\ibbu1g3\bigaccid\qh1{^f}\tqh1h\enotes
\Notes|\ibbu2i0\qh2k\qh2f\enotes
\Notes|\qh2h\tqh2k&\oct \cl e&&\tleg4\cu e\enotes
\temps\Notes\zhl c\raise27pt\qs\qupp d|\ds&\oct
  \ql d&na,&\qu d\enotes
\Notes|\ibbu1g3\qh1f\tqh1h\enotes
\Notes|\ibbu2i0\qh2k\qh2f\qh2h\tqh2k&\qp&&\qp\enotes
\end{verbatim}\end{quote}



 %\check
 \section{Layout parameters}


 Most layout parameters are set by \musictex\ to reasonable default values.
However, sophisticated scores\footnote{To our knowledge, the most complicated
scores are those written for the piano, during the romantic and post-romantic
periods.} may need more place below the lowest staff, between staves, etc.
\def\nochange{(\ital{NOT to be changed})} We give below a short list of the
most significant parameters.

%\check

\subsection{List of layout parameters}

{\sl REMARK : the mention ``\nochange'' does not mean that this parameter cannot
be changed, but that it should not be modified directly, e.g.\ by saying
something like {\Bslash interligne=14pt}. In other words, changing these
parameters must be performed using more comprehensive macros which not only
update them but also perform some other compulsory related changes.}

\begin{description}

 \item[\keyindex{interligne} : ]vertical interval between lines in a staff of
the current instrument, taking account of a possible specification of \keyindex{staffspacing$r$}
(see \ref{staffspace})
\nochange
 \item[\keyindex{Interligne} : ]vertical distance between the base of staff
lines  of the current instrument, taking account of a possible specification
of \keyindex{staffspacing$r$} (\keyindex{interligne} is the size of the blank
space between lines; the difference between them is the line thickness
\keyindex{lthick}) \nochange.
 \item[\keyindex{internote} : ]the vertical spacing of contiguous notes  of
the current instrument, taking account of a possible specification of
\keyindex{staffspacing$r$}, i.e.\ the half of {\Bslash Interligne} \nochange
 \item[\keyindex{Internote} : ]the vertical spacing of contiguous notes of
the instrument(s) whose \keyindex{staffspacing$r$} has the \ital{default
value} of one (1.0). Since this dimension is the same, regardless of the
actual value of the staff line spacing of any instrument, only this value
should be used to specify horizontal spacings. It should be used rather than
absolute dimensions in \verb|pt| or \verb|mm|, since \verb|\Internote| is
duly proportional to the general size (\keyindex{musicsize}) but not
dependent on specific changes in  \keyindex{staffspacing$r$} \nochange
 \item[\keyindex{nullthick} : ]reserved height above base line for zero staff
lines (text of songs)
 \item[\keyindex{staffbotmarg} : ]margin below the first staff of the lowest
instrument. If not already assigned a non zero dimension, it is set to
\keyindex{bottomfacteur}\keyindex{Interligne} at the next system.
 \item[\keyindex{stafftopmarg} : ]margin above the upper staff of the upper
instrument.
 \item[\keyindex{interbeam} : ]vertical distance between beams.
 \item[\keyindex{interportee} : ]the distance between the bottom of one staff and
the bottom of the next one. It is set to
2\keyindex{interfacteur}{\Bslash internote} at the next system.
 \item[\keyindex{Interportee} : ]the distance between the top of one staff
and the bottom of the next one. Re-computed at the at the next system.
 \item[\keyindex{interinstrument} : ]the additional vertical distance between
two different instruments (must be positive, otherwise weird results can
happen). This means that the distance between the upper
staff of the previous instrument and the lowest line of the current
instrument is equal to \keyindex{interportee+\bslash interinstrument}.
This value is normally zero, but it helps putting additional space between
distinct instruments for the sake of clarity. This is a general dimension
which holds for each of the vertical spaces between instruments, except the
upper one, in which case this interval is irrelevant. However, this parameter
can be overridden for the space above a specific instrument. For example (see
the example {\tty angescao.tex}) one can state:

\medskip
\begin{quote}\begin{verbatim}
\def\interinstrumenti{5pt}
\end{verbatim}\end{quote}

\medskip
\noindent to force an additional spacing of 5 points above instrument $i$,
whatever the value of \keyindex{interinstrument}.
 This feature can usefully be used to have more space before instruments
representing \itxem{voices}, in order to have enough place to put
\itxem{lyrics} without assigning these lyrics a zero staff specific instrument
(useful to avoid having too many declared insruments in a choir score).
 \item[\keyindex{systemheight} : ]the distance from the bottom of the
lowest staff to the top of the highest staff of the upper instrument. This is
the height of the vertical bars (single, double, repeats, etc.) \nochange.
\end{description}


\medskip
 In addition, when handling notes of a given staff of a given instrument, the
following dimensions are available (note these are not true registers, but
\ital{equivalenced symbols} through a {\Bslash def}):
\begin{itemize}
 \item\keyindex{altitude} : the altitude of the lowest line of the lowest
staff of the current instrument \nochange.
 \item\keyindex{altportee} : the altitude of the lowest line of the current
staff \nochange.
 \item\keyindex{stemfactor} : a parameter defining the size of half, quarter
and hooked eighth notes stems. Normally a stem has the length of one octave,
i.e. 3.5\keyindex{Interligne}. However, this is not valid for small size
notes and, therefore, the stem size is related to the \keyindex{interbeam}
dimensions which, in turn, is \ital{normally} equal to 0.75
\keyindex{Interligne}. Thus the normal value of \keyindex{stemfactor} is
4.66, but it can be shortened for any purpose by saying, for example:

  \verb|\def\stemfactor{3.5}|

Normal stem length is restored by calling the macro \keyindex{normalstems}.

\end{itemize}

%\check

\subsection{Changing layout parameters}

 Most of these values can be changed, but only between the end of the
previous system and the beginning of the next one. This can be inserted
between a \keyindex{suspmorceau} (or a \keyindex{finmorceau})
and a \keyindex{reprmorceau} (or a \keyindex{debutmorceau}), but it is
wiser to say, for example: 

\medskip
\begin{quote}\begin{verbatim}
\def\atnextline{\global\staffbotmarg=5\Interligne}
\end{verbatim}\end{quote}

\medskip

When doing so, the \musictex\ user should be aware that this could disturb
pending \ital{slurs} or \ital{ties}, since the altitude of these is stored in
an absolute way, starting from the baseline of the systems. Therefore,
changing the \keyindex{staffbotmarg} dimension can also be made by means of
{\tty\keyindex{advancebottom}\{$<dimension>$\}} which updates all pending slur
and tie altitudes by the given dimension. This has been used in {\tty
pacifiqn} and {\tty pacifiqb}.

\medskip The user may prefer to redefine {\Bslash
def\keyindex{bottomfacteur}} or {\Bslash def\keyindex{interfacteur}} to a
given integer number, but this can be done only between  a
\keyindex{suspmorceau} (or a \keyindex{finmorceau}) and a
\keyindex{reprmorceau} (or a \keyindex{debutmorceau}), but it is wiser to use
\keyindex{atnextline} as previously.

\medskip It is also wise to use \keyindex{atnextline} to change the the
number of instruments, the staff spacings, the number of staffs at the next
line... provided that the coding of the notes \ital{resists} an unexpected
line change executing the \keyindex{atnextline}.

\medskip In anycase it is a good idea to look at the procedures
\keyindex{computewidths} and \keyindex{computespecifics} to understant what
they really compute, and also to find the places where they are actually
invoked.

\subsection{Changing the vertical distance between consecutive systems}

If this has to be changed only once at the beginning of a piece, the simplest
is to say

\medskip
\begin{quote}\begin{verbatim}
\def\bottomfacteur{8}
\end{verbatim}\end{quote}

\medskip\noindent before invoking\keyindex{debutmorceau}. Note that
\keyindex{bottomfacteur} is invoked to compute the dimension register
\keyindex{staffbotmarg} whenever this register contains a zero dimension.
The number (it must be a number) is multiplied by \keyindex{Interligne} and
affected to \keyindex{staffbotmarg}.

\medskip If this has to be changed several times to meet page layout
requirements with scores sometimes -- but only sometimes -- going very deep
below the lowes staff of the systems, it is wiser to use
\keyindex{advancebottom}, namely:

\medskip
\begin{quote}\begin{verbatim}
\advancebottom{10pt}
\end{verbatim}\end{quote}

\medskip\noindent to advance the bottom margin register by that dimension at
the next system line break.

{\medskip\noindent\sl REMARK : it is also possible to change the system top
margin, i.e.\ \keyindex{stafftopmarg} between systems. Changing top and
bottom margin add together but it may influence the position of the whole in
the pages.}

\subsection{Changing staff distance within systems}

This can be done at the beginning by changing the definition of
\keyindex{interfacteur} (it is {\Bslash def}ined to a number) before the
first \keyindex{debutmorceau}. But it is also possible to give a new
dimension to the dimension register \keyindex{interportee} and, optionally,
to the dimension register \keyindex{interinstrument} (see above).

\subsection{Changing the number of lines in staffs}

Unless explicitely specified, staffs consist of five lines, in accordance to
the normal way of coding music scores. However, two exceptions might be
preferred when using \musictex~:

\begin{itemize}
 \item \itxem{gregorian music} is often writen using staff of
four lines instead of five,
 \item \itxem{percussion music} (e.g. drums, triangle) only needs one line
staffs, since the pitch cannot change.
\end{itemize}

Therefore, since its version 5.00, \musictex\ allows for choosing the number
of lines of the staffs of an instrument. This is done by defining the macro
\keyindex{stafflinesnb}$r$ --- where $r$ is the roman numeral of the wanted
instrument as usual --- to be the number of lines of the specific staff.
For example~:\label{gregorian}

\begin{quote}
\noindent\verb|\def\stafflinesii{4}|
\end{quote}
will make the instrument number 2 (i.e. {\tt ii}) to have staffs of four
lines, that is, fit for gregorian music.
%\check

\subsection{Resetting normal layout parameters}

Except \keyindex{musicsize} which has to be explicitely changed if needed, all
layour registers are reset to default values by \keyindex{resetfacteurs} which
put zero dimensions into \keyindex{staffbotmarg} and \keyindex{stafftopmarg},
so that the next \keyindex{debutmorceau} will recompute them, unless they have
been given non zero dimensions inbetween.

\subsection{Typesetting one-line excerpts rather than larges scores}

Very often, what is wanted is not to typeset a large comprehensive score of
several lines and pages, but an excerpt of one or two bars, preferably
centered such as the various examples of this manual. This can be done simply
by replacing \keyindex{debutmorceau} with
\keyindex{debutextrait}\footnote{\ital{Extrait} is the French word for
\ital{excerpt}.} and {\Bslash finmorceau} or {\Bslash suspmorceau} with
\keyindex{finextrait}.

 If you want to terminate it without a bar, you code
\keyindex{zfinextrait} which acts like {\Bslash zsuspmorceau}..

%\check

 \subsection{Lyrics}\index{lyrics}

 Lyrics can be introduced in several ways~:
 \begin{enumerate}

 \item The first one consists in dedicating
one instrument whose \keyindex{nbportees$r$} is zero. Then the text of the
lyrics is just inserted, note by note, by inserting it at the right position
between {\Bslash notes}...{\Bslash enotes}. The drawback is that a four voice
choir needs eight instruments which may lead to troubles if there is also an
orchestra score below the voices.

 \item Therefore, another solution consists in adjusting \keyindex{interinstrument}
and \keyindex{staffbotmarg} to give more place below the song
instruments\footnote{Beware of a shift of one position, since {\Bslash
intersinstrument$r$} indicates the interval \ital{above} the $r$-th
instrument.}. Then the texts of the lyrics are indicated with the song tune
instrument, using \keyindex{zcharnote} with a negative numeric position
value.

 By the way, the {\Bslash zcharnote} may have a {\Bslash vbox} as a second
argument, and this \keyindex{vbox} may contain several {\Bslash hbox}es
describing the text of the different couplets (see example {\tty ANGESCAO}).

  \item Of easier use are the commands \keyindex{zsong} (right of the note),
\keyindex{lsong} (left) and \keyindex{csong} (centered) which post the lyrics
at the lower staff line \ital{minus} the previous
\keyindex{interinstrument}~$n$ or the \keyindex{staffbotmarg} quantity. These
commands only have one argument, namely the lyrics text:


\begin{center}
\keyindex{zsong}\verb|{|\ital{text}\verb|}|\quad
\keyindex{lsong}\verb|{|\ital{text}\verb|}|\quad
\keyindex{csong}\verb|{|\ital{text}\verb|}|
\end{center}

 \medskip As an example, the following French song

 \begin{music}
 \generalsignature{1}
 \def\nbinstruments{1}
 \debutextrait
 \NOtes\zsong{Au }\qu g\enotes
 \NOtes\zsong{clair }\qu g\enotes
 \NOtes\zsong{de }\qu g\enotes
 \NOtes\zsong{la }\qu h\enotes
 \barre
 \NOTes\zsong{lu- }\hu i\enotes
 \NOTes\zsong{ne, }\hu h\enotes
 \barre
 \NOtes\zsong{mon }\qu g\enotes
 \NOtes\zsong{a- }\qu i\enotes
 \NOtes\zsong{mi }\qu h\enotes
 \NOtes\zsong{Pier- }\qu h\enotes
 \barre
 \NOTes\zsong{rot, }\wh g\sk\enotes
 \finextrait
 \end{music}

\noindent was coded as:

\begin{quote}\begin{verbatim}
 \generalsignature{1}
 \def\nbinstruments{1}
 \debutextrait
 \NOtes\zsong{Au }\qu g\enotes
 \NOtes\zsong{clair }\qu g\enotes
 \NOtes\zsong{de }\qu g\enotes
 \NOtes\zsong{la }\qu h\enotes
 \barre
 \NOTes\zsong{lu- }\hu i\enotes
 \NOTes\zsong{ne, }\hu h\enotes
 \barre
 \NOtes\zsong{mon }\qu g\enotes
 \NOtes\zsong{a- }\qu i\enotes
 \NOtes\zsong{mi }\qu h\enotes
 \NOtes\zsong{Pier- }\qu h\enotes
 \barre
 \NOTes\zsong{rot, }\wh g\sk\enotes
 \finextrait
\end{verbatim}\end{quote}

 \end{enumerate}


%\check

 \section{Other special ornaments}

 Since the first release of \musictex, many users have either asked for
new features of proposed their own procedures to be included in \musictex.
Some of them have been added because of their high utility, some others have
been left aside for several reasons. The major reason for not including many
proposals in the \musictex\ release is that -- at least on many computers and
especially PC's -- the \TeX\ memory is limited to 65~000 ``words''. This is a
very restricted value which causes many trials to exceed the capacity of \TeX\
(unless using ``big\TeX''). Therefore, only the proposed macros of very
general use can be introduced in the release as a standard and the users are
suggested to take their specific procedures and include them (directly of by
means of an {\Bslash input}) in their own source code.

  This is particularly valid for people who want to typeset \itxem{baroque}
music using the ancient \itxem{ornament codings} rather than the modern
equivalents. As a compromise, some macros kindly provided by Ian {\sc Collier}
in Great Britain have slightly been updated and provided in a separate file
named {\tty musicext.tex}.

  For example, brackets can be produced:

  \begin{music}
\def\nbinstruments{1}\relax
\computewidths
\nbporteesi=1\relax
\debutextrait
\normal
\temps\Notes\ovbkt j3{15}\ql{hij}\enotes
\finextrait

  \end{music}

%\check
 \section{The {\tt musicsty} macros}\label{musicsty}
% non musical procedures used in typesetting the non-musical
% texts together with musictex
 This file is made for non \TeX perts and/or lazy score typesetters. It
provides
 \begin{itemize}
 \item a set of font definitions of common use, such as \verb|\tenrm|,
\verb|\eightrm|, etc.,
 \item a reasonable setting of \keyindex{hsize}, \keyindex{vsize},
\keyindex{hoffset}, \keyindex{voffset} dimensions in order to have a good
layout fi for European A4 paper\footnote{People addicted to \itxem{legal} or
other paper sizes should correct it for their own purpose.}
 \item a set of text size commands:

 \begin{description}
  \item[\keyindex{eightpoint}] which sets the usual \keyindex{rm},
\keyindex{bf}, \keyindex{sl}, \keyindex{it} commands to 8 point font size;
  \item[\keyindex{tenpoint}] which sets the usual \keyindex{rm},
\keyindex{bf}, \keyindex{sl}, \keyindex{it} commands to 10 point font size;
  \item[\keyindex{twlpoint}] to get 12 point font size;
  \item[\keyindex{frtpoint}] to get 14.4 point font size;
  \item[\keyindex{svtpoint}] to get 17.28 point font size;
  \item[\keyindex{twtypoint}] to get 20.74 point font size;
  \item[\keyindex{twfvpoint}] to get 24.88 point font size;
 \end{description}
 \item a set of commands to make easy piece titles~:
  \begin{itemize} \item \keyindex{author} or
\keyindex{fullauthor} to be put at the right of the first page, below the
title of the piece; the calling sequence is, for example:

  \verb|     \author{Daniel TAUPIN\\organiste \`a Gif-sur-Yvette}|

  \noindent where the \verb|\\| makes theauthor name displayed on two or
several lines.

 
  \item \keyindex{shortauthor} to be put at the bottom of each page,
  \item \keyindex{fulltitle} which is the big main title of the piece,
  \item \keyindex{subtitle} is displayed below the main title of the piece,
  \item \keyindex{shorttitle} or \keyindex{title}
  which is the title repeated at the bottom of each page,
  \item \keyindex{othermention} which is displayed on the left of the page, in
front of the author's name (it may contain several \verb|\\| to display it on
several lines,
  \item \keyindex{maketitle}  which displays all the previous stuff.
  \end{itemize}

 \item  Some additional commands to make \itxem{footnotes}.
 These commands are
  \begin{itemize}
   \item The normal Plain-\TeX\ \keyindex{footnote} command which has two
arguments --- not only one as in \LaTeX\index{LaTeX=\protect\LaTeX} --- namely
the label of the footnote, i.e. any sequence of characters and not only
figures, and the text of the footnote.

\noindent{\sl IMPORTANT: the \verb|\footnote| command does not work inside
boxes\footnote{This is not a \TeX-bug, this is a feature!}, therefore this
command must not be issued within music. But another alternate feature is
provided (see below).}

 \item The \keyindex{Footnote} command, which counts the footnotes and uses a
number as the label of the foot note (equivalent to \LaTeX's \verb|\footnote|
command). The same restriction applies concerning footnotes within the music
coding.

 \item The \keyindex{vfootnote} command, taken from the Plain-\TeX, which
makes the footnote itself at the bottom of the current page, but does not put
the footnote label at the place it is referred in the main text.

 Thus, if a footnote is needed whose reference lies inside the music itself,
the music typesetter must perform it in two steps~:
 \begin{enumerate}
  \item quote the reference inside the music, using \verb|zcharnote| for
example,
  \item post the footnote itself, using \verb|\vfootnote| outside the music,
either before \keyindex{debutmorceau} or between \keyindex{suspmorceau} and
\keyindex{reprmorceau} or equivalent commands.
 \end{enumerate}

  \end{itemize}
 \end{itemize}

 Note that \verb|musicsty| should not be used with \LaTeX.


\section{Abnormal music coding}
\subsection{Gregorian chant}
\index{gregorian music}Gregorian chant is often coded using four line staffs
(see section \ref{gregorian}) and using special notes (called \itxem{neumes})
which are described in section \ref{gregnotes}. But the gregorian chant also
needs a special \itxem{alto clef} which is in fact the ancester of the modern
alto clef. The gregorian \itxem{alto clef} can be invoked instead of the
modern one by re-declaring the \keyindex{clefdut}$r$ macro. Thus

\begin{quote}\begin{verbatim}
\def\clefdutiv{\gclefdut}
\end{verbatim}\end{quote}
will cause the instrument number 4 (i.e. {\tt iv}) to exhibit a gregorian C
clef whenever the value of \verb|\cleftoskiv| refers to an alto clef.
The modern alto clef can be restored for that instrument by~:


\begin{quote}\begin{verbatim}
\def\clefdutiv{\clefdutsymbol}
\end{verbatim}\end{quote}

\subsection{Music score without clefs or with special clefs}\index{clefs
(empty)} Regardless of the number of lines of the staffs, an instrument may
have no clefs, e.g. for \itxem{percussion music} but also for any weird
purpose. This done by declaring some of the following items~:

\medskip
\verb|\def|\keyindex{clefdesol}$r$\verb|{\relax}|

\verb|\def|\keyindex{clefdut}$r$\verb|{\relax}|

\verb|\def|\keyindex{clefdefa}$r$\verb|{\relax}|

\medskip
 At first sight, it could be thought to be silly to cancel the G clef, the C
clef and/or the F clef to have an empty clef symbol. But the reason is that
the absence of clefs does not mean that notes should not be raised according
to their pitch. Thus, if the user uses the G clef coding of pitches, he is
welcome to use the present feature to cancel the G clef, but if he uses some
alto clef coding, then he should cancel the alto clef symbol. Of course, all
this is irrelevant if the engraver chooses to use numeric coding of note
position like \verb|\qu{10}|.

 Normal symbols for these clefs and for instrument or roman number $r$ can be
restored by~:


\medskip
\verb|\def\clefdesol|$r$\verb|{|\keyindex{clefdesolsymbol}\verb|}|

\verb|\def\clefdefa|$r$\verb|{|\keyindex{clefdefasymbol}\verb|}|

\verb|\def\clefdut|$r$\verb|{|\keyindex{clefdutsymbol}\verb|}|

\medskip Besides, a special \itxem{drum clef} (two heavy vertical bars) can
replace any of the standard clefs, for exemple the G clef by saying~:

\medskip
\verb|\def\clefdesol|$r$\verb|{|\keyindex{drumclefsymbol}\verb|}|


\medskip It is to be emphasised that these features are specific to one
instrument --- not one staff of a several staff instrument --- so that some
weird score for \ital{monks}, \ital{drum} and \ital{electronic keyboard}
such as

\begin{music}
\parindent 19mm
\def\nbinstruments{3}   \def\instrumenti{keyboard}
\def\instrumentii{drum} \def\instrumentiii{monks}
\def\stafflinesnbii{1}  \def\stafflinesnbiii{4}
\generalsignature{0}    \generalmeter{\relax}
\signi=-1\relax % one flat at keyboard
\nbporteesi=2\relax % 2 staffs at keyboard
\cleftoksi={6000}\cleftoksiii={3000}\cleftoksii={1000}
\def\clefdutiii{\gclefdut} % gregorian C clef at instrument iii
\def\clefdutii{\drumclefsymbol}   % cancel C clef at instrument ii
\def\interinstrumentii{-4\Interligne} % less spacing above drum
\normal
\debutextrait
\Notes\hu F|\zh c\hu h&\diamw 0&\carrg {acd}\enotes
\NOtes\qu I|\zq N\qu d&\diamg 0&\diamg f\enotes
\NOtes\qu J|\zq a\qu e&\diamg 0&\diamg e\enotes
\notes\hu G|\zh b\hu d&\zshaker 3\diamw {00}&\zcarrg d\carqg g\carrg {hgh}\enotes
\finextrait
\end{music}
\noindent could be coded as follows, regardless this is relevant~:
\begin{verbatim}
\parindent 19mm
\def\nbinstruments{3}   \def\instrumenti{keyboard}
\def\instrumentii{drum} \def\instrumentiii{monks}
\def\stafflinesnbii{1}  \def\stafflinesnbiii{4}
\generalsignature{0}    \generalmeter{\relax}
\signi=-1\relax            % one flat at keyboard
\nbporteesi=2\relax % 2 staffs at keyboard
\cleftoksi={6000}\cleftoksiii={3000}\cleftoksii={1000}
\def\clefdutiii{\gclefdut} % gregorian C clef at instrument iii
\def\clefdutii{\drumclefsymbol}   % cancel C clef at instrument ii
\def\interinstrumentii{-4\Interligne} % less spacing above drum
\normal
\debutextrait
\Notes\hu F|\zh c\hu h&\diamw 0&\carrg {acd}\enotes
\NOtes\qu I|\zq N\qu d&\diamg 0&\diamg f\enotes
\NOtes\qu J|\zq a\qu e&\diamg 0&\diamg e\enotes
\notes\hu G|\zh b\hu d&\zshaker 3\diamw {00}&\zcarrg d\carqg g\carrg {hgh}\enotes
\finextrait
\end{verbatim}

In the same way, a possible violin score with \itxem{harmonic notes} (see
\ref{othernotes}) could be~:\label{abnormalscores}

\begin{music}
\def\freqbarno{9999}% no bar numbers
\def\nbinstruments{1}\nbporteesi=1\relax
\generalsignature{-2}\generalmeter{\allabreve}\cleftoksi={0000}
\normal
\debutextrait
\NOtes\zd o\zh d\hu h\enotes
\Notes\ibu0k0\zq g\yh0k\qh0j\zq e\yh0i\tbu0\qh0j\enotes
\barre
\NOTes\zd g\hu k\enotes
\NOTes\hpause\enotes
\barre
\NOtes\zd o\zh d\hl h\enotes
\Notes\ibl0b0\zq g\yb0k\qb0j\zq e\yb0i\tbl0\qb0j\enotes
\barre
\NOTes\zd g\hu k\enotes
\NOTes\hpause\enotes
\finextrait
\end{music}

It was coded as follows:
\begin{quote}\begin{verbatim}
\def\nbinstruments{1}\nbporteesi=1\relax
\generalsignature{-2}\generalmeter{\allabreve}\cleftoksi={0000}
\normal
\debutextrait
\NOtes\zd o\zh d\hu h\enotes
\Notes\ibu0k0\zq g\yh0k\qh0j\zq e\yh0i\tbu0\qh0j\enotes
\barre
\NOTes\zd g\hu k\enotes
\NOTes\hpause\enotes
\barre
\NOtes\zd o\zh d\hl h\enotes
\Notes\ibl0b0\zq g\yb0k\qb0j\zq e\yb0i\tbl0\qb0j\enotes
\barre
\NOTes\zd g\hu k\enotes
\NOTes\hpause\enotes
\finextrait
\end{verbatim}\end{quote}


\subsection{Usual percussion music}
\label{genpercus} Besides single percussion scores usually
written using one-line staffs, percussion music involving several instruments
is often writen on five-line staffs with a \ital{drum clef}, where the
instruments are distinguished by the type of the note heads and the apparent
pitch of the note on the staff. We give an example --- kindly provided by
Agusti {\sc Mart\'in Domingo}~:

\medskip
\begin{music}
\def\freqbarno{9999}% no bar numbers
\def\nbinstruments{1}\generalsignature{0}\def\stafflinesbi{5}
\generalmeter{\meterfrac44}
\cleftoksi={0000}\def\clefdesoli{\drumclefsymbol}
\normal
\debutextrait
\leftrepeat
\Notes\zql f\rlap{\soupir}\ibu0m0\xh0{nn}\enotes
\Notes\zk d\zql f\zq j\xh0n\tbu0\xh0n\enotes
\Notes\zql f\rlap{\soupir}\ibu0m0\xh0{nn}\enotes
\Notes\zk d\zql f\zq j\xh0n\tbu0\xh0n\enotes
\barre
\Notes\zql f\rlap{\soupir}\ibu0m0\kh0{nn}\enotes
\Notes\zx d\zql f\zq j\kh0n\tbu0\kh0n\enotes
\Notes\zql f\rlap{\soupir}\ibu0m0\kh0{nn}\enotes
\Notes\zx d\zql f\zq j\kh0n\tbu0\kh0n\enotes
\barre
\Notes\zql f\rlap{\soupir}\ibu0m0\oxh0{nn}\enotes
\Notes\zox d\zql f\zq j\kh0n\tbu0\oxh0n\enotes
\Notes\zql f\rlap{\soupir}\ibu0m0\oxh0{nn}\enotes
\Notes\zox d\zql f\zq j\kh0n\tbu0\oxh0n\enotes
\setrightrepeat\finextrait
\end{music}
 Its coding was~:
 \begin{quote}\begin{verbatim}
\begin{music}
\def\nbinstruments{1}\generalsignature{0}\def\stafflinesbi{5}
\generalmeter{\meterfrac44}
\cleftoksi={0000}\def\clefdesoli{\drumclefsymbol}
\normal
\debutextrait
\leftrepeat
\Notes\zql f\rlap{\soupir}\ibu0m0\xh0{nn}\enotes
\Notes\zk d\zql f\zq j\xh0n\tbu0\xh0n\enotes
\Notes\zql f\rlap{\soupir}\ibu0m0\xh0{nn}\enotes
\Notes\zk d\zql f\zq j\xh0n\tbu0\xh0n\enotes
\barre
\Notes\zql f\rlap{\soupir}\ibu0m0\kh0{nn}\enotes
\Notes\zx d\zql f\zq j\kh0n\tbu0\kh0n\enotes
\Notes\zql f\rlap{\soupir}\ibu0m0\kh0{nn}\enotes
\Notes\zx d\zql f\zq j\kh0n\tbu0\kh0n\enotes
\barre
\Notes\zql f\rlap{\soupir}\ibu0m0\oxh0{nn}\enotes
\Notes\zox d\zql f\zq j\kh0n\tbu0\oxh0n\enotes
\Notes\zql f\rlap{\soupir}\ibu0m0\oxh0{nn}\enotes
\Notes\zox d\zql f\zq j\kh0n\tbu0\oxh0n\enotes
\setrightrepeat\finextrait
\end{music}
 \end{verbatim}\end{quote}

 To use these different note heads, one must
 \begin{itemize}
 \item either include a specific percussion file namely {\tt
\ixem{musicper.tex}} after the usual \verb|\input musictex| or invoke the
\LaTeX\ style option {\tt\ixem{musicper}}\index{musicper.sty};
 \item use special macro names to replace the usual elliptic back note head
with either a double sharp sign or with a $+$. These macros are~:
 \begin{itemize}
  \item \keyindex{zx}, \keyindex{xu}, \keyindex{xup}, \keyindex{xupp},
 \keyindex{xl}, \keyindex{xlp}, \keyindex{xlpp},
 \keyindex{xh}, \keyindex{xb}, \keyindex{zx},
 \keyindex{xcu}, \keyindex{xcup}, \keyindex{xcupp},
 \keyindex{xccu}, \keyindex{xcccu}, \keyindex{xccccu},
 \keyindex{xcl}, \keyindex{xclp}, \keyindex{xclpp},
 \keyindex{xccl}, \keyindex{xcccl} and \keyindex{xccccl}, which behave exactly like
 \verb|\zq|, \verb|\qu|, \verb|\qup|, \verb|\qupp|,
 \verb|\ql|, \verb|\qlp|, \verb|\qlpp|,
 \verb|\qh|, \verb|\qb|, \verb|\zq|,
 \verb|\cu|, \verb|\cup|, \verb|\cupp|,
 \verb|\ccu|, \verb|\cccu|, \verb|\ccccu|,
 \verb|\cl|, \verb|\clp|, \verb|\clpp|,
 \verb|\ccl|, \verb|\cccl| and  \verb|\ccccl|, except that the note head is
 \raise 0.6ex\hbox{\musicnorfont\char '65} instead of \raise
0.5ex\hbox{\musicnorfont\char '41}~.
  \item \keyindex{zox}, \keyindex{oxu}, \keyindex{oxup}, \keyindex{oxupp},
 \keyindex{oxl}, \keyindex{oxlp}, \keyindex{oxlpp},
 \keyindex{oxh}, \keyindex{oxb}, \keyindex{zx},
 \keyindex{oxcu}, \keyindex{oxcup}, \keyindex{oxcupp},
 \keyindex{oxccu}, \keyindex{oxcccu}, \keyindex{oxccccu},
 \keyindex{oxcl}, \keyindex{oxclp}, \keyindex{oxclpp},
 \keyindex{oxccl}, \keyindex{oxcccl} and \keyindex{oxccccl}, to get a note
head of
\raise 0.6ex\hbox{\musicnorfont\char 38} instead of \raise
0.5ex\hbox{\musicnorfont\char '41}~.
  \item \keyindex{zk}, \keyindex{ku}, \keyindex{kup}, \keyindex{kupp},
 \keyindex{kl}, \keyindex{klp}, \keyindex{klpp},
 \keyindex{kh}, \keyindex{kb}, \keyindex{zk},
 \keyindex{kcu}, \keyindex{kcup}, \keyindex{kcupp},
 \keyindex{kccu}, \keyindex{kcccu}, \keyindex{kccccu},
 \keyindex{kcl}, \keyindex{kclp}, \keyindex{kclpp},
 \keyindex{kccl}, \keyindex{kcccl} and \keyindex{kccccl}, to get a
note head of
 \raise 0.6ex\hbox{\musicnorfont\char 39} instead of \raise
0.6ex\hbox{\musicnorfont\char '41}~.
 \end{itemize}
 \end{itemize}


 \section{Writing your own macros: the {\Bslash catcode} problems}

As seen before, the \keyindex{catcodes} of the {\tty\|} and {\tty\&} symbols
are modified by \musictex, in the range of the actual scores but no
more\footnote{Since version 4.99.} in
the whole of the \TeX\ source. Thus, if you define your own macros to make
your writing easier, you are likely to invoke  the {\tty\|} or {\tty\&}
symbols in a part of text where their  \keyindex{catcode}s are not correctly set.
This may result typically in a diagnostic like~:

\verb|! Misplaced alignment tab character &.|

\noindent when you attempt, not to define, but to use your macro using the
{\tty\&} symbol to change the instrument. Smart \TeX ers know that the
\keyindex{catcode}s are attached to the characters \ital{when they are input}
and not when they are used; thus you must be sure that {\tty\|} and {\tty\&}
have the correct \musictex\ \keyindex{catcode} when the macro is defined,
which may well occur outside the actual score.

It is also worth pointing out that the same problem may occur with other
punctuation marks like ``{\tty<\relax}'', ``{\tty>\relax}'',  ``\verb|^|,
etc., if their \keyindex{catcode} has been changed by some other set of
macros, like {\tty \ixem{french.sty}}.


 \section{Musicla\TeX}

\subsection{The {\tt musictex.sty} style}
 
 As said before, the amount of memory and registers used by \musictex\ makes
it hardly compatible with \LaTeX. However, Nicolas {\sc
Brouard}\index{Brouard, N.} succeeded in building a {\tty\ixem{musictex.sty}}
which is now included in the distribution. This is not recommended to make
separate music scores. Its purpose is rather to provide a means of inserting
short musical excerpts in books or articles written with \LaTeX. Then, the
\keyindex{documentstyle} command should include {\tty musictex} in the
options.


\medskip
 The \LaTeX\ style file {\tty \ixem{musictex.sty}} simply \verb|\input|s
the following files (in that very order): \begin{itemize}
 \item {\tty musicpre.tex}
 \item {\tty musicnft.tex}
 \item {\tty musictex.tex}
 \item {\tty musicvbm.tex}
 \item {\tty musicpos.tex}
 \end{itemize}



\medskip
 In the case of a \LaTeX\index{LaTeX=\protect\LaTeX}
user wanting to use accidental
transposition facilities, he should invoke {\tty\ixem{musictrp}} in the
options of the \keyindex{documentstyle} command.

 In case of {\tty TeX capacity exceeded...}, use a ``Bigla\TeX'' (after
checking there is no visible error in the source
code).\index{BigLaTeX=Big\LaTeX}

\subsection{Wide music in \LaTeX}

  Another difficulty appears with \LaTeX: internal \LaTeX\ macros handle the
page size in a way which is not supposed to be changed within a given document.
This means that text horizontal and vertical sizes are somewhat frozen so that
one can hardly insert pieces of music of page size different from the size
specified by the LaTeX  \itxem{style}. Although a \itxem{largemusic} has been
provided, the main drawback is an unpredictable behaviour of top and bottom
printouts, especially page numberings.

 If the whole of a document has wide pages, it can be haddled with the {\tty
a4wide}\index{a4wide} style option, or any derivate of it.

%\check

 \subsection{The {\Bslash catcode} problems}
 The {\tty musicpos.tex} file
merely overrides the \keyindex{catcode}s of the {\tty\|} and {\tty\&} symbols
which are modified by \musictex. To have access to these symbols when coding
music, on should then enclose the scores or excerpts within {\Bslash
begin\{music\}} and {\Bslash end\{music\}}. But there is also another
possibility, i.e.\ to say \keyindex{nextinstrument} instead of {\tty
\&}\index{\&} and \keyindex{nextstaff} instead of {\tty\|}.\index{\|}

 Another problem comes from the {\tty \ixem{french.sty}} written by Bernard
{\sc Gaulle}\index{Gaulle, B.} which is the standard of the \ixem{GUTenberg}
French association. This style changes many \keyindex{catcode}s which lead
\musictex\ to fail in many cases. Therefore, since the version 4.99, the
\keyindex{catcode}s of all are forced to the adequate value at
\keyindex{debutmorceau}, \keyindex{debutextrait} and restored at their
original value at \keyindex{finmorceau}, \keyindex{suspmorceau} and
\keyindex{finextrait}. This means that some facilities like the
\itxem{guillemets} or the \itxem{tabulation} character are inhibited within
music scores (possible problem with sophisticated \itxem{lyrics}) but
perfectly available within the normal text.

 Anyway, in case of emergency, one can invoke \keyindex{catcodesmusic} to
establish the \keyindex{catcode}s at their value fit for music, and
\keyindex{endcatcodesmusic} to reset them at their external value, for example
those chosen by {\tty french.sty}.

 \section{Implementation and restrictions}

 The macroinstruction file \musictex\ contains approximately 2500 lines of
code, that is 80~000 bytes approximately. This requires your score to be
compiled by the most extended versions of \TeX\ (65~000 words of working
memory). It is therefore wise to set \hbox{\Bslash tracingsstats} to 2 in
order to have an information about the memory used in each page. In desperate
situations, we recommend using the ``Big\TeX'' processors which,
unfortunately, perform a great deal of disk input/outputs (on PCs with i286
processors) which make them awfully slow\footnote{Using i386 or i486
processors, this problem disappears with the specific version of {\tty
emTeX}}.\index{Big\protect\TeX}\index{em\TeX}

In particular, the number of registers it uses and the amount of memory used
by \LaTeX\ macros makes it doubtfully compatible with \LaTeX, unless using
Big\LaTeX.\index{BigLaTeX=Big\LaTeX}

Other precautions are necessary: beware of end-of-line spaces; they corrupt
layout and may cause unwanted line breakings after which music symbols seem
to \ital{float} in the air without staffs. To avoid that, it is recommended
to use \keyindex{relax} rather than {\tty\%} at the end of source lines.

\chapter{Installation}

\section{Getting the stuff}

As seen before, all the files are available at \itxem{anonymous ftp} {\tty
rsovax.lps.u-psud.fr} (193.55.39.100) in the directory\footnote{{\sc
unix} addicts whould beware that this primary server is an old MicroVax run
the VMS system. Therefore the brackets in the directory name are compulsory,
not optional.}
{\tty[anonymous.musictex]}, which can more easily be reached
using the simple command

{\tt cd musictex}


This directory normally contains {\tty musictex.zip} which contains
all the distribution for PC (\ixem{MS-DOS}) computers. This is only for
{\tty ftp}-ing convenience since all source files are directly available
in the same directory. In addition a set of examples (i.e. the files
whose name does not begin with ``{\tty music}'') is packed into
{\tty musicexa.zip}, and {\tty recueil.zip} contains the zipped DVIs of a large
subset of the examples. Finally the PK files of specific fonts are provided
in {\tty musicpk.zip}; getting this file is useless if you are able to {\tty
metafont} the files whose {\tty *.mf} are provided in the main package.

The \ixem{VMS} files are also packed into {\tty musictex.bck}\index{musictex.zip}
\index{musictex.bck}. Notwithstanding the fact that files are packed together
or not, the files provided are of two kinds:
\begin{enumerate}
\item the basic files;
\item the example files.
\end{enumerate}

All \itxem{basic files} are either of the form {\tty music*.*} (excluding of
course {\tty *.zip} and {\tty *.bck}), {\tty beamn*.*} and {\tty slurn*.*}. Other
files ({\tty *.tex} or {\tty *.dvi}) are example files.

\ital{Fonts}\index{fonts} are provided as {\tty *.mf} files but also as {\tty
*.tfm} and {\tty *.pk} files for 300~dpi printers or previewers. Additional
values of the \ixem{dpi} parameter are also provided in {\tty musicpk.zip}.
Normally needed fonts are {\tty musikn20}, {\tty musikn16}, {\tty musikn13},
{\tty musikn11},  {\tty musicbra} or {\tty musicbrb}\footnote{{\tt musicbra}
and {\tt musicbrb} exhibit some drawing differences, but they meet the same
purpose of typing \itxem{piano braces}.}, {\tty beamn20}, {\tty beamn16}, {\tty beamn13},
{\tty beamn11},  {\tty slurn20} and {\tty slurn16}.

\section{Installing the fonts}
 \textit{All} files with the extension \verb|.tfm|\footnote{\TeX\ font metric
files, needed directly from binary for \TeX ing the examples or the doc.} have
to be copied in the same directory as the other \verb|.tfm|'s.
Then, if you get the error message:

\verb|! Font ... not loadable: Metric (TFM) file not found.|

\noindent this means
you did not succeed in installing the \verb|.tfm|'s, or you installed them in
the wrong directory. Then look at your general \TeX\
documentation\footnote{The specific \TeX\ installation manual, not necessarily
the \TeX book.} and try again.


The \verb|.tfm| only contains the width, height and depth of each character
of a font and is the only font file needed for \TeX ing. But, to preview
and/or view you need the pixel fonts, i.e. not the metrics but the exact
drawing of the characters. On most systems they are packed and have either
the extension \verb|.pk| or the extension {\tt.}$<dpi>${\tty pk}, where $dpi$
is the actual resolution of the font needed in the requested size. This means
that the the given \verb|*.pk| files provided often have to be renamed as
\verb|*.300pk|, especially in \unix\ systems.

 Mostly needed and spreaded are the fonts for \verb|dvi|-driver with
resolution of 300dpi. Using Em\TeX the \verb|.pk|-fonts have to be copied in
\verb|...\pixel.lj\300dpi\|. In \unix\ systems they often have to be renamed
as  \verb|*.300pk| and put in a directory --- the name of which can be
provided by your local manual or \TeX-wizard --- of the form
\verb|.../fonts/pk|.

\section{Building a format}\label{formatbuild}

Introducing the basically needed files in a \itxem{format} (with
\ixem{INITEX}) is a means of saving computer time and memory. Besides, you
will have a format compatible with \musixtex\ and --- provided you made the
symmetrical format for \musixtex that is, including {\tt musixcpt} in your
\musixtex\ format --- you can compile exactly the same source files with both
\musictex\ and \musixtex, which is a good means of finding whether some
strange behaviour is specific to one implementation or the other, or whether
you made some general mistake.

\subsection{Starting from nil}
\begin{enumerate}
\item Build up a file called \verb|musictex.ini| with following contents:

\begin{verbatim}
\input plain % or your local plain-like TeX format
\input musicnft
\input musictex
\input musicvbm 
\input musictrp % optionally
\input musicsty % optionally
\def\fmtname{musictex}\def\fmtversion{5.08}
\dump
\end{verbatim}

\item Make your format with the command

\smallskip
 \verb|initex musictex.ini|\footnote{depends on your implementation.
{\tty initex} may require another form, such as
``\verb|tex386 -i|''.}

\item Then your format may me invoked for \TeX ing score by something like


 \verb|tex &musictex| \textit{jobname}\footnote{depends on your
implementation. Very often the formatless \TeX is invoked by \verb|virtex|
rather than \verb|tex| which already invokes a default plain format.}.
\end{enumerate}

\subsection{Starting from your usual plain format}
\begin{enumerate}
 \item First, try to find --- on your favourite system --- whether \verb|tex| is
an executable routine, or a \verb|tex.bat| command in MS-DOS or a {\sl
shell\/} procedure under \unix.
 \item If \verb|tex| is a command try to find the ``initex'' local command:
usually it is either \verb|initex| or \verb|tex -i|.
 \item Try to find the name of the ``plain \TeX'' format (usually posted
when \TeX ing anything.
 \item Then, {\it mutatis mutandis\/}, assuming the ``initex'' command has
the name \verb|initex| and the ``plain \TeX'' format is \verb|plain|, run the
shell command:

 \medskip
 \verb|initex  \&plain musictex.ins|

 \medskip\noindent which will produce a format file {\tt musictex.fmt} which
you shall put in the same directory  as the others formats (hoping you have
the access rights...). Note, in \unix\ systems, the backslash before the
\verb|&| which tells the system to consider this character as a member of the
command, not a batch execution indication. Once this is done, you can
\musictex\ any score you have written using a command such as:

 \medskip
 \verb|tex \&musictex my-score.tex|

 \medskip\noindent that is, specifying your new format {\tt musictex.fmt}
instead of the usual {\tt plain.fmt}.

 For MS-DOS/{\tt emTeX} users the format building command is:

 \verb|tex386 -i &plain musictex.ins|

 \noindent (you can change \verb|plain| into \verb|dc-plain| or any other
plain-like format you have) then:

 \verb|copy musictex.fmt \emtex\btexfmts\*.*|

 \noindent and the {\tt musictex.bat} command can be

% It makes no sense to state -mt20000, perhaps this is useful for MusicTeX,
% (it uses hyphenation to break a line) but not for MusiXTeX (it uses musixflx)
\begin{quote}\begin{verbatim}
if exist %1.tex goto tex
goto end

:tex
tex386 -mt20000 &musictex %1 %2 %3 %4 %5 %6 %7 %8 %9

:end
 \end{verbatim}\end{quote}
\end{enumerate}




 \input musicdoc.ind
 
 
\chapter{Examples}

\vskip -8mm
Due to compatibility problems with \LaTeX\ (used to produce this notice)
large examples must be \TeX -ed separately, i.e.\ using \TeX\ and not \LaTeX.
Therefore, the Music\TeX\ future user is suggested to produce
some of the following examples and to look carefully at the way some
special features have been coded.

\medskip When producing this examples, care should be taken about the fact
that many several files are supposed to be included (by means of
\keyindex{input}) in other files. Thus the only good files to be directly
\TeX-ed are those which begin with ``{\Bslash input musicnft}'' or
 ``\verb|% \input musixtex|''. This latter command indicates that this
exemple can be run, both with \musictex\ inserting

 \verb|\input musicnft| 

 \verb|\input musictex| 

 \verb|\input musicvbm| 

 \verb|\input musictrp| 

 \verb|\input musicsty|

 \noindent at the beginning, or inserting the correcponding files of
MusiX\TeX\ (see corresponding manual). A simplerway of doing consists in having
two formats, one for \musictex\ (see \ref{formatbuild}), one for MusiX\TeX\
plus its compatibility input files. 

In addition, it must be noted that most DVI previewers and laser printers
have their origin at one inch below and one inch right of the right upper
corner of the paper, while the musical examples have their upper left
significant corner only at one centimeter right and below the left top of the
paper. Therefore, special parameters have to be given to the DVI
transcription programs unless special \keyindex{hoffset} and
\keyindex{voffset} \TeX\ commands are introduced within the source \TeX\
text.

\medskip Suggested tests are:

\begin{itemize}
\item{\tty PACIFIQN} for a long original piano work (11 pages);
 {\tty PACIFIQB} is the same in 16~pt staff size;
\item{\tty CARILLON} for a sophisticated piano score (use Big\TeX~!);
\item{\tty PRAETORI} to get an ancient polyphonic song with three
 transpositions;
\item{\tty ANGESCAO} if you like Christmas carols with four voices, a
 three-staff organ score and the same transposed to meet ordinary
 singer's limitations; {\tty ANGESCAM} is the same without organ. See {\tty
 angescax.tex} to see how the same source has been used for two distinct
layouts. 
\item{\tty HPRELFUG} if you like imitations of J.-S. Bach
  (included in {\tty RECUEIL});
\item{\tty MARCELLX} if you like pre-baroque music  (included in {\tty RECUEIL});
\item{\tty RECUEIL} if you want to get all the organ works\footnote{All
subsets of {\tty RECUEIL} can be \musictex-ed separately.} of the author in a
single book (65 pages);
 \item{\tty HWIDOR} and {\tty NWIDOR} to get the Toccata by Charles-Marie
{\sc Widor} in two different sizes;
 \item{\tty AVEMARIX} to get the ``M\'editation'' (alias ``Ave Maria'') by
Charles {\sc Gounod} for organ and violin or song.\index{Gounod, C.} and {\tty
AVEMARIO} to get the same for organ solo;
 \item{\tty RACINE} to get the ``Cantique de Jean Racine'' by Gabriel {\sc
Faur\'e} in a transcription fit for organ rather than for
piano.\index{Faur\'e, G.}

\end{itemize}\index{Widor, C.-M.}\index{Gounod, C.}

\chapter{Summary of denotations}
\section{Pitches}
\begin{music}
\cleftoksi={6000}
\def\lbnt#1{\zcharnote{9}{\tty #1}\wh{!#1}}
\def\LBNT#1{\zcharnote{9}{\tty !\relax #1}\wh{!#1}}
\def\nbinstruments{1}\relax
\debutextrait\autolines{13}{1}{12}\temps\notes\qsk
\LBNT {`A}\LBNT {`B}\relax
\LBNT {`C}\LBNT {`D}\LBNT {`E}\relax
\LBNT {`F}\LBNT {`G}\relax
\lbnt A\lbnt B\lbnt C\lbnt D\lbnt E\lbnt F\enotes\finextrait
\def\lbnt#1{\zcharnote{-5}{\tty #1}\wh{!#1}}\relax
\debutextrait\autolines{13}{1}{12}\temps\notes\qsk
\lbnt G\lbnt H\lbnt I\relax
\lbnt J\lbnt K\lbnt L\lbnt M\lbnt N\lbnt a\lbnt b\lbnt c\lbnt d\lbnt e\relax
\enotes\finextrait
\def\nbinstruments{1}\relax
\cleftoksi={0000}
\def\lbnt#1{\zcharnote{10}{\tty #1}\wh{!#1}}\relax
\debutextrait\autolines{13}{1}{12}\temps\notes\qsk
\lbnt a\lbnt b\lbnt c\lbnt d\lbnt e\lbnt f\lbnt g\lbnt h\lbnt i\relax
\lbnt j\lbnt k\lbnt l\lbnt m\enotes\finextrait
\def\lbnt#1{\zcharnote{-4}{\tty #1}\wh{!#1}}\relax
\debutextrait\autolines{13}{1}{12}\temps\notes\qsk
\lbnt n\lbnt o\lbnt p\lbnt q\lbnt r\relax
\lbnt s\lbnt t\lbnt u\lbnt v\lbnt w\lbnt x\lbnt y\lbnt z\relax
\enotes
\finextrait
\end{music}
\section{Notes and Rests}
\begin{music}
\def\mcra#1{\zcharnote{10}{\hbox to 1.3\Interligne{\hss\Bslash #1\hss}}}
\def\Mcra#1{\zcharnote{14}{\hbox to 1.3\Interligne{\hss\Bslash #1\hss}}}
\def\nbinstruments{1}
\debutextrait\normal\temps\NOTes\sk
\mcra{zbv}\zbv i\sk\mcra{zsb}\zsb i\sk\enotes\Notes\relax
\mcra{wh}\wh i\qsk
\mcra{hu}\hu f\qsk\mcra{hl}\hl l\qsk
\mcra{qu}\qu f\qsk\mcra{ql}\ql l\qsk
\mcra{cu}\cu f\qsk\mcra{cl}\cl l\qsk
\qsk\mcra{ccu}\ccu f\qsk\qsk\mcra{ccl}\ccl l\qsk\relax
\qsk\qsk\mcra{cccu}\cccu d\qsk\qsk\qsk\mcra{cccl}\cccl l\qsk
\qsk\qsk\mcra{ccccu}\ccccu f\qsk\qsk\qsk\mcra{ccccl}\ccccl l\enotes\finextrait
\end{music}
%
\begin{music}
\def\mcra#1{\zcharnote{10}{\hbox to 1.3\Interligne{\hss\Bslash #1\hss}}}
\def\Mcra#1{\zcharnote{14}{\hbox to 1.3\Interligne{\hss\Bslash #1\hss}}}
\def\nbinstruments{1}
\debutextrait\normal\temps\NOTes\qsk
\mcra{whp}\whp i\qsk
\mcra{hup}\hup f\qsk\mcra{hlp}\hlp l\qsk
\mcra{qup}\qup f\qsk\mcra{qlp}\qlp l\qsk
\mcra{cup}\cup f\qsk\mcra{clp}\clp l\qsk\enotes
\Notes\qsk\mcra{qh0}\ibu0f0\qh0f\tbu0\sk
\qsk\mcra{qb0}\ibl0l0\qb0l\tbl0\sk\relax
\qsk\mcra{qhp0}\ibu0f0\qhp0f\tbu0\sk
\qsk\mcra{qbp0}\ibl0l0\qbp0l\tbl0\sk\relax
\qsk\mcra{qhpp0}\ibu0f0\qhpp0f\tbu0\sk
\qsk\mcra{qbpp0}\ibl0l0\qbpp0l\tbl0\sk\relax
\enotes
\finextrait
%
\def\mcrb#1#2{\zcharnote{14}{\hbox to 1.3\Interligne{\hss\Bslash #1\hss}}%
\mcra{#2}}%
\def\Mcrb#1#2{\zcharnote{18}{\hbox to 1.3\Interligne{\hss\Bslash #1\hss}}%
\Mcra{#2}}%
\def\mcrc#1#2#3{\zcharnote{18}{\hbox to 1.3\Interligne{\hss\Bslash #1\hss}}%
\mcrb{#2}{#3}}%
\debutextrait\normal\NOTES\qsk\relax
\Mcrb{seizsoupir}{qqs}\qqs\qsk%     hemi-demi-semi-quaver rest
\mcrb{huitsoupir}{hs}\hs\qsk%             demi-semi-quaver rest
\mcrc{quartsoupir}{qsoupir}{qs}\qs\qsk\qsk% semi-quaver rest
\mcrc{demisoupir}{dsoupir}{ds}\ds\qsk%  quaver rest
\enotes\finextrait
\debutextrait\normal\NOTEs
\mcrb{soupir}{qp}\qp\qsk%                  crotchet rest
\mcra{hpause}\hpause%                  minim rest
\mcra{pause}\pause%                    semibreve rest
\mcra{PAuse}\PAuse%                    ? rest
\enotes\NOTes
\mcra{PAUSe}\PAUSe%                    ?? rest
\enotes\finextrait
\end{music}
%
\section{Other symbols}


\begin{music}
\def\nbinstruments{1}\parindent 0pt
\def\mcra#1{\zcharnote{17}{\hbox to 1.3\Interligne{\hss\Bslash #1\hss}}}
\def\Mcra#1{\charnote{17}{\hbox to \noteskip{\Bslash #1\hss}}}
\debutextrait\NOTEs
\mcra{nTrille}\nTrille{n}4\sk\qsk
\mcra{ntrille}\ntrille{n}4\sk\sk
\mcra{pince}\pince{n}\sk\mcra{Pince}\Pince{n}\sk\relax
\mcra{mordant}\mordant{n}\sk
\mcra{turn}\turn{n}\sk\mcra{backturn}\backturn{n}\sk
\mcra{coda}\coda{n}\sk\mcra{segno}\segno{n}\sk
\enotes\finextrait
\debutextrait
\def\mcra#1{\zcharnote{10}{\hbox to \noteskip{\Bslash #1\hss}}}\relax
\NOTes\sk\relax
\mcra{dimin}\zcharnote b{\dimin}\sk\sk
\mcra{Dimin}\zcharnote b{\Dimin}\sk\sk\relax
\mcra{DImin}\zcharnote b{\DImin}\sk\sk\sk\relax
\mcra{DIMin}\zcharnote b{\DIMin}\sk\sk\sk\sk\relax
\enotes\finextrait
\debutextrait
\def\mcra#1{\zcharnote{10}{\hbox to \noteskip{\Bslash #1\hss}}}\relax
\NOTes\sk\relax
\mcra{cresc}\zcharnote b{\cresc}\sk\sk\relax
\mcra{Cresc}\zcharnote b{\Cresc}\sk\sk\relax
\mcra{CResc}\zcharnote b{\CResc}\sk\sk\sk\relax
\mcra{CREsc}\zcharnote b{\CREsc}\sk\sk\sk\sk\relax
\enotes\finextrait
\debutextrait
\def\mcra#1{\zcharnote{14}{\hbox to 2\Interligne{\hss\Bslash #1\hss}}}\relax
\NOTEs\sk
\mcra{pointdorgue}\pointdorgue{l}\wh k\sk\relax
\mcra{pointdurgue}\pointdurgue{e}\wh f\sk\enotes
\NOTes
\mcra{PED}\PED\sk\mcra{DEP}\DEP\sk\relax
\enotes\finextrait
\end{music}
\end{document}


