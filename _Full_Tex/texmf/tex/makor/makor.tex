%% This file provides the style for the Makor fonts.
%\input zeuppl 
%\font\5=pplr8a \5

%%
%% Now we can use ! to create private macros...
%%%
\def\LETTER{11 }
\def\MAKEEXCLAMLETTER{%
  \catcode`!=\LETTER % Now, macro names like \!tmp are OK!
}
\def\MAKEEXCLAMOTHER{\catcode`!=12 }
\MAKEEXCLAMLETTER

%%
%% REGISTERS
%%
\newcount\scr

%%
%% CATCODE CHANGES, CONTROL SYMBOLS
%%

%%
%% SYNONYMS, ETC.
%%
\def\active{13 }\let\ACTIVE=\active \let\VOWEL=\ACTIVE
\def\OTHER{12 }\def\SUPERSCRIPT{7 }\def\SUBSCRIPT{8 }
\let\COLON=: % store original value of colon.
\let\tictactoe=\# \let\vert=| 
\let\origcircum=^ \let\origunder=_ 
\let\lbrack=[ \let\rbrack=]
\let\!lt=\let \!lt\!ctcd=\catcode \!lt\!df=\def \!lt\!gdf=\gdef
\let\!glbl=\global \def\!glt{\global\let} \let\!shw=\show
\let\!shwth=\showthe
\let\PAR=\par \let\NOBOUNDARY=\noboundary
\def\'{\char127 }% the right quote

%%
%% UTILITY MACROS
%%

%% \boxme{stuff} put a box around stuff.
\newdimen\Vthickness \Vthickness=.2pt
\def\boxme#1{\vbox{\hrule height\Vthickness
	\hbox{\vrule width\Vthickness#1\vrule width\Vthickness}%
	\hrule height\Vthickness}}

%% POOR MAN'S BOLD (form the TeXbook) is usefulin cases where a genuine
%% bold-face font is not available.
\newbox\pmbbox
\def\pmb#1{\setbox\pmbbox=\hbox{#1}%
  \kern-.02em\copy\pmbbox\kern-\wd\pmbbox
  \kern.04em\copy\pmbbox\kern-\wd\pmbbox
  \kern-.02em\raise.04em\box\pmbbox }

%% grabbing and gobbling a single token...
\def\!gbbl#1{}

%% DIVISION.  \!dimdivide{\a}{\b}{\c} assumes that \a and \b represent
%% dimensions.  \!dimdivide computes the ratio (\a)/(\b) and stores
%% this dimensionless quantity as the definition of \c.
\newdimen\!tempdima \newdimen\!tempdimb
{\catcode`\p=\OTHER \catcode`\t=\OTHER
  \gdef\!remPT#1pt{\gdef\!sf{#1}}}
\def\!dimdivide#1#2#3{\!tempdima=#1\!tempdimb=#2%
  \divide\!tempdimb by4 \divide\!tempdimb by16384 % altogether by
						  % 65536
  \divide\!tempdima by\!tempdimb \expandafter\!remPT\the\!tempdima
  \let#3=\!sf}

%% TYPESETTING NUMBERS.  It's necessary to use the \NUM macro to
%% properly typeset left-to-right numbers in a Makor document.
\def\NUM#1{\!NUMBER#1MAKOR}\def\!EMPTY!{}
\def\!NUMBER#1#2MAKOR{\def\!tmp{#2}%
  \ifx\!tmp\!EMPTY! % no more tokens to reverse
    \def\next{#1}%
  \else
    \def\next{\NUM{#2}#1}%
  \fi \next}

%% Use these macros to scale dimension registers by scale factors.
%% \!scaledim{\dimen0}{\csname \!hfont !SF\endcsname} is the same
%% as \!scalebyfontSF{\dimen0}. 
\def\!scaledim#1#2{#1=#2#1\relax}
\def\!scalebyfontSF#1{#1\expandafter\csname \!hfont !SF\endcsname#1}

%% FONT-NAMING for PLAIN TeX.  Because using Hebrew fonts requires
%% computing and using scale factors, we need to hide this from the
%% innocent user by providing wrappers for defining/using Hebrew
%% fonts.  \hfontdef{a}{b}{size} defines a font command \a which relates
%% to the system font b at <size>.  NONE of the 3 arguments can be
%% control sequences.  Ex: \hfontdef{myfont}{oshomsehe}{12pt}
%% (I suppose the implicit assumption is that arg#2 represents a
%% scalable font.)  Since we NEVER get kerns or character numbers
%% except when a Hebrew font is current, it is OK to adjust these
%% kerns, etc by the scale factor
%%
%% At the same time a base font is created (\a in the above example),
%% we need also a corresponding vowel-less font.  If the system name
%% for the font is b, then b0 represents its vowelless counterpart.
%% If \a is the TeX name for the font, then \xa is vowels-off (the
%% `x' is supposed to suggest cancellling of the vowels).
\newif\if!Hebrew \!Hebrewfalse % default
\def\hfontdef#1#2#3{%
  \ifundefined{#1}% unused control sequence...
    \expandafter\font\csname#1\endcsname=#2 at #3
    \expandafter\font\csname x#1\endcsname=#20 at #3
    \stringdef{#1!size}{#3}% associate size with TeX font name
    \stringdef{#1!system}{#2}% associate system name with TeX font
    \stringdef{x#1!size}{#3}% associate size with TeX font name
    \stringdef{x#1!system}{#2}% associate system name with TeX font
    \!dimdivide{10pt}{#3}{\!!tmp}%
    \expandafter\let\csname  #1!SF\endcsname=\!!tmp
    \expandafter\let\csname x#1!SF\endcsname=\!!tmp
  \else
    \message{***** Font def error: #1 already in use as font name.
    Ignoring your command.}%
  \fi}

%% DECLARING A DEFAULT HEBREW FONT.  It's a nuisance to keep
%% invoking a hebrew font within the [...] environemnt.  Therefore,
%% we provide the optionof declaring a default font which will be
%% called automatically.  Of course, it can always be over-ridden.
%% USAGE: Aftar, say, \hfontdef{hebrew}{...}{...}, we can
%% say \declarehdefault{hebrew}.
\def\declarehdefault#1{\def\!hdefaultfont{#1}}

%% USING the font.  The second form calibrates \baselineskip as well.
%% This command establishes vowelled and voweless font calls, and sets
%% up the default vwoel state on the basis of the user's preferences.
\newif\if!vowels 
\def\declarevowelsdefault{\!vowelstrue}
\def\declarevowelsoff{\!vowelsfalse}
\declarevowelsdefault % AH's personal preference!

\def\usehfont#1{\gdef\!hfont{#1}\gdef\!xhfont{x#1}\!Hebrewtrue
   %\declarehdefault{#1}%
   \if!vowels % vowels on...
     \csname #1\endcsname
   \else % vowels off...
     \csname x#1\endcsname
  \fi}
\def\Usehfont#1{\usehfont{#1}\baselineskip=1.4\csname#1!size\endcsname
    }
\let\hfont=\usehfont \let\Hfont=\Usehfont % synonyms

%% BACK AND FORTH FROM VOWELS TO NO VOWELS, & VICE-VERSA.
\def\V{% use the vowelled font...
  \!vowelstrue\expandafter\csname\!hfont\endcsname}%
\def\CXLV{% vowels off!
  \!vowelsfalse\expandafter\csname\!xhfont\endcsname}%

%% \stringdef{str}{stuff stuff} defines \csname str\endcsname to
%% have the expansion <stuff stuff>.
\def\stringdef#1#2{\expandafter\def\csname#1\endcsname{#2}}

%% In the same category as \stringdef is \strxq (STRing eXeQute},
%% which enables a user to execute commands with wierd names or
%% command names containing vowels in environments where vowels are
%% non-letters. Ex: \strxq{hspace}{1pc} or \strxq{|"}.
\def\strxq#1{\csname #1\endcsname}

%% Here's the TeXbook's macro to test if \srt has been define.
\def\ifundefined#1{\expandafter\ifx\csname#1\endcsname\relax}

%% \truncdim\x shaves the value of a dimen register (here, \x) so that
%% all the wierd decimal expansion disappears.  For example,
%% \Truncdim{2.000034pt} yields 2.
\def\truncdim#1{\scr=#1 \divide\scr by 65536\relax}

%% \!getcurrchar relies on the fact that the italic corrections for
%% each character is equal to .nn.
%% For example, if the italic corr is .44pt, the character occupies
%% position 44 in the Makor encoding.
\newcount\currchar  % what's the current character position?
\def\!getcurrchar{\/\dimen0=\lastkern \kern-\dimen0\relax
  \!scalebyfontSF{\dimen0}%
  \advance\dimen0 by.0003pt \multiply\dimen0 by100
  \truncdim{\dimen0}\currchar=\scr
}

%% \getkern{a}{b} sores the kerns between characters a and b in the current
%% font in \dimen0.
\def\getkern#1#2{\setbox0=\hbox{#1#2}\dimen0=\wd0\relax
  \setbox0=\hbox{#1{}#2}\advance\dimen0 by-\wd0% dimen0 is the kern
}

%% In Makor, the kerns between the nullchar (\char254) and any other
%% character yiled the width of the offset from the medial axis
%% through the type.  A positive kern means the displacement is to the
%% right; to the left, otherwise.  \getloweraxis{ch} gets the lower
%% axis of the letter ch.
\newdimen\axisoffset
\def\nullchar{\char254 }
\def\getloweraxis#1{\getkern{\nullchar}{#1}\axisoffset=\dimen0 }
\def\getupperaxis#1{\getkern{#1}{\nullchar}\axisoffset=\dimen0 }

%% \viewnextchar examines the next character in the token list,
%% and leaves it for further typesetting.  \stealnextchar examines and
%% removes the next character from the token list.
\def\!makemakorspecials{\catcode`\^=\OTHER \catcode`\_=\ACTIVE }
\def\!makevowelsactive{\!makemakorspecials
  \!ctcd`\a=\VOWEL \!ctcd`\e=\VOWEL 
  \!ctcd`\i=\VOWEL \!ctcd`\u=\VOWEL \!ctcd`+=\VOWEL \!ctcd`:=\VOWEL 
  \!ctcd`|=\VOWEL \!ctcd`'=\VOWEL \!ctcd`"=\VOWEL }
\def\!nm!k!m!k!rsp!c!!ls{% unmakemakorspecials...
  \!ctcd`\^=\SUPERSCRIPT \ctcd`\_=\SUBSCRIPT }
\def\!mkvwlslttr{\!nm!k!m!k!rsp!c!!ls
  \!ctcd`a=\LETTER \!ctcd`e=\LETTER \!ctcd`i=\LETTER 
  \!ctcd`u=\LETTER 
  \!ctcd`+=\OTHER \!ctcd`:=\OTHER \!ctcd`|=\OTHER \!ctcd`"=\OTHER }
\def\!mkvwlsvwls{\!lta=\vwlA \!lte=\vwlE \!lti=\vwlI \!ltu=\vwlU \!lt'=\vwlSHEVA
  \!lt+=\vwlPLUS \!lt:=\vwlCOLON \!lt|=\vwlVERT \!lt"=\vwlQUOTES }
\def\viewnextchar#1{\def\!tmp{#1}\futurelet\nextchar\!tmp}
%% There doesn't appear to be an easy way to test if \nextchar
%% is a vowel other than explicit comparisons using \ifx...\fi
%% \testforvowelness is always invoked when the first vowel of a 1 or
%% 2 vowel sequence appears.  Really, we are testing the for the
%% vowelness of the next character...
\newif\ifnextisvowel \newcount\nextvowel
\let\lettera=a \let\lettere=e \let\letteri=i \let\letteru=u \let\letterplus=+ \let\letterquotes="
\def\testforvowelness{\nextisvowelfalse % default condition
  \let\!nxtvwl=X% default value---something is wrong!
  \ifx \vwlA\nextchar \let\!nxtvwl=\lettera \nextisvoweltrue \fi
  \ifx \vwlE\nextchar \let\!nxtvwl=\lettere \nextisvoweltrue \fi
  \ifx \vwlI\nextchar \let\!nxtvwl=\letteri \nextisvoweltrue \fi
  \ifx \vwlU\nextchar \let\!nxtvwl=\letteru \nextisvoweltrue \fi
  \ifx \vwlPLUS\nextchar \let\!nxtvwl=\letterplus \nextisvoweltrue \fi
  \ifx \vwlQUOTES\nextchar \let\!nxtvwl=\letterquotes \nextisvoweltrue \fi
}

%% \testforspecials sets a flag in case certain special conditions are
%% to be noted.  As of this writing, there are only two---is the
%% current character a final khaf or nun?  If so, we use special vowels.
\def\!au!{au}
\def\testforspecials{\ifnum\currchar=6 %now test for kamatz or
				       %sheva...
    \if+\currvowel \def\currvowel{\char61}\fi
    \if'\currvowel \def\currvowel{\char62}\fi 
  \fi 
  \ifnum\currchar=1 % is it a final nun?
    \if+\currvowel \def\currvowel{\hskip3.5pt\char43}\fi
  \fi }

%% Every time a vowel needs placement, here's what we need to do:
%%   \item is next letter a vowel?  (Sometimes, vowels are specified
%%         with 2-letter ligature combinations, such as :a or |+.)
%%   \item what character is the preceding consonant?
%%   \item is the consonant a special case?  (Special cases so far are only
%%         the sheva and kamatz used for k(h)afsophit.)
%%   \item If no special case: put vowel in enlarged vowelbox.
%%   \item Center the vowel box on the medial axis.
%%   \item Move to next consonant.
%%   
%%   \hebrewloweraccent{\char 100}{segol} puts the segol accent underneath
%%         the dalet (char100).
%% Because only vowels are {\it active\/}, we can assume that this machinery
%% is only activated whenthe current character is a vowel.
\newbox\vowelbox
\def\nulloffset{\axisoffset=0pt }
\def\taketwosteps#1#2{\skip0=#1 \advance\skip0 by#2\relax
  \dimen0=\skip0
  \hskip\dimen0\relax}
\def\olap#1{\hbox to0pt{\hss#1\hss}}
%% \def\hebrewloweraccent{\!getcurrchar \testforspecials
%%   \nulloffset\getloweraxis{\char\currchar}%
%%   \setbox0=\hbox{\char\currchar}%
%%   \taketwosteps{-.5\wd0}{-\axisoffset}\olap{\currvowel}%
%%   \taketwosteps{.5\wd0}{\axisoffset}\nobreak}

\def\voweldef#1#2{%
  \expandafter\def\csname !#1\endcsname{\def\currvowel{#2}\show\currvowel}}
\def\Voweldef#1{\voweldef{#1}{#1}}
\Voweldef{'}%
\Voweldef{a}\Voweldef{e}\Voweldef{i}\Voweldef{+}\Voweldef{u}%
\Voweldef{|}\Voweldef{"}

%% ACCENTS UNDER FINAL GUTTERAL.  If the final letter of a word is
%% gutteral, and takes an accent, then the vowel is vocalized before
%% the consonant.  The transpose operator _ allows us to enter text
%% the way it's read.  Note: we exploit the fact that the accent can be
%% represented by a single character (a) and the gutteral needs two
%% characters.
\def\!transpose#1#2#3{#2#3\vwlA}
{\catcode`\_=\ACTIVE \global\let_=\!transpose}

%% MAKING VOWELS ACTIVE.  First, NOTE WELL that neither `o' nor `O'
%% is a `vowel', in the Makor sense of the term. The vowel marks they
%% set appear on top of or next to a base letter, so special finagling
%% is simply unnecessary.
%%
%% The posssible vowels that appear in the `first' position are a, e,
%% i, u, +, :, and |.  
\newtoks\!vwl
\def\!checknextchar{\viewnextchar{\!ptvwl}}
\def\!!checknextchar{\testforvowelness 
  \ifnextisvowel 
    \!vwl=\expandafter{\the\!vwl\char\nextvowel}%
  \else % return char to token list...
    \nextchar
  \fi
  \!ptvwl
  }
\def\!ptvwl{\!getcurrchar \edef\currvowel{\the\!vwl}\testforspecials 
  \testforvowelness
  \ifnextisvowel
    \edef\currvowel{\currvowel\!nxtvwl}% put nxtvwl in vowel ligature...
    \let\!nxt=\!gbbl
  \else % next character is a consonant
    \let\!nxt=\relax
  \fi
  \!!ptvwl \!nxt
}
\def\!!ptvwl{\nulloffset\getloweraxis{\char\currchar}%
  \setbox0=\hbox{\char\currchar}%
  \nobreak\taketwosteps{-.5\wd0}{-\axisoffset}\olap{\currvowel}%
  \nobreak\taketwosteps{.5\wd0}{\axisoffset}}
\def\getvowelmacro#1#2{\expandafter\def\csname vwl#1\endcsname{%
  \noboundary\!vwl={#2}\!checknextchar}}
\getvowelmacro{A}{a}
\getvowelmacro{I}{i}
\getvowelmacro{E}{e}
\getvowelmacro{PLUS}{+}
\getvowelmacro{U}{u}
\getvowelmacro{SHEVA}{'}
\getvowelmacro{COLON}{:}
\getvowelmacro{VERT}{|}
\getvowelmacro{QUOTES}{"}

{\!makevowelsactive
  \!glta=\vwlA \!glt+=\vwlPLUS \!glti=\vwlI \!glte=\vwlE 
  \!gltu=\vwlU \!glt'=\vwlSHEVA
  \!glt:=\vwlCOLON \!glt|=\vwlVERT \!glt"=\vwlQUOTES
} % end of active vowel environment
\newif\if!HEBREW \!HEBREWfalse
\def\HEBREW{\beginR
   \!HEBREWtrue\endgraf\bgroup\!makevowelsactive }
\def\ENDHEBREW{\egroup\endgraf\!HEBREWfalse\endR}
\def\!offparindent{\leavevmode\setbox0=\lastbox}
\def\[{\!offparindent\bgroup\!HEBREWtrue
   \beginR\!makevowelsactive
   \hfont\!hdefaultfont% default hebrew
} 
\def\]{\endR\egroup}
%\MAKEEXCLAMOTHER
%%%%%%%%%%%%%%%%%%%%%%%%%%%%%%%%%%%%%%%%%%%%%%%%%%%%%%%%%%%%%%%%%%%%%%%%%%%%%
\TeXXeTstate=1 
%%%%%%%%%%%%%%%%%%%%%%%%%%%%%%%%%%%%%%%%%%%%%%%%%%%%%%%%%%%%%%%%%%%%%%%%%%%%%

\endinput

