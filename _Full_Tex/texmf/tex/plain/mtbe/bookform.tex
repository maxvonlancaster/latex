%%% File: BookForm.tex
%%% From `Mathematical TeX by Example' by Arvind Borde
%%% (c) 1993, Academic Press.
%%% Contents: The formatting commands for the `inner' right hand pages 
%%% of some of the Examples in `Mathematical TeX by Example'.

% The commands shown here should suggest how a relatively simple
% TeX file may be used to create a variety of different output formats:
% Examples 2--5 were all formatted with the commands given here, yet
% they each have a different `look'. 

\input EightPt  % To use small fonts.
\nopagenumbers  % Switch off normal page numbering.

%----------------------------------------------------------------------------
%%% TYPEFACE NAMES:

\font\Smallcaps=cmcsc10
\font\Sansserif=cmss10

%---------------------------------------------------------------------------_%
%%% FOR DISPLAYED FORMULAS:

\newcount\EquationNo     \EquationNo=1

% Some convolutions are needed in order to allow the equation number to 
% `leak out' and to make it correctly to the headline:
 
\def\EqLbl #1{(\the\EquationNo #1)\xdef\Store{.$\>$\the\EquationNo #1}%
   \global\postdisplaypenalty10000 
   \global\skip0=\belowdisplayskip 
   \global\skip1=\belowdisplayshortskip 
   \global\belowdisplayskip=-.8\baselineskip
       \global\advance\belowdisplayskip by\skip0
   \global\belowdisplayshortskip=-.8\baselineskip
       \global\advance\belowdisplayshortskip by\skip1
   \aftergroup\PlaceMark
    \def\AA{#1}\def\BB{}\ifx\AA\BB
             \global\advance\EquationNo by1 \fi}%
\def\PlaceMark{\mark{\Store}\par \global\postdisplaypenalty0 
               \global\belowdisplayskip=\skip0 
               \global\belowdisplayshortskip=\skip1 
               \noindent\ignorespaces}%
\def\FirstEquationOnPage{\edef\Testfm{\firstmark}%
                         \edef\Testtm{\topmark}%
                         \edef\Testbm{\botmark}%    
                         \ifx\Testfm\Testtm\else\firstmark\fi}%
\def\LastEquationOnPage{\edef\Testfm{\firstmark}%
                        \edef\Testtm{\topmark}%
                        \edef\Testbm{\botmark}%    
                        \ifx\Testbm\Testtm\else\botmark\fi}%
\def\PrevEq {{\advance\EquationNo by -1 \the\EquationNo}}
\def\PrevEqs #1{{\advance\EquationNo by -#1 \the\EquationNo}}
\def\ShiftEquationNo #1{\advance\EquationNo by #1}

%----------------------------------------------------------------------------
%%% FOR FOOTNOTES:

\newcount\FootnoteNo     \FootnoteNo=1
\let\OldFootnoteCommand=\footnote

\def\NumberFootnotes{\def\footnote{\OldFootnoteCommand
        {$^{\the\FootnoteNo}$\aftergroup\global\aftergroup\advance
            \aftergroup\FootnoteNo\aftergroup b\aftergroup y\aftergroup
            1\aftergroup\relax}}} 
\def\DontNumberFootNotes{\def\footnote{\OldFootnoteCommand}}        
\def\FootnoteSize #1{\dimen\footins=#1} % To set maximum vertical size.

%---------------------------------------------------------------------------_%
%%% FORMATTING CHAPTERS, SECTIONS, SUBSECTIONS:

\newcount\ChapIndicator  % New chapter indicator.
\newcount\ChapNo         % Chapter number.
\newcount\SectNo         % Section number.
\newcount\SubSectNo      % Subsection number.

\def\ChapterFonts #1#2{\font\ChpFont=#1
                       \font\ChpNameFont=#2}
\def\SectionFont #1{\font\SctFont=#1}
\def\SubSectionFont #1{\font\SbSctFont=#1}

\newskip\StartofChapSkip         \StartofChapSkip=.75in
\newskip\InChapTitleSkip         \InChapTitleSkip=3\bigskipamount
\newskip\AfterChapTitleSkip      \AfterChapTitleSkip=2\bigskipamount
\newskip\BetweenSectionSkip      \BetweenSectionSkip=2\bigskipamount
\newskip\AfterSectionTitleSkip   \AfterSectionTitleSkip=\bigskipamount
\newtoks\ChapName            % To store the chapter name.
\newtoks\SectionName         % To store the section name.
\newtoks\EveryChapter        % The end of this file shows the default.   
\newtoks\EverySection        % Ditto
\newtoks\EverySubSection     % Ditto

\newif\ifResetSectNo       \ResetSectNotrue
\def\DontResetSectNos{\ResetSectNofalse}

\def\StartBook{\ChapNo=0 \SectNo=0 \SubSectNo=0 \ChapIndicator=0 \pageno=1
               \ChapName{}\SectionName{}}%
\def\Chapter #1{\ChapIndicator=1 \ifResetSectNo \SectNo=0 \fi
                \advance\ChapNo by 1
                \topglue\StartofChapSkip\relax \ChapName={#1}%
                \ChPosition{\ChpFont\the\EveryChapter}%
                \nobreak\vskip\InChapTitleSkip\relax
                \Position{\ChpNameFont #1}%
                \nobreak\vskip\AfterChapTitleSkip\relax}%
\def\Section #1{\goodbreak \vskip\BetweenSectionSkip\relax \SectionName={#1}%
                \advance\SectNo by1  \EquationNo=1
                \Position{\SctFont\the\EverySection\ #1}%
                \nobreak\vskip\AfterSectionTitleSkip\relax}%
\def\SubSection #1{\goodbreak \vskip\BetweenSectionSkip\relax 
                     \advance\SubSectNo by1 
                \Position{\SbSctFont\the\EverySubSection\ #1}% 
                \nobreak\vskip\AfterSectionTitleSkip\relax}%
\def\SemiLeftTitles {\let\Position=\Leftlines \let\ChPosition=\rightline }
\def\LeftTitles {\let\Position=\Leftlines \let\ChPosition=\leftline }
\def\CenteredTitles {\let\Position=\Centerlines \let\ChPosition=\Centerlines }
\def\Centerlines #1{{\parindent0pt \pretolerance10000 
                       \rightskip 0pt plus 1fil \leftskip 0pt plus 1fil
                       \parfillskip 0pt #1 \par}}
\def\Leftlines #1{{\parindent0pt \pretolerance10000
                    \rightskip0pt plus 1fil \parfillskip 0pt #1 \par}} 

%---------------------------------------------------------------------------_%
%%% FOR THEOREMS, ETC.:

\def\PropositionFonts #1#2{\font\PropLabelFont=#1
                           \font\PropStateFont=#2}
\def\ProofFont #1{\font\PrfFont=#1}
\def\Proposition #1.#2\par{\removelastskip \vskip 1.5\bigskipamount
     \noindent {\PropLabelFont #1.\ } {\PropStateFont #2} \endgraf} 
\def\ProofsEtc #1.{\removelastskip \medskip 
     \noindent {\PrfFont #1.\quad}}

%----------------------------------------------------------------------------
%%% MISCELLANEOUS:

\def\EndLine{\hfil\break}
\def\EndPage{\vfil\eject}
\def\Lcase #1{\lowercase\expandafter{#1}}
\def\RomanNumeral #1{\uppercase\expandafter{\romannumeral #1}}
\def\Hfil {\aftergroup\hfil}

%----------------------------------------------------------------------------
%%% THE HEADLINE, ETC.:

\newtoks\LeftRunningHead
\newtoks\RightRunningHead
\def\TopLRPno {\headline{\ifnum\ChapIndicator=1 \hfil
  \else \ifodd\pageno \the\RightRunningHead\llap{\folio}\else
                      \rlap{\folio}\the\LeftRunningHead\fi \fi}
    \footline{\ifnum\ChapIndicator=1 \hfil\sevenrm\folio\hfil \else \hfil\fi}}

%----------------------------------------------------------------------------
%%% THE OUTPUT ROUTINE:

\newdimen\LMarginOdd    \LMarginOdd=1 in
\newdimen\LMarginEven   \LMarginEven=1 in
\output{\ifodd\pageno \hoffset=\LMarginOdd \else \hoffset=\LMarginEven \fi
        \advance\hoffset by -1in  % Because 1in is the Plain TeX left margin.
        \advance\hoffset by.375in % To conform to the publisher's rules.
        \plainoutput \global\ChapIndicator=0 }

%----------------------------------------------------------------------------
%%% DEFAULTS:

\EveryChapter={C$\>$H$\>$A$\>$P$\>$T$\>$E$\>$R\ \ \the\ChapNo}
\EverySection={\the\ChapNo.\the\SectNo}
\EverySubSection={\the\ChapNo.\the\SectNo.\the\SubSectNo.}
\SemiLeftTitles
\TopLRPno
\RightRunningHead={\hfil\the\ChapNo.\the\SectNo.\ \the\SectionName\hfil}
\LeftRunningHead={\hfil\the\ChapNo.\ {\Smallcaps\the\ChapName}\hfil}
\SectNo=0  \SubSectNo=0
\OnFootnoterule{.8125in}

%----------------------------------------------------------------------------
\endinput
