% This is piTeX, a set of macros I (Paul Isambert) use to
% typeset documentations for my packages (that's why it is
% archived on CTAN).
%
% Perhaps in the future, when this achieves some kind of
% format-like completude, there'll be a documentation and
% it'll be publicly announced.
%
% You can of course use those macros, but you are on your
% own, and the files will probably be modified without announcement.
% The file is supposed to be \input on plain TeX with LuaTeX, at least v.0.6.
%
%
% The files needed are:
%
% texapi.tex (an independent package \input in the following file)
% yax.tex (an independent package)
% fonts.ptx (piTeX font "management")
% luaotfload.sty (an independent package (not by me))
% sections.ptx (piTeX sectionning commands)
%
% Date: July 2010.
%
%
% User interface
\input yax % which itself \input's texapi

\setcatcodes{\@\_=11}
\def\ptx@error{\senderror{PiTeX}}

% This to write in latin1.
\directlua{%
  lualog = function (message)
    texio.write_nl("") texio.write_nl("")
    texio.write_nl(message)
    texio.write_nl("") texio.write_nl("")
  end
 function convert(buf)
   return string.gsub(buf,"(.)", function (ch)
       return unicode.utf8.char(string.byte(ch))
     end)
 end
 callback.register('process_input_buffer',convert)
 }

\newcount\ptx@attribute_count
\def\newattribute#1{%
  \advance\ptx@attribute_count1
  \edefcs{ptx@attribute_#1}{10\the\ptx@attribute_count}%
  }
\def\ptx@attribute_number#1{\usecs{ptx@attribute_#1}}
\def\ptx@attribute_set#1{\passcs\attribute{ptx@attribute_#1}=1\relax}
\def\ptx@attribute_unset#1{\passcs\attribute{ptx@attribute_#1}=-"7FFFFFFF\relax}

\newdimen\urvsize
\urvsize\vsize

\def\inputpitexfile#1 {\input #1.ptx }

\inputpitexfile fonts
\inputpitexfile sections
% No footnotes in documentations.
%\input footnotes.tex
% I redefine the output routine on every job.
%\input output.tex

\defactiveparameter document{%
  \pdfinfo{%
    /Author (\usevalueor #1 : author {Unknown author})
    /Title  (\usevalueor #1 : pdftitle  {\usevalueor #1 : title {No title}})
    \ifattribute #1 : subject {/Subject (\usevalue #1 : subject )}{}
    \ifattribute #1 : keywords {/Keywords (\usevalue #1 : keywords )}{}
    }%
  \pdfcatalog{%
    /PageMode
      \ifcasevalue #1 : display
        \val outlines   /UseOutlines
        \val signets    /UseOutlines % French!
        \val thumbs     /UseThumbs
        \val thumbnails /UseThumbs
        \val full       /FullScreen
        \val fullscreen /FullScreen
        \elseval        /UseNone
      \endval
    }%
  }

\restrictparameter document : author title pdftitle date subject keywords display version\par
\restrictattribute document:display outlines signets thumbs thumbnails full fullscreen\par


\defactiveparameter page{%
  \settovalue\pdfpagewidth  #1 : pagewidth
  \settovalue\pdfpageheight #1 : pageheight
  \settovalue\baselineskip  #1 : baselineskip
  \settovalueor\topskip     #1 : topskip {\topskip=\baselineskip}%
  \settovalue\pdfhorigin    #1 : left
  \ifattribute #1 : right
              {\hsize = \dimexpr(\pdfpagewidth-\pdfhorigin-\usevalue #1 : right )\relax}
              {\settovalue\hsize #1 : hsize }%
  \settovalue\pdfvorigin    #1 : top
  \settovalue\parindent     #1 : parindent
  \settovalue\parskip       #1 : parskip
  \ifattribute #1 : lines {\vsize=\usevalue #1 : lines \baselineskip}{}%
  }
\restrictparameter page : pagewidth
                          pageheight
                          hsize
                          baselineskip
                          topskip
                          left
                          right
                          top
                          lines
                          parindent
                          parskip
                          \par

% Defaults... they don't produce anything beautiful.
% I redefine them on every job.
\setparameter page:
  pageheight   = 28cm
  pagewidth    = 21cm
  hsize        = 15cm
  baselineskip = 12pt
  lines        = 42
  left         = 1in
  top          = 1in
  parskip      = 0pt


% Should be used with YaX to set parameters.
\long\pdef\newblock{%
  \ifnext*{\gobbleoneand\ptx@newblock_group}
          {\ptx@newblock_nogroup}%
  }
\def\ptx@newblock_nogroup#1#2#3{%
  \def#1{\ifnext/{\gobbleoneand{#3}}{#2}}%
  }
\def\ptx@newblock_group#1#2#3{%
  \def#1{\ifnext/{\gobbleoneand{#3\egroup}}{\bgroup#2}}%
  }
\def\antigobblespace{%
  \ifcatnext a{ }{\iffnext({ }}%
  }


% PDF links
\def\dest#1#2{%
  \ifcs{ptx@link_name=#1}
       {\ptx@error{Link `#1' already exists}#2}
       {\letcs{ptx@link_name=#1}\relax
        \bgroup
          \ifvmode
            \pdfdest name {#1} xyz #2%
          \else
            \setbox0=\hbox{#2}%
            \raise\ht0\hbox{\pdfdest name {#1} xyz}#2%
          \fi
        \egroup
        }%
  }
\def\link#1#2{%
  \pdfstartlink attr {/Border [0 0 0]} goto name {#1}#2\pdfendlink
  }

% Margin notes � la Bringhurst: aligned with the line they refer
% to, unless they bleed into the bottom margin, in which case they
% are shifted up so that their bottom line is aligned with the
% page's bottom line. Totally experimental. It'd be nice if I could
% do that with sections too (so that a note doesn't bleed into the
% next section).
\newattribute{note}
\def\note{\ifnext[{\ptx@note}{\ptx@note[0pt]}}
\def\ptx@note[#1]#2{%
  \ptx@attribute_set{note}%
  \setbox0=\hbox{\vtop{% We need the box's height minus its first line, hence the vtop.
    \notefont \baselineskip=9pt
    \hsize\dimexpr\pdfpagewidth-(\pdfhorigin+\hsize+4em)\relax
    \rightskip=0pt plus 1fil
    \it\noindent #2}}%
  \ptx@attribute_unset{note}%
  \ht0=0pt \dp0=0pt
  \vadjust{%
    \kern-\dimexpr(2pt+#1)\relax
    \moveright\dimexpr\hsize+1em
    \box0
    \kern\dimexpr2pt+#1}%
  \vrule height 0pt width 0pt depth 2pt
% Problem: if the line where the note appears is last of its
% box, the box has no depth and vfilling doesn't work properly.
  }%
\directlua{%
  %
  % Box255 is traversed from the last node upward.
  % Each node with dimension adds to the "remaining height",
  % which is used when a note is encountered: if the note's depth
  % (minus its last box's depth) is larger than the remaining height,
  % then it is shifted upward, so that it doesn't bleed in the bottom
  % margin. When computing the remaining height, kerns and glues aren't
  % taken into account until a box has been seen (and the depth of
  % that first box isn't added either), so the note doesn't even
  % bleed into available page space that isn't filled with text.
  %
  process_marginalia = function (head)
    local remainingheight, first, item = 0, true, node.slide(head)
    while item do
      if node.has_attribute(item,\ptx@attribute_number{note}) then
        local depth = item.list.depth - node.tail(item.list.list).depth
        if depth > remainingheight then
          lualog("A note has been shifted on page " .. tex.count[0] ..
                 " by an amount of " .. (depth-remainingheight)/65536 .. "pt.")
          item.list.shift = remainingheight-depth
        end
      elseif node.has_field(item, "height") then
        remainingheight = remainingheight+item.height
        if first then
          first = false
        else
          remainingheight = remainingheight + item.depth
        end
      elseif node.has_field(item,"kern") then
        if not first then
          remainingheight = remainingheight+item.kern
        end
      elseif node.has_field(item, "spec") then
        if not first then
          remainingheight = remainingheight+item.spec.width
        end
      end
      item = item.prev
    end
  end
  }
\def\processmarginalia{%
  \directlua{process_marginalia(tex.box[255].list)}%
  }

\def\stoppage{\penalty-10000\relax}
\def\clearpage{\vfil\stoppage\relax}
\def\needspace#1{%
  \penalty0
  \ifdim\dimexpr\pagegoal-\pagetotal<\dimexpr#1\relax
    \clearpage
  \fi
  }

% Verbatim facilities.
\newbox\ptx@box_temp

\def\tcode#1{{\codefont#1}}
\long\def\com#1{%
  \bgroup
    \codefont
    \string#1%
  \egroup
  \antigobblespace
  }
\def\arg#1{{\codefont\it<#1>}\iffnext\spacechar{\kern.2ex }}
\def\barg#1{{\codefont\char"007B\relax{\it<#1>}\char"007D\relax}}
\def\oarg#1{{\codefont[{\it<#1>}]}}


\def\exampleskip{\vskip\baselineskip}
\def\example{%
  \exampleskip
  \bgroup
  \parindent0pt
  \setcatcodes{\\\{\}\$\&\#\^\_\ \~\%=12,\^^M=13}%
  \codefont
  \readexample
  }%
\let\ptx@example_original\example
\bgroup
\setcatcodes{\\=12,\|=0,\^^M=13}%
|gdef|readexample#1^^M#2\example/{#2|egroup|exampleskip}%
|restorecatcodes%
\setcatcodes{\^^M=13}%
\gdef^^M{\quitvmode\endgraf}%
\egroup%
\def\verb#1{%
  \def\ptx@verb##1#1{##1\egroup}%
  \bgroup
  \setcatcodes{\\\{\}\$\&\#\^\_\ \~\%=12}%
  \codefont
  \ptx@verb
  }

%
% Read and process examples.
%
\newwrite\ptx@examplewrite

\bgroup
\setcatcodes{\%=12,\/=14}
\gdef\Example{/
  \immediate\openout\ptx@examplewrite=\jobname.pex
  \immediate\write\ptx@examplewrite{% This is a scratch file used to typeset examples in \jobname.tex.}/
  \immediate\write\ptx@examplewrite/
  \bgroup
  \setcatcodes{\\\#\%\^^M\ =12}/
  \writeexample
  }/
\egroup

\bgroup
\setcatcodes{\^^M=12}%
\gdef\writeexample#1^^M{%
  \immediate\write\ptx@examplewrite{\noexpand\example}%  
  \ptx@writeexample}%
\gdef\ptx@writeexample#1^^M{%
  \passexpanded{\ifstring{#1}}{\examplestring}%
               {\immediate\write\ptx@examplewrite{\string\example/}%
                \immediate\closeout\ptx@examplewrite\egroup%
                \processexample}%
               {\immediate\write\ptx@examplewrite{\primunexpanded{#1}}%
                \ptx@writeexample}%
  }%
{\setcatcodes{\\=12,\|=0}|gdef|examplestring{\Example/}}%
\egroup

\def\typesetexample{%
  {\input\jobname.pex }%
  }
\def\doexamplenogroup{\let\example\ptx@Example\input\jobname.pex \let\example\ptx@example_original}
\def\doexample{{\doexamplenogroup}}
\def\ptx@Example{%
  \iffnext/{\gobbleoneand\endinput}%
  }
% To be redefined for different layout.
\def\processexample{\typesetexample\doexample}

\def\verb#1{%
  \def\ptx@verb##1#1{##1\egroup}%
  \bgroup
  \setcatcodes{\\\{\}\$\&\#\^\_\ \~\%=12}%
  \codefont
  \ptx@verb
  }


\newdimen\extraboxspace
\newdimen\ptx@extraboxspace_top
\newdimen\ptx@extraboxspace_right
\newdimen\ptx@extraboxspace_bottom
\newdimen\ptx@extraboxspace_left

\newfornoempty\ptx@colorbox_loop{1}#2,{%
  \ifcase#1
    \ptx@extraboxspace_top    =#2
    \ptx@extraboxspace_right  =#2
    \ptx@extraboxspace_bottom =#2
    \ptx@extraboxspace_left   =#2
  \or
    \ptx@extraboxspace_right  =#2
    \ptx@extraboxspace_left   =#2
  \or
    \ptx@extraboxspace_bottom =#2
  \or
    \ptx@extraboxspace_left   =#2
  \fi
  \passarguments{\numexpr(#1+1)}%
  }
\def\colorbox{%
  \ifnextnospace[\ptx@colorbox_setborders
          {\ptx@extraboxspace_top    =\extraboxspace
           \ptx@extraboxspace_right  =\extraboxspace
           \ptx@extraboxspace_bottom =\extraboxspace
           \ptx@extraboxspace_left   =\extraboxspace
           \ptx@colorbox_do}%
  }
\def\ptx@colorbox_setborders[#1]{%
  \ptx@colorbox_loop{0}{#1,}%
  \ptx@colorbox_do
  }
{\setcatcodes{pt=12}
\gdef\noPT#1pt{#1 }}
\def\ptx@colorbox_do#1#2{%
  \bgroup
  \setbox\ptx@box_temp=\hbox{#2}%
  \hbox{%
    \pdfliteral{
      q #1 rg #1 RG
      -\expandafter\noPT\the\ptx@extraboxspace_left
       \expandafter\noPT\the\dimexpr(\ht\ptx@box_temp+\ptx@extraboxspace_top)\relax
       \expandafter\noPT\the\dimexpr(\wd\ptx@box_temp+\ptx@extraboxspace_left+\ptx@extraboxspace_right)\relax
      -\expandafter\noPT\the\dimexpr(\ht\ptx@box_temp+\ptx@extraboxspace_top+\dp0+\ptx@extraboxspace_bottom)\relax
      re f Q}%
    #2}%
  \egroup
  }

\def\trace{\tracingcommands4 \tracingmacros4 }
\def\untrace{\tracingcommands0 \tracingmacros0 }

\restorecatcodes