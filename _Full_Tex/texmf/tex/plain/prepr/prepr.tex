\newbox\leftpage
\newdimen\fullhsize 
\newdimen\hstitle
\newdimen\hsbody
\tolerance=1000\hfuzz=2pt

\def\bigans{b }
\message{ big or little (b/l)? }\read-1 to\answ

\ifx\answ\bigans\message{(This will come out unreduced.}
\magnification=1200\baselineskip=14pt
\font\bigfnt=cmbx10  \global\let\absfnt=\tenrm
\hsbody=\hsize \hstitle=\hsize %take default values for unreduced format

\else\message{(This will be reduced.}
\let\lr=L
\special{landscape}
\font\bigfnt=cmbx10 scaled\magstep1
\font\absfnt=cmr10 scaled\magstep1
\def\almostshipout#1{\if L\lr \count1=1
      \global\setbox\leftpage=#1 \global\let\lr=R
   \else \count1=2
      \shipout\vbox{\hbox to\fullhsize{\box\leftpage\hfil#1}}
      \global\let\lr=L\fi}

\output={\ifnum\count0=1 %%% This is the HUTP version
 \shipout\vbox{\hbox to \fullhsize{\hfill\pagebody\hfill}}\advancepageno
 \else
 \almostshipout{\leftline{\vbox{\pagebody\makefootline}}}\advancepageno 
 \fi}

\magnification=1000\baselineskip=14pt\voffset=-.31truein\hoffset=.05truein
\hstitle=8truein\hsbody=4.75truein\vsize=7truein\fullhsize=10truein
\fi

%%%%%%%%%%%%%%%%%%%%%%%%%%%%%%%%%%%%%%%%%%%%%%%%%%%%%%%%%%%%%%%%%%%%%


\def\tit#1#2{\nopagenumbers\hsize=\hstitle\null%
\vskip 5.5truecm\bigfnt\centerline{#1}\medskip\centerline{#2}
\absfnt\vskip 0.8truecm\message{#1}\message{#2}}


\def\date#1{\ifnum\folio=2\footline={\hss\tenrm\folio\hss}\fi
\vfill\line{ASTRPD \hfill #1}\tenrm\supereject\hsize=\hsbody%
\footline={\hss\tenrm\folio\hss}} % restores pagenumbers


% tagged sec numbers
\global\newcount\secno \global\secno=0
\global\newcount\meqno \global\meqno=1
\global\newcount\subsecno \global\subsecno=0

\def\section#1{\global\advance\secno by1\global\meqno=1\global\subsecno=0
\ifx\answ\bigans \vfill\eject
\else\bigbreak\bigskip\fi% (combination \goodbreak\bigskip\bigskip)
\noindent{\bf\the\secno. #1}\par\nobreak\medskip\nobreak\message{#1}}

\def\subsec#1{\global\advance\subsecno by1\bigskip\noindent
{\bf\the\secno.\the\subsecno\ #1}\par\nobreak\smallskip\nobreak\message{#1}}


%       \eqn\label{a+b=c}	gives a displayed equation with number
%				chosen consecutively within sections.
%     \eqnn and \eqna define labels in advance
\newwrite\efile \let\firsteqn=T
\def\writeqno#1%
{\if T\firsteqn \immediate\openout\efile=eqns.tmp\global\let\firsteqn=F\fi%
\immediate\write\efile{#1 \string#1}\global\advance\meqno by1}

\def\eqnn#1{\xdef #1{(\the\secno.\the\meqno)}\writeqno#1}
\def\eqna#1{\xdef #1##1{(\the\secno.\the\meqno##1)}\writeqno{#1{}}}

\def\eqn#1#2{\xdef #1{(\the\secno.\the\meqno)}%
$$#2\eqno(\the\secno.\the\meqno)$$\writeqno#1}

%			 footnotes
\global\newcount\ftno \global\ftno=1
\def\foot#1{\footnote{$^{\the\ftno}$}{#1}\ %
\global\advance\ftno by1}


%     \fig\label{text}
% generates a number, assigns it to \label, generates an entry.
% To list the figs on a separate page,  \listfigs

\global\newcount\figno \global\figno=1
\newwrite\ffile

%\def\fig#1#2{\item{Figure #1: }#2}

\def\fig#1#2{\the\figno\nfig#1{#2}}
\def\nfig#1#2{\xdef#1{\the\figno}%
\ifnum\figno=1\immediate\openout\ffile=figs.tmp\fi%
\immediate\write\ffile {\noexpand \item{Fig. \noexpand#1 :\ }#2}%
\global\advance\figno by1}

\def\semi{;\hfil\noexpand\break}

\def\listfigs{\vfill\eject\immediate\closeout\ffile%\parindent=20pt
\centerline{\bf Figure Captions}\bigskip\parindent=40pt%
\input figs.tmp\vfill\eject\parindent=20pt}
%
%     \ref\label{text}
% generates a number, assigns it to \label, generates an entry.
% To list the refs on a separate page,  \listrefs

\global\newcount\refno \global\refno=1
\newwrite\rfile

\def\ref#1#2{[\the\refno]\nref#1{#2}}
\def\nref#1#2{\xdef#1{[\the\refno]}%
\ifnum\refno=1\immediate\openout\rfile=refs.tmp\fi%
\immediate\write\rfile{\noexpand\item{\noexpand#1\ }#2.}%
\global\advance\refno by1}

\def\listrefs{\vfill\eject\immediate\closeout\rfile%\parindent=20pt
\centerline{{\bf References}}\bigskip\frenchspacing%
\input refs.tmp\vfill\eject\nonfrenchspacing}


%and finally acknowledgments

\def\ack#1{\vskip 4truecm\centerline{\bf Acknowledgments}\par\nobreak
\bigskip\nobreak #1}

%%%%%%%%%%%%%%%%%%%%%%%%%%%%%%%%%%%%%%%%%%%%%%%%%%%%%%%%%%%%%%%%%%%%%%%%%%%%%

\def\grad#1{\,\nabla\!_{{#1}}\,}
\def\om#1#2{\omega^#1{}_#2}
