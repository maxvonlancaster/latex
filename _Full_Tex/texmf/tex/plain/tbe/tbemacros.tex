%                  TBEMACROS, Version 1.0, July 10, 1991

%This file contains some new commands defined in the book `TeX by Example',
%by Arvind Borde (copyright (c) 1992, Academic Press). The commands may be
%copied and modified freely, as long as (1) such copies and modifications are
%not made or distributed for re-sale, (2) this notice is retained, and
%(3) the date of any modifications is added below this notice.

%The commands are provided `as is'. The author provides no warranty whatsoever,
%either express or implied, regarding them, or with respect to their fitness
%for any particular purpose.

%The commands were put together to demonstrate the versatility of TeX and the
%variety of applications possible with it. They are not all equally useful.
%An effort has been made to keep them reasonably simple and to avoid
%certain advanced constructions. Though this has meant sacrificing
%sophistication at various points, it should allow even new users to edit
%the definitions to suit their own purposes with relative ease.

%Samples of the uses of these commands are to be found in a companion file,
%`TBESAMPLES.TEX'.

%********** LETTERS ************************************************************
\def\letterhead {\hrule height 1pt depth 0pt width\hsize
      \medskip
      \rightline{\vbox{\hsize 1.5true in \baselineskip=12pt \parindent=0pt
      Your name, \hfil\break
      Your address, \hfil\break
      Etc. }}}

\def\date #1{\rightline{\vbox{\hsize 1.5true in\noindent #1}}\bigskip}

\def\address #1{\vbox{\baselineskip=12pt\halign{## \hfil\cr #1}}}

\def\sal #1{\noindent #1}

\def\endletter #1#2{\nobreak\medskip\rightline{\vbox{\hsize 1.5true in
      \parindent=0pt #1\vskip 1.25 true cm #2}}}

%********** MEMORANDA **********************************************************
\def\block #1{\vbox{\hsize 2.5 true cm\noindent \bf#1:}}

\def\memo TO:#1FROM:#2SUBJECT:#3DATE:#4\par{\centerline{\sl MEMO!}
      \bigskip \hrule height1pt \medskip
      \vbox{\parindent=75pt\parskip=1pt
      \item{\block{TO}}#1
      \item{\block{FROM}}#2
      \item{\block{SUBJECT}}#3
      \item{\block{DATE}}#4 }\medskip \hrule height1pt \bigskip}

%********** SCRIPTS ************************************************************
\newcount\spk
\def\beginscript {\bgroup \parindent=0pt \spk=1 \sl \rightskip.4in
      \def\par {\ifnum\spk=1 \endgraf \it \spk=2 \leftskip.4in \rightskip0in
               \else \endgraf \sl \spk=1 \leftskip0in \rightskip.4in \fi}}

\def\endscript {\egroup}

%********** FERMAT *************************************************************
\newcount\var \newcount\pw \newcount\tmp \newcount\cnt
\def\pow#1#2#3{\var=#1 \pw=#2 \tmp=\var \cnt=1
      \loop \multiply\var by\tmp \advance\cnt by1 \ifnum\cnt<\pw \repeat
      \global#3=\var}

\newcount\xf \newcount\xnf \newcount\yf \newcount\ynf \newcount\zf \newcount\znf
\def\n {\number}

\def\fermat#1#2#3{$\global\xf=#1 \global\yf=#2 \global\pw=#3
      \pow{\xf}{\pw}{\xnf} \pow{\yf}{\pw}{\ynf}
      \global\tmp=\ynf \global\advance\tmp by\xnf
      {\n\xf}^{\n\pw}+{\n\yf}^{\n\pw}={\n\tmp}$.\hfil\break
      \ifnum\xf>\yf \zf=\xf \else \zf=\yf \fi
      \loop {\pow{\zf}{\pw}{\znf}} \ifnum\znf<\tmp \advance\zf by1 \repeat
      \ifnum\znf=\tmp  The sum seems to be exactly ${\n\zf}^{\n\pw}$.
        \ifnum\pw=2 {\it Yawn!\/} Tell me something I don't know, will you?
        \else Incredible!  But, perhaps you'd better check my work.\fi
      \else \advance\zf by-1
      This lies between ${\n\zf}^{\n\pw}$\pow{\zf}{\pw}{\znf}($={\n\znf}$)
      \advance\zf by1
      and ${\n\zf}^{\n\pw}$\pow{\zf}{\pw}{\znf}($={\n\znf}$).\fi}%

%********** FRAMING ************************************************************
\def\frame #1#2#3#4{\vbox{\hrule height #1pt%    TOP RULE
 \hbox{\vrule width #1pt\kern #2pt%              RULE AND SPACE ON LEFT
 \vbox{\kern #2pt%                               SPACE AT TOP
 \vbox{\hsize #3\noindent #4}%                   MATERIAL THAT WILL BE BOXED
 \kern #2pt}%                                    SPACE AT BOTTOM
 \kern #2pt\vrule width #1pt}%                   SPACE AND RULE ON RIGHT
 \hrule height0pt depth #1pt}}%,                 BOTTOM RULE

\def\fitframe #1#2#3{\vbox{\hrule height#1pt%    TOP RULE
 \hbox{\vrule width#1pt\kern #2pt%               RULE AND SPACE ON LEFT
 \vbox{\kern #2pt\hbox{#3}\kern #2pt}%           TOP, MATERIAL, BOTTOM
 \kern #2pt\vrule width#1pt}%                    SPACE AND RULE ON RIGHT
 \hrule height0pt depth#1pt}}%                   BOTTOM RULE

\def\shframe #1#2#3#4{\vbox{\hrule height 0pt%   NO TOP SHADOW
 \hbox{\vrule width #1pt\kern 0pt%               LEFT SHADOW
 \vbox{\kern-#1pt\frame{.3}{#2}{#3}{#4}%         SHADOW STARTS #1 PT FROM TOP
 \kern-.3pt}%                                    MOVE UP RULE THICKNESS AT BOT.
 \kern-#1pt\vrule width 0pt}%                    STOPS #1 PT FROM RT; NO RT SHAD
 \hrule height #1pt}}%                           BOTTOM SHADOW

\def\s #1{\frame{.3}{2}{8pt}{\centerline{#1\vphantom{(}}}\ }

%********** RULED TRAPEZOID ****************************************************
\newcount\tw        %top width in pt
\newcount\bw        %bottom width in pt
\newcount\h         %height in pt
\newcount\bs        %bottom shift in pt
\newcount\th        %line thickness in 1/64th of a pt
\newcount\gp        %line gap in 1/64th of a point
\newcount\rs        %running shift
\newcount\rw        %running width
\newcount\rh        %running height
\newcount\tmp       %for temporarily storing variables

\def\trap #1#2#3#4#5#6{\vbox{\offinterlineskip
      \tw=#1 \bw=#2 \h=#3 \bs=#4 \th=#5 \gp=#6 \rh=0
      \multiply\tw by 65536 \multiply\bw by 65536 \multiply\bs by 65536
      \multiply\th by 1024 \multiply\gp by 1024
      \loop
      \tmp=\bs \multiply\tmp by\rh \divide\tmp by\h
      \rs=\tmp                                        %running shift calculated
      \tmp=\bw \advance\tmp by-\tw \multiply\tmp by\rh
      \divide\tmp by\h \advance\tmp by\tw \rw=\tmp    %running width calculated
      \hbox{\kern\rs sp\vrule height0sp depth\th sp width\rw sp}%RULE DRAWN HERE
      \vskip\gp sp                                              %GAP LEFT HERE
      \ifnum\rh<\h  \tmp=\rh \multiply\tmp by 65536
       \advance\tmp by\th \advance\tmp by\gp \divide\tmp by65536 \rh=\tmp
      \repeat}}%

\def\rect #1#2#3#4{\trap{#1}{#1}{#2}{0}{#3}{#4}}%

%********** MULTIPLE COLUMN ****************************************************
\newcount\colnumber \newbox\col \newdimen\tmpdim \newdimen\size
\newdimen\coljump \coljump=.2 true in                  %GAP BETWEEN COLUMNS

\def\niceskip {\vskip\baselineskip}
\def\nstrut {\vrule height\topskip depth0pt width0pt}  %TO PROP THINGS UP
\def\divider{\hskip\coljump}
\def\dividerule{\dimen0=.4pt \dimen1=\coljump \advance\dimen1 by-\dimen0
      \divide\dimen1 by2
      \def\divider{\hskip\dimen1 \vrule width\dimen0 \hskip\dimen1}}

\def\beginart #1/#2{\vbox\bgroup#1 \colnumber=#2  \parskip=0pt
      \advance\colnumber by-1 \tmpdim=\coljump \multiply\tmpdim by\colnumber
      \size=\hsize  \advance\size by-\tmpdim
      \advance\colnumber by1 \divide\size by\colnumber
      \vbadness=10000 \hbadness=2000 \tolerance=2000
      \setbox\col=\vbox\bgroup\hsize\size \noindent\nstrut}

\def\endart {\global\size=\baselineskip \vfil \egroup
      \multiply\size by\colnumber  \advance\size by-\baselineskip
      \tmpdim=\ht\col  \advance\tmpdim by\size  \divide\tmpdim by\colnumber
      \hbox{\splittopskip=\topskip \doittoit}\egroup}

\def\doittoit{\ifnum\colnumber>0 \vsplit\col to \tmpdim
      \global\advance\colnumber by-1
      \ifnum\colnumber>0 \divider \fi \doittoit \fi}

%********** FONT AND CODE TABLE ************************************************
\def\chartable #1{\smallbreak\vbox{\noindent%
     \underbar{Characters in the {\sl#1\/} font, with decimal codes:}
     \raggedright \hbadness5000 \tolerance10000 \medskip
     \font\ft=#1 \ft \dimen0=14pt \baselineskip=\dimen0
     \ifdim\dimen0<3.25ex \baselineskip=3.25ex \fi \count255=0 \dimen0=10pt
     \loop \setbox0=\hbox{\char\count255}
     \ifdim\wd0>\dimen0 \dimen0=\wd0 \fi \advance\count255 by1
     \ifnum\count255<128 \repeat \count255=0
     \advance\dimen0 by 25pt \noindent \loop
     \hbox to\dimen0{\hbox to23pt{\hfil\rm\the\count255:\ }\char\count255\hfil}
     \advance\count255 by1 \ifnum\count255<128 \quad\repeat
     \smallbreak} \font\tenrm=cmr10 }

%********** THE `IGNORE' COMMAND ***********************************************
\def\ignore {\count255=0 \begingroup
      \loop \catcode\count255=14  % Make everything a comment character.
         \advance\count255 by1 \ifnum\count255<127
      \repeat \catcode`\!=0 }     % Makes ! an escape character.
{\catcode`\!=0 !gdef!E{!endgroup}}% Defines the `stop ignoring' command.

%********** STACKSYMBOLS *******************************************************
\def\stacksymbols #1#2#3#4{\def\theguybelow{#2}
    \def\verticalposition{\lower#3pt}
    \def\spacingwithinsymbol{\baselineskip0pt\lineskip#4pt}
    \mathrel{\mathpalette\intermediary#1}}
\def\intermediary#1#2{\verticalposition\vbox{\spacingwithinsymbol
      \everycr={}\tabskip0pt
      \halign{$\mathsurround0pt#1\hfil##\hfil$\crcr#2\crcr
               \theguybelow\crcr}}}

\def\lapproxeq{\stacksymbols{<}{\sim}{2.5}{.2}}
\def\gapproxeq{\stacksymbols{>}{\sim}{3}{.5}}

%********** DATING *************************************************************
\def\monthname {\ifcase\month\or January\or February\or March\or
April\or May\or June\or July\or August\or September\or October\or
November\or December\fi}

\newcount\mins  \newcount\hours  \hours=\time \mins=\time
\def\now{\divide\hours by60 \multiply\hours by60 \advance\mins by-\hours
     \divide\hours by60         % NOTE: \divide only gives integer answers.
     \ifnum\hours>12 \advance\hours by-12
       \number\hours:\ifnum\mins<10 0\fi\number\mins\ P.M.\else
       \number\hours:\ifnum\mins<10 0\fi\number\mins\ A.M.\fi}
\def\today {\monthname\ \number\day, \number\year}

%********** TABLE COMMANDS *****************************************************
\newdimen\tempdim                 % For temporary storage.
\newdimen\othick   \othick=.4pt   % To set the outer rule thickness.
\newdimen\ithick   \ithick=.4pt   % To set the inner rule thickness.
\newdimen\spacing  \spacing=9pt   % To set the interline spacing.
\newdimen\abovehr  \abovehr=6pt   % Space above horizontal rules.
\newdimen\belowhr  \belowhr=8pt   % Space below horizontal rules.
\newdimen\nexttovr \nexttovr=8pt  % Space next to vertical rules.

\def\r{\hfil&\omit\vrsp\vrule width\othick\cr&}   % To start a new line.
\def\rr{\hfil\down{\abovehr}&\omit\vrsp\vrule width\othick\cr
     \noalign{\hrule height\ithick}\up{\belowhr}&}% To draw an \hrule.
\def\up#1{\tempdim=#1\advance\tempdim by1ex
     \vrule height\tempdim width0pt depth0pt}%   For space above a line.
\def\down#1{\vrule height0pt depth#1 width0pt}%  For space below a line.
\def\large#1#2{\setbox0=\vtop{\hsize#1 \lineskiplimit=0pt \lineskip=1pt
     \baselineskip\spacing \advance\baselineskip by 3pt \noindent
     #2}\tempdim=\dp0\advance\tempdim by\abovehr\box0\down{\tempdim}}
% `\large' allows you to make multi-line table entries.
\def\dig{{\hphantom0}}  % To leave a space, the width of a digit.
\def\hgap#1{\hskip-\nexttovr\hskip#1\hskip-\nexttovr\relax} % For a gap.
\def\vrsp{\hskip\nexttovr\relax}
\def\toprule#1{\def\startrule{\hrule height#1\relax}} % Set a top rule.
\toprule{\othick}                      % Picking the `\toprule' default.
\def\nstrut{\vrule height\spacing depth3.5pt width0pt}
\def\exclaim{\char`\!}                 % To print an exclamation mark.
\def\preamble#1{\def\startup{#1}}      % For `customized' preambles.
\preamble{&##}                         % Choosing the default preamble.
{\catcode`\!=\active
 \gdef!{\hfil\vrule width0pt\vrsp\vrule width\ithick\relax\vrsp&}}
% Setting up `!' as the entry separator.

\def\table #1{\vbox\bgroup \setbox0=\hbox{#1}
     \vbox\bgroup\offinterlineskip  \catcode`\!=\active
     \halign\bgroup##\vrule width\othick\vrsp&\span\startup\nstrut\cr
     \noalign{\medskip}
     \noalign{\startrule}\up{\belowhr}&}

\def\caption #1{\down{\abovehr}&\omit\vrsp\vrule width\othick\cr
     \noalign{\hrule height\othick}\egroup\egroup \setbox1=\lastbox
     \tempdim=\wd1 \hbox to\tempdim{\hfil \box0 \hfil} \box1 \smallskip
     \hbox to\tempdim{\advance\tempdim by-20pt\hfil\vbox{\hsize\tempdim
     \noindent #1}\hfil}\egroup}

%********** VERBATIM REPRODUCTION **********************************************
\let\cc=\catcode
{\cc`\^^M=\active %
\gdef\losenolines{\cc`\^^M=\active \def^^M{\leavevmode\endgraf}}}
\def\literal {\begingroup \cc`\\=12 \cc`\{=12 \cc`\}=12 \cc`\$=12 \cc`\&=12
 \cc`\#=12 \cc`\%=12 \cc`\~=12 \cc`\_=12 \cc`\^=12 \cc`\*=12 \cc`\@=0
 \cc`\`=\active \losenolines \obeyspaces \tt}%
{\obeyspaces\gdef {\hglue.5em\relax}}

{\cc`\`=\active \gdef`{\relax\lq}}

\def\vquotingon{\cc`\"=\active}
\def\vquotingoff{\cc`\"=12}
\vquotingon
\def"{\literal\leavevmode\hbox\bgroup\com}
%`\leavevmode' starts a new paragraph, if needed.

\def\beginliteral{\medskip \literal \cc`\"=12%
  \parskip0pt \baselineskip11pt \thatisit}

{\cc`\@=0 \cc`\\=12 @cc`@^^M=@active %
 @gdef@com#1"{#1@egroup@endgroup} %
 @gdef@thatisit^^M#1\endliteral{#1@endgroup@smallskip}}
\vquotingoff

%********** GENERAL ************************************************************
\def\pattern #1#2{\count0=0
      \loop #1\advance\count0 by 1 \ifnum\count0<#2 \repeat}

\def\raggedleft{\leftskip=0pt plus2em \parfillskip0pt
     \spaceskip=.3333em \xspaceskip=.5em\relax}

%*******************************************************************************
\endinput
