\topmatter

\title{Estimating Risk Aversion from \cr 
     Arrow-Debreu Portfolio Choice}

\runningname{Hal R. Varian}
\runningtitle{Estimating Risk Aversion}

\thanks{This research was supported in part by the National Science
Foundation.  I would like to thank Richard Green for helpful remarks.  I am
especially grateful to an anonymous referee whose comments significantly
improved the statements and proofs of the results.}

\author{Hal R. Varian}

\affil{University of Michigan}

\date{October 27, 1984}

\version{\today}

\abstract{This paper derives necessary and sufficient conditions for
Arrow-Debreu choices of contingent consumption to be compatible with the
maximization of a state independent expected utility function that exhibits
increasing or decreasing absolute risk aversion, or increasing or
decreasing relative risk aversion.  The conditions can be used to bound
different measures of risk aversion based on a single observation of
Arrow-Debreu portfolio choice.}

\keywords{Revealed preference, expected utility, risk aversion, portfolio
choice.}


\address{Hal R. Varian, Department of Economics, University of Michigan, 
Ann Arbor, MI 48109}

\endtopmatter

\document

\noindent {\scten The expected utility} hypothesis forms the basis for much
of our understanding of investor behavior under uncertainty.  It is
commonly agreed that a well-behaved expected utility function should be an
increasing and concave function of wealth, or, equivalently, that its first
derivative should be positive and its second derivative should be negative.
It is also widely accepted that the Arrow-Pratt measure of absolute risk
aversion should be declining with wealth.  There is much less agreement
about the behavior of the Arrow-Pratt measure of {\it relative\/} risk
aversion, although some investigators have argued that it should increase
with wealth.
  
In this note I derive necessary and sufficient conditions for choices of
contingent consumption across states of nature to satisfy various
hypotheses about the behavior of these measures of risk aversion.  If the
portfolio choice behavior of the consumer is consistent with the conditions
I derive, then the conditions can be used to bound the Arrow--Pratt
measures of absolute and relative risk aversion.   The conditions are
derived using methods of the ``nonparametric approach'' to optimizing
behavior introduced by Afriat (1967) and extended by Diewert (1973),
Diewert and Parkan (1978), and Varian (1982), (1983a).  Applications of
these methods to choice under uncertainty include Dybvig and Ross (1982),
Green and Srivastava (1983), and Varian (1983b).

\section The Maximization Problem

Consider an investor who chooses a pattern of consumption across states of
nature to solve the following problem:
  $$\max \sum_{s=1}^S \pi_s u(c_s) $$

\Refs

\ref  
\by{Afriat, S.} \yr{1967a} \paper{The Construction of a Utility
Function from Expenditure Data} \jour{International Economic Review}
\vol{8} \pages{67--77}
\endref

\ref  
\by{Afriat, S.} \yr{1967b} \paper{The Construction of Separable
Utility Functions from Expenditure Data} \paperinfo{mimeo, Purdue}
\endref

\ref  \by{Breeden, D. and R. Litzenberger} \yr{1978} \paper{Prices of
State-Contingent Claims Implicit in Option Prices} \jour{Journal of
Business} \pages{621--651}
\endref

\ref 
\by{Diewert, E.} \yr{1973} \paper{Afriat and Revealed Preference Theory} 
\jour{Review of Economic Studies} \vol{40} \pages{419--426}
\endref

\ref  \by{Diewert, E. and C. Parkan} \yr{1978} \paper{Tests for Consistency
of Consumer Data and Nonparametric Index Numbers} \paperinfo{Working Paper
78-27, University of British Columbia}
\endref

\ref  \by{Dybvig, P. and S. Ross} \yr{1982} \paper{Portfolio Efficient Sets}
\jour{Econometrica} \pages{1525--1546}
\endref

\ref  \by{Green, R. and S. Srivastava} \yr{1983} \paper{Preference
Restrictions, Asset Returns, and Consumption} \paperinfo{mimeo,
Carnegie--Mellon University}
\endref

\ref  \by{Varian, H.} \yr{1982} \paper{The Nonparametric Approach to Demand
Analysis} \jour{Econometrica} \vol{50} \pages{945--973}
\endref

\ref  
\by{Varian, H.} \yr{1983a} \paper{Nonparametric Tests of Models of
Consumer Behavior} \jour{Review of Economic Studies} \vol{50}
\pages{99--110}
\endref

\ref  
\by{Varian, H.} \yr{1983b} \paper{Nonparametric Tests of Models of
Investor Behavior} \jour{Journal of Financial and Quantitative Analysis}
\vol{18} \pages{269--278} 
\endref
