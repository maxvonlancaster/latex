\documentclass[12pt]{amsart}
% ----------------------------------------------------------------


\usepackage[english]{babel}

\usepackage{amsmath,amsfonts,amssymb,amsthm}
\usepackage[english]{babel}


\newtheorem{thm}{Theorem}
\newtheorem{cor}[thm]{Corollary}
\newtheorem{lem}[thm]{Lemma}
\newtheorem{prop}{Proposition}
\newtheorem{exm}[thm]{Example}
\theoremstyle{definition}
\newtheorem{defn}{Definition}
\theoremstyle{remark}
\newtheorem{rem}[thm]{Remark}
\numberwithin{equation}{section}

\setlength{\voffset}{-20mm}
\setlength{\hoffset}{-20mm}
\setlength{\textwidth}{170mm}
\setlength{\textheight}{240mm}

\tolerance=3000
 \sloppy

\begin{document}

\title{{\sc Spaces-intersections and spaces-unions  }}

\author{Volodymyr Maslyuchenko}

\address{Yuriy Fedkovych Chernivtsi National University, {\it e-mail: v.maslyuchenko@gmail.com}}

\maketitle
Spaces-intersections $X=\bigcap\limits_{n=1}^{\infty}X_{n}$ and spaces-unions  $X=\bigcup\limits_{n=1}^{\infty}X_{n}$ meet in the different section of analysis, in particular in the theory of perfect spaces of Kothe-Toeplitz [1] as weighing and weighable spaces that is construct on the basis of space $l_{1}$ all absolutely convergent numerical series, their generalizations construct on the basis of spaces $l_{p}$ sumacious from the  degree p of numerical sequences in article [2] at research of generalized spaces of Kothe (see [3] and the literature indicated there and [7]), in the theories of methods of summarization [4,5], quasi-reflexive spaces [6] and inductive limits [3] . In this talk it is planned to do the review of the received results and discuss the problems related to their generalization.




\begin{thebibliography}{99}
\bibitem{Kothe} Kothe G.{\it Die Stufenraume, eine einfache Klasse linearer volkommener Raume}, Math.Z. {\bf 51} (1948), 311--345.
\bibitem{Diedonne} Diedonne I., Gomes A. P. {\it Sur certain espacec vectoriels topologiques}, C. R. Arad. Sci. Paris {\bf 230} (1950), 1129--1130.
\bibitem{Maslyuchenko}  Maslyuchenko V.K. {\it Some questions of theories of generalized spaces (in Rassian)}, Dis. Candidate of Physics and Mathematics --Chernivtsi, (1983), --131 �.
\bibitem{Zeller} Zeller K. {\it Alleqemaine Eigenschaften von Limiterungsverfahren}, Math. Z. {\bf 53}  N 3 (1951), 463--487.
\bibitem{Wilansky} Wilansky A., Zeller K. {\it FH-spase and intersection of FK-spaces}, Michigan Math. Z. {\bf 6} N 4  (1959), 346--357.
\bibitem{Plichko }  Plichko A.M. , Maslyuchenko V.K. {\it Quasi-reflexive locally convex spaces without Banach subspaces(in Rassian)}, Function theory and functional analysis and their adjectives {\bf 44} (1985), 78--84.
\bibitem{ Maslyuchenko  } Maslyuchenko V.K., Mykhaylyuk V.V. {\it About the coincidence of spaces of Kothe in family $l_{p}(R): 0<p\leq \infty$ (in Ukraine) }, Matematychni Studii {\bf 17} N 41 (2002), 75--80.
 \end{thebibliography}

\end{document}
% ----------------------------------------------------------------
